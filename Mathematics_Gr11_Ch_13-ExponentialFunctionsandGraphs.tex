\chapter{Exponential Functions and Graphs - Grade 11}
\label{m:fg:e11}

\section{Introduction}
In Grade 10, you studied graphs of many different forms. In this chapter, you will learn a little more about the graphs of exponential functions.

%\begin{syllabus}
%\item Demonstrate the ability to work with various types of functions.
%\item Recognise relationships between variables in terms of numerical, graphical, verbal and symbolic representations and convert flexibly between these representations (tables, graphs, words and formulae).
%\item Generate as many graphs as necessary, initially by means of point-by-point plotting, supported by available technology, to make and test conjectures and hence to generalise the effects of the parameters $a$ and $q$ on the graphs of functions including:
%\begin{eqnarray*}
%y=ab^{(x+p)} + q \mbox{ ;$b>0$}
%\end{eqnarray*}
%\item Identify characteristics as listed below and hence use applicable characteristics to sketch graphs of functions including those listed above:
%\begin{itemize}
%\item domain and range;
%\item intercepts with the axes;
%\item turning points, minima and maxima;
%\item asymptotes;
%\item shape and symmetry;
%\item average gradient (average rate of change);
%\item intervals on which the function increases/decreases;
%\item the discrete or continuous nature of the graph.
%\end{itemize}
%\end{syllabus}

\section{Functions of the Form $y=ab^{(x+p)} + q$ for $b> 0$}
This form of the exponential function is slightly more complex than the form studied in Grade 10.

\begin{figure}[htbp]
\begin{center}
\begin{pspicture}(-5,-1)(5,4)
%\psgrid
\psset{yunit=0.75,xunit=0.75}
\psaxes[arrows=<->](0,0)(-5,-1)(5,5)
\psplot[plotstyle=curve,arrows=<->]{-5}{0.6}{2 x 1 add exp 2 add}
\psline[linestyle=dashed](-5,2)(5,2)
\end{pspicture}
\caption{General shape and position of the graph of a function of the form $f(x)=ab^{(x+p)} + q$.}
\label{fig:mf:g:exponential11}
\end{center}
\end{figure}

\Activity{Investigation}{Functions of the Form $y=ab^{(x+p)} + q$}{
\begin{enumerate}
\item{On the same set of axes, plot the following graphs:
\begin{enumerate}
\item{$a(x)=-2^{(x+1)} + 1$}
\item{$b(x)=-1^{(x+1)} + 1$}
\item{$d(x)=1^{(x+1)} + 1$}
\item{$e(x)=2^{(x+1)} + 1$}
\end{enumerate}
Use your results to deduce the effect of $a$.}
\item{On the same set of axes, plot the following graphs:
\begin{enumerate}
\item{$f(x)=2^{(x+1)} -2$}
\item{$g(x)=2^{(x+1)} -1$}
\item{$h(x)=2^{(x+1)} +0$}
\item{$j(x)=2^{(x+1)} +1$}
\item{$k(x)=2^{(x+1)} +2$}
\end{enumerate}
Use your results to deduce the effect of $q$.}
\item{Following the general method of the above activities, choose your own values of $a$ and $q$ to plot five different graphs of $y=ab^{(x+p)} + q$ to deduce the effect of $p$.}
\end{enumerate}}

You should have found that the value of $a$ affects whether the graph is above the asymptote ($a>0$) or below the asymptote ($a<0$).

You should have also found that the value of $p$ affects the position of the $x$-intercept.

You should have also found that the value of $q$ affects the position of the $y$-intercept.

These different properties are summarised in Table~\ref{tab:mf:graphs:summaryexp}. The axes of symmetry for each graph is shown as a dashed line.

\begin{table}[htb]
\begin{center}
\caption{Table summarising general shapes and positions of functions of the form \newline $y=ab^{(x+p)} + q$.}
\vspace{0.5cm}
\label{tab:mf:graphs:summaryexp}
\begin{tabular}{|c|c|c||c|c|}\hline
&\multicolumn{2}{c||}{$p<0$}&\multicolumn{2}{c|}{$p>0$}\\\hline
& $a>0$&$a<0$& $a>0$&$a<0$\\\hline\hline
$q>0$&
\begin{pspicture}(-1.2,-1.2)(1.2,1.2)
\psset{xunit=0.2,yunit=0.2}
\psaxes[arrows=<->,dx=0,Dx=10,dy=0,Dy=10](0,0)(-5,-5)(5,5)
\psplot[plotstyle=curve,arrows=<->]{-5}{2.5}{2 x 1 sub exp 2 add}
\end{pspicture}
&
\begin{pspicture}(-1.2,-1.2)(1.2,1.2)
%\psgrid
\psset{xunit=0.2,yunit=0.2}
\psaxes[arrows=<->,dx=0,Dx=10,dy=0,Dy=10](0,0)(-5,-5)(5,5)
\psplot[plotstyle=curve,arrows=<->]{-5}{3}{2 x 1 sub exp -1 mul 2 add}
\end{pspicture}
&
\begin{pspicture}(-1.2,-1.2)(1.2,1.2)
%\psgrid
\psset{xunit=0.2,yunit=0.2}
\psaxes[arrows=<->,dx=0,Dx=10,dy=0,Dy=10](0,0)(-5,-5)(5,5)
\psplot[plotstyle=curve,arrows=<->]{-5}{0.6}{2 x 1 add exp 2 add}
\end{pspicture}
&
\begin{pspicture}(-1.2,-1.2)(1.2,1.2)
%\psgrid
\psset{xunit=0.2,yunit=0.2}
\psaxes[arrows=<->,dx=0,Dx=10,dy=0,Dy=10](0,0)(-5,-5)(5,5)
\psplot[plotstyle=curve,arrows=<->]{-5}{1.9}{2 x 1 add exp -1 mul 3 add}
\end{pspicture}
\\\hline
$q<0$&
\begin{pspicture}(-1.2,-1.2)(1.2,1.2)
%\psgrid
\psset{xunit=0.2,yunit=0.2}
\psaxes[arrows=<->,dx=0,Dx=10,dy=0,Dy=10](0,0)(-5,-5)(5,5)
\psplot[plotstyle=curve,arrows=<->]{-5}{3}{2 x 1 sub exp 2 sub}
\end{pspicture}
&
\begin{pspicture}(-1.2,-1.2)(1.2,1.2)
%\psgrid
\psset{xunit=0.2,yunit=0.2}
\psaxes[arrows=<->,dx=0,Dx=10,dy=0,Dy=10](0,0)(-5,-5)(5,5)
\psplot[plotstyle=curve,arrows=<->]{-5}{2.5}{2 x 1 sub exp -1 mul 2 sub}
\end{pspicture}
&
\begin{pspicture}(-1.2,-1.2)(1.2,1.2)
%\psgrid
\psset{xunit=0.2,yunit=0.2}
\psaxes[arrows=<->,dx=0,Dx=10,dy=0,Dy=10](0,0)(-5,-5)(5,5)
\psplot[plotstyle=curve,arrows=<->]{-5}{1.9}{2 x 1 add exp 3 sub}
\end{pspicture}
&
\begin{pspicture}(-1.2,-1.2)(1.2,1.2)
%\psgrid
\psset{xunit=0.2,yunit=0.2}
\psaxes[arrows=<->,dx=0,Dx=10,dy=0,Dy=10](0,0)(-5,-5)(5,5)
\psplot[plotstyle=curve,arrows=<->]{-5}{0.6}{2 x 1 add exp -1 mul 2 sub}
\end{pspicture}
\\\hline
\end{tabular}
\end{center}
\end{table}

\subsection{Domain and Range}
For $y=ab^{(x+p)} + q$, the function is defined for all real values of $x$. Therefore, the domain is $\{x:x\in\mathbb{R}\}$.

The range of $y=ab^{(x+p)} + q$ is dependent on the sign of $a$.

If $a>0$ then:
\begin{eqnarray*}
b^{(x+p)}&>& 0\\
a\cdot b^{(x+p)} &>& 0\\
a\cdot b^{(x+p)}+q &>& q\\
f(x) &>& q
\end{eqnarray*}
Therefore, if $a>0$, then the range is $\{f(x):f(x)\in[q;\infty)\}$.

If $a<0$ then:
\begin{eqnarray*}
b^{(x+p)} &>& 0\\
a\cdot b^{(x+p)} &<& 0\\
a\cdot b^{(x+p)}+q &<& q\\
f(x) &<& q
\end{eqnarray*}
Therefore, if $a<0$, then the range is $\{f(x):f(x)\in(-\infty;q]\}$.

For example, the domain of $g(x)=3\cdot 2^{x+1} + 2$ is $\{x:x\in\mathbb{R}\}$.
For the range,
\begin{eqnarray*}
2^{x+1}&>&0\\
3 \cdot 2^{x+1}&>&0\\
3 \cdot 2^{x+1}+2&>&2
\end{eqnarray*}
Therefore the range is $\{g(x):g(x)\in[2;\infty)\}$.

\Exercise{Domain and Range}{
\begin{enumerate}
\item{Give the domain of  $y = 3^x$.}
\item{What is the domain and range of $f(x) = 2^x$ ?}
\item{Determine the domain and range of $y = (1,5)^{x+3}$.}

\end{enumerate}}

\subsection{Intercepts}
For functions of the form, $y=ab^{(x+p)} + q$, the intercepts with the $x$- and $y$-axis are calculated by setting $x=0$ for the $y$-intercept and by setting $y=0$ for the $x$-intercept.

The $y$-intercept is calculated as follows:
\begin{eqnarray}
y&=&ab^{(x+p)} + q\\
y_{int}&=&ab^{(0+p)} + q\\
&=&ab^p + q
\end{eqnarray}

For example, the $y$-intercept of $g(x)=3\cdot 2^{x+1} + 2$ is given by setting $x=0$ to get:
\begin{eqnarray*}
y&=&3\cdot 2^{x+1} + 2\\
y_{int}&=&3\cdot 2^{0+1} + 2\\
&=&3\cdot 2^{1} + 2\\
&=&3 \cdot 2 + 2\\
&=&8\\
\end{eqnarray*}

The $x$-intercepts are calculated by setting $y=0$ as follows:
\begin{eqnarray}
y&=&ab^{(x+p)} + q\\
0&=&ab^{(x_{int}+p)} + q\\
ab^{(x_{int}+p)}&=&-q\\
b^{(x_{int}+p)}&=&-\frac{q}{a}
\end{eqnarray}
Which only has a real solution if either $a<0$ or $q<0$ and $a\neq0$. Otherwise, the graph of the function of form $y=ab^{(x+p)} + q$ does not have any $x$-intercepts.

For example, the $x$-intercept of $g(x)=3\cdot 2^{x+1} + 2$ is given by setting $y=0$ to get:
\begin{eqnarray*}
y&=&3\cdot 2^{x+1} + 2\\
0&=&3\cdot 2^{x_{int}+1} + 2\\
-2&=&3\cdot 2^{x_{int}+1}\\
2^{x_{int}+1}&=&\frac{-2}{3}
\end{eqnarray*}
which has no real solution. Therefore, the graph of $g(x)=3\cdot 2^{x+1} + 2$ does not have a $x$-intercept.

\Exercise{Intercepts}{
\begin{enumerate}
\item{Give the $y$-intercept of the graph of $y = b^{x} + 2$.}
\item{Give the $x$- and $y$-intercepts of the graph of $y = \frac{1}{2}(1,5)^{x + 3} - 0,75$.}
\end{enumerate}
}

\subsection{Asymptotes}
The asymptote is the place at which the function is undefined.  For functions of the form $y=ab^{(x+p)} + q$ this is along the line where $y = q$.

For example, the asymptote of $g(x)=3\cdot 2^{x+1} + 2$ is $y = 2$.

\Exercise{Asymptotes}{
\begin{enumerate}
\item{Give the equation of the asymptote of the graph of $y = 3^x - 2$.}
\item{What is the equation of the horizontal asymptote of the \\graph of $y =
3(0,8)^{x-1} - 3$ ?}
\end{enumerate}
}

\subsection{Sketching Graphs of the Form $f(x)=ab^{(x+p)} + q$}
In order to sketch graphs of functions of the form, $f(x)=ab^{(x+p)} + q$, we need to determine four characteristics:
\begin{enumerate}
\item{domain and range}
\item{$y$-intercept}
\item{$x$-intercept}
\end{enumerate}

For example, sketch the graph of $g(x)=3\cdot 2^{x+1} + 2$. Mark the intercepts.

We have determined the domain to be $\{x:x\in\mathbb{R}\}$ and the range to be $\{g(x):g(x)\in(2;\infty)\}$.

The $y$-intercept is $y_{int}=8$ and there is no $x$-intercept.

\begin{figure}[H]
\begin{center}
\begin{pspicture}(-5,-1)(5,9)
%\psgrid
\psset{yunit=0.75,xunit=0.75}
\psaxes[arrows=<->](0,0)(-5,-1)(5,12)
\psplot[plotstyle=curve,arrows=<->]{-5}{0.73}{2 x 1 add exp 3 mul 2 add}
\end{pspicture}
\caption{Graph of $g(x)=3\cdot 2^{x+1} + 2$.}
\label{fig:mf:g:exponentialsketchexample}
\end{center}
\end{figure}

\Exercise{Sketching Graphs}{
\begin{enumerate}
\item{Draw the graphs of the following on the same set of axes.  Label the horizontal asymptotes and y-intercepts clearly.
\begin{enumerate}
\item{$y = 2^x + 2$}
\item{$y = 2^{x+2}$}
\item{$y = 2\cdot2^x$}
\item{$y = 2\cdot2^{x+2} + 2$}
\end{enumerate}}
\begin{enumerate}
\item{Draw the graph of $f(x) = 3^x$.}
\item{Explain where a solution of $3^x = 5$ can be read off the graph.}
\end{enumerate}
\end{enumerate}}

\section{End of Chapter Exercises}
\begin{enumerate}
\item{The following table of values has columns giving the $y$-values for the graph $y = a^x$, $y = a ^{x+1}$  and $y = a^x + 1$.  Match a graph to a column.
\begin{center}
\begin{tabular}{|c|c|c|c|}\hline
$x$ & $A$ & $B$ & $C$\\\hline
$-2$ & $7,25$ & $6,25$ & $2,5$\\\hline
$-1$ & $3,5$ & $2,5$ & $1$\\\hline
$0$ & $2$ & $1$ & $0,4$\\\hline
$1$ & $1,4$ & $0,4$ & $0,16$\\\hline
$2$ & $1,16$ & $0,16$ & $0,064$\\\hline
\end{tabular}
\end{center}
}
\item{The graph of $f(x) = 1+ a.2^x$ (a is a constant) passes through the origin.}{
\begin{enumerate}
 \item{Determine the value of $a$.}
 \item{Determine the value of $f(-15)$ correct to five decimal places.}
 \item{Determine the value of $x$, if $P(x; 0,5)$ lies on the graph of $f$.}
 \item{If the graph of $f$ is shifted $2$ units to the right to give the function $h$, write down the equation of $h$.}
  \end{enumerate}}
\item{The graph of $f(x) = a.b^x ~(a \neq 0)$ has the point $P(2;144)$ on $f$.}{
\begin{enumerate}
\item{If $b = 0,75$, calculate the value of $a$.}
\item{Hence write down the equation of $f$.}
\item{Determine, correct to two decimal places, the value of $f(13)$.}
\item{Describe the transformation of the curve of $f$ to $h$ if $h(x) = f(-x)$.}
\end{enumerate}}
\end{enumerate}



% CHILD SECTION END



% CHILD SECTION START

