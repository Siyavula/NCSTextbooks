\chapter{Introduction to Book}
\label{mathintro}

\section{The Language of Mathematics}
The purpose of any language, like English or Zulu, is to make it possible for people to communicate. All languages have an alphabet, which is a group of letters that are used to make up words. There are also rules of grammar which explain how words are supposed to be used to build up sentences. This is needed because when a sentence is written, the person reading the sentence understands exactly what the writer is trying to explain. Punctuation marks (like a full stop or a comma) are used to further clarify what is written.

Mathematics is a language, specifically it is the language of Science. Like any language, mathematics has letters (known as numbers) that are used to make up words (known as expressions), and sentences (known as equations). The punctuation marks of mathematics are the different signs and symbols that are used, for example, the plus sign (+), the minus sign (-), the multiplication sign ($\times$), the equals sign (=) and so on. There are also rules that explain how the numbers should be used together with the signs to make up equations that express some meaning.

%\section{Structure of Book}
%This book is divided into four parts, according to the four learning outcomes required by the National %Curriculum Statement for Mathematics (Grades 10 - 12). For the final Grade 12 examination, Parts 1 and 2 %will be examined in the first examination paper and Parts 3 and 4 will be examined in the second examination %paper.

%Each section in the text is marked as being for Grades 10, 11 or 12 and extra content not in the syllabus is %marked as \textbf{Advanced}. Where no indication of grade is given, these sections are to be considered as %introductory sections and are applicable for all grades, to clarify the content.

%The book is structured as follows:

%\begin{itemize}
%\item{\textbf{Building a Solid Foundation} contains a brief overview of key concepts that the reader should %be comfortable with, before the rest of the book is attempted.}
%\item{Part 1 - Learning Outcome 1: Numbers and Number Relationships}
%\begin{itemize}
%\item{Chapter: Numbers}
%\item{Chapter: Patterns in Numbers}
%\item{Chapter: Finance}
%\end{itemize}
%\item{Part 2 - Learning Outcome 2: Functions and Algebra}
%\begin{itemize}
%\item{Chapter: Functions and Graphs}
%\item{Chapter: Solving Equations}
%\item{Chapter: Differentiation}
%\item{Chapter: Linear Programming}
%\end{itemize}
%\item{Part 3 - Learning Outcome 3: Space, Shape and Measurement}
%\begin{itemize}
%\item{Chapter: Geometry}
%\item{Chapter: Trigonometry}
%\end{itemize}
%\begin{itemize}
%\item{Part 4 - Learning Outcome 4: Data Handling and Probability}
%\item{Chapter: Statistics}
%\item{Chapter: Probability}
%\item{Chapter: Combinations and Permutations}
%\end{itemize}
%\end{itemize}



%\section{How this book should be used}
%\nts{List how we see the book being used.}

% CHILD SECTION END 

% CHILD SECTION START 

\part{Grade 11}

% CHILD SECTION START 

