\chapter{The Chemical Industry}
\label{chap:chemindustry}

\section{Introduction}
\label{sec:chem:intro}

The chemical industry has been around for a very long time, but not always in the way we think of it today! Dyes, perfumes, medicines and soaps are all examples of products that have been made from chemicals that are found in either plants or animals. However, it was not until the time of the Industrial Revolution that the chemical industry as we know it today began to develop. At the time of the Industrial Revolution, the human population began to grow very quickly and more and more people moved into the cities to live. With this came an increase in the need for things like paper, glass, textiles and soaps. On the farms, there was a greater demand for fertilisers to help produce enough food to feed all the people in cities and rural areas. Chemists and engineers responded to these growing needs by using their technology to produce a variety of new chemicals. This was the start of the chemical industry.\\

In South Africa, the key event that led to the growth of the chemical industry was the discovery of diamonds and gold in the late 1800's. Mines needed explosives so that they could reach the diamonds and gold-bearing rock, and many of the main chemical companies in South Africa developed to meet this need for explosives. In this chapter, we are going to take a closer look at one of South Africa's major chemical companies, \textbf{Sasol}, and will also explore the \textbf{chloralkali} and \textbf{fertiliser} industries. 

\section{Sasol}
\label{sec:chemindustry:Sasol}

\textbf{Oil and natural gas} are important fuel resources. Unfortunately, South Africa has no large oil reserves and, until recently, had very little natural gas. One thing South Africa \textit{does} have however, is large supplies of \textbf{coal}. Much of South Africa's chemical industry has developed because of the need to produce oil and gas from coal, and this is where Sasol has played a very important role.\\

Sasol was established in 1950, with its main aim being to convert low grade coal into petroleum (crude oil) products and other chemical feedstocks. A 'feedstock' is something that is used to make another product. Sasol began producing oil from coal in 1955.

\begin{IFact}{The first interest in coal chemistry started as early as the 1920's. In the early 1930's a research engineer called Etienne Rousseau was employed to see whether oil could be made from coal using a new German technology called the \textbf{Fischer-Tropsch} process. After a long time, and after many negotiations, Rousseau was given the rights to operate a plant using this new process. As a result, the government-sponsored 'South African Coal, Oil and Gas Corporation Ltd' (commonly called 'Sasol') was formed in 1950 to begin making oil from coal. A manufacturing plant was established in the Free State and the town of \textbf{Sasolburg} developed around this plant. Production began in 1955. In 1969, the \textbf{Natref} crude oil refinery was established, and by 1980 and 1982 Sasol Two and Sasol Three had been built at \textbf{Secunda}.}
\end{IFact}

\subsection{Sasol today: Technology and production}
\label{subsec:chem:technology}

Today, Sasol is an oil and gas company with diverse chemical interests. Sasol has three main areas of operation: Firstly, \textbf{coal to liquid fuels technology}, secondly the production of \textbf{crude oil} and thirdly the conversion of \textbf{natural gas to liquid fuel}.

\begin{enumerate}
\item{\textbf{Coal to liquid fuels}

Sasol is involved in mining coal and converting it into synthetic fuels, using the \textbf{Fischer-Tropsch} technology. Figure \ref{fig:Sasol:gasification} is a simplified diagram of the process that is involved.

\begin{center}
\begin{figure}[h]
\begin{pspicture}(-8,-4)(8.5,4)
%\psgrid[gridcolor=lightgray]
\psframe[linewidth=1pt](-9,-1)(-7,1)
\rput(-8,0){\textbf{Coal mining}}
\psline[linewidth=2pt,arrows=->](-7,0)(-6,0)
\psframe[linewidth=2pt,fillstyle=solid,fillcolor=lightgray](-6,-2)(-4,2)
\rput(-5,0.6){\textbf{Coal}}
\rput(-5,0.2){\textbf{gasification}}
\rput(-5,-0.6){(Sasol/Lurgi}
\rput(-5,-1){process)}
\psframe[linewidth=1pt](-3,-0.5)(0,0.5)
\rput(-1.5,0.2){Crude synthesis}
\rput(-1.5,-0.2){gas}
\psline[linewidth=2pt,arrows=->](-4,0)(-3,0)
\psframe[linewidth=1pt](1,-3)(3,-2)
\rput(2,-2.3){Gas}
\rput(2,-2.7){purification}
\psline[linewidth=2pt,arrows=->](0,-0.5)(1,-2)
\psline[linewidth=2pt,arrows=->](0,0.5)(1,2)
\rput(4,2){Condensates from the gas are cooled}
\rput(4,1.6){to produce tars, oils and pitches.}
\rput(4,1.2){Ammonia, sulfur and phenolics are}
\rput(4,0.8){also recovered.}
\psline[linewidth=2pt,arrows=->](3,-2.5)(4,-2.5)
\psframe[linewidth=2pt,fillstyle=solid,fillcolor=lightgray](4,-4)(6,-1)
\rput(5,-2.3){\textbf{SAS}}
\rput(5,-2.7){\textbf{reactor}}
\psline[linewidth=2pt,arrows=->](6,-2.5)(7,-2.5)
\rput(8,-2.3){\textbf{C$_{1}$ to C$_{20}$}}
\rput(8,-2.7){\textbf{hydrocarbons}}
\end{pspicture}
\caption{The gasification of coal to produce liquid fuels}
\label{fig:Sasol:gasification}
\end{figure}
\end{center}
}

\textbf{Coal gasification} is also known as the \textbf{Sasol/Lurgi} gasification process, and involves converting low grade coal to a synthesis gas. Low grade coal has a low percentage carbon, and contains other impurities. The coal is put under extremely high pressure and temperature in the presence of steam and oxygen. The gas that is produced has a high concentration of hydrogen ($H_{2}$) and carbon monoxide (CO). That is why it is called a 'synthesis gas', because it is a mixture of more than one gas.\\

In the \textbf{Sasol Advanced Synthol (SAS) reactors}, the gas undergoes a high temperature Fischer-Tropsch conversion. Hydrogen and carbon monoxide react under high pressure and temperature and in the presence of an iron catalyst, to produce a range of hydrocarbon products. Below is the generalised equation for the process. Don't worry too much about the numbers that you see in front of the reactants and products. It is enough just to see that the reaction of hydrogen and carbon monoxide (the two gases in the \textit{synthesis gas}) produces a hydrocarbon and water. 

\begin{center}
$\rm{(2n+1)H_{2} + nCO \rightarrow C_{n}H_{2n+2} + nH_{2}O}$\\
\end{center}

A range of hydrocarbons are produced, including petrol, diesel, jet fuel, propane, butane, ethylene, polypropylene, alcohols and acetic acids.

\Tip{\textbf{Different types of fuels}\\

It is important to understand the difference between types of fuels and the terminology that is used for them. The table below summarises some of the fuels that will be mentioned in this chapter.\\

\begin{center}
\begin{tabular}{|p{2.5cm}|p{7.5cm}|}\hline
\textbf{Compound} & \textbf{Description} \\\hline
Petroleum (crude oil) & A naturally occurring liquid that forms in the earth's lithosphere (see Grade 11 notes). It is a mixture of hydrocarbons, mostly alkanes, ranging from C$_{5}$H$_{12}$ to C$_{18}$H$_{38}$. \\\hline
Natural gas & Natural gas has the same origin as petroleum, but is made up of shorter hydrocarbon chains.\\\hline
Paraffin wax & This is made up of longer hydrocarbon chains, making it a solid compound.\\\hline
Petrol (gasoline) & A liquid fuel that is derived from petroleum, but which contains extra additives to increase the octane rating of the fuel. Petrol is used as a fuel in combustion engines.\\\hline
Diesel & Diesel is also derived from petroleum, but is used in diesel engines.\\\hline
Liquid Petroleum Gas (LPG) & LPG is a mixture of hydrocarbon gases, and is used as a fuel in heating appliances and vehicles. Some LPG mixtures contain mostly propane, while others are mostly butane. LPG is manufactured when crude oil is refined, or is extracted from natural gas supplies in the ground.\\\hline
Paraffin & This is a technical name for the alkanes, but refers specifically to the \textit{linear} alkanes. \textit{Isoparaffin} refers to non-linear (branched) alkanes.\\\hline
Jet fuel & A type of aviation fuel designed for use in jet engined aircraft. It is an oil-based fuel and contains additives such as antioxidants, corrosion inhibitors and icing inhibitors.\\\hline
\end{tabular}
\end{center}
}

You will notice in the diagram that Sasol doesn't only produce liquid fuels, but also a variety of other chemical products. Sometimes it is the synthetic fuels themselves that are used as feedstocks to produce these chemical products. This is done through processes such as \textbf{hydrocracking} and \textbf{steamcracking}. Cracking is when heavy hydrocarbons are converted to simpler light hydrocarbons (e.g. LPG and petrol) through the breaking of C-C bonds. A heavy hydrocarbon is one that has a high number of hydrogen and carbon atoms (more solid), and a light hydrocarbon has fewer hydrogen and carbon atoms and is either a liquid or a gas.\\

\Definition{Hydrocracking}{
Hydrocracking is a cracking process that is assisted by the presence of an elevated partial pressure of hydrogen gas. It produces chemical products such as ethane, LPG, isoparaffins, jet fuel and diesel.
}

\Definition{Steam cracking}{
Steam cracking occurs under very high temperatures. During the process, a liquid or gaseous hydrocarbon is diluted with steam and then briefly heated in a furnace at a temperature of about $850^{circ}C$. Steam cracking is used to convert \textit{ethane} to \textit{ethylene}. Ethylene is a chemical that is needed to make plastics. Steam cracking is also used to make propylene, which is an important fuel gas.
}
 
\item{\textbf{Production of crude oil}

Sasol obtains crude oil off the coast of Gabon (a country in West Africa) and refines this at the Natref refinery (figure \ref{fig:Sasol:refining crude oil}). Sasol also sells liquid fuels through a number of service stations.

\begin{center}
\begin{figure}[h]
\begin{pspicture}(-8,-1)(8.5,1)
%\psgrid[gridcolor=lightgray]
\psframe[linewidth=1pt](-7,-1)(-5,1)
\rput(-6,0.2){\textbf{Imported}}
\rput(-6,-0.2){\textbf{crude oil}}
\psline[linewidth=2pt,arrows=->](-5,0)(-1,0)
\psframe[linewidth=1pt](-1,-0.8)(2,0.8)
\rput(0.5,0.4){Oil processed}
\rput(0.5,0){at Natref}
\rput(0.5,-0.4){refinery}
\psline[linewidth=2pt,arrows=->](2,0)(4,0)
\rput(6.2,0.2){Linear-chained hydrocarbons}
\rput(6.2,-0.2){e.g. waxes, paraffins}
\rput(6.2,-0.6){and diesel.}
\end{pspicture}
\caption{Crude oil is refined at Sasol's Natref refinery to produce liquid fuels}
\label{fig:Sasol:refining crude oil}
\end{figure}
\end{center}
}

\item{\textbf{Liquid fuels from natural gas}

Sasol produces natural gas in Mozambique and is expanding its 'gas to fuel' technology. The gas undergoes a complex process to produce linear-chained hydrocarbons such as waxes and paraffins (figure \ref{fig:Sasol:gas to fuel}).

\begin{center}
\begin{figure}[h]
\begin{pspicture}(-8,-2.5)(8,2)
%\psgrid[gridcolor=lightgray]
\psframe[linewidth=1pt](-7,-1)(-5,1)
\rput(-6,0.5){\textbf{Mozambique}}
\rput(-6,0.1){\textbf{natural}}
\rput(-6,-0.3){\textbf{gas}}
\psline[linewidth=2pt,arrows=->](-5,0)(-3,0)
\psframe[linewidth=2pt,fillstyle=solid,fillcolor=lightgray](-3,-2)(-1,2)
\rput(-2,0.2){\textbf{Autothermal}}
\rput(-2,-0.2){\textbf{reactor}}
\psline[linewidth=2pt,arrows=->](-1,0)(1,0)
\psframe[linewidth=2pt,fillstyle=solid,fillcolor=lightgray](1,-2)(3,2)
\rput(2,0.6){\textbf{Sasol Slurry}}
\rput(2,0.2){\textbf{Phase F-T}}
\rput(2,-0.2){\textbf{reactor}}
\psline[linewidth=2pt,arrows=->](3,0)(5,0)
\rput(7,0.2){Linear-chained hydrocarbons}
\rput(7,-0.2){e.g. waxes and paraffins}
\end{pspicture}
\caption{Conversion of natural gas to liquid fuels}
\label{fig:Sasol:gas to fuel}
\end{figure}
\end{center}
}

In the \textbf{autothermal reactor}, methane from natural gas reacts with steam and oxygen over an iron-based catalyst to produce a \textit{synthesis gas}. This is a similar process to that involved in coal gasification. The oxygen is produced through the \textbf{fractional distillation of air}. 

\Definition{Fractional distillation}{
Fractional distillation is the separation of a mixture into its component parts, or fractions. Since air is made up of a number of gases (with the major component being nitrogen), fractional distillation can be used to separate it into these different parts.
}

The syngas is then passes through a \textbf{Sasol Slurry Phase Distillate (SSPD)} process. In this process, the gas is reacted at far lower temperatures than in the SAS reactors. Apart from hard wax and candle wax, high quality diesel can also be produced in this process. Residual gas from the SSPD process is sold as pipeline gas while some of the lighter hydrocarbons are treated to produce kerosene and paraffin. Ammonia is also produced, which can be used to make fertilisers.
\end{enumerate}

\begin{IFact}{Sasol is a major player in the emerging Southern African natural gas industry, after investing 1.2 billion US dollars to develop onshore gas fields in central Mozambique. Sasol has been supplying natural gas from Mozambique's Temane field to customers in South Africa since 2004.}
\end{IFact}

\Exercise{Sasol processes\\}{
\textit{Refer to the diagrams summarising the three main Sasol processes, and use these to answer the following questions:\\}

\begin{enumerate}
\item{Explain what is meant by each of the following terms:
	\begin{enumerate}
	\item{crude oil}
	\item{hydrocarbon}
	\item{coal gasification}
	\item{synthetic fuel}
	\item{chemical feedstock}
	\end{enumerate}}

\item{
	\begin{enumerate}
	\item{What is diesel?}
	\item{Describe two ways in which diesel can be produced.}
	\end{enumerate}
}

\item{Describe one way in which lighter chemical products such as ethylene, can be produced.} 

\item{Coal and oil play an important role in Sasol's technology. }
	\begin{enumerate}
	\item{In the table below, summarise the similarities and differences between coal, oil and natural gas in terms of how they are formed ('origin'), their general chemical formula and whether they are solid, liquid or gas.}

\begin{tabular}{|p{2cm}|p{2cm}|p{2cm}|p{2cm}|}\hline
 & Coal & Oil & Natural gas \\\hline
Origin & & &\\\hline
General chemical formula & & & \\\hline
Solid, liquid or gas & & & \\\hline
\end{tabular}

	\item{In your own words, describe how coal is converted into liquid fuels.}
	\item{Explain why Sasol's 'coal to liquid fuels' technology is so important in meeting South Africa's fuel needs.}
	\item{Low grade coal is used to produce liquid fuels. What is the main use of higher grade coal in South Africa?}
	\end{enumerate}

\end{enumerate}

\insertpracticeinfo{4}
}

\Activity{Case Study}{Safety issues and risk assessments\\}{
Safety issues are important to consider when dealing with industrial processes. Read the following extract that appeared in the Business report on 6th February 2006, and then discuss the questions that follow.\\

\begin{quote}{
Cape Town - Sasol, the petrochemicals group, was likely to face prosecution on 10 charges of culpable homicide after an explosion at its Secunda plant in 2004 in which 10 people died, a Cape Town labour law specialist said on Friday.
The specialist, who did not want to be named, was speaking after the inquiry into the explosion was concluded last Tuesday. It was convened by the labour department.

The evidence led at the inquiry showed a failure on the part of the company to conduct a proper risk assessment and that:
Sasol failed to identify hazards associated with a high-pressure gas pipeline running through the plant, which had been shut for extensive maintenance work, in the presence of hundreds of people and numerous machines, including cranes, fitters, contractors, and welding and cutting machines. Because there had never been a risk assessment, the hazard of the high-pressure pipeline had never been identified.

Because Sasol had failed to identify the risk, it did not take any measures to warn people about it, mark the line or take precautions. There had also been inadequacy in planning the shutdown work. In the face of a barrage of criticism for the series of explosions that year, Sasol embarked on a comprehensive programme to improve safety at its operations and appointed Du Pont Safety Resources, the US safety consultancy, to benchmark the petrochemical giant's occupational health and safety performance against international best practice.}
\end{quote}

\begin{enumerate}
\item{Explain what is meant by a 'risk assessment'.}
\item{Imagine that you have been asked to conduct a risk assessment of the Sasol/Lurgi gasification process. What information would you need to know in order to do this assessment?}
\item{In groups, discuss the importance of each of the following in ensuring the safety of workers in the chemical industry:}

\begin{itemize}
\item{employing experienced Safety, Health and Environment personnel}
\item{regular training to identify hazards}
\item{equipment maintenance and routine checks}
\end{itemize}

\item{What other precautions would you add to this list to make sure that working conditions are safe?}

\end{enumerate}

}

\subsection{Sasol and the environment}

From its humble beginnings in 1950, Sasol has grown to become a major contributor towards the South African economy. Today, the industry produces more than 150 000 barrels of fuels and petrochemicals per day, and meets more than 40\% of South Africa's liquid fuel requirements. In total, more than 200 fuel and chemical products are manufactured at Sasolburg and Secunda, and these products are exported to over 70 countries worldwide. This huge success is largely due to Sasol's ability to diversify its product base. The industry has also helped to provide about 170 000 jobs in South Africa, and contributes around R40 billion to the country's Gross Domestic Product (GDP).\\

However, despite these obvious benefits, there are always environmental costs associated with industry. Apart from the vast quantities of resources that are needed in order for the industry to operate, the production process itself produces waste products and pollutants.

\Exercise{Consumption of resources\\}{Any industry will always use up huge amounts of resources in order to function effectively, and the chemical industry is no exception. In order for an industry to operate, some of the major resources that are needed are \textbf{energy} to drive many of the processes, \textbf{water}, either as a coolant or as part of a process and \textbf{land} for mining or operations.\\

Refer to the data table below which shows Sasol's water use between 2002 and 2005 (\textit{Sasol Sustainable Development Report 2005}), and answer the questions that follow.\\

\begin{center}
\begin{tabular}{|l|c|c|c|c|}\hline
\textbf{Water use} ($1000m^{3}$) & 2002 & 2003 & 2004 & 2005\\\hline\hline
River water & 113 722 & 124 179 & 131 309 & 124 301\\\hline
Potable water & 15 126 & 10 552 & 10 176 & 10 753\\\hline
\textbf{Total} & \textbf{157 617} & \textbf{178 439} & \textbf{173 319} & \textbf{163 203}\\\hline
\end{tabular}
\end{center}

	\begin{enumerate}
	\item{Explain what is meant by 'potable' water.}
	\item{Describe the trend in Sasol's water use that you see in the above statistics.}
	\item{Suggest possible reasons for this trend.}
	\item{List some of the environmental impacts  of using large amounts of river water for industry.}
	\item{Suggest ways in which these impacts could be reduced}
	\end{enumerate}

\insertpracticeinfo{1}
}

\Exercise{Industry and the environment\\}{

Large amounts of gases and pollutants are released during production, and when the fuels themselves are used. Refer to the table below, which shows greenhouse gas and atmospheric pollution data for Sasol between 2002 and 2005, and then answer the questions that follow. (\textit{Source: Sasol Sustainable Development Report 2005})\\

\begin{center}
\begin{tabular}{|p{4cm}|c|c|c|c|}\hline
\textbf{Greenhouse gases and air pollutants} (kilotonnes) & 2002 & 2003 & 2004 & 2005\\\hline\hline
Carbon dioxide ($CO_{2}$) & 57 476 & 62 873 & 66 838 & 60 925\\\hline
Hydrogen sulfide ($H_{2}S$) & 118 & 105 & 102 & 89\\\hline
Nitrogen oxides ($NO_{x}$) & 168 & 173 & 178 & 166\\\hline
Sulfur dioxide ($SO_{2}$) & 283 & 239 & 261 & 222\\\hline
\end{tabular}
\end{center}

\begin{enumerate}
\item{Draw line graphs to show how the quantity of each pollutant produced has changed between 2002 and 2005.}
\item{Describe what you see in the graphs, and suggest a reason for this trend.}
\item{Explain what is meant by each of the following terms:
	\begin{enumerate}
	\item{greenhouse gas}
	\item{global warming}
	\end{enumerate}}
\item{Describe some of the possible effects of global warming.}
\item{When sulfur dioxide is present in the atmosphere, it may react with water vapour to produce \textit{sulfuric acid}. In the same way, nitrogen dioxide and water vapour react to form \textit{nitric acid}. These reactions in the atmosphere may cause \textbf{acid rain}. Outline some of the possible consequences of acid rain.}

\item{Many industries are major contributors towards environmental problems such as global warming, environmental pollution, over-use of resources and acid rain. Industries are in a difficult position: On one hand they must meet the ever increasing demands of society, and on the other, they must achieve this with as little environmental impact as possible. This is a huge challenge.

\begin{itemize}
\item{Work in groups of 3-4 to discuss ways in which industries could be encouraged (or in some cases forced) to reduce their environmental impact.}
\item{Elect a spokesperson for each group, who will present your ideas to the class.}
\item{Are the ideas suggested by each group practical?}
\item{How easy or difficult do you think it would be to implement these ideas in South Africa?}
\end{itemize}}
\end{enumerate}

\insertpracticeinfo{1}
}

\begin{IFact}{Sasol is very aware of its responsibility towards creating cleaner fuels. From 1st January 2006, the South African government enforced a law to prevent lead from being added to petrol. Sasol has complied with this. One branch of Sasol, \textbf{Sasol Technology} also has a bio-diesel research and development programme focused on developing more environmentally friendly forms of diesel. One way to do this is to use renewable resources such as soybeans to make diesel. Sasol is busy investigating this new technology.
}
\end{IFact}

\section{The Chloralkali Industry}
\label{sec:chemical:chloralkali}

The chlorine-alkali (chloralkali) industry is an important part of the chemical industry, and produces \textbf{chlorine} and \textbf{sodium hydroxide} through the electrolysis of table salt (NaCl). The main raw material is \textbf{brine} which is a saturated solution of sodium chloride (NaCl) that is obtained from natural salt deposits. \\

The products of this industry have a number of important uses. \textbf{Chlorine} is used to purify water, and is used as a disinfectant. It is also used in the manufacture of many every-day items such as hypochlorous acid, which is used to kill bacteria in drinking water. Chlorine is also used in paper production, antiseptics, food, insecticides, paints, petroleum products, plastics (such as polyvinyl chloride or PVC), medicines, textiles, solvents, and many other consumer products. Many chemical products such as chloroform and carbon tetrachloride also contain chlorine.\\

\textbf{Sodium hydroxide} (also known as 'caustic soda') has a number of uses, which include making soap and other cleaning agents, purifying bauxite (the ore of aluminium), making paper and making rayon (artificial silk).

\subsection{The Industrial Production of Chlorine and Sodium Hydroxide}

Chlorine and sodium hydroxide can be produced through a number of different reactions. However, one of the problems is that when chlorine and sodium hydroxide are produced together, the chlorine combines with the sodium hydroxide to form chlorate ($ClO^{-}$) and chloride ($Cl^{-}$) ions. This produces sodium chlorate, NaClO, a component of household bleach. To overcome this problem the chlorine and sodium hydroxide must be separated from each other so that they don't react. There are three industrial processes that have been designed to overcome this problem, and to produce chlorine and sodium hydroxide. All three methods involve \textbf{electrolytic cells} (chapter \ref{chap:electrochemical}). 
\\

\Tip{\textbf{Electrolytic cells}\\

Electrolytic cells are used to transform reactants into products by using electric current. They are made up of an \textbf{electrolyte} and two electrodes, the \textbf{cathode} and the \textbf{anode}. An electrolytic cell is activated by applying an external electrical current. This creates an electrical potential across the cathode and anode, and forces a chemical reaction to take place in the electrolyte. Cations flow towards the cathode and are reduced. Anions flow to the anode and are oxidised. Two new products are formed, one product at the cathode and one at the anode. }


\begin{enumerate}

\item{\textbf{The Mercury Cell}}

In the mercury-cell (figure \ref{fig:mercury cell}), brine passes through a chamber which has a carbon electrode (the anode) suspended from the top. Mercury flows along the floor of this chamber and acts as the cathode. When an electric current is applied to the circuit, chloride ions in the electrolyte are oxidised to form chlorine gas.

\begin{center}
$\rm{2Cl^{-}_{(aq)} \rightarrow Cl_{2(g)} + 2e^{-}}$ 
\end{center}

At the cathode, sodium ions are reduced to sodium.  
\begin{center}
$\rm{2Na^{+}_{(aq)} + 2e^{-} \rightarrow 2Na_{(Hg)}}$ 
\end{center}

The sodium dissolves in the mercury, forming an amalgam of sodium and mercury. The amalgam is then poured into a separate vessel, where it decomposes into sodium and mercury. The sodium reacts with water in the vessel and produces sodium hydroxide (caustic soda) and hydrogen gas, while the mercury returns to the electrolytic cell to be used again.\\

\begin{center}
$\rm{2Na_{(Hg)} + 2H_{2}O_{(l)} \rightarrow 2NaOH_{(aq)} + H_{2(g)}}$ 
\end{center}

\begin{figure}[h]
\begin{center}
\begin{pspicture}(-5,-5)(5,3)
\def\whitearrow{\psframe[fillstyle=solid,fillcolor=white,linestyle=none](-0.2,-0.2)(0.2,0.2)\psline{->}(-0.5,0)(0.5,0)}

%\psgrid[gridcolor=gray]

\rput(0,2.5){\textbf{Main vessel}}
\rput(4,0.5){\textbf{Secondary vessel}}

\psframe[fillstyle=vlines,fillcolor=lightgray](-4,0)(0,0.6)
\uput[d](-2,0){mercury cathode (-)}
\psframe(-4,0)(0,2)
\psframe[fillstyle=solid,fillcolor=lightgray,linestyle=none](-2.3,1)(-1.7,2.6)
\psframe[fillstyle=solid,fillcolor=lightgray,linestyle=none](-3,1)(-1,1.6)
\rput(-2,1.3){Carbon anode (+)}


\rput(-3.5,1){\uput[l](-1,0){NaCl}\rput(-0.5,0){\whitearrow}}
\rput(0.5,1.6){\uput[r](0,0){NaCl}\rput(-0.5,0){\whitearrow}}
\rput(-3.5,1.5){\uput[u](0,1){Cl$_2$}\rput{90}(0,0.5){\whitearrow}}

\psframe(0,-4)(4,-1)
\psframe[fillstyle=solid,fillcolor=white,linestyle=none](-0.2,0.8)(0.2,1.2)
\psframe[fillstyle=solid,fillcolor=white,linestyle=none](0.8,-1.2)(1.2,-0.8)
\psline{->}(-0.5,1)(1,1)(1,-2)
\uput[r](1,-0.4){\parbox{2cm}{sodium mercury amalgam}}

\rput{90}(3.5,-1){\whitearrow}\uput[u](3.5,-0.5){H$_2$}
\rput(4.5,-2.0){\uput[r](0,0){NaOH}\rput(-0.5,0){\whitearrow}}
\rput(4.5,-4.5){\uput[r](0,1){H$_2$O}\rput{180}(-0.5,1.5){\whitearrow}}


\psframe[fillstyle=solid,fillcolor=white,linestyle=none](-0.2,-2.8)(0.2,-2.4)
\psline{->}(0.4,-2.6)(-4.4,-2.6)(-4.4,0.4)(-4,0.4)
\uput[dr](-4.4,-2.6){\parbox{3.9cm}{mercury returned to the electrolytic cell}}


%\psline(1.5,0)(1.5,5)\psline(1.5,0.6)(4.2,0.6)\uput[r](4.2,0.6){membrane}
\end{pspicture}
\caption{The Mercury Cell}
\label{fig:mercury cell}
\end{center}
\end{figure}

This method, however, only produces a fraction of the chlorine and sodium hydroxide that is used by industry as it has certain disadvantages: mercury is expensive and toxic, and although it is returned to the electrolytic cell, some always escapes with the brine that has been used. The mercury reacts with the brine to form mercury(II) chloride. In the past this effluent was released into lakes and rivers, causing mercury to accumulate in fish and other animals feeding on the fish. Today, the brine is treated before it is discharged so that the environmental impact is lower. 

\item{\textbf{The Diaphragm Cell}}

In the diaphragm-cell (figure \ref{fig:diaphragm cell}), a porous diaphragm divides the electrolytic cell, which contains brine, into an anode compartment and a cathode compartment. The brine is introduced into the anode compartment and flows through the diaphragm into the cathode compartment. When an electric current passes through the brine, the salt's chlorine ions and sodium ions move to the electrodes. Chlorine gas is produced at the anode. At the cathode, sodium ions react with water, forming caustic soda and hydrogen gas. Some salt remains in the solution with the caustic soda and can be removed at a later stage. 

\begin{figure}[htbp]
\begin{center}
\begin{pspicture}(-2,-0.2)(5,6.6)
\def\whitearrow{\psframe[fillstyle=solid,fillcolor=white,linestyle=none](0,0.25)(1,0.75)\psline{->}(0,0.5)(1,0.5)}
%\psgrid
\rput(0,0){\psframe(-1,0)(4,5)}
\psframe[fillstyle=solid,fillcolor=lightgray](0.25,1)(0.75,4.5)
\rput(2,0){\psframe(0.25,1)(0.75,4.5)}
\psline(0.5,4.5)(0.5,6)
\psline(2.5,4.5)(2.5,6)
\rput(2.5,6.25){\textbf{--}}
\rput(0.5,6.25){\textbf{+}}
\pscircle(2.5,6.25){0.25}
\pscircle(0.5,6.25){0.25}
\uput[l](-1,3){anode}\psline(-1.1,3)(0.5,3)
\uput[r](4,3){cathode}\psline(2.5,3)(4.1,3)
\rput(-0.5,0){\uput[l](-1,4){NaCl}\rput(-1,3.5){\whitearrow}}
\rput{90}(0,4.5){\whitearrow}\uput[u](-0.5,5.5){Cl$_2$}
\rput{90}(4,4.5){\whitearrow}\uput[u](3.5,5.5){H$_2$}
\psline[linestyle=dashed](1.5,0)(1.5,5)
\psline(1.5,0.6)(4.2,0.6)\uput[r](4.2,0.6){porous diaphragm}
\end{pspicture}
\caption{Diaphragm Cell}
\label{fig:diaphragm cell}
\end{center}
\end{figure}

This method uses less energy than the mercury cell, but the sodium hydroxide is not as easily concentrated and precipitated into a useful substance.

\begin{IFact}{To separate the chlorine from the sodium hydroxide, the two half-cells were traditionally separated by a porous asbestos diaphragm, which needed to be replaced every two months. This was damaging to the environment, as large quantities of asbestos had to be disposed. Today, the asbestos is being replaced by other polymers which do not need to be replaced as often.}
\end{IFact}

\item{\textbf{The Membrane Cell}}

The membrane cell (figure \ref{fig:membranecell}) is very similar to the diaphragm cell, and the same reactions occur. The main difference is that the two electrodes are separated by an ion-selective membrane, rather than by a diaphragm. The structure of the membrane is such that it allows cations to pass through it between compartments of the cell. It does not allow anions to pass through. This has nothing to do with the size of the pores, but rather with the charge on the ions. Brine is pumped into the anode compartment, and only the positively charged sodium ions pass into the cathode compartment, which contains pure water.\\

\begin{figure}[htbp]
\begin{center}
\begin{pspicture}(-2,-0.2)(5,6.6)
\def\whitearrow{\psframe[fillstyle=solid,fillcolor=white,linestyle=none](0,0.25)(1,0.75)\psline{->}(0,0.5)(1,0.5)}
%\psgrid
\rput(0,0){\psframe(-1,0)(4,5)}
\psframe[fillstyle=solid,fillcolor=lightgray](0.25,1)(0.75,4.5)
\rput(2,0){\psframe(0.25,1)(0.75,4.5)}
\psline(0.5,4.5)(0.5,6)
\psline(2.5,4.5)(2.5,6)

\rput(2.5,6.25){\textbf{--}}\pscircle(2.5,6.25){0.25}
\rput(0.5,6.25){\textbf{+}}\pscircle(0.5,6.25){0.25}

\uput[l](-1,3){anode}\psline(-1.1,3)(0.5,3)
\uput[r](4,3){cathode}\psline(2.5,3)(4.1,3)

\rput(-0.5,0){\uput[l](-1,4){NaCl}\rput(-1,3.5){\whitearrow}}
\rput{90}(0,4.5){\whitearrow}\uput[u](-0.5,5.5){Cl$_2$}
\rput{90}(4,4.5){\whitearrow}\uput[u](3.5,5.5){H$_2$}
\rput(-0.5,0){\uput[l](-1,1){NaCl}\rput{180}(0,1.5){\whitearrow}}

\rput(4.5,0){\uput[r](0,4){NaOH}\rput(-1,3.5){\whitearrow}}
\rput(4.5,0){\uput[r](0,1){H$_{2}$O}\rput{180}(0,1.5){\whitearrow}}



\psline(1.5,0)(1.5,5)
\psline(1.5,0.6)(4.2,0.6)\uput[r](4.2,0.6){membrane}
\end{pspicture}
\caption{Membrane Cell}
\label{fig:membranecell}
\end{center}
\end{figure}

At the positively charged anode, $Cl^{-}$ ions from the brine are oxidised to $Cl_{2}$ gas.

\begin{center}
$\rm{2Cl^{-} \rightarrow Cl_{2(g)} + 2e^{-}}$
\end{center}

At the negatively charged cathode, hydrogen ions in the water are reduced to hydrogen gas.

\begin{center}
$\rm{2H^{+}_{(aq)} + 2e^{-} \rightarrow H_{2(g)}}$
\end{center}

The Na$^{+}$ ions flow through the membrane to the cathode compartment and react with the remaining hydroxide ($OH^{-}$) ions from the water to form sodium hydroxide (NaOH). The chloride ions cannot pass through, so the chlorine does not come into contact with the sodium hydroxide in the cathode compartment. The sodium hydroxide is removed from the cell. The overall equation is as follows:

\begin{center}
$\rm{2NaCl + 2H_{2}O \rightarrow Cl_{2} + H_{2} + 2NaOH}$
\end{center}

The advantage of using this method is that the sodium hydroxide that is produced is very pure because it is kept separate from the sodium chloride solution. The caustic soda therefore has very little salt contamination. The process also uses less electricity and is cheaper to operate.
\end{enumerate}

\Exercise{The Chloralkali industry\\}{

\begin{enumerate}
\item{Refer to the flow diagram below which shows the reactions that take place in the membrane cell, and then answer the questions that follow.\\

\begin{pspicture}(-6,-5.2)(6,5.2)
%\psgrid[gridcolor=lightgray]
\rput(-2,5){\parbox{2.5cm}{ANODE}}
\rput(-2,4){\parbox{2.5cm}{NaCl is added to this compartment}}
\rput(2,5){\parbox{2.5cm}{CATHODE}}
\rput(2,4){\parbox{2.5cm}{(a)}}
\rput(-2,2){\parbox{2.5cm}{Cl$^{-}$ ions}}
\rput(-4,2){\parbox{2.5cm}{(b)}}
\rput(2,2){\parbox{2.5cm}{H$^{+}$ ions are reduced to H$_{2}$ gas}}
\rput(-2,0){\parbox{2.5cm}{Na$^{+}$ ions in solution}}
\rput(2,0){\parbox{2.5cm}{OH$^{-}$ ions in solution}}
\rput(0,-2){(c)}
\rput(0,-4){\parbox{2.5cm}{Na$^{+}$ and OH$^{-}$ ions react to form NaOH}}
\psline{->}(-2,3.2)(-2,2.4)
\psline{->}(2,3.2)(2,2.4)
\psline{->}(2,1.2)(2,0.4)
\psline{->}(-2,1.2)(-2,0.4)
\psline(-2,-0.5)(0,-1.5)
\psline(2,-0.5)(0,-1.5)
\psline{->}(0,-2.5)(0,-3.2)
\psline{->}(-3.5,2)(-4.5,2)
\end{pspicture}

	\begin{enumerate}
	\item{What liquid is present in the cathode compartment at (a)?}
	\item{Identify the gas that is produced at (b).}
	\item{Explain one feature of this cell that allows the Na$^{+}$ and OH$^{-}$ ions to react at (c).}
	\item{Give a balanced equation for the reaction that takes place at (c).}
	\end{enumerate}
}

\item{Summarise what you have learnt about the three types of cells in the chloralkali industry by completing the table below:

\begin{center}
\begin{tabular}{|p{3cm}|p{2cm}|p{2cm}|p{2cm}|}\hline
 & \textbf{Mercury cell} & \textbf{Diaphragm cell} & \textbf{Membrane cell} \\\hline
Main raw material & & & \\\hline
Mechanism of separating Cl$_{2}$ and NaOH & & & \\\hline
Anode reaction & & & \\\hline
Cathode reaction & & & \\\hline
Purity of NaOH produced & & & \\\hline
Energy consumption & & & \\\hline
Environmental impact & & & \\\hline
\end{tabular}
\end{center}
}

\end{enumerate}

\insertpracticeinfo{2}
}

\subsection{Soaps and Detergents}

Another important part of the chloralkali industry is the production of \textbf{soaps} and \textbf{detergents}. You will remember from an earlier chapter, that water has the property of \textit{surface tension}. This means that it tends to bead up on surfaces and this slows down the wetting process and makes cleaning difficult. You can observe this property of surface tension when a drop of water falls onto a table surface. The drop holds its shape and does not spread. When cleaning, this surface tension must be reduced so that the water can spread. Chemicals that are able to do this are called \textbf{surfactants}. Surfactants also loosen, disperse and hold particles in suspension, all of which are an important part of the cleaning process. Soap is an example of one of these surfactants. Detergents contain one or more surfactants. We will go on to look at these in more detail.

\Definition{Surfactant}{
A surfactant is a wetting agent that lowers the surface tension of a liquid, allowing it to spread more easily.
}

\begin{enumerate}

\item{\textbf{Soaps}}

In chapter \ref{chap:orgmac}, a number of important biological macromolecules were discussed, including carbohydrates, proteins and nucleic acids. \textbf{Fats} are also biological macromolecules. A fat is made up of an alcohol called glycerol, attached to three fatty acids (figure \ref{fig:fat}). Each \textbf{fatty acid} is made up of a carboxylic acid attached to a long hydrocarbon chain. An \textbf{oil} has the same structure as a fat, but is a liquid rather than a solid. Oils are found in plants (e.g. olive oil, sunflower oil) and fats are found in animals.\\

\begin{figure}[h]
\begin{center}
\begin{pspicture}(-5,-3.5)(2.5,3.8)
%\psgrid[gridcolor=lightgray]
\rput(-3,2){H}
\psline(-3,1.7)(-3,1.3)

\rput(-3,1){C}
\psline(-3.3,1)(-3.7,1)
\rput(-4,1){H}
\psline(-2.7,1)(-2.3,1)
\rput(-2,1){O}
\psline(-1.7,1)(-1.3,1)
\rput(-1,1){C}
\psline(-1.05,1.3)(-1.05,1.7)
\psline(-0.95,1.3)(-0.95,1.7)
\rput(-1,2){O}
\psline(-0.7,1)(-0.3,1)
\rput(0.8,1){(CH$_{2}$)$_{14}$CH$_{3}$}

\rput(0,-1.5){
\rput(-3,1){C}
\psline(-3.3,1)(-3.7,1)
\rput(-4,1){H}
\psline(-2.7,1)(-2.3,1)
\rput(-2,1){O}
\psline(-1.7,1)(-1.3,1)
\rput(-1,1){C}
\psline(-1.05,1.3)(-1.05,1.7)
\psline(-0.95,1.3)(-0.95,1.7)
\rput(-1,2){O}
\psline(-0.7,1)(-0.3,1)
\rput(0.8,1){(CH$_{2}$)$_{14}$CH$_{3}$}
}

\rput(0,-3){
\rput(-3,1){C}
\psline(-3.3,1)(-3.7,1)
\rput(-4,1){H}
\psline(-2.7,1)(-2.3,1)
\rput(-2,1){O}
\psline(-1.7,1)(-1.3,1)
\rput(-1,1){C}
\psline(-1.05,1.3)(-1.05,1.7)
\psline(-0.95,1.3)(-0.95,1.7)
\rput(-1,2){O}
\psline(-0.7,1)(-0.3,1)
\rput(0.8,1){(CH$_{2}$)$_{14}$CH$_{3}$}
}

\psline(-3,0.7)(-3,-0.2)
\psline(-3,-0.8)(-3,-1.7)
\psline(-3,-2.3)(-3,-2.7)
\rput(-3,-3){H}

\psline{<->}(-4,2.5)(-2,2.5)
\rput(-3,3){glycerol}

\psline{<->}(-1.8,2.5)(1.5,2.5)
\rput(-0.1,3){fatty acids}

\end{pspicture}
\caption{The structure of a fat, composed of an alcohol and three fatty acids}
\label{fig:fat}
\end{center}
\end{figure}

To make soap, sodium hydroxide (NaOH) or potassium hydroxide (KOH) must be added to a fat or an oil. During this reaction, the glycerol is separated from the fatty acid in the fat, and is replaced by either potassium or sodium ions (figure \ref{fig:soap}). Soaps are the water-soluble sodium or potassium salts of fatty acids. 

\begin{figure}[h]
\begin{center}
\begin{pspicture}(-4,-3.5)(13,3.8)
%\psgrid[gridcolor=lightgray]
\rput(-3,2){H}
\psline(-3,1.7)(-3,1.3)

\rput(-3,1){C}
\psline(-3.3,1)(-3.7,1)
\rput(-4,1){H}
\psline(-2.7,1)(-2.3,1)
\rput(-2,1){O}
\psline(-1.7,1)(-1.3,1)
\rput(-1,1){C}
\psline(-1.05,1.3)(-1.05,1.7)
\psline(-0.95,1.3)(-0.95,1.7)
\rput(-1,2){O}
\psline(-0.7,1)(-0.3,1)
\rput(0.8,1){(CH$_{2}$)$_{14}$CH$_{3}$}

\rput(0,-1.5){
\rput(-3,1){C}
\psline(-3.3,1)(-3.7,1)
\rput(-4,1){H}
\psline(-2.7,1)(-2.3,1)
\rput(-2,1){O}
\psline(-1.7,1)(-1.3,1)
\rput(-1,1){C}
\psline(-1.05,1.3)(-1.05,1.7)
\psline(-0.95,1.3)(-0.95,1.7)
\rput(-1,2){O}
\psline(-0.7,1)(-0.3,1)
\rput(0.8,1){(CH$_{2}$)$_{14}$CH$_{3}$}
}

\rput(0,-3){
\rput(-3,1){C}
\psline(-3.3,1)(-3.7,1)
\rput(-4,1){H}
\psline(-2.7,1)(-2.3,1)
\rput(-2,1){O}
\psline(-1.7,1)(-1.3,1)
\rput(-1,1){C}
\psline(-1.05,1.3)(-1.05,1.7)
\psline(-0.95,1.3)(-0.95,1.7)
\rput(-1,2){O}
\psline(-0.7,1)(-0.3,1)
\rput(0.8,1){(CH$_{2}$)$_{14}$CH$_{3}$}
}

\psline(-3,0.7)(-3,-0.2)
\psline(-3,-0.8)(-3,-1.7)
\psline(-3,-2.3)(-3,-2.7)
\rput(-3,-3){H}

\rput(2.6,0){\textbf{+ 3NaOH}}
\psline{->}(3.7,0)(4.5,0)

\rput(9,0){
\rput(-3,2){H}
\psline(-3,1.7)(-3,1.3)

\rput(-3,1){C}
\psline(-3.3,1)(-3.7,1)
\rput(-4,1){H}
\psline(-2.7,1)(-2.3,1)
\rput(-2,1){OH}
\rput(2,0){

\rput(-2,1){$^{-}$O}
\psline(-1.7,1)(-1.3,1)
\rput(-2.6,1){Na$^{+}$}

\rput(-1,1){C}
\psline(-1.05,1.3)(-1.05,1.7)
\psline(-0.95,1.3)(-0.95,1.7)
\rput(-1,2){O}
\psline(-0.7,1)(-0.3,1)
\rput(0.8,1){(CH$_{2}$)$_{14}$CH$_{3}$}
}
\rput(0,-1.5){
\rput(-3,1){C}
\psline(-3.3,1)(-3.7,1)
\rput(-4,1){H}
\psline(-2.7,1)(-2.3,1)
\rput(-2,1){OH}
\rput(2,0){

\rput(-2,1){$^{-}$O}
\psline(-1.7,1)(-1.3,1)
\rput(-2.6,1){Na$^{+}$}
\rput(-3.4,1){\textbf{+}}

\rput(-1,1){C}
\psline(-1.05,1.3)(-1.05,1.7)
\psline(-0.95,1.3)(-0.95,1.7)
\rput(-1,2){O}
\psline(-0.7,1)(-0.3,1)
\rput(0.8,1){(CH$_{2}$)$_{14}$CH$_{3}$}
}
}

\rput(0,-3){
\rput(-3,1){C}
\psline(-3.3,1)(-3.7,1)
\rput(-4,1){H}
\psline(-2.7,1)(-2.3,1)
\rput(-2,1){OH}

\rput(-3,-0.5){glycerol}
\rput(1,-0.5){sodium salts of fatty acids}
\rput(2,0){

\rput(-2,1){$^{-}$O}
\psline(-1.7,1)(-1.3,1)
\rput(-2.6,1){Na$^{+}$}


\rput(-1,1){C}
\psline(-1.05,1.3)(-1.05,1.7)
\psline(-0.95,1.3)(-0.95,1.7)
\rput(-1,2){O}
\psline(-0.7,1)(-0.3,1)
\rput(0.8,1){(CH$_{2}$)$_{14}$CH$_{3}$}
}
}

\psline(-3,0.7)(-3,-0.2)
\psline(-3,-0.8)(-3,-1.7)
\psline(-3,-2.3)(-3,-2.7)
\rput(-3,-3){H}
}

\end{pspicture}
\caption{Sodium hydroxide reacts with a fat to produce glycerol and sodium salts of the fatty acids}
\label{fig:soap}
\end{center}
\end{figure}

\begin{IFact}{Soaps can be made from either fats or oils. Beef fat is a common source of fat, and vegetable oils such as palm oil are also commonly used.}
\end{IFact}

Fatty acids consist of two parts: a carboxylic acid group and a hydrocarbon chain. The hydrocarbon chain is \textit{hydrophobic}, meaning that it is repelled by water. However, it is attracted to grease, oils and other dirt. The carboxylic acid is \textit{hydrophilic}, meaning that it is attracted to water. Let's imagine that we have added soap to water in order to clean a dirty rugby jersey. The hydrocarbon chain will attach itself to the soil particles in the jersey, while the carboxylic acid will be attracted to the water. In this way, the soil is pulled free of the jersey and is suspended in the water. In a washing machine or with vigourous handwashing, this suspension can be rinsed off with clean water. 

\Definition{Soap}{
Soap is a surfactant that is used with water for washing and cleaning. Soap is made by reacting a fat with either sodium hydroxide (NaOH) or potassium hydroxide (KOH).}


\item{\textbf{Detergents}}

\Definition{Detergent}{
Detergents are compounds or mixtures of compounds that are used to assist cleaning. The term is often used to distinguish between soap and other chemical surfactants for cleaning.
}

Detergents are also cleaning products, but are composed of one or more surfactants. Depending on the type of cleaning that is needed, detergents may contain one or more of the following:

\begin{itemize}
\item{\textit{Abrasives} to scour a surface.}
\item{\textit{Oxidants} for bleaching and disinfection.}
\item{\textit{Enzymes} to digest proteins, fats or carbohydrates in stains. These are called \textit{biological detergents}.}
\end{itemize}

\end{enumerate}

\Exercise{The choralkali industry\\}{

\begin{pspicture}(-6,1)(4,-4)
%\psgrid[gridcolor=lightgray]
\rput(-4,0){\parbox{3cm}{1. Raw material into main reaction vessel}}
\rput(0,0){\parbox{2.5cm}{2.Chlorine is produced}}
\rput(4,0){\parbox{2.5cm}{3. Na-Hg amalgam breaks into Na and Hg in second reaction vessel}}
\rput(4,-3){\parbox{3cm}{4. NaOH is produced}}
\rput(0,-3){\parbox{3cm}{5. H$_{2}$ gas released }}
\psline{->}(-2.3,0)(-1.7,0)
\psline{->}(1.5,0)(2.1,0)
\psline{->}(4,-1)(4,-2)
\psline{->}(2.1,-3)(1.3,-3)
\end{pspicture}

\begin{enumerate}
\item{The diagram above shows the sequence of steps that take place in the mercury cell.}

	\begin{enumerate}
	\item{Name the 'raw material' in step 1.}
	\item{Give the chemical equation for the reaction that produces chlorine in step 2.}
	\item{What other product is formed in step 2.}
	\item{Name the reactants in step 4.}
	\end{enumerate}

\item{Approximately 30 million tonnes of chlorine are used throughout the world annually. Chlorine is produced industrially by the electrolysis of brine. The diagram represents a membrane cell used in the production of Cl$_{2}$ gas.}


\begin{center}
\psset{unit=0.6}
\begin{pspicture}(-2,-0.2)(5,6.6)
\def\whitearrow{\psframe[fillstyle=solid,fillcolor=white,linestyle=none](0,0.25)(1,0.75)\psline{->}(0,0.5)(1,0.5)}
%\psgrid
\rput(0,0){\psframe(-1,0)(4,5)}
\psframe[fillstyle=solid,fillcolor=lightgray](0.25,1)(0.75,4.5)
\rput(2,0){\psframe(0.25,1)(0.75,4.5)}
\psline(0.5,4.5)(0.5,6)
\psline(2.5,4.5)(2.5,6)

\rput(2.5,6.25){\textbf{--}}\pscircle(2.5,6.25){0.25}
\rput(0.5,6.25){\textbf{+}}\pscircle(0.5,6.25){0.25}

\uput[l](-1,3){anode}\psline(-1.1,3)(0.5,3)
\uput[r](4,3){cathode}\psline(2.5,3)(4.1,3)

\rput(-0.5,0){\uput[l](-1,4){NaCl}\rput(-1,3.5){\whitearrow}}
\rput{90}(0,4.5){\whitearrow}\uput[u](-0.5,5.5){Cl$_2$}
\rput{90}(4,4.5){\whitearrow}\uput[u](3.5,5.5){H$_2$}
\rput(-0.5,0){\uput[l](-1,1){NaCl}\rput{180}(0,1.5){\whitearrow}}

\rput(4.5,0){\uput[r](0,4){NaOH}\rput(-1,3.5){\whitearrow}}
\rput(4.5,0){\uput[r](0,1){H$_{2}$O}\rput{180}(0,1.5){\whitearrow}}

\psline(1.5,0)(1.5,5)
\psline(1.5,0.6)(4.2,0.6)\uput[r](4.2,0.6){membrane}
\end{pspicture}
\end{center}

	\begin{enumerate}
	\item{What ions are present in the electrolyte in the left hand compartment of the cell?}
	\item{Give the equation for the reaction that takes place at the anode.}
	\item{Give the equation for the reaction that takes place at the cathode.}
	\item{What ion passes through the membrane while these reactions are taking place?}

Chlorine is used to purify drinking water and swimming pool water. The substance responsible for this process is the weak acid, hypochlorous acid (HOCl).

	\item{One way of putting HOCl into a pool is to bubble chlorine gas through the water. Give an equation showing how bubbling Cl$_{2}$(g) through water produces HOCl.}
	\item{A common way of treating pool water is by adding 'granular chlorine'. Granular chlorine consists of the salt calcium hypochlorite, Ca(OCl)$_{2}$. Give an equation showing how this salt dissolves in water. Indicate the phase of each substance in the equation.}
	\item{The OCl$^{-}$ ion undergoes hydrolysis , as shown by the following equation:
\begin{center}
$\rm{OCl^{-} + H_{2}O \rightleftharpoons HOCl + OH^{-}}$
\end{center}
Will the addition of granular chlorine to pure water make the water acidic, basic or will it remain neutral? Briefly explain your answer.}

	\end{enumerate}
(IEB Paper 2, 2003)
\end{enumerate}

\insertpracticeinfo{2}
}

\section{The Fertiliser Industry}
\label{sec:chemical:fertilisers}

\subsection{The value of nutrients}

Nutrients are very important for life to exist. An \textbf{essential nutrient} is any chemical element that is needed for a plant to be able to grow from a seed and complete its life cycle. The same is true for animals. A \textit{macronutrient} is one that is required in large quantities by the plant or animal, while a \textit{micronutrient} is one that only needs to be present in small amounts for a plant or an animal to function properly.

\Definition{Nutrient}{
A nutrient is a substance that is used in an organism's metabolism or physiology and which must be taken in from the environment.
}

In plants, the macronutrients include carbon (C), hydrogen (H), oxygen (O), nitrogen (N), phosphorus (P) and potassium (K). The source of each of these nutrients for plants, and their function, is summarised in table \ref{tab:plant macronutrients}. Examples of micronutrients in plants include iron, chlorine, copper and zinc.


\begin{table}[h]
\begin{center}
\caption{The source and function of the macronutrients in plants}
\label{tab:plant macronutrients}
\begin{tabular}{|p{2.8cm}|p{3.2cm}|p{4.2cm}|}\hline
\textbf{Nutrient} & \textbf{Source} & \textbf{Function}\\\hline
Carbon & Carbon dioxide in the air & Component of organic molecules such as carbohydrates, lipids and proteins \\\hline
Hydrogen & Water from the soil & Component of organic molecules \\\hline
Oxygen & Water from the soil & Component of organic molecules \\\hline
Nitrogen & Nitrogen compounds in the soil & Part of plant proteins and chlorophyll. Also boosts plant growth. \\\hline
Phosphorus & Phosphates in the soil & Needed for photosynthesis, blooming and root growth \\\hline
Potassium & Soil & Building proteins, part of chlorophyll and reduces diseases in plants \\\hline
\end{tabular}
\end{center}
\end{table}
 
Animals need similar nutrients in order to survive. However since animals can't photosynthesise, they rely on plants to supply them with the nutrients they need. Think for example of the human diet. We can't make our own food and so we either need to eat vegetables, fruits and seeds (all of which are direct plant products) or the meat of other animals which would have fed on plants during their life. So most of the nutrients that animals need are obtained either directly or indirectly from plants. Table \ref{tab:animal macronutrients} summarises the functions of some of the macronutrients in animals.\\

\begin{table}[h]
\begin{center}
\caption{The functions of animal macronutrients}
\label{tab:animal macronutrients}
\begin{tabular}{|p{3.5cm}|p{5cm}|}\hline
\textbf{Nutrient} & \textbf{Function}\\\hline
Carbon & Component of organic compounds \\\hline
Hydrogen & Component of organic compounds \\\hline
Oxygen & Component of organic compounds \\\hline
Nitrogen & Component of nucleic acids and proteins\\\hline
Phosphorus & Component of nucleic acids and phospholipids \\\hline
Potassium & Helps in coordination and regulating the water balance in the body \\\hline
\end{tabular}
\end{center}
\end{table}

Micronutrients also play an important function in animals. Iron for example, is found in haemoglobin, the blood pigment that is responsible for transporting oxygen to all the cells in the body.\\

Nutrients then, are essential for the survival of life. Importantly, obtaining nutrients starts with plants, which are able either to photosynthesise or to absorb the required nutrients from the soil. It is important therefore that plants are always able to access the nutrients that they need so that they will grow and provide food for other forms of life.

\subsection{The Role of fertilisers}

Plants are only able to absorb soil nutrients in a particular form. Nitrogen for example, is absorbed as \textbf{nitrates}, while phosphorus is absorbed as \textbf{phosphates}. The \textbf{nitrogen cycle} (Grade 10) describes the process that is involved in converting atmospheric nitrogen into a form that can be used by plants. \\ 

However, all these natural processes of maintaining soil nutrients take a long time. As populations grow and the demand for food increases, there is more and more strain on the land to produce food. Often, cultivation practices don't give the soil enough time to recover and to replace the nutrients that have been lost. Today, \textbf{fertilisers} play a very important role in restoring soil nutrients so that crop yields stay high. Some of these fertilisers are \textbf{organic} (e.g. compost, manure and fishmeal), which means that they started off as part of something living. Compost for example is often made up of things like vegetable peels and other organic remains that have been thrown away. Others are \textbf{inorganic} and can be made industrially. The advantage of these commercial fertilisers is that the nutrients are in a form that can be absorbed immediately by the plant.

\Definition{Fertiliser}{
A fertiliser is a compound that is given to a plant to promote growth. Fertilisers usually provide the three major plant nutrients and most are applied via the soil so that the nutrients are absorbed by plants through their roots.
}

When you buy fertilisers from the shop, you will see three numbers on the back of the packet e.g. 18-24-6. These numbers are called the \textbf{NPK ratio}, and they give the percentage of nitrogen, phosphorus and potassium in that fertiliser. Depending on the types of plants you are growing, and the way in which you would like them to grow, you may need to use a fertiliser with a slightly different ratio. If you want to encourage root growth in your plant for example, you might choose a fertiliser with a greater amount of phosphorus. Look at the table below, which gives an idea of the amounts of nitrogen, phosphorus and potassium there are in different types of fertilisers. Fertilisers also provide other nutrients such as calcium, sulfur and magnesium. 

\begin{table}[h]
\begin{center}
\caption{Common grades of some fertiliser materials}

\begin{tabular}{|l|c|}\hline
\textbf{Description} & \textbf{Grade (NPK \%)} \\\hline\hline
Ammonium nitrate & 34-0-0\\\hline
Urea & 46-0-0\\\hline
Bone Meal & 4-21-1\\\hline
Seaweed & 1-1-5\\\hline
Starter fertilisers & 18-24-6\\\hline
Equal NPK fertilisers & 12-12-12\\\hline
High N, low P and medium K fertilisers & 25-5-15\\\hline
\end{tabular}
\end{center}
\end{table}

\subsection{The Industrial Production of Fertilisers}
\label{sec:fertilisers:industrial}

The industrial production of fertilisers may involve several processes.\\
 
\begin{enumerate}
\item{\textbf{Nitrogen fertilisers}}

Making \textbf{nitrogen fertilisers} involves producing \textit{ammonia}, which is then reacted with \textit{oxygen} to produce \textit{nitric acid}. Nitric acid is used to acidify phosphate rock to produce nitrogen fertilisers. The flow diagram below illustrates the processes that are involved. Each of these steps will be examined in more detail.


\begin{figure}[h]
\begin{center}
\begin{pspicture}(-4,-2.6)(4,3.2)
%\psgrid[gridcolor=lightgray]
\rput(0,3){\textbf{(a) HABER PROCESS}}
\rput(0,2.6){The production of ammonia}
\rput(0,2.2){from nitrogen and hydrogen}
\psline[linewidth=1pt,arrows=->](0,2)(0,1.2)
\rput(0,1){\textbf{(b) OSTWALD PROCESS}}
\rput(0,0.6){Production of nitric acid}
\rput(0,0.2){from ammonia and oxygen}
\psline[linewidth=1pt,arrows=->](0,0)(0,-0.8)
\rput(0,-1){\textbf{(c) NITROPHOSPHATE PROCESS}}
\rput(0,-1.4){Acidification of phosphate rock}
\rput(0,-1.8){with nitric acid to produce phosphoric}
\rput(0,-2.2){acid and calcium nitrate}
\end{pspicture}
\caption{Flow diagram showing steps in the production of nitrogen fertilisers}
\end{center}
\end{figure} 


	\begin{enumerate}

	\item{\textbf{The Haber Process}}

The Haber process involves the reaction of nitrogen and hydrogen to produce ammonia. \textbf{Nitrogen} is produced through the \textbf{fractional distillation} of air. Fractional distillation is the separation of a mixture (remember that air is a mixture of different gases) into its component parts through various methods. \textbf{Hydrogen} can be produced through \textbf{steam reforming}. In this process, a hydrocarbon such as methane reacts with water to form carbon monoxide and hydrogen according to the following equation:

\begin{center}
$\rm{CH_{4} + H_{2}O \rightarrow CO + 3H_{2}}$
\end{center} 

Nitrogen and hydrogen are then used in the Haber process. The equation for the Haber process is:

\begin{center}
$\rm{N_{2}(g) + 3H_{2}(g) \rightarrow 2NH_{3}(g)}$ 
\end{center}
(The reaction takes place in the presence of an iron (Fe) catalyst under conditions of 200 atmospheres (atm) and 450-500 degrees Celsius)

\begin{IFact}{The Haber process developed in the early 20th century, before the start of World War 1. Before this, other sources of nitrogen for fertilisers had included saltpeter (NaNO$_{3}$) from Chile and guano. Guano is the droppings of seabirds, bats and seals. By the 20th century, a number of methods had been developed to 'fix' atmospheric nitrogen. One of these was the Haber process, and it advanced through the work of two German men, Fritz Haber and Karl Bosch (The process is sometimes also referred to as the 'Haber-Bosch process').  They worked out what the best conditions were in order to get a high yield of ammonia, and found these to be high temperature and high pressure. They also experimented with different catalysts to see which worked best in that reaction. During World War 1, the ammonia that was produced through the Haber process was used to make explosives. One of the advantages for Germany was that, having perfected the Haber process, they did not need to rely on other countries for the chemicals that they needed to make them. }
\end{IFact}


	\item{\textbf{The Ostwald Process}}

The Ostwald process is used to produce nitric acid from ammonia. Nitric acid can then be used in reactions that produce fertilisers. Ammonia is converted to nitric acid in two stages. First, it is oxidised by heating with oxygen in the presence of a platinum catalyst to form nitric oxide and water. This step is strongly exothermic, making it a useful heat source.

\begin{center}
$\rm{4NH_{3}(g) + 5O_{2}(g) \rightarrow 4NO(g) + 6H_{2}O(g)}$\\
\end{center}
Stage two, which combines two reaction steps, is carried out in the presence of water. Initially nitric oxide is oxidised again to yield nitrogen dioxide:
\begin{center}
$\rm{2NO(g) + O_{2}(g) \rightarrow 2NO_{2}(g)}$\\
\end{center}
This gas is then absorbed by the water to produce nitric acid. Nitric oxide is also a product of this reaction. The nitric oxide  (NO) is recycled, and the acid is concentrated to the required strength. 
\begin{center}
$\rm{3NO_{2}(g) + H_{2}O(l) \rightarrow 2HNO_{3}(aq) + NO(g)}$\\
\end{center}


	\item{\textbf{The Nitrophosphate Process}}

The nitrophosphate process involves acidifying phosphate rock with nitric acid to produce a mixture of phosphoric acid and calcium nitrate:
\begin{center}
$\rm{Ca_{3}(PO_{4})_{2} + 6HNO_{3} + 12H_{2}O \rightarrow 2H_{3}PO_{4} + 3Ca(NO_{3})_{2} + 12H_{2}O}$\\
\end{center}

When calcium nitrate and phosphoric acid react with ammonia, a compound fertiliser is produced.
\begin{center}
$\rm{Ca(NO_{3})_{2} + 4H_{3}PO_{4} + 8NH_{3} \rightarrow CaHPO_{4} + 2NH_{4}NO_{3} + 8(NH_{4})_2HPO_{4}}$ 
\end{center}

If potassium chloride or potassium sulphate is added, the result will be NPK fertiliser. 

	\item{\textbf{Other nitrogen fertilisers}}
\begin{itemize}
\item{
Urea ((NH$_{2}$)$_{2}$CO) is a nitrogen-containing chemical product which is produced on a large scale worldwide. Urea has the highest nitrogen content of all solid nitrogeneous fertilisers in common use (46.4\%) and is produced by reacting ammonia with carbon dioxide.\\

Two reactions are involved in producing urea:\\

\begin{enumerate}

\item{
$\rm{2NH_{3} + CO_{2} \rightarrow H_{2}N-COONH_{4}}$
}

\item{
$\rm{H_{2}N-COONH_{4} \rightarrow (NH_{2})_{2}CO + H_{2}O}$\\
}

\end{enumerate}
}

\item{Other common fertilisers are ammonium nitrate and ammonium sulphate. Ammonium nitrate is formed by reacting ammonia with nitric acid. 

\begin{center}
$\rm{NH_{3} + HNO_{3} \rightarrow NH_{4}NO_{3}}$
\end{center}

Ammonium sulphate is formed by reacting ammonia with sulphuric acid.

\begin{center}
$\rm{2NH_{3} + H_{2}SO_{4} \rightarrow (NH_{4})_{2}SO_{4}}$
\end{center}

}
\end{itemize}
 
	\end{enumerate}

\item{\textbf{Phosphate fertilisers}}

The production of phosphate fertilisers also involves a number of processes. The first is the production of sulfuric acid through the \textbf{contact process}. Sulfuric acid is then used in a reaction that produces phosphoric acid. Phosphoric acid can then be reacted with phosphate rock to produce triple superphosphates. 

	\begin{enumerate}
\item{\textit{The production of sulfuric acid}}

Sulfuric acid is produced from sulfur, oxygen and water through the contact process. In the first step, sulfur is burned to produce sulfur dioxide.

\begin{center}
$\rm{S(s) + O_{2}(g) \rightarrow SO_{2}(g)}$
\end{center}
 
This is then oxidised to sulfur trioxide using oxygen in the presence of a vanadium(V) oxide catalyst.

\begin{center}
$\rm{2SO_{2} + O_{2}(g) \rightarrow 2SO_{3}(g)}$
\end{center} 

Finally the sulfur trioxide is treated with water to produce 98-99\% sulfuric acid.

\begin{center}
$\rm{SO_{3}(g) + H_{2}O(l) \rightarrow H_{2}SO_{4}(l)}$
\end{center}

\item{\textit{The production of phosphoric acid}}

The next step in the production of phosphate fertiliser is the reaction of sulfuric acid with phosphate rock to produce phosphoric acid (H$_{3}$PO$_{4}$). In this example, the phosphate rock is fluoropatite (Ca$_{5}$F(PO$_{4}$)$_{3}$). 

\begin{center}
$\rm{Ca_{5}F(PO_{4})_{3} + 5H_{2}SO_{4} + 8H_{2}O \rightarrow 5CaSO_{4} + HF + 3H_{3}PO_{4}}$
\end{center} 

\item{\textit{The production of phosphates and superphosphates}}

When concentrated phosphoric acid reacts with ground phosphate rock, triple superphosphate is produced.

\begin{center}
$\rm{3Ca_{3}(PO_{4})_{2}{\cdot}CaF_{2} + 12H_{3}PO_{4} \rightarrow 9Ca(H_{2}PO_{4})_{2} + 3CaF_{2}}$ 
\end{center}
	\end{enumerate}

\item{\textbf{Potassium}}

Potassium is obtained from \textbf{potash}, an impure form of potassium carbonate (K$_{2}$CO$_{3}$). Other potassium salts (e.g. KCl AND K$_{2}$O) are also sometimes included in fertilisers.

\end{enumerate}

\subsection{Fertilisers and the Environment: Eutrophication}

Eutrophication is the enrichment of an ecosystem with chemical nutrients, normally by compounds that contain nitrogen or phosphorus. Eutrophication is considered a form of pollution because it promotes plant growth, favoring certain species over others. In aquatic environments, the rapid growth of  certain types of plants can disrupt the normal functioning of an ecosystem, causing a variety of problems. Human society is impacted as well because eutrophication can decrease the resource value of rivers, lakes, and estuaries making recreational activities less enjoyable. Health-related problems can also occur if eutrophic conditions interfere with the treatment of drinking water.

\Definition{Eutrophication}{
Eutrophication refers to an increase in chemical nutrients in an ecosystem. These chemical nutrients usually contain nitrogen or phosphorus.
}

In some cases, eutrophication can be a natural process that occurs very slowly over time. However, it can also be accelerated by certain human activities. Agricultural runoff, when excess fertilisers are washed off fields and into water, and sewage are two of the major causes of eutrophication. There are a number of impacts of eutrophication.

\begin{itemize}
\item{\textit{A decrease in biodiversity (the number of plant and animal species in an ecosystem)}

When a system is enriched with nitrogen, plant growth is rapid. When the number of plants increases in an aquatic system, they can block light from reaching deeper. Plants also consume oxygen for respiration, and if the oxygen content of the water decreases too much, this can cause other organisms such as fish to die.}

\item{\textit{Toxicity}

Sometimes, the plants that flourish during eutrophication can be toxic and may accumulate in the food chain.}  
\end{itemize}  

\begin{IFact}{
South Africa's Department of Water Affairs and Forestry has a 'National Eutrophication Monitoring Programme' which was set up to monitor eutrophication in impoundments such as dams, where no monitoring was taking place.
}
\end{IFact}

Despite the impacts, there are a number of ways of preventing eutrophication from taking place. \textbf{Cleanup measures} can directly remove the excess nutrients such as nitrogen and phosphorus from the water. Creating \textbf{buffer zones} near farms, roads and rivers can also help. These act as filters and cause nutrients and sediments to be deposited there instead of in the aquatic system. \textbf{Laws} relating to the treatment and discharge of sewage can also help to control eutrophication. A final possible intervention is \textbf{nitrogen testing and modeling}. By assessing exactly how much fertiliser is needed by crops and other plants, farmers can make sure that they only apply just enough fertiliser. This means that there is no excess to run off into neighbouring streams during rain. There is also a cost benefit for the farmer.

\Activity{Discussion}{Dealing with the consequences of eutrophication\\}{

In many cases, the damage from eutrophication is already done. In groups, do the following:\\

\begin{enumerate}
\item{List all the possible consequences of eutrophication that you can think of.}
\item{Suggest ways to solve these problems, that arise because of eutrophication.}
\end{enumerate}
}

\Exercise{Chemical industry: Fertilisers\\}{

\begin{quote}{
\textbf{Why we need fertilisers}

There is likely to be a gap between food production and demand in several parts of the world by 2020.  Demand is influenced by population growth and urbanisation, as well as income levels and changes in dietary preferences. \\ 

The facts are as follows:
\begin{itemize}
\item{There is an increasing world population to feed}
\item{Most soils in the world used for large-scale, intensive production of crops lack the necessary nutrients for the crops}
\end{itemize}

Conclusion: Fertilisers are needed!\\}
\end{quote}

The flow diagram below shows the main steps in the industrial preparation of two important solid fertilisers.

\begin{center}
\begin{pspicture}(-4,-5)(4,5)
%\psgrid[gridcolor=lightgray]
\psframe(-4,3.5)(-2,4.5)
\rput(-3,4){Air}
\psline[arrows=->](-3,3.5)(-3,3)
\psframe(-4,2)(-2,3)
\rput(-3,2.5){Nitrogen}
\psline[arrows=->](-3,2)(-3,1.5)
\rput(6,0){
\psframe(-4,3.5)(-2,4.5)
\rput(-3,4){Methane}
\psline[arrows=->](-3,3.5)(-3,3)
\psframe(-4,2)(-2,3)
\rput(-3,2.5){Hydrogen}
\psline[arrows=->](-3,2)(-3,1.5)
}
\psline(-3,1.5)(3,1.5)
\rput(0,1.8){Haber process}
\psline[arrows=->](-1,1.5)(-1,1)
\psframe(-2,0)(0,1)
\rput(-1,0.5){NH$_{3}$}
\psline[arrows=->](-2,0.5)(-3,0.5)
\rput(-2.5,0.8){\small{H$_{2}$SO$_{4}$}}
\psframe(-5,0)(-3,1)
\rput(-4,0.5){Fertiliser C}
\psline[arrows=->](0,0.5)(1,0.5)
\rput(0.5,1){Process}
\rput(0.5,0.7){Y}
\psframe(1,0)(2,1)
\rput(1.5,0.5){NO}
\psline[arrows=->](2,0.5)(3,0.5)
\psframe(3,0)(4.5,1)
\rput(3.8,0.7){Brown}
\rput(3.8,0.4){gas}
\psline[arrows=->](3.8,0)(3.8,-1)
\psframe(3,-2)(4.5,-1)
\rput(3.8,-1.5){Liquid E}
\psline[arrows=->](-1,0)(-1,-3)
\psframe(-2,-4)(0,-3)
\rput(-1,-3.5){Fertiliser D}
\psline[arrows=->](3,-1.5)(-1,-1.5)


\end{pspicture}
\end{center}


\begin{enumerate}
\item{Write down the balanced chemical equation for the formation of the brown gas.}
\item{Write down the name of process Y.}
\item{Write down the chemical formula of liquid E.}
\item{Write down the chemical formulae of fertilisers C and D respectively.}

The following extract comes from an article on fertilisers:
\begin{center}
\textit{A world without food for its people}

\textit{A world with an environment poisoned through the actions of man }

\textit{Are two contributing factors towards a disaster scenario.}

\end{center}

\item{Write down THREE ways in which the use of fertilisers poisons the environment.}
\end{enumerate}

\insertpracticeinfo{1}

}

\section{Electrochemistry and batteries}
\label{sec:chemindustry:batteries}

You will remember from chapter \ref{chap:electrochemical} that a \textbf{galvanic} cell (also known as a \textit{voltaic} cell) is a type of electrochemical cell where a chemical reaction produces electrical energy. The \textbf{electromotive force} (emf) of a galvanic cell is the difference in voltage between the two half cells that make it up. Galvanic cells have a number of applications, but one of the most important is their use in \textbf{batteries}. You will know from your own experience that we use batteries in a number of ways, including cars, torches, sound systems and cellphones to name just a few.

\subsection{How batteries work}

A battery is a device in which \textbf{chemical energy} is directly converted to \textbf{electrical energy}. It consists of one or more voltaic cells, each of which is made up of two half cells that are connected in series by a conductive electrolyte. The voltaic cells are connected in series in a battery. Each cell has a positive electrode (cathode), and a negative electrode (anode). These do not touch each other but are immersed in a solid or liquid electrolyte.\\

Each half cell has a net electromotive force (emf) or voltage. The voltage of the battery is the difference between the voltages of the half-cells. This potential difference between the two half cells is what causes an electric current to flow. \\

Batteries are usually divided into two broad classes:

\begin{itemize}
\item{\textit{Primary batteries} irreversibly transform chemical energy to electrical energy. Once the supply of reactants has been used up, the battery can't be used any more.}
 
\item{\textit{Secondary batteries} can be recharged, in other words, their chemical reactions can be reversed if electrical energy is supplied to the cell. Through this process, the cell returns to its original state. Secondary batteries can't be recharged forever because there is a gradual loss of the active materials and electrolyte. Internal corrosion can also take place.}
\end{itemize}

\subsection{Battery capacity and energy}

The \textbf{capacity} of a battery, in other words its ability to produce an electric charge, depends on a number of factors. These include:

\begin{itemize}
\item{\textbf{Chemical reactions}

The chemical reactions that take place in each of a battery's half cells will affect the voltage across the cell, and therefore also its capacity. For example, nickel-cadmium (NiCd) cells measure about 1.2 V, and alkaline and carbon-zinc cells both measure about 1.5 V. However, in other cells such as Lithium cells, the changes in electrochemical potential are much higher because of the reactions of lithium compounds, and so lithium cells can produce as much as 3 volts or more. The concentration of the chemicals that are involved will also affect a battery's capacity. The higher the concentration of the chemicals, the greater the capacity of the battery.
}

\item{\textbf{Quantity of electrolyte and electrode material in cell}

The greater the amount of electrolyte in the cell, the greater its capacity. In other words, even if the chemistry in two cells is the same, a larger cell will have a greater capacity than a small one. Also, the greater the surface area of the electrodes, the greater will be the capacity of the cell.
}

\item{\textbf{Discharge conditions}

A unit called an \textbf{Ampere hour} (Ah) is used to describe how long a battery will last. An ampere hour (more commonly known as an \textbf{amp hour}) is the amount of electric charge that is transferred by a current of one ampere for one hour. Battery manufacturers use a standard method to rate their batteries. So, for example, a 100 Ah battery will provide a current of 5 A for a period of 20 hours at room temperature. The capacity of the battery will depend on the rate at which it is discharged or used. If a 100 Ah battery is discharged at 50 A (instead of 5 A), the capacity will be \textit{lower} than expected and the battery will run out \textit{before} the expected 2 hours.\\

The relationship between the current, discharge time and capacity of a battery is expressed by \textbf{Peukert's law}:

\begin{eqnarray*}
C_{p} = I^{k}t
\end{eqnarray*}

In the equation, 'C$_{p}$' represents the battery's capacity (Ah), I is the discharge current (A), k is the Peukert constant and t is the time of discharge (hours).
}
\end{itemize}

\subsection{Lead-acid batteries}

In a \textbf{lead-acid battery}, each cell consists of electrodes of lead (Pb) and lead (IV) oxide (PbO$_{2}$) in an electrolyte of sulfuric acid (H$_{2}$SO$_{4}$). When the battery discharges, both electrodes turn into lead (II) sulphate (PbSO$_{4}$) and the electrolyte loses sulfuric acid to become mostly water.\\

The chemical half reactions that take place at the anode and cathode when the battery is \textbf{discharging} are as follows:\\

Anode (oxidation): $\rm{Pb_{(s)} + SO_{4\ (aq)}^{2-} \rightleftharpoons PbSO_{4\ (s)} + 2e^{-}}$ (E$^{0}$ = -0.356 V)\\



Cathode (reduction): $\rm{PbO_{2\ (s)} + SO_{4\ (aq)}^{2-} + 4H^{+} + 2e^{-} \rightleftharpoons PbSO_{4\ (s)}+ 2H_{2}O_{(l)}}$ (E$^{0}$ = 1.685 V)\\


The overall reaction is as follows:

\begin{center}
$\rm{PbO_{2}(s) + 4H^{+}(aq) + 2SO_{4}^{2-}(aq) + Pb(s) \rightarrow 2PbSO_{4}(s) + 2H_{2}O(l)}$
\end{center}

The emf of the cell is calculated as follows:\\


EMF = E (cathode)- E (anode)

EMF = +1.685 V - (-0.356 V)

EMF = +2.041 V\\

Since most batteries consist of six cells, the total voltage of the battery is approximately 12 V.

One of the important things about a lead-acid battery is that it can be \textbf{recharged}. The recharge reactions are the \textit{reverse} of those when the battery is discharging.\\

The lead-acid battery is made up of a number of \textit{plates} that maximise the surface area on which chemical reactions can take place. Each plate is a rectangular grid, with a series of holes in it. The holes are filled with a mixture of lead and sulfuric acid. This paste is pressed into the holes and the plates are then stacked together, with suitable separators between them. They are then placed in the battery container, after which acid is added (figure \ref{fig:chemindustry:lead-acid}).

\begin{center}
\begin{figure}[h]
\begin{pspicture}(-7,-4)(7,4)
%\psgrid[gridcolor=lightgray]
\psframe(-4,-4)(3.5,3)
\psline[linewidth=1pt](-3,3)(-3,-2.5)
\psline[linewidth=1pt](-3,2)(2,2)
\psline[linewidth=1pt](-2,2)(-2,-2.5)
\psline[linewidth=1pt](-1,2)(-1,-2.5)
\psline[linewidth=1pt](0,2)(0,-2.5)
\psline[linewidth=1pt](1,2)(1,-2.5)
\psline[linewidth=1pt](2,2)(2,-2.5)

\psline[linewidth=0.7pt](2.5,3)(2.5,-3)
\psline[linewidth=0.7pt](2.5,-3)(-2.5,-3)
\psline[linewidth=0.7pt](1.5,-3)(1.5,1.5)
\psline[linewidth=0.7pt](0.5,-3)(0.5,1.5)
\psline[linewidth=0.7pt](-0.5,-3)(-0.5,1.5)
\psline[linewidth=0.7pt](-1.5,-3)(-1.5,1.5)
\psline[linewidth=0.7pt](-2.5,-3)(-2.5,1.5)

\psframe(-3.2,3)(-2.8,3.4)
\rput(-3,3.7){\Large{-}}
\psframe(2.3,3)(2.7,3.4)
\rput(2.5,3.7){\textbf{+}}

\psline(-3,1)(-5,1)
\rput(-6.5,1){Lead anode plates}
\psline(-3,-3)(-5,-3)
\rput(-5.5,-3){H$_{2}$SO$_{4}$}
\psline(2.5,0)(4.5,0)
\rput(6.5,0.2){Lead cathode plates}
\rput(6.5,-0.2){coated with PbO$_{2}$}
\end{pspicture}
\caption{A lead-acid battery}
\label{fig:chemindustry:lead-acid}
\end{figure}
\end{center}

Lead-acid batteries have a number of applications. They can supply high surge currents, are relatively cheap, have a long shelf life and can be recharged. They are ideal for use in cars, where they provide the high current that is needed by the starter motor. They are also used in forklifts and as standby power sources in telecommunication facilities, generating stations and computer data centres. One of the disadvantages of this type of battery is that the battery's lead must be recycled so that the environment doesn't become contaminated. Also, sometimes when the battery is charging, hydrogen gas is generated at the cathode and this can cause a small explosion if the gas comes into contact with a spark.

\subsection{The zinc-carbon dry cell}

A simplified diagram of a zinc-carbon cell is shown in figure \ref{fig:chemindustry:zinc-carbon}.

\begin{center}
\begin{figure}[h]
\begin{pspicture}(-6,-5)(6,5)
%\psgrid[gridcolor=lightgray]
\psframe(-3,-4)(3,4)
\psframe(-2.8,-3.8)(2.8,4)
\psframe(-1.5,-3.4)(1.5,4)
\psframe(-0.8,-3)(0.8,4)
\psframe(-0.8,4)(0.8,4.4)
\psline(0.8,4.2)(4.5,4.2)
\rput(5.5,4.2){metal cap}
\psline(0,3)(4.5,3)
\rput(6.3,3){carbon rod (cathode)}
\psline(3,1)(4.5,1)
\rput(5.4,1){zinc case}
\psline(1.5,-1)(4.5,-1)
\rput(6.4,-1){manganese (IV) oxide}
\psline(2.5,-2)(4.5,-2)
\rput(5.9,-2){paste of NH$_{4}$Cl}
\psline(2.8,-3)(4.5,-3)
\rput(6.4,-2.8){separator between zinc}
\rput(6.4,-3.2){and the electrolyte}
\end{pspicture}
\caption{A zinc-carbon dry cell}
\label{fig:chemindustry:zinc-carbon}
\end{figure}
\end{center}


A zinc-carbon cell is made up of an outer zinc container, which acts as the \textbf{anode}. The \textbf{cathode} is the central carbon rod, surrounded by a mixture of carbon and manganese (IV) oxide (MnO$_{2}$). The \textbf{electrolyte} is a paste of ammonium chloride (NH$_{4}$Cl). A fibrous fabric separates the two electrodes, and a brass pin in the centre of the cell conducts electricity to the outside circuit. \\

The paste of ammonium chloride reacts according to the following half-reaction:

$\rm{2NH_{4}^{+}(aq) + 2e^{-} \rightarrow 2NH_{3}(g) + H_{2}(g)}$\\

The manganese(IV) oxide in the cell removes the hydrogen produced above, according to the following reaction:

$\rm{2MnO_{2}(s) + H_{2}(g) \rightarrow Mn_{2}O_{3}(s) + H_{2}O(l)}$\\

The combined result of these two reactions can be represented by the following half reaction, which takes place at the cathode:\\

Cathode: $\rm{2NH_{4}^{+}(aq) + 2MnO_{2}(s) + 2e^{-} \rightarrow Mn_{2}O_{3}(s) + 2NH_{3}(g) + H_{2}O(l)}$\\

The anode half reaction is as follows:

Anode: $\rm{Zn(s) \rightarrow Zn^{2+} + 2e^{-}}$\\

The overall equation for the cell is:

\begin{center}
$\rm{Zn(s) + 2MnO_{2}(s) + 2NH_{4}^{+} \rightarrow Mn_{2}O_{3}(s) + H_{2}O + Zn(NH_{3})_{2}^{2+}(aq)}$ (E$^{0}$ = 1.5 V)
\end{center}

\textbf{Alkaline batteries} are almost the same as zinc-carbon batteries, except that the electrolyte is potassium hydroxide (KOH), rather than ammonium chloride. The two half reactions in an alkaline battery are as follows:\\

Anode: $\rm{Zn(s) + 2OH^{-}(aq) \rightarrow Zn(OH)_{2}(s) + 2e^{-}}$

Cathode: $\rm{2MnO_{2}(s) + H_{2}O(l) + 2e^{-} \rightarrow Mn_{2}O_{3}(s) + 2OH^{-}(aq)}$\\


Zinc-carbon and alkaline batteries are cheap primary batteries and are therefore very useful in appliances such as remote controls, torches and radios where the power drain is not too high. The disadvantages are that these batteries can't be recycled and can leak. They also have a short shelf life. Alkaline batteries last longer than zinc-carbon batteries.

\begin{IFact}{
The idea behind today's common 'battery' was created by Georges Leclanche in France in the 1860's. The anode was a zinc and mercury alloyed rod, the cathode was a porous cup containing crushed MnO$_{2}$. A carbon rod was inserted into this cup. The electrolyte was a liquid solution of ammonium chloride, and the cell was therefore called a \textit{wet cell}. This was replaced by the \textit{dry cell} in the 1880's. In the dry cell, the zinc can which contains the electrolyte, has become the anode, and the electrolyte is a paste rather than a liquid. 
}
\end{IFact} 

\subsection{Environmental considerations}

While batteries are very convenient to use, they can cause a lot of damage to the environment. They use lots of valuable resources as well as some potentially hazardous chemicals such as lead, mercury and cadmium. Attempts are now being made to recycle the different parts of batteries so that they are not disposed of in the environment, where they could get into water supplies, rivers and other ecosystems.

\Exercise{Electrochemistry and batteries\\}{
A dry cell, as shown in the diagram below, does not contain a liquid electrolyte. The electrolyte in a typical zinc-carbon cell is a moist paste of ammonium chloride and zinc chloride.\\

\nts{Insert diagram}\\

The paste of ammonium chloride reacts according to the following half-reaction:\\

\begin{center}
$\rm{2NH_{4}^{+}(aq) + 2e^{-} \rightarrow 2NH_{3}(g) + H_{2}(g)}$  (a)\\
\end{center}

Manganese(IV) oxide is included in the cell to remove the hydrogen produced during half-reaction (a), according to the following reaction:

\begin{center}
$\rm{2MnO_{2}(s) + H_{2}(g) \rightarrow Mn_{2}O_{3}(s) + H_{2}O(l)}$ (b)
\end{center}

The combined result of these two half-reactions can be represented by the following half reaction:

\begin{center}
$\rm{2NH_{4}^{+}(aq) + 2MnO_{2}(s) + 2e^{-} \rightarrow Mn_{2}O_{3}(s) + 2NH_{3}(g) + H_{2}O(l)}$ (c)
\end{center}

\begin{enumerate}
\item{Explain why it is important that the hydrogen produced in half-reaction (a) is removed by the manganese(IV) oxide.\\}

\textit{In a zinc-carbon cell, such as the one above, half-reaction (c) and the half-reaction that takes place in the  Zn/Zn$^{2+}$ half-cell,  produce  an emf of 1,5 V under standard conditions.}\\
\item{Write down the half-reaction occurring at the anode.}
\item{Write down the net ionic equation occurring in the zinc-carbon cell.}
\item{Calculate the reduction potential for the cathode half-reaction.}
\item{When in use the zinc casing of the dry cell becomes thinner, because it is oxidised.  When not in use, it still corrodes.  Give a reason for the latter observation.}
\item{Dry cells are generally discarded when 'flat'.  Why is the carbon rod the most useful part of the cell, even when the cell is flat?}
\end{enumerate}

(DoE Exemplar Paper 2, 2007)

\insertpracticeinfo{1}
}

\summary{aaa}

\begin{itemize}
\item{The growth of South Africa's \textbf{chemical industry} was largely because of the mines, which needed explosives for their operations. One of South Africa's major chemical companies is \textbf{Sasol}. Other important chemical industries in the country are the \textbf{chloralkali} and \textbf{fertiliser} industries.}
\item{All countries need energy resources such as oil and natural gas. Since South Africa doesn't have either of these resources, Sasol technology has developed to convert \textbf{coal} into liquid fuels.}
\item{Sasol has three main operation focus areas: Firstly, the conversion of \textbf{coal to liquid fuel}, secondly the production and \textbf{refinement of crude oil} which has been imported, and thirdly the production of \textbf{liquid fuels from natural gas}.}
\item{The conversion of coal to liquid fuels involves a \textbf{Sasol/Lurgi gasification} process, followed by the conversion of this synthesis gas into a range of hydrocarbons, using the \textbf{Fischer-Tropsch technology} in \textbf{SAS reactors}.}
\item{Heavy hydrocarbons can be converted into light hydrcarbons through a process called \textbf{cracking}. Common forms of cracking are \textbf{hydrocracking} and \textbf{steam cracking}.}
\item{With regard to crude oil, Sasol imports crude oil from Gabon and then refines this at the \textbf{Natref refinery}.}
\item{Gas from Mozambique can be used to produce liquid fuels, through two processes: First, the gas must pass through an autothermal reactor to produce a synthesis gas. Secondly, this synthesis gas is passed through a \textbf{Sasol Slurry Phase Distillate} process to convert the gas to hydrocarbons.}
\item{All industries have an impact on the \textbf{environment} through the consumption of natural resources such as water, and through the production of pollution gases such as carbon dioxide, hydrogen sulfides, nitrogen oxides and others.}
\item{The \textbf{chloralkali industry} produces \textbf{chlorine} and \textbf{sodium hydroxide}. The main raw material is \textbf{brine} (NaCl).}
\item{In industry, \textbf{electrolytic cells} are used to split the sodium chloride into its component ions to produce chlorine and sodium hydroxide. One of the challenges in this process is to keep the products of the electrolytic reaction (i.e. the chlorine and the sodium hydroxide) separate so that they don't react with each other. Specially designed electrolytic cells are needed to do this.}
\item{There are three types of electrolytic cells that are used in this process: \textbf{mercury cell}, the \textbf{diaphragm cell} and the \textbf{membrane cell}.}
\item{The \textbf{mercury cell} consists of two reaction vessels. The first reaction vessel contains a mercury cathode and a carbon anode. An electric current passed through the brine produces Cl$^{-}$ and Na$^{+}$ ions. The Cl$^{-}$ ions are oxidised to form chlorine gas at the anode. Na$^{+}$ ions combine with the mercury cathode to form a sodium-mercury amalgam.  The sodium-mercury amalgam passes into the second reaction vessel containing water, where the Na$^{+}$ ions react with hydroxide ions from the water. Sodium hydroxide is the product of this reaction.}
\item{One of the \textbf{environmental impacts} of using this type of cell, is the use of \textbf{mercury}, which is highly toxic.}
\item{In the \textbf{diaphragm cell}, a porous diaphragm separates the anode and the cathode compartments. Chloride ions are oxidised to chlorine gas at the anode, while sodium ions produced at the cathode react with water to produce sodium hydroxide.}
\item{The \textbf{membrane cell} is very similar to the diaphragm cell, except that the anode and cathode compartments are separated by an \textbf{ion-selective membrane} rather than by a diaphragm. Brine is only pumped into the anode compartment. Positive sodium ions pass through the membrane into the cathode compartment, which contains water. As with the other two cells, chlorine gas is produced at the anode and sodium hydroxide at the cathode.}
\item{One use of sodium hydroxide is in the production of \textbf{soaps and detergents}, and so this is another important part of the chloralkali industry.}
\item{To make soap, sodium hydroxide or potassium hydroxide react with a fat or an oil. In the reaction, the sodium or potassium ions replace the alcohol in the fat or oil. The product, a \textbf{sodium or potassium salt of a fatty acid}, is what soap is made of.}
\item{The fatty acids in soap have a \textbf{hydrophilic} and a \textbf{hydrophobic} part in each molecule, and this helps to loosen dirt and clean items.}
\item{\textbf{Detergents} are also cleaning products, but are made up of a mixture of compounds. They may also have other components added to them to give certain characteristics. Some of these additives may be abrasives, oxidants or enzymes.}
\item{The \textbf{fertiliser industry} is another important chemical industry.}
\item{All plants need certain \textbf{macronutrients} (e.g. carbon, hydrogen, oxygen, potassium, nitrogen and phosphorus) and \textbf{micronutrients} (e.g. iron, chlorine, copper and zinc) in order to survive. Fertilisers provide these nutrients.}
\item{In plants, most nutrients are obtained from the atmosphere or from the soil.}
\item{Animals also need similar nutrients, but they obtain most of these directly from plants or plant products. They may also obtain them from other animals, whcih may have fed on plants during their life.}
\item{The fertiliser industry is very important in ensuring that plants and crops receive the correct nutrients in the correct quantities to ensure maximum growth.} 
\item{\textbf{Nitrogen fertilisers} can be produced industrially using a number of chemical processes: The \textbf{Haber process} reacts nitrogen and hydrogen to produce \textbf{ammonia}; the \textbf{Ostwald process} reacts oxygen and ammonia to produce \textbf{nitric acid}; the \textbf{nitrophosphate process} reacts nitric acid with phosphate rock to produce compound fertilisers.}
\item{\textbf{Phosphate fertilisers} are also produced through a series of reactions. The \textbf{contact process} produces \textbf{sulfuric acid}. Sulfuric acid then reacts with phosphate rock to produce \textbf{phosphoric acid}, after which phosphoric acid reacts with ground phosphate rock to produce fertilisers such as \textbf{triple superphosphate}.}
\item{Potassium is obtained from \textbf{potash}.}
\item{Fertilisers can have a damaging effect on the environment when they are present in high quantities in ecosystems. They can lead to \textbf{eutrophication}. A number of preventative actions can be taken to reduce these impacts.}
\item{Another important part of the chemical industry is the production of \textbf{batteries}.}
\item{A battery is a device that changes chemical energy into electrical energy.}
\item{A battery consists of one or more \textbf{voltaic cells}, each of which is made up of two half cells that are connected in series by a conductive electrolyte. Each half cell has a net electromotive force (emf) or voltage. The net voltage of the battery is the difference between the voltages of the half-cells. This potential difference between the two half cells is what causes an electric current to flow. }
\item{A \textbf{primary battery} cannot be recharged, but a \textbf{secondary battery} can be recharged.}
\item{The \textbf{capacity} of a battery depends on the \textbf{chemical reactions} in the cells, the \textbf{quantity of electrolyte and electrode material} in the cell, and the \textbf{discharge conditions} of the battery.}
\item{The relationship between the current, discharge time and capacity of a battery is expressed by \textbf{Peukert's law}:

\begin{eqnarray*}
C_{p} = I^{k}t
\end{eqnarray*}

In the equation, 'C$_{p}$' represents the battery's capacity (Ah), I is the discharge current (A), k is the Peukert constant and t is the time of discharge (hours).}
\item{Two common types of batteries are \textbf{lead-acid batteries} and the \textbf{zinc-carbon dry cell}.}
\item{In a \textbf{lead-acid battery}, each cell consists of electrodes of lead (Pb) and lead (IV) oxide (PbO$_{2}$) in an electrolyte of sulfuric acid (H$_{2}$SO$_{4}$). When the battery discharges, both electrodes turn into lead (II) sulphate (PbSO$_{4}$) and the electrolyte loses sulfuric acid to become mostly water.}
\item{A \textbf{zinc-carbon cell} is made up of an outer zinc container, which acts as the \textbf{anode}. The cathode is the central carbon rod, surrounded by a mixture of carbon and manganese (IV) oxide (MnO$_{2}$). The electrolyte is a paste of ammonium chloride (NH$_{4}$Cl). A fibrous fabric separates the two electrodes, and a brass pin in the centre of the cell conducts electricity to the outside circuit. }
\item{Despite their many advantages, batteries are made of potentially toxic materials and can be damaging to the \textbf{environment}.}
\end{itemize}

\begin{eocexercises}{}

\begin{enumerate}
\item{Give one word or term for each of the following descriptions:}
	\begin{enumerate}
	\item{A solid organic compound that can be used to produce liquid fuels.}
	\item{The process used to convert heavy hydrocarbons into light hydrocarbons.}
	\item{The process of separating nitrogen from liquid air.}
	\item{The main raw material in the chloralkali industry.}
	\item{A compound given to a plant to promote growth.}
	\item{An electrolyte used in lead-acid batteries.}
	\end{enumerate}

\item{Indicate whether each of the following statements is true or false. If the statement is false, rewrite the statement correctly.}
	\begin{enumerate}
	\item{The longer the hydrocarbon chain in an organic compound, the more likely it is to be a solid at room temperature.}
	\item{The main elements used in fertilisers are nitrogen, phosphorus and potassium.}
	\item{A soap molecule is composed of an alcohol molecule and three fatty acids.}
	\item{During the industrial preparation of chlorine and sodium hydroxide, chemical energy is converted to electrical energy.}
	\end{enumerate}

\item{For each of the following questions, choose the one correct answer from the list provided.}

	\begin{enumerate}
	\item{The sequence of processes that best describes the conversion of coal to liquid fuel is:}
		\begin{enumerate}
		\item{coal $\rightarrow$ gas purification $\rightarrow$ SAS reactor $\rightarrow$ liquid hydrocarbon}
		\item{coal $\rightarrow$ autothermal reactor $\rightarrow$ Sasol slurry phase F-T reactor $\rightarrow$ liquid hydrocarbon}
		\item{coal $\rightarrow$ coal purification $\rightarrow$ synthesis gas $\rightarrow$ oil}
		\item{coal $\rightarrow$ coal gasification $\rightarrow$ gas purification $\rightarrow$ SAS reactor $\rightarrow$ liquid hydrocarbons}
		\end{enumerate}

	\item{The half-reaction that takes place at the cathode of a mercury cell in the chloralkali industry is:}
		\begin{enumerate}
		\item{$\rm{2Cl^{-} \rightarrow Cl_{2} + 2e^{-}}$}


		\item{$\rm{2Na^{+} + 2e^{-} \rightarrow 2Na}$}
		\item{$\rm{2H^{+} + 2e^{-} \rightarrow H_{2}}$}
		\item{$\rm{NaCl + H_{2}O \rightarrow NaOH + HCl}$}
		\end{enumerate}



	\item{In a zinc-carbon dry cell...}
		\begin{enumerate}
		\item{the electrolyte is manganese (IV) oxide}
		\item{zinc is oxidised to produce electrons}
		\item{zinc is reduced to produce electrons}
		\item{manganese (IV) dioxide acts as a reducing agent}
		\end{enumerate}

	\end{enumerate}

\item{\textbf{Chloralkali manufacturing process}\\

The chloralkali (also called 'chlorine-caustic') industry is one of the largest electrochemical technologies in the world.  Chlorine is produced using three types of electrolytic cells.  The simplified diagram below shows a membrane cell.

\begin{center}
\begin{pspicture}(-4,0)(4,6)
%\psgrid[gridcolor=lightgray]
\psline(-4,0)(4,0)
\psline(-4,0)(-4,0.5)
\psline(-4,0.5)(-5,0.5)
\psline(-4,1)(-5,1)
\psline[arrows=->](-4.3,0.75)(-5.25,0.75)
\psline(-4,1)(-4,3)



\psline(-4,3)(-5,3)
\psline(-4,3.5)(-5,3.5)
\psline(-4,3.5)(-4,4)
\psline[arrows=->](-5.25,3.25)(-4.25,3.25)
\psline(-4,4)(-3.5,4) \psline(-3.5,4)(-3.5,5) \psline(-3,5)(-3,4)
\psline[arrows=->](-3.25,4.3)(-3.25,5.3)
\psline(-3,4)(3,4) \psline(3,4)(3,5)
\psline(3.5,5)(3.5,4)
\psline[arrows=->](3.25,4.3)(3.25,5.3)
\psline(3.5,4)(4,4) 
\psline(4,4)(4,1)
\psline(4,1)(5,1)
\psline(5,0.5)(4,0.5)
\psline[arrows=->](4.3,0.75)(5.3,0.75)
\psline[linestyle=dashed](0,0)(0,4)
\psline(4,0.5)(4,0)
\psframe(-2.5,1.5)(-2,3)
\rput(-2.25,2.5){+}
\psline(-2.25,3)(-2.25,5.5)
\psline(-2.25,5.5)(-1.5,5.5)
\psframe(-1.5,5)(1.5,6)
\rput(0,5.5){Power supply}
\psline(1.5,5.5)(2.25,5.5)
\psline(2.25,5.5)(2.25,3)
\psframe(2,1.5)(2.5,3)
\rput(2.25,2.5){\textbf{-}}
\rput(-5.5,4){Saturated}
\rput(-5.5,3.7){NaCl}
\rput(-5.5,1.5){Depleted}
\rput(-5.5,1.2){NaCl}
\psline[arrows=->](-4.3,2)(-2.5,2)
\rput(-4.5,2){M}
\rput(3.5,5.5){Gas B}
\rput(-3.5,5.5){Gas A}
\psline[arrows=->](4.3,2)(2.5,2)
\rput(4.5,2){N}
\rput(5.5,1.2){NaOH}
\end{pspicture}
\end{center}

	\begin{enumerate}
	\item{Give two reasons why the membrane cell is the preferred cell for the preparation of chlorine.}
	\item{Why do you think it is advisable to use inert electrodes in this process?}
	\item{Write down the equation for the half-reaction taking place at electrode M.}
	\item{Which gas is chlorine gas? Write down only Gas A or Gas B.}
	\item{Briefly explain how sodium hydroxide forms in this cell.}
	\end{enumerate}

(DoE Exemplar Paper 2,2007)
}

\item{
The production of nitric acid is very important in the manufacture of fertilisers. Look at the diagram below, which shows part of the fertiliser production process, and then answer the questions that follow.

\begin{center}
\begin{pspicture}(-6,6)(4,0.2)
%\psgrid[gridcolor=lightgray]
\psframe(-6,5)(-4,6)
\rput(-5,5.5){(1)}
\psline[arrows=->](-3.7,5.5)(-3.3,5.5)
\rput(-2.7,5.5){N$_{2}$(g)}
\psline[arrows=->](-3,5)(-3,4)
\rput(-2.3,4.5){+ H$_{2}$}
\rput(-1.5,4.5){(3)}
\psframe(-4,2.7)(-2,3.7)
\rput(-3,3.2){(2)}
\rput(-1.3,3.2){+ O$_{2}$}
\psline[arrows=->](-0.4,3.2)(0.6,3.2)
\rput(1.6,3.2){NO + H$_{2}$O}
\psline[arrows=->](1.6,3)(1.6,2)
\rput(3.4,2.5){Ostwald Process}
\psframe(0.6,0.5)(2.6,1.5)
\rput(1.6,1){(4)}
\end{pspicture}
\end{center}

\begin{enumerate}
\item{Name the process at (1).}
\item{Name the gas at (2).}
\item{Name the process at (3) that produces gas (2).}
\item{Name the product at (4).}
\item{Name two fertilisers that can be produced from nitric acid.}
\end{enumerate}
}

\item{
A lead-acid battery has a number of different components. Match the description in Column A with the correct word or phrase in Column B. All the descriptions in Column A relate to lead-acid batteries.

\begin{center}
\begin{tabular}{ll}
\textbf{Column A} & \textbf{Column B} \\ \hline
The electrode metal \ \ \ \ \ \ \ & Lead sulphate \\
Electrolyte & Mercury \\
A product of the overall cell reaction & Electrolytic \\
An oxidising agent in the cathode half-reaction & Lead \\
Type of cells in a lead-acid battery & Sulfuric acid \\
 & Ammonium chloride \\
 & Lead oxide \\
 & Galvanic \\
\end{tabular}

\end{center}

}
\end{enumerate}
\insertpracticeinfo{6}
\end{eocexercises}

