\chapter{Organic Macromolecules}
\label{chap:orgmac}

As its name suggests, a macromolecule is a large molecule that forms when lots of smaller molecules are joined together. In this chapter, we will be taking a closer look at the structure and properties of different macromolecules, and at how they form. 


% CHILD SECTION START 

\section{Polymers}
\label{sec:orgmac:polymers}

Some macromolecules are made up of lots of repeating structural units called \textbf{monomers}. To put it more simply, a monomer is like a building block. When lots of similar monomers are joined together by covalent bonds, they form a \textbf{polymer}. In an \textbf{organic polymer}, the monomers are joined by the \textit{carbon} atoms of the polymer 'backbone'. A polymer can also be \textbf{inorganic}, in which case there may be atoms such as \textit{silicon} in the place of carbon atoms. The key feature that makes a polymer different from other macromolecules, is the repetition of identical or similar monomers in the polymer chain. The examples shown below will help to make these concepts clearer. 


\Definition{Polymer}{Polymer is a term used to describe large molecules consisting of repeating structural units, or monomers, connected by covalent chemical bonds.}

\begin{enumerate}
\item{\textbf{Polyethene}

Chapter \ref{chap:om} looked at the structure of a group of hydrocarbons called the \textit{alkenes}. One example is the molecule \textbf{ethene}. The structural formula of ethene is is shown in figure \ref{fig:orgmac:polyethene}. When lots of ethene molecules bond together, a polymer called \textbf{polyethene} (commonly called polyethylene) is formed. Ethene is the \textit{monomer} which, when joined to other ethene molecules, forms the \textit{polymer} \textbf{polyethene}. Polyethene is a cheap plastic that is used to make plastic bags and bottles. 

\begin{figure}[h]
\begin{center}
\begin{pspicture}(-3,-1.5)(7,1.5)
%\psgrid[gridcolor=lightgray]
\rput(-3,1){\textbf{(a)}}
\rput(-1,0){\textbf{C}}
\rput(0,0){\textbf{C}}
\psline(-0.8,-0.05)(-0.2,-0.05)
\psline(-0.8,0.05)(-0.2,0.05)
\psline(0.2,0.2)(0.7,0.7)
\psline(0.2,-0.2)(0.7,-0.7)
\psline(-1.2,0.2)(-1.7,0.7)
\psline(-1.2,-0.2)(-1.7,-0.7)
\rput(1,1){\textbf{H}}
\rput(1,-1){\textbf{H}}
\rput(-2,1){\textbf{H}}
\rput(-2,-1){\textbf{H}}
\rput(2,1){\textbf{(b)}}
\rput(3,0){\textbf{C}}
\rput(4,0){\textbf{C}}
\rput(5,0){\textbf{C}}
\rput(6,0){\textbf{C}}
\rput(3,1){\textbf{H}}
\rput(4,1){\textbf{H}}
\rput(5,1){\textbf{H}}
\rput(6,1){\textbf{H}}
\rput(3,-1){\textbf{H}}
\rput(4,-1){\textbf{H}}
\rput(5,-1){\textbf{H}}
\rput(6,-1){\textbf{H}}
\psline(2.8,0)(2.2,0)
\psline(3.8,0)(3.2,0)
\psline(4.8,0)(4.2,0)
\psline(5.8,0)(5.2,0)
\psline(6.8,0)(6.2,0)
\psline(3,0.2)(3,0.8)
\psline(4,0.2)(4,0.8)
\psline(5,0.2)(5,0.8)
\psline(6,0.2)(6,0.8)
\psline(3,-0.2)(3,-0.8)
\psline(4,-0.2)(4,-0.8)
\psline(5,-0.2)(5,-0.8)
\psline(6,-0.2)(6,-0.8)
\end{pspicture}
\end{center}
\caption{(a) Ethene monomer and (b) polyethene polymer}
\label{fig:orgmac:polyethene}
\end{figure}

A polymer may be a chain of thousands of monomers, and so it is impossible to draw the entire polymer. Rather, the structure of a polymer can be condensed and represented as shown in figure \ref{fig:orgmac:polyethene n}. The monomer is enclosed in brackets and the 'n' represents the number of ethene molecules in the polymer, where 'n' is any whole number. What this shows is that the ethene monomer is repeated an indefinite number of times in a molecule of polyethene.\\

\begin{figure}[H]
\begin{center}
\begin{pspicture}(-3,-1.5)(2,1.5)
%\psgrid[gridcolor=lightgray]
\rput(-1,0){\textbf{C}}
\rput(-1,1){\textbf{H}}
\rput(-1,-1){\textbf{H}}
\psline(-1.2,0)(-1.8,0)
\psline(-1,0.2)(-1,0.8)
\psline(-1,-0.2)(-1,-0.8)
\rput(1,0){
\rput(-1,0){\textbf{C}}
\rput(-1,1){\textbf{H}}
\rput(-1,-1){\textbf{H}}
\psline(-1.2,0)(-1.8,0)
\psline(-1,0.2)(-1,0.8)
\psline(-1,-0.2)(-1,-0.8)
}
\psline(0.2,0)(0.8,0)
\psline(-2,1)(-2.2,1)

\psline(-2,-1)(-2.2,-1)
\psline(-2.2,1)(-2.2,-1)
\psline(1,1)(1.2,1)
\psline(1,-1)(1.2,-1)
\psline(1.2,1)(1.2,-1)
\rput(-2.4,-.2){\textbf{n}}
\end{pspicture}
\end{center}
\caption{A simplified representation of a polyethene molecule}
\label{fig:orgmac:polyethene n}
\end{figure}
}

\item{\textbf{Polypropene}

Another example of a polymer is \textit{polypropene} (fig \ref{fig:orgmac:polypropene}). Polypropene (commonly known as polypropylene) is also a plastic, but is stronger than polyethene and is used to make crates, fibres and ropes. In this polymer, the monomer is the alkene called \textbf{propene}.\\

\begin{figure}[h]
\begin{center}
\scalebox{.8}{
\begin{pspicture}(-3,-1.5)(11.5,1.5)
%\psgrid[gridcolor=lightgray]
\rput(-3,1){\textbf{(a)}}
\rput(-1,0){\textbf{C}}
\rput(0,0){\textbf{C}}
\psline(-0.8,-0.05)(-0.2,-0.05)
\psline(-0.8,0.05)(-0.2,0.05)
\psline(0.2,0.2)(0.7,0.7)
\psline(0.2,-0.2)(0.7,-0.7)
\psline(-1.2,0.2)(-1.7,0.7)
\psline(-1.2,-0.2)(-1.7,-0.7)
\rput(1,1){\textbf{H}}
\rput(1,-1){\textbf{H}}
\rput(-2,1){\textbf{CH$_{3}$}}
\rput(-2,-1){\textbf{H}}

\rput(2,1){\textbf{(b)}}
\rput(3,0){\textbf{C}}
\rput(4,0){\textbf{C}}
\rput(5,0){\textbf{C}}
\rput(6,0){\textbf{C}}
\rput(3,1){\textbf{CH$_{3}$}}
\rput(4,1){\textbf{H}}
\rput(5,1){\textbf{CH$_{3}$}}
\rput(6,1){\textbf{H}}
\rput(3,-1){\textbf{H}}
\rput(4,-1){\textbf{H}}
\rput(5,-1){\textbf{H}}
\rput(6,-1){\textbf{H}}

\psline(2.8,0)(2.2,0)
\psline(3.8,0)(3.2,0)
\psline(4.8,0)(4.2,0)
\psline(5.8,0)(5.2,0)
\psline(6.8,0)(6.2,0)
\psline(3,0.2)(3,0.8)
\psline(4,0.2)(4,0.8)
\psline(5,0.2)(5,0.8)
\psline(6,0.2)(6,0.8)
\psline(3,-0.2)(3,-0.8)
\psline(4,-0.2)(4,-0.8)
\psline(5,-0.2)(5,-0.8)
\psline(6,-0.2)(6,-0.8)
\rput(7.3,0){\textbf{or}}
\rput(9,0){\textbf{C}}
\rput(9,1){\textbf{CH$_{3}$}}
\rput(9,-1){\textbf{H}}
\psline(8.8,0)(8.2,0)
\psline(9,0.2)(9,0.8)
\psline(9,-0.2)(9,-0.8)
\rput(10,0){\textbf{C}}
\rput(10,1){\textbf{H}}
\rput(10,-1){\textbf{H}}
\psline(9.8,0)(9.2,0)
\psline(10,0.2)(10,0.8)
\psline(10,-0.2)(10,-0.8)
\psline(10.2,0)(10.8,0)
\psline(8,1)(7.8,1)
\psline(8,-1)(7.8,-1)
\psline(7.8,1)(7.8,-1)
\psline(11,1)(11.2,1)
\psline(11,-1)(11.2,-1)
\psline(11.2,1)(11.2,-1)
\rput(7.6,-.6){n}
\end{pspicture}}
\end{center}
\caption{(a) Propene monomer and (b) polypropene polymer}
\label{fig:orgmac:polypropene}
\end{figure}
}
\end{enumerate}



% CHILD SECTION END 



% CHILD SECTION START 

\section{How do polymers form?}
\label{subsec:orgmac:formation}

Polymers are formed through a process called \textbf{polymerisation}, where monomer molecules react together to form a polymer chain. Two types of polymerisation reactions are \textbf{addition polymerisation} and \textbf{condensation polymerisation}.

\Definition{Polymerisation}{
In chemistry, polymerisation is a process of bonding monomers, or \textit{single units} together through a variety of reaction mechanisms to form longer chains called polymers.  
}

\subsection{Addition polymerisation}

In this type of reaction, monomer molecules are added to a growing polymer chain one at a time. No small molecules are eliminated in the process. An example of this type of reaction is the formation of \textit{polyethene} from \textit{ethene} (fig \ref{fig:orgmac:polyethene}). When molecules of ethene are joined to each other, the only thing that changes is that the double bond between the carbon atoms in each ethene monomer is replaced by a single bond so that a new carbon-carbon bond can be formed with the next monomer in the chain. In other words, the monomer is an \textit{unsaturated} compound which, after an addition reaction, becomes a \textit{saturated} compound.

\Extension{Initiation, propagation and termination\\}{

There are three stages in the process of addition polymerisation. \textbf{Initiation} refers to a chemical reaction that triggers off another reaction. In other words, initiation is the starting point of the polymerisation reaction. \textbf{Chain propagation} is the part where monomers are continually added to form a longer and longer polymer chain. During chain propagation, it is the reactive end groups of the polymer chain that react in each propagation step, to add a new monomer to the chain. Once a monomer has been added, the reactive part of the polymer is now in this last monomer unit so that propagation will continue. \textbf{Termination} refers to a chemical reaction that destroys the reactive part of the polymer chain so that propagation stops.
}

\begin{wex}{Polymerisation reactions\\}{A polymerisation reaction takes place and the following polymer is formed:
\begin{center}
\begin{pspicture}(-2,-1.5)(2,1.5)
%\psgrid[gridcolor=lightgray]
\rput(0,0){C}
\psline(-0.3,0)(-0.7,0)
\psline(0,0.3)(0,0.7)
\rput(0,1){W}
\psline(0,-0.3)(0,-0.7)
\rput(0,-1){Y}
\psline(0.3,0)(0.7,0)
\rput(1,0){C}
\psline(1,0.3)(1,0.7)
\rput(1,1){X}
\psline(1,-0.3)(1,-0.7)
\rput(1,-1){Z}
\psline(1.3,0)(1.7,0)
\psline(-1,1)(-1.3,1)
\psline(-1.3,1)(-1.3,-1)
\psline(-1.3,-1)(-1,-1)
\psline(2,1)(2.3,1)
\psline(2.3,1)(2.3,-1)
\psline(2.3,-1)(2,-1)
\rput(3.0,-1){n}
\end{pspicture}
\end{center}

Note: W, X, Y and Z could represent a number of different atoms or combinations of atoms e.g. H, F, Cl or CH$_{3}$. \\

\begin{enumerate}
\item{Give the structural formula of the monomer of this polymer.}
\item{To what group of organic compounds does this monomer belong?}
\item{What type of polymerisation reaction has taken place to join these monomers to form the polymer?}
\end{enumerate}
}{\westep{Look at the structure of the repeating unit in the polymer to determine the monomer.}

The monomer is:

\begin{center}
\scalebox{.8}{
\begin{pspicture}(-1,-1)(1,1)
\rput(-1,0){\textbf{C}}
\rput(0,0){\textbf{C}}
\psline(-0.8,-0.05)(-0.2,-0.05)
\psline(-0.8,0.05)(-0.2,0.05)
\psline(0.2,0.2)(0.7,0.7)
\psline(0.2,-0.2)(0.7,-0.7)
\psline(-1.2,0.2)(-1.7,0.7)
\psline(-1.2,-0.2)(-1.7,-0.7)
\rput(1,1){\textbf{H}}
\rput(1,-1){\textbf{H}}
\rput(-2,1){\textbf{CH$_{3}$}}
\rput(-2,-1){\textbf{H}}
\end{pspicture}}
\end{center}
\westep{Look at the atoms and bonds in the monomer to determine which group of organic compounds it belongs to.}

The monomer has a double bond between two carbon atoms. The monomer must be an alkene.
\westep{Determine the type of polymerisation reaction.}

In this example, unsaturated monomers combine to form a saturated polymer. No atoms are lost or gained for the bonds between monomers to form. They are simply added to each other. This is an addition reaction.
}
\end{wex}

\subsection{Condensation polymerisation}

In this type of reaction, two monomer molecules form a covalent bond and a small molecule such as water is lost in the bonding process. Nearly all biological reactions are of this type. \textbf{Polyester} and \textbf{nylon} are examples of polymers that form through condensation polymerisation.

\begin{enumerate}
\item{\textbf{Polyester}

Polyesters are a group of polymers that contain the \textbf{ester} functional group in their main chain. Although there are many forms of polyesters, the term \textit{polyester} usually refers to polyethylene terephthalate (PET). PET is made from ethylene glycol (an alcohol) and terephthalic acid (a carboxylic acid). In the reaction, a hydrogen atom is lost from the alcohol, and a hydroxyl group is lost from the carboxylic acid. Together these form one water molecule which is lost during condensation reactions. A new bond is formed between an oxygen and a carbon atom. This bond is called an \textbf{ester linkage}. The reaction is shown in figure \ref{fig:orgmac:polyester2}. 

\begin{figure}[h]
\begin{center}
\scalebox{.8} % Change this value to rescale the drawing.
{
\begin{pspicture}(0,-1.4539063)(15.874375,1.4539063)
\usefont{T1}{ptm}{m}{n}
\rput(0.17046875,1.2739062){(a)}
\usefont{T1}{ptm}{m}{n}
\rput(1.1595312,0.27390626){\textbf{HO}}
\psline[linewidth=0.028222222cm](1.5853125,0.27390626)(2.3853126,0.27390626)
\usefont{T1}{ptm}{m}{n}
\rput(3.1595314,0.27390626){\textbf{CH$_{2}$CH$_{2}$}}
\psline[linewidth=0.028222222cm](3.9853125,0.27390626)(4.7853127,0.27390626)
\usefont{T1}{ptm}{m}{n}
\rput(5.159531,0.27390626){\textbf{O H}}
\usefont{T1}{ptm}{m}{n}
\rput(6.659531,0.27390626){\textbf{+}}
\usefont{T1}{ptm}{m}{n}
\rput(8.159532,0.27390626){\textbf{H O}}
\psline[linewidth=0.028222222cm](8.785313,0.27390626)(9.585313,0.27390626)
\usefont{T1}{ptm}{m}{n}
\rput(9.759531,0.27390626){\textbf{C}}
\psline[linewidth=0.028222222cm](9.735312,0.47390625)(9.735312,1.0739063)
\psline[linewidth=0.028222222cm](9.835313,0.47390625)(9.835313,1.0739063)
\usefont{T1}{ptm}{m}{n}
\rput(9.759531,1.2739062){\textbf{O}}
\psline[linewidth=0.028222222cm](9.985312,0.27390626)(10.785313,0.27390626)
\psline[linewidth=0.028222222cm](12.785313,0.27390626)(13.385312,0.27390626)
\usefont{T1}{ptm}{m}{n}
\rput(13.559531,0.27390626){\textbf{C}}
\psline[linewidth=0.028222222cm](13.535313,0.47390625)(13.535313,1.0739063)
\psline[linewidth=0.028222222cm](13.635312,0.47390625)(13.635312,1.0739063)
\usefont{T1}{ptm}{m}{n}
\rput(13.559531,1.2739062){\textbf{O}}
\psline[linewidth=0.028222222cm](13.785313,0.27390626)(14.585313,0.27390626)
\usefont{T1}{ptm}{m}{n}
\rput(14.959531,0.27390626){\textbf{OH}}
\psellipse[linewidth=0.028222222,dimen=outer](5.3853126,0.27390626)(0.25,0.25)
\psellipse[linewidth=0.028222222,dimen=outer](8.185312,0.27390626)(0.4,0.4)
\psline[linewidth=0.028222222,arrowsize=0.05291667cm 2.0,arrowlength=1.4,arrowinset=0.4]{->}(6.6853123,-0.32609376)(6.8853126,-0.72609377)(7.1853123,-0.92609376)
\usefont{T1}{ptm}{m}{n}
\rput(7.6825,-1.2260938){H$_{2}$O molecule lost}
\usefont{T1}{ptm}{m}{n}
\rput(2.6595314,-0.72609377){\textbf{ethylene glycol}}
\usefont{T1}{ptm}{m}{n}
\rput(11.659532,-0.72609377){\textbf{terephthalic acid}}
\pspolygon[linewidth=0.04](11.246667,1.1405728)(12.306666,1.1405729)(12.826667,0.26057294)(12.326667,-0.57942706)(11.306666,-0.57942706)(10.766666,0.32057294)
\pscircle[linewidth=0.04,dimen=outer](11.8,0.29390624){0.74}
\end{pspicture} 
}
% Generated with LaTeXDraw 2.0.5
% Wed Sep 01 18:39:57 SAST 2010
% \usepackage[usenames,dvipsnames]{pstricks}
% \usepackage{epsfig}
% \usepackage{pst-grad} % For gradients
% \usepackage{pst-plot} % For axes
\scalebox{.8} % Change this value to rescale the drawing.
{
\begin{pspicture}(0,-1.1637759)(13.397188,1.143776)
\usefont{T1}{ptm}{m}{n}
\rput(0.6175,-0.7904427){(a)}
\pspolygon[linewidth=0.04](8.753698,0.57622385)(9.813698,0.576224)(10.333698,-0.303776)(9.833698,-1.143776)(8.813698,-1.143776)(8.273698,-0.243776)
\pscircle[linewidth=0.04,dimen=outer](9.307032,-0.2704427){0.74}
\usefont{T1}{ptm}{m}{n}
\rput(0.18046875,0.7153385){(b)}
\usefont{T1}{ptm}{m}{n}
\rput(1.6823438,-0.28466144){\textbf{HO}}
\psline[linewidth=0.028222222cm](2.1260936,-0.28466144)(2.9260938,-0.28466144)
\usefont{T1}{ptm}{m}{n}
\rput(3.6823437,-0.28466144){\textbf{CH$_{2}$CH$_{2}$}}
\psline[linewidth=0.028222222cm](4.526094,-0.28466144)(5.326094,-0.28466144)
\usefont{T1}{ptm}{m}{n}
\rput(5.832344,-0.28466144){\textbf{O}}
\psline[linewidth=0.028222222cm](6.326094,-0.28466144)(7.126094,-0.28466144)
\usefont{T1}{ptm}{m}{n}
\rput(7.2823434,-0.28466144){\textbf{C}}
\psline[linewidth=0.028222222cm](7.276094,-0.084661454)(7.276094,0.51533854)
\psline[linewidth=0.028222222cm](7.376094,-0.084661454)(7.376094,0.51533854)
\usefont{T1}{ptm}{m}{n}
\rput(7.2823434,0.7153385){\textbf{O}}
\psline[linewidth=0.028222222cm](7.526094,-0.28466144)(8.326094,-0.28466144)
\psline[linewidth=0.028222222cm](10.326094,-0.28466144)(10.926094,-0.28466144)
\usefont{T1}{ptm}{m}{n}
\rput(11.082343,-0.28466144){\textbf{C}}
\psline[linewidth=0.028222222cm](11.076094,-0.084661454)(11.076094,0.51533854)
\psline[linewidth=0.028222222cm](11.176094,-0.084661454)(11.176094,0.51533854)
\usefont{T1}{ptm}{m}{n}
\rput(11.082343,0.7153385){\textbf{O}}
\psline[linewidth=0.028222222cm](11.326094,-0.28466144)(12.126093,-0.28466144)
\usefont{T1}{ptm}{m}{n}
\rput(12.482344,-0.28466144){\textbf{OH}}
\psframe[linewidth=0.028222222,linestyle=dashed,dash=0.17638889cm 0.10583334cm,dimen=outer](8.627031,1.143776)(5.887031,-0.7846615)
\end{pspicture} 
}

% \begin{pspicture}(-4,-1.5)(9,2.5)
% %\psgrid[gridcolor=lightgray]
% \rput(-4.5,1){(b)}
% \rput(-3,0){\textbf{HO}}
% \psline(-2.6,0)(-1.8,0)
% \rput(-1,0){\textbf{CH$_{2}$CH$_{2}$}}
% \psline(-0.2,0)(0.6,0)
% \rput(1.15,0){\textbf{O}}
% \psline(1.6,0)(2.4,0)
% \rput(2.6,0){\textbf{C}}
% \psline(2.55,0.2)(2.55,0.8)
% \psline(2.65,0.2)(2.65,0.8)
% \rput(2.6,1){\textbf{O}}
% \psline(2.8,0)(3.6,0)
% \psline(3.6,0)(4.2,0.7)
% \psline(4.2,0.7)(5,0.7)
% \psline(5,0.7)(5.6,0)
% \psline(5.6,0)(5,-0.7)
% \psline(5,-0.7)(4.2,-0.7)
% \psline(4.2,-0.7)(3.6,0)
% \psellipse(4.6,0)(0.3,0.3)
% \psline(5.6,0)(6.2,0)
% \rput(6.4,0){\textbf{C}}
% \psline(6.35,0.2)(6.35,0.8)
% \psline(6.45,0.2)(6.45,0.8)
% \rput(6.4,1){\textbf{O}}
% \psline(6.6,0)(7.4,0)
% \rput(7.8,0){\textbf{OH}}
% \psframe[linestyle=dashed](-0.3,-0.5)(2.4,0.5)
% \rput(1,-1.1){ester linkage}
% \end{pspicture}
\caption{An acid and an alcohol monomer react (a) to form a molecule of the polyester 'polyethylene terephthalate' (b).}
\label{fig:orgmac:polyester2}
\end{center}
\end{figure}
}

Polyesters have a number of characteristics which make them very useful. They are resistant to stretching and shrinking, they are easily washed and dry quickly, and they are resistant to mildew. It is for these reasons that polyesters are being used more and more in \textbf{textiles}. Polyesters are stretched out into fibres and can then be made into fabric and articles of clothing. In the home, polyesters are used to make clothing, carpets, curtains, sheets, pillows and upholstery. 

\begin{IFact}{
Polyester is not just a textile. Polyethylene terephthalate is in fact a plastic which can also be used to make plastic drink bottles. Many drink bottles are recycled by being reheated and turned into polyester fibres. This type of recycling helps to reduce disposal problems.}
\end{IFact}

\item{\textbf{Nylon}

Nylon was the first polymer to be commercially successful. Nylon replaced silk, and was used to make parachutes during World War 2. Nylon is very strong and resistant, and is used in fishing line, shoes, toothbrush bristles, guitar strings and machine parts to name just a few. Nylon is formed from the reaction of an amine (1,6-diaminohexane) and an acid monomer (adipic acid) (figure \ref{fig:orgmac:nylon}). The bond that forms between the two monomers is called an \textbf{amide linkage}. An amide linkage forms between a nitrogen atom in the amine monomer and the carbonyl group in the carboxylic acid.

\begin{figure}[!h]
\begin{center}
\begin{pspicture}(-4,-2)(10.5,1.5)
%\psgrid[gridcolor=lightgray]
\rput(-4,1){(a)}
\rput(-3,0){\textbf{H}}
\psline(-2.8,0)(-2.2,0)
\rput(-2,0){\textbf{N}}
\psline(-1.8,0)(-1.2,0)
\rput(-0.5,0){\textbf{(CH$_{2}$)$_{4}$}}
\psline(-2,0.2)(-2,0.8)
\rput(-2,1){\textbf{H}}
\psline(0.2,0)(0.8,0)
\rput(1,0){\textbf{N}}
\psline(1,0.2)(1,0.8)
\rput(1,1){\textbf{H}}
\psline(1.2,0)(1.8,0)
\rput(2,0){\textbf{ H}}
\rput(3,0){\textbf{+}}
\rput(4,0){\textbf{HO}}
\psline(4.6,0)(5.2,0)
\rput(5.4,0){\textbf{C}}
\psline(5.35,0.2)(5.35,0.8)
\psline(5.45,0.2)(5.45,0.8)
\rput(5.4,1){\textbf{O}}
\psline(5.6,0)(6.2,0)
\rput(6.8,0){\textbf{(CH$_{2}$)$_{4}$}}
\psline(7.6,0)(8.2,0)
\rput(8.4,0){\textbf{C}}
\psline(8.35,0.2)(8.35,0.8)
\psline(8.45,0.2)(8.45,0.8)
\rput(8.4,1){\textbf{O}}
\psline(8.6,0)(9.2,0)
\rput(9.6,0){\textbf{OH}}
\psellipse(2.1,0)(0.25,0.25)
\psellipse(4,0)(0.4,0.4)
\psline[linearc=0.25]{->}(3,-0.6)(3.2,-1)(3.5,-1.2)


\rput(3.5,-1.5){H$_{2}$O molecule is lost}
\end{pspicture}

\begin{pspicture}(-4,-1)(8,2)
%\psgrid[gridcolor=lightgray]
\rput(-5,1){(b)}
\rput(-3,0){\textbf{H}}
\psline(-2.8,0)(-2.2,0)
\rput(-2,0){\textbf{N}}
\psline(-1.8,0)(-1.2,0)
\rput(-0.5,0){\textbf{(CH$_{2}$)$_{4}$}}
\psline(-2,0.2)(-2,0.8)
\rput(-2,1){\textbf{H}}
\psline(0.2,0)(0.8,0)
\rput(1,0){\textbf{N}}
\psline(1,0.2)(1,0.8)
\rput(1,1){\textbf{H}}
\psline(1.2,0)(1.8,0)
\rput(2.2,0){\textbf{C}}
\psline(2.15,0.2)(2.15,0.8)
\psline(2.25,0.2)(2.25,0.8)
\rput(2.2,1){\textbf{O}}
\psline(2.4,0)(3,0)
\rput(3.8,0){\textbf{(CH$_{2}$)$_{4}$}}
\psline(4.6,0)(5.2,0)
\rput(5.4,0){\textbf{C}}
\psline(5.35,0.2)(5.35,0.8)
\psline(5.45,0.2)(5.45,0.8)
\rput(5.4,1){\textbf{O}}
\psline(5.6,0)(6.2,0)
\rput(6.6,0){\textbf{OH}}
\psline[linestyle=dashed](0.2,-0.4)(2.4,1.5)
\psline[linestyle=dashed](2.4,1.5)(2.4,-0.4)
\psline[linestyle=dashed](2.4,-0.4)(0.2,-0.4)
\rput(1.3,-1){amide linkage}
\end{pspicture}
\end{center}
\caption{An amine and an acid monomer (a) combine to form a section of a nylon polymer (b).}
\label{fig:orgmac:nylon}
\end{figure}
}
\end{enumerate}

\begin{IFact}{
Nylon was first introduced around 1939 and was in high demand to make stockings. However, as World War 2 progressed, nylon was used more and more to make parachutes, and so stockings became more difficult to buy. After the war, when manufacturers were able to shift their focus from parachutes back to stockings, a number of riots took place as women queued to get stockings. In one of the worst disturbances, 40 000 women queued up for 13 000 pairs of stockings, which led to fights breaking out!
}
\end{IFact}

\Exercise{Polymers\\}{

\begin{enumerate}
\item{The following monomer is a reactant in a polymerisation reaction:

\begin{center}
\scalebox{.8}{
\begin{pspicture}(-1,-1)(1,1)
\rput(-1,0){\textbf{C}}
\rput(0,0){\textbf{C}}
\psline(-0.8,-0.05)(-0.2,-0.05)
\psline(-0.8,0.05)(-0.2,0.05)
\psline(0.2,0.2)(0.7,0.7)
\psline(0.2,-0.2)(0.7,-0.7)
\psline(-1.2,0.2)(-1.7,0.7)
\psline(-1.2,-0.2)(-1.7,-0.7)
\rput(1,1){\textbf{H}}
\rput(1,-1){\textbf{H}}
\rput(-2,1){\textbf{CH$_{3}$}}
\rput(-2,-1){\textbf{H}}
\end{pspicture}}
\end{center}

	\begin{enumerate}
	\item{What is the IUPAC name of this monomer?}
	\item{Give the structural formula of the polymer that is formed in this polymerisation reaction.}
	\item{Is the reaction an addition or condensation reaction?}
	\end{enumerate}
}

\item{The polymer below is the product of a polymerisation reaction.

\begin{center}
\begin{pspicture}(-2,-1.8)(2,1.5)
%\psgrid[gridcolor=lightgray]
\rput(-2,0){
\rput(-1,0){
\rput(0,0){C}
\psline(0,0.3)(0,0.7)
\rput(0,1){H}
\psline(0,-0.3)(0,-0.7)
\rput(0,-1){H}
\psline(0.3,0)(0.7,0)
\psline(0.3,0)(0.7,0)
\rput(1,0){C}
\psline(1,0.3)(1,0.7)
\rput(1,1){Cl}
\psline(1,-0.3)(1,-0.7)
\rput(1,-1){H}
\psline(-0.3,0)(-0.7,0)
}
\rput(2,0){
\rput(-1,0){
\rput(0,0){C}
\psline(0,0.3)(0,0.7)
\rput(0,1){H}
\psline(0,-0.3)(0,-0.7)
\rput(0,-1){H}
\psline(0.3,0)(0.7,0)
\psline(0.3,0)(0.7,0)
\rput(1,0){C}
\psline(1,0.3)(1,0.7)
\rput(1,1){Cl}
\psline(1,-0.3)(1,-0.7)
\rput(1,-1){H}
\psline(-0.3,0)(-0.7,0)
}
}
\rput(4,0){
\rput(-1,0){
\rput(0,0){C}
\psline(0,0.3)(0,0.7)
\rput(0,1){H}
\psline(0,-0.3)(0,-0.7)
\rput(0,-1){H}
\psline(0.3,0)(0.7,0)
\psline(0.3,0)(0.7,0)
\rput(1,0){C}
\psline(1,0.3)(1,0.7)
\rput(1,1){Cl}
\psline(1,-0.3)(1,-0.7)
\rput(1,-1){H}
\psline(-0.3,0)(-0.7,0)
}
}
}
\psline(2.3,0)(2.7,0)
\end{pspicture}
\end{center}

	\begin{enumerate}
	\item{Give the structural formula of the monomer in this polymer.}
	\item{What is the name of the monomer?}
	\item{Draw the abbreviated structural formula for the polymer.}
	\item{Has this polymer been formed through an addition or condensation polymerisation reaction?}
	\end{enumerate}

}

\item{A condensation reaction takes place between methanol and methanoic acid.}
	\begin{enumerate}
	\item{Give the structural formula for:}
		\begin{enumerate}
		\item{methanol}	
		\item{methanoic acid}
		\item{the product of the reaction}
		\end{enumerate}
	\item{What is the name of the product? (Hint: The product is an ester)}
	\end{enumerate}

\end{enumerate}

% Automatically inserted shortcodes - number to insert 3
\par \practiceinfo
\par \begin{tabular}[h]{cccccc}
% Question 1
(1.)	01pd	&
% Question 2
(2.)	01pe	&
% Question 3
(3.)	01pf	&
\end{tabular}
% Automatically inserted shortcodes - number inserted 3
}


% CHILD SECTION END 



% CHILD SECTION START 

\section{The chemical properties of polymers}
\label{subsec:orgmac:properties}

The attractive forces between polymer chains play a large part in determining a polymer's properties. Because polymer chains are so long, these interchain forces are very important. It is usually the side groups on the polymer that determine what types of intermolecular forces will exist. The greater the strength of the intermolecular forces, the greater will be the tensile strength and melting point of the polymer. Below are some examples:

\begin{itemize}
\item{\textbf{Hydrogen bonds between adjacent chains}

Polymers that contain amide or carbonyl groups can form \textit{hydrogen bonds} between adjacent chains. The positive hydrogen atoms in the N-H groups of one chain are strongly attracted to the oxygen atoms (more precisely, the lone-pairs on the oxygen) in the C=O groups on another. Polymers that contain \textit{urea} linkages would fall into this category. The structural formula for urea is shown in figure \ref{fig:orgmac:urea}. Polymers that contain urea linkages have high tensile strength and a high melting point.

\begin{figure}[h]
\begin{center}
\begin{pspicture}(-2,-1.5)(2,1.5)
%\psgrid[gridcolor=lightgray]
\rput(0,0){\textbf{C}}
\rput(0,1){\textbf{O}}
\psline(-0.05,0.2)(-0.05,0.8)
\psline(0.05,0.2)(0.05,0.8)
\rput(-1,-1){\textbf{H$_{2}$N}}
\rput(1,-1){\textbf{NH$_{2}$}}
\psline(-0.2,-0.2)(-0.8,-0.8)
\psline(0.2,-0.2)(0.8,-0.8)
\end{pspicture}
\end{center}
\caption{The structural formula for urea}
\label{fig:orgmac:urea}
\end{figure}
}

\item{
\textbf{Dipole-dipole bonds between adjacent chains}

Polyesters have \textit{dipole-dipole bonding} between their polymer chains. Dipole bonding is not as strong as hydrogen bonding, so a polyester's melting point and strength are lower than those of the polymers where there are hydrogen bonds between the chains. However, the weaker bonds between the chains means that polyesters have greater flexibility. The greater the flexibility of a polymer, the more likely it is to be moulded or stretched into fibres.
}
\item{\textbf{Weak van der Waal's forces}

Other molecules such as ethene do not have a permanent dipole and so the attractive forces between polyethene chains arise from weak \textit{van der Waals} forces. Polyethene therefore has a lower melting point than many other polymers.
}
\end{itemize}




% CHILD SECTION END 



% CHILD SECTION START 

\section{Types of polymers}
\label{subsec:orgmac:types}

There are many different types of polymers. Some are organic, while others are inorganic. Organic polymers can be broadly grouped into either synthetic/semi-synthetic (artificial) or biological (natural) polymers. We are going to take a look at two groups of organic polymers: \textit{plastics}, which are usually synthetic or semi-synthetic and \textit{biological macromolecules} which are natural polymers. Both of these groups of polymers play a very important role in our lives.


% CHILD SECTION END 



% CHILD SECTION START 

\section{Plastics}
\label{sec:orgmac:plastics}

In today's world, we can hardly imagine life without plastic. From cellphones to food packaging, fishing line to plumbing pipes, compact discs to electronic equipment, plastics have become a very important part of our daily lives.

\Definition{Plastic}{The term plastic covers a range of synthetic or semisynthetic organic polymers. Plastics may contain other substances to improve their performance. Their name comes from the fact that many of them are malleable, in other words they have the property of plasticity.}

It was only in the nineteenth century that it was discovered that plastics could be made by chemically changing natural polymers. For centuries before this, only \text{natural} organic polymers had been used. Examples of natural organic polymers include \textit{waxes} from plants, \textit{cellulose} (a plant polymer used in fibres and ropes) and \textit{natural rubber} from rubber trees. But in many cases, these natural organic polymers didn't have the characteristics that were needed for them to be used in specific ways. Natural rubber for example, is sensitive to temperature and becomes sticky and smelly in hot weather and brittle in cold weather. \\

In 1834 two inventors, Friedrich Ludersdorf of Germany and Nathaniel Hayward of the US, independently discovered that adding sulfur to raw rubber helped to stop the material from becoming sticky. After this, Charles Goodyear discovered that heating this modified rubber made it more resistant to abrasion, more elastic and much less sensitive to temperature. What these inventors had done was to improve the properties of a natural polymer so that it could be used in new ways. An important use of rubber now is in vehicle tyres, where these properties of rubber are critically important.

\begin{IFact}{
The first true plastic (i.e. one that was not based on any material found in nature) was \textit{Bakelite}, a cheap, strong and durable plastic. Some of these plastics are still used for example in electronic circuit boards, where their properties of insulation and heat resistance are very important.}
\end{IFact}

\subsection{The uses of plastics}

There is such a variety of different plastics available, each having their own specific properties and uses. The following are just a few examples.

\begin{itemize}
\item{
\textbf{Polystyrene}

Polystyrene (figure \ref{fig:orgmac:polystyrene}) is a common plastic that is used in model kits, disposable eating utensils and a variety of other products. In the polystyrene polymer, the monomer is \textit{styrene}, a liquid hydrocarbon that is manufactured from petroleum.

\begin{figure}[H]
\begin{center}
\scalebox{.8} % Change this value to rescale the drawing.
{
\begin{pspicture}(0,-2.11)(14.273173,2.13)
\psline[linewidth=0.028222222cm](0.85411114,-0.1184375)(0.85411114,0.6815625)
\usefont{T1}{ptm}{m}{n}
\rput(0.96832985,0.8615625){\textbf{CH}}
\usefont{T1}{ptm}{m}{n}
\rput(1.7883298,1.8815625){\textbf{CH$_{2}$}}
\psline[linewidth=0.028222222cm,arrowsize=0.05291667cm 2.0,arrowlength=1.4,arrowinset=0.4]{->}(2.3741112,-0.9384375)(4.374111,-0.9384375)
\usefont{T1}{ptm}{m}{n}
\rput(3.3739548,-0.7384375){polymerisation}
\usefont{T1}{ptm}{m}{n}
\rput(6.34833,0.8615625){\textbf{CH}}
\psline[linewidth=0.028222222cm](6.424111,1.0615625)(7.22,1.75)
\usefont{T1}{ptm}{m}{n}
\rput(7.60833,1.8815625){\textbf{CH$_{2}$}}
\usefont{T1}{ptm}{m}{n}
\rput(8.84833,0.8615625){\textbf{CH}}
\usefont{T1}{ptm}{m}{n}
\rput(11.34833,0.8615625){\textbf{CH}}
\psline[linewidth=0.028222222cm](7.88,1.71)(8.674111,1.0615625)
\usefont{T1}{ptm}{m}{n}
\rput(13.861455,-0.9384375){etc}
\pspolygon[linewidth=0.04](0.84,-2.07)(0.0,-1.518)(0.0,-0.57336)(0.86,-0.09)(1.74,-0.59)(1.74,-1.53)
\pscircle[linewidth=0.04,dimen=outer](0.88,-1.07){0.72}
\psline[linewidth=0.028222222cm](6.394111,-0.0984375)(6.394111,0.7015625)
\pspolygon[linewidth=0.04](6.36,-2.05)(5.52,-1.498)(5.52,-0.55336)(6.38,-0.07)(7.26,-0.57)(7.26,-1.51)
\pscircle[linewidth=0.04,dimen=outer](6.4,-1.05){0.72}
\psline[linewidth=0.028222222cm](8.874111,-0.0984375)(8.874111,0.7015625)
\pspolygon[linewidth=0.04](8.84,-2.07)(8.0,-1.518)(8.0,-0.57336)(8.86,-0.09)(9.74,-0.59)(9.74,-1.53)
\pscircle[linewidth=0.04,dimen=outer](8.88,-1.07){0.72}
\psline[linewidth=0.028222222cm](11.374111,-0.1184375)(11.374111,0.6815625)
\pspolygon[linewidth=0.04](11.36,-2.09)(10.52,-1.538)(10.52,-0.59336)(11.38,-0.11)(12.26,-0.61)(12.26,-1.55)
\pscircle[linewidth=0.04,dimen=outer](11.4,-1.09){0.72}
\psline[linewidth=0.028222222cm](9.124111,1.0815625)(9.92,1.77)
\usefont{T1}{ptm}{m}{n}
\rput(10.22833,1.9415625){\textbf{CH$_{2}$}}
\psline[linewidth=0.028222222cm](10.58,1.73)(11.374111,1.0815625)
\psline[linewidth=0.028222222cm](11.724112,1.1015625)(12.52,1.79)
\usefont{T1}{ptm}{m}{n}
\rput(12.82833,1.9215626){\textbf{CH$_{2}$}}
\psline[linewidth=0.028222222cm](13.02,1.75)(13.814111,1.1015625)
\psline[linewidth=0.04cm,doubleline=true,doublesep=0.06](0.96,1.03)(1.52,1.73)
\end{pspicture} 
}

\end{center}
\caption{The polymerisation of a styrene monomer to form a polystyrene polymer}
\label{fig:orgmac:polystyrene}
\end{figure}
}

\item{
\textbf{Polyvinylchloride (PVC)}

Polyvinyl chloride (PVC) (figure \ref{fig:orgmac:pvc}) is used in plumbing, gutters, electronic equipment, wires and food packaging. The side chains of PVC contain chlorine atoms, which give it its particular characteristics.

\begin{figure}[H]
\begin{center}
\begin{pspicture}(-3,-1)(2,1.3)
%\psgrid[gridcolor=lightgray]
\rput(-1,0){\textbf{C}}
\rput(0,0){\textbf{C}}
\rput(-2,1){\textbf{H}}
\rput(-2,-1){\textbf{H}}
\rput(1,1){\textbf{Cl}}
\rput(1,-1){\textbf{H}}
\psline(-1.2,0.2)(-1.8,0.8)
\psline(-1.2,-0.2)(-1.8,-0.8)
\psline(0.2,0.2)(0.8,0.8)
\psline(0.2,-0.2)(0.8,-0.8)
\psline(-0.8,0.05)(-0.2,0.05)
\psline(-0.8,-0.05)(-0.2,-0.05)
\psline(-2.2,1)(-2.4,1)
\psline(-2.2,-1)(-2.4,-1)
\psline(-2.4,1)(-2.4,-1)
\psline(1.2,1)(1.4,1)
\psline(1.2,-1)(1.4,-1)
\psline(1.4,1)(1.4,-1)
\rput(-2.6,-.2){\textbf{n}}
\end{pspicture}
\end{center}
\caption{Polyvinyl chloride}
\label{fig:orgmac:pvc}
\end{figure}
}

\begin{IFact}{

Many vinyl products have other chemicals added to them to give them particular properties. Some of these chemicals, called additives, can leach out of the vinyl products. In PVC, \textit{plasticizers} are used to make PVC more flexible. Because many baby toys are made from PVC, there is concern that some of these products may leach into the mouths of the babies that are chewing on them. In the USA, most companies have stopped making PVC toys. There are also concerns that some of the plasticizers added to PVC may cause a number of health conditions including cancer.
}
\end{IFact}

\item{
\textbf{Synthetic rubber}

Another plastic that was critical to the World War 2 effort was \textit{synthetic rubber}, which was produced in a variety of forms. Not only were worldwide natural rubber supplies limited, but most rubber-producing areas were under Japanese control. Rubber was needed for tyres and parts of war machinery. After the war, synthetic rubber also played an important part in the space race and nuclear arms race.
}

\item{
\textbf{Polyethene/polyethylene (PE)}

Polyethylene (figure \ref{fig:orgmac:polyethene}) was discovered in 1933. It is a cheap, flexible and durable plastic and is used to make films and packaging materials, containers and car fittings. One of the most well known polyethylene products is 'Tupperware', the sealable food containers designed by Earl Tupper and promoted through a network of housewives!
}

\item{
\textbf{Polytetrafluoroethylene (PTFE)}

Polytetrafluoroethylene (figure \ref{fig:orgmac:teflon}) is more commonly known as 'Teflon' and is most well known for its use in non-stick frying pans. Teflon is also used to make the breathable fabric Gore-Tex.

\begin{figure}[H]
\begin{center}
\begin{pspicture}(-4,-2)(5,2)
%\psgrid[gridcolor=lightgray]
\rput(-3,0){
\rput(-1,0){\textbf{C}}
\rput(0,0){\textbf{C}}
\psline(-0.8,-0.05)(-0.2,-0.05)
\psline(-0.8,0.05)(-0.2,0.05)
\psline(0.2,0.2)(0.7,0.7)
\psline(0.2,-0.2)(0.7,-0.7)
\psline(-1.2,0.2)(-1.7,0.7)
\psline(-1.2,-0.2)(-1.7,-0.7)
\rput(1,1){\textbf{F}}
\rput(1,-1){\textbf{F}}
\rput(-2,1){\textbf{F}}
\rput(-2,-1){\textbf{F}}}
\psline[arrows=->](-1.5,0)(0,0)
\rput(5,0){
\rput(-3,0){\textbf{C}}
\rput(-2,0){\textbf{C}}
\rput(-3,1){\textbf{F}}
\rput(-3,-1){\textbf{F}}
\rput(-2,1){\textbf{F}}
\rput(-2,-1){\textbf{F}}
\psline(-2.8,0)(-2.2,0)
\psline(-3,0.2)(-3,0.8)
\psline(-3,-0.2)(-3,-0.8)
\psline(-2,0.2)(-2,0.8)
\psline(-2,-0.2)(-2,-0.8)
\psline(-3.2,0)(-3.8,0)
\psline(-1.8,0)(-1.2,0)
}
\psline(1,1)(0.8,1)
\psline(1,-1)(0.8,-1)
\psline(0.8,1)(0.8,-1)
\psline(4,1)(4.2,1)
\psline(4,-1)(4.2,-1)
\psline(4.2,1)(4.2,-1)
\rput(.5,-.2){\textbf{n}}
\end{pspicture}
\end{center}
\caption{A tetra fluoroethylene monomer and polytetrafluoroethylene polymer}
\label{fig:orgmac:teflon}
\end{figure}
}
\end{itemize} 

Table \ref{tab:plastics} summarises the formulae, properties and uses of some of the most common plastics.

\begin{table}[H]
\begin{center}
\begin{tabular}{|p{3.2cm}|p{2.8cm}|p{2cm}|p{3cm}|p{3cm}|}\hline
\textbf{Name} & \textbf{Formula} & \textbf{Monomer} & \textbf{Properties} & \textbf{Uses} \\\hline
Polyethene (low density) & -(CH$_{2}$-CH$_{2}$)$_{n}$- & CH$_{2}$=CH$_{2}$ & soft, waxy solid & film wrap and plastic bags \\\hline
Polyethene (high density) & -(CH$_{2}$-CH$_{2}$)$_{n}$- & CH$_{2}$=CH$_{2}$ & rigid & electrical insulation, bottles and toys \\\hline
Polypropene & -[CH$_{2}$-CH(CH$_{3}$)]$_{n}$- & CH$_{2}$=CHCH$_{3}$ & different grades: some are soft and others hard & carpets and upholstery \\\hline
Polyvinylchloride (PVC) & -(CH$_{2}$-CHCl)$_{n}$- & CH$_{2}$=CHCl & strong, rigid & pipes, flooring \\\hline
Polystyrene & -[CH$_{2}$-CH(C$_{6}$H$_{5}$)]$_{n}$ & CH$_{2}$=CHC$_{6}$H$_{5}$ & hard, rigid & toys, packaging \\\hline
Polytetrafluoroethylene & -(CF$_{2}$-CF$_{2}$)$_{n}$- & CF$_{2}$=CF$_{2}$ & resistant, smooth, solid & non-stick surfaces, electrical insulation \\\hline
\end{tabular}
\caption{A summary of the formulae, properties and uses of some common plastics}
\label{tab:plastics}
\end{center}
\end{table}

\Exercise{Plastics\\}{

\begin{enumerate}
\item{
It is possible for macromolecules to be composed of more than one type of repeating monomer. The resulting polymer is called a \textbf{copolymer}. Varying the monomers that combine to form a polymer, is one way of controlling the properties of the resulting material. Refer to the table below which shows a number of different copolymers of rubber, and answer the questions that follow:}

\begin{center}
\begin{tabular}{|l|l|p{2cm}|p{2cm}|}\hline
\textbf{Monomer A} & \textbf{Monomer B} & \textbf{Copolymer} & \textbf{Uses}\\\hline
H$_{2}$C=CHCl & H$_{2}$C=CCl$_{2}$ & Saran & films and fibres \\\hline 
H$_{2}$C=CHC$_{6}$H$_{5}$ & H$_{2}$C=C-CH=CH$_{2}$ & SBR (styrene butadiene rubber) & tyres \\\hline 
H$_{2}$C=CHCN &  H$_{2}$C=C-CH=CH$_{2}$ & Nitrile rubber & adhesives and hoses \\\hline
H$_{2}$C=C(CH$_{3}$)$_{2}$ & H$_{2}$C=C-CH=CH$_{2}$ & Butyl rubber & inner tubes \\\hline
F$_{2}$C=CF(CF$_{3}$) & H$_{2}$C=CHF & Viton & gaskets \\\hline
\end{tabular}
\end{center}

	\begin{enumerate}
	\item{Give the structural formula for each of the monomers of nitrile rubber.}
	\item{Give the structural formula of the copolymer viton.}
	\item{In what ways would you expect the properties of SBR to be different from nitrile rubber?}
	\item{Suggest a reason why the properties of these polymers are different.}
	\end{enumerate}

\item{In your home, find as many examples of different types of plastics that you can. Bring them to school and show them to your group. Together, use your examples to complete the following table:}

\begin{center}
\begin{tabular}{|l|l|l|l|}\hline
\textbf{Object} & \textbf{Type of plastic} & \textbf{Properties} & \textbf{Uses}\\\hline
 & & & \\\hline
 & & & \\\hline
 & & & \\\hline
 & & & \\\hline
 & & & \\\hline
\end{tabular}
\end{center}
\end{enumerate}

% Automatically inserted shortcodes - number to insert 2
\par \practiceinfo
\par \begin{tabular}[h]{cccccc}
% Question 1
(1.)	01pg	&
% Question 2
(2.)	01ph	&
\end{tabular}
% Automatically inserted shortcodes - number inserted 2

}


\subsection{Thermoplastics and thermosetting plastics}

A \textbf{thermoplastic} is a plastic that can be melted to a liquid when it is heated and freezes to a brittle, glassy state when it is cooled enough. These properties of thermoplastics are mostly due to the fact that the forces between chains are weak. This also means that these plastics can be easily stretched or moulded into any shape. Examples of thermoplastics include nylon, polystyrene, polyethylene, polypropylene and PVC. Thermoplastics are more easily recyclable than some other plastics.\\

\textbf{Thermosetting plastics} differ from thermoplastics because once they have been formed, they cannot be remelted or remoulded. Examples include bakelite, vulcanised rubber, melanine (used to make furniture), and many glues. Thermosetting plastics are generally stronger than thermoplastics and are better suited to being used in situations where there are high temperatures. They are not able to be recycled. Thermosetting plastics have strong covalent bonds between chains and this makes them very strong.

\Activity{Case Study}{Biodegradable plastics}{
\textit{Read the article below and then answer the questions that follow.}\\

Our whole world seems to be wrapped in plastic. Almost every product we buy, most of the food we eat and many of the liquids we drink come encased in plastic. Plastic packaging provides excellent protection for the product, it is cheap to manufacture and seems to last forever. Lasting forever, however, is proving to be a major environmental problem. Another problem is that traditional plastics are manufactured from non-renewable resources - oil, coal and natural gas.  In an effort to overcome these problems, researchers and engineers have been trying to develop biodegradable plastics that are made from renewable resources, such as plants.\\ 

The term biodegradable means that a substance can be broken down into simpler substances by the activities of living organisms, and therefore is unlikely to remain in the environment. The reason most plastics are not biodegradable is because their long polymer molecules are too large and too tightly bonded together to be broken apart and used by decomposer organisms. However, plastics based on natural plant polymers that come from wheat or corn starch have molecules that can be more easily broken down by microbes.\\

Starch is a natural polymer. It is a white, granular carbohydrate produced by plants during photosynthesis and it serves as the plant's energy store. Many plants contain large amounts of starch. Starch can be processed directly into a bioplastic but, because it is soluble in water, articles made from starch will swell and deform when exposed to moisture, and this limits its use. This problem can be overcome by changing starch into a different polymer. First, starch is harvested from corn, wheat or potatoes, then microorganisms transform it into lactic acid, a monomer. Finally, the lactic acid is chemically treated to cause the molecules of lactic acid to link up into long chains or polymers, which bond together to form a plastic called polylactide (PLA).\\

PLA can be used for products such as plant pots and disposable nappies. It has been commercially available in some countries since 1990, and certain blends have proved successful in medical implants, sutures and drug delivery systems because they are able to dissolve away over time. However, because PLA is much more expensive than normal plastics, it has not become as popular as one would have hoped.\\

\textbf{Questions}

\begin{enumerate}
\item{In your own words, explain what is meant by a 'biodegradable plastic'.}
\item{Using your knowledge of chemical bonding, explain why some polymers are biodegradable and others are not.}
\item{Explain why lactic acid is a more useful monomer than starch, when making a biodegradable plastic.}
\item{If you were a consumer (shopper), would you choose to buy a biodegradable plastic rather than another? Explain your answer.}
\item{What do you think could be done to make biodegradable plastics more popular with consumers?}
\end{enumerate}
}

\subsection{Plastics and the environment}

Although plastics have had a huge impact globally, there is also an environmental price that has to be paid for their use. The following are just some of the ways in which plastics can cause damage to the environment.

\begin{enumerate}
\item{\textbf{Waste disposal}

Plastics are not easily broken down by micro-organisms and therefore most are not easily biodegradeable. This leads to waste dispoal problems.
}
\item{\textbf{Air pollution}

When plastics burn, they can produce toxic gases such as carbon monoxide, hydrogen cyanide and hydrogen chloride (particularly from PVC and other plastics that contain chlorine and nitrogen).
}

\item{\textbf{Recycling}

It is very difficult to recycle plastics because each type of plastic has different properties and so different recycling methods may be needed for each plastic. However, attempts are being made to find ways of recycling plastics more effectively. Some plastics can be remelted and re-used, while others can be ground up and used as a filler. However, one of the problems with recycling plastics is that they have to be sorted according to plastic \textit{type}. This process is difficult to do with machinery, and therefore needs a lot of labour. Alternatively, plastics should be re-used. In many countries, including South Africa, shoppers must now pay for plastic bags. This encourages people to collect and re-use the bags they already have.
}
\end{enumerate}


\Activity{Case Study}{Plastic pollution in South Africa\\}{

\textit{Read the following extract, taken from 'Planet Ark' (September 2003), and then answer the questions that follow.}\\

\begin{quote}{
South Africa launches a programme this week to exterminate its "national flower" - the millions of used plastic bags that litter the landscape. \\ 

Beginning on Friday, plastic shopping bags used in the country must be both thicker and more recyclable, a move officials hope will stop people from simply tossing them away. "Government has targeted plastic bags because they are the most visible kind of waste," said Phindile Makwakwa, spokeswoman for the Department of Environmental Affairs and Tourism. "But this is mostly about changing people's mindsets about the environment."\\

South Africa is awash in plastic pollution. Plastic bags are such a common eyesore that they are dubbed "roadside daisies" and referred to as the national flower. Bill Naude of the Plastics Federation of South Africa said the country used about eight billion plastic bags annually, a figure which could drop by 50 percent if the new law works.\\}
\end{quote}

It is difficult sometimes to imagine exactly how much waste is produced in our country every year. Where does all of this go to? You are going to do some simple calculations to try to estimate the volume of plastic packets that is produced in South Africa every year.

\begin{enumerate}
\item{Take a plastic shopping packet and squash it into a tight ball.} 
	\begin{enumerate}
	\item{Measure the approximate length, breadth and depth of your squashed plastic bag.}
	\item{Calculate the approximate volume that is occupied by the packet.}
	\item{Now calculate the approximate volume of your classroom by measuring its length, breadth and height.}
	\item{Calculate the number of squashed plastic packets that would fit into a classroom of this volume.}
	\item{If South Africa produces an average of 8 billion plastic bags each year, how many clasrooms would be filled if all of these bags were thrown away and not re-used?}
	\end{enumerate}
\item{What has South Africa done to try to reduce the number of plastic bags that are produced?}
\item{Do you think this has helped the situation?}
\item{What can \textit{you} do to reduce the amount of plastic that you throw away?}
\end{enumerate}
}


% CHILD SECTION END 



% CHILD SECTION START 

\section{Biological Macromolecules}
\label{sec:orgmac:bm}

A \textit{biological macromolecule} is one that is found in living organisms. Biological macromolecules include molecules such as carbohydrates, proteins and nucleic acids. Lipids are also biological macromolecules. They are essential for all known forms of life to survive.

\Definition{Biological macromolecule}{A biological macromolecule is a polymer that occurs naturally in living organisms. These molecules are essential to the survival of life.}

\subsection{Carbohydrates}
\label{subsec:orgmac:carbohydrates}

\textbf{Carbohydrates} include the sugars and their polymers. One key characteristic of the carbohydrates is that they contain only the elements carbon, hydrogen and oxygen. In the carbohydrate monomers, every carbon except one has a hydroxyl group attached to it, and the remaining carbon atom is double bonded to an oxygen atom to form a carbonyl group. One of the most important monomers in the carbohydrates is \textbf{glucose} (figure \ref{fig:orgmac:glucose}). The glucose molecule can exist in an open-chain (acyclic) and ring (cyclic) form. 

\begin{figure}[h]
\begin{center}
\begin{pspicture}(-4,-3)(4,6)
%\psgrid[gridcolor=lightgray]
\rput(-4.5,1.5){(b)}
\rput(-4.5,5.5){(a)}
\rput(-3,4){\textbf{C$_{1}$}}
\psline(-3,3.8)(-3.6,3.2)
\rput(-3.7,3){\textbf{H}}
\psline(-3.05,4.2)(-3.6,4.8)
\psline(-2.95,4.2)(-3.5,4.8)
\rput(-3.6,5){\textbf{O}}
\psline(-2.8,4)(-2.2,4)
\rput(-2,4){\textbf{C$_{2}$}}
\psline(-2,4.2)(-2,4.8)
\rput(-2,5){\textbf{OH}}
\psline(-2,3.8)(-2,3.2)
\rput(-2,3){\textbf{H}}
\rput(1,0){
\psline(-2.8,4)(-2.2,4)
\rput(-2,4){\textbf{C$_{3}$}}
\psline(-2,4.2)(-2,4.8)
\rput(-2,5){\textbf{H}}
\psline(-2,3.8)(-2,3.2)
\rput(-2,3){\textbf{OH}}
}
\rput(2,0){
\psline(-2.8,4)(-2.2,4)
\rput(-2,4){\textbf{C$_{4}$}}
\psline(-2,4.2)(-2,4.8)
\rput(-2,5){\textbf{OH}}
\psline(-2,3.8)(-2,3.2)
\rput(-2,3){\textbf{H}}
}
\rput(3,0){
\psline(-2.8,4)(-2.2,4)
\rput(-2,4){\textbf{C$_{5}$}}
\psline(-2,4.2)(-2,4.8)
\rput(-2,5){\textbf{OH}}
\psline(-2,3.8)(-2,3.2)
\rput(-2,3){\textbf{H}}
}
\rput(4,0){
\psline(-2.8,4)(-2.2,4)
\rput(-2,4){\textbf{C$_{6}$}}
\psline(-2,4.2)(-2,4.8)
\rput(-2,5){\textbf{OH}}
\psline(-2,3.8)(-2,3.2)
\rput(-2,3){\textbf{H}}
}
\psline(2.2,4)(2.8,4)
\rput(3,4){\textbf{H}}
%................................... ring structure of sugar..........

\rput(1.5,0){
\rput(-3,0){\textbf{C$_{4}$}}
\psline(-3.1,0.2)(-3.6,0.8)
\rput(-3.8,1){\textbf{H}}
\psline(-3.1,-0.2)(-3.6,-0.8)
\rput(-3.8,-1){\textbf{OH}}
\psline(-2.9,0.2)(-2.4,1)
\rput(-2.2,1.2){\textbf{C$_{5}$}}
\psline(-2.2,1.4)(-2.2,1.9)
\rput(-2.2,2.1){\textbf{CH$_{2}$OH}}
\psline(-2.2,1)(-2.2,0.5)
\rput(-2.2,0.3){\textbf{H}}
\psline(-2,1.2)(-1.4,1.2)
\rput(-1.2,1.2){\textbf{O}}
\psline(-1,1)(-0.5,0.2)
\rput(-0.3,0){\textbf{C$_{1}$}}
\psline(-0.1,0.2)(0.4,0.8)
\rput(0.6,1){\textbf{H}}
\psline(-0.1,-0.2)(0.4,-0.8)
\rput(0.6,-1){\textbf{OH}}
\psline(-0.5,-0.2)(-1,-0.8)
\rput(-1.2,-1){\textbf{C$_{2}$}}
\psline(-1.2,-0.8)(-1.2,-0.3)
\psline(-1.2,-1.2)(-1.2,-1.7)
\rput(-1.2,-0.1){\textbf{H}}
\rput(-1.2,-1.9){\textbf{OH}}
\psline(-1.4,-1)(-2,-1)
\rput(-2.2,-1){\textbf{C$_{3}$}}
\psline(-2.2,-0.8)(-2.2,-0.3)
\rput(-2.2,-0.1){\textbf{OH}}
\psline(-2.2,-1.2)(-2.2,-1.7)
\rput(-2.2,-1.9){\textbf{H}}
\psline(-2.4,-0.8)(-2.9,-0.3)
}
\end{pspicture}
\caption{The open chain (a) and cyclic (b) structure of a glucose molecule}
\label{fig:orgmac:glucose}
\end{center}
\end{figure}

Glucose is produced during \textbf{photosynthesis}, which takes place in plants. During photosynthesis, sunlight (solar energy), water and carbon dioxide are involved in a chemical reaction that produces glucose and oxygen. This glucose is stored in various ways in the plant. 

The photosynthesis reaction is as follows:

\begin{center}
$\rm{6CO_{2} + 6H_{2}O + sunlight \rightarrow C_{6}H_{12}O_{6} + 6O_{2}}$\\
\end{center}

Glucose is an important source of \textbf{energy} for both the plant itself, and also for the other animals and organisms that may feed on it. Glucose plays a critical role in \textbf{cellular respiration}, which is a chemical reaction that occurs in the cells of all living organisms. During this reaction, glucose and oxygen react to produce carbon dioxide, water and Adenosine Triphosphate (ATP). ATP is a molecule that cells use for energy so that the body's cells can function normally. The purpose of \textit{eating} then, is to obtain glucose which the body can then convert into the ATP it needs to be able to survive.\\

The reaction for cellular respiration is as follows:

\begin{center}
$\rm{6C_{6}H_{12}O_{6} + 60_{2} \rightarrow 6CO_{2} + 6H_{2}O + ATP}$\\
\end{center}


We don't often eat glucose in its simple form. More often, we eat complex carbohydrates that our bodies have to break down into individual glucose molecules before they can be used in cellular respiration. These complex carbohydrates are polymers, which form through condensation polymerisation reactions (figure \ref{fig:orgmac:glucosepolym}). \textit{Starch} and \textit{cellulose} are two example of carbohydrates that are polymers composed of glucose monomers. 

\begin{figure}[h]
\begin{center}
\begin{pspicture}(-8,-8)(4,3)
%\psgrid[gridcolor=lightgray]
\rput(-8.5,2){(a)}
\rput(-8.5,-4){(b)}
\rput(-4,0){
\rput(-3,0){\textbf{C$_{4}$}}
\psline(-3.1,0.2)(-3.6,0.8)
\rput(-3.8,1){\textbf{H}}
\psline(-3.1,-0.2)(-3.6,-0.8)
\rput(-3.8,-1){\textbf{OH}}
\psline(-2.9,0.2)(-2.4,1)
\rput(-2.2,1.2){\textbf{C$_{5}$}}
\psline(-2.2,1.4)(-2.2,1.9)
\rput(-2.2,2.1){\textbf{CH$_{2}$OH}}
\psline(-2.2,1)(-2.2,0.5)
\rput(-2.2,0.3){\textbf{H}}
\psline(-2,1.2)(-1.4,1.2)
\rput(-1.2,1.2){\textbf{O}}
\psline(-1,1)(-0.5,0.2)
\rput(-0.3,0){\textbf{C}}
\psline(-0.1,0.2)(0.4,0.8)
\rput(0.6,1){\textbf{H}}
\psline(-0.1,-0.2)(0.4,-0.8)
\rput(0.6,-1){\textbf{OH}}
\psline(-0.5,-0.2)(-1,-0.8)
\rput(-1.2,-1){\textbf{C$_{2}$}}
\psline(-1.2,-0.8)(-1.2,-0.3)
\psline(-1.2,-1.2)(-1.2,-1.7)
\rput(-1.2,-0.1){\textbf{H}}
\rput(-1.2,-1.9){\textbf{OH}}
\psline(-1.4,-1)(-2,-1)
\rput(-2.2,-1){\textbf{C$_{3}$}}
\psline(-2.2,-0.8)(-2.2,-0.3)
\rput(-2.2,-0.1){\textbf{OH}}
\psline(-2.2,-1.2)(-2.2,-1.7)
\rput(-2.2,-1.9){\textbf{H}}
\psline(-2.4,-0.8)(-2.9,-0.3)
}
\rput(-2,0){\textbf{+}}
\rput(3,0){
\rput(-3,0){\textbf{C$_{4}$}}
\psline(-3.1,0.2)(-3.6,0.8)
\rput(-3.8,1){\textbf{H}}
\psline(-3.1,-0.2)(-3.6,-0.8)
\rput(-3.8,-1){\textbf{OH}}
\psline(-2.9,0.2)(-2.4,1)
\rput(-2.2,1.2){\textbf{C$_{5}$}}
\psline(-2.2,1.4)(-2.2,1.9)
\rput(-2.2,2.1){\textbf{CH$_{2}$OH}}
\psline(-2.2,1)(-2.2,0.5)
\rput(-2.2,0.3){\textbf{H}}
\psline(-2,1.2)(-1.4,1.2)
\rput(-1.2,1.2){\textbf{O}}
\psline(-1,1)(-0.5,0.2)
\rput(-0.3,0){\textbf{C}}
\psline(-0.1,0.2)(0.4,0.8)
\rput(0.6,1){\textbf{H}}
\psline(-0.1,-0.2)(0.4,-0.8)
\rput(0.6,-1){\textbf{OH}}
\psline(-0.5,-0.2)(-1,-0.8)
\rput(-1.2,-1){\textbf{C$_{2}$}}
\psline(-1.2,-0.8)(-1.2,-0.3)
\psline(-1.2,-1.2)(-1.2,-1.7)
\rput(-1.2,-0.1){\textbf{H}}
\rput(-1.2,-1.9){\textbf{OH}}
\psline(-1.4,-1)(-2,-1)
\rput(-2.2,-1){\textbf{C$_{3}$}}
\psline(-2.2,-0.8)(-2.2,-0.3)
\rput(-2.2,-0.1){\textbf{OH}}
\psline(-2.2,-1.2)(-2.2,-1.7)
\rput(-2.2,-1.9){\textbf{H}}
\psline(-2.4,-0.8)(-2.9,-0.3)
}

\rput(-3,-6){
\rput(-3,0){\textbf{C$_{4}$}}
\psline(-3.1,0.2)(-3.6,0.8)
\rput(-3.8,1){\textbf{H}}
\psline(-3.1,-0.2)(-3.6,-0.8)
\rput(-3.8,-1){\textbf{OH}}
\psline(-2.9,0.2)(-2.4,1)
\rput(-2.2,1.2){\textbf{C$_{5}$}}
\psline(-2.2,1.4)(-2.2,1.9)
\rput(-2.2,2.1){\textbf{CH$_{2}$OH}}
\psline(-2.2,1)(-2.2,0.5)
\rput(-2.2,0.3){\textbf{H}}
\psline(-2,1.2)(-1.4,1.2)
\rput(-1.2,1.2){\textbf{O}}
\psline(-1,1)(-0.5,0.2)
\rput(-0.3,0){\textbf{C}
\psline(-0.1,-0.3)(-0.1,-0.7)
\rput(-0.1,-1){\textbf{H}}
}

\psline(-0.5,-0.2)(-1,-0.8)
\rput(-1.2,-1){\textbf{C$_{2}$}}
\psline(-1.2,-0.8)(-1.2,-0.3)
\psline(-1.2,-1.2)(-1.2,-1.7)
\rput(-1.2,-0.1){\textbf{H}}
\rput(-1.2,-1.9){\textbf{OH}}
\psline(-1.4,-1)(-2,-1)
\rput(-2.2,-1){\textbf{C$_{3}$}}
\rput(-2.2,-1.9){\textbf{H}}
\psline(-2.4,-0.8)(-2.9,-0.3)
}
\rput(2,-6){
\rput(-3,0){\textbf{C$_{4}$}}
\psline(-2.9,0.2)(-2.4,1)
\rput(-2.2,1.2){\textbf{C$_{5}$}}
\psline(-2.2,1.4)(-2.2,1.9)
\rput(-2.2,2.1){\textbf{CH$_{2}$OH}}
\psline(-2.2,1)(-2.2,0.5)
\rput(-2.2,0.3){\textbf{H}}
\psline(-2,1.2)(-1.4,1.2)
\rput(-1.2,1.2){\textbf{O}}
\psline(-1,1)(-0.5,0.2)
\rput(-0.3,0){\textbf{C}}
\psline(-0.1,0.2)(0.4,0.8)
\rput(0.6,1){\textbf{H}}
\psline(-0.1,-0.2)(0.4,-0.8)
\rput(0.6,-1){\textbf{OH}}
\psline(-0.5,-0.2)(-1,-0.8)
\rput(-1.2,-1){\textbf{C$_{2}$}}
\psline(-1.2,-0.8)(-1.2,-0.3)
\psline(-1.2,-1.2)(-1.2,-1.7)
\rput(-1.2,-0.1){\textbf{H}}
\rput(-1.2,-1.9){\textbf{OH}}
\psline(-1.4,-1)(-2,-1)
\rput(-2.2,-1){\textbf{C$_{3}$}}
\psline(-2.2,-0.8)(-2.2,-0.3)
\rput(-2.2,-0.1){\textbf{OH}}
\psline(-2.2,-1.2)(-2.2,-1.7)
\rput(-2.2,-1.9){\textbf{H}}
\psline(-2.4,-0.8)(-2.9,-0.3)
}
\psline(-1.1,-5.7)(-1.1,-5.3)
\rput(-1.1,-5){\textbf{H}}
\psline(-1.4,-6)(-1.9,-6)
\rput(-2.2,-6){\textbf{O}}
\psline(-2.5,-6)(-2.9,-6)
\rput(3,-6){\textbf{+}}
\rput(3.7,-6){\textbf{H$_{2}$O}}
\end{pspicture}
\end{center}
\caption{Two glucose monomers (a) undergo a condensation reaction  to produce a section of a carbohydrate polymer (b). One molecule of water is produced for every two monomers that react.}
\label{fig:orgmac:glucosepolym}
\end{figure}

\begin{itemize}
\item{\textbf{Starch}

Starch is used by plants to store excess glucose, and consists of long chains of glucose monomers. Potatoes are made up almost entirely of starch. This is why potatoes are such a good source of energy. Animals are also able to store glucose, but in this case it is stored as a compound called \textbf{glycogen}, rather than as starch.}

\item{\textbf{Cellulose}

Cellulose is also made up of chains of glucose molecules, but the bonding between the polymers is slightly different from that in starch. Cellulose is found in the cell walls of plants and is used by plants as a building material.}

\begin{IFact}
{
It is very difficult for animals to digest the cellulose in plants that they may have been feeding on. However, fungi and some protozoa are able to break down cellulose. Many animals, including termites and cows, use these organisms to break cellulose down into glucose, which they can then use more easily.
}
\end{IFact}
\end{itemize}

\subsection{Proteins}

Proteins are an incredibly important part of any cell, and they carry out a number of functions such as support, storage and transport within the body. The monomers of proteins are called \textbf{amino acids}. An amino acid is an organic molecule that contains a carboxyl and an amino group, as well as a carbon side chain. The carbon side chain varies from one amino acid to the next, and is sometimes simply represented by the letter 'R' in a molecule's structural formula. Figure \ref{fig:orgmac:aminoacids} shows some examples of different amino acids.

\begin{figure}[!h]
\begin{center}
\begin{pspicture}(-4,-8)(6,1.5)
%\psgrid[gridcolor=lightgray]
\rput(-1,0){
\rput(-3.3,0){\textbf{H$_{2}$N}}
\psellipse(-3.3,0)(0.6,0.6)
\psline(-2.8,0)(-2.2,0)
\rput(-2,0){\textbf{C}}
\psline(-2,0.2)(-2,0.8)
\psline(-2,-0.2)(-2,-0.8)
\rput(-2,1){\textbf{H}}
\rput(-2,-1){\textbf{H}}
\psline(-1.2,0)(-1.8,0)
\rput(-1,0){\textbf{C}}
\psline(-0.75,0.2)(-0.25,0.7)
\psline(-0.85,0.2)(-0.35,0.7)
\rput(-0.1,1){\textbf{O}}
\psline(-0.8,-0.2)(-0.3,-0.7)
\rput(-0.1,-1){\textbf{OH}}
\psellipse[linestyle=dashed](-0.3,0)(1.2,1.2)
\rput(0.2,1.5){Carboxyl group}
\rput(-3.5,-0.9){Amino group}
\rput(-2,-2.5){glycine}
}

\rput(1,0){
\rput(5,0){
\rput(-3.3,0){\textbf{H$_{2}$N}}
\psline(-2.8,0)(-2.2,0)
\rput(-2,0){\textbf{C}}
\psline(-2,0.2)(-2,0.8)
\psline(-2,-0.2)(-2,-0.8)
\rput(-2,1){\textbf{H}}
\rput(-2,-1){\textbf{CH$_{3}$}}
\psline(-1.2,0)(-1.8,0)
\rput(-1,0){\textbf{C}}
\psline(-0.75,0.2)(-0.25,0.7)
\psline(-0.85,0.2)(-0.35,0.7)
\rput(-0.1,1){\textbf{O}}
\psline(-0.8,-0.2)(-0.3,-0.7)
\rput(-0.1,-1){\textbf{OH}}
\psellipse[linestyle=dashed](-0.3,0)(1.2,1.2)
\psellipse(-3.3,0)(0.6,0.6)
\psframe(-2.4,-1.5)(-1.6,-0.5)
\rput(-2,-2.5){alanine}
\rput(-3.7,-1.5){Side chain ('R')}
}
}

\rput(3,-5){
\rput(-3.3,0){\textbf{H$_{2}$N}}
\psline(-2.8,0)(-2.2,0)
\rput(-2,0){\textbf{C}}
\psline(-2,0.2)(-2,0.8)
\psline(-2,-0.2)(-2,-0.8)
\rput(-2,1){\textbf{H}}
\rput(-2,-1){\textbf{CH$_{2}$}}
\psline(-1.2,0)(-1.8,0)
\rput(-1,0){\textbf{C}}
\psline(-0.75,0.2)(-0.25,0.7)
\psline(-0.85,0.2)(-0.35,0.7)
\rput(-0.1,1){\textbf{O}}
\psline(-0.8,-0.2)(-0.3,-0.7)
\rput(-0.1,-1){\textbf{OH}}
\rput(-2,-3){serine}
\psline(-2,-1.2)(-2,-1.8)
\rput(-2,-2){\textbf{OH}}
\psellipse[linestyle=dashed](-0.3,0)(1.2,1.2)
\psellipse(-3.3,0)(0.6,0.6)
\psframe(-2.4,-2.5)(-1.6,-0.5)

}
\end{pspicture}
\end{center}
\caption{Three amino acids: glycine, alanine and serine}
\label{fig:orgmac:aminoacids}
\end{figure}


Although each of these amino acids has the same basic structure, their side chains ('R' groups) are different. In the amino acid glycine, the side chain consists only of a hydrogen atom, while alanine has a \textit{methyl} side chain. The 'R' group in serine is CH$_{2}$ - OH. Amongst other things, the side chains affect whether the amino acid is \textit{hydrophilic} (attracted to water) or \textit{hydrophobic} (repelled by water). If the side chain is \textit{polar}, then the amino acid is hydrophilic, but if the side chain is \textit{non-polar} then the amino acid is hydrophobic. Glycine and alanine both have non-polar side chains, while serine has a polar side chain.

\Extension{Charged regions in an amino acid\\}{
In an amino acid, the amino group acts as a base because the nitrogen atom has a pair of unpaired electrons which it can use to bond to a hydrogen ion. The amino group therefore attracts the hydrogen ion from the carboxyl group, and ends up having a charge of +1. The carboxyl group from which the hydrogen ion has been taken then has a charge of -1. The amino acid glycine can therefore also be represented as shown in the figure below.

\begin{center}
\begin{pspicture}(-4,-3)(0,2)
%\psgrid[gridcolor=lightgray]
\rput(-3.3,0){\textbf{H$_{3}$N$^{+}$}}
\psline(-2.8,0)(-2.2,0)
\rput(-2,0){\textbf{C}}
\psline(-2,0.2)(-2,0.8)
\psline(-2,-0.2)(-2,-0.8)
\rput(-2,1){\textbf{H}}
\rput(-2,-1){\textbf{H}}
\psline(-1.2,0)(-1.8,0)
\rput(-1,0){\textbf{C}}
\psline(-0.75,0.2)(-0.25,0.7)
\psline(-0.85,0.2)(-0.35,0.7)
\rput(-0.1,1){\textbf{O}}
\psline(-0.8,-0.2)(-0.3,-0.7)
\rput(-0.1,-1){\textbf{O$^{-}$}}
\rput(-2,-2){glycine}
\end{pspicture}
\end{center}
}

When two amino acid monomers are close together, they may be joined to each other by \textbf{peptide bonds} (figure \ref{fig:orgmac:peptide}) to form a \textbf{polypeptide} chain. . The reaction is a condensation reaction. Polypeptides can vary in length from a few amino acids to a thousand or more. The polpeptide chains are then joined to each other in different ways to form a \textbf{protein}. It is the sequence of the amino acids in the polymer that gives a protein its particular properties.\\

The sequence of the amino acids in the chain is known as the protein's \textbf{primary structure}. As the chain grows in size, it begins to twist, curl and fold upon itself. The different parts of the polypeptide are held together by hydrogen bonds, which form between hydrogen atoms in one part of the chain and oxygen or nitrogen atoms in another part of the chain. This is known as the \textbf{secondary structure} of the protein. Sometimes, in this coiled helical structure, bonds may form between the side chains (R groups) of the amino acids. This results in even more irregular contortions of the protein. This is called the \textbf{tertiary structure} of the protein.\\

\begin{figure}[H]
\begin{center}
\begin{pspicture}(-4,-6)(4,2)
%\psgrid[gridcolor=lightgray]

\rput(-1,0){
\rput(-3.3,0){\textbf{H$_{2}$N}}
\psline(-2.8,0)(-2.2,0)
\rput(-2,0){\textbf{C}}
\psline(-2,0.2)(-2,0.8)
\psline(-2,-0.2)(-2,-0.8)
\rput(-2,1){\textbf{H}}
\rput(-2,-1){\textbf{H}}
\psline(-1.2,0)(-1.8,0)
\rput(-1,0){\textbf{C}}
\psline(-0.75,0.2)(-0.25,0.7)
\psline(-0.85,0.2)(-0.35,0.7)
\rput(-0.1,1){\textbf{O}}
\psline(-0.8,-0.2)(-0.3,-0.7)
\rput(-0.1,-1){\textbf{OH}}
}

\rput(1,0){
\rput(5,0){
\rput(-3.3,0){\textbf{H$_{2}$N}}
\psline(-2.8,0)(-2.2,0)
\rput(-2,0){\textbf{C}}
\psline(-2,0.2)(-2,0.8)
\psline(-2,-0.2)(-2,-0.8)
\rput(-2,1){\textbf{H}}
\rput(-2,-1){\textbf{CH$_{3}$}}
\psline(-1.2,0)(-1.8,0)
\rput(-1,0){\textbf{C}}
\psline(-0.75,0.2)(-0.25,0.7)
\psline(-0.85,0.2)(-0.35,0.7)
\rput(-0.1,1){\textbf{O}}
\psline(-0.8,-0.2)(-0.3,-0.7)
\rput(-0.1,-1){\textbf{OH}}
}
}

\rput(0,-4){
\rput(-3.3,0){\textbf{H$_{2}$N}}
\psline(-2.8,0)(-2.2,0)
\rput(-2,0){\textbf{C}}
\psline(-2,0.2)(-2,0.8)
\psline(-2,-0.2)(-2,-0.8)
\rput(-2,1){\textbf{H}}
\rput(-2,-1){\textbf{H}}
\psline(-1.2,0)(-1.8,0)
\rput(-1,0){\textbf{C}}
\psline(-0.8,0)(-0.2,0)
\psline(-0.95,0.2)(-0.95,0.8)
\psline(-1.05,0.2)(-1.05,0.8)
\rput(-1,1){\textbf{O}}

\rput(3,0){
\rput(-3,0){\textbf{N}}
\psline(-3,-0.2)(-3,-0.8)
\rput(-3,-1){\textbf{H}}
\psline(-2.8,0)(-2.2,0)
\rput(-2,0){\textbf{C}}
\psline(-2,0.2)(-2,0.8)
\psline(-2,-0.2)(-2,-0.8)
\rput(-2,1){\textbf{H}}
\rput(-2,-1){\textbf{CH$_{3}$}}
\psline(-1.2,0)(-1.8,0)
\rput(-1,0){\textbf{C}}
\psline(-0.75,0.2)(-0.25,0.7)
\psline(-0.85,0.2)(-0.35,0.7)
\rput(-0.1,1){\textbf{O}}
\psline(-0.8,-0.2)(-0.3,-0.7)
\rput(-0.1,-1){\textbf{OH}}
}
}
\rput(0.5,0){\textbf{+}}
\rput(3.3,-4){\textbf{+}}
\rput(4,-4){\textbf{H$_{2}$O}}
\rput(-4,1){(a)}
\rput(-4,-3){(b)}
\psellipse[linestyle=dashed](-0.5,-4)(0.8,1.6)
\rput(-0.5,-2){Peptide bond}
\end{pspicture}
\end{center}
\caption{Two amino acids (glycine and alanine) combine to form part of a polypeptide chain. The amino acids are joined by a peptide bond between a carbon atom of one amino acid and a nitrogen atom of the other amino acid.}
\label{fig:orgmac:peptide}
\end{figure}


\begin{IFact}{
There are twenty different amino acids that exist in nature. All cells, both plant and animal, build their proteins from only twenty amino acids. At first, this seems like a very small number, especially considering the huge number of different proteins that exist. However, if you consider that most proteins are made up of polypeptide chains that contain at least 100 amino acids, you will start to realise the endless possible combinations of amino acids that are available. 
}
\end{IFact} 

\subsubsection*{The functions of proteins}

Proteins have a number of functions in living organisms. 

\begin{itemize}
\item{\textit{Structural proteins} such as collagen in animal connective tissue and keratin in hair, horns and feather quills, all provide support.}
\item{\textit{Storage proteins} such as albumin in egg white provide a source of energy. Plants store proteins in their seeds to provide energy for the new growing plant.}
\item{\textit{Transport proteins} transport other substances in the body. Haemoglobin in the blood for example, is a protein that contains iron. Haemoglobin has an affinity (attraction) for oxygen and so this is how oxygen is transported around the body in the blood.}
\item{\textit{Hormonal proteins} coordinate the body's activities. Insulin for example, is a hormonal protein that controls the sugar levels in the blood.}
\item{\textit{Enzymes} are chemical catalysts and speed up chemical reactions. Digestive enzymes such as amylase in your saliva, help to break down polymers in food. Enzymes play an important role in all cellular reactions such as respiration, photosynthesis and many others.} 
\end{itemize}

\Activity{Research Project}{Macromolecules in our daily diet\\}{

\begin{enumerate}
\item{
In order to keep our bodies healthy, it is important that we eat a balanced diet with the right amounts of carbohydrates, proteins and fats. Fats are an important source of energy, they provide insulation for the body, and they also provide a protective layer around many vital organs. Our bodies also need certain essential vitamins and minerals. Most food packaging has a label that provides this information.\\

Choose a number of different food items that you eat. Look at the food label for each, and then complete the following table:\\

\begin{tabular}{|p{2.3cm}|p{2.3cm}|p{2.3cm}|p{2.3cm}|}\hline
\textbf{Food} & \textbf{Carbohydrates (\%)} & \textbf{Proteins (\%)} & \textbf{Fats (\%)} \\\hline
 & & & \\\hline
 & & & \\\hline
 & & & \\\hline
 & & & \\\hline
 & & & \\\hline
\end{tabular}

	\begin{enumerate}
	\item{Which food type contains the largest proportion of protein?}
	\item{Which food type contains the largest proportion of carbohydrates?}
	\item{Which of the food types you have listed would you consider to be the 'healthiest'? Give a reason for your answer.}
	\end{enumerate}
}

\item{In an effort to lose weight, many people choose to \textit{diet}. There are many diets on offer, each of which is based on particular theories about how to lose weight most effectively. Look at the list of diets below:

\begin{itemize}
\item{Low fat diet}
\item{Atkin's diet}
\item{Weight Watchers}
\end{itemize}

For each of these diets, answer the following questions:

	\begin{enumerate}
	\item{What theory of weight loss does each type of diet propose?}
	\item{What are the \textit{benefits} of the diet?}
	\item{What are the potential \textit{problems} with the diet?}
	\end{enumerate}

}
\end{enumerate}
}

\Exercise{Carbohydrates and proteins\\}{

\begin{enumerate}
\item{Give the structural formula for each of the following:}
	\begin{enumerate}
	\item{A polymer chain, consisting of three glucose molecules.}
	\item{A polypeptide chain, consisting of two molecules of alanine and one molecule of serine.}
	\end{enumerate}

\item{Write balanced equations to show the polymerisation reactions that produce the polymers described above.}

\item{The following polypeptide is the end product of a polymerisation reaction:}

\begin{center}
\begin{pspicture}(-4,-2.3)(4,2)
%\psgrid[gridcolor=lightgray]

\rput(-2,0){
\rput(-3.3,0){\textbf{H$_{2}$N}}
\psline(-2.8,0)(-2.2,0)
\rput(-2,0){\textbf{C}}
\psline(-2,0.2)(-2,0.8)
\psline(-2,-0.2)(-2,-0.8)
\rput(-2,1){\textbf{H}}
\rput(-2,-1){\textbf{CH$_{3}$}}
\psline(-1.2,0)(-1.8,0)
\rput(-1,0){\textbf{C}}
\psline(-0.8,0)(-0.2,0)
\psline(-0.95,0.2)(-0.95,0.8)
\psline(-1.05,0.2)(-1.05,0.8)
\rput(-1,1){\textbf{O}}

\rput(3,0){
\rput(-3,0){\textbf{N}}
\psline(-3,-0.3)(-3,-0.7)
\rput(-3,-1){\textbf{H}}
\psline(-2.8,0)(-2.2,0)
\rput(-2,0){\textbf{C}}
\psline(-2,0.2)(-2,0.8)
\psline(-2,-0.2)(-2,-0.8)
\rput(-2,1){\textbf{H}}
\rput(-2,-1){\textbf{CH$_{2}$}}
\psline(-2,-1.3)(-2,-1.7)
\rput(-2,-2){\textbf{SH}}
\psline(-1.2,0)(-1.8,0)
\rput(-1,0){\textbf{C}}
\psline(-0.8,0)(-0.2,0)
\psline(-0.95,0.2)(-0.95,0.8)
\psline(-1.05,0.2)(-1.05,0.8)
\rput(-1,1){\textbf{O}}
}

\rput(6,0){
\rput(-3,0){\textbf{N}}
\psline(-3,-0.2)(-3,-0.8)
\rput(-3,-1){\textbf{H}}
\psline(-2.8,0)(-2.2,0)
\rput(-2,0){\textbf{C}}
\psline(-2,0.2)(-2,0.8)
\psline(-2,-0.2)(-2,-0.8)
\rput(-2,1){\textbf{H}}
\rput(-2,-1){\textbf{H}}
\psline(-1.2,0)(-1.8,0)
\rput(-1,0){\textbf{C}}
\psline(-0.75,0.2)(-0.25,0.7)
\psline(-0.85,0.2)(-0.35,0.7)
\rput(-0.1,1){\textbf{O}}
\psline(-0.8,-0.2)(-0.3,-0.7)
\rput(-0.1,-1){\textbf{OH}}
}
}


\end{pspicture}
\end{center}


	\begin{enumerate}
	\item{Give the structural formula of the monomers that make up the polypeptide.}
	\item{On the structural formula of the first monomer, label the amino group and the carboxyl group.}
	\item{What is the chemical formula for the carbon side chain in the second monomer?}
	\item{Name the bond that forms between the monomers of the polypeptide.}
	\end{enumerate}
\end{enumerate}

% Automatically inserted shortcodes - number to insert 3
\par \practiceinfo
\par \begin{tabular}[h]{cccccc}
% Question 1
(1.)	01pi	&
% Question 2
(2.)	01pj	&
% Question 3
(3.)	01pk	&
\end{tabular}
% Automatically inserted shortcodes - number inserted 3
}

\subsection{Nucleic Acids}

You will remember that we mentioned earlier that each protein is different because of its unique sequence of amino acids. But what controls how the amino acids arrange themselves to form the specific proteins that are needed by an organism? This task is for the \textbf{gene}. A gene contains DNA (deoxyribonucleic acid) which is a polymer that belongs to a class of compounds called the \textbf{nucleic acids}. DNA is the genetic material that organisms inherit from their parents. It is DNA that provides the genetic coding that is needed to form the specific proteins that an organism needs. Another nucleic acid is RNA (ribonucleic acid). The diagram in figure \ref{fig:orgmac:nucleotide} shows an RNA molecule.\\

The DNA polymer is made up of monomers called \textbf{nucleotides}. Each nucleotide has three parts: a sugar, a phosphate and a nitrogenous base. DNA is a double-stranded helix (a helix is basically a coil). Or you can think of it as two RNA molecules bonded together. \\

\begin{figure}[!h]
\begin{center}
\begin{pspicture}(-6,-6)(6,3)
%\psgrid[gridcolor=lightgray]
\psset{unit=0.7}
\psellipse(-2.2,0)(0.5,0.5)
\psline(-1.8,-0.2)(-1,-1)
\psline(-1,-1)(0,-0.2)
\psline(0,-0.2)(1,-1)
\psline(1,-1)(0.6,-2)
\psline(-1,-1)(-0.6,-2)
\psline(-0.6,-2)(0.6,-2)
\psline(1,-1)(2,-1)
\psframe(2,-1.5)(4,-0.5)
\psline(-0.6,-2)(-1.9,-2.7)
\psline(-3,0)(-3.2,0)
\psline(-3,-2.5)(-3.2,-2.5)
\psline(-3.2,0)(-3.2,-2.5)
\rput(-4.5,-1.5){nucleotide}

\psline(4.5,3.5)(4.7,3.5)
\psline(4.5,-9.5)(4.7,-9.5)
\psline(4.7,3.5)(4.7,-9.5)
\rput(7.5,-1.5){DNA polymer made up of}
\rput(6.8,-1.8){four nucleotides}


\rput(0,-3){
\psellipse(-2.2,0)(0.5,0.5)
\psline(-1.8,-0.2)(-1,-1)
\psline(-1,-1)(0,-0.2)
\psline(0,-0.2)(1,-1)
\psline(1,-1)(0.6,-2)
\psline(-1,-1)(-0.6,-2)
\psline(-0.6,-2)(0.6,-2)
\psline(1,-1)(2,-1)
\psframe(2,-1.5)(4,-0.5)
\psline(-0.6,-2)(-1.9,-2.7)
}
\rput(0,-6){
\psellipse(-2.2,0)(0.5,0.5)
\psline(-1.8,-0.2)(-1,-1)
\psline(-1,-1)(0,-0.2)
\psline(0,-0.2)(1,-1)
\psline(1,-1)(0.6,-2)
\psline(-1,-1)(-0.6,-2)
\psline(-0.6,-2)(0.6,-2)
\psline(1,-1)(2,-1)
\psframe(2,-1.5)(4,-0.5)
\psline(-0.6,-2)(-1.9,-2.7)
}
\rput(0,3){
\psellipse(-2.2,0)(0.5,0.5)
\psline(-1.8,-0.2)(-1,-1)
\psline(-1,-1)(0,-0.2)
\psline(0,-0.2)(1,-1)
\psline(1,-1)(0.6,-2)
\psline(-1,-1)(-0.6,-2)
\psline(-0.6,-2)(0.6,-2)
\psline(1,-1)(2,-1)
\psframe(2,-1.5)(4,-0.5)
\psline(-0.6,-2)(-1.9,-2.7)
}
\rput(-2,3.9){phosphate}
\rput(0,3.9){sugar}
\rput(3.5,3.9){nitrogenous base}

\end{pspicture}
\end{center}
\caption{Nucleotide monomers make up the RNA polymer}
\label{fig:orgmac:nucleotide}
\end{figure}


There are five different nitrogenous bases: adenine (A), guanine (G), cytosine (C), thymine (T) and uracil (U). It is the sequence of the nitrogenous bases in a DNA polymer that will determine the genetic code for that organism. Three consecutive nitrogenous bases provide the coding for one amino acid. So, for example, if the nitrogenous bases on three nucleotides are \textit{uracil}, \textit{cytosine} and \textit{uracil} (in that order), one \textbf{serine} amino acid will become part of the polypeptide chain. The polypeptide chain is built up in this way until it is long enough (and with the right amino acid sequence) to be a protein. Since proteins control much of what happens in living organisms, it is easy to see how important nucleic acids are as the starting point of this process.\\

\begin{IFact}{
A single defect in even one nucleotide, can be devastating to an organism. One example of this is a disease called \textbf{sickle cell anaemia}. Because of one wrong nucletide in the genetic code, the body produces a protein called \textbf{sickle haemoglobin}. Haemoglobin is the protein in red blood cells that helps to transport oxygen around the body. When sickle haemoglobin is produced, the red blood cells change shape. This process damages the red blood cell membrane, and can cause the cells to become stuck in blood vessels. This then means that the red blood cells, whcih are carrying oxygen, can't get to the tissues where they are needed. This can cause serious organ damage. Individuals who have sickle cell anaemia generally have a lower life expectancy. 
}
\end{IFact}

Table \ref{tab:amino acids} shows some other examples of genetic coding for different amino acids.\\

\begin{table}[h]
\begin{center}
\caption{Nitrogenouse base sequences and the corresponding amino acid}
\label{tab:amino acids}
\begin{tabular}{|c|c|}\hline
\textbf{Nitrogenous base sequence} & \textbf{Amino acid} \\\hline
UUU & Phenylalanine \\\hline
CUU & Leucine \\\hline
UCU & Serine \\\hline
UAU & Tyrosine \\\hline
UGU & Cysteine \\\hline
GUU & Valine \\\hline
GCU & Alanine \\\hline
GGU & Glycine \\\hline
\end{tabular}
\end{center}
\end{table}

\Exercise{Nucleic acids\\}{
\begin{enumerate}
\item{For each of the following, say whether the statement is \textbf{true} or \textbf{false}. If the statement is \textit{false}, give a reason for your answer.}
	\begin{enumerate}
	\item{Deoxyribonucleic acid (DNA) is an example of a \textit{polymer} and a nucleotide is an example of a \textit{monomer}.}
	\item{Thymine and uracil are examples of nucleotides.}
	\item{A person's DNA will determine what proteins their body will produce, and therefore what characteristics they will have. }
	\item{An amino acid is a protein monomer.}
	\item{A polypeptide that consists of five amino acids, will also contain five nucleotides.}
	\end{enumerate}

\item{For each of the following sequences of nitrogenous bases, write down the amino acid/s that will be part of the polypeptide chain.}
	\begin{enumerate}
	\item{UUU}
	\item{UCUUUU}
	\item{GGUUAUGUUGGU}
	\end{enumerate}

\item{A polypeptide chain consists of three amino acids. The sequence of nitrogenous bases in the nucleotides of the DNA is GCUGGUGCU. Give the structural formula of the polypeptide.}
\end{enumerate}

% Automatically inserted shortcodes - number to insert 3
\par \practiceinfo
\par \begin{tabular}[h]{cccccc}
% Question 1
(1.)	01pm	&
% Question 2
(2.)	01pn	&
% Question 3
(3.)	01pp	&
\end{tabular}
% Automatically inserted shortcodes - number inserted 3
}

\summary{aaa}

\begin{itemize}
\item{A \textbf{polymer} is a macromolecule that is made up of many repeating structural units called \textbf{monomers} which are joined by covalent bonds.}
\item{Polymers that contain carbon atoms in the main chain are called \textbf{organic polymers}.}
\item{Organic polymers can be divided into \textbf{natural organic polymers} (e.g. natural rubber) or \textbf{synthetic organic polymers} (e.g. polystyrene).}
\item{The polymer \textbf{polyethene} for example, is made up of many ethene monomers that have been joined into a polymer chain.}
\item{Polymers form through a process called \textbf{polymerisation}.}
\item{Two examples of polymerisation reactions are \textbf{addition} and \textbf{condensation} reactions.}
\item{An \textbf{addition reaction} occurs when unsaturated monomers (e.g. alkenes) are added to each other one by one. The breaking of a double bond between carbon atoms in the monomer, means that a bond can form with the next monomer. The polymer \textbf{polyethene} is formed through an addition reaction.}
\item{In a \textbf{condensation reaction}, a molecule of water is released as a product of the reaction. The water molecule is made up of atoms that have been lost from each of the monomers. Polyesters and nylon are polymers that are produced through a condensation reaction.}
\item{The \textbf{chemical properties} of polymers (e.g. tensile strength and melting point) are determined by the types of atoms in the polymer, and by the strength of the bonds between adjacent polymer chains. The stronger the bonds, the greater the strength of the polymer, and the higher its melting point.}
\item{One group of synthetic organic polymers, are the \textbf{plastics}.}
\item{\textbf{Polystyrene} is a plastic that is made up of styrene monomers. Polystyrene is used a lot in packaging.}
\item{\textbf{Polyvinyl chloride} (PVC) consists of vinyl chloride monomers. PVC is used to make pipes and flooring.}
\item{\textbf{Polyethene}, or \textbf{polyethylene}, is made from ethene monomers. Polyethene is used to make film wrapping, plastic bags, electrical insulation and bottles.}
\item{\textbf{Polytetrafluoroethylene} is used in non-stick frying pans and electrical insulation.}
\item{A \textbf{thermoplastic} can be heated and melted to a liquid. It freezes to a brittle, glassy state when cooled very quickly. Examples of thermoplastics are polyethene and PVC.}
\item{A \textbf{thermoset} plastic cannot be melted or re-shaped once formed. Examples of thermoset plastics are vulcanised rubber and melanine.}
\item{It is not easy to \textbf{recycle} all plastics, and so they create environmental problems.}
\item{Some of these \textbf{environmental problems} include issues of waste disposal, air pollution and recycling.}
\item{A \textbf{biological macromolecule} is a polymer that occurs naturally in living organisms.}
\item{Examples of biological macromolecules include \textbf{carbohydrates} and \textbf{proteins}, both of which are essential for life to survive.}
\item{Carbohydrates include the \textbf{sugars} and their polymers, and are an important source of \textbf{energy} in living organisms.}
\item{\textbf{Glucose} is a carbohydrate monomer. Glucose is the molecule that is needed for \textbf{cellular respiration}.}
\item{The glucose monomer is also a building block for carbohydrate polymers such as \textbf{starch}, \textbf{glycogen} and \textbf{cellulose}.}
\item{\textbf{Proteins} have a number of important functions. These include their roles in structures, transport, storage, hormonal proteins and enzymes.}
\item{A protein consists of monomers called \textbf{amino acids}, which are joined by \textbf{peptide bonds}.}
\item{A protein has a \textbf{primary}, \textbf{secondary} and \textbf{tertiary} structure.}
\item{An amino acid is an organic molecule, made up of a \textbf{carboxyl} and an \textbf{amino} group, as well as a carbon \textbf{side chain} of varying lengths.} 
\item{It is the \textbf{sequence} of amino acids that determines the nature of the protein.}
\item{It is the \textbf{DNA} of an organism that determines the order in which amino acids combine to make a protein.}
\item{DNA is a \textbf{nucleic acid}. DNA is a polymer, and is made up of monomers called \textbf{nucleotides}.}
\item{Each nucleotide consists of a \textbf{sugar}, a \textbf{phosphate} and a \textbf{nitrogenous base}. It is the sequence of the nitrogenous bases that provides the 'code' for the arrangement of the amino acids in a protein. }
\end{itemize}


\begin{eocexercises}{}
\begin{enumerate}

\item{Give one word for each of the following descriptions:}
	\begin{enumerate}
	\item{A chain of monomers joined by covalent bonds.}
	\item{A polymerisation reaction that produces a molecule of water for every two monomers that bond.}
	\item{The bond that forms between an alcohol and a carboxylic acid monomer during a polymerisation reaction.}
	\item{The name given to a protein monomer.}
	\item{A six-carbon sugar monomer.}
	\item{The monomer of DNA, which determines the sequence of amino acids that will make up a protein.}
	\end{enumerate}

	\item{A polymer is made up of monomers, each of which has the formula CH$_{2}$=CHCN. The formula of the polymer is:}
		\begin{enumerate}
		\item{-(CH$_{2}$=CHCN)$_{n}$-}	
		\item{-(CH$_{2}$-CHCN)$_{n}$-}
		\item{-(CH-CHCN)$_{n}$-}
		\item{-(CH$_{3}$-CHCN)$_{n}$-}
		\end{enumerate}

	\item{A polymer has the formula -[CO(CH$_{2}$)$_{4}$CO-NH(CH$_{2}$)6NH]$_{n}$-. Which of the following statements is \textbf{true}?}
		\begin{enumerate}
		\item{The polymer is the product of an addition reaction.}
		\item{The polymer is a polyester.}
		\item{The polymer contains an amide linkage.}
		\item{The polymer contains an ester linkage.}
		\end{enumerate}
	
	\item{Glucose:}
		\begin{enumerate}
		\item{is a monomer that is produced during cellular respiration}
		\item{is a sugar polymer}
		\item{is the monomer of starch}
		\item{is a polymer produced during photosynthesis}
		\end{enumerate}

\item{The following monomers are involved in a polymerisation reaction:

\begin{center}
\begin{pspicture}(-2,-1.8)(2,1.5)
%\psgrid[gridcolor=lightgray]
\rput(-1,0){
\rput(-3.3,0){\textbf{H$_{2}$N}}
\psline(-2.8,0)(-2.2,0)
\rput(-2,0){\textbf{C}}
\psline(-2,0.2)(-2,0.8)
\psline(-2,-0.2)(-2,-0.8)
\rput(-2,1){\textbf{H}}
\rput(-2,-1){\textbf{H}}
\psline(-1.2,0)(-1.8,0)
\rput(-1,0){\textbf{C}}
\psline(-0.8,0)(-0.2,0)
\psline(-0.95,0.2)(-0.95,0.8)
\psline(-1.05,0.2)(-1.05,0.8)
\rput(-1,1){\textbf{O}}
\rput(0,0){\textbf{OH}}

\rput(5,0){
\rput(-3.3,0){\textbf{H$_{2}$N}}
\psline(-2.8,0)(-2.2,0)
\rput(-2,0){\textbf{C}}
\psline(-2,0.2)(-2,0.8)
\psline(-2,-0.2)(-2,-0.8)
\rput(-2,1){\textbf{H}}
\rput(-2,-1){\textbf{H}}
\psline(-1.2,0)(-1.8,0)
\rput(-1,0){\textbf{C}}
\psline(-0.8,0)(-0.2,0)
\psline(-0.95,0.2)(-0.95,0.8)
\psline(-1.05,0.2)(-1.05,0.8)
\rput(-1,1){\textbf{O}}
\rput(0,0){\textbf{OH}}
}
}
\rput(-0.2,0){\textbf{+}}
\end{pspicture}
\end{center}

	\begin{enumerate}
	\item{Give the structural formula of the polymer that is produced.}
	\item{Is the reaction an addition or condensation reaction?}
	\item{To what group of organic compounds do the two monomers belong?}
	\item{What is the name of the monomers?}
	\item{What type of bond forms between the monomers in the final polymer?}
	\end{enumerate}	
}

\item{The table below shows the melting point for three plastics. Suggest a reason why the melting point of PVC is \textit{higher} than the melting point for polyethene, but \textit{lower} than that for polyester.}

	\begin{center}
	\begin{tabular}{|l|c|}\hline
	\textbf{Plastic} & \textbf{Melting point} ($^{0}$C)\\\hline
	Polyethene & 105 - 115\\\hline
	PVC & 212\\\hline
	Polyester & 260\\\hline
	\end{tabular}
	\end{center}


\item{An amino acid has the formula H$_{2}$NCH(CH$_{2}$CH$_{2}$SCH$_{3}$)COOH.}
	\begin{enumerate}
	\item{Give the structural formula of this amino acid.}
	\item{What is the chemical formula of the carbon side chain in this molecule?}
	\item{Are there any peptide bonds in this molecule? Give a reason for your answer.}
	\end{enumerate}
\end{enumerate}
% Automatically inserted shortcodes - number to insert 5
\par \practiceinfo
\par \begin{tabular}[h]{cccccc}
% Question 1
(1.)	01pq	&
% Question 2
(2.)	01pr	&
% Question 3
(3.)	01ps	&
% Question 4
(4.)	01pt	&
% Question 5
(5.)	01pu	&
\end{tabular}
% Automatically inserted shortcodes - number inserted 5
\end{eocexercises}

