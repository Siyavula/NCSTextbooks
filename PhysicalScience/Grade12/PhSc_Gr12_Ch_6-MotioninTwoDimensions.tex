\chapter{Motion in Two Dimensions}
\label{p:m:m2d12}

\section{Introduction}
In Grade 10, we studied motion in one dimension and briefly looked at vertical motion. In this chapter we will discuss vertical motion and also look at motion in two dimensions. In Grade 11, we studied the conservation of momentum and looked at applications in one dimension. In this chapter we will look at momentum in two dimensions.
\chapterstartvideo{VPnio}
\section{Vertical Projectile Motion}
%\begin{syllabus}
%\item The learner must be able to understand and explain that for vertical projectile motion (near the surface of the Earth if air friction is ignored) projectiles
%\begin{itemize}
%\item fall freely with gravitational acceleration 'g' 
%\item accelerate downwards with a constant acceleration whether the projectile is moving upward or downward
%\item have zero velocity at their greatest height
%\item take the same the time to reach their greatest height from the point of upward launch as the time they take to fall back to the point of launch
%\item can have their motion described by a single set of equations for the upward and downward motion
%\end{itemize}
%\item The learner must be able to use equations of motion, e.g. to determine (1) the greatest height reached given the velocity with which the projectile is launched upward (initial velocity) (2) the time at which a projectile is at a particular height given its initial velocity (3) the height relative to the ground of the position of a projectile shot vertically upward at launch, given the time for the projectile to reach the ground
%\item The learner must be able to draw position vs time (x vs t), velocity vs time (v vs t) and acceleration vs time (a vs t) graphs for projectile motion 
%\item The learner must be able to give equations for position versus time and velocity versus time for the graphs of motion of particular projectiles and vice versa.
%\item Given x vs t, v vs t or a vs t graphs, the learner must be able to  (1) determine position, displacement, velocity or acceleration at any time t (2) describe the motion of the object e.g. graphs showing a ball, (a)	bouncing, (b) thrown vertically upwards (c) thrown vertically downward, and so on 
%\item Notes: Link to Grade 10 motion in one dimension. Link to Grade 11 Newton's second Law and the Law of Universal Gravitation. The equations of motion are identical to the equations introduced in Grade 10. Do not derive a special set of equations - use the general equations and set a=g for the vertical component of the motion.
%\end{syllabus}

In Grade 10, we studied the motion of objects in free fall and we saw that an object in free fall falls with gravitational acceleration $g$.
Now we can consider the motion of objects that are thrown upwards and then fall back to the Earth. We call this \textit{projectile motion} and we will only consider the situation where the object is thrown straight upwards and then falls straight downwards - this means that there is no horizontal displacement of the object, only a vertical displacement.

\subsection{Motion in a Gravitational Field}
When an object is in the earth's gravitational field, it always accelerates downwards, towards the centre of the earth, with a constant acceleration $g$, no matter whether the object is moving upwards or downwards. This is shown in Figure~\ref{fig:p:m:m2d12:projectile}.

\Tip{Projectiles moving upwards or downwards in the earth's gravitational field always accelerate downwards with a constant acceleration $g$. Note: acceleration means that the velocity \textit{changes}; it either becomes greater \textit{or} smaller.}

\begin{figure}[htbp]
\begin{center}
\begin{pspicture}(0,1)(2,3)
\SpecialCoor
%\psgrid[gridcolor=gray]
\psline{->}(0,2)(0,3)
\psdots[dotsize=6pt](0,2)
\uput[l](0,2){object moving upwards}
\psline{->}(0.2,2.2)(0.2,1.8)

\uput[r](0.2,2){$g$}
\rput(2,0){\psline{->}(0,2)(0,1)
\psdots[dotsize=6pt](0,2)
\uput[r](0,2){object moving downwards}}
\rput(1.6,0){\psline{->}(0.2,2.2)(0.2,1.8)
\uput[l](0.2,2){$g$}}
\end{pspicture}
\caption{Objects moving upwards or downwards, always accelerate downwards.}
\label{fig:p:m:m2d12:projectile}
\end{center}
\end{figure}

This means that if an object is moving upwards, its velocity decreases until it stops ($v_f=0$~\ms). This is the maximum height that the object reaches, because after this, the object starts to fall.

\Tip{Projectiles have zero velocity at their greatest height.}

Consider an object thrown upwards from a vertical height $h_o$. We have seen that the object will travel upwards with decreasing velocity until it stops, at which point it starts falling. The time that it takes for the object to fall down to height $h_o$ is the same as the time taken for the object to reach its maximum height from height $h_o$.

\begin{figure}[htbp]
\begin{center}
\subfigure[time = 0~s]{
\begin{pspicture}(-1,1)(1,4)
%\psgrid[gridcolor=gray]
\psline[linestyle=dashed](-1,2)(1,2)
\uput[l](-1,2){initial height $h_0$}
\psline[linestyle=dashed](-1,4)(1,4)
\psline[linestyle=dashed](-1,4)(1,4)
\uput[l](-1,4){maximum height}
\psline{->}(0,2)(0,3)
\psdots[dotsize=6pt](0,2)
\end{pspicture}}
\subfigure[time = $t_m$]{
\begin{pspicture}(-1,1)(1,4)
%\psgrid[gridcolor=gray]
\psline[linestyle=dashed](-1,2)(1,2)
\psline[linestyle=dashed](-1,4)(1,4)
\psdots[dotsize=6pt](0,4)
\end{pspicture}}
\subfigure[time = $2t_m$]{
\begin{pspicture}(-1,1)(1,4)
%\psgrid[gridcolor=gray]
\psline[linestyle=dashed](-1,2)(1,2)
\psline[linestyle=dashed](-1,4)(1,4)
\psline{->}(0,2)(0,1)
\psdots[dotsize=6pt](0,2)
\end{pspicture}}
\caption{(a) An object is thrown upwards from height $h_0$. (b) After time $t_m$, the object reaches its maximum height, and starts to fall. (c) After a time $2t_m$ the object returns to height $h_0$.}
\label{fig:p:m:m2d12:maxheighttime}
\end{center}
\end{figure}

\Tip{Projectiles take the same time to go from the point of launch to the greatest height as the time they take to fall back to the point of launch.}

\subsection{Equations of Motion}
The equations of motion that were used in Chapter~\ref{p:gme10} to describe free fall can be used for projectile motion. These equations are the same as those equations that were derived in Chapter~\ref{p:m1d10}, but with acceleration from gravity: $a=g$. We use $g=9,8\emss$ for our calculations. 

Remember that when you use these equations, you are dealing with vectors which have magnitude \textit{and} direction. Therefore, you need to decide which direction will be the positive direction so that your vectors have the correct signs. 

\begin{eqnarray*}
v_i &=& \mbox{initial velocity (\ms) at $t$ = 0 s} \\
v_f &=& \mbox{final velocity (\ms) at time $t$}\\
\Delta x &=& \mbox{vertical displacement (m)} \\
t &=& \mbox{time (s)} \\
\Delta t &=& \mbox{time interval (s)} \\
g &=& \mbox{acceleration due to gravity (\mss)}
\end{eqnarray*}

\begin{eqnarray}
v_f &=& v_i + gt \label{eq:pg:eq1}\\
\Delta x &=& \frac{(v_i + v_f)}{2} t\label{eq:pg:eq2}\\
\Delta x &=& v_it + \frac{1}{2}gt^2 \label{eq:pg:eq3}\\
v_f^2 &=& v_i^2 + 2g\Delta x \label{eq:pg:eq4}
\end{eqnarray}
% PhET simulation on projectile motion: SIYAVULA-SIMULATION:http://cnx.org/content/m39546/latest/#projectile-motion
\simulation{PhET sim on motion}{VPnlc}
\begin{wex}{Projectile motion}{A ball is thrown upwards with an initial velocity of 10~\ms. \begin{enumerate}
	\item Determine the maximum height reached above the thrower's hand.
	\item Determine the time it takes the ball to reach its maximum height.
	\end{enumerate}}
{\westep{Identify what is required and what is given}
We are required to determine the maximum height reached by the ball and how long it takes to reach this height. We are given the initial velocity $v_i$ = 10 \ms and the acceleration due to gravity g = 9,8 \mss.

\westep{Determine how to approach the problem}
Choose down as positive. We know that at the maximum height the velocity of the ball is 0~\ms. We therefore have the following:
\begin{itemize}
\item{$v_i=-10\ems$ (it is negative because we chose downwards as positive)}
\item{$v_f=0\ems$}
\item{$g=+9,8\emss$}
\end{itemize}

\westep{Identify the appropriate equation to determine the height.}
We can use:
\nequ{v_f^2 = v_i^2 + 2g \Delta x}
to solve for the height.

\westep{Substitute the values in and find the height.}
\begin{eqnarray*}
v_f^2 &=& v_i^2 + 2g\Delta x\\
(0)^{2} &=& (-10)^{2} + (2)(9,8)  (\Delta x)\\
-100 &=& 19,6 \Delta x\\
\Delta x &=& -5,102 \rm{m}
\end{eqnarray*}
The value for the displacement will be negative because the displacement is upwards and we have chosen downward as positive (and upward as negative). The height will be a positive number, $h=5,10$m.

\westep{Identify the appropriate equation to determine the time.}
We can use:
\nequ{v_f = v_i + gt}
to solve for the time.

\westep{Substitute the values in and find the time.}
\begin{eqnarray*}
v_f &=& v_i + gt\\
0  &=& -10 + 9,8t\\
10 &=& 9,8 t\\
t  &=& 1,02 \rm{s}
\end{eqnarray*}

\westep{Write the final answer.}
The ball reaches a maximum height of 5,10 m.\\
The ball takes 1,02 s to reach the top.}
\end{wex}

\begin{wex}{Height of a projectile}{A cricketer hits a cricket ball from the ground so that it goes directly upwards. If the ball takes, 10~s to return to the ground, determine its maximum height.}{
\setcounter{stepcounter}{1}
\westep{Identify what is required and what is given}
We need to find how high the ball goes. We know that it takes 10 seconds to go up and down. We do not know what the initial velocity of the ball ($v_i$) is.\\
%We are required to determine the maximum height of a ball, thrown upwards, that takes 10~s to return to the ground.

\westep{Determine how to approach the problem}
\begin{minipage}{0.5\textwidth}
A problem like this one can be looked at as if there are two parts to the motion. The first is the ball going up with an initial velocity and stopping at the top (final velocity is zero). The second motion is the ball falling, its initial velocity is zero and its final velocity is unknown.\\
\end{minipage}
\begin{minipage}{0.05\textwidth}
\begin{center}
\end{center}
\end{minipage}
\begin{minipage}{0.45\textwidth}
\begin{center}
\begin{pspicture}(0,0)(3,3)
%\psgrid
\psline{->}(1,0)(1,2)
\pscurve(1,2)(1.2,2.2)(1.4,2)
\psline{->}(1.4,2)(1.4,0)
\rput[r](0.5,0){$v_i$ = ?}
%\rput[r](0.5,0){$v_i$ = ?}
\rput[r](0.5,2){$v_f$ = 0 \ms}
\rput[l](1.6,2){$v_i$ = 0 \ms}
\rput[l](1.6,0){$v_f$ = ?}
\rput[l](1.6,1){$g$ = 9,8 \mss}
\end{pspicture}
\end{center}
\end{minipage}
\vspace{1cm}

Choose down as positive. We know that at the maximum height, the velocity of the ball is 0 \ms. We also know that the ball takes the same time to reach its maximum height as it takes to travel from its maximum height to the ground. This time is half the total time. We therefore know the following for the second motion of the ball going down:
\begin{itemize}
\item{$t=5\es$, half of the total time}
\item{$v_{top}=v_i=0\ems$}
\item{$g=9,8\emss$}
\item{$\Delta x=$ ?}
\end{itemize}

\westep{Find an appropriate equation to use}
We do not know the final velocity of the ball coming down. We need to choose an equation that does not have $v_f$ in it. We can use the following equation to solve for $\Delta x$:
\nequ{\Delta x = v_it + \frac{1}{2}gt^2}

\westep{Substitute values and find the height.}
\begin{eqnarray*}
\Delta x &=& (0)(5) + \frac{1}{2}(9,8)(5)^2\\
\Delta x &=& 0 + 122,5 \rm{m}\\
%\rm{height} &=& 122,5 \rm{m}
\end{eqnarray*}

In the second motion, the displacement of the ball is 122,5m downwards. This means that the height was 122,5m, h=122,5m.

\westep{Write the final answer}
The ball reaches a maximum height of 122,5 m.}
\end{wex}

\Exercise{Equations of Motion}{
\begin{enumerate}
\item A cricketer hits a cricket ball straight up into the air. The cricket ball has an initial velocity of 20 \ms. 
	\begin{enumerate}
	\item What height does the ball reach before it stops to fall back to the ground. 
	\item How long has the ball been in the air for?
	\end{enumerate}
\item Zingi throws a tennis ball straight up into the air. It reaches a height of 80 cm.
	\begin{enumerate}
	\item Determine the initial velocity of the tennis ball.
	\item How long does the ball take to reach its maximum height?
	\end{enumerate}
\item A tourist takes a trip in a hot air balloon. The hot air balloon is ascending (moving up) at a velocity of 4 \ms. He accidentally drops his camera over the side of the balloon's  basket, at a height of 20 m. Calculate the velocity with which the camera hits the ground.
\end{enumerate}
\scalebox{1} % Change this value to rescale the drawing.
{
\begin{pspicture}(0,-3.36)(9.702812,3.36)
\definecolor{color597b}{rgb}{0.8,0.8,0.8}
\psbezier[linewidth=0.04](5.075483,1.6850274)(5.3500285,0.9347714)(5.8591995,-0.17347302)(6.3209376,-0.2)(6.7826757,-0.22652698)(7.6056128,1.1356267)(7.703275,1.8485246)(7.8009377,2.5614226)(7.261549,3.32)(6.3587193,3.32)(5.4558897,3.32)(4.8009377,2.4352832)(5.075483,1.6850274)
\psline[linewidth=0.04cm](5.0209374,1.82)(7.6809373,1.82)
\psframe[linewidth=0.04,dimen=outer,fillstyle=solid,fillcolor=color597b](6.8609376,0.14)(5.8009377,-0.92)
\pscircle[linewidth=0.02,dimen=outer](5.9609375,0.58){0.12}
\psline[linewidth=0.02cm](5.9409375,0.48)(5.9409375,0.12)
\psline[linewidth=0.02cm](5.9409375,0.34)(5.5809374,0.2)
\pscircle[linewidth=0.02,dimen=outer](5.9109373,0.61){0.03}
\psline[linewidth=0.02cm](5.9609375,0.56)(5.8809376,0.54)
\psframe[linewidth=0.02,dimen=outer](5.6009374,0.28)(5.3409376,0.02)
\pscircle[linewidth=0.02,dimen=outer](5.4709377,0.17){0.05}
\psline[linewidth=0.03cm,linestyle=dotted,dotsep=0.16cm](5.8209376,0.04)(1.3409375,0.04)
\psline[linewidth=0.04cm,arrowsize=0.05291667cm 3.0,arrowlength=2.0,arrowinset=0.4]{<->}(1.4409375,0.06)(1.4409375,-3.34)
\psline[linewidth=0.03cm](1.4409375,-3.28)(8.820937,-3.28)
\psline[linewidth=0.024cm,arrowsize=0.05291667cm 3.0,arrowlength=2.4,arrowinset=0.4]{->}(5.4409375,0.04)(5.4409375,-0.56)
\psline[linewidth=0.04cm,arrowsize=0.05291667cm 4.0,arrowlength=2.0,arrowinset=0.4]{->}(7.8809376,1.72)(7.8809376,3.34)
\usefont{T1}{ptm}{m}{n}
\rput(9.002344,2.73){$4 \ems$}
\usefont{T1}{ptm}{m}{n}
\rput(0.51234376,-1.69){20 m}
\end{pspicture} 
}

% Automatically inserted shortcodes - number to insert 3
\par \practiceinfo
\par \begin{tabular}[h]{cccccc}
% Question 1
(1.)	01rz	&
% Question 2
(2.)	01s0	&
% Question 3
(3.)	01s1	&
\end{tabular}
% Automatically inserted shortcodes - number inserted 3
}

%\pagebreak[4]
\subsection{Graphs of Vertical Projectile Motion}
Vertical projectile motion is the same as motion at constant acceleration. In Grade 10 you learnt about the graphs for motion at constant acceleration. The graphs for vertical projectile motion are therefore identical to the graphs for motion under constant acceleration.\\
When we draw the graphs for vertical projectile motion, we consider two main situations: an object moving upwards and an object moving downwards.\\
If we take the upwards direction as positive then for an object moving upwards we get the graphs shown in Figure~\ref{fig:p:m:m2d12:pm:up}.

\begin{figure}[htbp]
\begin{center}
\begin{pspicture}(-1,-1.4)(12.4,3.6)
%\psgrid[gridcolor=lightgray]
\rput(0,0){
\psaxes[labels=none,ticks=none]{->}(0,0)(3,3)
\psplot[unit=2,plotstyle=curve]{0}{1.02}{x 2 exp -4.9 mul 5 x mul add}
\uput[u](0,3){$h$ (m)}
\uput[r](3,0){$t$ (s)}
\psline[linestyle=dashed](1.02,0)(1.02,2.75)
\uput[d](1.02,0){$t_m$}
\uput[d](2.04,0){$t_f$}
\uput[l](0,2.55){$h_m$}
\uput[d](1.5,-0.5){(a)}
\uput[l](0,0){0}}
\rput(4.5,1){
\psaxes[labels=none,ticks=none]{->}(0,0)(0,-2)(2.5,2)
\psplot[xunit=2,yunit=0.3]{0}{1.02}{ -9.8 x mul 5 add}
\uput[u](0,2){$v$ (\ms)}
\uput[r](2.5,0){$t$ (s)}
\uput[d](1.02,0){$t_m$}
\uput[u](2.04,0){$t_f$}
\psline[linestyle=dashed](2.04,-1.6)(2.04,0)
\uput[l](0,0){0}
\uput[l](0,5){$v_{i}$}
\uput[l](0,-5){$v_{f}$}
\uput[d](1.5,-1.5){(b)}}
\rput(9,1){\psaxes[labels=none,ticks=none]{->}(0,0)(0,-2)(2.5,2)
\psline[linewidth=2pt](0,-1)(2,-1)
\uput[l](0,-1){$g$}
\uput[u](0,2){$a$ (\mss)}
\uput[r](2.5,0){$t$ (s)}
\uput[d](1.5,-1.5){(c)}
\uput[l](0,0){0}}
\end{pspicture}
\caption{Graphs for an object thrown upwards with an initial velocity $v_i$. The object takes $t_m$~s to reach its maximum height of $h_m$~m after which it falls back to the ground. (a) position vs. time graph (b) velocity vs. time graph (c) acceleration vs. time graph.}\label{fig:p:m:m2d12:pm:up}
\end{center}
\end{figure}

\begin{wex}{Drawing Graphs of Projectile Motion}{Stanley is standing on the a balcony 20 m above the ground. Stanley tosses up a rubber ball with an initial velocity of 4,9~\ms. The ball travels upwards and then falls to the ground. Draw graphs of position vs. time, velocity vs. time and acceleration vs. time. Choose upwards as the positive direction.}{
\westep{Determine what is required}
We are required to draw graphs of 
	\begin{enumerate}
	\item $\Delta x$ vs. $t$
	\item $v$ vs. $t$
	\item $a$ vs. $t$
	\end{enumerate}

\westep{Determine how to approach the problem}
%\begin{minipage}{0.5\textwidth}
There are two parts to the motion of the ball:
\begin{enumerate}
\item ball travelling upwards from the building
\item ball falling to the ground
\end{enumerate}
We examine each of these parts separately. To be able to draw the graphs, we need to determine the time taken and displacement for each of the motions.
%\end{minipage}


\westep{Find the height and the time taken for the first motion.}
\begin{minipage}{0.49\textwidth}
For the first part of the motion we have:
\begin{itemize}
\item{$v_i=+4,9\ems$}
\item{$v_f=0\ems$}
\item{$g=-9,8\emss$}
\end{itemize}
\end{minipage}

\begin{minipage}{0.49\textwidth}
\begin{center}
\begin{pspicture}(0,0)(4.4,5)
%\psgrid[gridcolor=lightgray]
\psframe(0,0)(1,5)
\psframe[fillcolor=black](1,4)(1.4,4.2)
\psline(1,0)(2.4,0)
%\psline[linewidth=1pt,linestyle=dashed]{|-|}(0.5,0)(0.5,4.2)
%\rput(0.55,2.2){20 m}
%\psline[linewidth=1pt,]{->}(2,2.6)(2,2.2)
%\uput[r](2,2.4){$g=9,8$\mss}
\psline[linewidth=1pt,]{->}(2.5,3.95)(2.5,3.55)
\uput[r](2.5,3.75){$g=-9,8$\mss}
%\psline[linewidth=1pt,]{->}(2,5.0)(2,4.6)
%\uput[r](2,4.8){$v_i=0$\ms}
%\psline[linewidth=1pt,]{->}(2,3.2)(2,2.8)
%\uput[r](2,3){$\Delta x=20+1,225$ m}
\psline[linewidth=1pt,]{->}(2.5,3.1)(2.5,3.5)
\uput[r](2.5,3.3){$v_i=4,9$\ms}
\uput[r](2.5,4.2){$v_f=0$\ms}
\psline[linewidth=1pt](1.4,4.2)(1.4,4.8)
\pscurve(1.4,4.8)(1.6,5)(1.8,4.8)
\psline[linewidth=1pt]{->}(1.8,4.8)(1.8,0)
\psline[linewidth=1pt,linestyle=dashed](1,5)(2.5,5)
\end{pspicture}
\end{center}
\end{minipage}

Therefore we can use $v_f^2 = v_i^2 + 2g \Delta x$ to solve for the height and $v_f=v_i+gt$ to solve for the time.\\
\hspace*{-2cm}
\begin{minipage}{0.35\textwidth}
\begin{eqnarray*}
v_f^2 &=& v_i^2 + 2g \Delta x\\
(0)^2 &=& (4,9)^2 + 2 \times (-9,8) \times \Delta x\\
19,6 \Delta x &=& (4,9)^2\\
\Delta x &=& 1,225~m\\
\end{eqnarray*}
\end{minipage}
\hspace*{2cm}
\begin{minipage}{0.35\textwidth}
\begin{eqnarray*}
v_f &=& v_i + gt\\
0 &=& 4,9 + (-9,8) \times t\\
9,8 t &=& 4,9\\
t &=& 0,5~s\\
\end{eqnarray*}
\end{minipage}
\begin{center}
\begin{pspicture}(0,0)(4.4,5)
%\psgrid[gridcolor=lightgray]
\psframe(0,0)(1,5)
\psframe[fillcolor=black](1,4)(1.4,4.2)
\psline(1,0)(4.4,0)
\uput[r](-2.5,4.8){$t=0,5$s}
\uput[r](-2.5,4.4){$\Delta x=1,225$m}
\psline[linewidth=1pt](1.4,4.2)(1.4,4.8)
\pscurve(1.4,4.8)(1.6,5)(1.8,4.8)
\psline[linewidth=1pt]{->}(1.8,4.8)(1.8,0)
\psline[linewidth=1pt,linestyle=dashed](1,5)(2.5,5)
\end{pspicture}
\end{center}

\westep{Find the height and the time taken for the second motion.}
\begin{minipage}{0.49\textwidth}
For the second part of the motion we have:
\begin{itemize}
\item{$v_i=0\ems$}
\item{$\Delta x=-(20 + 1,225)$ m}
\item{$g=-9,8\emss$}
\end{itemize}
Therefore we can use $\Delta x = v_it + \frac{1}{2}gt^2$ to solve for the time.
\begin{eqnarray*}
\Delta x &=& v_it + \frac{1}{2}gt^2\\
-(20+1,225) &=& (0) \times t + \frac{1}{2} \times (-9,8) \times t^2\\
-21,225 &=& 0 - 4,9t^2\\
t^2 &=& 4,33163 ...\\
t &=& 2,08125 ...~s\\
\end{eqnarray*}
\end{minipage}

\begin{minipage}{0.49\textwidth}
\begin{center}
\begin{pspicture}(0,0)(4.4,5)
%\psgrid[gridcolor=lightgray]
\psframe(0,0)(1,5)
\psframe[fillcolor=black](1,4)(1.4,4.2)
\psline(1,0)(4.4,0)
\psline[linewidth=1pt,linestyle=dashed]{|-|}(0.5,0)(0.5,4.2)
\rput(0.5,2.2){20 m}
\psline[linewidth=1pt,]{->}(2,2.6)(2,2.2)
\uput[r](2,2.4){$g=-9,8$ \mss}
%\psline[linewidth=1pt,]{->}(2,5.0)(2,4.6)
\uput[r](2,4.8){$v_i=0$ \ms}
\psline[linewidth=1pt,]{->}(2,3.2)(2,2.8)
\uput[r](2,3){$\Delta x=-21,225$ m}
\psline[linewidth=1pt](1.4,4.2)(1.4,4.8)
\pscurve(1.4,4.8)(1.6,5)(1.8,4.8)
\psline[linewidth=1pt]{->}(1.8,4.8)(1.8,0)
\psline[linewidth=1pt,linestyle=dashed](1,5)(2.5,5)
\end{pspicture}
\end{center}
\end{minipage}

\westep{Graph of position vs. time}
The ball starts from a position of 20 m (at t = 0 s) from the ground and moves upwards until it reaches (20 + 1,225) m (at t = 0,5 s). It then falls back to 20 m (at t = 0,5 + 0,5 = 1,0 s) and then falls to the ground, $\Delta$ x = 0 m at (t = 0,5 + 2,08 = 2,58 s).

\begin{center}
\scalebox{1} % Change this value to rescale the drawing.
{
\begin{pspicture}(0,-2.76)(6.26,2.76)
\psline[linewidth=0.042cm,arrowsize=0.05291667cm 2.0,arrowlength=1.4,arrowinset=0.4]{->}(0.58,-2.26)(0.56,2.34)
\psline[linewidth=0.042cm,arrowsize=0.05291667cm 2.0,arrowlength=1.4,arrowinset=0.4]{->}(0.56,-2.26)(5.6,-2.26)
\psline[linewidth=0.042cm](1.36,-2.04)(1.36,-2.44)
\psline[linewidth=0.042cm](2.16,-2.06)(2.16,-2.46)
\psline[linewidth=0.042cm](2.96,-2.06)(2.96,-2.46)
\psline[linewidth=0.042cm](3.76,-2.04)(3.76,-2.44)
\psline[linewidth=0.042cm](4.56,-2.06)(4.56,-2.46)
\psline[linewidth=0.042cm](0.38,-0.08)(0.78,-0.08)
\usefont{T1}{ptm}{m}{n}
\rput(0.65,2.565){x [m]}
\usefont{T1}{ptm}{m}{n}
\rput(5.94,-2.255){t [s]}
\rput(0.61,-2.555){\scriptsize 0}
\rput(1.37,-2.575){\scriptsize 0,5}
\rput(2.17,-2.575){\scriptsize 1,0}
\rput(2.96,-2.575){\scriptsize 1,5}
\rput(3.78,-2.575){\scriptsize 2,0}
\rput(4.57,-2.575){\scriptsize 2,5}
\rput(0.39,-2.235){\scriptsize 0}
\rput(0.22,-0.095){\scriptsize 20}
\psbezier[linewidth=0.042](0.58,-0.08)(0.64,-0.14)(2.0,1.2)(4.68,-2.26)
\psline[linewidth=0.042cm,linestyle=dotted,dotsep=0.16cm](1.38,0.08)(1.36,-2.28)
\psline[linewidth=0.042cm,linestyle=dotted,dotsep=0.16cm](0.6,0.12)(1.38,0.1)
\psline[linewidth=0.042cm,linestyle=dotted,dotsep=0.16cm](0.56,-0.08)(2.16,-0.06)
\psline[linewidth=0.042cm,linestyle=dotted,dotsep=0.16cm](2.16,-0.06)(2.16,-2.28)
\rput(0.25,0.165){\scriptsize 21,2}
\end{pspicture} 
}
\end{center}

\westep{Graph of velocity vs. time}
The ball starts off with a velocity of +4,9 \ms at t = 0 s, it then reaches a velocity of 0 \ms at t = 0,5 s. It stops and falls back to the Earth. At t = 1,0s (i.e. after a further 0,5s) it has a velocity of -4,9 \ms. This is the same as the initial upwards velocity but it is downwards. It carries on at constant acceleration until t = 2,58 s. In other words, the velocity graph will be a straight line.
The final velocity of the ball can be calculated as follows:
\begin{eqnarray*}
v_f&=&v_i+g t\\
&=&0+(-9,8)(2,08...)\\
&=&-20,396... \ems
\end{eqnarray*}

\begin{center}
\begin{pspicture}(0,-3.6742187)(9.502812,3.6992188)
\psline[linewidth=0.03cm,arrowsize=0.05291667cm 3.0,arrowlength=2.0,arrowinset=0.4]{<->}(1.4409375,3.0207813)(1.4409375,-3.6592188)
\psline[linewidth=0.03cm,arrowsize=0.05291667cm 3.0,arrowlength=2.0,arrowinset=0.4]{->}(1.4409375,1.2407813)(8.380938,1.2407813)
\psline[linewidth=0.04cm](1.4409375,2.5007813)(8.140938,-2.9792187)
\psline[linewidth=0.03cm,linestyle=dashed,dash=0.16cm 0.16cm](4.2809377,1.2407813)(4.2809377,0.16078125)
\psline[linewidth=0.03cm,linestyle=dashed,dash=0.16cm 0.16cm](4.3209376,0.16078125)(1.4609375,0.16078125)
\psline[linewidth=0.03cm,linestyle=dashed,dash=0.16cm 0.16cm](7.2609377,1.2407813)(7.3409376,-2.2792187)
\psline[linewidth=0.03cm,linestyle=dashed,dash=0.16cm 0.16cm](7.3409376,-2.3192186)(1.4409375,-2.3192186)
\psdots[dotsize=0.14](4.2809377,0.16078125)
\psdots[dotsize=0.14](7.3009377,-2.2792187)
\psdots[dotsize=0.14](2.9809375,1.2607813)
\usefont{T1}{ptm}{m}{n}
\rput(9.042344,1.2307812){$t(s)$}
\usefont{T1}{ptm}{m}{n}
\rput(1.5223438,3.5107813){$v (\ems)$}
\psdots[dotsize=0.12](1.4609375,2.5007813)
\usefont{T1}{ptm}{m}{n}
\rput(0.8123438,2.4707813){$4,9$}
\usefont{T1}{ptm}{m}{n}
\rput(0.7523438,0.17078125){$-4,9$}
\usefont{T1}{ptm}{m}{n}
\rput(0.6123437,-2.3092186){$-20,40$}
\usefont{T1}{ptm}{m}{n}
\rput(3.0723438,1.8107812){$0,5$}
\usefont{T1}{ptm}{m}{n}
\rput(4.552344,1.5907812){$1,0$}
\usefont{T1}{ptm}{m}{n}
\rput(7.4423437,1.7307812){$2,58$}
\end{pspicture}
\end{center}
\westep{Graph of a vs t}
We chose upwards to be positive. The acceleration of the ball is downward. $g=-9.8 \emss$. Because the acceleration is constant throughout the motion, the graph looks like this:
\begin{center}
\begin{pspicture}(0,-3.5742188)(9.002812,3.5992188)
\psline[linewidth=0.03cm,arrowsize=0.05291667cm 3.0,arrowlength=2.0,arrowinset=0.4]{<->}(1.2609375,2.9607813)(1.2609375,-3.5592186)
\psline[linewidth=0.03cm,arrowsize=0.05291667cm 3.0,arrowlength=2.0,arrowinset=0.4]{->}(1.2609375,0.06078125)(7.8209376,0.06078125)
\psline[linewidth=0.04cm](1.2609375,-1.5792187)(6.5809374,-1.5792187)
\psline[linewidth=0.03cm,linestyle=dashed,dash=0.16cm 0.16cm](6.5609374,-1.5592188)(6.5609374,0.08078125)
\usefont{T1}{ptm}{m}{n}
\rput(8.542344,0.03078125){$t(s)$}
\usefont{T1}{ptm}{m}{n}
\rput(1.4523437,3.4107811){$a(\emss)$}
\usefont{T1}{ptm}{m}{n}
\rput(6.5623436,0.43078125){$2,58$}
\usefont{T1}{ptm}{m}{n}
\rput(0.43234375,-1.5692188){$-9,8$}
\end{pspicture}
\end{center}
}
\end{wex}

\begin{wex}{Analysing Graphs of Projectile Motion}
{The graph below (not drawn to scale) shows the motion of tennis ball that was thrown vertically upwards from an open window some distance from the ground. It takes the ball 0,2 s to reach its highest point before falling back to the ground. Study the graph given and calculate 
	\begin{enumerate}
	\item how high the window is above the ground.
	\item the time it takes the ball to reach the maximum height.
	\item the initial velocity of the ball.
	\item the maximum height that the ball reaches.
	\item the final velocity of the ball when it reaches the ground.
	\end{enumerate}
\begin{center}
\begin{pspicture}(-1,-1)(5,5.5)
\psline{->}(0,0)(0,5.5)
\psline{->}(0,0)(5,0)
\pscurve(0,4)(1,4.5)(2,4)(4,0)
\rput(2,-0.6){time (s)}
\rput(-2,2){Position (m)}
\psline(1,-0.1)(1,0.1)
\psline(2,-0.1)(2,0.1)
%\psline(3,-0.1)(3,0.1)
\psline(4,-0.1)(4,0.1)
\rput(1,-0.3){0,2}
\rput(2,-0.3){0,4}
\rput(4,-0.3){?}
\psline(-0.1,1)(0.1,1)
\psline(-0.1,2)(0.1,2)
\psline(-0.1,3)(0.1,3)
\psline(-0.1,4)(0.1,4)
\psline(-0.1,5)(0.1,5)
\rput(-0.3,1){1}
\rput(-0.3,2){2}
\rput(-0.3,3){3}
\rput(-0.3,4){4}
\rput(-0.3,5){5}
\psline[linestyle=dotted]{-}(1,0)(1,4.5)
\psline[linestyle=dotted]{-}(2,0)(2,4)
\psline[linestyle=dotted]{-}(0,4.5)(1,4.5)
\psline[linestyle=dotted]{-}(0,4)(2,4)
\end{pspicture}
\end{center}}
{\westep{Find the height of the window.}
The initial position of the ball will tell us how high the window is. From the y-axis on the graph we can see that the ball is 4 m from the ground.\\
The window is therefore 4 m above the ground.\\
\westep{Find the time taken to reach the maximum height.}
The maximum height is where the position-time graph show the maximum position - the top of the curve. This is when t = 0,2 s.\\
It takes the ball 0,2s to reach the maximum height.\\
\westep{Find the initial velocity ($v_i$) of the ball.}
To find the initial velocity we only look at the first part of the motion of the ball. That is from when the ball is released until it reaches its maximum height. We have the following for this: In this case, let's choose upwards as positive.
\begin{eqnarray*}
t &=& 0,2~\rm{s}\\
g &=& -9,8~\emss\\
v_f &=& 0~\ems \rm{(because~the~ball~stops)}
\end{eqnarray*}
To calculate the initial velocity of the ball ($v_i$), we use:
\begin{eqnarray*}
v_f &=& v_i + gt\\
0 &=& v_i + (-9,8)(0,2)\\
v_i &=& 1,96 \ems
\end{eqnarray*}
The initial velocity of the ball is 1,96 \ms upwards.\\

\westep{Find the maximum height ($\Delta x$) of the ball.}
To find the maximum height we look at the initial motion of the ball. We have the following:
\begin{eqnarray*}
t &=& 0,2~\rm{s}\\
g &=& -9,8~\emss\\
v_f &=& 0~\ems \rm{(because~the~ball~stops)}\\
v_i &=& +1,96 \ems \rm{(calculated~above)}
\end{eqnarray*}
To calculate the displacement from the window to the maximum height ($\Delta x$) we use:
\begin{eqnarray*}
\Delta x &=& v_it + \frac{1}{2}gt^2\\
\Delta x &=& (1,96)(0,2) + \frac{1}{2}(-9,8)(0,2)^2\\
\Delta x &=& 0,196 \rm{m}\\
\end{eqnarray*}
The maximum height of the ball is (4 + 0,196) = 4,196 m above the ground.\\

\westep{Find the final velocity ($v_f$) of the ball.}
To find the final velocity of the ball we look at the second part of the motion. For this we have:
\begin{eqnarray*}
\Delta x &=& -4,196~\rm{m} \rm~({because~upwards~is~positive})\\
g &=& -9,8~\emss\\
v_i &=& 0~\ems
\end{eqnarray*}
We can use $(v_f)^2 = (v_i)^2 + 2g\Delta x$ to calculate the final velocity of the ball.
\begin{eqnarray*}
(v_f)^2 &=& (v_i)^2 + 2g\Delta x\\
(v_f)^2 &=& (0)^2 + 2(-9,8)(-4,196)\\
(v_f)^2 &=& 82,2416\\
v_f &=& 9,0687 ... \ems\\
\end{eqnarray*}
The final velocity of the ball is 9,07 \ms \textit{downwards}.
}
\end{wex}

\begin{wex}{Describing Projectile Motion}
%{\nts{Need a worked example: given a graph, describe the motion.}}{A cricket ball is hit by a cricketer and the following graph of velocity vs. time was drawn:
{A cricketer hits a cricket ball from the ground and the following graph of velocity vs. time was drawn. Upwards was taken as positive. Study the graph and follow the instructions below:
	\begin{enumerate}
	\item Describe the motion of the ball according to the graph.
	\item Draw a sketch graph of the corresponding displacement-time graph. Label the axes.
	\item Draw a sketch graph of the corresponding acceleration-time graph. Label the axes.
	\end{enumerate}
\begin{center}
\scalebox{1.2} % Change this value to rescale the drawing.
{\begin{pspicture}(0,-3.17)(5.7029686,3.175)
\psline[linewidth=0.05cm,arrowsize=0.05291667cm 2.0,arrowlength=1.4,arrowinset=0.4]{->}(0.40359375,-2.45)(0.40359375,3.15)
\psline[linewidth=0.05cm,arrowsize=0.05291667cm 2.0,arrowlength=1.4,arrowinset=0.4]{->}(0.40359375,0.15)(4.4035935,0.15)
\psline[linewidth=0.04cm](0.40359375,2.15)(2.4035938,-1.85)
\psdots[dotsize=0.12](0.40359375,2.15)
\psdots[dotsize=0.12](1.4035938,0.15)
\psdots[dotsize=0.12](2.4035938,-1.85)
\psline[linewidth=0.04cm,linestyle=dashed,dash=0.16cm 0.16cm](2.4035938,-1.75)(2.4035938,0.15)
\psline[linewidth=0.04cm,linestyle=dashed,dash=0.16cm 0.16cm](1.6035937,-3.15)(1.6035937,-3.15)
\psline[linewidth=0.04cm,linestyle=dashed,dash=0.16cm 0.16cm](2.4035938,-1.85)(0.40359375,-1.85)
\usefont{T1}{ptm}{m}{n}
\rput(0,2.15){\small 19,6}
\usefont{T1}{ptm}{m}{n}
\rput(0,-1.85){\small -19,6}
\usefont{T1}{ptm}{m}{n}
\rput(1.4676563,0.35){\small 2}
\usefont{T1}{ptm}{m}{n}
\rput(2.3746874,0.35){\small 4}
\usefont{T1}{ptm}{m}{n}
\rput(5.1315627,0.15){\small time (s)}
\usefont{T1}{ptm}{m}{n}
\rput{-270.0}(0.5859375,0.2765625){\rput(0.151875,0.45){\small velocity (\ms)}}
\psdots[dotsize=0.12](2.4035938,0.15)
\psdots[dotsize=0.12](0.40359375,-1.85)
\end{pspicture}}
\end{center}
}
{\westep{Describe the motion of the ball.}
We need to study the velocity-time graph to answer this question. We will break the motion of the ball up into two time zones: t = 0 s to t = 2 s and t = 2 s to t = 4 s.\\
From t = 0 s to t = 2 s the following happens:\\
The ball starts to move at an initial velocity of 19,6 \ms and decreases its velocity to 0 \ms at t = 2 s. 
At t = 2 s the velocity of the ball is 0 \ms and therefore it stops.\\
From t = 2 s to t = 4 s the following happens:\\
The ball moves from a velocity of 0 \ms to 19,6 \ms in the opposite direction to the original motion.\\
If we assume that the ball is hit straight up in the air (and we take upwards as positive), it reaches its maximum height at t = 2 s, stops, turns around and falls back to the Earth to reach the ground at t = 4 s.  \\

\westep{Draw the displacement-time graph.}
To draw this graph, we need to determine the displacements at t = 2 s and t = 4 s.\\
At t = 2 s: \\
The displacement is equal to the area under the graph:\\
Area under graph = Area of triangle\\
Area = $\frac{1}{2}$bh\\
Area = $\frac{1}{2}$ $\times$ 2 $\times$ 19,6\\
Displacement = 19,6 m

At t = 4 s: \\
The displacement is equal to the area under the whole graph (top and bottom). Remember that an area under the time line must be subtracted:\\
Area under graph = Area of triangle 1 + Area of triangle 2\\
Area = $\frac{1}{2}$bh + $\frac{1}{2}$bh\\
Area = ($\frac{1}{2}$ $\times$ 2 $\times$ 19,6) + ($\frac{1}{2}$ $\times$ 2 $\times$ (-19,6))\\
Area = 19,6 - 19,6 \\
Displacement = 0 m

The displacement-time graph for motion at constant acceleration is a curve. The graph will look like this:\\
\begin{center}
\scalebox{1.2} % Change this value to rescale the drawing.
{
\begin{pspicture}(0,-0.17)(5.4684377,3.175)
\psline[linewidth=0.05cm,arrowsize=0.05291667cm 2.0,arrowlength=1.4,arrowinset=0.4]{->}(1.1690625,0.15)(1.1690625,3.15)
\psline[linewidth=0.05cm,arrowsize=0.05291667cm 2.0,arrowlength=1.4,arrowinset=0.4]{->}(1.1690625,0.15)(4.1690626,0.15)
\psdots[dotsize=0.12](1.1690625,2.15)
\psdots[dotsize=0.12](2.1690626,0.15)
\psdots[dotsize=0.12](3.1690626,0.15)
\pscurve(1.1690625,0.15)(2.1690625,2.15)(3.1690625,0.15)
\psline[linewidth=0.04cm,linestyle=dashed,dash=0.16cm 0.16cm](2.3690624,-3.15)(2.3690624,-3.15)
\usefont{T1}{ptm}{m}{n}
\rput(0.7475,2.15){\small 19,6}
\usefont{T1}{ptm}{m}{n}
\rput(2.3434375,0.35){\small 2}

\usefont{T1}{ptm}{m}{n}
\rput(3.2454687,0.35){\small 4}
\usefont{T1}{ptm}{m}{n}
\rput(4.8970313,0.15){\small time (s)}
\usefont{T1}{ptm}{m}{n}
\rput{-270.0}(1.6859375,1.3771875){\rput(0.143125,1.55){\small displacement (m)}}
\psline[linewidth=0.04cm,linestyle=dashed,dash=0.16cm 0.16cm](2.1690626,2.15)(2.1690626,0.15)
\psline[linewidth=0.04cm,linestyle=dashed,dash=0.16cm 0.16cm](2.1690626,2.15)(1.1690625,2.15)
\psdots[dotsize=0.12](2.1690626,2.15)
\end{pspicture}}
\end{center}

\westep{Draw the acceleration-time graph.}
To draw the acceleration vs. time graph, we need to know what the acceleration is. The velocity-time graph is a straight line which means that the acceleration is constant. The gradient of the line will give the acceleration.

The line has a negative slope (goes down towards the left) which means that the acceleration has a negative value.

Calculate the gradient of the line:\\
gradient = $\frac{\Delta v}{t}$\\
gradient = $\frac{0-19,6}{2-0}$\\
gradient = $\frac{-19,6}{2}$\\
gradient = -9,8\\
acceleration = 9,8 \mss downwards

\begin{center}
\scalebox{1} % Change this value to rescale the drawing.
{
\begin{pspicture}(0,0)(5.237969,2.175)
\psline[linewidth=0.05cm,arrowsize=0.05291667cm 2.0,arrowlength=1.4,arrowinset=0.4]{<->}(0.93859375,-0.85)(0.93859375,2.15)
\psline[linewidth=0.05cm,arrowsize=0.05291667cm 2.0,arrowlength=1.4,arrowinset=0.4]{->}(0.93859375,1.15)(3.9385939,1.15)
\psdots[dotsize=0.12](1.9385937,1.15)
%\psline[linewidth=0.04cm,linestyle=dashed,dash=0.16cm 0.16cm](2.1385937,-2.15)(2.1385937,-2.15)
\usefont{T1}{ptm}{m}{n}
\rput(0.53703123,0.15){\small -9,8}
\usefont{T1}{ptm}{m}{n}
\rput(2.0129688,1.35){\small 2}
\usefont{T1}{ptm}{m}{n}
\rput(2.915,1.35){\small 4}
\usefont{T1}{ptm}{m}{n}
\rput(4.6665626,1.15){\small time (s)}
\usefont{T1}{ptm}{m}{n}
\rput{-270.0}(0.8859375,0.5771875){\rput(0.1421875,0.75){\small acceleration (\mss)}}
\psdots[dotsize=0.12](2.9385939,1.15)
\psline[linewidth=0.04cm,linestyle=dashed,dash=0.16cm 0.16cm](1.9385937,1.15)(1.9385937,0.15)
\psline[linewidth=0.04cm,linestyle=dashed,dash=0.16cm 0.16cm](2.9385939,1.15)(2.9385939,0.15)
\psline[linewidth=0.04cm](0.93859375,0.15)(2.9385939,0.15)
\psdots[dotsize=0.12](0.93859375,0.15)
\end{pspicture}}
\end{center}
}
\end{wex}
\vspace{1cm}


\Exercise{Graphs of Vertical Projectile Motion}{
\renewcommand{\labelenumii}{\alph{enumii}}

\begin{enumerate}
\item{Amanda throws a tennisball from a height of $1,5 $m straight up into the air and then lets it fall to the ground. Draw graphs of $\Delta x$ vs $t$; $v$ vs $t$ and $a$ vs $t$ for the motion of the ball. The initial velocity of the tennisball is $2\ems$. Choose upwards as positive.}
\item{A bullet is shot straight upwards from a gun. The following graph is drawn. Downwards was chosen as positive}
\begin{enumerate}
\item {Describe the motion of the bullet.}
\item {Draw a displacement - time graph.}
\item {Draw an acceleration - time graph.}
\end{enumerate}
\begin{center}
\begin{pspicture}(0,-3.0442188)(9.922812,3.0692186)
\psline[linewidth=0.03cm,arrowsize=0.05291667cm 3.0,arrowlength=2.0,arrowinset=0.4]{<->}(1.3009375,2.3107812)(1.3009375,-3.0292187)
\psline[linewidth=0.03cm,arrowsize=0.05291667cm 3.0,arrowlength=2.0,arrowinset=0.4]{->}(1.3009375,-0.50921875)(8.380938,-0.48921874)
\psline[linewidth=0.04cm](1.3009375,-2.3892188)(6.7209377,1.7707813)
\psline[linewidth=0.03cm,linestyle=dashed,dash=0.16cm 0.16cm](6.7009373,1.7707813)(6.7009373,-0.46921876)
\psline[linewidth=0.03cm,linestyle=dashed,dash=0.16cm 0.16cm](6.6609373,1.7507813)(1.2609375,1.7507813)
\psline[linewidth=0.03cm](3.7809374,-0.42921874)(3.7809374,-0.68921876)
\usefont{T1}{ptm}{m}{n}
\rput(9.462344,-0.51921874){$t(s)$}
\usefont{T1}{ptm}{m}{n}
\rput(1.4523437,2.8807812){$v(\ems)$}
\usefont{T1}{ptm}{m}{n}
\rput(0.63234377,1.7007812){$200$}
\usefont{T1}{ptm}{m}{n}
\rput(0.47234374,-2.3992188){$-200$}
\usefont{T1}{ptm}{m}{n}
\rput(4.0623436,-1.0792187){$20,4$}
\usefont{T1}{ptm}{m}{n}
\rput(7.202344,-0.81921875){$40,8$}
\end{pspicture} 
\end{center}
\end{enumerate}

% Automatically inserted shortcodes - number to insert 2
\par \practiceinfo
\par \begin{tabular}[h]{cccccc}
% Question 1
(1.)	01s2	&
% Question 2
(2.)	01s3	&
\end{tabular}
% Automatically inserted shortcodes - number inserted 2
}

%\begin{pspicture}(0,-3.825)(5.902969,3.825)
%\psline[linewidth=0.05cm,arrowsize=0.05291667cm 2.0,arrowlength=1.4,arrowinset=0.4]{<->}(0.60359377,-3.8)(0.60359377,3.8)
%\psline[linewidth=0.05cm,arrowsize=0.05291667cm 2.0,arrowlength=1.4,arrowinset=0.4]{->}(0.60359377,0.8)(4.603594,0.8)
%\psdots[dotsize=0.12](0.60359377,2.8)
%\psdots[dotsize=0.12](1.6035937,0.8)
%\psdots[dotsize=0.12](3.6035938,-3.2)
%\psline[linewidth=0.04cm,linestyle=dashed,dash=0.16cm 0.16cm](3.6035938,-3.2)(3.6035938,0.7)
%\psline[linewidth=0.04cm,linestyle=dashed,dash=0.16cm 0.16cm](1.8035938,-2.5)(1.8035938,-2.5)
%\psline[linewidth=0.04cm,linestyle=dashed,dash=0.16cm 0.16cm](3.6035938,-3.2)(0.60359377,-3.2)
%\usefont{T1}{ptm}{m}{n}
%\rput(0.265625,2.8){\small 30}
%\usefont{T1}{ptm}{m}{n}
%\rput(0.36453125,-3.2){\small ?}
%\usefont{T1}{ptm}{m}{n}
%\rput(1.6676563,1.0){\small 3}
%\usefont{T1}{ptm}{m}{n}
%\rput(3.5745313,1.0){\small 9}
%\usefont{T1}{ptm}{m}{n}
%\rput(5.3315625,0.8){\small time (s)}
%\usefont{T1}{ptm}{m}{n}
%\rput{-270.0}(0.4359375,0.1265625){\rput(0.151875,0.3){\small velocity (\ms)}}
%\psline[linewidth=0.04cm](0.60359377,2.8)(3.6035938,-3.2)
%\psdots[dotsize=0.12](3.6035938,0.8)
%\psdots[dotsize=0.12](0.60359377,-3.2)
%\end{pspicture} 

%\Extension{Projectile Motion in Two Dimensions}{This section does not appear in the current syllabus, but appears in the %detailed notes we received from DJG. If there is time, then this section can be written up.
%\begin{syllabus}
%\item calculate the horizontal and vertical components of a projectile launched at an angle, $\theta$, to the horizontal
%\item Use the equations of motion to calculate displacement and velocity for the vertical component of projectile motion, %setting the acceleration equal to $g$
%\item use the equations of motion to calculate displacement and velocity for the horizontal component of projectile motion, %setting the acceleration equal to zero
%\item Use the equations of motion for the horizontal and vertical components of projectile motion to calculate the time taken %for the motion.
%\item Calculate the initial and final velocity from the horizontal and vertical components, i.e. calculate the magnitude and %angle relative to the horizontal.
%\item The equations of motion are identical to the equations introduced in Grade 10. Do not derive a special set of equations %- use the general equations and set a=g for the vertical component of the motion and a=0 for the horizontal component. Be %sure to select positive and negative directions, and make sure that the signs of the displacement, velocity and acceleration %are consistent with the directions you choose. The acceleration of gravity always points downwards. When the velocity has a %horizontal and a vertical component, the magnitude of the velocity can be obtained using Pythagoras theorem, %$v=sqrt(v1^2+vf^2)$ and the angle of elevation from theta = arctan(vy/vx)
%\end{syllabus}}

\section{Conservation of Momentum in Two Dimensions}
%\begin{syllabus}
%\item The learner must know that the momentum of a system is conserved when no external forces act on it
%\item The learner must be able to calculate x and y components of momentum
%\item The learner must be able to solve problems involving conservation of momentum in both the x direction and the y direction
%\item The learner must know that an external force causes the momentum to change. The impulse delivered by the force is $F\Delta t=\Delta p$. 
%\item The learner must be able to solve problems involving impulse and momentum when the applied force is in the horizontal or vertical direction.
%\item The learner must be able to distinguish between elastic and inelastic collisions.
%\item The learner must be able to solve problems involving elastic and inelastic collisions for objects moving parallel or at right angles to each other.
%\item Notes: Link to Grade 11 conservation of momentum. As before, use the equation:pi=pfexcept now use one equation for components in the x direction and one equation for components in the y directionLink to grade 11 impulse.Link to Grade 10 "kinetic energy" Traffic accidents provide a very good context for studying inelastic collisions.
%\end{syllabus}

We have seen in Grade 11 that the momentum of a system is conserved when there are no external forces acting on the system. Conversely, an external force causes a change in momentum $\Delta p$, with the impulse delivered by the force, $F$ acting for a time $\Delta t$ given by:
\nequ{\Delta p=F\cdot \Delta t}
The same principles that were studied in applying the conservation of momentum to problems in one dimension, can be applied to solving problems in two dimensions. 

The calculation of momentum is the same in two dimensions as in one dimension. The calculation of momentum in two dimensions is broken down into determining the $x$ and $y$ components of momentum and applying the conservation of momentum to each set of components. 

Consider two objects moving towards each other as shown in Figure~\ref{fig:p:m:m2d12:momentum:2deg}. We analyse this situation by calculating the $x$ and $y$ components of the momentum of each object.

\begin{figure}[htbp]
\begin{center}
\subfigure[Before the collision]{
\begin{pspicture}(-0.4,-0.4)(5.4,3.4)
%\psgrid[gridcolor=gray]
\psline[linecolor=lightgray](0,0)(1,0)
\psline{->}(0,0)(1;45)
\psline{->}(0,0)(0,0.707)
\psline{->}(0,0)(0.707,0)
\uput[u](0,0.707){$v_{i1y}$}
\uput[r](0.707,0){$v_{i1x}$}
\psline[linestyle=dotted](1;45)(0,0.707)
\psline[linestyle=dotted](1;45)(0.707,0)
\rput(1.2;45){$v_{i1}$}
%\psarc{<->}(0,0){0.75}{0}{45}
\rput(0.5;22.5){$\theta_1$}
\pscircle[fillcolor=white,fillstyle=solid](0,0){0.25}
\rput(0,0){$m_1$}
\rput(5,0){
\psline[linecolor=lightgray](0,0)(-1,0)
\psline{->}(0,0)(1;135)
\psline{->}(0,0)(0,0.707)
\psline{->}(0,0)(-0.707,0)
\uput[u](0,0.707){$v_{i2y}$}
\uput[l](-0.707,0){$v_{i2x}$}
\psline[linestyle=dotted](1;135)(0,0.707)
\psline[linestyle=dotted](1;135)(-0.707,0)
\rput(1.2;135){$v_{i2}$}
%\psarc{<->}(0,0){0.75}{135}{180}
\rput(0.5;157.5){$\theta_2$}
\pscircle[fillcolor=white,fillstyle=solid](0,0){0.25}
\rput(0,0){$m_2$}
}
\psdot(2.5,2.5)
\uput[u](2.5,2.5){P}
\end{pspicture}}
\subfigure[After the collision]{
\begin{pspicture}(-0.4,-0.4)(5.4,3.4)
%\psgrid[gridcolor=gray]
\rput(4,2){\psline[linecolor=lightgray](0,0)(1,0)
\psline{->}(0,0)(1;45)
\psline{->}(0,0)(0,0.707)
\psline{->}(0,0)(0.707,0)
\uput[u](0,0.707){$v_{f1y}$}
\uput[r](0.707,0){$v_{f1x}$}
\psline[linestyle=dotted](1;45)(0,0.707)
\psline[linestyle=dotted](1;45)(0.707,0)
\rput(1.2;45){$v_{f1}$}
%\psarc{<->}(0,0){0.75}{0}{45}
\rput(0.5;22.5){$\phi_1$}
\pscircle[fillcolor=white,fillstyle=solid](0,0){0.25}
\rput(0,0){$m_1$}}
\rput(1,2){
\psline[linecolor=lightgray](0,0)(-1,0)
\psline{->}(0,0)(1;135)
\psline{->}(0,0)(0,0.707)
\psline{->}(0,0)(-0.707,0)
\uput[u](0,0.707){$v_{f2y}$}
\uput[l](-0.707,0){$v_{f2x}$}
\psline[linestyle=dotted](1;135)(0,0.707)
\psline[linestyle=dotted](1;135)(-0.707,0)
\rput(1.2;135){$v_{f2}$}
%\psarc{<->}(0,0){0.75}{135}{180}
\rput(0.5;157.5){$\phi_2$}
\pscircle[fillcolor=white,fillstyle=solid](0,0){0.25}
\rput(0,0){$m_2$}
}
\psdot(2.5,0)
\uput[u](2.5,0){P}
\end{pspicture}
}
\caption{Two balls collide at point P.}
\label{fig:p:m:m2d12:momentum:2deg}
\end{center}
\end{figure}

\subsubsection*{Before the collision}
Total momentum:
\begin{eqnarray*}
p_{i1} &=& m_1 v_{i1}\\
p_{i2} &=& m_2 v_{i2}
\end{eqnarray*}

$x$-component of momentum:
\begin{eqnarray*}
p_{i1x} &=& m_1 v_{i1x} =m_1 v_{i1} \cos \theta_1\\
p_{i2x} &=& m_2 u_{i2x} = m_2 v_{i2} \cos \theta_2
\end{eqnarray*}

$y$-component of momentum:
\begin{eqnarray*}
p_{i1y} &=& m_1 v_{i1y} =m_1 v_{i1} \sin \theta_1\\
p_{i2y} &=& m_2 v_{i2y} = m_2 v_{i2} \sin \theta_2
\end{eqnarray*}

\subsubsection*{After the collision}
Total momentum:
\begin{eqnarray*}
p_{f1} &=& m_1 v_{f1}\\
p_{f2} &=& m_2 v_{f2}
\end{eqnarray*}

$x$-component of momentum:
\begin{eqnarray*}
p_{f1x} &=& m_1 v_{f1x} =m_1 v_{f1} \cos \phi_1\\
p_{f2x} &=& m_2 v_{f2x} = m_2 v_{f2} \cos \phi_2
\end{eqnarray*}

$y$-component of momentum:
\begin{eqnarray*}
p_{f1y} &=& m_1 v_{f1y} =m_1 v_{f1} \sin \phi_1\\
p_{f2y} &=& m_2 v_{f2y} = m_2 v_{f2} \sin \phi_2
\end{eqnarray*}

\subsubsection*{Conservation of momentum}

The initial momentum is equal to the final momentum:
\nequ{p_{i}=p_{f}}

\begin{eqnarray*}
p_{i} &=& p_{i1}+p_{i2}\\
p_{f} &=& p_{f1}+p_{f2}
\end{eqnarray*}

This forms the basis of analysing momentum conservation problems in two dimensions.
% PhET simulation on momentum: SIYAVULA-SIMULATION:http://cnx.org/content/m39550/latest/#momentum
\simulation{PhET sim on momentum}{VPnlm}
\begin{wex}{2D Conservation of Momentum}
{In a rugby game, Player 1 is running with the ball at 5~m$\cdot$s$^{-1}$ straight down the field parallel to the edge of the field. Player 2 runs at 6~m$\cdot$s$^{-1}$ an angle of $60^{\circ}$ to the edge of the field and tackles Player 1. In the tackle, Player 2 stops completely while Player 1 bounces off Player 2. Calculate the velocity (magnitude and direction) at which Player 1 bounces off Player 2. Both the players have a mass of 90 kg.  }
{
\westep{Identify what is required and what is given}
The first step is to draw the picture to work out what the situation is. Mark the initial velocities of both players in the picture.

\begin{center}
\scalebox{1} % Change this value to rescale the drawing.
{
\begin{pspicture}(0,-2.88)(4.7,2.88)
\psframe[linewidth=0.04,dimen=outer](4.06,2.88)(0.0,-2.88)
\psline[linewidth=0.04cm,arrowsize=0.05291667cm 2.0,arrowlength=1.4,arrowinset=0.4]{->}(1.7,-1.12)(1.7192655,0.739855)
\psline[linewidth=0.04cm,arrowsize=0.05291667cm 2.0,arrowlength=1.4,arrowinset=0.4]{->}(3.68,-1.04)(1.7,0.72)
\psline[linewidth=0.04cm,linestyle=dashed,dash=0.16cm 0.16cm,arrowsize=0.05291667cm 2.0,arrowlength=1.4,arrowinset=0.4]{<-}(3.68,0.68)(3.66,-1.0)
\rput(3.45,-0.475){60$^\circ$}
\psline[linewidth=0.04cm,linestyle=dashed,dash=0.16cm 0.16cm,arrowsize=0.05291667cm 2.0,arrowlength=1.4,arrowinset=0.4]{<-}(1.7,0.66)(3.68,0.64)
\rput{89.425285}(1.1057717,-1.7169194){\rput(1.42,-0.28){\footnotesize $v_{1i}$=5 ms$^-1$}}
\rput{-44.034668}(1.1511331,1.7666745){\rput(2.76,-0.52){\footnotesize $v_{2i}$=8 ms$^-1$}}
\rput(2.66,0.88){\footnotesize $v_{2xi}$}
\rput{-89.722336}(4.081485,3.541312){\rput(3.82,-0.26){\footnotesize $v_{2yi}$}}
\end{pspicture} 
}
\end{center}
We also know that $m_{1}=m_{2}=90$ kg and $v_{f2}$ = 0 ms$^{-1}$. \\
We need to find the final velocity and angle at which Player 1 bounces off Player 2.

\westep{Use conservation of momentum to solve the problem. First find the initial total momentum:}
Total initial momentum = Total final momentum. 
But we have a two dimensional problem, and we need to break up the initial momentum into $x$ and $y$ components. 
\begin{eqnarray*}
p_{ix} &=& p_{fx}\\
p_{iy} &=& p_{fy}
\end{eqnarray*}

For Player 1:
\begin{eqnarray*}
p_{ix1} &=& m_{1}v_{i1x} = 90 \times 0 = 0 \\
p_{iy1} &=& m_{1}v_{i1y} = 90 \times 5 
\end{eqnarray*}

For Player 2:
\begin{eqnarray*}
p_{ix2} &=& m_{2}v_{i2x} = 90 \times 8\times \sin{60^{\circ}} \\
p_{iy2} &=& m_{2}v_{i2y} = 90 \times 8\times \cos{60^{\circ}} 
\end{eqnarray*}

\westep{Now write down what we know about the final momentum:}
For Player 1:
\begin{eqnarray*}
p_{fx1} &=& m_{1}v_{fx1} = 90 \times v_{fx1}\\
p_{fy1} &=& m_{1}v_{fy1} = 90 \times v_{fy1}
\end{eqnarray*}

For Player 2:
\begin{eqnarray*}
p_{fx2} &=& m_{2}v_{fx2} = 90 \times 0 = 0 \\
p_{fy2} &=& m_{2}v_{fy2} = 90 \times 0 = 0
\end{eqnarray*}

\westep{Use conservation of momentum:}
The initial total momentum in the $x$ direction is equal to the final total momentum in the $x$ direction.\\
The initial total momentum in the $y$ direction is equal to the final total momentum in the $y$ direction.\\
If we find the final $x$ and $y$ components, then we can find the final $total$ momentum.

\begin{eqnarray*}
p_{ix1} + p_{ix2} &=& p_{fx1} + p_{fx2} \\
0 + 90 \times 8\times \sin{60^{\circ}} &=& 90 \times v_{fx1} + 0 \\
v_{fx1} &=& \frac{90 \times 8\times \sin{60^{\circ}} }{90} \\
v_{fx1} &=& 6.928 \rm{ms^{-1}}
\end{eqnarray*}

\begin{eqnarray*}
p_{iy1} + p_{iy2} &=& p_{fy1} + p_{fy2} \\
90\times 5 + 90\times 8\times \cos{60^{\circ}} &=& 90\times v_{fy1} + 0 \\
v_{fy1} &=& \frac{90\times 5 + 90\times 8\times \cos{60^{\circ}}}{90} \\
v_{fy1} &=& 9.0 \rm{ms^{-1}}
\end{eqnarray*}

\westep{Using the $x$ and $y$ components, calculate the final total $v$}
Use Pythagoras's theorem to find the total final velocity:

\begin{center}
\scalebox{1} % Change this value to rescale the drawing.
{
\begin{pspicture}(0,-1.58)(2.561371,1.56)
\psline[linewidth=0.04cm,arrowsize=0.05291667cm 2.0,arrowlength=1.4,arrowinset=0.4]{->}(2.541371,-1.14)(0.30137104,-1.14)
\psline[linewidth=0.04cm,arrowsize=0.05291667cm 2.0,arrowlength=1.4,arrowinset=0.4]{->}(0.36137104,-1.16)(0.36137104,1.54)
\psline[linewidth=0.04cm,linestyle=dashed,dash=0.16cm 0.16cm,arrowsize=0.05291667cm 2.0,arrowlength=1.4,arrowinset=0.4]{->}(2.541371,-1.16)(0.36137104,1.48)
\rput(2.0013711,-0.9){\footnotesize $\theta$}
\rput(1.4513711,-1.36){\footnotesize $v_{fx1}$}
\rput{89.053925}(0.14854419,-0.19101743){\rput(0.17137103,0.0){\footnotesize $v_{fy1}$}}
\rput{-52.768417}(0.4449666,1.4970001){\rput(1.731371,0.32){\footnotesize $v_{ftot}$}}
\end{pspicture} 
}
\end{center}

\begin{eqnarray*}
v_{ftot} &=& \sqrt{v_{fx1}^{2} + v_{fy1}^{2}}\\
&=& \sqrt{6.928^{2} + 9^{2}} \\
&=& 11.36
\end{eqnarray*}

Calculate the angle $\theta$ to find the direction of Player 1's final velocity:
\begin{eqnarray*}
\sin{\theta} &=& \frac{v_{fy1}}{v_{ftot}}\\
\theta &=& 52.4^{\circ}
\end{eqnarray*}

Therefore Player 1 bounces off Player 2 with a final velocity of 11.36 m$\cdot \rm{s}^{-1}$ at an angle of 52.4$^\circ$ from the horizontal.
}
\end{wex}



\begin{wex}{2D Conservation of Momentum: II}
{In a soccer game, Player 1 is running with the ball at 5~m$\cdot$s$^{-1}$ across the pitch at an angle of $75^{\circ}$ from the horizontal. Player 2 runs towards Player 1 at 6~m$\cdot$s$^{-1}$ an angle of $60^{\circ}$ to the horizontal and tackles Player 1. In the tackle, the two players bounce off each other. Player 2 moves off with a velocity in the opposite $x$-direction of 0.3 m$\cdot$s$^{-1}$ and a velocity in the $y$-direction of 6 m$\cdot$s$^{-1}$.
Both the players have a mass of 80 kg. 
What is the final total velocity of Player 1?  }
{
\westep{Identify what is required and what is given}
The first step is to draw the picture to work out what the situation is. Mark the initial velocities of both players in the picture.


\begin{center}
\scalebox{1} % Change this value to rescale the drawing.
{
\begin{pspicture}(0,-1.85)(4.225025,1.83)
\psline[linewidth=0.04cm,arrowsize=0.05291667cm 2.0,arrowlength=1.4,arrowinset=0.4]{->}(0.18502514,-1.43)(1.4050251,1.81)
\psline[linewidth=0.04cm,linestyle=dashed,dash=0.16cm 0.16cm,arrowsize=0.05291667cm 2.0,arrowlength=1.4,arrowinset=0.4]{->}(0.22502513,-1.41)(1.4050251,-1.41)
\psline[linewidth=0.04cm,linestyle=dashed,dash=0.16cm 0.16cm,arrowsize=0.05291667cm 2.0,arrowlength=1.4,arrowinset=0.4]{->}(1.3650252,-1.41)(1.3850251,1.77)
\psline[linewidth=0.04cm,arrowsize=0.05291667cm 2.0,arrowlength=1.4,arrowinset=0.4]{<-}(1.3850251,1.79)(4.185025,-0.21)
\psline[linewidth=0.04cm,linestyle=dashed,dash=0.16cm 0.16cm,arrowsize=0.05291667cm 2.0,arrowlength=1.4,arrowinset=0.4]{->}(4.205025,-0.21)(1.4250251,-0.25)
\rput(0.6550251,-1.21){\footnotesize 75$^\circ$}
\rput(3.4550252,-0.03){\footnotesize 60$^\circ$}
\rput(0.70502514,-1.67){\footnotesize $v_{ix1}$}
\rput(2.825025,-0.43){\footnotesize $v_{ix2}$}
\rput{89.21085}(2.253403,-0.8646554){\rput(1.5650251,0.73){\footnotesize $v_{iy2}$}}
\rput{88.30284}(0.9400208,-1.3882858){\rput(1.1850251,-0.19){\footnotesize $v_{iy1}$}}
\rput{67.6102}(0.6611979,-0.36749542){\rput(0.6050251,0.33){\footnotesize $v_{i1}$=5 ms$^{-1}$}}
\rput{-35.449585}(0.0028357925,1.8688724){\rput(2.9250252,0.95){\footnotesize $v_{i2}$=6 ms$^{-1}$}}
\end{pspicture} 
}
\end{center}

We need to define a reference frame: For y, choose the direction they are both running in as positive. For x, the direction player 2 is running in is positive.

We also know that $m_{1}=m_{2}=80$ kg. And $v_{fx2}$=-0.3 ms$^{-1}$ and $v_{fy2}$=6 ms$^{-1}$. \\
We need to find the final velocity and angle at which Player 1 bounces off Player 2.

\westep{Use conservation of momentum to solve the problem. First find the initial total momentum:}
Total initial momentum = Total final momentum. 
But we have a two dimensional problem, and we need to break up the initial momentum into $x$ and $y$ components. Remember that momentum is a vector and has direction which we will indicate with a '+' or '-' sign.
\begin{eqnarray*}
p_{ix} &=& p_{fx}\\
p_{iy} &=& p_{fy}
\end{eqnarray*}

For Player 1:
\begin{eqnarray*}
p_{ix1} &=& m_{1}v_{i1x} = 80 \times (-5) \times \cos{75^{\circ}} \\
p_{iy1} &=& m_{1}v_{i1y} = 80 \times 5 \times \sin{75^{\circ}}
\end{eqnarray*}

For Player 2:
\begin{eqnarray*}
p_{ix2} &=& m_{2}v_{i2x} = 80 \times 6\times \cos{60^{\circ}} \\
p_{iy2} &=& m_{2}v_{i2y} = 80 \times 6\times \sin{60^{\circ}} 
\end{eqnarray*}

\westep{Now write down what we know about the final momentum:}
For Player 1:
\begin{eqnarray*}
p_{fx1} &=& m_{1}v_{fx1} = 80 \times v_{fx1}\\
p_{fy1} &=& m_{1}v_{fy1} = 80 \times v_{fy1}
\end{eqnarray*}

For Player 2:
\begin{eqnarray*}
p_{fx2} &=& m_{2}v_{fx2} = 80 \times (-0.3) \\
p_{fy2} &=& m_{2}v_{fy2} = 80 \times 6 
\end{eqnarray*}

\westep{Use conservation of momentum:}
The initial total momentum in the $x$ direction is equal to the final total momentum in the $x$ direction.\\
The initial total momentum in the $y$ direction is equal to the final total momentum in the $y$ direction.\\
If we find the final $x$ and $y$ components, then we can find the final $total$ momentum.

\begin{eqnarray*}
p_{ix1} + p_{ix2} &=& p_{fx1} + p_{fx2} \\
-80 \times 5\cos{75^{\circ}} + 80 \times 6 \times \cos{60^{\circ}} &=& 80 \times v_{fx1} + 80\times(-0.3) \\
v_{fx1} &=& \frac{-80 \times 5\cos{75^{\circ}} + 80 \times 6 \times \cos{60^{\circ}}}{80} \\ 
&\phantom{=}&-\dfrac{80\times(-0.3) }{80} \\
v_{fx1} &=& 2.0~ \rm{ms^{-1}}
\end{eqnarray*}

\begin{eqnarray*}
p_{iy1} + p_{iy2} &=& p_{fy1} + p_{fy2} \\
80\times 5\times \sin{75^{\circ}} + 80\times 6 \times \sin{60^{\circ}} &=& 80\times v_{fy1} + 80\times 6 \\
v_{fy1} &=& \frac{80\times 5\sin{75^{\circ}} + 80\times 6 \times \sin{60^{\circ}}}{80} \\
& & -\dfrac{80 \times 6 }{80}\\
v_{fy1} &=& 4.0~ \rm{ms^{-1}}
\end{eqnarray*}

\westep{Using the $x$ and $y$ components, calculate the final total $v$}
Use Pythagoras's theorem to find the total final velocity:

\begin{center}
\scalebox{1} % Change this value to rescale the drawing.
{
\begin{pspicture}(0,-1.58)(2.561371,1.56)
\psline[linewidth=0.04cm,arrowsize=0.05291667cm 2.0,arrowlength=1.4,arrowinset=0.4]{->}(2.541371,-1.14)(0.30137104,-1.14)
\psline[linewidth=0.04cm,arrowsize=0.05291667cm 2.0,arrowlength=1.4,arrowinset=0.4]{->}(0.36137104,-1.16)(0.36137104,1.54)
\psline[linewidth=0.04cm,linestyle=dashed,dash=0.16cm 0.16cm,arrowsize=0.05291667cm 2.0,arrowlength=1.4,arrowinset=0.4]{->}(2.541371,-1.16)(0.36137104,1.48)
\rput(2.0013711,-0.9){\footnotesize $\theta$}
\rput(1.4513711,-1.36){\footnotesize $v_{fx1}$}
\rput{89.053925}(0.14854419,-0.19101743){\rput(0.17137103,0.0){\footnotesize $v_{fy1}$}}
\rput{-52.768417}(0.4449666,1.4970001){\rput(1.731371,0.32){\footnotesize $v_{ftot}$}}
\end{pspicture} 
}
\end{center}

\begin{eqnarray*}
v_{ftot} &=& \sqrt{v_{fx1}^{2} + v_{fy1}^{2}}\\
&=& \sqrt{2^{2} + 4^{2}} \\
&=& 4.5
\end{eqnarray*}

Calculate the angle $\theta$ to find the direction of Player 1's final velocity:
\begin{eqnarray*}
\tan{\theta} &=& \frac{v_{fy1}}{v_{fx1}}\\
\theta &=& 63.4^{\circ}
\end{eqnarray*}

Therefore Player 1 bounces off Player 2 with a final velocity of 4.5 m$\cdot s^{-1}$ at an angle of 63.4$^\circ$ from the horizontal.
}
\end{wex}





%\Exercise{Conservation of Momentum in Two Dimensions}{
%\begin{enumerate}
%\item{\nts{Exercises are needed. Problems should involve conservation of momentum in both the x direction and the y %direction. Problems should also involve impulse and momentum when the applied force is in the horizontal or vertical %direction.}}
%\end{enumerate}
%}


\section{Types of Collisions}
\label{pc:types}
Two types of collisions are of interest:

\begin{itemize}
\item elastic collisions
\item inelastic collisions
\end{itemize}

In both types of collision, total momentum is \emph{always} conserved. Kinetic energy is conserved for elastic collisions, but not for inelastic collisions.

\subsection{Elastic Collisions}

\Definition{Elastic Collisions}{An elastic collision is a collision where total momentum and total kinetic energy are both conserved.}

This means that in an elastic collision the total momentum \emph{and} the total kinetic energy before the collision is the same as after the collision. For these kinds of collisions, the kinetic energy is not changed into another type of energy.

\subsubsection{Before the Collision}
\label{pc:types:elast:before}
Figure~\ref{fig:pc:types:elast:before} shows two balls rolling toward each other, about to collide:
\begin{figure}[h!tbp]
\begin{center}
\begin{pspicture}(0,0)(8,2)
%\psgrid
\pscircle[fillstyle=crosshatch](0.5,1){.5}
\rput*(0.5,1){1}
\psline{->}(1,1)(3,1)
\rput(2,0.6){$p_{i1}$, $KE_{i1}$}
\pscircle[fillstyle=hlines](7.5,1){0.5}
\rput*(7.5,1){2}
\psline{->}(7,1)(5,1)
\rput(6, 0.6){$p_{i2}$, $KE_{i2}$}
\end{pspicture}
\caption{Two balls before they collide.}
\label{fig:pc:types:elast:before}
\end{center}
\end{figure}

Before the balls collide, the total momentum of the system is equal to all the individual momenta added together. Ball 1 has a momentum which we call $p_{i1}$ and ball 2 has a momentum which we call $p_{i2}$, it means the total momentum before the collision is:

\begin{equation*}
\label{eq:pc:types:elast:before:P}
p_{i} = p_{i1}+p_{i2}
\end{equation*}
We calculate the total kinetic energy of the system in the same way. Ball 1 has a kinetic energy which we call $\kener_{i1}$ and the ball 2 has a kinetic energy which we call \kener$_{i2}$, it means that the total kinetic energy
before the collision is:
\begin{equation*}
\label{eq:pc:types:elast:before:K}
\kener_{i} = \kener_{i1}+\kener_{i2}
\end{equation*}

\subsubsection{After the Collision}
\label{pc:types:elast:after}
Figure~\ref{fig:pc:types:elast:after} shows two balls after they have collided:
\begin{figure}[h!tbp]
\begin{center}
\begin{pspicture}(0,0)(8,2)
\pscircle[fillstyle=crosshatch](3.5,1){.5}
\rput*(3.5,1){1}
\psline{<-}(1,1)(3,1)
\rput(2,0.6){$p_{f1}$, $\kener_{f1}$}
\pscircle[fillstyle=hlines](4.5,1){0.5}
\rput*(4.5,1){2}
\psline{<-}(7,1)(5,1)
\rput(6, 0.6){$p_{f2}$, $\kener_{f2}$}
\end{pspicture}
\caption{Two balls after they collide.}
\label{fig:pc:types:elast:after}
\end{center}
\end{figure}

After the balls collide and bounce off each other, they have new momenta and new kinetic energies. Like before, the total momentum of the system is equal to all the individual momenta added together. Ball 1 now has a momentum
which we call $p_{f1}$ and ball 2 now has a momentum which we call $p_{f2}$, it means the total momentum after the collision is
\begin{equation*}
\label{eq:pc:types:elast:after:P}
p_{f} = p_{f1}+p_{f2}
\end{equation*}
Ball 1 now has a kinetic energy which we call $\kener_{f1}$ and ball 2 now has a kinetic energy which we call $\kener_{f2}$, it means that the total kinetic energy after the collision is:
\begin{equation*}
\label{eq:pc:types:elast:after:K}
\kener_{f} = \kener_{f1}+\kener_{f2}
\end{equation*}
Since this is an \emph{elastic} collision, the total momentum before the collision equals the total momentum after the collision \textbf{and} the total kinetic energy before the collision equals the total kinetic energy after the collision. Therefore:
\begin{eqnarray*}
\nonumber
\mbox{Initial}&&\mbox{Final}\\
\label{eq:pc:types:elast:conserveP}
p_{i} &=& p_{f}\\\nonumber
p_{i1}+p_{i2} &=& p_{f1}+p_{f2} \\\nonumber
&\rm{\textbf{and}}&\\
\label{eq:pc:types:elast:conserveK}
\kener_{i} &=& \kener_{f} \\\nonumber
\kener_{i1}+\kener_{i2} &=& \kener_{f1}+\kener_{f2}
\end{eqnarray*}

\begin{wex}{An Elastic Collision}{Consider a collision between two pool balls. Ball 1 is at rest and ball 2 is moving towards it with a speed of 2 \ms. The mass of each ball is 0.3 kg. After the balls collide \emph{elastically}, ball 2 comes to an immediate stop and ball 1 moves off. What is the final velocity of ball 1?}{

\westep{Determine how to approach the problem}

We are given:
\begin{itemize}
\item mass of ball 1, $m_1$ = 0.3 kg
\item mass of ball 2, $m_2$ = 0.3 kg
\item initial velocity of ball 1, $v_{i1}$ = 0 \ms
\item initial velocity of ball 2, $v_{i2}$ = 2 \ms
\item final velocity of ball 2, $v_{f2}$ = 0 \ms
\item the collision is elastic
\end{itemize}

All quantities are in SI units. We are required to determine the final velocity of ball 1, $v_{f1}$. Since the collision is elastic, we know that 
\begin{itemize}
\item momentum is conserved, $m_1v_{i1}+m_2v_{i2}=m_1v_{f1}+m_2v_{f2}$
\item energy is conserved, $\frac{1}{2}(m_1v_{i1}^2+m_2v_{i2}^2)=\frac{1}{2}(m_1v_{f1}^2+m_2v_{f2}^2)$
\end{itemize}

\westep{Choose a frame of reference}
Choose to the right as positive.

\westep{Draw a rough sketch of the situation}
\begin{center}
\begin{pspicture}(0,-0.5)(9,1.5)
\pscircle[fillstyle=solid](.5,1){.5}
\rput(0.5,1){2}
\psline{->}(1,1)(1.5,1)
\uput[d](0.5,0.6){$m_2$, $v_{i2}$}
\pscircle[fillstyle=solid](2.5,1){.5}
\rput(2.5,1){1}
\uput[d](2.5,0.6){$m_1$, $v_{i1}$}
\rput(5,0){\pscircle[fillstyle=solid](.5,1){.5}
\rput(0.5,1){2}
\uput[d](0.5,0.6){$m_2$, $v_{f2}$}
\pscircle[fillstyle=solid](2.5,1){.5}
\rput(2.5,1){1}
\uput[d](2.5,0.6){$m_1$, $v_{f1}$}
\psline{->}(3,1)(3.5,1)}
\uput[d](1.5,0){Before collision}
\uput[d](6.5,0){After collision}
\end{pspicture}
\end{center}

\westep{Solve the problem}
Momentum is conserved. Therefore:
\begin{eqnarray*}
p_i&=&p_f \nonumber\\
m_{1}v_{i1} + m_{2}{v}_{i2} &=& m_{1}{v}_{f1}+m_{2}{v}_{f2}\\
(0,3)(0)+(0,3)(2)&=&(0,3)v_{f1}+0\\
v_{f1}&=&2\ems
\end{eqnarray*}

\westep{Write down the final answer}
The final velocity of ball 1 is 2 \ms to the right.}
\end{wex}

\begin{wex}{Another Elastic Collision}{Consider two 2 marbles. Marble 1 has mass 50~g and marble 2 has mass 100~g. Edward rolls marble 2 along the ground towards marble 1 in the positive $x$-direction. Marble 1 is initially at rest and marble 2 has a velocity of 3 \ms\ in the positive $x$-direction. After they collide \emph{elastically}, both marbles are moving. What is the final velocity of each marble?}
{\westep{Decide how to approach the problem}

We are given:
\begin{itemize}
\item mass of marble 1, $m_1$=50 g
\item mass of marble 2, $m_2$=100 g
\item initial velocity of marble 1, $v_{i1}$=0 \ms
\item initial velocity of marble 2, $v_{i2}$=3 \ms
\item the collision is elastic
\end{itemize}

The masses need to be converted to SI units.
\begin{eqnarray*}
m_{1} &=& 0,05\ \rm{kg}\\
m_{2} &=& 0,1\ \rm{kg}
\end{eqnarray*}

We are required to determine the final velocities:
\begin{itemize}
\item final velocity of marble 1, $v_{f1}$
\item final velocity of marble 2, $v_{f2}$
\end{itemize}

Since the collision is elastic, we know that
\begin{itemize}

\item momentum is conserved, $p_i=p_f$.
\item energy is conserved, $\kener_{i}$=$\kener_{f}$.
\end{itemize}

We have two equations and two unknowns ($v_1$, $v_2$) so it is a simple case of solving a set of simultaneous equations.

\westep{Choose a frame of reference}
Choose to the right as positive.

\westep{Draw a rough sketch of the situation}
\begin{center}
\scalebox{0.8}{
\begin{pspicture}(0,-2.2639062)(12.742812,2.2639062)
\pscircle[linewidth=0.04,dimen=outer](1.1809375,-0.02609375){0.85}
\pscircle[linewidth=0.04,dimen=outer](4.0209374,-0.02609375){0.36}
\pscircle[linewidth=0.04,dimen=outer](8.750937,-0.02609375){0.85}
\pscircle[linewidth=0.04,dimen=outer](11.600938,-0.02609375){0.36}
\psline[linewidth=0.04cm,arrowsize=0.05291667cm 3.0, arrowlength=2.0,arrowinset=0.4]{->}(2.0009375,-0.02609375)(3.3209374,-0.02609375)
\psline[linewidth=0.04cm,arrowsize=0.05291667cm 3.0, arrowlength=2.0,arrowinset=0.4]{->}(9.600938,-0.02609375)(10.480938,-0.02609375)
\psline[linewidth=0.04cm,arrowsize=0.05291667cm 3.0, arrowlength=2.0,arrowinset=0.4]{->}(11.9609375,-0.02609375)(12.580937,-0.02609375)
\usefont{T1}{ptm}{m}{n}
\rput(1.1995312,0.08390625){2}
\usefont{T1}{ptm}{m}{n}
\rput(8.759531,0.00390625){2}
\usefont{T1}{ptm}{m}{n}
\rput(4.005156,-0.00609375){\small 1}
\usefont{T1}{ptm}{m}{n}
\rput(11.582656,-0.01109375){\footnotesize 1}
\usefont{T1}{ptm}{m}{n}
\rput(2.4171875,2.0839062){Before Collision}
\usefont{T1}{ptm}{m}{n}
\rput(9.81875,2.0839062){After Collision}
\usefont{T1}{ptm}{m}{n}
\rput(1.1523438,-1.2760937){$m_2=100g$}
\usefont{T1}{ptm}{m}{n}
\rput(1.1623437,-2.0160937){$v_{i2}=3\ems$}
\usefont{T1}{ptm}{m}{n}
\rput(4.1423435,-1.1960938){$m_1=50g$}
\usefont{T1}{ptm}{m}{n}
\rput(4.4023438,-2.0360937){$v_{i1}=0$}
\usefont{T1}{ptm}{m}{n}
\rput(8.732344,-1.2360938){$m_2=100g$}
\usefont{T1}{ptm}{m}{n}
\rput(11.842343,-1.2560937){$m_1=50g$}
\end{pspicture}}
\end{center}


\westep{Solve the problem}

Momentum is conserved. Therefore:
\begin{eqnarray}
\nonumber
p_{i}&=&p_{f}\nonumber\\
p_{i1}+p_{i2}&=&p_{f1}+p_{f2}\nonumber\\
m_{1}v_{i1} + m_{2}{v}_{i2} &=& m_{1}{v}_{f1}+m_{2}{v}_{f2}\nonumber\\
(0,05)(0)+(0,1)(3)&=&(0,05)v_{f1}+(0,1)v_{f2}\nonumber\\
0,3&=&0,05v_{f1}+0,1v_{f2}
\label{eq:wex1}
\end{eqnarray}
Energy is also conserved. Therefore:
\begin{eqnarray}
\kener_{i}&=&\kener_{f}\nonumber\\
\kener_{i1}+\kener_{i2}&=&\kener_{f1}+\kener_{f2}\nonumber \\
\frac{1}{2}m_{1}v_{i1}^2 + \frac{1}{2}m_{2}{v}_{i2}^2 &=& \frac{1}{2}m_{1}{v}_{f1}^2+\frac{1}{2}m_{2}{v}_{f2}^2\nonumber\\
(\frac{1}{2})(0,05)(0)^2+(\frac{1}{2})(0,1)(3)^2&=&
\frac{1}{2}(0,05)(v_{f1})^2+(\frac{1}{2})(0,1)(v_{f2})^2\nonumber\\
0,45&=&0,025v_{f1}^2+0,05v_{f2}^2
\label{eq:wex2}
\end{eqnarray}

Substitute Equation~\ref{eq:wex1} into Equation~\ref{eq:wex2} and solve for $v_{f2}$.

\begin{eqnarray*}
m_{2}{v}_{i2}^2 &=& m_{1}{v}_{f1}^2+m_{2}{v}_{f2}^2\\
&=& m_{1} \left( \frac{m_2}{m_1}({v}_{i2} - {v}_{f2}) \right) ^2 +m_{2}{v}_{f2}^2\\
&=& m_{1} \frac{m_2^2}{m_1^2}\left({v}_{i2} - {v}_{f2} \right) ^2 +m_{2}{v}_{f2}^2\\
&=& \frac{m_2^2}{m_1}\left({v}_{i2} - {v}_{f2} \right) ^2 +m_{2}{v}_{f2}^2\\
{v}_{i2}^2&=& \frac{m_2}{m_1}\left({v}_{i2} - {v}_{f2} \right) ^2 +{v}_{f2}^2\\
&=&\frac{m_2}{m_1}\left({v}_{i2}^2 - 2\cdot v_{i2}\cdot v_{f2} + {v}_{f2}^2 \right) +{v}_{f2}^2\\
0&=&\left(\frac{m_2}{m_1}-1 \right){v}_{i2}^2 - 2 \frac{m_2}{m_1} v_{i2}\cdot v_{f2} + \left(\frac{m_2}{m_1}+1\right){v}_{f2}^2\\
&=&\left(\frac{0.1}{0.05}-1 \right)(3)^2 - 2 \frac{0.1}{0.05} (3)\cdot v_{f2} + \left(\frac{0.1}{0.05}+1\right){v}_{f2}^2\\
&=&(2-1)(3)^2 - 2 \cdot 2 (3)\cdot v_{f2} + (2+1){v}_{f2}^2\\
&=&9 - 12v_{f2} + 3{v}_{f2}^2\\
&=&3 - 4v_{f2} + {v}_{f2}^2\\
&=&(v_{f2}-3)(v_{f2}-1)
\end{eqnarray*}

Substituting back into Equation~\ref{eq:wex1}, we get:
\begin{eqnarray*}
{v}_{f1} &=& \frac{m_2}{m_1}({v}_{i2} - {v}_{f2})\\
&=& \frac{0.1}{0.05}(3 - 3)\\
&=&0\ \ems\\
&\mbox{or}&\\
{v}_{f1} &=& \frac{m_2}{m_1}({v}_{i2} - {v}_{f2})\\
&=& \frac{0.1}{0.05}(3 - 1)\\
&=&4\ \ems
\end{eqnarray*}
But according to the question, marble 1 is moving after the collision, therefore marble 1 moves to the right at 4 \ms.
Substituting this value for ${v}_{f1} $ into Equation~\ref{eq:wex1}, we can calculate:
\begin{eqnarray*}
v_{f2} & = &  \frac{0,3 - 0,05v_{f1}}{0,1} \\
& = & \frac{0,3 - 0,05\times4}{0,1} \\
& = & 1\ \ems
\end{eqnarray*}
Therefore marble 2 moves to the right with a velocity of 1 \ms.
 
 }
\end{wex}

\begin{wex}{Colliding Billiard Balls}
{Two billiard balls each with a mass of $150g$ collide head-on in an elastic collision. Ball 1 was travelling at a speed of $2\ems$ and ball 2 at a speed of $1,5 \ems$. After the collision, ball 1 travels away from ball 2 at a velocity of $1,5 \ems$.
\renewcommand{\labelenumii}{\alpha{enumii}}
\begin{enumerate}

\item {Calculate the velocity of ball 2 after the collision.}
\item {Prove that the collision was elastic. Show calculations.}
\end{enumerate}
}
{
\renewcommand{\labelenumii}{\alpha{enumii}}
\begin{enumerate}
\item {
\westep{Draw a rough sketch of the situation}

\begin{center}
\begin{pspicture}(0,-1.445)(12.58,1.485)
\pscircle[linewidth=0.04,dimen=outer](0.64,-0.805){0.64}
\pscircle[linewidth=0.04,dimen=outer](3.58,-0.805){0.64}
\pscircle[linewidth=0.04,dimen=outer](8.14,-0.805){0.64}
\pscircle[linewidth=0.04,dimen=outer](10.98,-0.805){0.64}
\psline[linewidth=0.04cm,arrowsize=0.05291667cm 3.0, arrowlength=2.0,arrowinset=0.4]{->}(1.28,-0.805)(2.02,-0.805)
\psline[linewidth=0.04cm,arrowsize=0.05291667cm 3.0, arrowlength=2.0,arrowinset=0.4]{->}(2.96,-0.805)(2.3,-0.805)
\psline[linewidth=0.04cm,arrowsize=0.05291667cm 3.0, arrowlength=2.0,arrowinset=0.4]{->}(7.54,-0.805)(6.1,-0.805)
\psline[linewidth=0.04cm,arrowsize=0.05291667cm 3.0, arrowlength=2.0,arrowinset=0.4]{->}(11.62,-0.805)(12.56,-0.80679995)
\usefont{T1}{ptm}{m}{n}
\rput(0.626875,-0.835){1}
\usefont{T1}{ptm}{m}{n}
\rput(3.5985937,-0.835){2}
\usefont{T1}{ptm}{m}{n}
\rput(8.086875,-0.835){1}
\usefont{T1}{ptm}{m}{n}
\rput(10.998593,-0.835){2}
\usefont{T1}{ptm}{m}{n}
\rput(2.49625,1.305){Before Collision}
\usefont{T1}{ptm}{m}{n}
\rput(9.517813,1.305){After Collision}
\usefont{T1}{ptm}{m}{n}
\rput(12.044531,-0.175){?}
\usefont{T1}{ptm}{m}{n}
\rput(6.851406,-0.255){$1,5\ems$}
\usefont{T1}{ptm}{m}{n}
\rput(1.1114062,0.165){$2\ems$}
\usefont{T1}{ptm}{m}{n}
\rput(3.5114062,0.185){$1,5\ems$}
\end{pspicture} 
\end{center}

\westep{Decide how to approach the problem}
Since momentum is conserved in all kinds of collisions, we can use conservation of momentum to solve for the velocity of ball 2 after the collision.
\westep{Solve problem}

\begin{eqnarray*}
p_{before}&=&p_{after}\\
m_1v_{i1}+m_2v_{i2}&=&m_1v_{f1}+m_2v_{f2}\\
\left( \frac{150}{1000}\right) (2)+\left( \frac{150}{1000} \right)(-1,5) &=& \left( \frac{150}{1000} \right)(-1,5)+\left( \frac{150}{1000} \right)(v_{f2})\\
0,3-0,225&=&-0,225+0,15v_{f2}\\
v_{f2}&=&3\ems
\end{eqnarray*}

So after the collision, ball 2 moves with a velocity of $3\ems$.
}
\item {


The fact that characterises an elastic collision is that the total kinetic energy of the particles before the collision is the same as the total kinetic energy of the particles after the collision. This means that if we can show that the initial kinetic energy is equal to the final kinetic energy, we have shown that the collision is elastic.

Calculating the initial total kinetic energy:

\begin{eqnarray*}
EK_{before}&=&\frac{1}{2}m_1v_{i1}^2+\frac{1}{2}m_2v_{i2}^2\\
&=&\left(\frac{1}{2}\right) (0,15)(2)^2+\left(\frac{1}{2}\right)(0,15)(-1,5)^2\\
&=&0.469....J
\end{eqnarray*}

Calculating the final total kinetic energy:

\begin{eqnarray*}
EK_{after}&=&\frac{1}{2}m_1v_{f1}^2+\frac{1}{2}m_2v_{f2}^2\\
&=&\left(\frac{1}{2}\right)(0,15)(-1,5)^2+\left(\frac{1}{2}\right) (0,15)(2)^2\\
&=&0.469....J
\end{eqnarray*}

So $EK_{before}=EK_{after}$ and hence the collision is elastic.
}
\end{enumerate}
}
\end{wex}


\subsection{Inelastic Collisions}
\Definition{Inelastic Collisions}{An inelastic collision is a collision in which \textbf{total momentum} is conserved but \textbf{total kinetic energy} is \emph{not} conserved. The kinetic energy is \emph{transformed} into other kinds of energy.}

So the total momentum before an inelastic collisions is the same as after the collision. But the total \emph{kinetic} energy before and after the inelastic collision is \emph{different}. Of course this does not mean that total energy has not been conserved, rather the energy has been \emph{transformed} into another type of energy.

As a rule of thumb, inelastic collisions happen when the colliding objects are distorted in some way. Usually they change their shape. The modification of the shape of an object requires energy and this is where the ``missing'' kinetic energy goes. A classic example of an inelastic collision is a motor car accident. The cars change shape and there is a noticeable change in the kinetic energy of the cars before and after the collision. This energy was used to bend the metal and deform the cars. Another example of an inelastic collision is shown in Figure~\ref{fig:p:m2d12:inelastic:example}.

\begin{figure}[htbp]
\begin{center}

\begin{pspicture}(-0.2,-0.4)(10,2.4)
\pscircle[fillcolor=gray,fillstyle=solid](0.75,1){0.75}
\psline{->}(1.5,1)(2,1)
\uput[u](0.75,1.75){$p_{im}$, $\kener_{im}$}
\pscircle[fillcolor=black,fillstyle=solid](5,1){0.15}
\psline{->}(4.85,1)(3.5,1)
\uput[u](5,1.75){$p_{ia}$, $\kener_{ia}$}
\uput[d](3,0.2){Before collision}
\pscircle[fillcolor=gray,fillstyle=solid](8,1){.75}
\pscircle[fillstyle=crosshatch](8,1){.75}
\pscircle[linecolor=white,fillstyle=solid](8.7,1){0.15}
\psarc{-}(8.7,1){0.2}{90}{270} 
\psline{->}(8.7,1)(10,1)
\uput[u](8,1.75){$p_{f}$, $\kener_{f}$}
\uput[d](8.1,0.2){After collision}
\end{pspicture}
\end{center}
\caption{Asteroid moving towards the Moon.}
\label{fig:p:m2d12:inelastic:example}
\end{figure}

An asteroid is moving through space towards the Moon. Before the asteroid crashes into the Moon, the total momentum of the system is:
\begin{equation*}
p_{i} = p_{im}+p_{ia}
\end{equation*}
The total kinetic energy of the system is:
\begin{equation*}
\kener_{i} = \kener_{im}+\kener_{ia}
\end{equation*}

When the asteroid collides \textbf{inelastically} with the Moon, its kinetic energy is transformed mostly into heat energy. If this heat energy is large enough, it can cause the asteroid and the area of the Moon's surface that it hits, to melt into liquid rock! From the force of impact of the asteroid, the molten rock flows outwards to form a crater on the Moon.

After the collision, the total momentum of the system will be the same as before. But since this collision is \emph{inelastic}, (and you can see that a change in the shape of objects has taken place!), total kinetic energy is \textbf{not} the same as before the collision.\\
Momentum is conserved:
\begin{equation*}
p_{i} = p_{f}
\end{equation*}
But the total kinetic energy of the system is not conserved:
\begin{equation*}
\kener_{i} \neq \kener_{f}
\end{equation*}

\begin{wex}{An Inelastic Collision}{Consider the collision of two cars. Car 1 is at rest and Car 2 is moving at a speed of 2 \ms\ to the left. Both cars each have a mass of 500~kg. The cars collide \emph{inelastically} and stick together. What is the resulting velocity of the resulting mass of metal?}{

\westep{Draw a rough sketch of the situation}
\begin{center}
\begin{pspicture}(0,-0.6)(11,1.4)
\def\car{\psframe(0,0.25)(2,1.25)
\psframe[fillstyle=solid,fillcolor=black](0.375,1.1)(0.625,1.3)
\psframe[fillstyle=solid,fillcolor=black](1.375,1.1)(1.625,1.3)
\psframe[fillstyle=solid,fillcolor=black](0.375,0.4)(0.625,0.2)
\psframe[fillstyle=solid,fillcolor=black](1.375,0.4)(1.625,0.2)}

\rput(0,0){\car}
\rput(1,0.75){Car 1}
\rput(1,0){$p_{i1}=0$}

\rput(3,0){\car}
\rput(4,0.75){Car 2}
\rput(4,0){$p_{i2}$}
\psline{<-}(2.5,0.75)(3,0.75)
\uput[d](2.5,0){Before collision}

\rput(7,0){\car}
\rput(8,0.75){Car 1}
\rput(9,0){\car}
\rput(10,0.75){Car 2}
\psline{<-}(6.5,0.75)(7,0.75)
\uput[d](9,0){After collision}
\uput[u](6.5,0.75){$p_{f}$}

\end{pspicture}
\end{center}

\westep{Determine how to approach the problem}
We are given:
\begin{itemize}
\item mass of car 1, $m_1$ = 500 kg
\item mass of car 2, $m_2$ = 500 kg
\item initial velocity of car 1, $v_{i1}$ = 0 \ms
\item initial velocity of car 2, $v_{i2}$ = 2 \ms to the left
\item the collision is inelastic
\end{itemize}

All quantities are in SI units. We are required to determine the final velocity of the resulting mass, $v_{f}$. 

Since the collision is inelastic, we know that
\begin{itemize}
\item momentum is conserved, $m_1v_{i1}+m_2v_{i2}=m_1v_{f1}+m_2v_{f2}$
\item kinetic energy is not conserved
\end{itemize}

\westep{Choose a frame of reference}
Choose to the left as positive.

\westep{Solve problem}
So we must use conservation of momentum to solve this problem. 

\begin{eqnarray*}
p_{i} & = & p_{f} \\
p_{i1} + p_{i2} &=& p_{f} \\
m_{1}v_{i1} + m_{2}v_{i2} &=& (m_{1} +m_{2})v_f \\
(500)(0) + (500)(2) &=& (500+500)v_f \\
1 000 &=& 1000 v_f \\
v_f &=& 1\ \ems
\end{eqnarray*}
Therefore, the final velocity of the resulting mass of cars is 1 \ms to the left.
}
\end{wex}
\clearpage
\begin{wex}{Colliding balls of clay}{
Two balls of clay, $200g$ each, are thrown towards each other according to the following diagram. When they collide, they stick together and move off together. All motion is taking place in the horizontal plane. Determine the velocity of the clay after the collision.
\begin{center}
\begin{pspicture}(0,-4.53)(10.631562,4.51)
\pscircle[linewidth=0.04,dimen=outer](6.94,3.64){0.87}
\psline[linewidth=0.04cm,arrowsize=0.05291667cm 3.0,arrowlength=2.0,arrowinset=0.4]{->}(6.94,2.79)(6.94,-1.19)
\rput(6.93625,3.705){\large 2}
\rput(5.001406,3.64){$200 g$}
\rput(7.9914064,1.34){$4\ems$}
\pscircle[linewidth=0.04,dimen=outer](0.87,-2.83){0.87}
\psline[linewidth=0.04cm,arrowsize=0.05291667cm 3.0,arrowlength=2.0,arrowinset=0.4]{->}(1.72,-2.83)(5.14,-2.83)
\rput(0.83734375,-2.855){\large 1}
\rput(0.94140625,-4.16){$200 g$}
\rput(3.1514063,-3.32){$3\ems$}
\psarc[linewidth=0.04](6.14,-2.77){0.84}{87.510445}{338.52322}
\psarc[linewidth=0.04](6.92,-2.25){0.84}{270.0}{160.01689}
\rput(6.525156,-2.535){\large 1+2}
\psline[linewidth=0.04cm,arrowsize=0.05291667cm 3.0,arrowlength=2.0,arrowinset=0.4]{->}(6.9,-3.09)(8.88,-4.51)
\psline[linewidth=0.04cm,arrowsize=0.05291667cm 3.5,arrowlength=2.4,arrowinset=0.4]{->}(10.44,-2.21)(10.44,2.09)
\rput(10.43875,2.645){\large N}
\rput(8.239062,-3.455){\large ?}
\end{pspicture} 
\end{center}
}{
\westep{Analyse the problem}
This is an inelastic collision where momentum is conserved.\\ 
The momentum before = the momentum after. \\
The momentum after can be calculated by drawing a vector diagram.
\westep{Calculate the momentum before the collision}
\begin{eqnarray*}
p_{1}\rm{(before)}=m_1v_{i1}=(0,2)(3)=0,6\ep\rm{east}\\
p_{2}\rm{(before)}=m_2v_{i2}=(0,2)(4)=0,8\ep\rm{south}
\end{eqnarray*}

\westep{Calculate the momentum after the collision.}
Here we need to draw a diagram:
\begin{center}
\begin{pspicture}(0,-1.9367187)(3.58625,1.9617188)
\psline[linewidth=0.03cm,arrowsize=0.05291667cm 3.0,arrowlength=2.0,arrowinset=0.4]{->}(0.686875,1.4782813)(3.186875,1.4782813)
\psline[linewidth=0.03cm,arrowsize=0.05291667cm 3.0,arrowlength=2.0,arrowinset=0.4]{->}(0.686875,1.4782813)(0.686875,-1.9217187)
\psline[linewidth=0.03cm,linestyle=dashed,dash=0.16cm 0.16cm](3.106875,-1.8817188)(3.126875,1.4982812)
\psline[linewidth=0.03cm,linestyle=dashed,dash=0.16cm 0.16cm](0.686875,-1.8817188)(3.106875,-1.8817188)
\psline[linewidth=0.04cm,arrowsize=0.05291667cm 4.0,arrowlength=2.4,arrowinset=0.4]{->}(0.706875,1.4782813)(3.106875,-1.8617188)
\usefont{T1}{ptm}{m}{n}
\rput(1.2682812,1.1682812){$\theta$}
\usefont{T1}{ptm}{m}{n}
\rput(1.9434375,1.7882812){0,6}
\usefont{T1}{ptm}{m}{n}
\rput(0.2,-0.39171875){0,8}
\usefont{T1}{ptm}{m}{n}
\rput(2.3176563,0.47828126){\small $p_{1+2}$(after)}
\psarc[linewidth=0.04](1.296875,1.0882813){0.39}{263.65982}{88.5312}
\end{pspicture}  
\end{center}
\begin{eqnarray*}
p_{1+2}\rm{(after)}&=&\sqrt{(0,8)^2+(0,6)^2}\\
&=&1
\end{eqnarray*}
\westep{Calculate the final velocity}
First we have to find the direction of the final momentum:
\begin{eqnarray*}
\tan \theta&=&\frac{0,8}{0,6}\\
\theta&=&53,13\degree
\end{eqnarray*}
Now we have to find the magnitude of the final velocity:\
\begin{eqnarray*}
p_{1+2}&=&m_{1+2}v_f\\
1&=&(0,2+0,2)v_f\\
v_f&=&2,5\ems \ \ 53,13\degree \rm{South \ of \  East}
\end{eqnarray*}
}
\end{wex}

\Exercise{Collisions}{
\renewcommand{\labelenumii}{\alph{enumii}}
\begin{enumerate}
\item{A truck of mass 4500~kg travelling at 20 \ms hits a car from behind. The car (mass 1000~kg) was travelling at 15 \ms. The two vehicles, now connected carry on moving in the same direction.
\begin{enumerate}
\item Calculate the final velocity of the truck-car combination after the collision.
\item Determine the kinetic energy of the system before and after the collision.
\item Explain the difference in your answers for b).
\item Was this an example of an elastic or inelastic collision? Give reasons for your answer.
\end{enumerate}}
\item{
Two cars of mass 900~kg each collide and stick together at an angle of 90$\degree$. Determine the final velocity of the cars if \\
car 1 was travelling at 15\ms and \\
car 2 was travelling at 20\ms.
\begin{center}
\begin{pspicture}(0,-3.95)(7.74,3.93)
\psframe[linewidth=0.03,dimen=outer](6.2,3.93)(5.02,2.15)
\psline[linewidth=0.04cm,arrowsize=0.05291667cm 3.0, arrowlength=2.0,arrowinset=0.4]{->}(5.62,2.15)(5.62,0.95)
\usefont{T1}{ptm}{m}{n}
\rput(5.61625,3.125){\large 2}
\usefont{T1}{ptm}{m}{n}
\rput(6.501406,1.68){$20\ems$}
\rput{90.0}(-0.41,-2.19){\psframe[linewidth=0.03,dimen=outer](1.48,-0.41)(0.3,-2.19)}
\psline[linewidth=0.04cm,arrowsize=0.05291667cm 3.0, arrowlength=2.0, arrowinset=0.4]{->}(1.78,-1.29)(2.9,-1.29)
\usefont{T1}{ptm}{m}{n}
\rput(2.7414062,-1.8){$15\ems$}
\usefont{T1}{ptm}{m}{n}
\rput(0.8773438,-1.235){\large 1}
\pspolygon[linewidth=0.03](4.98,-0.71)(3.84,-0.69)(3.84,-1.81)(6.18,-1.81)(6.16,0.31)(4.98,0.33)
\psline[linewidth=0.04cm,arrowsize=0.05291667cm 3.0, arrowlength=2.0, arrowinset=0.4]{->}(6.18,-1.81)(7.72,-3.93)
\usefont{T1}{ptm}{m}{n}
\rput(7.1390624,-2.495){\large ?}
\usefont{T1}{ptm}{m}{n}
\rput(5.325156,-1.135){\large 1+2}
\end{pspicture}
\end{center}
}
\end{enumerate}

% Automatically inserted shortcodes - number to insert 2
\par \practiceinfo
\par \begin{tabular}[h]{cccccc}
% Question 1
(1.)	01s4	&
% Question 2
(2.)	01s5	&
\end{tabular}
% Automatically inserted shortcodes - number inserted 2
}

\Extension{Tiny, Violent Collisions}{

\essayauthor[Thomas D. Gutierrez]

\essayauthorblurb[Tom Gutierrez received his Bachelor of Science and
Master degrees in Physics
from San Jose State University in his home town of San Jose, California. As
a Master's student he helped work on a laser spectrometer at NASA Ames
Research Centre. The instrument measured the ratio of different isotopes of
carbon in CO$_2$ gas and could be used for such diverse applications as
medical diagnostics and space exploration. Later, he received his PhD in
physics from the University of California, Davis where he performed
calculations for various reactions in high energy physics collisions. He
currently lives in Berkeley, California where he studies proton-proton
collisions seen at the STAR experiment at Brookhaven National Laboratory on
Long Island, New York.]

\essaytitle[High Energy Collisions]
Take an orange and expand it to the size of the earth. The atoms of the
earth-sized orange would themselves be about the size of regular oranges and
would fill the entire ``earth-orange''. Now, take an atom and expand it to
the size of a football field. The nucleus of that atom would be about the
size of a tiny seed in the middle of the field. From this analogy, you can
see that atomic nuclei are very small objects by human standards. They are
roughly $10^{-15}$ meters in diameter -- one-hundred thousand times smaller
than a typical atom. These nuclei cannot be seen or studied via any
conventional means such as the naked eye or microscopes. So how do
scientists study the structure of very small objects like atomic nuclei?

The simplest nucleus, that of hydrogen, is called the proton. Faced with
the inability to isolate a single proton, open it up, and directly examine
what is inside, scientists must resort to a brute-force and somewhat
indirect means of exploration: high energy collisions. By colliding protons
with other particles (such as other protons or electrons) at very high
energies, one hopes to learn about what they are made of and how they work.
The American physicist Richard Feynman once compared this process to
slamming delicate watches together and figuring out how they work by only
examining the broken debris. While this analogy may seem pessimistic, with
sufficient mathematical models and experimental precision, considerable
information can be extracted from the debris of such high energy subatomic
collisions. One can learn about both the nature of the forces at work and
also about the sub-structure of such systems.

The experiments are in the category of ``high energy physics'' (also known
as ``subatomic'' physics). The primary tool of scientific exploration in
these experiments is an extremely violent collision between two very, very
small subatomic objects such as nuclei. As a general rule, the higher the
energy of the collisions, the more detail of the original system you are
able to resolve. These experiments are operated at laboratories such as
CERN, SLAC, BNL, and Fermilab, just to name a few. The giant machines that
perform the collisions are roughly the size of towns. For example, the RHIC
collider at BNL is a ring about 1 km in diameter and can be seen from space.
The newest machine currently being built, the LHC at CERN, is a ring 9 km in
diameter!}

\Activity{Casestudy}{Atoms and its Constituents}
{\textbf{Questions:}
\begin{enumerate}
\item{What are isotopes?  (2)}
\item{What are atoms made up of?  (3)}
\item{Why do you think protons are used in the experiments and not atoms like carbon?  (2)}
\item {Why do you think it is necessary to find out what atoms are made up of and how they behave during collisions?  (2)}
\item{Two protons (mass $1,67\times 10^{-27}$~kg) collide and somehow stick together after the collision. If each proton travelled with an initial velocity of $5,00\times 10^{7}\ems$ and they collided at an angle of 90$\degree$, what is the velocity of the combination after the collision.   (9)}
\end{enumerate}
%\nts{The answers to these questions have been commented out in the code.}


%\comment{

%\textbf{Answers:}\\
%\begin{enumerate}
%\item Isotopes are atoms of the same element but with different number of neutrons.
%\item Protons, neutrons and electrons. (accept other subatomic particles as well)
%\item Carbon atoms have other particles like neutrons and electrons around that will interfere with the experiments.\\ %Protons are very small particles that can be accelerated easily.

%\item Any logical explanation.
%\item 
%\begin{center}
%\begin{pspicture}(0,-1.9542187)(3.8203125,1.9792187)
%\psline[linewidth=0.03cm,arrowsize=0.05291667cm 3.0, arrowlength=2.0,arrowinset=0.4]{->}(0.9209375,1.4607812)(3.4209375,1.4607812)
%\psline[linewidth=0.03cm,arrowsize=0.05291667cm 3.0, arrowlength=2.0,arrowinset=0.4]{->}(0.9209375,1.4607812)(0.9209375,-1.9392188)
%\psline[linewidth=0.03cm,linestyle=dashed,dash=0.16cm 0.16cm](3.3409376,-1.8992188)(3.3609376,1.4807812)
%\psline[linewidth=0.03cm,linestyle=dashed,dash=0.16cm 0.16cm](0.9209375,-1.8992188)(3.3409376,-1.8992188)
%\psline[linewidth=0.04cm,arrowsize=0.05291667cm 4.0, arrowlength=2.4,arrowinset=0.4]{->}(0.9409375,1.4607812)(3.3409376,-1.8792187)
%\usefont{T1}{ptm}{m}{n}
%\rput(1.5023438,1.1507813){$\theta$}
%\usefont{T1}{ptm}{m}{n}
%\rput(2.2323437,1.7907813){$p_1$}
%\usefont{T1}{ptm}{m}{n}
%\rput(0.41234374,-0.36921874){$p_2$}
%\usefont{T1}{ptm}{m}{n}
%\rput(2.5517187,0.46078125){\small $p_{1+2}$(after)}
%\psarc[linewidth=0.04](1.5309376,1.0707812){0.39}{263.65982}{88.5312}
%\end{pspicture}
%\end{center}
%\begin{eqnarray*}
%p\rm{after}&=&p_1+p_2\\
%&=&2p_1\\
%&=&(2)m_1v_{i1}\\
%&=&(2)(1,67\times 10^{-27})(5\times 10^7)\\
%&=&1,67\times 10^{-19}
%\end{eqnarray*}
%\begin{eqnarray*}
%p&=&mv_f\\
%&=&(1,67\times 10^{-27}+1,67\times 10^{-27})v_f\\
%v_f&=&5\times 10^7 \ems
%\end{eqnarray*}
%\begin{eqnarray*}
%\tan \theta&=&\frac{p_1}{p_2}\\
%&=&\frac{p_1}{p_1}\\
%&=&=1\\
%\theta&=&45\degree
%\end{eqnarray*}
%\end{enumerate}

%}
}


\section{Frames of Reference}
%\begin{syllabus}
%\item The learner must be able to define a frame of reference
%\item The learner must be able to give examples of the importance of specifying the frame of reference
%\item The learner must be able to define relative velocity
%\item The learner must be able to specify the velocity of an object relative to different frames of reference, e.g. for a person walking inside a train give the velocity relative to the train and relative to the ground.
%\item The learner must be able to use vectors to find the velocity of an object that moves relative to something else that is itself moving, e.g. if the velocity of a bird relative to the air is ba and of the air relative to the ground is ag then the velocity of the bird relative to the ground is bg= ba + ag
%\end{syllabus}

\subsection{Introduction}

\MarginCompass
\begin{figure}[htbp]
\begin{center}
\begin{pspicture}(0,0)(8,4)
\def\car{\psset{unit=0.75}\psframe[fillcolor=white,fillstyle=solid](0,0.25)(2,1.25)
\psframe[fillstyle=solid,fillcolor=black](0.375,1.1)(0.625,1.3)
\psframe[fillstyle=solid,fillcolor=black](1.375,1.1)(1.625,1.3)
\psframe[fillstyle=solid,fillcolor=black](0.375,0.4)(0.625,0.2)
\psframe[fillstyle=solid,fillcolor=black](1.375,0.4)(1.625,0.2)}
%\psgrid
\psframe[fillstyle=solid,fillcolor=black](0,1)(8,3)
\psline[linecolor=white,linewidth=3pt](0,2)(1,2)
\rput(2,0){\psline[linecolor=white,linewidth=3pt](0,2)(1,2)}
\rput(4,0){\psline[linecolor=white,linewidth=3pt](0,2)(1,2)}
\rput(6,0){\psline[linecolor=white,linewidth=3pt](0,2)(1,2)}
\rput(5,1){\car}
\psline[linecolor=white,linewidth=2pt]{->}(5,1.5)(4,1.5)
\psdots[dotsize=6pt](3,0.5)(3,3.5)
\uput[d](3,0.5){A}
\uput[u](3,3.5){B}
\psline{<->}(4,0.5)(2,0.5)
\psline{<->}(4,3.5)(2,3.5)
\uput[r](4,0.5){A's right}
\uput[l](2,0.5){A's left}
\uput[r](4,3.5){B's left}
\uput[l](2,3.5){B's right}
\end{pspicture}
\caption{Top view of a road with two people standing on opposite sides. A car drives past.}
\end{center}
\end{figure}

Consider two people standing, facing each other on either side of a road. A car drives past them, heading West. For the person facing South, the car was moving toward the right. However, for the person facing North, the car was moving toward the left. This discrepancy is due to the fact that the two people used two different \textit{frames of reference} from which to investigate this system. If each person were asked in what direction the car were moving, they would give a different answer. The answer would be relative to their frame of reference.

\subsection{What is a \textit{frame of reference}?}
\Definition{Frame of Reference}{
A \textit{frame of reference} is the point of view from which a system is observed.}
 In practical terms, a frame of reference is a set of axes (specifying directions) with an origin. An observer can then measure the position and motion of all points in a system, as well as the orientation of objects in the system relative to the frame of reference.

There are two types of reference frames: inertial and non-inertial. An inertial frame of reference travels at a constant velocity, which means that Newton's first law (inertia) holds true. A non-inertial frame of reference, such as a moving car or a rotating carousel, accelerates. Therefore, Newton's first law does not hold true in a non-inertial reference frame, as objects appear to accelerate without the appropriate forces.

\subsection{Why are frames of reference important?}
Frames of reference are important because (as we have seen in the introductory example) the velocity of a car can differ depending on which frame of reference is used.

\Extension{Frames of Reference and Special Relativity}{Frames of reference are especially important in special relativity, because when a frame of reference is moving at some significant fraction of the speed of light, then the flow of time in that frame does not necessarily apply in another reference frame. The speed of light is considered to be the only true constant between moving frames of reference.}
The next worked example will explain this.
\subsection{Relative Velocity}
The velocity of an object is frame dependent. More specifically, the perceived velocity of an object depends on the velocity of the observer. For example, a person standing on shore would observe the velocity of a boat to be different than a passenger on the boat. 

\begin{wex}{Relative Velocity}{The speedometer of a motor boat reads 5 \ms. The boat is moving East across a river which has a current travelling 3 \ms\ North. What would the velocity of the motor boat be according to an observer on the shore?}
{\westep{First, draw a diagram showing the velocities involved.}
%\MarginCompass
\begin{center}
\begin{pspicture}(-4.6,-2)(2,1)
%\psgrid[gridcolor=lightgray]
\psline[arrowscale=2]{->}(-3.0,-2.0)(-3.0,0)
\psline[arrowscale=2]{->}(-3.0,0)(2.0,0)
%\psline[arrowscale=2]{->}(-3.0,-2.0)(2.0,0)
\rput(-4.0,-1.0){3 \ms}
\rput(-0.5,0.3){5 \ms}
%\rput(-0.5,-1.5){5.8 \ms}
\end{pspicture}
\end{center}

\westep{Use the Theorem of Pythagoras to solve for the resultant of the two velocities.}

\begin{eqnarray*}
R & =& \sqrt{(3)^2 + (5)^2} \\
& = & \sqrt{34} \\
& = & 5,8\ \ems
\end{eqnarray*}
\begin{eqnarray*}
\tan \theta &=&\frac{5}{3}\\
\theta&=&59,04\degree
\end{eqnarray*}
%\MarginCompass
\begin{center}
\begin{pspicture}(0,-1.2341111)(6.139424,1.26)
\psline[linewidth=0.028222222cm,arrowsize=0.05291667cm 2.0,arrowlength=1.4,arrowinset=0.4]{->}(1.1253124,-1.22)(1.1253124,0.78)
\psline[linewidth=0.028222222cm,arrowsize=0.05291667cm 2.0,arrowlength=1.4,arrowinset=0.4]{->}(1.1253124,0.78)(6.1253123,0.78)
\psline[linewidth=0.028222222cm,arrowsize=0.05291667cm 2.0,arrowlength=1.4,arrowinset=0.4]{->}(1.1253124,-1.22)(6.1253123,0.78)
\usefont{T1}{ptm}{m}{n}
\rput(0.37171876,-0.22){3 \ms}
\usefont{T1}{ptm}{m}{n}
\rput(3.8704689,1.08){5 \ms}
\usefont{T1}{ptm}{m}{n}
\rput(3.8704689,-0.72){5,8 \ms}
\usefont{T1}{ptm}{m}{n}
\rput(1.5014062,-0.7415625){$\theta$}
\end{pspicture}  
\end{center}
The observer on the shore sees the boat moving with a velocity of 5,8 \ms\ at 59,04$\degree$east of north due to the current pushing the boat perpendicular to its velocity. This is contrary to the perspective of a passenger on the boat who perceives the velocity of the boat to be 5 \ms\ due East. Both perspectives are correct as long as the frame of the observer is considered.}\end{wex}

\begin{wex}{Relative Velocity 2}{It takes a man 10 seconds to ride down an escalator. It takes the same man 15 s to walk back up the escalator against its motion. How long will it take the man to walk down the escalator at the same rate he was walking before?}
{\westep{Identify what is required and what is given}
We are required to determine the time taken for a man to walk down an escalator, with its motion.

We are given the time taken for the man to ride down the escalator and the time taken for the man to walk up the escalator, against it motion.

\westep{Determine how to approach the problem}
Select down as positive and assume that the escalator moves at a velocity $v_e$. If the distance of the escalator is $x_e$ then:
\equ{v_e=\frac{x_e}{10\es}}{eq:we1}
Now, assume that the man walks at a velocity $v_m$. Then we have that:
\equ{v_e-v_m=\frac{x_e}{15\es}}{eq:we2}
We are required to find $t$ in:
\equ{v_e+v_m=\frac{x_e}{t}}{eq:we3}

\westep{Solve the problem}
We find that we have three equations and three unknowns ($v_e$, $v_m$ and $t$).

Add (\ref{eq:we2}) to (\ref{eq:we3}) to get:
\nequ{2v_e=\frac{x_e}{15\es}+\frac{x_e}{t}}
Substitute from (\ref{eq:we1}) to get:
\nequ{2\frac{x_e}{10\es}=\frac{x_e}{15\es}+\frac{x_e}{t}}
Since $x_e$ is not equal to zero we can divide throughout by $x_e$.
\nequ{\frac{2}{10\es}=\frac{1}{15\es}+\frac{1}{t}}
Re-write:
\nequ{\frac{2}{10\es}-\frac{1}{15\es}=\frac{1}{t}}
Multiply by $t$:
\nequ{t(\frac{2}{10\es}-\frac{1}{15\es})=1}
Solve for $t$:

\nequ{t=\frac{1}{\frac{2}{10\es}-\frac{1}{15\es}}}
to get:
\nequ{t=\frac{15}{2}\es}
\westep{Write the final answer}
The man will take 15/2 = 7,5 s
}
\end{wex}


\Exercise{Frames of Reference}{
\begin{enumerate}
\item{A woman walks north at 3 \kph\ on a boat that is moving east at 4 \kph. This situation is illustrated in the diagram below.
\begin{enumerate}
\item{How fast is the woman moving according to her friend who is also on the boat?}
\item{What is the woman's velocity according to an observer watching from the river bank?}
\end{enumerate}
\begin{center}
\begin{pspicture}(0,-2.655625)(6.3459377,2.675625)
\psline[linewidth=0.04cm](1.2259375,0.684375)(1.2259375,-0.695625)
\psbezier[linewidth=0.04](1.2259375,0.6515234)(2.8859375,0.704375)(3.4659376,0.62179434)(4.1059375,0.064375)
\psbezier[linewidth=0.04](1.2059375,-0.70847654)(2.8659375,-0.6556249)(3.1459374,-0.895625)(4.1059375,0.064375)
\psdots[dotsize=0.14](1.8659375,-0.395625)
\psline[linewidth=0.04cm,arrowsize=0.05291667cm 3.0,arrowlength=2.0,arrowinset=0.4]{->}(1.8659375,-0.375625)(1.8659375,0.344375)
\usefont{T1}{ptm}{m}{n}
\rput(2.7275,-0.125625){3\kph}
\psline[linewidth=0.04cm,arrowsize=0.05291667cm 3.0,arrowlength=2.0,arrowinset=0.4]{->}(3.7259376,-1.095625)(6.3259373,-1.075625)
\usefont{T1}{ptm}{m}{n}
\rput(4.912031,-0.665625){4\kph}
\psline[linewidth=0.04cm,arrowsize=0.05291667cm 4.0,arrowlength=2.4,arrowinset=0.4]{->}(0.1459375,-2.635625)(0.1459375,2.084375)
\usefont{T1}{ptm}{m}{n}
\rput(0.1646875,2.479375){\large N}
\end{pspicture} 
\end{center}


}
\item{A boy is standing inside a train that is moving at 10 \ms to the left. The boy throws a ball in the air with a velocity of 4 \ms. What is the resultant velocity of the ball
\begin{enumerate}

\item according to the boy?
\item according to someone outside the train?
\end{enumerate}}
\end{enumerate}

% Automatically inserted shortcodes - number to insert 2
\par \practiceinfo
\par \begin{tabular}[h]{cccccc}
% Question 1
(1.)	01s6	&
% Question 2
(2.)	01s7	&
\end{tabular}
% Automatically inserted shortcodes - number inserted 2
}
\summary{VPnmi}

\begin{enumerate}
\item Projectiles are objects that move through the air.
\item Objects that move up and down (vertical projectiles) on the earth accelerate with a constant acceleration g which is approximately equal to 9,8 \mss directed downwards towards the centre of the earth.
\item The equations of motion can be used to solve vertical projectile problems.
\begin{eqnarray*}
v_f &=& v_i + gt \\
\Delta x &=& \frac{(v_i + v_f)}{2} t\\
\Delta x &=& v_it + \frac{1}{2}gt^2\\
v_f^2 &=& v_i^2 + 2g\Delta x
\end{eqnarray*}
\item Graphs can be drawn for vertical projectile motion and are similar to the graphs for motion at constant acceleration. If upwards is taken as positive the $\Delta x$ vs $t$, $v$ vs $t$ ans $a$ vs $t$ graphs for an object being thrown upwards look like this:
\begin{center}
\begin{pspicture}(-1,-1.4)(12.4,3.6)
%\psgrid[gridcolor=lightgray]
\rput(0,0){
\psaxes[labels=none,ticks=none]{->}(0,0)(3,3)
\psplot[unit=2,plotstyle=curve]{0}{1.02}{x 2 exp -4.9 mul 5 x mul add}
\uput[u](0,3){$h$ (m)}
\uput[r](3,0){$t$ (s)}
\psline[linestyle=dashed](1.02,0)(1.02,2.75)
\uput[d](1.02,0){$t_m$}
\uput[d](2.04,0){$t_f$}
\uput[l](0,2.55){$h_m$}
\uput[d](1.5,-0.5){(a)}
\uput[l](0,0){0}}
\rput(4.5,1){
\psaxes[labels=none,ticks=none]{->}(0,0)(0,-2)(2.5,2)
\psplot[xunit=2,yunit=0.3]{0}{1.02}{ -9.8 x mul 5 add}
\uput[u](0,2){$v$ (\ms)}
\uput[r](2.5,0){$t$ (s)}
\uput[d](1.02,0){$t_m$}
\uput[u](2.04,0){$t_f$}
\psline[linestyle=dashed](2.04,-1.6)(2.04,0)
\uput[l](0,0){0}
\uput[l](0,5){$v_{i}$}
\uput[l](0,-5){$v_{f}$}
\uput[d](1.5,-1.5){(b)}}
\rput(9,1){\psaxes[labels=none,ticks=none]{->}(0,0)(0,-2)(2.5,2)
\psline[linewidth=2pt](0,-1)(2,-1)
\uput[l](0,-1){$g$}
\uput[u](0,2){$a$ (\mss)}
\uput[r](2.5,0){$t$ (s)}
\uput[d](1.5,-1.5){(c)}
\uput[l](0,0){0}}
\end{pspicture}
\end{center}


\item Momentum is conserved in one and two dimensions.\\
\begin{eqnarray*}
p&=&mv\\
\Delta p&=& m\Delta v\\
\Delta p&=& F\Delta t
\end{eqnarray*}
\item An elastic collision is a collision where both momentum and kinetic energy is conserved.
\begin{eqnarray*}
p_{\rm{before}}&=&p_{\rm{after}}\\
\kener_{\rm{before}}&=&\kener_{\rm{after}}
\end{eqnarray*}
\item An inelastic collision is where momentum is conserved but kinetic energy is not conserved.
\begin{eqnarray*}
p_{\rm{before}}&=&p_{\rm{after}}\\
\kener_{\rm{before}}&\neq&\kener_{\rm{after}}
\end{eqnarray*}
\item The frame of reference is the point of view from which a system is observed.
\end{enumerate}



\begin{eocexercises}{}
\begin{enumerate}
\item{[IEB 2005/11 HG] Two friends, Ann and Lindiwe decide to race each other by swimming across a river to the other side. They swim at identical speeds relative to the water. The river has a current flowing to the east. 
\MarginCompass
\begin{center}
\begin{pspicture*}(-1,-0.2)(3,2.4)
%\psgrid
\psline[linewidth=2pt](0,0)(3,0)					%lower river line
\rput(0,1.8){\psline[linewidth=2pt](0,0)(3,0)}		%upper river line
\psdots[dotsize=5pt](1,0)(1,1.8)(2.4,1.8)		%dots
\psline{->}(0,0.9)(1,0.9)		%current arrow
\uput[l](0,0){start}
\uput[l](0,1.8){finish}
\uput[u](1,1.8){Ann}
\uput[u](2.4,1.8){Lindiwe}
\uput[d](0.5,0.9){current}
\end{pspicture*}
\end{center}
Ann heads a little west of north so that she reaches the other side directly across from the starting point. Lindiwe heads north but is carried downstream, reaching the other side downstream of Ann. 
\renewcommand{\labelenumii}{\Alph{enumii}}
Who wins the race?
\begin{enumerate}
\item{Ann}
\item{Lindiwe}
\item{It is a dead heat}
\item{One cannot decide without knowing the velocity of the current.}
\end{enumerate}}


\item{[SC 2001/11 HG1] A bullet fired vertically upwards reaches a maximum height and falls back to the ground.
\begin{center}
\begin{pspicture}(0,0)(1,4.6)
\SpecialCoor
%\psgrid[gridcolor=lightgray]
\psline{->}(0,0)(0,4)
\rput(0.5,0){\psline{<-}(0,0)(0,4)}
\psarc(0.25,4){0.25}{0}{180}
\end{pspicture}
\end{center}
Which \textbf{one} of the following statements is \textbf{true} with reference to the acceleration of the bullet during its motion, if air resistance is ignored? The acceleration:
\renewcommand{\labelenumii}{\Alph{enumii}}
\begin{enumerate}
\item{is always downwards}
\item{is first upwards and then downwards}
\item{is first downwards and then upwards}
\item{decreases first and then increases}
\end{enumerate}}

\item{[SC 2002/03 HG1] Thabo suspends a bag of tomatoes from a spring balance held vertically. The balance itself weighs 10~N and he notes that the balance reads 50~N. He then lets go of the balance and the balance and tomatoes fall freely. What would the reading be on the balance while falling?
\begin{center}
\begin{pspicture}(-0.8,-1)(1,1.6)
%\psgrid[gridcolor=lightgray]
\def\springbalance{\psframe(0,0.25)(0.25,1)\pscircle(0.125,0.125){0.125}\psline(0.125,1)(0.125,1.5)}
\rput(0,0){\springbalance}
\psarc(0.125,-0.5){0.5}{180}{360}
\psline(-0.375,-0.5)(0,0.125)
\psline(0.625,-0.5)(0.25,0.125)
\rput(0.25,-0.8){\psellipse[fillcolor=gray,fillstyle=solid](0,0)(0.2,0.1)}
\rput(0.2,-0.6){\psellipse[fillcolor=gray,fillstyle=solid](0,0)(0.2,0.1)}
\rput(0.,-0.8){\psellipse[fillcolor=gray,fillstyle=solid](0,0)(0.2,0.1)}
\psline{->}(1,0.5)(1,0)\uput[r](1,0){falls freely}
\end{pspicture}
\end{center}
\renewcommand{\labelenumii}{\Alph{enumii}}
\begin{enumerate}
\item{50~N}
\item{40~N}
\item{10~N}
\item{0~N}
\end{enumerate}}


		\item{[IEB 2002/11 HG1] Two balls, P and Q, are simultaneously thrown into the air from the same height above the ground. P is thrown vertically upwards and Q vertically downwards with the same initial speed. Which of the following is true of both balls just before they hit the ground? (Ignore any air resistance. Take downwards as the positive direction.)
\begin{center}
\begin{tabular}{|l|l|l|}\hline
&\textbf{Velocity}&\textbf{Acceleration}\\ \hline
A&The same &The same \\\hline
B& P has a greater velocity than Q&P has a negative acceleration; Q has a positive acceleration \\\hline
C&P has a greater velocity than Q & The same\\\hline
D&The same &P has a negative acceleration; Q has a positive acceleration \\\hline
\end{tabular}
\end{center}
}

\item{[IEB 2002/11 HG1] An observer on the ground looks up to see a bird flying overhead along a straight line on bearing 130$\degree$ (40$\degree$ S of E). There is a steady wind blowing from east to west. In the vector diagrams below, I, II and III represent the following:

I\,\,\,\,\,\,\,the velocity of the bird relative to the air\\
II\,\,\,\,\,the velocity of the air relative to the ground\\
III\,\,\,the resultant velocity of the bird relative to the ground\\

Which diagram correctly shows these three velocities?	
\begin{center}
\scalebox{0.8}{
\begin{pspicture}(0,-2.1125)(16.035,2.1125)
\psline[linewidth=0.03cm,arrowsize=0.05291667cm 3.0,arrowlength=2.0,arrowinset=0.4]{<->}(0.58,1.5740625)(0.58,-1.3459375)
\psline[linewidth=0.03cm,arrowsize=0.05291667cm 3.0,arrowlength=2.0,arrowinset=0.4]{<->}(0.0,1.0940624)(3.4,1.0940624)
\usefont{T1}{ptm}{m}{n}
\rput(0.58984375,1.9440625){N}
\psline[linewidth=0.04cm,arrowsize=0.05291667cm 2.0,arrowlength=1.4,arrowinset=0.4]{->}(0.58,1.1140625)(2.68,-0.1859375)
\psline[linewidth=0.04cm,arrowsize=0.05291667cm 2.0,arrowlength=1.4,arrowinset=0.4]{->}(0.6,1.0940624)(1.24,-0.2259375)
\psline[linewidth=0.04cm,arrowsize=0.05291667cm 2.0,arrowlength=1.4,arrowinset=0.4]{->}(2.64,-0.1859375)(1.24,-0.2059375)
\psarc[linewidth=0.04](1.39,0.8640625){0.23}{267.70938}{100.00798}
\usefont{T1}{ptm}{m}{n}
\rput(1.2335937,0.9290625){\footnotesize $40\degree$}
\usefont{T1}{ptm}{m}{n}
\rput(1.876875,-0.5459375){\small ||}
\usefont{T1}{ptm}{m}{n}
\rput(2.266875,0.5740625){\small |}
\usefont{T1}{ptm}{m}{n}
\rput(0.766875,0.1140625){\small |||}
\psline[linewidth=0.03cm,arrowsize=0.05291667cm 3.0,arrowlength=2.0,arrowinset=0.4]{<->}(13.2,1.5540625)(13.2,-1.3659375)
\psline[linewidth=0.03cm,arrowsize=0.05291667cm 3.0,arrowlength=2.0,arrowinset=0.4]{<->}(12.62,1.0740625)(16.02,1.0740625)
\usefont{T1}{ptm}{m}{n}
\rput(13.209844,1.9240625){N}
\psline[linewidth=0.04cm,arrowsize=0.05291667cm 2.0,arrowlength=1.4,arrowinset=0.4]{->}(13.2,1.0940624)(15.3,-0.2059375)
\psline[linewidth=0.04cm,arrowsize=0.05291667cm 2.0,arrowlength=1.4,arrowinset=0.4]{->}(13.22,1.0740625)(13.86,-0.2459375)
\psarc[linewidth=0.04](13.51,0.7840625){0.29}{267.70938}{100.00798}
\usefont{T1}{ptm}{m}{n}
\rput(14.373593,0.8490625){\footnotesize $40\degree$}
\usefont{T1}{ptm}{m}{n}
\rput(14.786875,0.5540625){\small |||}
\psline[linewidth=0.04cm,arrowsize=0.05291667cm 2.0,arrowlength=1.4,arrowinset=0.4]{->}(15.26,-0.1659375)(13.86,-0.1859375)
\usefont{T1}{ptm}{m}{n}
\rput(13.486875,-0.0259375){\small |}
\usefont{T1}{ptm}{m}{n}
\rput(14.496875,-0.5259375){\small ||}
\usefont{T1}{ptm}{m}{n}
\rput(14.066719,-1.9109375){\large D}
\psline[linewidth=0.03cm,arrowsize=0.05291667cm 3.0,arrowlength=2.0,arrowinset=0.4]{<->}(4.8,1.5340625)(4.8,-1.3859375)
\psline[linewidth=0.03cm,arrowsize=0.05291667cm 3.0,arrowlength=2.0,arrowinset=0.4]{<->}(4.22,1.0540625)(7.62,1.0540625)
\usefont{T1}{ptm}{m}{n}
\rput(4.8098435,1.9040625){N}
\psline[linewidth=0.04cm,arrowsize=0.05291667cm 2.0,arrowlength=1.4,arrowinset=0.4]{->}(4.8,1.0740625)(6.9,-0.2259375)
\psline[linewidth=0.04cm,arrowsize=0.05291667cm 2.0,arrowlength=1.4,arrowinset=0.4]{->}(4.82,1.0540625)(5.46,-0.2659375)
\psline[linewidth=0.04cm,arrowsize=0.05291667cm 2.0,arrowlength=1.4,arrowinset=0.4]{->}(6.86,-0.2259375)(5.46,-0.2459375)
\psarc[linewidth=0.04](5.31,0.7240625){0.33}{233.53076}{94.398705}
\usefont{T1}{ptm}{m}{n}
\rput(6.213594,0.7890625){\footnotesize $40\degree$}
\usefont{T1}{ptm}{m}{n}
\rput(6.096875,-0.5859375){\small ||}
\usefont{T1}{ptm}{m}{n}
\rput(6.486875,0.5340625){\small |}
\usefont{T1}{ptm}{m}{n}
\rput(4.986875,0.0740625){\small |||}
\psline[linewidth=0.03cm,arrowsize=0.05291667cm 3.0,arrowlength=2.0,arrowinset=0.4]{<->}(8.98,1.5540625)(8.98,-1.3659375)
\psline[linewidth=0.03cm,arrowsize=0.05291667cm 3.0,arrowlength=2.0,arrowinset=0.4]{<->}(8.4,1.0740625)(11.8,1.0740625)
\usefont{T1}{ptm}{m}{n}
\rput(8.989843,1.9240625){N}
\psline[linewidth=0.04cm,arrowsize=0.05291667cm 2.0,arrowlength=1.4,arrowinset=0.4]{->}(8.98,1.0940624)(11.08,-0.2059375)
\psline[linewidth=0.04cm,arrowsize=0.05291667cm 2.0,arrowlength=1.4,arrowinset=0.4]{->}(9.0,1.0740625)(9.64,-0.2459375)
\psline[linewidth=0.04cm,arrowsize=0.05291667cm 2.0,arrowlength=1.4,arrowinset=0.4]{<-}(11.02,-0.2059375)(9.62,-0.2259375)
\psarc[linewidth=0.04](9.21,1.2040625){0.23}{267.70938}{140.44034}
\usefont{T1}{ptm}{m}{n}
\rput(10.313594,1.4690624){\footnotesize $130\degree$}
\usefont{T1}{ptm}{m}{n}
\rput(10.276875,-0.5659375){\small ||}
\usefont{T1}{ptm}{m}{n}
\rput(9.146875,0.1540625){\small |}
\usefont{T1}{ptm}{m}{n}
\rput(10.486875,0.5340625){\small |||}
\usefont{T1}{ptm}{m}{n}
\rput(1.4146875,-1.8909374){\large A}
\usefont{T1}{ptm}{m}{n}
\rput(5.651094,-1.9109375){\large B}
\usefont{T1}{ptm}{m}{n}
\rput(9.811093,-1.9309375){\large C}
\end{pspicture}} 
\end{center}
}


\item{[SC 2003/11] A ball X of mass $m$ is projected vertically upwards at a speed $u_x$ from a bridge 20~m high. A ball Y of mass $2m$ is projected vertically downwards from the same bridge at a speed of $u_y$. The two balls reach the water at the same speed. Air friction can be ignored.

Which of the following is true with reference to the speeds with which the balls are projected?
\renewcommand{\labelenumii}{\Alph{enumii}}
\begin{enumerate}
\item $u_x=\frac{1}{2}u_y$
\item $u_x=u_y$
\item $u_x=2u_y$
\item $u_x=4u_y$
\end{enumerate}}

%\item{[SC 2001/11 HG1] A sphere is attached to a string, which is suspended from a horizontal ceiling.
%\begin{center}
%\begin{pspicture}(0,0)(5,4)
%\SpecialCoor
%%\psgrid[gridcolor=lightgray]
%\psframe(0,3)(5,4)
%\psline(2.5,3)(2.5,1)
%\pscircle(2.5,0.5){0.5}
%\uput[ur](5,3){ceiling}
%\uput[ur](2.5,1){string}
%\uput[r](3,0.5){sphere}
%\end{pspicture}
%\end{center}
%The reaction force to the gravitational force exerted by the earth on the sphere is ...
%\renewcommand{\labelenumii}{\Alph{enumii}}
%\begin{enumerate}
%\item{the force of the sphere on the earth.}
%\item{the force of the ceiling on the string.}
%\item{the force of the string on the sphere.}
%\item{the force of the ceiling on the sphere.}
%\end{enumerate}}

\item{[SC 2002/03 HG1] A stone falls freely from rest from a certain height. Which one of the following quantities could be represented on the $y$-axis of the graph below?
\begin{center}
\begin{pspicture}(0,0)(3,3)
%\psgrid[gridcolor=lightgray]
\psline{<->}(0,3)(0,0)(3,0)
\uput[l](0,3){Y}
\uput[d](3,0){time}
\psplot{0}{1.5}{x 2 exp}
\end{pspicture}
\end{center}
\renewcommand{\labelenumii}{\Alph{enumii}}
\begin{enumerate}
\item{velocity}
\item{acceleration}
\item{momentum}
\item{displacement}
\end{enumerate}}

\renewcommand{\labelenumii}{\Alph{enumii}}
\item{
A man walks towards the back of a train at 2 \ms while the train moves forward at 10 \ms. The magnitude of the man's velocity with respect to the ground is

\begin{enumerate}
\item 2 \ms
\item 8 \ms
\item 10 \ms
\item 12 \ms
\end{enumerate}}

\item{A stone is thrown vertically upwards and it returns to the ground. If friction is ignored, its acceleration as it reaches the highest point of its motion is
\begin{enumerate}
\item greater than just after it left the throwers hand.
\item less than just before it hits the ground.
\item the same as when it left the throwers hand.
\item less than it will be when it strikes the ground.
\end{enumerate}}

\item{
An exploding device is thrown vertically upwards. As it reaches its highest point, it explodes and breaks up into three pieces of \textbf{equal mass}. Which one of the following combinations is possible for the motion of the three pieces if they all move in a vertical line?

\begin{center}
\begin{tabular}{|l|l|l|l|}\hline
&\textbf{Mass 1}&\textbf{Mass 2}&\textbf{Mass 3}\\ \hline
A&v downwards&v downwards&v upwards \\\hline
B& v upwards &2v downwards& v upwards \\\hline
C&2v upwards &v downwards& v upwards\\\hline
D&v upwards &2v downwards& v downwards\\\hline
\end{tabular}
\end{center}
}

\item{[IEB 2004/11 HG1] A stone is thrown vertically up into the air. Which of the following graphs best shows the resultant force exerted on the stone against time while it is in the air? (Air resistance is negligible.)

\begin{figure}[H]
\begin{center}
\begin{pspicture}(-1,-1.4)(12.4,3.6)
%\psgrid[gridcolor=lightgray]
\rput(0,0){
\psaxes[labels=none,ticks=none]{->}(0,0)(3,3)
\psplot[unit=2,plotstyle=curve]{0}{1.02}{x 2 exp -4.9 mul 5 x mul add}
\uput[u](0,3){$F_{res}$}
\uput[r](3,0){$t$}
\psline[linestyle=dashed](1.02,0)(1.02,2.75)
\uput[d](1.5,-0.5){A}
\uput[l](0,0){0}}
\rput(3.5,1){
\psaxes[labels=none,ticks=none]{->}(0,0)(0,-2)(2.5,2)
\psplot[xunit=2,yunit=0.3]{0}{1.02}{ -9.8 x mul 5 add}
\uput[u](0,2){$F_{res}$}
\uput[r](2.5,0){$t$}
\psline[linestyle=dashed](2.04,-1.6)(2.04,0)
\uput[l](0,0){0}
\uput[d](1.5,-1.5){B}}
\rput(7,0){\psaxes[labels=none,ticks=none]{->}(0,0)(2.5,3)
\psline[linewidth=2pt](0,1)(2,1)
\uput[u](0,2){$F_{res}$}
\uput[r](2.5,0){$t$}
\uput[d](1.5,-0.5){C}
\uput[l](0,0){0}}
\end{pspicture}
\label{fig:p:m:m2d12:pm:up}
\end{center}
\end{figure}
	}









\item{What is the velocity of a ball just as it hits the ground if it is thrown upward at 10 \ms from a height 5 meters above the ground?}


\item{[IEB 2005/11 HG1]
A breeze of 50 \kph\ blows towards the west as a pilot flies his light plane from town A to village B. The trip from A to B takes 1 h. He then turns west, flying for $\frac{1}{2}$ h until he reaches a dam at point C. He turns over the dam and returns to town A. The diagram shows his flight plan. It is not to scale.

\MarginCompass
\begin{center}
\begin{pspicture}(-0.2,0)(5.6,3.5)
%	\psgrid[gridcolor=lightgray]
\pnode(0,3){C}
\pnode(3,3){B}
\pnode(3,0){A}
\psline{->}(C)(A)
\psline{->}(A)(B)
\psline{->}(B)(C)
\psline{->}(5,1.5)(4,1.5)
\uput[u](C){C}
\uput[u](B){B}
\uput[r](A){A}
\uput[u](4.5,1.5){Wind velocity}
\uput[d](4.5,1.5){50~\kph}
\end{pspicture}
\end{center}

The pilot flies at the same altitude at a constant speed of 130 km.h$^{-1}$ relative to the air throughout this flight.
\renewcommand{\labelenumii}{\alph{enumii}}
\begin{enumerate}
\item{Determine the magnitude of the pilot's resultant velocity from the town A to the village B.}
\item{How far is village B from town A?}
\item{What is the plane's speed relative to the ground as it travels from village B to the dam at C?}\\
\item{Determine the following, by calculation or by scale drawing:
\begin{enumerate}
\item{The distance from the village B to the dam C.}
\item{The displacement from the dam C back home to town A.}
\end{enumerate}}
\end{enumerate}}



\item{A cannon (assumed to be at ground level) is fired off a flat surface at an angle, $\theta$ above the horizontal with an initial speed of $v_0$.
\renewcommand{\labelenumii}{\alph{enumii}}
\begin{enumerate}
\item{What is the initial horizontal component of the velocity?}
\item{What is the initial vertical component of the velocity?}
\item{What is the horizontal component of the velocity at the highest point of the trajectory?}
\item{What is the vertical component of the velocity at that point?}
\item{What is the horizontal component of the velocity when the projectile lands?}
\item{What is the vertical component of the velocity when it lands?}
\end{enumerate}}





	\item{[IEB 2004/11 HG1] Hailstones fall vertically on the hood of a car parked on a horizontal stretch of road. The average terminal velocity of the hailstones as they descend is 8,0 m.s$^{-1}$ and each has a mass of 1,2 g.
\renewcommand{\labelenumii}{\alph{enumii}}
	\begin{enumerate}
%		\item{Explain why a hailstone falls with a \textbf{terminal} velocity.}
		\item{Calculate the magnitude of the momentum of a hailstone just before it strikes the hood of the car.}
		\item{If a hailstone rebounds at 6,0 m.s$^{-1}$ after hitting the car's hood, what is the magnitude of its change in momentum?}
		\item{The hailstone is in contact with the car's hood for 0,002 s during its collision with the hood of the car. What is the magnitude of the resultant force exerted on the hood if the hailstone rebounds at 6,0 m.s$^{-1}$?}
		\item{A car's hood can withstand a maximum impulse of 0,48 N$\cdot$s without leaving a permanent dent. Calculate the minimum mass of a hailstone that will leave a dent in the hood of the car, if it falls at 8,0 m.s$^{-1}$ and rebounds at 6,0 m.s$^{-1}$ after a collision lasting 0,002 s.}
	\end{enumerate}}




\item{[IEB 2003/11 HG1 - Biathlon] Andrew takes part in a biathlon race in which he first swims across a river and then cycles. The diagram below shows his points of entry and exit from the river, A and P, respectively.

\begin{center}
\begin{pspicture}(0,-3.1826563)(10.831562,3.1626563)
\psline[linewidth=0.04cm](0.0,-0.5573437)(8.38,3.1426563)
\psline[linewidth=0.04cm](0.7,-2.0973437)(9.08,1.6026562)
\psline[linewidth=0.03cm](2.08,0.36265624)(2.8,-1.1773437)
\psline[linewidth=0.03cm,arrowsize=0.05291667cm 3.0,arrowlength=2.0,arrowinset=0.4]{->}(2.8,-1.1773437)(10.3,-1.1773437)
\psline[linewidth=0.03cm](2.06,0.38265625)(8.94,1.5426563)
\usefont{T1}{ptm}{m}{n}
\rput(4.0675,-0.8773438){\small $30\degree$}
\psarc[linewidth=0.03](4.42,-0.8773438){0.36}{300.96375}{108.43495}
\usefont{T1}{ptm}{m}{n}
\rput(2.0865624,1.0326562){A}
\usefont{T1}{ptm}{m}{n}
\rput(2.8367188,-1.5473437){Q}
\usefont{T1}{ptm}{m}{n}
\rput(8.910781,1.8326563){P}
\usefont{T1}{ptm}{m}{n}
\rput(1.838125,-0.5673438){100 m}
\usefont{T1}{ptm}{m}{n}
\rput(0.48796874,-1.3273437){River}
\psline[linewidth=0.03cm,arrowsize=0.05291667cm 3.0,arrowlength=2.0,arrowinset=0.4]{->}(6.2,1.6826563)(8.32,2.6226563)
\usefont{T1}{ptm}{m}{n}
\rput(7.624219,1.9326563){current}
\psline[linewidth=0.03cm,arrowsize=0.05291667cm 3.0,arrowlength=2.0,arrowinset=0.4]{<->}(8.9,-2.7173438)(8.9,0.32265624)
\usefont{T1}{ptm}{m}{n}
\rput(8.889844,0.71265626){N}
\usefont{T1}{ptm}{m}{n}
\rput(10.682032,-1.2473438){E}
\usefont{T1}{ptm}{m}{n}
\rput(8.879219,-3.0273438){S}
\psdots[dotsize=0.14](2.8,-1.1373438)
\psdots[dotsize=0.14](2.08,0.36265624)
\psdots[dotsize=0.14](8.88,1.5226562)
\end{pspicture} 
\end{center}

During the swim, Andrew maintains a constant velocity of 1,5 m.s$^{-1}$ East relative to the water. The water in the river flows at a constant velocity of 2,5 m.s$^{-1}$ in a direction 30$\degree$ North of East. The width of the river is 100 m.

The diagram below is a velocity-vector diagram used to determine the resultant velocity of Andrew relative to the river bed.	
\begin{center}
\begin{pspicture}(0,-1.281875)(7.6646876,1.281875)
\psline[linewidth=0.04cm,arrowsize=0.05291667cm 3.0,arrowlength=2.0,arrowinset=0.4]{->}(0.3971875,-0.801875)(4.3571873,-0.801875)
\psline[linewidth=0.04cm,arrowsize=0.05291667cm 3.0,arrowlength=2.0,arrowinset=0.4]{->}(0.3771875,-0.781875)(7.2771873,1.018125)
\psline[linewidth=0.04cm,arrowsize=0.05291667cm 3.0,arrowlength=2.0,arrowinset=0.4]{->}(4.2971873,-0.801875)(7.3371873,1.038125)
\usefont{T1}{ptm}{m}{n}
\rput(0.12375,-1.011875){A}
\usefont{T1}{ptm}{m}{n}
\rput(4.364375,-1.131875){B}
\usefont{T1}{ptm}{m}{n}
\rput(7.504375,1.108125){C}
\end{pspicture} 
\end{center}
\renewcommand{\labelenumii}{\alph{enumii}}
\begin{enumerate}
\item{Which of the vectors (AB, BC and AC) refer to each of the following?}
\begin{enumerate}
\item{The velocity of Andrew relative to the water.}
\item{The velocity of the water relative to the water bed.}
\item{The resultant velocity of Andrew relative to the river bed.}	
\end{enumerate}	

\item{Determine the magnitude of Andrew's velocity relative to the river bed either by calculations or by scale drawing, showing your method clearly.}	
\item{How long (in seconds) did it take Andrew to cross the river?}
\item{At what distance along the river bank (QP) should Peter wait with Andrew's bicycle ready for the next stage of the race?}
\end{enumerate}}


%The following questions were moved from grade 10:


\item{[IEB 2002/11 HG1 - Bouncing Ball]

A ball bounces vertically on a hard surface after being thrown vertically up into the air by a boy standing on the ledge of a building.

Just before the ball hits the ground for the first time, it has a velocity of magnitude 15 m.s$^{-1}$. Immediately, after bouncing, it has a velocity of magnitude 10 m.s$^{-1}$.

The graph below shows the velocity of the ball as a function of time from the moment it is thrown upwards into the air until it reaches its maximum height after bouncing once.

\begin{center}
\begin{pspicture}(-1.4,-3.5)(9.4,3)
%\psgrid[gridcolor=lightgray]
\psset{xunit=2}
\psline{->}(0,0)(4,0)
\uput[r](4,0){time (s)}

\psline{->}(0,-3.5)(0,2.5)
\uput[u](0,2.5){velocity (\ms)}

\psline(0,1)(2,-3)(2,2)(3,0)

\psline[linestyle=dashed](0,2)(2,2)
\psline[linestyle=dashed](0,-3)(2,-3)
\psline(1,0.2)(1,-0.2)

\uput[l](0,2){10}
\uput[l](0,1){5}
\uput[l](0,0){0}
\uput[l](0,-1){-5}
\uput[l](0,-2){-10}
\uput[l](0,-3){-15}

\uput[d](1,0){1,0}
\uput[d](2,0){2,0}
\end{pspicture}
\end{center}
\renewcommand{\labelenumii}{\alph{enumii}}
\begin{enumerate}
\item{At what velocity does the boy throw the ball into the air?}
\item{What can be determined by calculating the gradient of the graph during the first two seconds?}
\item{Determine the gradient of the graph over the first two seconds. State its units.}
\item{How far below the boy's hand does the ball hit the ground?}
\item{Use an equation of motion to calculate how long it takes, from the time the ball was thrown, for the ball to reach its maximum height after bouncing.}
\item{What is the position of the ball, measured from the boy's hand, when it reaches its maximum height after bouncing?}
\end{enumerate}}

\item{[IEB 2001/11 HG1] - \textbf{Free Falling?}

A parachutist steps out of an aircraft, flying high above the ground. She falls for the first 8 seconds before opening her parachute. A graph of her velocity is shown in Graph A below.

\begin{center}
\begin{pspicture}(-1.4,-0.6)(11.4,6.6)
%\psgrid[gridcolor=lightgray]
\psset{unit=0.6}
\psline{->}(0,0)(16.5,0)
\uput[r](16.5,0){time (s)}

\psline{->}(0,0)(0,10)
\uput[u](0,10){velocity (\ms)}

\psline(0,0)(3,7)
\psline(4,8)(8,8)(9,1)(15,1)

\psplot{3}{4}{x 2 exp -1 mul 8 x mul add 8 sub}
\psplot{15}{16}{x 2 exp 32 x mul sub 256 add}
\rput(13,8){Graph A}

\psline[linestyle=dashed](0,8)(4,8)
\psline[linestyle=dashed](0,1)(9,1)
\psline[linestyle=dashed](4,0)(4,8)
\psline[linestyle=dashed](8,0)(8,8)
\psline[linestyle=dashed](9,0)(9,1)
\psline[linestyle=dashed](15,0)(15,1)

\uput[l](0,8){40}
\uput[l](0,1){5}
\uput[l](0,0){0}

\uput[d](4,0){4}
\uput[d](8,0){8}
\uput[d](9,0){9}
\uput[d](15,0){15}
\uput[d](16,0){16}

\end{pspicture}
\end{center}
\renewcommand{\labelenumii}{\alph{enumii}}
\begin{enumerate}
%\item{Describe her motion between A and B.}
\item{Use the information from the graph to calculate an approximate height of the aircraft when she stepped out of it (to the nearest 10 m).}
\item{What is the magnitude of her velocity during her descent with the parachute fully open?}

The air resistance acting on the parachute is related to the speed at which the parachutist descends. Graph B shows the relationship between air resistance and velocity of the parachutist descending with the parachute open.

\begin{center}
\begin{pspicture}(-1.4,-1)(5.2,7.2)
\psset{unit=0.75}
\psgrid[gridcolor=lightgray,gridlabels=0](0,0)(7,10)
\psaxes[dy=1,Dy=100]{<->}(0,0)(7,9.5)
\psplot{0}{6}{x 2 exp 0.2368 mul 0.0127 x mul add}
\pcline[offset=0.8cm,linestyle=none](0,0)(0,10)
\aput{:U}{Air resistance on parachutist (N)}
\pcline[offset=-0.4cm,linestyle=none](0,0)(7,0)
\bput{:U}{velocity (\ms)}
\rput(2,8.5){Graph B}
\end{pspicture}
\end{center}

\item{Use Graph B to find the magnitude of the air resistance on her parachute when she was descending with the parachute open.}
\item{Assume that the mass of the parachute is negligible. Calculate the mass of the parachutist showing your reasoning clearly.}
\end{enumerate}}

\item{
An aeroplane travels from Cape Town and the pilot must reach Johannesburg, which is situated 1300~km from Cape Town on a bearing of 50$\degree$ in 5 hours. At the height at which the plane flies, a wind is blowing at 100 \kph on a bearing of 130 $\degree$ for the whole trip.
\begin{center}
\begin{pspicture}(0,-2.6140625)(7.8934374,2.6140625)
\psline[linewidth=0.03cm,arrowsize=0.05291667cm 3.0,arrowlength=2.0,arrowinset=0.4]{<-}(1.16,2.0028124)(1.16,-2.0571876)
\psline[linewidth=0.03cm](0.0,-1.5971875)(2.38,-1.5971875)
\psline[linewidth=0.04cm,arrowsize=0.05291667cm 3.0,arrowlength=2.0,arrowinset=0.4]{->}(1.14,-1.6171875)(6.32,1.4828125)

\usefont{T1}{ptm}{m}{n}
\rput(1.7075,-0.8571875){\small 50$\degree$}
\usefont{T1}{ptm}{m}{n}
\rput(1.1026562,-2.3871875){Cape Town}
\usefont{T1}{ptm}{m}{n}
\rput(6.862969,1.8728125){Johannesburg}
\usefont{T1}{ptm}{m}{n}
\rput(1.13875,2.4178126){\large N}
\end{pspicture} 
\end{center}
\renewcommand{\labelenumii}{\alph{enumii}}
\begin{enumerate}
\item Calculate the magnitude of the average resultant velocity of the aeroplane, in \kph, if it is to reach its destination on time.
\item Calculate their average velocity, in \kph, in which the aeroplane should be travelling in order to reach Johannesburg in the prescribed 5 hours. Include a labelled, rough vector diagram in your answer.\\
(If an accurate scale drawing is used, a scale of 25 \kph = 1 cm must be used.)
\end{enumerate}}


\item{
Niko, in the basket of a hot-air balloon, is stationary at a height of 10 m above the level from where his friend, Bongi, will throw a ball. Bongi intends throwing the ball upwards and Niko, in the basket, needs to \textbf{descend} (move downwards) to catch the ball at its maximum height.
\begin{center}
\begin{pspicture}(0,-4.06)(10.46,4.06)
\definecolor{color597b}{rgb}{0.8,0.8,0.8}
\psbezier[linewidth=0.04](5.6945453,2.4050274)(5.969091,1.6547714)(6.478262,0.54652697)(6.94,0.52)(7.401738,0.49347302)(8.224675,1.8556266)(8.322338,2.5685246)(8.42,3.2814226)(7.8806114,4.04)(6.977782,4.04)(6.0749526,4.04)(5.42,3.1552832)(5.6945453,2.4050274)
\psline[linewidth=0.04cm](5.64,2.54)(8.3,2.54)
\psframe[linewidth=0.04,dimen=outer,fillstyle=solid,fillcolor=color597b](7.48,0.86)(6.42,-0.2)
\psline[linewidth=0.03cm,linestyle=dotted,dotsep=0.16cm](6.44,0.76)(1.96,0.76)
\psline[linewidth=0.04cm,arrowsize=0.05291667cm 3.0,arrowlength=2.0,arrowinset=0.4]{<->}(2.06,0.78)(2.06,-2.62)
\psline[linewidth=0.03cm,linestyle=dotted,dotsep=0.16cm](2.06,-2.56)(9.44,-2.56)
\psline[linewidth=0.04cm](0.0,-4.04)(10.44,-4.04)
\psline[linewidth=0.03cm](4.44,-2.84)(4.44,-3.64)
\pscircle[linewidth=0.03,dimen=outer](4.45,-2.63){0.25}
\psline[linewidth=0.03cm](4.44,-3.62)(4.14,-4.0)
\psline[linewidth=0.03cm](4.44,-3.6)(4.76,-4.02)
\psline[linewidth=0.03cm](4.44,-3.18)(5.2,-2.74)
\psline[linewidth=0.03cm](4.42,-3.16)(4.1,-2.96)
\psline[linewidth=0.03cm](4.14,-2.96)(3.84,-3.14)
\psdots[dotsize=0.1](4.36,-2.6)
\psdots[dotsize=0.1](4.56,-2.6)
\rput{-179.88995}(8.866179,-5.2129245){\psarc[linewidth=0.03](4.430587,-2.6107197){0.14993295}{53.75402}{144.51685}}
\psdots[dotsize=0.18](5.2,-2.64)
\psline[linewidth=0.03cm,arrowsize=0.05291667cm 3.0,arrowlength=2.0,arrowinset=0.4]{->}(5.2,-2.54)(5.2,-1.54)
\usefont{T1}{ptm}{m}{n}
\rput(6.041406,-1.89){$13\ems$}
\usefont{T1}{ptm}{m}{n}
\rput(1.0496875,-0.97){$10$ m}
\end{pspicture} 
\end{center}
Bongi throws the ball upwards with a velocity of 13 \ms. Niko starts his descent at the same instant the ball is thrown upwards, by letting air escape from the balloon, causing it to accelerate downwards. Ignore the effect of air friction on the ball.
\renewcommand{\labelenumii}{\alph{enumii}}
\begin{enumerate}
\item Calculate the maximum height reached by the ball.
\item Calculate the magnitude of the minimum average acceleration the balloon must have in order for Niko to catch the ball, if it takes 1,3 s for the ball to reach its maximum height.
\end{enumerate}}

\item{
Lesedi (mass 50 kg) sits on a massless trolley. The trolley is travelling at a constant speed of 3 \ms. His friend Zola (mass 60 kg) jumps on the trolley with a velocity of 2 \ms. What is the final velocity of the combination (Lesedi, Zola and trolley) if Zola jumps on the trolley from
\renewcommand{\labelenumii}{\alph{enumii}}
\begin{enumerate}
\item the front
\item behind
\item the side
\end{enumerate}
(Ignore all kinds of friction)
\begin{center}
\begin{pspicture}(0,-2.7076561)(11.845312,2.7076561)
\psframe[linewidth=0.03,dimen=outer](7.6053123,1.7492187)(2.5453124,-0.5107812)
\psline[linewidth=0.04cm](3.1453125,-0.53078127)(4.3053126,-0.53078127)
\psline[linewidth=0.04cm](5.8253126,1.7692188)(6.9853125,1.7692188)
\psline[linewidth=0.04cm](3.1453125,1.7692188)(4.3053126,1.7692188)

\psline[linewidth=0.04cm](5.8253126,-0.53078127)(6.9853125,-0.53078127)
\psline[linewidth=0.04cm,arrowsize=0.05291667cm 3.0,arrowlength=2.0,arrowinset=0.4]{->}(8.885312,1.7292187)(11.825313,1.7292187)
\usefont{T1}{ptm}{m}{n}
\rput(10.556719,2.5192187){$3\ems$}
\usefont{T1}{ptm}{m}{n}
\rput(5.0754685,0.6592187){Trolley + Lesedi}
\psline[linewidth=0.04cm,arrowsize=0.05291667cm 3.0,arrowlength=2.0,arrowinset=0.4]{->}(0.5453125,0.62921876)(3.0853126,0.62921876)
\psline[linewidth=0.04cm,arrowsize=0.05291667cm 3.0,arrowlength=2.0,arrowinset=0.4]{->}(5.1453123,-2.2107813)(5.1453123,-0.07078125)
\psline[linewidth=0.04cm,arrowsize=0.05291667cm 3.0,arrowlength=2.0,arrowinset=0.4]{->}(9.765312,0.62921876)(7.1253123,0.62921876)
\usefont{T1}{ptm}{m}{n}
\rput(10.070469,0.61921877){(a)}
\usefont{T1}{ptm}{m}{n}
\rput(5.1604686,-2.4807813){(b)}
\usefont{T1}{ptm}{m}{n}
\rput(0.17046875,0.5992187){(c)}
\end{pspicture} 
\end{center}
}
\end{enumerate}

% Automatically inserted shortcodes - number to insert 21
\par \practiceinfo
\par \begin{tabular}[h]{cccccc}
% Question 1
(1.)	01s8	&
% Question 2
(2.)	01s9	&
% Question 3
(3.)	01sa	&
% Question 4
(4.)	01sb	&
% Question 5
(5.)	01sc	&
% Question 6
(6.)	01sd	\\ % End row of shortcodes
% Question 7
(7.)	01se	&
% Question 8
(8.)	01sf	&
% Question 9
(9.)	01sg	&
% Question 10
(10.)	01sh	&
% Question 11
(11.)	01si	&
% Question 12
(12.)	01sj	\\ % End row of shortcodes
% Question 13
(13.)	01sk	&
% Question 14
(14.)	01sm	&
% Question 15
(15.)	01sn	&
% Question 16
(16.)	01sp	&
% Question 17
(17.)	01sq	&
% Question 18
(18.)	01sr	\\ % End row of shortcodes
% Question 19
(19.)	01ss	&
% Question 20
(20.)	01st	&
% Question 21
(21.)	01su	&
\end{tabular}
% Automatically inserted shortcodes - number inserted 21
\end{eocexercises}
% CHILD SECTION END 



% CHILD SECTION START 

