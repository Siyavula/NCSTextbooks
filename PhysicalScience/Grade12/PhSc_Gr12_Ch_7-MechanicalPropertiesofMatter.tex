\chapter{Mechanical Properties of Matter}
\label{p:mm:ep12}

\section {Introduction}

In this chapter we will look at some mechanical (physical) properties of various materials that we use. The mechanical properties of a material are those properties that are affected by forces being applied to the material. These properties are important to consider when we are constructing buildings, structures or modes of transport like an aeroplane.
\chapterstartvideo{VPnng}  
\section{Deformation of materials}
%\begin{syllabus}
%\item The learner must be able to Appreciate that deformation is caused by a force that can either be compressive or tensile when applied in 1 plane
%\item The learner must be able to Describe behaviour of a spring in terms of the relationship between applied force and extension of a spring [Hooke's Law].
%\item The learner must be able to Demonstrate an understanding of the similarities and differences between force-extension graphs for typical ductile, brittle and polymeric materials 
%\item The learner must be able to Recognize the point of ultimate tensile stress and the point beyond which permanent deformation takes place (elastic limit) and the point beyond which Hooke's Law is no longer obeyed (Limit of proportionality)
%\end{syllabus}

\subsection{Hooke's Law}

Deformation (change of shape) of a solid is caused by a force that can either be compressive or tensile when applied in one direction (plane). Compressive forces try to compress the object (make it smaller or more compact) while tensile forces try to tear it apart.
We can study these effects by looking at what happens when you compress or expand a spring.
 
Hooke's Law relates the restoring force of a spring to its displacement from equilibrium length.

The equilibrium length of a spring is its length when no forces are applied to it. When a force is applied to a spring, e.g., by attaching a weight to one end, the spring will expand and become longer. The difference between the new length and the equilibrium length is the displacement.
 
\HistoricalNote{Hooke's Law}{Hooke's law is named after the seventeenth century physicist Robert Hooke who discovered it in 1660 (18 July 1635 - 3 March 1703).}

\Definition{Hooke's Law}{In an elastic spring, the extension varies linearly with the force applied.

$F=-kx$ \\
where $F$ is the restoring force in newtons (N), $k$ is the spring constant in $N \cdot m^{-1}$ and $x$ is the displacement of the spring from its equilibrium length in metres (m).}

\begin{figure}[H]
\begin{center}
\scalebox{1} % Change this value to rescale the drawing.
{
\begin{pspicture}(0,-2.8784375)(5.5779686,2.8584375)
\rput(0.85796875,-1.7615625){\psaxes[linewidth=0.04,dx=1.0cm,dy=1.0cm,Dx=0.1](0,0)(0,0)(4,4)}
\psline[linewidth=0.04cm](0.85796875,-1.7615625)(4.857969,2.2384374)
\psline[linewidth=0.04cm](0.85796875,-1.7615625)(5.5579686,-1.7615625)
\psline[linewidth=0.04cm](0.85796875,-1.7615625)(0.85796875,2.8384376)
\psdots[dotsize=0.12](1.8579688,-0.7615625)
\psdots[dotsize=0.12](2.8579688,0.2384375)
\psdots[dotsize=0.12](3.8579688,1.2384375)
\psdots[dotsize=0.12](4.857969,2.2384374)
\usefont{T1}{ptm}{m}{n}
\rput{-270.0}(0.6884375,0.3615625){\rput(0.1575,0.5484375){Force (N)}}
\usefont{T1}{ptm}{m}{n}
\rput(3.2567186,-2.6515625){Extension (m)}
\end{pspicture} 
}
\caption{Hooke's Law - the relationship between a spring's restoring force and its displacement from equilibrium length.}
\end{center}
\end{figure}

\Activity{Experiment}{Hooke's Law}
{
\Aim{Verify Hooke's Law.}
\Apparatus{
\begin{itemize}
\item weights
\item spring
\item ruler
\end{itemize}
}
\Method{
\begin{enumerate}
\item Set up a spring vertically in such a way that you are able to hang weights from it. 
\item Measure the equilibrium length, $x_{0}$, of the spring (i.e. the length of the spring when nothing is attached to it).
\item Measure the extension of the spring for a range of different weights. Note: the extension is the difference between the spring's equilibrium length and the new length when a weight is attached to it, $x-x_{0}$. 
\item Draw a table of force (weight) in newtons and corresponding extension.
\item Draw a graph of force versus extension for your experiment. 
\end{enumerate}
 }
\Conclusions{
\begin{enumerate}
\item What do you observe about the relationship between the applied force and the extension?
\item Determine the gradient (slope) of the graph.
\item Now calculate the spring constant for your spring.
\end{enumerate} }
}
% Phet simulation on Hooke's Law: SIYAVULA-SIMULATION:http://cnx.org/content/m39533/latest/#springs
\simulation{phet on springs}{VPnni}
% Khan Academy video on springs and Hooke's Law: SIYAVULA-VIDEO:http://cnx.org/content/m39533/latest/#springs1
\mindsetvid{khan on springs}{VPnnz}
\begin{wex}{Hooke's Law I}{
A spring is extended by 7 cm by a force of 56 N.\\
Calculate the spring constant for this spring.
}

{
\begin{eqnarray*}
\textrm{F} &=& -\textrm{kx}\\
56 & = & -\textrm{k}\times0,07\\
\end{eqnarray*}
\begin{eqnarray*}
\textrm{k} &=& \frac{-56}{0,07}\\
 & = & -800\textrm{ N}\cdot \rm{m}^{-1}
\end{eqnarray*}
}

\end{wex}

\begin{wex}{Hooke's Law II}{
A spring of length 20cm stretches to 24cm when a load of 0,6N is applied to it.

\begin{enumerate}
\item Calculate the spring constant for the spring.\\

\item Determine the extension of the spring if a load of 0,5N is applied to it.\\

\end{enumerate}

}

{
\westep{Determine what information you have:}
We know: \\
$F$ = 0,6 N \\
The equilibrium spring length is 20 cm \\
The expanded spring length is 24 cm \\

\westep{Use Hooke's Law to find the spring constant}
First we need to calculate the displacement of the spring from its equilibrium length: \\
\begin{eqnarray*}
x & = & 24 \textrm{ cm} - 20 \textrm{ cm}\\
& = & 4 \textrm{ cm}\\
& = & 0,04 \textrm{ m}
\end{eqnarray*}

Now use Hooke's Law to find the spring constant: \\
\begin{eqnarray*}
F & = & -kx \\
0,6 & = & -k \cdot \: 0,04\\
k = -15  \textrm{ N.m}^{-1}
\end{eqnarray*}

\westep{The second part of the question asks us to find the spring's extension if a 0,5 N load is attached to it. We have:}
$F$ = 0,5 N \\
We know from the first part of the question that \\
$k$ = -15 $\textrm{ N.m}^{-1}$ \\

So, using Hooke's Law: \\

\begin{eqnarray*}
F & = & -kx \\
x & = & -\frac{F}{k} \\
& = & -\frac{0,5}{-15}\\
& = & 0,033 \textrm{ m}\\
& = & 3,3 \textrm{ cm}\\
\end{eqnarray*}
}

\end{wex}

\begin{wex}{Hooke's Law III}{
A spring has a spring constant of $-400$ N.m$^{-1}$. By how much will it stretch if a load of 50 N is applied to it?
}


{
\begin{eqnarray*}
\textrm{F} &=& -\textrm{kx}\\
50 & = & -(-400)\textrm{x}\\
\end{eqnarray*}
\begin{eqnarray*}
\textrm{x} &=& \frac{50}{400}\\
 & = & 0,125\textrm{ m}\\
& =& 12,5 \textrm{ cm}
\end{eqnarray*}
}
\end{wex}



\subsection{Deviation from Hooke's Law}

We know that if you have a small spring and you pull it apart too much it stops 'working'. It bends out of shape and loses its springiness. When this happens, Hooke's Law no longer applies, the spring's behaviour deviates from Hooke's Law.

Depending on what type of material we are dealing with, the manner in which it deviates from Hooke's Law is different. We give classify materials by this deviation. The following graphs show the relationship between force and extension for different materials and they all deviate from Hooke's Law. Remember that a straight line show proportionality so as soon as the graph is no longer a straight line, Hooke's Law no longer applies.

\subsubsection{Brittle material}


\begin{figure}[H] 
\begin{center}
\scalebox{1} % Change this value to rescale the drawing.
{
\begin{pspicture}(0,-1.8475)(4.3,1.8275)
\psline[linewidth=0.04cm](0.31546876,1.8075)(0.31546876,-1.2925)
\psline[linewidth=0.04cm](0.31546876,-1.3925)(4.28,-1.3925)
\psbezier[linewidth=0.04](0.31546876,-1.4227326)(0.41546875,-0.8925)(0.8559975,0.9618657)(0.9154688,1.1726162)(0.9749401,1.3833667)(1.1154687,1.5075)(1.3154687,1.5075)
\rput{-270.0}(0.55328125,0.29328126){\rput(0.13,0.41828126){Force}}
\rput(1.3779687,-1.6825){extension}
\psline[linewidth=0.04cm](1.18,1.6675)(1.48,1.3875)
\psline[linewidth=0.04cm](1.2160971,1.3720329)(1.4839028,1.6829672)
\rput(1.72,1.2275){\footnotesize fracture}
\end{pspicture} 
}
\caption {A hard, brittle substance}
\end{center}
\end{figure}


This graph shows the relationship between force and extension for a brittle, but strong material. Note that there is very little extension for a large force but then the material suddenly fractures.
Brittleness is the property of a material that makes it break easily without bending. 

Have you ever dropped something made of glass and seen it shatter? Glass does this because it is brittle.

\subsubsection{Plastic material}

\begin{figure}[H]
\begin{center}
\scalebox{1} % Change this value to rescale the drawing.
{
\begin{pspicture}(0,-1.7925)(3.6339064,1.7725)
\psline[linewidth=0.04cm](0.31390625,1.7525)(0.31390625,-1.3475)
\psline[linewidth=0.04cm](0.31390625,-1.3475)(3.6139061,-1.3475)
\usefont{T1}{ptm}{m}{n}
\rput(2.1764061,-1.6375){extension}
\usefont{T1}{ptm}{m}{n}
\rput{-270.0}(0.5909375,0.3475){\rput(0.11578125,0.4625){Force}}
\psbezier[linewidth=0.04](0.31390625,-1.3475)(1.3139062,-0.6475)(2.4139063,-0.6475)(3.3139062,-0.5475)
\end{pspicture} 
}
\caption{A plastic material's response to an applied force.}
\end{center}
\end{figure}

Here the graph shows the relationship between force and extension for a plastic material. The material extends under a small force but it does not fracture easily, and it does not return to its original length when the force is removed.


\subsubsection{Ductile material}

\begin{figure}[H]
\begin{center}
\scalebox{1} % Change this value to rescale the drawing.
{
\begin{pspicture}(0,-1.7975)(4.78,1.7775)
\psline[linewidth=0.04cm](0.31546876,1.7575)(0.31546876,-1.3425)
\psline[linewidth=0.04cm](0.31546876,-1.3425)(4.3154683,-1.3425)
\psbezier[linewidth=0.04](0.31546876,-1.3425)(0.9,-0.4025)(1.4394965,0.36422902)(1.94,0.8775)(2.4405034,1.390771)(4.0396514,1.2796583)(4.06,1.2775)
\rput(2.1779685,-1.6325){extension}
\rput{-270.0}(0.60328126,0.34328124){\rput(0.13,0.46828124){Force}}
\psline[linewidth=0.04cm](3.92,1.4175)(4.24,1.1175)
\psline[linewidth=0.04cm](4.22,1.4175)(3.94,1.1375)
\rput(4.26,1.5975){\footnotesize fracture}
\rput(2.98,0.9625){\scriptsize plastic region}
\rput(1.67,-0.6375){\scriptsize elastic region}
\end{pspicture} 
}

\caption{A ductile substance.}
\end{center}
\end{figure}

In this graph the relationship between force and extension is for a material that is ductile. The material shows plastic behaviour over a range of forces before the material finally fractures. Ductility is the ability of a material to be stretched into a new shape without breaking. Ductility is one of the characteristic properties of metals.

A good example of this is aluminium, many things are made of aluminium. Aluminium is used for making everything from cooldrink cans to aeroplane parts and even engine blocks for cars. Think about squashing and bending a cooldrink can.

Brittleness is the opposite of ductility.


 
When a material reaches a point where Hooke's Law is no longer valid, we say it has reached its \emph{limit of proportionality}. After this point, the material will not return to its original shape after the force has been removed. We say it has reached its \emph{elastic limit}.

\Definition{Elastic limit}{The elastic limit is the point beyond which permanent deformation takes place.}

\Definition{Limit of proportionality}{The limit of proportionality is the point beyond which Hooke's Law is no longer obeyed.}


\Exercise{Hooke's Law and deformation of materials}{
\begin{enumerate}
\item What causes deformation?
\item Describe Hooke's Law in words and mathematically.
\item List similarities and differences between ductile, brittle and plastic (polymeric) materials, with specific reference to their force-extension graphs.
\item Describe what is meant by the \textit{elastic limit}.
\item Describe what is meant by the \textit{limit of proportionality}.
\item A spring of length 15~cm stretches to 27~cm when a load of 0,4~N is applied to it.
\begin{enumerate}
\item Calculate the spring constant for the spring.
\item Determine the extension of the spring if a load of 0,35~N is applied to it.
\end{enumerate}
\item A spring has a spring constant of $-200$~N.m$^{-1}$. By how much will it stretch if a load of 25~N is applied to it?
\item A spring of length 20~cm stretches to 24~cm when a load of 0,6~N is applied to it.
\begin{enumerate}
\item Calculate the spring constant for the spring.\\
\item Determine the extension of the spring if a load of 0,8~N is applied to it.\\
\end{enumerate}
\end{enumerate}

% Automatically inserted shortcodes - number to insert 8
\par \practiceinfo
\par \begin{tabular}[h]{cccccc}
% Question 1
(1.)	01sv	&
% Question 2
(2.)	01sw	&
% Question 3
(3.)	01sx	&
% Question 4
(4.)	01sy	&
% Question 5
(5.)	01sz	&
% Question 6
(6.)	01t0	\\ % End row of shortcodes
% Question 7
(7.)	01t1	&
% Question 8
(8.)	01t2	&
\end{tabular}
% Automatically inserted shortcodes - number inserted 8
}

\section{Elasticity, plasticity, fracture, creep}
%\begin{syllabus}
%\item The learner must be able to Compare and contrast elastic and plastic deformation
%\item The learner must be able to Compare and contrast creep and fracture as modes of failure in materials.
%\end{syllabus}

\subsection{Elasticity and plasticity}

Materials are classified as plastic or elastic depending on how they respond to an applied force. It is important to note that plastic substances are not necessarily a type of plastic (polymer) they only behave like plastic. Think of them as being like plastic which you will be familiar with.

A rubber band is a material that has elasticity. It returns to its original shape after an applied force is removed, providing that the material is not stretched beyond its elastic limit.

Plasticine is an example of a material that is plastic. If you flatten a ball of plasticine, it will stay flat. A plastic material does not return to its original shape after an applied force is removed.

\begin{itemize}
\item Elastic materials return to their original shape.
\item Plastic materials deform easily and do not return to their original shape.
\end{itemize}

\subsection {Fracture, creep and fatigue}

Some materials are neither plastic nor elastic. These substances will break or fracture when a large enough force is applied to them. The brittle glass we mentioned earlier is an example.
 
Creep occurs when a material deforms over a long period of time because of an applied force. An example of creep is the bending of a shelf over time when a heavy object is put on it. Creep may eventually lead to the material fracturing. The application of heat may lead to an increase in creep in a material.

Fatigue is similar to creep. The difference between the two is that fatigue results from the force being applied and then removed repeatedly over a period of time. With metals this results in failure because of metal fatigue. 

\begin{itemize}
\item Fracture is an abrupt breaking of the material.
\item Creep is a slow deformation process due to a continuous force over a long time.
\item Fatigue is weakening of the material due to short forces acting many many times.
\end{itemize}

\Exercise{Elasticity, plasticity, fracture and creep}{
\begin{enumerate}
\item List the similarities and differences between elastic and plastic deformation.
\item List the similarities and differences between creep and fracture as modes of failure in material.
\end{enumerate}
% Automatically inserted shortcodes - number to insert 2
\par \practiceinfo
\par \begin{tabular}[h]{cccccc}
% Question 1
(1.)	01t3	&
% Question 2
(2.)	01t4	&
\end{tabular}
% Automatically inserted shortcodes - number inserted 2

}

\section{Failure and strength of materials}
%\begin{syllabus}
%\item The learner must be able to Compare and contrast the brittle and ductile modes of failure
%\item The learner must be able to Explain the behaviour and properties of materials by using an understanding of structure of materials. Terms should include vacancies, dislocations, impurities, and terms like grain boundaries and slip planes
%\item The learner must be able to Understand that a material's mechanical properties can be described in terms of ductile, malleable, tough and elastic.
%\item The learner must be able to Demonstrate understanding of how a material's mechanical properties can be controlled by cold working, annealing, tempering, and introduction of impurities, alloying and sintering.
%\end{syllabus}

\subsection{The properties of matter}

The strength of a material is defined as the stress (the force per unit cross-sectional area) that it can withstand. Strength is measured in newtons per square metre ($N \cdot m^{-2}$).

\emph{Stiffness} is a measure of how flexible a material is. In Science we measure the stiffness of a material by calculating its Young's Modulus. The Young's modulus is a ratio of how much it bends to the load applied to it. Stiffness is measure in newtons per metre ($N \cdot m^{-1}$).
 
\emph{Hardness} of a material can be measured by determining what force will cause a permanent deformation in the material. Hardness can also be measured using a scale like Mohs hardness scale. On this scale, diamond is the hardest at 10 and talc is the softest at 1.

\begin{IFact}{Remembering that the Mohs scale is the hardness scale and that the softest substance is talc will often come in handy for general knowledge quizzes.}\end{IFact}

The \emph{toughness} of a material is a measure of how it can resist breaking when it is stressed. It is scientifically defined as the amount of energy that a material can absorb before fracturing.

A ductile material is a substance that can undergo large plastic deformation without fracturing. Many metals are very ductile and they can be drawn into wires, e.g. copper, silver, aluminium and gold.

A \emph{malleable} material is a substance that can easily undergo plastic deformation by hammering or rolling. Again, metals are malleable substances, e.g. copper can be hammered into sheets and aluminium can be rolled into aluminium foil.
 
A brittle material fractures with very little or no plastic deformation. Glassware and ceramics are brittle.

\subsection{Structure and failure of materials}

Many substances fail because they have a weakness in their atomic structure. There are a number of problems that can cause these weaknesses in structure. These are vacancies, dislocations, grain boundaries and impurities.

\emph{Vacancies} occur when there are spaces in the structure of a crystalline solid. These vacancies cause weakness and such materials often fail at these places. Think about bricks in a wall, if you started removing bricks the wall would get weaker.

\emph{Dislocations} result in weakened bonding between layers of atoms in a crystal lattice and this creates a critical boundary. If sufficient force is applied along the boundary, it can break the weakened bonds, allowing the two sides of the crystal to slide against one another. The two pieces of the crystal keep their shape and structure.

\emph{Impurities} in a crystal structure can cause a weak region in the crystal lattice around the impurity. Like vacancies, the substance often fail from these places in the lattice. This you can think of as bricks in a wall which don't fit properly, they are the wrong kind of bricks (atoms) to make the structure strong.

%A difference in \emph{grain size} in a crystal lattice will result in rusting or oxidation at the boundary which again will result in failure when sufficient force is applied. 

\subsection{Controlling the properties of materials}

There are a number of processes that can be used to make materials less likely to fail. We shall look at a few methods in this section.
\subsection*{Cold working}

Cold working is a process in which a metal is \emph{strengthened} by repeatedly being reshaped. This is carried out at a temperature below the melting point of the metal. The repeated shaping of the metal results in dislocations which then prevent restrict the motion of dislocations in the metal. Cold working increases the strength of the metal but in so doing, the metal loses its ductility. We say the metal is \emph{work-hardened}.\\

\subsection*{Annealing}
Annealing is a process of heating and cooling a material to relax the crystal structure and reduce weakness due to impurities and structural flaws. 
During annealing, the material is heated to a high temperature that is below the material's melting point. At a sufficiently high temperature, atoms with weakened bonds can rearrange themselves into a stronger structure. Slowly cooling the material ensures that the atoms will remain in these stronger locations. Annealing is often used before cold working. 
 
\subsection*{Introduction of Impurities}
Most pure metals are relatively weak because dislocations can move easily within them. However, if impurities are added to a metal (e.g., carbon is added to an iron sample), they can disturb the regular structure of the metal and so prevent dislocations from spreading. This makes the metal stronger. 
 
\subsection*{Alloying}
An alloy is a mixture of a metal with other substances. In other words, alloying involves adding impurities to a metal sample. The other substances can be metal or non-metal. An alloy often has properties that are very different to the properties of the substances from which it is made. The added substances strengthen the metal by preventing dislocations from spreading. Ordinary steel is an alloy of iron and carbon. The carbon impurities trap dislocations. There are many types of steel that also include other metals with iron and carbon. Brass is an alloy of copper and Zinc. Bronze is an alloy of copper and tin. Gold and silver that is used in coins or jewellery are also alloyed.

\subsection*{Tempering}

Tempering is a process in which a metal is melted then quickly cooled. The rapid cooling is called quenching. Usually tempering is done a number of times before a metal has the correct properties that are needed for a particular application. 

\subsection*{Sintering}

Sintering is used for making ceramic objects among other things. In this process the substance is heated so that its particles stick together. It is used with substances that have a very high melting point. The resulting product is often very pure and it is formed in the process into the shape that is wanted. Unfortunately, sintered products are brittle.

\subsection{Steps of Roman Swordsmithing}
\begin{itemize}
\item Purifying the iron ore. 
\item Heating the iron blocks in a furnace with charcoal. 
\item Hammering and getting into the needed shape. The smith used a hammer to pound the metal into blade shape. He usually used tongs to hold the iron block in place. 
\item Reheating. When the blade cooled, the smith reheated it to keep it workable. While reheated and hammered repeatedly.
\item \emph{Quenching} which involved the process of white heating and cooling in water. Quenching made the blade harder and stronger. At the same time it made the blade quite brittle, which was a considerable problem for the sword smiths.
\item \emph{Tempering} was then done to avoid brittleness the blade was tempered. In another words it was reheated a final time to a very specific temperature. How the Romans do balanced the temperature? The smith was guided only by the blade's colour and his own experience. 
\end{itemize}

\Exercise{Failure and strength of materials}{
\begin{enumerate}
\item List the similarities and differences between the brittle and ductile modes of failure.
\item What is meant by the following terms:
\begin{enumerate}
\item vacancies
\item dislocations
\item impurities

\item grain boundaries
\end{enumerate}
\item What four terms can be used to describe a material's mechanical properties?
\item What is meant by the following:
\begin{enumerate}
\item cold working
\item annealing
\item tempering
\item introduction of impurities
\item alloying
\item sintering
\end{enumerate}
\end{enumerate}

% Automatically inserted shortcodes - number to insert 4
\par \practiceinfo
\par \begin{tabular}[h]{cccccc}
% Question 1
(1.)	01t5	&
% Question 2
(2.)	01t6	&
% Question 3
(3.)	01t7	&
% Question 4
(4.)	01t8	&
\end{tabular}
% Automatically inserted shortcodes - number inserted 4
}

\summary{VPnqj}
\begin{enumerate}
\item Hooke's Law gives the relationship between the extension of a spring and the force applied to it. The law says they are proportional.
\item Materials can be classified as plastic or elastic depending on how they respond to an applied force. 
\item Materials can fracture or undergo creep or fatigue when forces are applied to them.
\item Materials have the following mechanical properties to a greater or lesser degree: strength, hardness, ductility, malleability, brittleness, stiffness.
\item Materials can be weakened by have the following problems in their crystal lattice: vacancies, dislocations, impurities, difference in grain size.
\item Materials can have their mechanical properties improved by one or more of the following processes: cold working, annealing, adding impurities, tempering, sintering.

\end{enumerate}

\begin{eocexercises}{}
\begin{enumerate}

\item State Hooke's Law in words.

\item What do we mean by the following terms with respect to Hooke's Law?
\begin{enumerate}
\item elastic limit
\item limit of proportionality
\end{enumerate}

\item A spring is extended by 18~cm by a force of 90~N. Calculate the spring constant for this spring.

\item A spring of length 8~cm stretches to 14~cm when a load of 0,8~N is applied to it.
\begin{enumerate}
\item Calculate the spring constant for the spring.
\item Determine the extension of the spring if a load of 0,7~N is applied to it.
\end{enumerate}

\item A spring has a spring constant of $-150$~N.m$^{-1}$. By how much will it stretch if a load of 80~N is applied to it?

\item What do we mean by the following terms when speaking about properties of materials?
\begin{enumerate}
\item hardness
\item toughness
\item ductility
\item malleability
\item stiffness
\item strength
\end{enumerate}

\item What is Young's modulus?

\item In what different ways can we improve the material properties of substances?

\item What is a metal alloy?

\item What do we call an alloy of:
\begin{enumerate}
\item iron and carbon
\item copper and zinc 
\item copper and tin
\end{enumerate}

\item Do some research on what added substances can do to the properties of steel. Present your findings in a suitable table.
 
\end{enumerate}

% Automatically inserted shortcodes - number to insert 11
\par \practiceinfo
\par \begin{tabular}[h]{cccccc}
% Question 1
(1.)	01t9	&
% Question 2
(2.)	01ta	&
% Question 3
(3.)	01tb	&
% Question 4
(4.)	01tc	&
% Question 5
(5.)	01td	&
% Question 6
(6.)	01te	\\ % End row of shortcodes
% Question 7
(7.)	01tf	&
% Question 8
(8.)	01tg	&
% Question 9
(9.)	01th	&
% Question 10
(10.)	01ti	&
% Question 11
(11.)	01tj	&
\end{tabular}
% Automatically inserted shortcodes - number inserted 11

\end{eocexercises}
% CHILD SECTION END 



% CHILD SECTION START 

