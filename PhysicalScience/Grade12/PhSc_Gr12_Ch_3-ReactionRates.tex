\chapter{Reaction Rates}
\label{chap:ReactionRates}


% CHILD SECTION START 

\section{Introduction}

Before we begin this section, it might be useful to think about some different types of reactions and how quickly or slowly they occur.

\Exercise{Thinking about reaction rates\\}{

Think about each of the following reactions:

\begin{itemize}
\item{rusting of metals}
\item{photosynthesis}
\item{weathering of rocks (e.g. limestone rocks being weathered by water)}
\item{combustion}
\end{itemize}

\begin{enumerate}
\item{For each of the reactions above, write a chemical equation for the reaction that takes place.}
\item{How fast is each of these reactions? Rank them in order from the fastest to the slowest.}
\item{How did you decide which reaction was the fastest and which was the slowest?}
\item{Try to think of some other examples of chemical reactions. How fast or slow is each of these reactions, compared with those listed earlier?}
\end{enumerate}
}

In a chemical reaction, the substances that are undergoing the reaction are called the \textbf{reactants}, while the substances that form as a result of the reaction are called the \textbf{products}. The \textbf{reaction rate} describes how quickly or slowly the reaction takes place. So how do we know whether a reaction is slow or fast? One way of knowing is to look either at how quickly the \textit{reactants are used up} during the reaction or at how quickly the \textit{product forms}. For example, iron and sulfur react according to the following equation:

\begin{center}
\rm${Fe + S \rightarrow FeS}$
\end{center}

In this reaction, we can observe the speed of the reaction by measuring how long it takes before there is no iron or sulfur left in the reaction vessel. In other words, the reactants have been used up. Alternatively, one could see how quickly the iron sulfide product forms. Since iron sulfide looks very different from either of its reactants, this is easy to do. \\

In another example:

\begin{eqnarray*}
2Mg(s) + O_{2} \rightarrow 2Mg O(s)
\end{eqnarray*}

In this case, the reaction rate depends on the speed at which the reactants (solid
magnesium and oxygen gas) are used up, or the speed at which the product (magnesium oxide) is formed.

\Definition{Reaction rate}{The \textit{rate of a reaction} describes how quickly \textit{reactants are used up} or how quickly \textit{products are formed} during a chemical reaction. The units used are: moles per second (mols/second or mol.s$^{-1}$).
}

The average rate of a reaction is expressed as the number of
moles of reactant used up, divided by the total reaction time, or as the number of moles of product formed, divided by the reaction time. Using the magnesium reaction shown earlier:

\begin{equation*}
Average \ reaction \ rate \ of \ Mg = \frac{moles \ Mg \ used}{reaction \ time \ (s)}
\end{equation*}

\begin{center}
or
\end{center}

\begin{equation*}
Average \ reaction \ rate \ of \ O_{2} = \frac{moles \ O_{2} \ used}{reaction \ time \ (s)}
\end{equation*}

\begin{center}
or
\end{center}

\begin{equation*}
Average \ reaction \ rate \ of \ MgO = \frac{moles \ MgO \ produced}{reaction \ time \ (s)}
\end{equation*}

\begin{wex}{Reaction rates}{The following reaction takes place: 
\begin{center}
\rm${4Li + O_{2} \rightarrow 2Li_{2}O}$
\end{center}

After two minutes , 4 g of Lithium has been used up. Calculate the rate of the reaction.\\
}

{\westep{Calculate the number of moles of Lithium that are used up in the reaction.}

\begin{equation*}
n = \frac{m}{M} = \frac{4}{6.94} = 0.58 mols 
\end{equation*}
\westep{Calculate the time (in seconds) for the reaction.}

\begin{equation*}
t = 2 \times 60 = 120 s
\end{equation*}
\westep{Calculate the rate of the reaction.}

Rate of reaction = \begin{equation*} \frac{moles \ of \ Lithium \ used}{time} = \frac{0.58}{120} = 0.005\end{equation*} 

The rate of the reaction is 0.005 mol.s$^{-1}$
}\end{wex}

\Exercise{Reaction rates\\}{
\begin{enumerate}
\item{
A number of different reactions take place. The table below shows the number of moles of reactant that are used up in a particular time for each reaction.

\begin{center}
\begin{tabular}{|c|c|c|c|}\hline
\textbf{Reaction} & \textbf{Mols used up} & \textbf{Time} & \textbf{Reaction rate}\\\hline
1 & 2 & 30 secs & \\\hline
2 & 5 & 2 mins & \\\hline
3 & 1 & 1.5 mins & \\\hline
4 & 3.2 & 1.5 mins & \\\hline
5 & 5.9 & 30 secs & \\\hline
\end{tabular}
\end{center}

	\begin{enumerate}
	\item{Complete the table by calculating the rate of each reaction.}
	\item{Which is the \textit{fastest} reaction?}
	\item{Which is the \textit{slowest} reaction?}
	\end{enumerate}
}

\item{
Two reactions occur simultaneously in separate reaction vessels. The reactions are as follows:\\
\begin{center}
\rm${Mg + Cl_{2} \rightarrow MgCl_{2}}$

\rm${2Na + Cl_{2} \rightarrow 2NaCl}$
\end{center}

After 1 minute, 2 g of MgCl$_{2}$ have been produced in the first reaction.

	\begin{enumerate}
	\item{How many moles of MgCl$_{2}$ are produced after 1 minute?}
	\item{Calculate the rate of the reaction, using the amount of product that is produced.}
	\item{Assuming that the second reaction also proceeds at the same rate, calculate...}
		\begin{enumerate}
		\item{the number of moles of NaCl produced after 1 minute.}
		\item{the mass (in g) of sodium that is needed for this reaction to take place.}
		\end{enumerate}
	\end{enumerate}	
}
\end{enumerate}
}



% CHILD SECTION END 



% CHILD SECTION START 

\section{Factors affecting reaction rates}
\label{sec:reactionrates:factors affecting}
Several factors affect the rate of a reaction. It is important to know
these factors so that reaction rates can be controlled. This is particularly important when it comes to industrial reactions, so that productivity can be maximised. The following are some of the factors that affect the rate of a reaction.

\begin{enumerate}
\item{\textbf{Nature of reactants}}

Substances have different chemical properties and therefore react differently and at different rates.

\item{\textbf{Concentration} (or \textbf{pressure} in the case of gases)}

As the concentration of the reactants increases, so does the reaction rate.

\item{\textbf{Temperature}}

If the temperature of the reaction increases, so does the rate of the reaction. 
	
\item{\textbf{Catalyst}} 

Adding a catalyst increases the reaction rate.

\item{\textbf{Surface area of solid reactants}} 

Increasing the surface area of the reactants (e.g. if a solid reactant is broken or ground up into smaller pieces) will increase the reaction rate. 
\end{enumerate}  

\Activity{Experiment}{The nature of reactants.\\}{

\Aim{

To determine the effect of the nature of reactants on the rate of a reaction.}

\Apparatus{

Oxalic acid ((COOH)$_{2}$), iron(II) sulphate (FeSO$_{4}$), potassium permanganate (KMnO$_{4}$), concentrated sulfuric acid (H$_{2}$SO$_{4}$), spatula, test tubes, medicine dropper, glass beaker and glass rod.}

\begin{center}
\begin{pspicture}(-4,-1)(4,2.5)
%\psgrid[gridcolor=lightgray]
\rput(-1,0){
%\psset{unit=2}

\def\testtube{\psarc(0.2,0){0.2}{180}{0}\psline(0,0)(0,2)\psline(0.4,0)(0.4,2)}
\def\chip{\psframe[fillstyle=solid,fillcolor=lightgray](0,0)(0.1,0.1)}
\rput(-0.4,0){\rput(0,0.2){\psarc[fillstyle=solid,fillcolor=lightgray,linestyle=none](0.2,0){0.2}{180}{0}\psframe[fillstyle=solid,fillcolor=lightgray,linestyle=none](0,0)(0.4,0.6)\testtube}

\uput[d](0,0){Test tube 1}
\uput[d](0,-0.3){Iron (II) sulphate solution}}

\rput(2,0){\rput(1,0.2){\psarc[fillstyle=solid,fillcolor=lightgray,linestyle=none](0.2,0){0.2}{180}{0}\psframe[fillstyle=solid,fillcolor=lightgray,linestyle=none](0,0)(0.4,0.6)\testtube
}
\uput[d](1.4,0){Test tube 2}
\uput[d](2,-0.3){Oxalic acid solution}}
}

\psline[linearc=0.25]{->}(-2.1,2.2)(-1.6,2.4)(-1.1,2.2)
\rput(-3,1.8){H$_{2}$SO$_{4}$}
\rput(-3,1.4){KMnO$_{4}$}

\psline[linearc=0.25]{<-}(2.2,2.2)(1.7,2.4)(1.2,2.2)
\rput(0.7,1.8){H$_{2}$SO$_{4}$}
\rput(0.7,1.4){KMnO$_{4}$}


\end{pspicture}
\end{center}

\Method{

\begin{enumerate}
\item{In the first test tube, prepare an iron (II) sulphate solution by dissolving about two spatula points of iron (II) sulphate in 10 cm$^{3}$ of water.}
\item{In the second test tube, prepare a solution of oxalic acid in the same way.}
\item{Prepare a solution of sulfuric acid by adding 1 cm$^{3}$ of the concentrated acid to about 4 cm$^{3}$ of water. Remember always to add the \textit{acid to the water}, and never the other way around.}
\item{Add 2 cm$^{3}$ of the sulfuric acid solution to the iron(II)  and oxalic acid solutions respectively.}
\item{Using the medicine dropper, add a few drops of potassium permanganate to the two test tubes. Once you have done this, observe how quickly each solution discolours the potassium permanganate solution.\\} 
\end{enumerate}
}

\Results{

\begin{itemize}
\item{You should have seen that the oxalic acid solution discolours the potassium permanganate much more slowly than the iron(II) sulphate.}
\item{It is the oxalate ions (C$_{2}$O$_{4}^{2-}$) and the Fe$^{2+}$ ions that cause the discolouration. It is clear that the Fe$^{2+}$ ions act much more quickly than the C$_{2}$O$_{4}^{2-}$ ions. The reason for this is that there are no covalent bonds to be broken in the ions before the reaction can take place. In the case of the oxalate ions, covalent bonds between carbon and oxygen atoms must be broken first.}
\end{itemize}
}

\Conclusions{

The nature of the reactants can affect the rate of a reaction.}
}

\begin{IFact}{Oxalic acids are abundant in many plants. The leaves of the tea plant (\textit{Camellia sinensis}) contain very high concentrations of oxalic acid relative to other plants. Oxalic acid also occurs in small amounts in foods such as parsley, chocolate, nuts and berries. Oxalic acid irritates the lining of the gut when it is eaten, and can be fatal in very large doses.}
\end{IFact}

\Activity{Experiment}{Surface area and reaction rates.\\}{

Marble ($CaCO_{3}$) reacts with hydrochloric acid (HCl) to form calcium chloride, water and carbon dioxide gas according to the following equation:\\

\begin{center}
\rm${CaCO_{3} + 2HCl \rightarrow CaCl_{2} + H_{2}O + CO_{2}}$\\
\end{center}

\Aim{

To determine the effect of the surface area of reactants on the rate of the reaction.}

\Apparatus{

2 g marble chips, 2 g powdered marble, hydrochloric acid, beaker, two test tubes.}

\begin{center}
\begin{pspicture}(-1,-1)(4,5)
%\psgrid[gridcolor=gray]
%\psset{unit=2}

\def\testtube{\psarc(0.2,0){0.2}{180}{0}\psline(0,0)(0,2)\psline(0.4,0)(0.4,2)}
\def\chip{\psframe[fillstyle=solid,fillcolor=lightgray](0,0)(0.1,0.1)}
\rput(-0.4,0){\rput(0,0.2){\testtube}
\rput(0.2,0.01){\chip}
\rput{45}(0.1,0.05){\chip}
\rput{30}(0.3,0.05){\chip}
\rput{60}(0.2,0.2){\chip}
\rput{90}(0.35,0.2){\chip}
\rput{96}(0.15,0.2){\chip}
\rput{120}(0.3,0.1){\chip}
\uput[d](0,0){Test tube 1}
\uput[d](0,-0.3){marble chips}}

\rput(2,0){\rput(1,0.2){\psarc[fillstyle=solid,fillcolor=lightgray,linestyle=none](0.2,0){0.2}{180}{0}\psframe[fillstyle=solid,fillcolor=lightgray,linestyle=none](0,0)(0.4,0.6)\testtube
}
\uput[d](1,0){Test tube 2}
\uput[d](1,-0.3){powdered marble}}

\rput(0.75,2.6){\filledbeaker}
\uput[r](2.4,3.6){\parbox[l]{3.5cm}{beaker containing dilute hydrochloric acid}}


\end{pspicture}
\end{center}

\Method{

\begin{enumerate}
\item{Prepare a solution of hydrochloric acid in the beaker by adding 2 cm$^{3}$ of the concentrated solution to 20 cm$^{3}$ of water.}
\item{Place the marble chips and powdered marble into separate test tubes.}
\item{Add 10 cm$^{3}$ of the dilute hydrochloric acid to each of the test tubes and observe the rate at which carbon dioxide gas is produced.\\}
\end{enumerate}
}

\Results{

\begin{itemize}
\item{Which reaction proceeds the fastest?}
\item{Can you explain this?}
\end{itemize}
}

\Conclusions{

The reaction with powdered marble is the fastest. The smaller the pieces of marble are, the greater the surface area for the reaction to take place. The greater the surface area of the reactants, the faster the reaction rate will be.}
}

\Activity{Experiment}{Reactant concentration and reaction rate.}{

\Aim{

To determine the effect of reactant concentration on reaction rate.}

\Apparatus{

Concentrated hydrochloric acid (HCl), magnesium ribbon, two beakers, two test tubes, measuring cylinder.}

\Method{

\begin{enumerate}
\item{Prepare a solution of dilute hydrochloric acid in one of the beakers by diluting 1 part concentrated acid with 10 parts water. For example, if you measure 1 cm$^{3}$ of concentrated acid in a measuring cylinder and pour it into a beaker, you will need to add 10 cm$^{3}$ of water to the beaker as well. In the same way, if you pour 2 cm$^{3}$ of concentrated acid into a beaker, you will need to add 20 cm$^{3}$ of water. Both of these are 1:10 solutions. Pour 10 cm$^{3}$ of the 1:10 solution into a test tube and mark it 'A'. Remember to add the \textit{acid} to the \textit{water}, and not the other way around.} 
\item{Prepare a second solution of dilute hydrochloric acid by diluting 1 part concentrated acid with 20 parts water. Pour 10cm$^{3}$ of this 1:20 solution into a second test tube and mark it 'B'.}
\item{Take two pieces of magnesium ribbon of the \textbf{same length}. At the same time, put one piece of magnesium ribbon into test tube A and the other into test tube B, and observe closely what happens.\\}
\end{enumerate}

\begin{center}
\begin{pspicture}(-4,-1.3)(4,2.5)
%\psgrid[gridcolor=lightgray]
\rput(-1,0){
%\psset{unit=2}

\def\testtube{\psarc(0.2,0){0.2}{180}{0}\psline(0,0)(0,2)\psline(0.4,0)(0.4,2)}
\def\chip{\psframe[fillstyle=solid,fillcolor=lightgray](0,0)(0.1,0.1)}
\rput(-0.4,0){\rput(0,0.2){\psarc[fillstyle=solid,fillcolor=lightgray,linestyle=none](0.2,0){0.2}{180}{0}\psframe[fillstyle=solid,fillcolor=lightgray,linestyle=none](0,0)(0.4,0.6)\testtube}

\uput[d](0,0){Test tube A}
\uput[d](0,-0.3){1:10 HCl solution}}

\rput(2,0){\rput(1,0.2){\psarc[fillstyle=solid,fillcolor=lightgray,linestyle=none](0.2,0){0.2}{180}{0}\psframe[fillstyle=solid,fillcolor=lightgray,linestyle=none](0,0)(0.4,0.6)\testtube
}
\uput[d](1,0){Test tube B}
\uput[d](1,-0.3){1:20 HCl solution}}
}

\psline[linearc=0.25]{->}(-2.1,2.2)(-1.6,2.4)(-1.1,2.2)
\psframe(-3,1.7)(-2,2)
\rput(-2.5,1.2){Mg ribbon}

\psline[linearc=0.25]{<-}(2.2,2.2)(1.7,2.4)(1.2,2.2)
\psframe(0.2,1.7)(1.2,2)
\rput(0.5,1.2){Mg ribbon}
\end{pspicture}
\end{center}

The equation for the reaction is:

\begin{center}
\rm${2HCl + Mg \rightarrow MgCl_{2} + H_{2}}$\\
\end{center}
}

\Results{

\begin{itemize}
\item{Which of the two solutions is more concentrated, the 1:10 or 1:20 hydrochloric acid solution?}
\item{In which of the test tubes is the reaction the fastest? Suggest a reason for this.}
\item{How can you measure the rate of this reaction?}
\item{What is the gas that is given off?}
\item{Why was it important that the same length of magnesium ribbon was used for each reaction?}
\end{itemize}  
}

\Conclusions{

The 1:10 solution is more concentrated and this reaction therefore proceeds faster. The greater the concentration of the reactants, the faster the rate of the reaction. The rate of the reaction can be measured by the rate at which hydrogen gas is produced.}
}

\Activity{Group work}{The effect of temperature on reaction rate\\}{

\begin{enumerate}
\item{
In groups of 4-6, design an experiment that will help you to see the effect of temperature on the reaction time of 2 cm of magnesium ribbon and 20 ml of vinegar. During your group discussion, you should think about the following:

\begin{itemize}
\item{What equipment will you need?}
\item{How will you conduct the experiment to make sure that you are able to compare the results for different temperatures?}
\item{How will you record your results?}
\item{What safety precautions will you need to take when you carry out this experiment?}
\end{itemize}
}
\item{Present your experiment ideas to the rest of the class, and give them a chance to comment on what you have done.}
\item{Once you have received feedback, carry out the experiment and record your results.}
\item{What can you conclude from your experiment?}
\end{enumerate} 
}



% CHILD SECTION END 



% CHILD SECTION START 

\section{Reaction rates and collision theory}

It should be clear now that the rate of a reaction varies depending on a number of factors. But how can we \textit{explain} why reactions take place at different speeds under different conditions? Why, for example, does an increase in the surface area of the reactants also increase the rate of the reaction? One way to explain this is to use \textbf{collision theory.}\\
 
For a reaction to occur, the particles that are reacting must collide with one another. Only a fraction of all the collisions that take place actually cause a chemical change. These are called 'successful' collisions. When there is an increase in the \textit{concentration} of reactants, the chance that reactant particles will collide with each other also increases because there are more particles in that space. In other words, the \textit{collision frequency} of the reactants increases. The number of \textit{successful} collisions will therefore also increase, and so will the rate of the reaction. In the same way, if the \textit{surface area} of the reactants increases, there is also a greater chance that successful collisions will occur.

\Definition{Collision theory}{Collision theory is a theory that explains how chemical reactions occur and why reaction rates differ for different reactions. The theory states that for a reaction to occur the reactant particles must collide, but that only a certain fraction of the total collisions, the \textit{effective collisions}, actually cause the reactant molecules to change into products. This is because only a small number of the molecules have enough energy and the right orientation at the moment of impact to break the existing bonds and form new bonds.}

When the \textit{temperature} of the reaction increases, the average kinetic energy of the reactant particles increases and they will move around much more actively. They are therefore more likely to collide with one another (Figure \ref{fig:reactionrates:collision}). Increasing the temperature also increases the number of particles whose energy will be greater than the activation energy for the reaction (refer section \ref{sec:reactionrates:energy}).

\begin{figure}[htbp]
\begin{center}
\begin{pspicture}(0,-0.8)(9.2,4.4)
\SpecialCoor
%\psgrid[gridcolor=lightgray]
\def\fe{\pscircle(0,0){0.4}\rput(0,0){A}\psline{->}(0.4,0)(0.6,0)}
\def\s{\pscircle(0,0){0.2}\rput(0,0){B}\psline{->}(0.2,0)(0.4,0)}

\psframe(0,0)(4.2,4.2)
\rput(1.5,2){\s}
\rput{30}(3,1){\rput(1,1){\fe}\rput(1.5,2){\s}}
\rput{65}(1.55,1.33){\rput(1,1){\fe}\rput(1.5,2){\s}}
\rput{265}(2,3){\rput(-0.5,0.6){\fe}\rput(1.5,1.6){\s}}
\rput{27}(2,0.6){\s}
\rput{3}(1,1){\fe}
\rput{-20}(3,0.7){\fe}
\rput(2.4,2){\fe}
\rput{-80}(1.4,3.6){\s}
\rput(2,-0.5){Low Temperature}

\def\fe{\pscircle(0,0){0.4}\rput(0,0){A}\psline[linewidth=2pt]{->}(0.4,0)(0.8,0)}
\def\s{\pscircle(0,0){0.2}\rput(0,0){B}\psline[linewidth=2pt]{->}(0.2,0)(0.6,0)}

\rput(5,0){
\psframe(0,0)(4.2,4.2)
\rput(1.5,2){\s}
\rput{30}(3,1){\rput(1,1){\fe}\rput(1.5,2){\s}}
\rput{65}(1.55,1.33){\rput(1,1){\fe}\rput(1.5,2){\s}}
\rput{265}(2,3){\rput(-0.5,0.6){\fe}\rput(1.5,1.6){\s}}
\rput{27}(2,0.6){\s}
\rput{3}(1,1){\fe}
\rput{-20}(3,0.7){\fe}
\rput(2.4,2){\fe}
\rput{-80}(1.4,3.6){\s}
\rput(2,-0.5){{High Temperature}}
}

\end{pspicture}
\caption{An increase in the temperature of a reaction increases the chances that the reactant particles (A and B) will collide because the particles have more energy and move around more.}
\label{fig:reactionrates:collision}
\end{center}
\end{figure}


\Exercise{Rates of reaction\\}{Hydrochloric acid and calcium carbonate react according to the following equation:

\begin{center}
\rm${CaCO_{3} + 2HCl \rightarrow CaCl_{2} + H_{2}O + CO_{2}}$
\end{center}

The volume of carbon dioxide that is produced during the reaction is measured at different times. The results are shown in the table below.

\begin{center}
\begin{tabular}{|c|c|}\hline
\textbf{Time (mins)} & \textbf{Total Volume of CO$_{2}$ produced (cm$^{3}$)}\\\hline
1 & 14 \\\hline
2 & 26 \\\hline
3 & 36 \\\hline
4 & 44 \\\hline
5 & 50 \\\hline
6 & 58 \\\hline
7 & 65 \\\hline
8 & 70 \\\hline
9 & 74 \\\hline
10 & 77 \\\hline
\end{tabular}
\end{center}

\begin{center}
\textit{Note: On a graph of production against time, it is the \textit{gradient} of the graph that shows the rate of the reaction.}
\end{center}


\textbf{Questions:}

\begin{enumerate}
\item{Use the data in the table to draw a graph showing the volume of gas that is produced in the reaction, over a period of 10 minutes.}
\item{At which of the following times is the reaction \textit{fastest}? Time = 1 minute; time = 6 minutes or time = 8 minutes.}
\item{Suggest a reason why the reaction slows down over time.}
\item{Use the graph to estimate the volume of gas that will have been produced after 11 minutes.}
\item{After what time do you think the reaction will stop?}
\item{If the experiment was repeated using a more concentrated hydrochloric acid solution...}
	\begin{enumerate}
	\item{would the rate of the reaction increase or decrease from the one shown in the graph?}
	\item{draw a rough line on the graph to show how you would expect the reaction to proceed with a more concentrated HCl solution.}
	\end{enumerate}  
\end{enumerate}
}


% CHILD SECTION END 



% CHILD SECTION START 

\section{Measuring Rates of Reaction}
\label{sec:reactionrates:measuring}

How the rate of a reaction is measured will depend on what the reaction is, and what product forms. Look back to the reactions that have been discussed so far. In each case, how was the rate of the reaction measured? The following examples will give you some ideas about other ways to measure the rate of a reaction:

\begin{itemize}
\item{\textit{Reactions that produce hydrogen gas:} 

When a metal dissolves in an acid, hydrogen gas is produced. A lit splint can be used to test for hydrogen. The 'pop' sound shows that hydrogen is present. For example, magnesium reacts with sulfuric acid to produce magnesium sulphate and hydrogen.
}
\begin{center}
\rm${Mg(s) + H_{2}SO_{4} \rightarrow MgSO_{4} + H_{2}}$
\end{center}
 
\item{\textit{Reactions that produce carbon dioxide:} 

When a carbonate dissolves in an acid, carbon dioxide gas is produced. When carbon dioxide is passes through limewater, it turns the limewater milky. This is a simple test for the presence of carbon dioxide. For example, calcium carbonate reacts with hydrochloric acid to produce calcium chloride, water and carbon dioxide.
}
\begin{center}
\rm${CaCO_{3}(s) + 2HCl(aq) \rightarrow CaCl_{2}(aq) + H_{2}O(l) + CO_{2}(g)}$
\end{center}

\item{\textit{Reactions that produce gases such as oxygen or carbon dioxide:} 

Hydrogen peroxide decomposes to produce oxygen. The volume of oxygen produced can be measured using the gas syringe method (figure \ref {fig:reactionrates:gassyringe}). The gas collects in the syringe, pushing out against the plunger. The volume of gas that has been produced can be read from the markings on the syringe. For example, hydrogen peroxide decomposes in the presence of a manganese(IV) oxide catalyst to produce oxygen and water.
}

\begin{center}
\rm${2H_{2}O_{2}(aq) \rightarrow 2H_{2}O(l) + O_{2}(g)}$
\end{center}

\begin{figure}[htbp]
\begin{center}
\begin{pspicture}(-2,0)(8,6.2)
\SpecialCoor
%\psgrid[gridcolor=lightgray]
\def\syringe{
\psframe[fillstyle=solid,fillcolor=white,linestyle=none](0,0)(5.5,1)
\multirput(0.5,0)(0.5,0){10}{\psline(0,0.4)(0,0.6)}
\psline(5.5,1.1)(5.5,1)(0,1)(0,0.5)
\psline(5.5,-0.1)(5.5,0)(0,0)(0,0.4)
\pspolygon[linewidth=0.05cm](5.6,1.2)(5.6,0.95)(3,0.95)(3,0.05)(5.6,0.05)(5.6,-0.2)(5.8,-0.2)(5.8,1.2)
}
\rput(2,2){\pstTubeEssais[glassType=erlen,niveauLiquide1=40,tubeCoude]}
\rput(2.2,5){\syringe}
\uput[d](4.5,5){Gas Syringe System}
\rput(-0.5,0.5){Reactants}
\pcline{->}(-0.5,1.6)(-0.5,2.2)
\aput{:U}{Gas}
\end{pspicture}
\caption{Gas Syringe Method}
\label{fig:reactionrates:gassyringe}
\end{center}
\end{figure}

\item{\textit{Precipitate reactions:} 

In reactions where a \textit{precipitate} is formed, the amount of precipitate formed in a period of time can be used as a measure of the reaction rate. For example, when sodium thiosulphate reacts with an acid, a yellow precipitate of sulfur is formed. The reaction is as follows:}
\begin{center}
\rm${Na_{2}S_{2}O_{3}(aq) + 2HCl(aq) \rightarrow 2NaCl(aq) + SO_{2}(aq) + H_{2}O(l) + S(s)}$ 
\end{center}

One way to estimate the rate of this reaction is to carry out the investigation in a conical flask and to place a piece of paper with a black cross underneath the bottom of the flask. At the beginning of the reaction, the cross will be clearly visible when you look into the flask (figure \ref{fig:reactionrates:cross}). However, as the reaction progresses and more precipitate is formed, the cross will gradually become less clear and will eventually disappear altogether. Noting the time that it takes for this to happen will give an idea of the reaction rate. Note that it is not possible to collect the SO$_{2}$ gas that is produced in the reaction, because it is very soluble in water.

\begin{figure}[htbp]
\begin{center}
\begin{pspicture}(-2,-4)(2,2)
\SpecialCoor
%\psgrid[gridcolor=lightgray]
\rput(0,0){\pstTubeEssais[glassType=erlen,niveauLiquide1=40]}
\pscircle(0,-3){1}
\rput(0,-3){\psset{unit=0.5}\psline(-1,-1)(1,1)\psline(1,-1)(-1,1)}
\end{pspicture}
\caption{At the beginning of the reaction beteen sodium thiosulphate and hydrochloric acid, when no precipitate has been formed, the cross at the bottom of the conical flask can be clearly seen.}
\label{fig:reactionrates:cross}
\end{center}
\end{figure}

\item{\textit{Changes in mass:} 

The rate of a reaction that produces a gas can also be measured by calculating the mass loss as the gas is formed and escapes from the reaction flask. This method can be used for reactions that produce carbon dioxide or oxygen, but are not very accurate for reactions that give off hydrogen because the mass is too low for accuracy. Measuring changes in mass may also be suitable for other types of reactions.}
\end{itemize}


\Activity{Experiment}{Measuring reaction rates\\}{

\Aim{

To measure the effect of concentration on the rate of a reaction.}

\Apparatus{

\begin{itemize}
\item{300 cm$^{3}$ of sodium thiosulphate (Na$_{2}$S$_{2}$O$_{3}$) solution. Prepare a solution of sodium thiosulphate by adding 12 g of Na$_{2}$S$_{2}$O$_{3}$ to 300 cm$^{3}$ of water. This is solution 'A'.}
\item{300 cm$^{3}$ of water}
\item{100 cm$^{3}$ of 1:10 dilute hydrochloric acid. This is solution 'B'.}
\item{Six 100 cm$^{3}$ glass beakers}
\item{Measuring cylinders}
\item{Paper and marking pen}
\item{Stopwatch or timer\\}
\end{itemize}
}

\Method{

One way to measure the rate of this reaction is to place a piece of paper with a cross underneath the reaction beaker to see how quickly the cross is made invisible by the formation of the sulfur precipitate. 

\begin{enumerate}
\item{Set up six beakers on a flat surface and mark them from 1 to 6. Under each beaker you will need to place a piece of paper with a large black cross.}
\item{Pour 60 cm$^{3}$ solution A into the first beaker and add 20 cm$^{3}$ of water}
\item{Use the measuring cylinder to measure 10 cm$^{3}$ HCl. Now add this HCl to the solution that is already in the first beaker (NB: Make sure that you always clean out the measuring cylinder you have used before using it for another chemical).}
\item{Using a stopwatch with seconds, record the time it takes for the precipitate that forms to block out the cross.}
\item{Now measure 50 cm$^{3}$ of solution A into the second beaker and add 30 cm$^{3}$ of water. To this second beaker, add 10 cm$^{3}$ HCl, time the reaction and record the results as you did before.}
\item{Continue the experiment by diluting solution A as shown below.}
\end{enumerate}

\begin{center}
\begin{tabular}{|p{1.3cm}|p{2.3cm}|p{2.3cm}|p{2.3cm}|p{1.3cm}|}\hline
\textbf{Beaker} & \textbf{Solution A (cm$^{3}$)} & \textbf{Water (cm$^{3}$)} & \textbf{Solution B (cm$^{3}$)} & \textbf{Time (s)}\\\hline
1 & 60 & 20 & 10 & \\\hline
2 & 50 & 30 & 10 & \\\hline
3 & 40 & 40 & 10 & \\\hline
4 & 30 & 50 & 10 & \\\hline
5 & 20 & 60 & 10 & \\\hline
6 & 10 & 70 & 10 & \\\hline
\end{tabular}
\end{center}

The equation for the reaction between sodium thiosulphate and hydrochloric acid is:

\begin{center}
\rm${Na_{2}S_{2}O_{3}(aq) + 2HCl(aq) \rightarrow 2NaCl(aq) + SO_{2}(aq) + H_{2}O(l) + S(s)}$ 
\end{center}
}

\Results{
\begin{itemize}
\item{Calculate the reaction rate in each beaker. This can be done using the following equation:

\begin{equation*}
Rate \ of \ reaction = \frac{1}{time}
\end{equation*}
}
\item{Represent your results on a graph. \textbf{Concentration} will be on the x-axis and \textbf{reaction rate} on the y-axis. Note that the original volume of Na$_{2}$S$_{2}$O$_{3}$ can be used as a measure of concentration.}
\item{Why was it important to keep the volume of HCl constant?}
\item{Describe the relationship between concentration and reaction rate.}
\end{itemize}
}

\Conclusions{

The rate of the reaction is fastest when the concentration of the reactants was the highest.}
}



% CHILD SECTION END 



% CHILD SECTION START 

\section{Mechanism of reaction and catalysis}
\label{sec:reactionrates:energy}

Earlier it was mentioned that it is the \textit{collision} of particles that causes reactions to occur and that only some of these collisions are 'successful'. This is because the reactant particles have a wide range of kinetic energy, and only a small fraction of the particles will have enough energy to actually break bonds so that a chemical reaction can take place. The minimum energy that is needed for a reaction to take place is called the \textbf{activation energy}. For more information on the energy of reactions, refer to Grade 11.

\Definition{Activation energy}{The energy that is needed to break the bonds in reactant molecules so that a chemical reaction can proceed.}

Even at a fixed temperature, the energy of the particles varies, meaning that only some of them will have enough energy to be part of the chemical reaction, depending on the activation energy for that reaction. This is shown in figure \ref{fig:reactionrates:particle energy distribution}. Increasing the reaction temperature has the effect of increasing the number of particles with enough energy to take part in the reaction, and so the reaction rate increases.

\begin{figure}[htbp]
\begin{center}
\begin{pspicture}(-1,-0.6)(7,5)
\SpecialCoor
%\psgrid[gridcolor=lightgray]
\psplot[xunit=2,yunit=5]{0}{3}{2.7128 x x mul neg exp x 2 exp mul 2 mul}
\pcline{->}(0,0)(0,5)
\aput{:U}{\parbox[c]{4cm}{\centering Probability of particle with that KE}}
\pcline{->}(0,0)(7,0)
\bput{:U}{Kinetic Energy of Particle (KE)}
\uput[r](0.6,4.4){\parbox[l]{5cm}{The distribution of particle kinetic energies at a fixed temperature.}}
\psline[linestyle=dashed](2.4,0)(2.4,3.4)
\uput[ul](2.4,0.2){Average KE}
\psline{->}(2.3,0.4)(2.4,0)
\end{pspicture}
\caption{The distribution of particle kinetic energies at a fixed temperature}
\label{fig:reactionrates:particle energy distribution}
\end{center}
\end{figure}

A \textbf{catalyst} functions slightly differently. The function of a catalyst is to lower the activation energy so that more particles now have enough energy to react. The catalyst itself is not changed during the reaction, but simply provides an alternative pathway for the reaction, so that it needs less energy. Some \textit{metals} e.g. platinum, copper and iron can act as catalysts in certain reactions. In our own human bodies, \textit{enzymes} are catalysts that help to speed up biological reactions. Catalysts generally react with one or more of the reactants to form a chemical intermediate which then reacts to form the final product. The chemical intermediate is sometimes called the \textbf{activated complex}. \\

The following is an example of how a reaction that involves a catalyst might proceed. C represents the catalyst, A and B are reactants and D is the product of the reaction of A and B.\\

\textbf{Step 1:} A + C $\rightarrow$ AC 
 
\textbf{Step 2:} B + AC $\rightarrow$ ABC  

\textbf{Step 3:} ABC $\rightarrow$ CD 

\textbf{Step 4:} CD $\rightarrow$ C + D\\

In the above, ABC represents the intermediate chemical. Although the catalyst (C) is consumed by reaction 1, it is later produced again by reaction 4, so that the overall reaction is as follows:

\begin{center}
A + B + C $\rightarrow$ D + C 
\end{center}

You can see from this that the catalyst is released at the end of the reaction, completely unchanged.

\Definition{Catalyst}{A catalyst speeds up a chemical reaction, without being altered in any way. It increases the reaction rate by lowering the activation energy for a reaction.}

Energy diagrams are useful to illustrate the effect of a \textbf{catalyst} on reaction rates. Catalysts decrease the activation energy required for a reaction to proceed (shown by the smaller 'hump' on the energy diagram in figure \ref{fig:reactionrates:catalyst}), and therefore increase the reaction rate. 

\begin{figure}[h]
\begin{center}
\begin{pspicture}(-1,-1)(10,5.5)
%  \psgrid(0,0)(0,0)(10,5.5)
  \psline{->}(0,0)(10,0)
  \psline{->}(0,0)(0,5.5)
  \pscurve[showpoints=false](0,1)(1.5,1)(1.75,1)(2,1)(2.4,1.1)
  (4.5,5)(6.6,2.36)(7,2.3)(7.25,2.3)(7.5,2.3)(9,2.3)
  \pscurve[linestyle=dashed](0,1)(1.5,1)(1.75,1)
  (2,1)(2.4,1.09)(4.5,3.6)(6.5,2.33)(7,2.3)(7.5,2.3)(9,2.3)
  \psline[linestyle=dotted](0,1)(5,1)
  \psline[linestyle=dotted](4.6,5)(1.75,5)
  \psline{<->}(2,5)(2,1)
  \psline{<->}(4.61,3.6)(4.61,1)  
  \rput[bl](4.45,5.1){\small activated complex}
  \rput[b](8,2.4){\small products}
  \rput[t](1,0.9){\small reactants}
  \rput[rb](1.9,4){\small activation}
  \rput[rt](1.7,3.9){\small energy}
  \rput[lb](4.76,1.65){\small activation energy}
  \rput[lt](4.86,1.6){\small with a catalyst}
  \rput[r](4.45,-0.5){Time}
  \psline{->}(4.5,-0.5)(5.3,-0.5)
  \rput[r]{90}(-0.5,3.65){Potential energy}
  \psline{->}(-0.5,3.75)(-0.5,4.75)
\end{pspicture}
\end{center}
\caption{The effect of a catalyst on the activation energy of a reaction}
\label{fig:reactionrates:catalyst} 
\end{figure}

\Activity{Experiment}{Catalysts and reaction rates\\}{

\Aim{

To determine the effect of a catalyst on the rate of a reaction}

\Apparatus{ 

Zinc granules, 0.1 M hydrochloric acid, copper pieces, one test tube and a glass beaker.}

\Method{

\begin{enumerate}
\item{Place a few of the zinc granules in the test tube.}
\item{Measure the mass of a few pieces of copper and keep them separate from the rest of the copper.}
\item{Add about 20 cm$^{3}$ of HCl to the test tube. You will see that a gas is released. Take note of how quickly or slowly this gas is released. Write a balanced equation for the chemical reaction that takes place.}
\item{Now add the copper pieces to the same test tube. What happens to the rate at which the gas is produced?}
\item{Carefully remove the copper pieces from the test tube (do not get HCl on your hands), rinse them in water and alcohol and then weigh them again. Has the mass of the copper changed since the start of the experiment?\\}
\end{enumerate}
}

\Results{

During the reaction, the gas that is released is hydrogen. The rate at which the hydrogen is produced \textit{increases} when the copper pieces (the catalyst) are added. The mass of the copper does not change during the reaction.}

\Conclusions{

The copper acts as a \textit{catalyst} during the reaction. It speeds up the rate of the reaction, but is not changed in any way itself.}
}

\Exercise{Reaction rates\\}{

\begin{enumerate}
\item{For each of the following, say whether the statement is \textbf{true} or \textbf{false}. If it is false, re-write the statement correctly.}
	\begin{enumerate}
	\item{A catalyst increases the energy of reactant molecules so that a chemical reaction can take place.}
	\item{Increasing the temperature of a reaction has the effect of increasing the number of reactant particles that have more energy that the activation energy.}
	\item{A catalyst does not become part of the final product in a chemical reaction.}
	\end{enumerate}

\item{5 g of zinc granules are added to 400 cm$^{3}$ of 0.5 mol.dm$^{-3}$ hydrochloric acid. To investigate the rate of the reaction, the change in the mass of the flask containing the zinc and the acid was measured by placing the flask on a direct reading balance. The reading on the balance shows that there is a decrease in mass during the reaction. The reaction which takes place is given by the following equation:

\begin{center}
\rm${Zn(s) + 2HCl(aq) \rightarrow ZnCl_{2}(aq) + H_{2}(g)}$
\end{center}

	\begin{enumerate}
	\item{Why is there a decrease in mass during the reaction?}
	\item{The experiment is repeated, this time using 5 g of powdered zinc instead of granulated zinc. How will this influence the rate of the reaction?}
	\item{The experiment is repeated once more, this time using 5 g of granulated zinc and 600 cm$^{3}$ of 0.5 mol.dm$^{-3}$ hydrochloric acid. How does the rate of this reaction compare to the \textbf{original} reaction rate?}
	\item{What effect would a catalyst have on the rate of this reaction?}
	\end{enumerate}
}

(IEB Paper 2 2003)

\item{Enzymes are catalysts. Conduct your own research to find the names of common enzymes in the human body and which chemical reactions they play a role in.}

\item{5 g of calcium carbonate powder reacts with 20 cm$^{3}$ of a 0.1 mol.dm$^{-3}$ solution of hydrochloric acid. The gas that is produced at a temperature of 25$^{circ}$C is collected in a gas syringe.}
	\begin{enumerate}
	\item{Write a balanced chemical equation for this reaction.}
	\item{The rate of the reaction is determined by measuring the volume of gas thas is produced in the first minute of the reaction. How would the rate of the reaction be affected if:}
		\begin{enumerate}
		\item{a lump of calcium carbonate of the same mass is used}
		\item{40 cm$^{3}$ of 0.1 mol.dm$^{-3}$ hydrochloric acid is used}
		\end{enumerate} 
	\end{enumerate}

\end{enumerate}
}



% CHILD SECTION END 



% CHILD SECTION START 

\section{Chemical equilibrium}
\label{sec:reactionrates:equilibrium}

Having looked at factors that affect the rate of a reaction, we now need to ask some important questions. Does a reaction always proceed in the same direction or can it be reversible? In other words, is it always true that a reaction proceeds from \textit{reactants to products}, or is it possible that sometimes, the reaction will reverse and the \textit{products will be changed back into the reactants}? And does a reaction always run its full course so that all the reactants are used up, or can a reaction reach a point where reactants are still present, but there does not seem to be any further change taking place in the reaction? The following demonstration might help to explain this.

\Activity{Demonstration}{Liquid-vapour phase equilibrium\\}{

\textbf{Apparatus and materials:\\}

2 beakers; water; bell jar\\

\textbf{Method:\\}

\begin{enumerate}
\item{Half fill two beakers with water and mark the level of the water in each case.}
\item{Cover one of the beakers with a bell jar.}
\item{Leave the beakers and, over the course of a day or two, observe how the water level in the two beakers changes. What do you notice? Note: You could speed up this demonstration by placing the two beakers over a bunsen burner to heat the water. In this case, it may be easier to cover the second beaker with a glass cover.\\}
\end{enumerate}

\textbf{Observations:\\}

You should notice that in the beaker that is uncovered, the water level drops quickly because of evaporation. In the beaker that is covered, there is an initial drop in the water level, but after a while evaporation appears to stop and the water level in this beaker is higher than that in the one that is open. Note that the diagram below shows the situation ate time=0. \\

\begin{center}
\begin{pspicture}(-2,0)(8,8)
\SpecialCoor
%\psgrid[gridcolor=lightgray]
\def\longuparrow{\psline[linewidth=0.2cm,linecolor=gray]{->}(0,0)(0,1.6)}
\def\uparrow{\psline[linewidth=0.2cm,linecolor=gray]{->}(0,0)(0,0.8)}
\def\downarrow{\psline[linewidth=0.2cm,linecolor=gray]{<-}(0,0)(0,0.8)}
\def\longdownarrow{\psline[linewidth=0.2cm,linecolor=gray]{<-}(0,0)(0,1.6)}

\uput[r](-2,5.25){\uparrow \uput[r](0,0.5){ = evaporation}}
\uput[r](-2,6.5){\downarrow \uput[r](0,0.5){ = condensation}}

\rput(-2,0){
\rput(0,0){\psset{unit=2}\psline[linearc=7pt](0,2)(0,0)(1.5,0)(1.5,2)
\psline[fillstyle=solid,fillcolor=lightgray,linearc=7pt](0,0.7)(0,0)(1.5,0)(1.5,0.7)\psellipse(0.75,2)(0.8,0.1)} 
\rput(1,2.2){\longuparrow}
\rput(2,2.2){\downarrow}
}

\rput(3,0){
\psline[linearc=1.5cm](0,0.25)(0,7)(4,7)(4,0.25)
\psarc(-0.25,0.25){0.25}{-90}{0}
\psarc(4.25,0.25){0.25}{180}{270}
\rput(-0.1,0){\psset{xunit=1.05,yunit=1.025}\psline[linearc=1.5cm](0,0.25)(0,7)(4,7)(4,0.25)
\psarc(-0.25,0.25){0.25}{-90}{0}
\psarc(4.25,0.25){0.25}{180}{270}
\pscircle(2,7.25){0.25}}
\rput(0.5,0){\psset{unit=2}\filledbeaker} 
\rput(1.5,2.2){\longuparrow}
\rput(2.5,2.2){\longdownarrow}
}

\uput{0.1cm}[u](7,7.4){bell jar}
\psline(7,7.4)(6.6,6.8)
\end{pspicture}
\end{center}

\textbf{Discussion:\\}

In the first beaker, liquid water becomes water vapour as a result of evaporation and the water level drops. In the second beaker, evaporation also takes place. However, in this case, the vapour comes into contact with the surface of the bell jar and it cools and condenses to form liquid water again. This water is returned to the beaker. Once condensation has begun, the rate at which water is lost from the beaker will start to decrease. At some point, the rate of evaporation will be equal to the rate of condensation above the beaker, and there will be no change in the water level in the beaker. This can be represented as follows:

\begin{center}
\rm${liquid \rightleftharpoons vapour}$
\end{center}

In this example, the reaction (in this case, a change in the phase of water) can proceed in either direction. In one direction there is a change in phase from liquid to vapour. But the reverse can also take place, when vapour condenses to form water again. 
}   

In a \textbf{closed system} it is possible for reactions to be reversible, such as in the demonstration above. In a closed system, it is also possible for a chemical reaction to reach \textbf{equilibrium}. We will discuss these concepts in more detail.

\subsection{Open and closed systems}

An \textbf{open system} is one in which matter or energy can flow into or out of the system. In the liquid-vapour demonstration we used, the first beaker was an example of an open system because the beaker could be heated or cooled (a change in \textit{energy}), and water vapour (the \textit{matter}) could evaporate from the beaker. \\

A \textbf{closed system} is one in which energy can enter or leave, but matter cannot. The second beaker covered by the bell jar is an example of a closed system. The beaker can still be heated or cooled, but water vapour cannot leave the system because the bell jar is a barrier. Condensation changes the vapour to liquid and returns it to the beaker. In other words, there is no loss of matter from the system. \\

\Definition{Open and closed systems}{
An \textit{open system} is one whose borders allow the movement of energy and matter into and out of the system. A \textit{closed system} is one in which only energy can be exchanged, but not matter.
}

\subsection{Reversible reactions}
Some reactions can take place in two directions. In one direction the reactants combine to form the products. This is called the \textbf{forward reaction}. In the other, the products react to form reactants again. This is called the \textbf{reverse reaction}. A special double-headed arrow is used to show this type of \textbf{reversible reaction}:

\begin{center}
$ XY + Z \rightleftharpoons X + YZ $
\end{center}

So, in the following reversible reaction:
\begin{center}
\rm${H_{2}(g) + I_{2}(g) \rightleftharpoons 2HI (g)}$
\end{center}

The forward reaction is ${H_{2}(g) + I_{2}(g) \rightarrow 2HI(g)}$. The reverse reaction is ${2HI(g) \rightarrow H_{2}(g) + I_{2}(g)}$.

\Definition{A reversible reaction}{A reversible reaction is a chemical reaction that can proceed in both the forward and reverse directions. In other words, the reactant and product of one reaction may reverse roles.}

\Activity{Demonstration}{The reversibility of chemical reactions\\}{

\textbf{Apparatus and materials:\\}

Lime water (Ca(OH)$_{2}$); calcium carbonate (CaCO$_{3}$); hydrochloric acid; 2 test tubes with rubber stoppers; delivery tube; retort stand and clamp; bunsen burner.\\

\textbf{Method and observations:\\}

\begin{enumerate}
\item{Half-fill a test tube with clear lime water (Ca(OH)$_{2}$).}
\item{In another test tube, place a few pieces of calcium carbonate (CaCO$_{3}$) and cover the pieces with dilute hydrochloric acid. Seal the test tube with a rubber stopper and delivery tube.}
\item{Place the other end of the delivery tube into the test tube containing the lime water so that the carbon dioxide that is produced from the reaction between calcium carbonate and hydrochloric acid passes through the lime water. Observe what happens to the appearance of the lime water. 

The equation for the reaction that takes place is:
\begin{center}
\rm${Ca(OH)_{2} + CO_{2} \rightarrow CaCO_{3} + H_{2}O }$
\end{center}
}

CaCO$_{3}$ is insoluble and it turns the limewater milky.

\item{Allow the reaction to proceed for a while so that carbon dioxide continues to pass through the limewater. What do you notice? The equation for the reaction that takes place is:

\begin{center}
\rm${CaCO_{3}(s) + H_{2}O + CO_{2} \rightarrow Ca(HCO_{3})_{2}}$
\end{center}

In this reaction, calcium carbonate becomes one of the reactants to produce hydrogen carbonate (Ca(HCO$_{3}$)$_{2}$) and so the solution becomes clear again.
}

\item{Heat the solution in the test tube over a bunsen burner. What do you observe? You should see bubbles of carbon dioxide appear and the limewater turns milky again. The reaction that has taken place is:

\begin{center}
\rm${Ca(HCO_{3})_{2} \rightarrow CaCO_{3}(s) + H_{2}O + CO_{2}}$
\end{center}
}
\end{enumerate}

\begin{center}
\begin{pspicture}(-2,-0.4)(6,7)
%\psgrid[gridcolor=gray]
\def\stand{\psline[linewidth=5pt](-1.5,0)(1.5,0)\psline[linewidth=5pt](0,0)(0,5)\psframe[fillstyle=solid,fillcolor=gray,linestyle=none](-0.5,2.5)(0.5,3.5)\psline[linewidth=3pt,linecolor=gray](0.5,3)(2,3)}
\def\tube{\psarc(0,0){1.5}{0}{180}\psarc(0,0){1.75}{0}{180}\psline(1.5,0)(1.5,-1)\psline(1.75,0)(1.75,-1)}
\rput(0,-0.2){\stand}
\rput(1.5,3){\pstTubeEssais[glassType=tube,bouchon=true,niveauLiquide1=30]}
\rput(4.8,2){\pstTubeEssais[glassType=tube,bouchon=true,niveauLiquide1=60]}
\rput(3.2,5){\tube}
\rput(3.2,5){\psline(-1.5,-0.9)(-1.5,-1.25)\psline(-1.75,-0.9)(-1.75,-1.25)}
\rput(5,4){\psline(0,-0.9)(0,-1.25)\psline(-0.25,-0.9)(-0.25,-1.25)}
\uput[d](1.95,1){\parbox[l]{3.2cm}{calcium carbonate \& hydrochloric acid}}
\uput[d](4.8,0){limewater}
\uput[r](4.4,6.4){delivery tube}
\uput[l](1,5){rubber stopper}
\uput[r](5.4,4){rubber stopper}

\end{pspicture}
\end{center}


\textbf{Discussion:}\\

\begin{itemize}
\item{If you look at the last two equations you will see that the one is the reverse of the other. In other words, this is a \textit{reversible reaction} and can be written as follows:
\begin{center}
\rm${CaCO_{3}(s) + H_{2}O + CO_{2} \rightleftharpoons Ca(HCO_{3})_{2}}$
\end{center}
}
\item{Is the forward reaction endothermic or exothermic? Is the reverse reaction endothermic or exothermic? You should have noticed that the reverse reaction only took place when the solution was heated. Sometimes, changing the temperature of a reaction can change its direction.}
\end{itemize}
}

\subsection{Chemical equilibrium}

Using the same reversible reaction that we used in an earlier example:

\begin{center}
\rm${H_{2}(g) + I_{2}(g) \rightleftharpoons 2HI (g)}$
\end{center}

The forward reaction is:

\begin{equation*}
H_{2} + I_{2} \rightarrow 2HI
\end{equation*}

The reverse reaction is:

\begin{equation*}
2HI \rightarrow H_{2} + I_{2}
\end{equation*}

When the rate of the forward reaction and the reverse reaction are equal, the system is said to be in \textbf{equilbrium}. Figure \ref{fig:reactionrates:equilibrium} shows this. Initially (time = 0), the rate of the forward reaction is high and the rate of the reverse reaction is low. As the reaction proceeds, the rate of the forward reaction decreases and the rate of the reverse reaction increases, until both occur at the same rate. This is called equilibrium. \\

Although it is not always possible to observe any macroscopic changes, this does not mean that the reaction has stopped. The forward and reverse reactions continue to take place and so microscopic changes still occur in the system. This state is called \textbf{dynamic equilibrium}. In the liquid-vapour phase equilibrium demonstration, dynamic equilibrium was reached when there was no observable change in the level of the water in the second beaker even though evaporation and condensation continued to take place.

\begin{figure}[htbp]
\begin{center}
\begin{pspicture}(-0.6,-0.6)(7,5.5)
\SpecialCoor
%\psgrid[gridcolor=lightgray]
\pcline{->}(0,0)(0,5.5)
\aput{:U}{Rate of Reaction}
\pcline{->}(0,0)(7,0)
\bput{:U}{Time}
\uput[u](5,2.5){equilibrium}
\pscurve[showpoints=false,curvature=1 1 1](0,5)(1,3.5)(3,2.5)
\pscurve[showpoints=false,curvature=1 1 1](0,0)(1,1.5)(3,2.5)
\psline(3,2.5)(7,2.5)
\uput[r](1,1){2HI$\rightarrow$H$_2$+I$_2$}
\uput[r](1,4){H$_2$+I$_2\rightarrow$2HI}

\end{pspicture}
\caption{The change in rate of forward and reverse reactions in a closed system}
\label{fig:reactionrates:equilibrium}
\end{center}
\end{figure}

There are, however, a number of factors that can change the chemical equilibrium of a reaction. Changing the \textbf{concentration}, the \textbf{temperature} or the \textbf{pressure} of a reaction can affect equilibrium. These factors will be discussed in more detail later in this chapter.

\Definition{Chemical equilibrium}{
Chemical equilibrium is the state of a chemical reaction, where the concentrations of the reactants and products have no net change over time. Usually, this state results when the forward chemical reactions proceed at the same rate as their reverse reactions. 
}


% CHILD SECTION END 



% CHILD SECTION START 

\section{The equilibrium constant}

\Definition{Equilibrium constant}{
The equilibrium constant (K$_{c}$), relates to a chemical reaction at equilibrium. It can be calculated if the equilibrium concentration of each reactant and product in a reaction at equilibrium is known.
}

\subsection{Calculating the equilibrium constant}

Consider the following generalised reaction which takes place in a \textbf{closed} container at a \textbf{constant temperature}:
\begin{center}
\rm${A + B \rightleftharpoons C + D}$ 
\end{center}

We know from section \ref{sec:reactionrates:factors affecting} that the rate of the forward reaction is directly proportional to the concentration of the reactants. In other words, as the concentration of the reactants increases, so does the rate of the forward reaction. This can be shown using the following equation:
\begin{center}
Rate of forward reaction $\propto$ [A][B] 

or

Rate of forward reaction = k$_{1}$[A][B]
\end{center}

Similarly, the rate of the reverse reaction is directly proportional to the concentration of the products. This can be shown using the following equation:
\begin{center}
Rate of reverse reaction $\propto$ [C][D]

or

Rate of reverse reaction = k$_{2}$[C][D]
\end{center}

At equilibrium, the rate of the forward reaction is equal to the rate of the reverse reaction. This can be shown using the following equation:

\begin{equation*}
k_{1}[A][B] = k_{2}[C][D]
\end{equation*}

\begin{center}
or
\end{center}

\begin{equation*}
\frac{k_{1}}{k_{2}} = \frac{[C][D]}{[A][B]}
\end{equation*}

or, if the constants k$_{1}$ and k$_{2}$ are simplified to a single constant, the equation becomes:

\begin{equation*}
k_{c} = \frac{[C][D]}{[A][B]} 
\end{equation*}

A more general form of the equation for a reaction at chemical equilibrium is:
\begin{center}
${aA + bB \rightleftharpoons cC + dD}$
\end{center}

where A and B are reactants, C and D are products and a, b, c, and d are the coefficients of the respective reactants and products. A more general formula for calculating the equilibrium constant is therefore:

\begin{equation*}
K_{c} = \frac{[C]^{c}[D]^{d}}{[A]^{a}[B]^{b}} 
\end{equation*}

It is important to note that if a reactant or a product in a chemical reaction is in either the \textbf{liquid} or \textbf{solid} phase, the concentration stays constant during the reaction. Therefore, these values can be left out of the equation to calculate K$_{c}$. For example, in the following reaction:\\

\begin{center}
\rm${C(s) + H_{2}O(g) \rightleftharpoons CO(g) + H_{2}(g)}$
\end{center}

\begin{equation*}
K_{c} = \frac{[CO][H_{2}]}{[H_{2}O]}
\end{equation*}

\Tip{

\begin{enumerate}
\item{The constant $K_{c}$ is affected by temperature and so, if the values of $K_{c}$ are being compared for different reactions, it is important that all the reactions have taken place at the same temperature.}
\item{K$_{c}$ values do not have units. If you look at the equation, the units all cancel each other out.}
\end{enumerate}
}

\subsection{The meaning of K$_{c}$ values}

The formula for K$_{c}$ has the concentration of the products in the numerator and the concentration of reactants in the denominator. So a \textit{high K$_{c}$} value means that the concentration of products is high and the reaction has a high yield. We can also say that the equilibrium lies far to the right. The opposite is true for a low K$_{c}$ value. A \textit{low K$_{c}$ value} means that, at equilibrium, there are more reactants than products and therefore the yield is low. The equilibrium for the reaction lies far to the left.

\Tip{Calculations made easy\\}{
When you are busy with calculations that involve the equilibrium constant, the following tips may help:\\

\begin{enumerate}
\item{
Make sure that you always read the question carefully to be sure of what you are being asked to calculate. If the equilibrium constant is involved, make sure that the concentrations you use are the concentrations \textbf{at equilibrium}, and not the concentrations or quantities that are present at some other time in the reaction. 
}

\item{When you are doing more complicated calculations, it sometimes helps to draw up a table like the one below and fill in the \textbf{mole values} that you know or those you can calculate. This will give you a clear picture of what is happening in the reaction and will make sure that you use the right values in your calculations. 


\begin{center}
\begin{tabular}{|l|c|c|c|}\hline
 & \textbf{Reactant 1} & \textbf{Reactant 2} & \textbf{Product 1}\\\hline
Start of reaction & & &  \\\hline
Used up & & &  \\\hline
Produced & & &  \\\hline
Equilibrium & & &  \\\hline
\end{tabular}
\end{center}
}
\end{enumerate}
}

\begin{wex}{Calculating K$_{c}$}{For the reaction:

\begin{center}
\rm${SO_{2}(g) + NO_{2}(g) \rightarrow NO(g) + SO_{3}(g)}$
\end{center}

the concentration of the reagents is as follows:

[SO$_{3}$] = 0.2 mol.dm$^{-3}$

[NO$_{2}$] = 0.1 mol.dm$^{-3}$

[NO] = 0.4 mol.dm$^{-3}$

[SO$_{2}$] = 0.2 mol.dm$^{-3}$\\

Calculate the value of K$_{c}$.\\
}

{
\westep{Write the equation for k$_{c}$}

\begin{equation*}
K_{c} = \frac{[NO][SO_{3}]}{[SO_{2}][NO_{2}]}
\end{equation*}

\westep{Fill in the values you know for this equation and calculate K$_{c}$}

\begin{equation*}
K_{c} = \frac{(0.4 \times 0.2)}{(0.2 \times 0.1)} = 4
\end{equation*}
}
\end{wex}

\begin{wex}{Calculating reagent concentration}{For the reaction:
\begin{center}
$S(s) + O_{2}(g) \rightleftharpoons SO_{2}(g)$
\end{center}

\begin{enumerate}
\item{Write an equation for the equilibrium constant.}
\item{Calculate the equilibrium concentration of $O_{2}$ if K$_{c}$=6 and
  [$SO_{2}]=3 mol. dm^{-3}$ at equilibrium.}   
\end{enumerate} }
{\westep{Write the equation for K$_{c}$}

\begin{equation*}
K_{c} = \frac{[SO_{2}]}{[O_{2}]}
\end{equation*}

(Sulfur is left out of the equation because it is a solid and its concentration stays constant during the reaction)\\
\westep{Re-arrange the equation so that oxygen is on its own on one side of the equation}

\begin{equation*}
[O_{2}] = \frac{[SO_{2}]}{K_{c}}
\end{equation*}
\westep{Fill in the values you know and calculate [O$_{2}$]}

\begin{equation*}
[O_{2}] = \frac{3 mol.dm^{-3}}{6} = 0.5 mol.dm^{-3}
\end{equation*}
}
\end{wex}  

\begin{wex}{Equilibrium calculations}{Initially 1.4 moles of NH$_{3}$(g) is introduced into a sealed 2.0 dm$^{-3}$ reaction vessel. The ammonia decomposes when the temperature is increased to 600K and reaches equilibrium as follows:

\begin{center}
\rm${2NH_{3}(g) \rightleftharpoons N_{2}(g) + 3H_{2}(g)}$
\end{center}

When the equilibrium mixture is analysed, the concentration of NH$_{3}$(g) is 0.3 mol$\cdot$dm$^{-3}$

\begin{enumerate}
\item{Calculate the concentration of N$_{2}$(g) and H$_{2}$(g) in the equilibrium mixture.}
\item{Calculate the equilibrium constant for the reaction at 900 K.}
\end{enumerate}
}
{\westep{Calculate the number of moles of NH$_{3}$ at equilibrium.}

\begin{equation*}
c = \frac{n}{V}
\end{equation*}

Therefore,

\begin{equation*}
n = c \times V = 0.3 \times 2 = 0.6 mol
\end{equation*}
\westep{Calculate the number of moles of ammonia that react (are 'used up') in the reaction.}
Moles used up = 1.4 - 0.6 = 0.8 moles\\
\westep{Calculate the number of moles of product that are formed.}
Remember to use the mole ratio of reactants to products to do this. In this case, the ratio of NH$_{3}$:N$_{2}$:H$_{2}$ = 2:1:3. Therefore, if 0.8 moles of ammonia are used up in the reaction, then 0.4 moles of nitrogen are produced and 1.2 moles of hydrogen are produced.\\
\westep{Complete the following table}

\begin{center}
\begin{tabular}{|l|c|c|c|}\hline
 & \textbf{NH$_{3}$} & \textbf{N$_{2}$} & \textbf{H$_{2}$}\\\hline
Start of reaction & 1.4 & 0 & 0  \\\hline
Used up & 0.8 & 0 & 0  \\\hline
Produced & 0 & 0.4 & 1.2  \\\hline
Equilibrium & 0.6 & 0.4 & 1.2  \\\hline
\end{tabular}
\end{center}
\westep{Using the values in the table, calculate [N$_{2}$] and [H$_{2}$]}

\begin{equation*}
[N_{2}] = \frac{n}{V} = \frac{0.4}{2} = 0.2 \ mol.dm^{-3}
\end{equation*}

\begin{equation*}
[H_{2}] = \frac{n}{V} = \frac{1.2}{2} = 0.6 \ mol.dm^{-3}
\end{equation*}
\westep{Calculate K$_{c}$}

\begin{equation*}
K_{c} = \frac{[H_{2}]^{3}[N_{2}]}{[NH_{3}]^{2}} = \frac{(0.6)^{3}(0.2)}{(0.3)^{2}} = 0.48
\end{equation*}
}
\end{wex}

\begin{wex}{Calculating K$_{c}$\\}{
Hydrogen and iodine gas react according to the following equation:\\

$H_{2}(g) + I_{2}(g) \rightleftharpoons 2HI(g)$\\

When 0.496 mol $H_{2}$ and 0.181 mol $I_{2}$ are heated at $450^{o}$C in a 1 dm$^{3}$ container, the equilibrium mixture is found to contain 0.00749 mol $I_{2}$. Calculate the equilibrium constant for the reaction at $450^{o}$C. \\}

{\westep{Calculate the number of moles of iodine used in the reaction.}

Moles of iodine used = 0.181 - 0.00749 = 0.1735 mol\\
}

{\westep{Calculate the number of moles of hydrogen that are used up in the reaction.}

The mole ratio of hydrogen:iodine = 1:1, therefore 0.1735 moles of hydrogen must also be used up in the reaction.\\
\westep{Calculate the number of moles of hydrogen iodide that are produced.}

The mole ratio of H$_{2}$:I$_{2}$:HI = 1:1:2, therefore the number of moles of HI produced is 0.1735 $\times$ 2 = 0.347 mol.\\

So far, the table can be filled in as follows:
 
\begin{center}
\begin{tabular}{|l|c|c|c|}\hline
 & \textbf{H$_{2}$ (g)} & \textbf{I$_{2}$} & \textbf{2HI}\\\hline
Start of reaction & 0.496 & 0.181 & 0 \\\hline
Used up & 0.1735 & 0.1735 & 0\\\hline
Produced & 0 & 0 & 0.347 \\\hline
Equilibrium & 0.3225 & 0.0075 & 0.347  \\\hline
\end{tabular}
\end{center}
\westep{Calculate the concentration of each of the reactants and products at equilibrium.}

\begin{equation*}
c = \frac{n}{V}
\end{equation*}

Therefore the equilibrium concentrations are as follows:

[H$_{2}$] = 0.3225 mol.dm$^{-3}$

[I$_{2}$] = 0.0075 mol.dm$^{-3}$

[HI] = 0.347 mol.dm$^{-3}$\\
\westep{Calculate K$_{c}$}

\begin{equation*}
K_{c} = \frac{[HI]}{[H_{2}][I_{2}]} = \frac{0.347}{0.3225 \times 0.0075} = 143.47
\end{equation*}
}
\end{wex}

\Exercise{The equilibrium constant\\}{

\begin{enumerate}
\item{Write the equilibrium constant expression, K$_{c}$ for the following reactions:}
	\begin{enumerate}
	\item{$\rm{2NO (g) + Cl_{2} (g) \rightleftharpoons 2NOCl}$}
	\item{$\rm{H_{2} (g) + I_{2} (g) \rightleftharpoons 2HI (g)}$}
	\end{enumerate}

\item{The following reaction takes place:

\begin{center}
$\rm{Fe^{3+} (aq) + 4Cl^{-} \rightleftharpoons FeCl_{4}^{-} (aq)}$
\end{center}

K$_{c}$ for the reaction is 7.5 $\times$ 10$^{-2}$ mol.dm$^{-3}$. At equilibrium, the concentration of FeCl$_{4}^{-}$ is 0.95 $\times$ 10$^{-4}$ mol.dm$^{-3}$ and the concentration of free iron (Fe$^{3+}$) is 0.2 mol.dm$^{-3}$. Calculate the concentration of chloride ions at equilibrium.
}

\item{Ethanoic acid (CH$_{3}$COOH) reacts with ethanol (CH$_{3}$CH$_{2}$OH) to produce ethyl ethanoate and water. The reaction is:

\begin{center}
$\rm{CH_{3}COOH + CH_{3}CH_{2}OH \rightarrow CH_{3}COOCH_{2}CH_{3} + H_{2}O}$
\end{center}

At the beginning of the reaction, there are 0.5 mols of ethanoic acid and 0.5 mols of ethanol. At equilibrium, 0.3 mols of ethanoic acid was left unreacted. The volume of the reaction container is 2 dm$^{3}$. Calculate the value of K$_{c}$.}

\end{enumerate}
}    



% CHILD SECTION END 



% CHILD SECTION START 

\section{Le Chatelier's principle}
\label{sec:reactionrates:lechatelier}

A number of factors can influence the equilibrium of a reaction. These are:
\begin{enumerate}
\item concentration
\item temperature
\item pressure
\end{enumerate}

\textbf{Le Chatelier's Principle} helps to predict what a change in temperature, concentration or pressure will have on the position of the equilibrium in a chemical reaction. This is very important, particularly in industrial applications, where yields must be accurately predicted and maximised.

\Definition{Le Chatelier's Principle}{If a chemical system at equilibrium experiences a change in concentration, temperature or total pressure the equilibrium will shift in order to minimise that change and a new equilibrium is established.}

\subsection{The effect of concentration on equilibrium}

If the concentration of a substance is increased, the equilibrium will shift so that this concentration decreases. So for example, if the concentration of a reactant was increased, the equilibrium would shift in the direction of the reaction that \textit{uses up} the reactants, so that the reactant concentration decreases and equilibrium is restored. In the reaction between nitrogen and hydrogen to produce ammonia:

\begin{center}
\rm${N_{2}(g) + 3H_{2}(g) \rightleftharpoons 2NH_{3}(g)}$ 
\end{center}

\begin{itemize}
\item{If the nitrogen or hydrogen concentration was increased, Le Chatelier's principle predicts that equilibrium will shift to favour the \textit{forward reaction} so that the excess nitrogen and hydrogen are used up to produce ammonia. Equilibrium shifts to the \textit{right}.} 

\item{If the nitrogen or hydrogen concentration was decreased, the \textit{reverse reaction} would be favoured so that some of the ammonia would change back to nitrogen and hydrogen to restore equilibrium.}

\item{The same would be true if the concentration of the product (NH$_{3}$) was changed. If [NH$_{3}$] decreases, the forward reaction is favoured and if [NH$_{3}$] increases, the reverse reaction is favoured.}
\end{itemize}


\subsection{The effect of temperature on equilibrium}

If the temperature of a reaction mixture is increased, the equilibrium will shift to decrease the temperature. So
it will favour the reaction which will \textit{use up} heat energy, in other words the endothermic reaction. The opposite is true if the temperature is decreased. In this case, the reaction that \textit{produces} heat energy will be favoured, in other words, the exothermic reaction.\\

The reaction shown below is exothermic (shown by the negative value for $\Delta$ H). This means that the forward reaction, where nitrogen and hydrogen react to form ammonia, gives off heat. In the reverse reaction, where ammonia is broken down into hydrogen and nitrogen gas, heat is used up and so this reaction is endothermic.

\begin{eqnarray*}
  & e.g. & N_{2}(g) + 3H_{2}(g) \rightleftharpoons  2NH_{3}(g) \ \rm{and} \
  \Delta H = -92kJ
\end{eqnarray*}

An increase in temperature favours the reaction that is endothermic (the reverse reaction) because it uses up energy. If the temperature is increased, then the yield of ammonia (NH$_{3}$) decreases. \\

A decrease in temperature favours the reaction that is exothermic (the forward reaction) because it produces energy. Therefore, if the temperature is decreased, then the yield of NH$_{3}$ increases. \\

\Activity{Experiment}{Le Chatelier's Principle\\}{

\Aim{

To determine the effect of a change in concentration and temperature on chemical equilibrium}

\Apparatus{

0.2 M CoCl$_{2}$ solution, concentrated HCl, water, test tube, bunsen burner}

\Method{

\begin{enumerate}
\item{Put 4-5 drops of 0.2M CoCl$_{2}$ solution into a test tube.}
\item{Add 20-25 drops of concentrated HCl.}
\item{Add 10-12 drops of water.}
\item{Heat the solution for 1-2 minutes.}
\item{Cool the solution for 1 minute under a tap.}
\item{Observe and record the colour changes that take place during the reaction.\\}
\end{enumerate}

The equation for the reaction that takes place is:
\begin{eqnarray*}
  & e.g. & \underbrace{CoCl_{4}^{2-} + 6H_{2}O}_{\rm{blue}} \rightleftharpoons \underbrace{Co(H_{2}O)_{6}^{2+} + 4Cl^{-}}_{\rm{pink}}\\ 
\end{eqnarray*}
}

\Results{

Complete your observations in the table below, showing the colour changes that take place, and also indicating whether the concentration of each of the ions in solution increases or decreases.

\begin{center}
\begin{tabular}{|p{1.4cm}|p{1.4cm}|p{1.4cm}|p{1.4cm}|p{1.4cm}|p{1.4cm}|}\hline
 & \textbf{Initial colour} & \textbf{Final colour} & \textbf{[Co$^{2+}$]} & \textbf{[Cl$^{-}$]} & \textbf{[CoCl$_{4}^{2-}$]}\\\hline
Add Cl$^{-}$ & & & & & \\\hline
Add H$_{2}$O & & & & & \\\hline
Increase temp. & & & & & \\\hline
Decrease temp. & & & & & \\\hline
\end{tabular}
\end{center}
}

\Conclusions{

Use your knowledge of equilibrium principles to explain the changes that you recorded in the table above. Draw a conclusion about the effect of a change in concentration of either the reactants or products on the equilibrium position. Also draw a conclusion about the effect of a change in temperature on the equilibrium position.}
}

\subsection{The effect of pressure on equilibrium}

In the case of gases, we refer to pressure instead of concentration. Similar principles apply as those that were described before for concentration. When the pressure of a system increases, there are more particles in a particular space. The equilibrium will shift in a direction that reduces the number of gas particles so that the pressure is also reduced. To predict what will happen in a reaction, we need to look at the number of moles of gas that are in the reactants and products. Look at the example below:

\begin{eqnarray*}
  & e.g. & 2SO_{2}(g) + O_{2}(g) \rightleftharpoons 2SO_{3}(g)
\end{eqnarray*}

In this reaction, two moles of product are formed for every three moles of reactants. If we increase the pressure on the closed system, the equilibrium will shift to the right because the forward reaction reduces the number of moles of gas that are present. This means that the yield of SO$_{3}$ will increase. The opposite will apply if the pressure on the system decreases. the equilibrium will shift to the left, and the concentration of SO$_{2}$ and O$_{2}$ will increase. 

\Tip{
The following rules will help in predicting the changes that take place in equilibrium reactions:\\

\begin{enumerate}
\item{If the forward reaction that forms the product is endothermic, then an increase in temperature will favour this reaction and the yield of product will increase. Lowering the temperature will decrease the product yield.}

\item{If the forward reaction that forms the product is exothermic, then a decrease in temperature will favour this reaction and the product yield will increase. Increasing the temperature will decrease the product yield.}

\item{Increasing the pressure favours the side of the equilibrium with the least number of gas molecules. This is shown in the balanced symbol equation. This rule applies in reactions with one or more gaseous reactants or products.}

\item{Decreasing the pressure favours the side of the equilibrium with the most number of gas molecules. This rule applies in reactions with one or more gaseous reactants or products.}

\item{If the concentration of a reactant (on the left) is increased, then some of it must change to the products (on the right) for equilibrium to be maintained. The equilibrium position will shift to the right.}

\item{If the concentration of a reactant (on the left) is decreased, then some of  the products (on the right) must change back to reactants for equilibrium to be maintained. The equilibrium position will shift to the left.}

\item{A \textbf{catalyst} does not affect the equilibrium position of a reaction. It only influences the \textit{rate of the reaction}, in other words, how quickly equilibrium is reached.}
\end{enumerate}
}

\begin{wex}{Reaction Rates 1}{
$2NO_{2}(g)\rightleftharpoons 2NO(g) + O_{2}(g)$ and  $\Delta H > 0$
How will the rate of the reverse reaction be affected by:
\begin{enumerate}
  \item a decrease in temperature?
  \item the addition of a catalyst?
  \item the addition of more NO gas?
\end{enumerate}}
  {
\begin{enumerate}
  \item         The rate of the forward reaction will increase since it is the forward reaction that is exothermix and therefore \textit{produces} energy to balance the loss of energy from the decrease in temperature. The rate of the reverse reaction will decrease. 
 
  \item The rate of the reverse and the forward reaction will increase.
  \item         The rate of the reverse reaction
  will increase so that the extra NO gas is converted into NO$_{2}$ gas. 
\end{enumerate}}  
\end{wex}

\begin{wex}{Reaction Rates 2}{
\begin{enumerate}
  \item Write a balanced equation for the exothermic reaction between
  Zn(s) and HCl. 
  \item Name 3 ways to increase the reaction rate between hydrochloric
  acid and zinc metal.     
\end{enumerate}}  
   {
   \begin{enumerate}
     \item $Zn(s) + 2HCl(aq) \rightleftharpoons  ZnCl_{2}(aq)+ H_{2}(g)$
     \item A catalyst could be added, the zinc solid could be ground
     into a fine powder to increase its surface area, the HCl
     concentration could be increased or the reaction temperature
     could be increased.        
   \end{enumerate}}  
\end{wex}  

\Exercise{Reaction rates and equilibrium\\}{

\begin{enumerate}
\item{The following reaction reaches equilibrium in a closed container:
\begin{center}
\rm${CaCO_{3}(s) \rightleftharpoons CaO(s) + CO_{2}(g)}$
\end{center}

The pressure of the system is increased by decreasing the volume of the container. How will the number of moles and the concentration of the CO$_{2}$(g) have changed when a new equilibrium is reached at the same temperature?

\begin{center}
\begin{tabular}{|c|c|c|}\hline
 & \textbf{moles of CO$_{2}$} & \textbf{[CO$_{2}$]} \\\hline
A & decreased & decreased \\\hline
B & increased & increased \\\hline
C & decreased & stays the same \\\hline
D & decreased & increased \\\hline
\end{tabular}
\end{center}
}
(IEB Paper 2, 2003)

\item{The following reaction has reached equilibrium in a closed container:

\begin{center}
\rm${C(s) + H_{2}O(g) \rightleftharpoons CO(g) + H_{2}(g)}$ $\Delta H$ > 0
\end{center}

The pressure of the system is then decreased by increasing the volume of the container. How will the concentration of the H$_{2}$(g) and the value of K$_{c}$ be affected when the new equilibrium is established? Assume that the temperature of the system remains unchanged.

\begin{center}
\begin{tabular}{|c|c|c|}\hline
 & \textbf{[H$_{2}$]} & \textbf{K$_{c}$} \\\hline
A & increases & increases \\\hline
B & increases & unchanged \\\hline
C & unchanged & unchanged \\\hline
D & decreases & unchanged \\\hline
\end{tabular}
\end{center}
}
(IEB Paper 2, 2004)

\item{During a classroom experiment copper metal reacts with concentrated nitric acid to produce NO$_{2}$ gas, which is collected in a gas syringe. When enough gas has collected in the syringe, the delivery tube is clamped so that no gas can escape. The brown NO$_{2}$ gas collected reaches an equilibrium with colourless N$_{2}$O$_{4}$ gas as represented by the following equation:

\begin{equation*}
2NO_{2} (g) \rightleftharpoons N_{2}O_{4} (g)
\end{equation*}

Once this equilibrium has been established, there are 0.01 moles of NO$_{2}$ gas and 0.03 moles of N$_{2}$O$_{4}$ gas present in the syringe.

	\begin{enumerate}
	\item{A learner, noticing that the colour of the gas mixture in the syringe is no longer changing, comments that all chemical reactions in the syringe must have stopped. Is this assumption correct? Explain.}
	\item{The gas in the syringe is cooled. The volume of the gas is kept constant during the cooling process. Will the gas be lighter or darker at the lower temperature? Explain your answer.}
	\item{The volume of the syringe is now reduced to 75 cm$^{3}$ by pushing the plunger in and holding it in the new position. There are 0.032 moles of N$_{2}$O$_{4}$ gas present once the equilibrium has been re-established at the reduced volume (75 cm$^{3}$). Calculate the value of the equilibrium constant for this equilibrium.}

(IEB Paper 2, 2004)
	\end{enumerate}
}

\item{Consider the following reaction, which takes place in a closed container:

\begin{center}
$\rm{A (s) + B (g) \rightarrow AB (g)}$ $\Delta$H $<$ 0
\end{center}

If you wanted to increase the rate of the reaction, which of the following would you do?
	\begin{enumerate}
	\item{decrease the concentration of B}
	\item{decrease the temperature of A}
	\item{grind A into a fine powder}
	\item{decrease the pressure}
	\end{enumerate}
(IEB Paper 2, 2002)
}

\item{Gases X and Y are pumped into a 2 dm$^{3}$ container. When the container is sealed, 4 moles of gas X and 4 moles of gas Y are present. The following equilibrium is established:

\begin{center}
$\rm{2X (g) + 3Y (g) \rightleftharpoons X_{2}Y_{3}}$
\end{center}

The graph below shows the number of moles of gas X and gas X$_{2}$Y$_{3}$ that are present from the time the container is sealed.

\scalebox{1} % Change this value to rescale the drawing.
{
\begin{pspicture}(0,-4.052031)(12.7675,4.032031)
\psline[linewidth=0.04cm](1.7675,4.012031)(1.7875,-1.9479686)
\psline[linewidth=0.04cm](1.8075,-1.9479686)(12.747499,-1.9279686)
\psline[linewidth=0.04cm,linestyle=dashed,dash=0.16cm 0.16cm](4.7675,-1.9079688)(4.7675,2.2320313)
\psline[linewidth=0.04cm,linestyle=dashed,dash=0.16cm 0.16cm](8.767501,-1.9079688)(8.787499,2.2920313)
\psline[linewidth=0.04cm,linestyle=dashed,dash=0.16cm 0.16cm](11.787499,-1.9279686)(11.807501,2.2320313)
\usefont{T1}{ptm}{m}{n}
\rput(4.74625,-2.1829689){\small 30}
\usefont{T1}{ptm}{m}{n}
\rput(10.775625,-3.0629687){\small time (s)}
\usefont{T1}{ptm}{m}{n}
\rput(1.250625,-1.4629687){\small 0,5}
\usefont{T1}{ptm}{m}{n}
\rput(1.3915626,1.9970312){\small 4}
\usefont{T1}{ptm}{m}{n}
\rput(0.596875,1.5570314){\small number}
\usefont{T1}{ptm}{m}{n}
\rput(8.684063,-2.2229688){\small 70}
\usefont{T1}{ptm}{m}{n}
\rput(11.839687,-2.2629688){\small 100}
\usefont{T1}{ptm}{m}{n}
\rput(0.531875,0.8770313){\small moles}
\usefont{T1}{ptm}{m}{n}
\rput(0.5640625,1.2170312){\small of}
\psline[linewidth=0.04cm,linestyle=dashed,dash=0.16cm 0.16cm](1.8075,-1.5079689)(2.7075,-1.5079689)
\psline[linewidth=0.04cm](2.6075,-1.5079688)(8.807501,-1.5079688)
\psline[linewidth=0.04cm](4.8075,1.0720314)(8.807501,1.0720314)
\psline[linewidth=0.04cm](9.6675,1.4120313)(12.12,1.432031)
\psline[linewidth=0.04cm](9.9875,-1.8279686)(12.1675,-1.8279686)
\psline[linewidth=0.04cm](11.6075,-3.0479689)(12.367499,-3.0479689)
\psline[linewidth=0.04cm](12.367499,-3.0479689)(12.1675,-2.8079689)
\psline[linewidth=0.04cm](12.347501,-3.0679688)(12.187501,-3.2679687)
\usefont{T1}{ptm}{m}{n}
\psbezier[linewidth=0.04](4.8,1.092031)(3.66,1.1320311)(2.34,1.052031)(1.78,2.092031)
\psbezier[linewidth=0.04](2.66,-1.5079689)(2.5085714,-1.447969)(1.8752381,-1.6783689)(1.78,-1.927969)
\psbezier[linewidth=0.04](8.66,-1.5079689)(8.96,-1.4879689)(9.2,-1.547969)(9.42,-1.687969)(9.64,-1.827969)(9.72,-1.807969)(10.04,-1.827969)
\psbezier[linewidth=0.04](8.62,1.072031)(9.22,1.052031)(9.26,1.3720311)(9.72,1.412031)
\end{pspicture} 
}


	\begin{enumerate}
	\item{How many moles of gas X$_{2}$Y$_{3}$ are formed by the time the reaction reaches equilibrium at 30 seconds?}
	\item{Calculate the value of the equilibrium constant at t = 50 s.}
	\item{At 70 s the temperature is increased. Is the forward reaction endothermic or exothermic? Explain in terms of Le Chatelier's Principle.}
	\item{How will this increase in temperature affect the value of the equilibrium constant?}
	\end{enumerate}
}

\end{enumerate}
}



% CHILD SECTION END 



% CHILD SECTION START 

\section{Industrial applications}
\label{sec:reactionrates:industrial}

The \textbf{Haber process} is a good example of an industrial process which uses the equilibrium principles that have been discussed. The equation for the process is as follows:

\begin{equation*}
N_{2}(g) + 3H_{2}(g) \rightleftharpoons 2NH_{3}(g) + energy 
\end{equation*}

Since the reaction is \textbf{exothermic}, the forward reaction is favoured at low temperatures, and the reverse reaction at high temperatures. If the purpose of the Haber process is to produce ammonia, then the temperature must be maintained at a level that is low enough to ensure that the reaction continues in the forward direction.\\

The forward reaction is also favoured by high \textit{pressures} because there are four moles of reactant for every two moles of product formed.\\

The K value for this reaction will be calculated as follows:

\begin{equation*}
K = \frac{[NH_{3}]^{2}}{[N_{2}][H_{2}]^{3}}
\end{equation*}

\Exercise{Applying equilibrium principles\\}{

Look at the values of k calculated for the Haber process reaction at different temperatures, and then answer the questions that follow:\\

\begin{center}
\begin{tabular}{|l|l|}\hline
\textbf{$T^{oC}$} & k \\\hline
25 & 6.4 x $10^{2}$\\\hline
200 & 4.4 x $10^{-1}$\\\hline
300 & 4.3 x $10^{-3}$\\\hline
400 & 1.6 x $10^{-4}$\\\hline
500 & 1.5 x $10^{-5}$\\\hline
\end{tabular}
\end{center}

\begin{enumerate}
\item{What happens to the value of K as the temperature increases?}
\item{Which reaction is being favoured when the temperature is 300 degrees celsius?}
\item{According to this table, which temperature would be best if you wanted to produce as much ammonia as possible? Explain.}
\end{enumerate}
}

\section{Summary}

\begin{itemize}
\item{The \textbf{rate of a reaction} describes how quickly reactants are used up, or how quickly products form. The units used are moles per second.}
\item{A number of factors can affect the rate of a reaction. These include the \textbf{nature of the reactants}, the \textbf{concentration} of reactants, \textbf{temperature} of the reaction, the presence or absence of a \textbf{catalyst} and the \textbf{surface area} of the reactants.}
\item{\textbf{Collision theory} provides one way of explaining why each of these factors can affect the rate of a reaction. For example, higher temperatures mean increased reaction rates because the reactant particles have more energy and are more likely to collide successfully with each other.}
\item{Different methods can be used to \textbf{measure the rate of a reaction}. The method used will depend on the nature of the product. Reactions that produce gases can be measured by collecting the gas in a syringe. Reactions that produce a precipitate are also easy to measure because the precipitate is easily visible.}
\item{For any reaction to occur, a minimum amount of energy is needed so that bonds in the reactants can break, and new bonds can form in the products. The minimum energy that is required is called the \textbf{activation energy} of a reaction.}
\item{In reactions where the particles do not have enough energy to overcome this activation energy, one of two methods can be used to facilitate a reaction to take place: increase the temperature of the reaction or add a catalyst.}
\item{\textbf{Increasing the temperature of a reaction} means that the average energy of the reactant particles increases and they are more likely to have enough energy to overcome the activation energy.}
\item{A \textbf{catalyst} is used to lower the activation energy so that the reaction is more likely to take place. A catalyst does this by providing an alternative, lower energy pathway, for the reaction.}
\item{A catalyst therefore \textbf{speeds up a reaction} but does not become part of the reaction in any way.}
\item{\textbf{Chemical equilibrium} is the state of a reaction, where the concentrations of the reactants and the products have no net change over time. Usually this occurs when the rate of the forward reaction is the same as the rate of the reverse reaction.}
\item{The \textbf{equilibrium constant} relates to reactions at equilibrium, and can be calculated using the following equation:

\begin{equation*}
K_{c} = \frac{[C]^{c}[D]^{d}}{[A]^{a}[B]^{b}} 
\end{equation*}

where A and B are reactants, C and D are products and a, b, c, and d are the coefficients of the respective reactants and products. }
\item{A \textbf{high K$_{c}$ value} means that the concentration of products at equilibrium is high and the reaction has a high yield. A \textbf{low K$_{c}$ value} means that the concentration of products at equilibrium is low and the reaction has a low yield.}
\item{\textbf{Le Chatelier's Principle} states that if a chemical system at equilibrium experiences a change in concentration, temperature or total pressure the equilibrium will shift in order to minimise that change and to re-establish equilibrium. For example, if the pressure of a gaseous system at eqilibrium was increased, the equilibrium would shift to favour the reaction that produces the lowest quantity of the gas. If the temperature of the same system was to increase, the equilibrium would shift to favour the endothermic reaction. Similar principles apply for changes in concentration of the reactants or products in a reaction.}
\item{The principles of equilibrium are very important in \textbf{industrial applications} such as the Haber process, so that productivity can be maximised.}

\end{itemize}


\Exercise{Summary Exercise\\}{
\begin{enumerate}
\item{For each of the following questions, choose the one correct answer from the list provided.}

	\begin{enumerate}	
	\item{Consider the following reaction that has reached equilibrium after some time in a sealed 1 dm$^{3}$ flask:
\begin{center}
\rm${PCl_{5}(g) \rightleftharpoons PCl_{3}(g) + Cl_{2}(g)}$; $\Delta H$ is positive
\end{center}

Which one of the following reaction conditions applied to the system would decrease the rate of the reverse reaction?

		\begin{enumerate}
		\item{increase the pressure}
		\item{increase the reaction temperature}
		\item{continually remove Cl$_{2}$(g) from the flask}
		\item{addition of a suitable catalyst}
		\end{enumerate}
}
(IEB Paper 2, 2001)

	\item{The following equilibrium constant expression is given for a particular reaction:
\begin{center}
\rm${K_{c} = [H_{2}O]^{4}[CO_{2}]^{3}/[C_{3}H_{8}][O_{2}]^{5}}$
\end{center}

For which one of the following reactions is the above expression of K$_{c}$ is correct?

		\begin{enumerate}
		\item{\rm${C_{3}H_{8}(g) + 5O_{2}(g) \rightleftharpoons 4H_{2}O(g) + 3CO_{2}(g)}$}
		\item{\rm${4H_{2}O(g) + 3CO_{2}(g) \rightleftharpoons C_{3}H_{8}(g) + 5O_{2}(g)}$}
		\item{\rm${2C_{3}H_{8}(g) + 7O_{2}(g) \rightleftharpoons 6CO(g) + 8H_{2}O(g)}$}
		\item{\rm${C_{3}H_{8}(g) + 5O_{2}(g) \rightleftharpoons 4H_{2}O(l) + 3CO_{2}(g)}$}
		\end{enumerate}
}
(IEB Paper 2, 2001)

	\end{enumerate}

\item{10 g of magnesium ribbon reacts with a 0.15 mol.dm$^{-3}$ solution of hydrochloric acid at a temperature of 25$^{0}$C.
	\begin{enumerate}
	\item{Write a balanced chemical equation for the reaction.}
	\item{State \textbf{two} ways of increasing the rate of production of H$_{2}$(g).}
	\item{A table of the results is given below:}
\begin{center}
\begin{tabular}{|c|c|}\hline
\textbf{Time elapsed (min)} & \textbf{Vol of H$_{2}$(g) (cm$^{3}$)}\\\hline
0 & 0 \\\hline
0.5 & 17 \\\hline
1.0 & 25 \\\hline
1.5 & 30 \\\hline
2.0 & 33 \\\hline
2.5 & 35 \\\hline
3.0 & 35 \\\hline
\end{tabular}
\end{center}
		\begin{enumerate}
		\item{Plot a graph of volume versus time for these results.}
		\item{Explain the shape of the graph during the following two time intervals: t = 0 to t = 2.0 min and then t = 2.5 and t = 3.0 min by referring to the volume of H$_{2}$(g) produced.}
		\end{enumerate}
	\end{enumerate}
}
(IEB Paper 2, 2001)

\item{Cobalt chloride crystals are dissolved in a beaker containing ethanol and then a few drops of water are added. After a period of time, the reaction reaches equilibrium as follows:
\begin{center}
\rm${CoCl_{4}^{2-}}$ (blue) \rm${ + 6H_{2}O \rightleftharpoons Co(H_{2}O)_{6}^{2+}}$ (pink) \rm${ + 4Cl^{-}}$
\end{center}

The solution, which is now just blue, is poured into three test tubes. State, in each case, what colour changes will be observed (if any) if the following are added in turn to each test tube:

	\begin{enumerate}
	\item{1 cm$^{3}$ of distilled water}
	\item{A few crystals of sodium chloride}
	\item{The addition of dilute hydrochloric acid to the third test tube causes the solution to turn pink. Explain why this occurs.}
	\end{enumerate}
}
(IEB Paper 2, 2001)


\end{enumerate}
}


% CHILD SECTION END 



% CHILD SECTION END 



% CHILD SECTION START 

