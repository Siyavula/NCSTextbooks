\chapter{Force, Momentum and Impulse}
\label{p:m:fmi11}



\section{Introduction}
In Grade 10 we studied motion but not what caused the motion. In this chapter we will learn that a net force is needed to cause motion. We recall what a force is and learn about how force and motion are related. We are introduced to two new concepts, momentum and impulse, and we learn more about turning forces and the force of gravity.\\
\chapterstartvideo{VPkgi}
\section{Force}
%\nts{This section could do with more activities, notably some simple experiments.}

\subsection{What is a \textit{force}?}
A force is anything that can cause a change to objects. Forces can:
\begin{itemize}
\item change the shape of an object
\item accelerate or stop an object
\item change the direction of a moving object.
\end{itemize}

A force can be classified as either a \textit{contact force} or a \textit{non-contact force}.

A contact force must touch or \textit{be in contact} with an object to cause a change. Examples of contact forces are:
\begin{itemize}
\item the force that is used to push or pull things, like on a door to open or close it
\item the force that a sculptor uses to turn clay into a pot
\item the force of the wind to turn a windmill
\end{itemize}

%\Activity{Contact Forces}{Write down 5 examples (excluding those given above) of contact forces that you see on your way to school.}

A non-contact force does not have to touch an object to cause a change. Examples of non-contact forces are:
\begin{itemize}
\item the force due to gravity, like the Earth pulling the Moon towards itself
\item the force due to electricity, like a proton and an electron attracting each other
\item the force due to magnetism, like a magnet pulling a paper clip towards itself
\end{itemize}

The unit of force is the \emph{newton} (symbol \emph{N}). This unit is named after Sir Isaac Newton who first defined force. Force is a vector quantity and has a magnitude and a direction. We use the abbreviation \emph{F} for force.

\begin{IFact}{There is a popular story that while Sir Isaac Newton was sitting under an apple tree, an apple fell on his head, and he suddenly thought of the Universal Law of Gravitation. Coincidentally, the weight of a small apple is approximately 1~N.}\end{IFact}

\begin{IFact}{Force was first described by Archimedes of Syracuse (circa 287 BC - 212 BC). Archimedes was a Greek mathematician, astronomer, philosopher, physicist and engineer. He was killed by a Roman soldier during the sack of the city, despite orders from the Roman general, Marcellus, that he was not to be harmed.}
\end{IFact}

This chapter will often refer to the \emph{resultant force} acting on an object. The resultant force is simply the vector sum of all the forces acting on the object. It is very important to remember that all the forces must be acting on the \emph{same} object. The resultant force is the force that has the same effect as all the other forces added together.

\subsection{Examples of Forces in Physics}
Most of Physics revolves around the study of forces. Although there are many different forces, all are handled in the same way. All forces in Physics can be put into one of four groups. These are gravitational forces, electromagnetic forces, strong nuclear forces and weak nuclear forces. You will mostly come across gravitational or electromagnetic forces at school.

\subsubsection{Gravitational Forces}
Gravity is the attractive force between two objects due to the mass of the objects. When you throw a ball in the air, its mass and the Earth's mass attract each other, which leads to a force between them. The ball falls back towards the Earth, and the Earth accelerates towards the ball. The movement of the Earth towards the ball is, however, so small that you couldn't possibly measure it.

\subsubsection{Electromagnetic Forces}
Almost all of the forces that we experience in everyday life are
electromagnetic in origin. They have this unusual name because long ago people thought that electric forces and magnetic forces were different things. After much work and experimentation, it has been realised that they are actually different manifestations of the same underlying theory.

\subsubsection{Electric or Electrostatic Forces}
If we have objects carrying electrical charge, which are not moving, then we are dealing with electrostatic forces (Coulomb's Law). This force is actually much stronger than gravity. This may seem strange, since gravity is obviously very powerful, and holding a balloon to the wall seems to be the most impressive thing electrostatic forces have done, but if we think about it: for gravity to be detectable, we need to have a very large mass nearby. But a balloon rubbed against someone's hair can stick to a wall with a force so strong that it overcomes the force of gravity between the entire Earth and the balloon---with just the charges in the balloon and the wall!

\subsubsection{Magnetic Forces}
The magnetic force is a different manifestation of the electromagnetic force. It stems from the interaction between \emph{moving charges} as opposed to the \emph{fixed} charges involved in Coulomb's Law. Examples of the magnetic force in action include magnets, compasses, car engines and computer data storage. Magnets are also used in the wrecking industry to pick up cars and move them around sites.

\subsubsection{Friction}
According to Newton's First Law (we will discuss this later in the chapter) an object moving \emph{without a force acting on it} will keep on moving. Then why does a box sliding on a table come to a stop? The answer is friction. Friction arises where two surfaces are in contact and moving relative to each other as a result of the interaction between the molecules of the two contact surfaces---for instance the interactions between the molecules on the bottom of the box with molecules on the top of the table. This interaction is electromagnetic in origin, hence friction is just another view of the electromagnetic force. Later in this chapter we will discuss frictional forces a little more.

\subsubsection{Drag Forces}
This is the force an object experiences while travelling through a medium like an aeroplane flying through air. When something travels through the air it needs to displace air as it travels and because of this, the air exerts a force on the object. This becomes an important force when you move fast and a lot of thought is taken to try and reduce the amount of drag force a sports car or an aeroplane experiences. The drag force is very useful for parachutists. They jump from high altitudes and if there was no drag force, then they would continue accelerating all the way to the ground. Parachutes are wide because the more surface area you have, the greater the drag force and hence the slower you hit the ground.

\subsection{Systems and External Forces}
The concepts of systems and forces external to such systems are very important in Physics. A system is any collection of objects. If one draws an imaginary box around such a system then an external force is one that
is applied by an object or person outside the box. Imagine for example a car pulling two trailers.

\begin{figure}[H]
\begin{center}
\scalebox{1} % Change this value to rescale the drawing.
{
\begin{pspicture}(0,-1.62)(8.82,1.64)
\psline[linewidth=0.04cm](5.9,-0.4)(5.4,-0.4)
\psline[linewidth=0.04cm](5.4,-0.4)(5.4,-1.4)
\psline[linewidth=0.04cm](5.4,-1.4)(8.4,-1.4)
\psline[linewidth=0.04cm](8.4,-1.4)(8.4,-0.5)
\psline[linewidth=0.04cm](8.4,-0.5)(7.7,-0.5)
\psline[linewidth=0.04cm](7.7,-0.5)(7.5,0.2)
\psline[linewidth=0.04cm](7.5,0.2)(6.1,0.2)
\psline[linewidth=0.04cm](6.1,0.2)(5.9,-0.4)
\psline[linewidth=0.08cm](5.4,-0.9)(4.7,-0.9)
\psframe[linewidth=0.04,dimen=outer](4.7,-0.4)(2.7,-1.4)
\psframe[linewidth=0.04,dimen=outer](2.4,-0.4)(0.4,-1.4)
\pscircle[linewidth=0.04,dimen=outer](0.9,-1.4){0.2}
\pscircle[linewidth=0.04,dimen=outer](1.9,-1.4){0.2}
\pscircle[linewidth=0.04,dimen=outer](3.2,-1.4){0.2}
\pscircle[linewidth=0.04,dimen=outer](4.2,-1.4){0.2}
\pscircle[linewidth=0.04,dimen=outer](6.15,-1.25){0.35}
\pscircle[linewidth=0.04,dimen=outer](7.65,-1.25){0.35}
\psline[linewidth=0.08cm](2.7,-0.9)(2.4,-0.9)
\psline[linewidth=0.08cm](0.6,1.6)(0.6,1.6)
\psline[linewidth=0.04cm](0.0,-1.6)(8.8,-1.6)
\psframe[linewidth=0.04,linestyle=dashed,dash=0.16cm 0.16cm,dimen=outer](5.0,0.1)(0.2,-1.6)
\usefont{T1}{ptm}{m}{n}
\rput(1.4071875,-0.89){B}
\usefont{T1}{ptm}{m}{n}
\rput(3.6265626,-0.89){A}
\end{pspicture}
}
\end{center}
\end{figure}

If we draw a box around the two trailers they can be considered a closed system or unit. When we look at the forces on this closed system the following forces will apply (we assume drag forces are absent):
\begin{itemize}
\item The force of the car pulling the unit (trailer A and B)
\item The force of friction between the wheels of the trailers and the road (opposite to the direction of motion)
\item The force of the Earth pulling downwards on the system (gravity)
\item The force of the road pushing upwards on the system
\end{itemize}

These forces are called external forces to the system.\\
The following forces will not apply:
\begin{itemize}
\item The force of A pulling B
\item The force of B pulling A
\item The force of friction between the wheels of the car and the road (opposite to the direction of motion)
\end{itemize}

We can also draw a box around trailer A or B, in which case the forces will be different.

\begin{figure}[H]
\begin{center}
\scalebox{1} % Change this value to rescale the drawing.
{
\begin{pspicture}(0,-1.62)(8.82,1.64)
\psline[linewidth=0.04cm](5.9,-0.4)(5.4,-0.4)
\psline[linewidth=0.04cm](5.4,-0.4)(5.4,-1.4)
\psline[linewidth=0.04cm](5.4,-1.4)(8.4,-1.4)
\psline[linewidth=0.04cm](8.4,-1.4)(8.4,-0.5)
\psline[linewidth=0.04cm](8.4,-0.5)(7.7,-0.5)
\psline[linewidth=0.04cm](7.7,-0.5)(7.5,0.2)
\psline[linewidth=0.04cm](7.5,0.2)(6.1,0.2)
\psline[linewidth=0.04cm](6.1,0.2)(5.9,-0.4)
\psline[linewidth=0.08cm](5.4,-0.9)(4.7,-0.9)
\psframe[linewidth=0.04,dimen=outer](4.7,-0.4)(2.7,-1.4)
\psframe[linewidth=0.04,dimen=outer](2.4,-0.4)(0.4,-1.4)
\pscircle[linewidth=0.04,dimen=outer](0.9,-1.4){0.2}
\pscircle[linewidth=0.04,dimen=outer](1.9,-1.4){0.2}
\pscircle[linewidth=0.04,dimen=outer](3.2,-1.4){0.2}
\pscircle[linewidth=0.04,dimen=outer](4.2,-1.4){0.2}
\pscircle[linewidth=0.04,dimen=outer](6.15,-1.25){0.35}
\pscircle[linewidth=0.04,dimen=outer](7.65,-1.25){0.35}
\psline[linewidth=0.08cm](2.7,-0.9)(2.4,-0.9)
\psline[linewidth=0.08cm](0.6,1.6)(0.6,1.6)
\psline[linewidth=0.04cm](0.0,-1.6)(8.8,-1.6)
\psframe[linewidth=0.04,linestyle=dashed,dash=0.16cm 0.16cm,dimen=outer](4.9,0.1)(2.5,-1.6)
\usefont{T1}{ptm}{m}{n}
\rput(1.4071875,-0.89){B}
\usefont{T1}{ptm}{m}{n}
\rput(3.6265626,-0.89){A}
\end{pspicture}
}
\end{center}
\end{figure}

If we consider trailer A as a system, the following external forces will apply:
\begin{itemize}
\item The force of the car pulling on A (towards the right)
\item The force of B pulling on A (towards the left)
\item The force of the Earth pulling downwards on the trailer (gravity)
\item The force of the road pushing upwards on the trailer
\item The force of friction between the wheels of A and the road (opposite to the direction of motion)
\end{itemize}

\subsection{Force Diagrams}
If we look at the example above and draw a force diagram of all the forces acting on the two-trailer-unit, the diagram would look like this:


\begin{figure}[H]
\begin{center}
\scalebox{1} % Change this value to rescale the drawing.
{
\begin{pspicture}(0,-2.75)(14.839063,2.75)
\psframe[linewidth=0.04,dimen=outer](5.5540624,0.61)(3.7940626,-0.45)
\psframe[linewidth=0.04,dimen=outer](7.9740624,0.59)(6.1940627,-0.47)
\psline[linewidth=0.08cm](5.5540624,0.05)(6.2140627,0.05)
\pscircle[linewidth=0.04,dimen=outer](4.3140626,-0.45){0.22}
\pscircle[linewidth=0.04,dimen=outer](5.0140624,-0.45){0.22}
\pscircle[linewidth=0.04,dimen=outer](6.6840625,-0.46){0.21}
\pscircle[linewidth=0.04,dimen=outer](7.4440627,-0.46){0.21}
\psframe[linewidth=0.04,linestyle=dashed,dash=0.16cm 0.16cm,dimen=outer](8.254063,1.19)(3.4140625,-0.97)
\psline[linewidth=0.04cm,arrowsize=0.05291667cm 2.0,arrowlength=1.4,arrowinset=0.4]{->}(8.254063,0.15)(9.974063,0.13)
\psline[linewidth=0.04cm,arrowsize=0.05291667cm 2.0,arrowlength=1.4,arrowinset=0.4]{->}(3.4140625,0.19)(2.4340625,0.19)
\psline[linewidth=0.04cm,arrowsize=0.05291667cm 2.0,arrowlength=1.4,arrowinset=0.4]{->}(5.8340626,-0.97)(5.8540626,-2.73)
\psline[linewidth=0.04cm,arrowsize=0.05291667cm 2.0,arrowlength=1.4,arrowinset=0.4]{->}(5.8140626,1.19)(5.8340626,2.73)
\psline[linewidth=0.04cm](5.6940627,2.05)(5.9540625,2.05)
\psline[linewidth=0.04cm](5.6940627,1.99)(5.9540625,1.99)
\psline[linewidth=0.04cm](5.6940627,-1.65)(5.9540625,-1.65)
\psline[linewidth=0.04cm](5.7140627,-1.71)(5.9740624,-1.71)
\usefont{T1}{ptm}{m}{n}
\rput(11.663593,0.56){F$_1$: Force of car on trailers (to the right)}
\usefont{T1}{ptm}{m}{n}
\rput(1.6010938,0.52){F$_f$: Frictional force }
\usefont{T1}{ptm}{m}{n}
\rput(8.964844,1.94){F$_N$: Upward force of road on trailers}
\usefont{T1}{ptm}{m}{n}
\rput(9.514844,-1.8){F$_g$: Downward force of Earth on trailers}
\usefont{T1}{ptm}{m}{n}
\rput(1.6904688,-0.12){on trailers (to the left)}
\end{pspicture}
}
\end{center}
\end{figure}

It is important to keep the following in mind when you draw force diagrams:
\begin{itemize}
\item Make your drawing large and clear.
\item You must use arrows and the direction of the arrow will show the direction of the force.
\item The length of the arrow will indicate the size of the force, in other words, the longer arrows in the diagram (F$_1$ for example) indicates a bigger force than a shorter arrow (F$_f$). Arrows of the same length indicate forces of equal size (F$_N$ and F$_g$). Use ``little lines'' like in maths to show this.
\item Draw neat lines using a ruler. The arrows must touch the system or object.
\item All arrows must have labels. Use letters with a key on the side if you do not have enough space on your drawing.
\item The labels must indicate what is applying the force (the force of the car?) on what the force is applied (?on the trailer?) and in which direction (to the right)
\item If the values of the forces are known, these values can be added to the diagram or key.
\end{itemize}

\begin{wex}{Force diagrams}
{Draw a labelled force diagram to indicate all the forces acting on trailer A in the example above.}
{
\westep{Draw a large diagram of the picture from your question}
%diagram 2a
\begin{figure}[H]
\begin{center}
\scalebox{1} % Change this value to rescale the drawing.
{
\begin{pspicture}(0,-0.72)(2.1,0.72)
\psframe[linewidth=0.04,dimen=outer](2.1,0.72)(0.0,-0.48)
\pscircle[linewidth=0.04,dimen=outer](0.69,-0.47){0.23}
\pscircle[linewidth=0.04,dimen=outer](1.51,-0.49){0.23}
\end{pspicture}
}
\end{center}
\end{figure}

\westep{Add all the forces}
%diagram 2b
\begin{figure}[H]
\begin{center}
\scalebox{1} % Change this value to rescale the drawing.
{
\begin{pspicture}(0,-2.17)(5.48,2.17)
\psframe[linewidth=0.04,dimen=outer](3.64,0.57)(1.54,-0.63)
\pscircle[linewidth=0.04,dimen=outer](2.23,-0.62){0.23}
\psline[linewidth=0.04cm,arrowsize=0.05291667cm 2.0,arrowlength=1.4,arrowinset=0.4]{->}(3.64,0.03)(5.46,0.03)
\psline[linewidth=0.04cm,arrowsize=0.05291667cm 2.0,arrowlength=1.4,arrowinset=0.4]{->}(1.54,0.03)(0.9,0.01)
\psline[linewidth=0.04cm,arrowsize=0.05291667cm 2.0,arrowlength=1.4,arrowinset=0.4]{->}(2.62,-0.63)(2.62,-2.15)
\psline[linewidth=0.04cm,arrowsize=0.05291667cm 2.0,arrowlength=1.4,arrowinset=0.4]{->}(2.62,0.55)(2.62,2.15)
\pscircle[linewidth=0.04,dimen=outer](3.05,-0.64){0.23}
\psdots[dotsize=0.12](2.62,-0.63)
\psline[linewidth=0.04cm,arrowsize=0.05291667cm 2.0,arrowlength=1.4,arrowinset=0.4]{->}(1.54,-0.31)(0.0,-0.33)
\psline[linewidth=0.04cm](2.46,1.49)(2.78,1.49)
\psline[linewidth=0.04cm](2.44,1.39)(2.76,1.39)
\psline[linewidth=0.04cm](2.44,-1.19)(2.78,-1.19)
\psline[linewidth=0.04cm](2.46,-1.29)(2.8,-1.29)
\psdots[dotsize=0.12](2.62,0.57)
\psdots[dotsize=0.12](1.56,0.05)
\psdots[dotsize=0.12](1.56,-0.31)
\psdots[dotsize=0.12](3.62,0.01)
\end{pspicture}
}
\end{center}
\end{figure}

\westep{Add the labels}

\begin{figure}[H]
\begin{center}
\scalebox{1} % Change this value to rescale the drawing.
{
\begin{pspicture}(0,-2.17)(11.779062,2.17)
\psframe[linewidth=0.04,dimen=outer](5.1740627,0.63)(3.3940625,-0.43)
\pscircle[linewidth=0.04,dimen=outer](3.8840625,-0.42){0.21}
\pscircle[linewidth=0.04,dimen=outer](4.6440625,-0.42){0.21}
\psline[linewidth=0.04cm,arrowsize=0.05291667cm 2.0,arrowlength=1.4,arrowinset=0.4]{->}(5.1540623,0.13)(7.3340626,0.11)
\psline[linewidth=0.04cm,arrowsize=0.05291667cm 2.0,arrowlength=1.4,arrowinset=0.4]{->}(3.3940625,0.25)(2.7340624,0.23)
\psline[linewidth=0.04cm,arrowsize=0.05291667cm 2.0,arrowlength=1.4,arrowinset=0.4]{->}(4.2940626,-0.39)(4.3140626,-2.15)
\psline[linewidth=0.04cm,arrowsize=0.05291667cm 2.0,arrowlength=1.4,arrowinset=0.4]{->}(4.2740626,0.61)(4.2940626,2.15)
\psline[linewidth=0.04cm](4.1540623,1.47)(4.4140625,1.47)
\psline[linewidth=0.04cm](4.1540623,1.41)(4.4140625,1.41)
\psline[linewidth=0.04cm](4.1540623,-1.05)(4.4140625,-1.05)
\psline[linewidth=0.04cm](4.1740627,-1.13)(4.4340625,-1.13)
\usefont{T1}{ptm}{m}{n}
\rput(8.503593,0.4){F$_1$: Force of car on trailer A (to the right)}
\usefont{T1}{ptm}{m}{n}
\rput(1.6010938,0.52){F$_f$: Frictional force }
\usefont{T1}{ptm}{m}{n}
\rput(7.5390625,1.36){F$_N$: Upward force of road on trailer A}
\usefont{T1}{ptm}{m}{n}
\rput(8.029062,-1.14){F$_g$: Downward force of Earth on trailer A}
\usefont{T1}{ptm}{m}{n}
\rput(1.8710938,-0.58){F$_B$: Force of trailer B }
\usefont{T1}{ptm}{m}{n}
\rput(1.7304688,-1.04){on trailer A (to the left)}
\psline[linewidth=0.04cm,arrowsize=0.05291667cm 2.0,arrowlength=1.4,arrowinset=0.4]{->}(3.4140625,-0.05)(2.0940626,-0.05)
\usefont{T1}{ptm}{m}{n}
\rput(4.280625,0.14){A}
\end{pspicture}
}
\end{center}
\end{figure}
}
\end{wex}

\subsection{Free Body Diagrams}
In a free-body diagram, the object of interest is drawn as a dot and all the forces acting on it are drawn as arrows pointing away from the dot.
A free body diagram for the two-trailer-system will therefore look like this:\\

\begin{figure}[H]
\begin{center}
\scalebox{1} % Change this value to rescale the drawing.
{
\begin{pspicture}(0,-1.9392188)(10.521563,1.9592187)
\psline[linewidth=0.04cm,arrowsize=0.05291667cm 2.0,arrowlength=1.4,arrowinset=0.4]{->}(1.3540626,-0.17921875)(3.0740626,-0.19921875)
\psline[linewidth=0.04cm,arrowsize=0.05291667cm 2.0,arrowlength=1.4,arrowinset=0.4]{->}(1.3140625,-0.17921875)(0.3340625,-0.17921875)
\psline[linewidth=0.04cm,arrowsize=0.05291667cm 2.0,arrowlength=1.4,arrowinset=0.4]{->}(1.3540626,-0.15921874)(1.3740625,-1.9192188)
\psline[linewidth=0.04cm,arrowsize=0.05291667cm 2.0,arrowlength=1.4,arrowinset=0.4]{->}(1.3540626,-0.19921875)(1.3740625,1.3407812)
\psline[linewidth=0.04cm](1.2340626,0.66078126)(1.4940625,0.66078126)
\psline[linewidth=0.04cm](1.2340626,0.60078126)(1.4940625,0.60078126)
\psline[linewidth=0.04cm](1.2340626,-0.77921873)(1.4940625,-0.77921873)
\psline[linewidth=0.04cm](1.2340626,-0.83921874)(1.4940625,-0.83921874)
\psdots[dotsize=0.18](1.3540626,-0.17921875)
\usefont{T1}{ptm}{m}{n}
\rput(6.8035936,1.7707813){F$_1$: Force of car on trailers (to the right)}
\usefont{T1}{ptm}{m}{n}
\rput(6.873594,1.2907813){F$_f$: Frictional force on trailers (to the left)}
\usefont{T1}{ptm}{m}{n}
\rput(7.054844,0.81078124){F$_g$: Downward force of Earth on trailers}
\usefont{T1}{ptm}{m}{n}
\rput(6.5248437,0.39078125){F$_N$: Upward force of road on trailers}
\usefont{T1}{ptm}{m}{n}
\rput(2.1420312,0.03078125){F$_1$}
\usefont{T1}{ptm}{m}{n}
\rput(1.9320313,-0.86921877){F$_g$}
\usefont{T1}{ptm}{m}{n}
\rput(0.88203124,0.93078125){F$_N$}
\usefont{T1}{ptm}{m}{n}
\rput(0.41203126,-0.46921876){F$_f$}
\end{pspicture}
}
\end{center}
\end{figure}

\begin{wex}{Free body diagram}
{Draw a free body diagram of all the forces acting on trailer A in the example above.}
{
\westep{Draw a dot to indicate the object}
%[Drawing of a dot]
\begin{figure}[H]
\begin{center}
\scalebox{1} % Change this value to rescale the drawing.
{
\begin{pspicture}(0,-0.1)(0.18,0.1)
\psdots[dotsize=0.16](0.08,0.0)
\end{pspicture}
}
\end{center}
\end{figure}

\westep{Draw arrows to indicate all the forces acting on the object}
%Drawing 3b:
\begin{figure}[H]
\begin{center}
\scalebox{1} % Change this value to rescale the drawing.
{
\begin{pspicture}(0,-1.65)(3.94,1.65)
\psline[linewidth=0.04cm,arrowsize=0.05291667cm 2.0,arrowlength=1.4,arrowinset=0.4]{->}(1.72,0.13)(3.92,0.13)
\psline[linewidth=0.04cm,arrowsize=0.05291667cm 2.0,arrowlength=1.4,arrowinset=0.4]{->}(1.68,0.11)(0.7,0.11)
\psline[linewidth=0.04cm,arrowsize=0.05291667cm 2.0,arrowlength=1.4,arrowinset=0.4]{->}(1.72,0.13)(1.74,-1.63)
\psline[linewidth=0.04cm,arrowsize=0.05291667cm 2.0,arrowlength=1.4,arrowinset=0.4]{->}(1.72,0.09)(1.74,1.63)
\psline[linewidth=0.04cm](1.6,0.95)(1.86,0.95)
\psline[linewidth=0.04cm](1.6,0.89)(1.86,0.89)
\psline[linewidth=0.04cm](1.6,-0.49)(1.86,-0.49)
\psline[linewidth=0.04cm](1.6,-0.55)(1.86,-0.55)
\psdots[dotsize=0.18](1.72,0.11)
\psline[linewidth=0.04cm,arrowsize=0.05291667cm 2.0,arrowlength=1.4,arrowinset=0.4]{->}(1.7,0.01)(0.0,-0.01)
\end{pspicture}
}
\end{center}
\end{figure}

\westep{Label the forces}
\begin{figure}[H]
\begin{center}
\scalebox{1} % Change this value to rescale the drawing.
{
\begin{pspicture}(0,-1.9192188)(11.375937,1.9392188)
\psline[linewidth=0.04cm,arrowsize=0.05291667cm 2.0,arrowlength=1.4,arrowinset=0.4]{->}(1.72,-0.13921875)(3.92,-0.13921875)
\psline[linewidth=0.04cm,arrowsize=0.05291667cm 2.0,arrowlength=1.4,arrowinset=0.4]{->}(1.68,-0.15921874)(0.7,-0.15921874)
\psline[linewidth=0.04cm,arrowsize=0.05291667cm 2.0,arrowlength=1.4,arrowinset=0.4]{->}(1.72,-0.13921875)(1.74,-1.8992188)
\psline[linewidth=0.04cm,arrowsize=0.05291667cm 2.0,arrowlength=1.4,arrowinset=0.4]{->}(1.72,-0.17921875)(1.74,1.3607812)
\psline[linewidth=0.04cm](1.6,0.68078125)(1.86,0.68078125)
\psline[linewidth=0.04cm](1.6,0.62078124)(1.86,0.62078124)
\psline[linewidth=0.04cm](1.6,-0.75921875)(1.86,-0.75921875)
\psline[linewidth=0.04cm](1.6,-0.81921875)(1.86,-0.81921875)
\psdots[dotsize=0.18](1.72,-0.15921874)
\psline[linewidth=0.04cm,arrowsize=0.05291667cm 2.0,arrowlength=1.4,arrowinset=0.4]{->}(1.7,-0.25921875)(0.0,-0.27921876)
\usefont{T1}{ptm}{m}{n}
\rput(7.489531,1.7507813){F$_1$: Force of car on trailer A (to the right)}
\usefont{T1}{ptm}{m}{n}
\rput(3.0079687,0.15078124){F$_1$}
\usefont{T1}{ptm}{m}{n}
\rput(7.789531,1.3507812){F$_B$: Force of trailer B on trailer A (to the left)}
\usefont{T1}{ptm}{m}{n}
\rput(0.53796875,-0.44921875){F$_B$}
\usefont{T1}{ptm}{m}{n}
\rput(7.579531,0.9507812){F$_f$: Frictional force on trailer A (to the left)}
\usefont{T1}{ptm}{m}{n}
\rput(1.0779687,0.25078124){F$_f$}
\usefont{T1}{ptm}{m}{n}
\rput(7.795,0.55078125){F$_g$: Downward force of Earth on trailer A}
\usefont{T1}{ptm}{m}{n}
\rput(2.2979689,-1.1492188){F$_g$}
\usefont{T1}{ptm}{m}{n}
\rput(7.285,0.15078124){F$_N$: Upward force of road on trailer A}
\usefont{T1}{ptm}{m}{n}
\rput(2.3479688,1.3507812){F$_N$}
\end{pspicture}
}
\end{center}
\end{figure}
}
\end{wex}


\subsection{Finding the Resultant Force}
The easiest way to determine a resultant force is to draw a free body diagram. Remember from Chapter~\ref{chap:vectors} that we use the length of the arrow to indicate the vector's magnitude and the direction of the arrow to show which direction it acts in.

After we have done this, we have a diagram of vectors and we simply find the sum of the vectors to get the resultant force.
\begin{figure}[H]
\begin{center}
\begin{pspicture}(-1,-0.5)(5,1)
%\psgrid
\psline[linewidth=2pt]{->}(-1,0.5)(0,0.5)

\psline[linewidth=2pt]{<-}(1,0.5)(2.5,0.5)
\psframe[linewidth=1pt](0,0)(1,1)
\uput[u](-0.5,0.5){4 N}
\uput[u](1.5,0.5){6 N}
\uput[d](0.5,0){(a)}
\rput(4.5,0.5){
\psline[linewidth=1pt]{<->}(-1.5,0)(1,0)
\psdot[dotsize=0.2](0,0)
\uput[u](-0.5,0){6 N}
\uput[u](0.5,0){4 N}
\uput[d](0,-0.5){(b)}
}
\end{pspicture}
\caption{(a) Force diagram of 2 forces acting on a box. (b) Free body diagram of the box.}
\label{forces:1}
\end{center}
\end{figure}

For example, two people push on a box from opposite sides with forces of 4~N and 6~N respectively as shown in Figure~\ref{forces:1}(a). The free body diagram in Figure~\ref{forces:1}(b) shows the object represented by a dot and the two forces are represented by arrows with their tails on the dot.

As you can see, the arrows point in opposite directions and have different lengths. The resultant force is 2~N to the left. This result can be obtained algebraically too, since the two forces act along the same line. First, as in motion in one direction, choose a frame of reference. Secondly, add the two vectors taking their directions into account.

For the example, assume that the positive direction is to the right, then:
\begin{eqnarray*}
F_{R} &=& (+4\eN)+(-6\eN)\\
&=& -2\eN\\
&=& 2 \eN~ \mbox{to the left}
\end{eqnarray*}

Remember that a negative answer means that the force acts in the \emph{opposite} direction to the one that you chose to be positive. You can \emph{choose} the positive direction to be any way you want, but once you have chosen it you \emph{must} keep it.

As you work with more force diagrams in which the forces exactly balance, you may notice that you get a zero answer (e.g.\@{} 0~N). This simply means that the forces are balanced and that the object will not accelerate.

Once a force diagram has been drawn the techniques of vector addition introduced in Chapter~\ref{chap:vectors} can be used. Depending on the situation you might choose to use a graphical technique such as the tail-to-head method or the parallelogram method, or else an algebraic approach to determine the resultant. Since force is a vector quantity all of these methods apply.

\begin{wex}{Finding the resultant force}
{A car (mass 1200 kg) applies a force of 2000 N on a trailer (mass 250 kg). A constant frictional force of 200 N is acting on the trailer, and a constant frictional force of 300 N is acting on the car.
\begin{enumerate}
\item Draw a force diagram of all the forces acting on the car.
\item Draw a free body diagram of all the horizontal forces acting on the trailer.
\item Use the force diagram to determine the resultant force on the trailer.
\end{enumerate}
}
{
\westep{Draw the force diagram for the car.}
The question asks us to draw all the forces on the car. This means that we must include horizontal and vertical forces.
\begin{figure}[H]
\begin{center}
\scalebox{1} % Change this value to rescale the drawing.
{
\begin{pspicture}(0,-2.17)(9.690625,2.17)
\pscircle[linewidth=0.04,dimen=outer](5.0528126,-0.62){0.23}
\psline[linewidth=0.04cm,arrowsize=0.05291667cm 2.0,arrowlength=1.4,arrowinset=0.4]{->}(4.9828124,-0.83)(4.3428125,-0.85)
\psline[linewidth=0.04cm,arrowsize=0.05291667cm 2.0,arrowlength=1.4,arrowinset=0.4]{->}(5.4428124,-0.63)(5.4428124,-2.15)
\psline[linewidth=0.04cm,arrowsize=0.05291667cm 2.0,arrowlength=1.4,arrowinset=0.4]{->}(5.4428124,0.55)(5.4428124,2.15)
\pscircle[linewidth=0.04,dimen=outer](5.8728123,-0.64){0.23}
\psdots[dotsize=0.12](5.4428124,-0.63)
\psline[linewidth=0.04cm,arrowsize=0.05291667cm 2.0,arrowlength=1.4,arrowinset=0.4]{->}(4.3628125,-0.31)(2.8228126,-0.33)
\psline[linewidth=0.04cm](5.2828126,1.49)(5.6028123,1.49)
\psline[linewidth=0.04cm](5.2628126,1.39)(5.5828123,1.39)
\psline[linewidth=0.04cm](5.2628126,-1.19)(5.6028123,-1.19)
\psline[linewidth=0.04cm](5.2828126,-1.29)(5.6228123,-1.29)
\psdots[dotsize=0.12](5.4428124,0.57)
\psdots[dotsize=0.12](5.0428123,-0.87)
\psdots[dotsize=0.12](4.3828125,-0.31)
\psline[linewidth=0.04cm](4.3828125,-0.61)(6.4628124,-0.61)
\psline[linewidth=0.04cm](6.4428124,-0.61)(6.4428124,-0.07)
\psline[linewidth=0.04cm](6.4428124,-0.07)(5.9828124,-0.03)
\psline[linewidth=0.04cm](5.9828124,-0.03)(5.7828126,0.57)
\psline[linewidth=0.04cm](5.7828126,0.57)(4.8828125,0.57)
\psline[linewidth=0.04cm](4.8828125,0.57)(4.6828127,0.09)
\psline[linewidth=0.04cm](4.6828127,0.13)(4.3828125,0.05)
\psline[linewidth=0.04cm](4.3628125,0.05)(4.3628125,-0.65)
\usefont{T1}{ptm}{m}{n}
\rput(6.9142184,1.66){F$_N$: Upward force}
\usefont{T1}{ptm}{m}{n}
\rput(7.327657,-1.32){F$_g$: Downward force of}
\usefont{T1}{ptm}{m}{n}
\rput(2.7990625,-1.06){F$_f$: Frictional force on car }
\usefont{T1}{ptm}{m}{n}
\rput(2.1590624,0.34){F$_1$: Force of trailer on car}
\usefont{T1}{ptm}{m}{n}
\rput(1.4904687,-0.06){(to the left) (2000 N)}
\usefont{T1}{ptm}{m}{n}
\rput(2.143125,-1.46){(to the left) (300 N)}
\usefont{T1}{ptm}{m}{n}
\rput(7.646406,-1.72){the Earth on car (12 000 N)}
\usefont{T1}{ptm}{m}{n}
\rput(7.5920315,1.26){of road on car (12000 N)}
\end{pspicture}  
}
\end{center}
\end{figure}
\westep{Draw the free body diagram for the trailer.}
The question only asks for horizontal forces. We will therefore not include the force of the Earth on the trailer, or the force of the road on the trailer as these forces are in a vertical direction.
\begin{figure}[H]
\begin{center}
\scalebox{1} % Change this value to rescale the drawing.
{
\begin{pspicture}(0,-0.638125)(14.759063,0.638125)
\psline[linewidth=0.04cm,arrowsize=0.05291667cm 2.0,arrowlength=1.4,arrowinset=0.4]{->}(6.3,0.0)(8.5,0.0)
\psline[linewidth=0.04cm,arrowsize=0.05291667cm 2.0,arrowlength=1.4,arrowinset=0.4]{->}(6.2,0.0)(5.2,0.0)
\psdots[dotsize=0.18](6.3,0)
\usefont{T1}{ptm}{m}{n}
\rput(9.2,0.8){F$_1$: Force of car on trailer}
\rput(9.2,0.4){(to the right) (2 000 N)}
\usefont{T1}{ptm}{m}{n}
\rput(3.8,-0.4){F$_f$: Frictional force on trailer (to the left) (200 N)}
\end{pspicture}
}
\end{center}
\end{figure}

\westep{Determine the resultant force on the trailer.}

To find the resultant force we need to add all the horizontal forces together. We do not add vertical forces as the movement of the car and trailer will be in a horizontal direction, and not up or down.
F$_{R}$ = 2000 + (-200) = 1800 N to the right.
}
\end{wex}
% Phet simulation on forces and motion: SIYAVULA-SIMULATION:http://cnx.org/content/m38958/latest/#id63458
\simulation{sim on forces and motion}{VPkgj}
\Exercise{title}{
\begin{enumerate}
\item{A force acts on an object. Name three effects that the force can have on the object.}
\item{Identify each of the following forces as contact or non-contact forces.
\begin{enumerate}
\item The force between the north pole of a magnet and a paper clip.
\item The force required to open the door of a taxi.
\item The force required to stop a soccer ball.
\item The force causing a ball, dropped from a height of 2 m, to fall to the floor.
\end{enumerate}
}
\item{A book of mass 2 kg is lying on a table. Draw a labelled force diagram indicating all the forces on the book.}
\item{A boy pushes a shopping trolley (mass 15 kg) with a constant force of 75 N. A constant frictional force of 20 N is present.
\begin{enumerate}
\item Draw a labelled force diagram to identify all the forces acting on the shopping trolley.
\item Draw a free body diagram of all the horizontal forces acting on the trolley.
\item Determine the resultant force on the trolley.
\end{enumerate}
}
\item {A donkey (mass 250 kg) is trying to pull a cart (mass 80 kg) with a force of 400 N. The rope between the donkey and the cart makes an angle of 30$\degree$ with the cart. The cart does not move.
\begin{enumerate}
\item Draw a free body diagram of all the forces acting on the donkey.
\item Draw a force diagram of all the forces acting on the cart.
\item Find the magnitude and direction of the frictional force preventing the cart from moving.
\end{enumerate}
}
\end{enumerate}
\practiceinfo

\begin{tabular}[h]{cccccc}
(1.) 01vw & (2.) 01vx & (3.) 01vy & (4.) 01vz & (5.) 01w0 & 
 \end{tabular}
}

\section{Newton's Laws}
In grade 10 you learned about motion, but did not look at how things start to move. You have also learned about forces. In this section we will look at the effect of forces on objects and how we can make things move.

\subsection{Newton's First Law}
Sir Isaac Newton was a scientist who lived in England (1642-1727) who was interested in the motion of objects under various conditions. He suggested that a stationary object will remain stationary unless a force acts on it and that a moving object will continue moving unless a force slows it down, speeds it up or changes its direction of motion. From this he formulated what is known as Newton's First Law of Motion:

\Definition{Newton's First Law of Motion}{An object will remain in a state of rest or continue travelling at constant velocity, unless acted upon by an unbalanced (net) force. }

Let us consider the following situations:\\

An ice skater pushes herself away from the side of the ice rink and skates across the ice. She will continue to move in a straight line across the ice unless something stops her. Objects are also like that. If we kick a soccer ball across a soccer field, according to Newton's First Law, the soccer ball should keep on moving forever! However, in real life this does not happen. Is Newton's Law wrong? Not really. Newton's First Law applies to situations where there aren't any external forces present. This means that friction is not present. In the case of the ice skater, the friction between the skates and the ice is very little and she will continue moving for quite a distance. In the case of the soccer ball, air resistance (friction between the air and the ball) and friction between the grass and the ball is present and this will slow the ball down.

\scalebox{0.5} % Change this value to rescale the drawing.
{
\begin{pspicture}(0,-7.465)(11.31,7.465)
\definecolor{color190}{rgb}{0.09019607,0.04313725,0.04705882}
\psbezier[linewidth=0.01,linecolor=color190](0.4833267,2.5238152)(0.7387859,2.9576032)(0.2,2.28)(0.12,2.16)(0.04,2.04)(0.0,1.88)(0.0,1.8)(0.0,1.72)(0.27895933,0.8280989)(0.33005118,0.5914873)(0.38114303,0.35487574)(0.5855104,-0.07891217)(0.7387859,-0.82818216)(0.8920614,-1.5774522)(2.078942,-2.8869464)(2.0671737,-2.878816)(2.0554054,-2.8706853)(3.684431,-2.3762686)(3.7021127,-2.3661575)(3.719794,-2.3560462)(4.6728578,-0.98592323)(4.3663063,-1.1830995)(4.0597553,-1.3802758)(2.680276,-0.07891217)(2.0671737,0.0)(1.4540716,0.07882888)(1.4540716,0.35487574)(1.351888,0.9069694)(1.2497042,1.459063)(1.6962553,1.7939805)(1.42,2.3)(1.1437447,2.8060195)(1.28,2.68)(1.1475207,2.326639)(1.0150412,1.9732778)(0.2278675,2.0900273)(0.4833267,2.5238152)
\psbezier[linewidth=0.01,linecolor=color190](0.4833267,2.5632505)(0.62,2.68)(0.8809696,2.838168)(0.74,3.18)
\psbezier[linewidth=0.01,linecolor=color190](0.06,2.12)(0.0,2.8020444)(0.08692311,3.964742)(0.2720482,4.5990224)(0.4571733,5.2333026)(0.9291566,6.135911)(1.0751808,6.348711)(1.2212049,6.561511)(1.26798,6.7100544)(1.6106024,6.987111)(1.9532248,7.264168)(2.0,7.46)(1.6349398,6.774311)(1.2698795,6.088622)(1.8053012,6.8452444)(1.7566265,6.656089)(1.7079518,6.4669333)(1.5375904,6.254133)(1.3185542,6.1122665)(1.0995181,5.9704)(0.7101205,5.095556)(0.58843374,4.5990224)(0.466747,4.102489)(0.5840964,3.3058667)(0.52,2.66)
\psbezier[linewidth=0.01,linecolor=color190](1.54,2.02)(2.0905557,2.8174257)(2.6522222,3.349802)(2.8194444,3.728515)(2.9866667,4.107228)(3.345,5.0651484)(3.4405556,5.287921)(3.536111,5.510693)(3.56,5.4884157)(3.56,5.688911)(3.56,5.8894057)(3.4405556,6.089901)(3.3211112,6.3349504)(3.2016666,6.58)(3.3688889,5.688911)(3.2972221,5.6220794)(3.2255557,5.5552473)(2.9866667,6.000792)(3.0344443,5.733465)(3.0822222,5.466139)(3.2255557,5.466139)(3.1777778,5.1765347)(3.13,4.8869305)(2.6522222,4.1963367)(2.4372222,3.7730694)(2.2222223,3.349802)(1.2427778,2.5878217)(1.26,2.6)(1.2772223,2.6121783)(0.98,2.7037623)(1.2905556,3.1047525)
\psbezier[linewidth=0.01,linecolor=color190](1.3,3.06)(0.96,3.06)(0.8,3.12)(0.72,3.18)(0.64,3.24)(0.6,3.4)(0.72,3.84)(0.84,4.28)(1.06,4.54)(1.16,4.64)(1.26,4.74)(1.42,5.1)(2.02,5.16)(2.62,5.22)(2.88,5.06)(3.08,4.98)
\psbezier[linewidth=0.01,linecolor=color190](1.3,3.06)(1.78,3.34)(1.86,3.76)(2.26,3.54)
\psbezier[linewidth=0.01,linecolor=color190](1.78,3.82)(2.02,3.88)(1.98,3.56)(1.78,3.5)
\psbezier[linewidth=0.01,linecolor=color190](1.82,4.06)(1.85625,4.175)(2.32,4.3)(2.48,4.3)
\psbezier[linewidth=0.01,linecolor=color190](1.78,4.22)(1.94,4.34)(2.32,4.42)(2.48,4.48)
\psbezier[linewidth=0.01,linecolor=color190](1.2,4.66)(1.68,4.76)(1.9,4.48)(1.8,4.24)
\psbezier[linewidth=0.01,linecolor=color190](3.3,4.8)(3.62,4.54)(3.88,4.46)(4.78,4.96)(5.68,5.46)(3.94,4.04)(4.28,4.08)(4.62,4.12)(3.2,3.6)(3.46,3.6)(3.72,3.6)(2.98,3.24)(2.66,3.4)
\psbezier[linewidth=0.01,linecolor=color190](2.06,-2.88)(2.62,-3.92)(5.92,-6.12)(6.34,-6.34)(6.76,-6.56)(5.9,-7.02)(5.9,-7.18)(5.9,-7.34)(7.42,-7.04)(7.26,-6.82)(7.1,-6.6)(4.48,-4.28)(4.16,-3.78)(3.84,-3.28)(3.4,-3.38)(2.9,-2.62)
\psbezier[linewidth=0.01,linecolor=color190](3.86,-2.08)(4.22,-2.36)(4.9,-2.58)(5.22,-2.72)(5.54,-2.86)(9.578232,-3.0499182)(9.9,-3.08)(10.221768,-3.1100817)(10.94,-3.8)(11.08,-3.72)(11.22,-3.64)(10.96,-3.3)(10.82,-3.1)(10.68,-2.9)(10.46,-2.48)(10.3,-2.7)(10.14,-2.92)(10.34,-2.96)(9.74,-2.82)(9.14,-2.68)(5.94,-2.06)(5.58,-2.02)(5.22,-1.98)(4.76,-1.58)(4.36,-1.34)
\psline[linewidth=0.01cm,linecolor=color190](5.94,-7.46)(5.94,-7.46)
\psline[linewidth=0.02cm,linecolor=color190](5.88,-7.42)(8.02,-6.94)
\psline[linewidth=0.02cm,linecolor=color190](11.3,-3.66)(10.32,-2.06)
\psline[linewidth=0.02cm,linecolor=color190](11.08,-5.82)(11.08,-5.82)
\psline[linewidth=0.02cm,linecolor=color190](5.96,-7.22)(5.88,-7.42)
\psline[linewidth=0.02cm,linecolor=color190](7.1,-7.02)(7.16,-7.14)
\psline[linewidth=0.02cm,linecolor=color190](11.1,-3.64)(11.3,-3.66)
\psline[linewidth=0.02cm,linecolor=color190](10.52,-2.7)(10.66,-2.62)
\end{pspicture}
}
\scalebox{0.5} % Change this value to rescale the drawing.
{
\begin{pspicture}(0,-5.829227)(13.14,5.8306108)
\definecolor{color0}{rgb}{0.93333333,0.03529411,0.03529411}
\definecolor{color262}{rgb}{0.06666666,0.03529411,0.03529411}
\definecolor{color3818b}{rgb}{0.12549019,0.07058823,0.07058823}
%\psbezier[linewidth=0.04,linecolor=color0](7.14,5.3506107)(4.9,2.7106109)(4.26,-1.1093891)(13.12,-3.4693892)
%\psbezier[linewidth=0.04,linecolor=color0](5.94,5.450611)(1.76,1.1906109)(4.82,-2.589389)(11.14,-3.929389)
%\psbezier[linewidth=0.04,linecolor=color0](4.22,3.0106108)(0.0,-1.3493891)(10.32,-1.0493891)(6.18,-5.789389)
%\rput{-17.584528}(-0.9773556,1.9422451){\psellipse[linewidth=0.04,linecolor=color0,dimen=outer](5.79,4.130611)(0.73,1.2)}
\psbezier[linewidth=0.04,linecolor=color262](5.48,5.550611)(5.66,5.650611)(5.98,5.810611)(6.08,5.710611)(6.18,5.610611)(6.18,5.550611)(6.2,5.510611)(6.22,5.470611)(6.44,5.590611)(6.62,5.470611)(6.8,5.3506107)(6.7144594,5.0064254)(6.68,5.050611)(6.6455407,5.0947967)(7.24,4.8706107)(7.34,4.590611)(7.44,4.310611)(7.0870843,4.548773)(6.82,4.690611)(6.552916,4.8324494)(7.2906103,4.236197)(6.94,4.250611)(6.58939,4.2650247)(6.44,4.510611)(6.4,4.730611)(6.36,4.950611)(5.74,3.830611)(5.86,4.430611)(5.98,5.030611)(6.28,4.730611)(5.46,4.690611)(4.64,4.650611)(6.020804,5.046645)(5.92,5.010611)(5.819196,4.974577)(4.8104544,4.889989)(4.82,5.050611)(4.8295455,5.211233)(5.52,5.3506107)(5.66,5.430611)
\psbezier[linewidth=0.04,linecolor=color262](6.56,4.3706107)(6.56,3.7706108)(6.14,3.130611)(5.86,2.910611)(5.58,2.690611)(5.02,3.350611)(5.04,3.7706108)(5.06,4.190611)(5.2,4.470611)(5.3,4.690611)
\psbezier[linewidth=0.04,linecolor=color262](3.34,2.070611)(3.76,2.5306108)(4.32,3.150611)(4.36,3.150611)(4.4,3.150611)(4.56,2.690611)(4.84,2.670611)(5.12,2.650611)(5.26,2.9706109)(5.26,3.0106108)(5.26,3.050611)(6.0,2.690611)(6.56,2.430611)(7.12,2.170611)(7.22,2.550611)(6.52,1.7706109)(5.82,0.9906109)(6.34,1.6906109)(5.38,1.8906109)
\psbezier[linewidth=0.04,linecolor=color262](5.04,4.050611)(4.94,3.890611)(4.7,3.5306108)(4.36,3.170611)
\psbezier[linewidth=0.04,linecolor=color262](3.34,1.7106109)(3.22,2.2306108)(3.34,2.090611)(3.34,2.090611)(3.34,2.090611)(3.74,1.990611)(3.96,1.7106109)(4.18,1.4306109)(4.2,1.4106109)(4.2,1.2106109)(4.2,1.0106109)(3.96,0.9706109)(3.78,1.2706109)
\psbezier[linewidth=0.04,linecolor=color262](3.48,2.050611)(3.3,1.4706109)(3.02,0.9706109)(3.1,0.89061093)(3.18,0.8106109)(3.38,0.07061091)(4.64,-1.0893891)
\psbezier[linewidth=0.04,linecolor=color262](3.96,1.6906109)(3.72,1.370611)(3.66,0.8706109)(3.58,0.95061094)(3.5,1.0306109)(4.14,-0.20938909)(4.8,-0.9493891)
\psbezier[linewidth=0.04,linecolor=color262](5.72,1.7706109)(5.62,1.5706109)(6.94,0.19061092)(7.0,0.17061092)(7.06,0.15061091)(6.58,-0.18938908)(6.02,-0.22938909)(5.46,-0.2693891)(4.74,-0.10938909)(4.58,-0.66938907)
\psbezier[linewidth=0.04,linecolor=color262](3.88,1.130611)(3.84,1.0106109)(3.96,0.4106109)(4.02,0.09061091)
\psbezier[linewidth=0.04,linecolor=color262](4.7839284,-0.92938906)(5.395357,-1.009389)(5.51,-0.9093891)(5.624643,-0.9693891)(5.739286,-1.0293891)(5.3735714,-1.1093891)(5.18,-1.0893891)(4.9864287,-1.0693891)(6.03,-1.129389)(6.24,-1.129389)(6.45,-1.129389)(6.08,-1.2893891)(5.5,-1.2293891)(4.92,-1.1693891)(6.459643,-1.509389)(6.28,-1.5893891)(6.100357,-1.6693891)(5.7246428,-1.5293891)(5.36,-1.389389)(4.995357,-1.249389)(6.124643,-1.9093891)(5.74,-1.7893891)(5.355357,-1.6693891)(4.812857,-1.3093891)(4.62,-1.0693891)
\psbezier[linewidth=0.04,linecolor=color262](4.06,-0.5093891)(4.06,-0.6293891)(4.16,-0.8493891)(4.14,-0.86938906)(4.12,-0.8893891)(4.2,-0.86938906)(4.38,-0.8293891)
\psbezier[linewidth=0.04,linecolor=color262](6.98,2.2506108)(7.34,2.130611)(7.82,1.850611)(7.62,1.9506109)(7.42,2.050611)(8.48,2.390611)(9.5,3.430611)
\psbezier[linewidth=0.04,linecolor=color262](6.62,1.8306109)(7.2,1.630611)(7.7,1.3906109)(7.74,1.4306109)(7.78,1.4706109)(9.36,2.870611)(9.84,3.430611)
\psbezier[linewidth=0.04,linecolor=color262](9.5,3.430611)(8.96,4.610611)(9.44,4.110611)(9.44,3.9906108)(9.44,3.870611)(9.48,3.7106109)(9.58,3.670611)(9.68,3.630611)(9.58,4.430611)(9.76,4.730611)(9.94,5.030611)(9.64,3.410611)(9.76,3.790611)(9.88,4.170611)(10.16,4.8906107)(10.18,4.670611)(10.2,4.450611)(10.02,4.030611)(9.88,3.7506108)(9.74,3.4706109)(10.4,4.530611)(10.32,4.290611)(10.24,4.050611)(10.04,3.610611)(9.84,3.430611)
\psbezier[linewidth=0.04,linecolor=color262](6.54,-0.12938909)(7.02,-0.60938907)(7.78,-1.1093891)(8.14,-1.1693891)(8.5,-1.2293891)(7.64,-1.4693891)(7.38,-1.5293891)(7.12,-1.5893891)(6.5,-2.449389)(6.52,-2.2893891)(6.54,-2.129389)(6.24,-2.0093892)(5.64,-1.4893891)
\psbezier[linewidth=0.04,linecolor=color262](7.6,-1.4493891)(8.2,-1.5893891)(8.991248,-1.7386874)(9.26,-1.9093891)(9.528752,-2.0800908)(11.28,-3.109389)(12.28,-3.2893891)
\psbezier[linewidth=0.04,linecolor=color262](6.72,-2.109389)(7.78,-2.4893892)(8.88,-2.329389)(8.96,-2.429389)(9.04,-2.5293891)(9.38,-3.0293891)(12.14,-3.569389)
\psbezier[linewidth=0.04,linecolor=color262,fillstyle=solid,fillcolor=color3818b](5.78,-1.629389)(5.64,-1.4493891)(6.44,-2.129389)(6.4,-2.109389)(6.36,-2.089389)(5.9813805,-2.357349)(5.96,-2.369389)(5.938619,-2.3814292)(5.92,-1.8093891)(5.78,-1.629389)
\psbezier[linewidth=0.04,linecolor=color262](6.06,-2.2893891)(6.06,-3.089389)(6.9564505,-2.4663787)(7.14,-3.449389)(7.3235497,-4.4323993)(7.2471433,-5.809227)(6.26,-5.6493893)
\psbezier[linewidth=0.04,linecolor=color262](6.68,-2.189389)(6.68,-2.9893892)(6.8233037,-2.2186728)(7.38,-3.0493891)(7.936696,-3.8801053)(9.121574,-4.161758)(8.26,-4.6693892)
\end{pspicture}
}

%do a minipage split
% Khan Academy video on Newton's First Law of motion: SIYAVULA-VIDEO:http://cnx.org/content/m38960/latest/#newton-1
\mindsetvid{Khan on newton 1}{VPkhy}
\subsubsection{Newton's First Law in action}
We experience Newton's First Law in every day life. Let us look at the following examples:\\

{\bf{Rockets}}:\\

A spaceship is launched into space. The force of the exploding gases pushes the rocket through the air into space. Once it is in space, the engines are switched off and it will keep on moving at a constant velocity. If the astronauts want to change the direction of the spaceship they need to fire an engine. This will then apply a force on the rocket and it will change its direction.

%rocket
\begin{figure}[h]
\begin{center}
\scalebox{0.6} % Change this value to rescale the drawing.
{
\begin{pspicture}(0,-3.75)(7.004871,3.75)
\psline[linewidth=0.04cm](1.0264977,3.5304644)(1.6233093,2.582722)
\psline[linewidth=0.04cm](3.3689935,3.5638368)(1.0264977,3.5304644)
\psline[linewidth=0.04cm](3.1855485,0.13940285)(2.6313665,1.0194496)
\psline[linewidth=0.04cm](4.232239,2.1929948)(3.1855485,0.13940285)
\psline[linewidth=0.08cm,arrowsize=0.05291667cm 2.0,arrowlength=1.4,arrowinset=0.4]{->}(1.6378409,0.7956044)(0.6900982,0.19879293)
\psline[linewidth=0.08cm,arrowsize=0.05291667cm 2.0,arrowlength=1.4,arrowinset=0.4]{->}(1.3971126,1.1403506)(0.47695127,0.53727245)
\psline[linewidth=0.08cm,arrowsize=0.05291667cm 2.0,arrowlength=1.4,arrowinset=0.4]{->}(1.1670417,1.4681727)(0.25314695,0.89267594)
\psline[linewidth=0.08cm,arrowsize=0.05291667cm 2.0,arrowlength=1.4,arrowinset=0.4]{->}(0.97081864,1.8173095)(0.04,1.2311554)
\pspolygon[linewidth=0.04](4.8107257,3.5263188)(4.415806,3.703062)(3.9832861,3.7143176)(3.3520694,3.5531795)(0.75140506,2.1282077)(1.8340638,0.44646758)(4.2385054,2.220576)(4.6440415,2.665031)(4.820785,3.059951)
\pscircle[linewidth=0.04,dimen=outer](5.534871,-2.28){1.47}
\usefont{T1}{ptm}{m}{n}
\rput{0.9892365}(-0.03724805,-0.09629055){\rput(5.5483084,-2.23){\Large Earth}}
\end{pspicture}
}
\end{center}
\caption{Newton's First Law and rockets}
\end{figure}

{\bf{Seat belts}:}\\

We wear seat belts in cars. This is to protect us when the car is involved in an accident. If a car is travelling at 120 \kph, the passengers in the car is also travelling at 120 \kph. When the car suddenly stops a force is exerted on the car (making it slow down), but not on the passengers. The passengers will carry on moving forward at 120 \kph according to Newton I. If they are wearing seat belts, the seat belts will stop them by exerting a force on them and so prevent them from getting hurt.

\begin{center}
\scalebox{0.3} % Change this value to rescale the drawing.
{
\begin{pspicture}(0,-5.494752)(20.32,5.504752)
\definecolor{color1003b}{rgb}{0.047,0.0196,0.0196}
\definecolor{color322b}{rgb}{0.0901,0.043,0.0431}
\psbezier[linewidth=0.04](1.0558287,-4.128509)(1.0385357,-3.3749824)(5.470862,-3.4501786)(6.405431,-3.5556967)(7.34,-3.6612148)(6.9114156,-4.4102597)(6.3087516,-4.942506)(5.706087,-5.474752)(1.0731217,-4.882036)(1.0558287,-4.128509)
\psbezier[linewidth=0.04](1.6252369,-3.7279913)(0.4915808,-4.182889)(0.06,1.4399428)(0.3390591,2.2825954)(0.6181182,3.1252482)(0.9366128,3.0862517)(1.4516722,2.5301535)(1.9667317,1.9740555)(2.758893,-3.2730937)(1.6252369,-3.7279913)
\psbezier[linewidth=0.04](2.0426986,1.3452481)(2.0720036,1.8772011)(2.439214,2.285248)(3.0119586,2.085248)(3.584703,1.8852481)(5.1707644,0.40524808)(5.3029366,0.32524806)(5.435108,0.24524808)(7.66,1.8852481)(7.5498567,1.3652481)(7.4397135,0.8452481)(7.634057,0.21440855)(7.5058,0.20524807)(7.3775425,0.19608761)(5.5452514,-0.49475193)(5.1927934,-0.5747519)(4.840335,-0.6547519)(2.0133939,0.813295)(2.0426986,1.3452481)
\psbezier[linewidth=0.04,fillstyle=solid,fillcolor=color322b](8.472562,2.5386589)(8.151501,2.1382349)(8.020529,1.2670984)(7.96804,0.8941427)(7.915552,0.5211869)(7.7015057,-0.6861639)(7.9367714,-0.9481519)(8.172037,-1.2101399)(8.793623,2.9390829)(8.472562,2.5386589)
\psbezier[linewidth=0.04](2.02,1.2452481)(2.02,1.6252481)(2.2388248,-2.1947522)(2.52,-3.134752)(2.801175,-4.074752)(3.8274896,-2.9169118)(4.22,-2.254752)(4.61251,-1.592592)(4.34,-0.43475193)(4.36,-0.3547519)(4.38,-0.27475193)(2.02,0.8652481)(2.02,1.2452481)
\psbezier[linewidth=0.04](8.84,-3.0547519)(8.44,-2.3147519)(8.631,-1.854752)(8.0139,-1.7747519)(7.3968,-1.694752)(4.3098,-1.9147519)(4.36,-2.014752)(4.4102,-2.1147518)(3.4764,-3.3147519)(2.9682,-3.474752)(2.46,-3.634752)(5.2551,-3.514752)(6.0537,-3.3147519)(6.8523,-3.1147518)(7.542,-2.8347518)(7.542,-2.8547518)(7.542,-2.8747518)(7.8732,-3.274752)(8.04,-4.274752)
\psline[linewidth=0.06cm](1.72,3.705248)(4.9,5.3252482)
\rput{26.82587}(2.1476142,-0.7192726){\psarc[linewidth=0.06](2.581892,4.1432247){0.9600791}{0.0}{180.0}}
\psline[linewidth=0.06](1.7,3.725248)(1.72,3.005248)(1.9,3.6252482)(2.12,3.325248)(2.26,3.785248)(2.86,3.4052482)(2.84,4.185248)(3.18,3.965248)(3.3,4.345248)(3.8,4.2852483)(3.46,4.585248)
\psbezier[linewidth=0.06](1.82,3.345248)(2.28,2.4452481)(3.28,1.9652481)(3.48,2.1252482)(3.68,2.285248)(3.82,3.725248)(3.42,4.3052483)
\psbezier[linewidth=0.06,fillstyle=solid,fillcolor=color1003b](0.02,2.785248)(0.0,2.785248)(0.34210426,3.295159)(0.28,3.345248)(0.21789576,3.395337)(1.86,2.605248)(2.86,2.225248)(3.86,1.8452481)(4.3,1.2052481)(4.02,1.4652481)(3.74,1.7252481)(0.04,2.785248)(0.02,2.785248)
\psbezier[linewidth=0.06,fillstyle=solid,fillcolor=color1003b](2.42,-2.974752)(2.46,-2.8747518)(1.786792,-3.47513)(1.8,-3.654752)(1.8132079,-3.8343737)(2.3512,-3.3747518)(2.4256,-3.254752)(2.5,-3.134752)(2.38,-3.0747519)(2.42,-2.974752)
\psbezier[linewidth=0.04,fillstyle=solid](7.52,1.1252481)(7.52,1.4852481)(7.86,1.7052481)(7.98,1.7052481)(8.1,1.7052481)(8.4,1.645248)(8.06,1.545248)(7.72,1.4452481)(8.54,1.685248)(8.52,1.165248)(8.5,0.64524806)(8.26,0.82524806)(7.9,0.7852481)(7.54,0.7452481)(7.52,0.76524806)(7.52,1.1252481)
\psline[linewidth=0.06cm](8.46,1.6252481)(8.94,1.6252481)
\psline[linewidth=0.06cm](8.34,0.4252481)(8.9,-0.41475192)
\psbezier[linewidth=0.04](12.095829,-4.128509)(12.078536,-3.3749824)(16.510862,-3.4501786)(17.44543,-3.5556967)(18.38,-3.6612148)(17.951416,-4.4102597)(17.348751,-4.942506)(16.746088,-5.474752)(12.113122,-4.882036)(12.095829,-4.128509)
\psbezier[linewidth=0.04](12.665236,-3.7279913)(11.531581,-4.182889)(11.1,1.4399428)(11.379059,2.2825954)(11.658118,3.1252482)(11.976613,3.0862517)(12.4916725,2.5301535)(13.006732,1.9740555)(13.798893,-3.2730937)(12.665236,-3.7279913)
\psbezier[linewidth=0.04](15.64,1.8252481)(15.98,2.3852482)(16.58,2.1452482)(16.76,1.5652481)(16.94,0.9852481)(16.65497,0.09874813)(16.54,0.02524807)(16.42503,-0.04825198)(18.056879,1.4226693)(18.28,1.6252481)(18.50312,1.8278269)(18.5,0.70524806)(18.58,0.4252481)(18.66,0.14524807)(15.958311,-1.3236657)(15.82,-0.97475195)(15.681689,-0.6258381)(15.3,1.2652481)(15.64,1.8252481)
\psbezier[linewidth=0.04,fillstyle=solid,fillcolor=color322b](19.512562,2.5386589)(19.191502,2.1382349)(19.060528,1.2670984)(19.00804,0.8941427)(18.955551,0.5211869)(18.741507,-0.6861639)(18.976772,-0.9481519)(19.212038,-1.2101399)(19.833624,2.9390829)(19.512562,2.5386589)
\psbezier[linewidth=0.04](15.5,1.5052481)(15.62,1.8452481)(12.398825,-3.2947521)(13.88,-3.474752)(15.361175,-3.6547518)(16.98,-0.55475193)(16.78,-0.73475194)(16.58,-0.91475195)(15.86,-1.094752)(15.82,-1.074752)(15.78,-1.0547519)(15.38,1.165248)(15.5,1.5052481)
\psbezier[linewidth=0.04](20.3,-3.014752)(19.9,-2.274752)(20.091,-1.814752)(19.4739,-1.7347519)(18.8568,-1.6547519)(16.1098,-1.694752)(15.98,-2.014752)(15.8502,-2.3347518)(14.9364,-3.234752)(14.42,-3.414752)(13.9036,-3.5947518)(16.7151,-3.474752)(17.5137,-3.274752)(18.3123,-3.0747519)(19.002,-2.794752)(19.002,-2.8147519)(19.002,-2.8347518)(19.3332,-3.234752)(19.5,-4.2347517)
\psline[linewidth=0.06cm](15.529593,2.927153)(18.028902,5.474752)
\rput{45.378242}(7.3970747,-10.459514){\psarc[linewidth=0.06](16.207344,3.616599){0.9600791}{0.0}{180.0}}
\psline[linewidth=0.06](15.50427,2.9397504)(15.752314,2.26353)(15.725694,2.9085815)(16.029713,2.694169)(16.016077,3.1748085)(16.705803,3.0054586)(16.43867,3.7385612)(16.830997,3.638172)(16.823856,4.0366054)(17.316963,4.1388087)(16.899181,4.3150406)
\psbezier[linewidth=0.06](15.738938,2.6176784)(16.461388,1.9108073)(17.562143,1.7739227)(17.700844,1.9892423)(17.839542,2.204562)(17.5141,3.6142738)(16.950348,4.0368643)
\psbezier[linewidth=0.06,fillstyle=solid,fillcolor=color1003b](11.06,2.785248)(11.0,2.825248)(11.182104,3.2551591)(11.1,3.345248)(11.017896,3.435337)(14.86,2.505248)(15.78,2.305248)(16.7,2.105248)(17.0,1.4452481)(16.68,1.785248)(16.36,2.1252482)(11.12,2.745248)(11.06,2.785248)
\psbezier[linewidth=0.06,fillstyle=solid,fillcolor=color1003b](13.46,-2.974752)(13.5,-2.8747518)(12.826792,-3.47513)(12.84,-3.654752)(12.853208,-3.8343737)(13.3912,-3.3747518)(13.4656,-3.254752)(13.54,-3.134752)(13.42,-3.0747519)(13.46,-2.974752)
\psbezier[linewidth=0.04,fillstyle=solid](18.56,1.1252481)(18.56,1.4852481)(18.9,1.7052481)(19.02,1.7052481)(19.14,1.7052481)(19.44,1.645248)(19.1,1.545248)(18.76,1.4452481)(19.58,1.685248)(19.56,1.165248)(19.54,0.64524806)(19.3,0.82524806)(18.94,0.7852481)(18.58,0.7452481)(18.56,0.76524806)(18.56,1.1252481)
\psline[linewidth=0.06cm](19.5,1.6252481)(19.98,1.6252481)
\psline[linewidth=0.06cm](19.38,0.4252481)(19.94,-0.41475192)
\psbezier[linewidth=0.04](16.74,1.645248)(16.92,1.2652481)(16.88,0.88524806)(16.92,0.34524807)
\psbezier[linewidth=0.06](16.6,1.8852481)(17.08,1.5052481)(16.9,0.34524807)(16.92,1.0852481)
\end{pspicture}
}
\end{center}

\begin{wex}{Newton's First Law in action}{Why do passengers get thrown to the side when the car they are driving in goes around a corner?}
{\westep{What happens before the car turns}
Before the car starts turning both the passengers and the car are travelling at the same velocity. (picture A)

\westep{What happens while the car turns}
The driver turns the wheels of the car, which then exert a force on the car and the car turns. This force acts on the car but not the passengers, hence (by Newton's First Law) the passengers continue moving with the same original velocity. (picture B)

\westep{Why passengers get thrown to the side?}
If the passengers are wearing seat belts they will exert a force on the passengers until the passengers' velocity is the same as that of the car (picture C). Without a seat belt the passenger may hit the side of the car.

\begin{center}
\begin{pspicture}(-6,-3)(6,3)
%before the turn
%car
\psline(-4.5,-1)(-4.5,1)(-3.5,1)(-3.5, -1)(-4.5, -1)
\psline[linewidth=1pt,arrowscale=2]{->}(-4,1)(-4,1.7)

%wheels
\psline[linewidth=0.2cm](-4.5,0.4)(-4.5, 0.6)
\psline[linewidth=0.2cm](-4.5,-0.4)(-4.5, -0.6)
\psline[linewidth=0.2cm](-3.5,0.4)(-3.5, 0.6)
\psline[linewidth=0.2cm](-3.5,-0.4)(-3.5, -0.6)

%person
\psdot[dotsize=0.2](-4,0)
\psline[linewidth=1pt,arrowscale=2]{->}(-4,0)(-4,0.7)

\rput(-4,-2){\parbox{3.5cm}{A: Both the car and the person travelling
at the same velocity}}
\psline[linestyle=dashed](-2,-3)(-2,3)

%the car turns but not the person

\rput{45}(3,3){
\psline(-4.5,-1)(-4.5,1)(-3.5,1)(-3.5, -1)(-4.5, -1)
\psline[linewidth=1pt,arrowscale=2]{->}(-4,1)(-4,1.7)
\psdot[dotsize=0.2](-4,0)

%wheels
\psline[linewidth=0.2cm](-4.5,0.4)(-4.5, 0.6)
\psline[linewidth=0.2cm](-4.5,-0.4)(-4.5, -0.6)
\psline[linewidth=0.2cm](-3.5,0.4)(-3.5, 0.6)
\psline[linewidth=0.2cm](-3.5,-0.4)(-3.5, -0.6)
}

\psline[linewidth=1pt,arrowscale=2]{->}(0.18,0.2)(0.18,0.8)

\rput(0,-2){\parbox{3.5cm}{B: The cars turns but not the person}}

\psline[linestyle=dashed](2,-3)(2,3)

%after the turn
\rput{45}(7,3){
\psline(-4.5,-1)(-4.5,1)(-3.5,1)(-3.5, -1)(-4.5, -1)
\psline[linewidth=1pt,arrowscale=2]{->}(-4,1)(-4,1.7)

%wheels
\psline[linewidth=0.2cm](-4.5,0.4)(-4.5, 0.6)
\psline[linewidth=0.2cm](-4.5,-0.4)(-4.5, -0.6)
\psline[linewidth=0.2cm](-3.5,0.4)(-3.5, 0.6)
\psline[linewidth=0.2cm](-3.5,-0.4)(-3.5, -0.6)

\psdot[dotsize=0.2](-4,0)

\psline[linewidth=1pt,arrowscale=2]{->}(-4,0)(-4,0.7)
}
\rput(4,-2){\parbox{3.5cm}{C: Both the car and
the person are travelling
at the same velocity again}}
\end{pspicture}
\end{center}}
\end{wex}

\subsection{Newton's Second Law of Motion}
According to Newton I, things 'like to keep on doing what they are doing'. In other words, if an object is moving, it tends to continue moving (in a straight line and at the same speed) and if an object is stationary, it tends to remain stationary. So how do objects start moving?

Let us look at the example of a 10 kg box on a rough table. If we push lightly on the box as indicated in the diagram, the box won't move. Let's say we applied a force of 100 N, yet the box remains stationary. At this point a frictional force of 100 N is acting on the box, preventing the box from moving. If we increase the force, let's say to 150 N and the box almost starts to move, the frictional force is 150 N. To be able to move the box, we need to push hard enough to overcome the friction and then move the box. If we therefore apply a force of 200 N remembering that a frictional force of 150 N is present, the 'first' 150 N will be used to overcome or 'cancel' the friction and the other 50 N will be used to move (accelerate) the block. In order to accelerate an object we must have a resultant force acting on the block.

\begin{center}
\begin{pspicture}(-3,-1)(4,1)
%\psgrid[gridcolor=lightgray]
\psline[linewidth=2pt](-3,-1)(-3,0)(3,0)(3, -1)
\multido{\n=-2.8+0.1}{59}
{\rput(\n,-0.2){\psline(-0.2,0)(0,0.2)}}
\psframe[linewidth=1pt](-0.5,0)(0.5,1)
\psline[linewidth=2pt]{<-}(0.5,0.5)(1.5,0.5)
\rput(0,0.5){box}
\uput[ur](-3,0){rough table}
\uput[r](1.5,0.5){applied force}
\end{pspicture}
\end{center}

Now, what do you think will happen if we pushed harder, lets say 300 N? Or, what do you think will happen if the mass of the block was more, say 20 kg, or what if it was less? Let us investigate how the motion of an object is affected by mass and force.

\Activity{Investigation}{Newton's Second Law of Motion}{
\Aim{To investigate the relation between the acceleration of objects and the application of a constant resultant force.}
\Method{
\begin{figure}[H]
\begin{center}
\scalebox{1} % Change this value to rescale the drawing.
{
\begin{pspicture}(0,-1.3)(4.42,1.32)
\psframe[linewidth=0.04,dimen=outer](2.0,-0.1)(0.0,-1.1)
\pscircle[linewidth=0.04,dimen=outer](0.5,-1.1){0.2}
\pscircle[linewidth=0.04,dimen=outer](1.5,-1.1){0.2}
\psline[linewidth=0.04cm,arrowsize=0.05291667cm 2.0,arrowlength=1.4,arrowinset=0.4]{->}(2.0,-0.1)(3.9,1.3)
\psline[linewidth=0.04cm,linestyle=dashed,dash=0.16cm 0.16cm](2.0,-0.1)(4.4,-0.1)
\usefont{T1}{ptm}{m}{n}
\rput(3.3635938,0.11){60$\degree$}
\end{pspicture}
}
\end{center}
\end{figure}

\begin{enumerate}
\item A constant force of 20 N, acting at an angle of 60$\degree$ to the horizontal, is applied to a dynamics trolley.
\item Ticker tape attached to the trolley runs through a ticker timer of frequency 20 Hz as the trolley is moving on the frictionless surface.
\item The above procedure is repeated 4 times, each time using the same force, but varying the mass of the trolley as follows:
\begin{itemize}
\item Case 1: 6,25~kg
\item Case 2: 3,57~kg
\item Case 3: 2,27~kg
\item Case 4: 1,67~kg
\end{itemize}
\item Shown below are sections of the four ticker tapes obtained. The tapes are marked with the letters A, B, C, D, etc. A is the first dot, B is the second dot and so on. The distance between each dot is also shown.

\end{enumerate}

\begin{figure}[H]
\begin{center}
\scalebox{1} % Change this value to rescale the drawing.
{
\begin{pspicture}(0,-4.2501564)(12.317187,4.2501564)
\psline[linewidth=0.04cm](0.0971875,3.8667188)(11.097187,3.8667188)
\psline[linewidth=0.04cm](0.0971875,2.8667188)(11.197187,2.8667188)
\psdots[dotsize=0.12](0.5971875,3.3667188)
\psdots[dotsize=0.12](1.0971875,3.3667188)
\psdots[dotsize=0.12](1.9971875,3.3667188)
\psdots[dotsize=0.12](3.2971876,3.3667188)
\psdots[dotsize=0.12](4.9971876,3.3667188)
\psdots[dotsize=0.12](7.0971875,3.3667188)
\psdots[dotsize=0.12](9.597187,3.3667188)
\rput(0.6796875,4.076719){Tape 1}
\rput(0.80328125,3.1617188){\footnotesize 5mm}
\rput(1.6045313,3.1617188){\footnotesize 9mm}
\rput(2.6721876,3.1617188){\footnotesize 13mm}
\rput(4.1721873,3.1617188){\footnotesize 17mm}
\rput(5.9875,3.1617188){\footnotesize 21mm}
\rput(8.4875,3.1617188){\footnotesize 25mm}
\psline[linewidth=0.04cm](0.0971875,1.8667188)(12.097187,1.8667188)
\psline[linewidth=0.04cm](0.0971875,0.86671877)(12.097187,0.86671877)
\psdots[dotsize=0.12](0.5971875,1.3667188)
\psdots[dotsize=0.12](0.8971875,1.3667188)
\psdots[dotsize=0.12](1.7971874,1.3667188)
\psdots[dotsize=0.12](3.4971876,1.3667188)
\psdots[dotsize=0.12](5.6971874,1.3667188)
\psdots[dotsize=0.12](8.797188,1.3667188)
\psdots[dotsize=0.12](11.897187,1.3667188)
\rput(0.69421875,2.0767188){Tape 2}
\rput(0.604375,1.1617187){\footnotesize 3mm}
\rput(1.4721875,1.1617187){\footnotesize 10mm}
\rput(2.6721876,1.1617187){\footnotesize 17mm}
\rput(4.5875,1.1617187){\footnotesize 24mm}
\rput(7.084375,1.1617187){\footnotesize 31mm}
\rput(10.284375,1.1617187){\footnotesize 38mm}
\psline[linewidth=0.04cm](0.0971875,-0.13328125)(12.097187,-0.13328125)
\psline[linewidth=0.04cm](0.0971875,-1.1332812)(12.097187,-1.1332812)
\psdots[dotsize=0.12](0.1971875,-0.63328123)
\psdots[dotsize=0.12](0.3971875,-0.63328123)
\psdots[dotsize=0.12](1.4971875,-0.63328123)
\psdots[dotsize=0.12](3.2971876,-0.63328123)
\psdots[dotsize=0.12](5.4971876,-0.63328123)
\psdots[dotsize=0.12](8.497188,-0.63328123)
\psdots[dotsize=0.12](11.997188,-0.63328123)
\rput(0.68671876,0.07671875){Tape 3}
\rput(0.3075,-0.8382813){\footnotesize 2mm}
\rput(1.0721875,-0.8382813){\footnotesize 13mm}
\rput(2.3875,-0.8382813){\footnotesize 24mm}
\rput(4.684375,-0.8382813){\footnotesize 35mm}
\rput(7.3884373,-0.8382813){\footnotesize 46mm}
\rput(10.683281,-0.8382813){\footnotesize 57mm}
\psline[linewidth=0.04cm](0.0971875,-2.1332812)(11.097187,-2.1332812)
\psline[linewidth=0.04cm](0.0971875,-3.1332812)(11.197187,-3.1332812)
\psdots[dotsize=0.12](0.5971875,-2.6332812)
\psdots[dotsize=0.12](1.0971875,-2.6332812)
\psdots[dotsize=0.12](1.9971875,-2.6332812)
\psdots[dotsize=0.12](3.2971876,-2.6332812)
\psdots[dotsize=0.12](4.9971876,-2.6332812)
\psdots[dotsize=0.12](7.0971875,-2.6332812)
\psdots[dotsize=0.12](9.597187,-2.6332812)
\rput(0.6954687,-1.9232812){Tape 4}
\rput(0.8045313,-2.8382812){\footnotesize 9mm}
\rput(1.6875,-2.8382812){\footnotesize 24mm}
\rput(2.684375,-2.8382812){\footnotesize 39mm}
\rput(4.1832814,-2.8382812){\footnotesize 54mm}
\rput(5.9840627,-2.8382812){\footnotesize 69mm}
\rput(8.48125,-2.8382812){\footnotesize 84mm}
\rput(2.2665625,-4.023281){Tapes are not drawn to scale.}
\rput(0.62375,3.6767187){A}
\rput(1.104375,3.6767187){B}
\rput(2.004375,3.6767187){C}
\rput(3.3171875,3.6767187){D}
\rput(4.999219,3.6767187){E}
\rput(7.086875,3.6767187){F}
\rput(9.518281,3.6767187){G}
\rput(0.62375,-2.3232813){A}
\rput(1.104375,-2.3232813){B}
\rput(2.004375,-2.3232813){C}
\rput(3.3171875,-2.3232813){D}
\rput(4.999219,-2.3232813){E}
\rput(7.086875,-2.3232813){F}
\rput(9.518281,-2.3232813){G}
\rput(0.12375,-0.32328126){A}
\rput(0.404375,-0.32328126){B}
\rput(1.504375,-0.32328126){C}
\rput(3.3171875,-0.32328126){D}
\rput(5.3992186,-0.32328126){E}
\rput(8.486875,-0.32328126){F}
\rput(11.918282,-0.32328126){G}
\psline[linewidth=0.04cm](0.2971875,1.8667188)(12.297188,1.8667188)
\rput(0.52375,1.6767187){A}
\rput(0.904375,1.6767187){B}
\rput(1.704375,1.6767187){C}
\rput(3.5171876,1.6767187){D}
\rput(5.599219,1.6767187){E}
\rput(8.786875,1.6767187){F}
\rput(11.818281,1.6767187){G}
\end{pspicture}
}
\end{center}
\end{figure}
}
Instructions:\\
\begin{enumerate}
\item Use each tape to calculate the instantaneous velocity (in \ms) of the trolley at points B and F (remember to convert the distances to m first!). Use these velocities to calculate the trolley's acceleration in each case.
%\item Use Newton's second law to calculate the mass of the trolley in each case.
\item Tabulate the mass and corresponding acceleration values as calculated in each case. Ensure that each column and row in your table is appropriately labelled.
\item Draw a graph of acceleration vs.\@ mass, using a scale of 1 cm = 1 \mss on the y-axis and 1 cm = 1 kg on the x-axis.
\item Use your graph to read off the acceleration of the trolley if its mass is 5 kg.
\item Write down a conclusion for the experiment.
\end{enumerate}
}

You will have noted in the investigation above that the heavier the trolley is, the slower it moved. The acceleration is \textit{inversely} proportional to the mass. In mathematical terms:
\begin{equation*}
a \propto \frac{1}{m}
\end{equation*}

In a similar investigation where the mass is kept constant, but the applied force is varied, you will find that the bigger the force is, the faster the object will move. The acceleration of the trolley is therefore \textit{directly} proportional to the resultant force. In mathematical terms:
\begin{equation*}
a \propto F.
\end{equation*}

Rearranging the above equations, we get $a$ $\propto$ $\frac{F}{m}$ OR $F~=~ma$\\

Newton formulated his second law as follows:\\

\Definition{Newton's Second Law of Motion}{
If a resultant force acts on a body, it will cause the body to accelerate in the direction of the resultant force. The acceleration of the body will be directly proportional to the resultant force and inversely proportional to the mass of the body. The mathematical representation is:
\begin{equation*}
F = ma.
\end{equation*}
}
% Khan Academy video on Newton's Second Law of motion: SIYAVULA-VIDEO:http://cnx.org/content/m38963/latest/#newton-2
\mindsetvid{Khan on newton 2}{VPkiq}
\subsubsection{Applying Newton's Second Law}
Newton's Second Law can be applied to a variety of situations. We will look at the main types of examples that you need to study.

%\subsubsection{Box on surface I}

\begin{wex}{Newton II - Box on a surface 1}{
A 10 kg box is placed on a table. A horizontal force of 32 N is applied to the box. A frictional force of 7 N is present between the surface and the box.
\begin{enumerate}
\item Draw a force diagram indicating all the horizontal forces acting on the box.
\item Calculate the acceleration of the box.
\end{enumerate}
%one box on table
\begin{figure}[H]
\begin{center}
\scalebox{1} % Change this value to rescale the drawing.
{
\begin{pspicture}(0,-1.62)(6.24625,1.62)
\psframe[linewidth=0.04,dimen=outer](4.42625,1.62)(2.38625,0.18)
\psframe[linewidth=0.04,dimen=outer,fillstyle=solid](5.88625,0.18)(0.88625,-0.06)
\psframe[linewidth=0.04,dimen=outer,fillstyle=solid](5.52625,-0.02)(5.12625,-1.62)
\psframe[linewidth=0.04,dimen=outer,fillstyle=solid](1.62625,-0.02)(1.22625,-1.6)
\usefont{T1}{ptm}{m}{n}
\rput(3.43375,0.92){10 kg}
\psline[linewidth=0.08cm,arrowsize=0.05291667cm 2.0,arrowlength=1.4,arrowinset=0.4]{->}(6.20625,1.06)(4.40625,1.06)
\usefont{T1}{ptm}{m}{n}
\rput(5.4265623,1.32){32 N}
\usefont{T1}{ptm}{m}{n}
\rput(0.956875,1.31){friction = 7 N}
\end{pspicture}
}
\end{center}
\end{figure}
}{\westep{Identify the horizontal forces and draw a force diagram}
We only look at the forces acting in a horizontal direction (left-right) and not vertical (up-down) forces. The applied force and the force of friction will be included. The force of gravity, which is a vertical force, will not be included.

%force diagram
\begin{figure}[H]
\begin{center}
\scalebox{1} % Change this value to rescale the drawing.
{
\begin{pspicture}(0,-1.335)(9.885,1.375)
\psframe[linewidth=0.04,dimen=outer,fillstyle=solid](3.44,-0.095)(1.36,-1.335)
\psline[linewidth=0.04cm,arrowsize=0.05291667cm 2.0,arrowlength=1.4,arrowinset=0.4]{->}(1.38,-0.495)(0.0,-0.515)
\psline[linewidth=0.04cm,arrowsize=0.05291667cm 2.0,arrowlength=1.4,arrowinset=0.4]{->}(3.4,-1.035)(4.42,-1.055)
\psline[linewidth=0.04cm,arrowsize=0.05291667cm 2.0,arrowlength=1.4,arrowinset=0.4]{->}(3.0,0.845)(1.08,0.825)
\usefont{T1}{ptm}{m}{n}
\rput(2.7632813,1.195){direction of motion}
\usefont{T1}{ptm}{m}{n}
\rput(1.76625,0.575){a = ?}
\usefont{T1}{ptm}{m}{n}
\rput(0.645,-0.245){F$_{1}$}
\usefont{T1}{ptm}{m}{n}
\rput(3.8553126,-0.805){F$_{f}$}
\usefont{T1}{ptm}{m}{n}
\rput(7.3,0.415){F$_{1}$ = applied force on box (32 N)}
\usefont{T1}{ptm}{m}{n}
\rput(6.501875,-0.105){F$_{f}$ = Frictional force (7 N)}
\end{pspicture}
}
\end{center}
\end{figure}

\westep{Calculate the acceleration of the box}
We have been given:\\
Applied force F$_{1}$ = 32 N\\
Frictional force F$_{f}$ = - 7 N\\
Mass m = 10 kg\\

To calculate the acceleration of the box we will be using the equation $F_{R} = ma$.
Therefore:
\begin{eqnarray*}
F_{R} &=& ma\\
F_{1} + F_{f} &=& (10)(a)\\
32 - 7 &=& 10~a\\
25 &=& 10~a\\
a &=& 2,5 \ems \mbox{towards the left}\\
\end{eqnarray*}
}
\end{wex}

\begin{wex}{Newton II - box on surface 2}{Two crates, 10 kg and 15 kg respectively, are connected with a thick rope according to the diagram. A force of 500 N is applied. The boxes move with an acceleration of 2 \mss. One third of the total frictional force is acting on the 10 kg block and two thirds on the 15 kg block. Calculate:
\begin{enumerate}
\item the magnitude and direction of the frictional force present.
\item the magnitude of the tension in the rope at T.
\end{enumerate}
%two boxes on surface
\begin{figure}[H]
\begin{center}
\scalebox{1} % Change this value to rescale the drawing.
{
\begin{pspicture}(0,-1.425)(7.0,1.465)
\definecolor{color1b}{rgb}{0.8,0.8,0.8}
\psframe[linewidth=0.04,dimen=outer,fillstyle=solid,fillcolor=color1b](7.0,-1.125)(0.0,-1.425)
\psframe[linewidth=0.04,dimen=outer](2.0,-0.125)(1.0,-1.125)
\psframe[linewidth=0.04,dimen=outer](4.5,0.275)(3.0,-1.125)
\psline[linewidth=0.1cm](2.0,-0.625)(3.0,-0.625)
\psline[linewidth=0.1cm](3.0,-0.625)(3.0,-0.625)
\psline[linewidth=0.06cm,arrowsize=0.05291667cm 2.0,arrowlength=1.4,arrowinset=0.4]{->}(4.5,-0.125)(6.3,-0.125)
\psline[linewidth=0.04cm,arrowsize=0.05291667cm 2.0,arrowlength=1.4,arrowinset=0.4]{->}(2.0,1.075)(3.9,1.075)
\usefont{T1}{ptm}{m}{n}
\rput(3.785625,-0.415){15 kg}
\usefont{T1}{ptm}{m}{n}
\rput(2.9373438,1.285){a = 2 \mss}
\usefont{T1}{ptm}{m}{n}
\rput(5.438906,0.085){500 N}
\usefont{T1}{ptm}{m}{n}
\rput(1.485625,-0.615){10 kg}
\usefont{T1}{ptm}{m}{n}
\rput(2.5332813,-0.385){\Large T}
\end{pspicture}
}
\end{center}
% \caption{Two crates on a surface}
\end{figure}
}{\westep{Draw a force diagram}
Always draw a force diagram although the question might not ask for it. The acceleration of the whole system is given, therefore a force diagram of the whole system will be drawn. Because the two crates are seen as a unit, the force diagram will look like this:

%force diagram two boxes
\begin{figure}[H]
\begin{center}
\scalebox{1} % Change this value to rescale the drawing.
{
\begin{pspicture}(0,-1.425)(8.769062,1.465)
\definecolor{color1b}{rgb}{0.8,0.8,0.8}
\psframe[linewidth=0.04,dimen=outer,fillstyle=solid,fillcolor=color1b](7.7940626,-1.125)(0.7940625,-1.425)
\psframe[linewidth=0.04,dimen=outer](2.7940626,-0.125)(1.7940625,-1.125)
\psframe[linewidth=0.04,dimen=outer](5.2940626,0.275)(3.7940626,-1.125)
\psline[linewidth=0.1cm](2.7940626,-0.625)(3.7940626,-0.625)
\psline[linewidth=0.1cm](3.7940626,-0.625)(3.7940626,-0.625)
\psline[linewidth=0.06cm,arrowsize=0.05291667cm 2.0,arrowlength=1.4,arrowinset=0.4]{->}(5.2940626,-0.125)(7.0940623,-0.125)
\psline[linewidth=0.04cm,arrowsize=0.05291667cm 2.0,arrowlength=1.4,arrowinset=0.4]{->}(2.7940626,1.075)(4.6940627,1.075)
\usefont{T1}{ptm}{m}{n}
\rput(4.5796876,-0.415){15 kg}
\usefont{T1}{ptm}{m}{n}
\rput(3.7314062,1.285){a = 2 \mss}
\usefont{T1}{ptm}{m}{n}
\rput(7.110156,0.085){Applied force = 500 N}
\usefont{T1}{ptm}{m}{n}
\rput(2.2796874,-0.615){10 kg}
\psframe[linewidth=0.04,linestyle=dashed,dash=0.16cm 0.16cm,dimen=outer](5.3940625,0.775)(1.6940625,-1.125)
\psline[linewidth=0.06cm,arrowsize=0.05291667cm 2.0,arrowlength=1.4,arrowinset=0.4]{->}(1.7940625,-0.425)(0.5940625,-0.425)
\usefont{T1}{ptm}{m}{n}
\rput(0.81375,-0.215){Friction = ?}
\end{pspicture}
}
\end{center}
% \caption{Force diagram for two crates on a surface}
\end{figure}

\westep{Calculate the frictional force}
To find the frictional force we will apply Newton's Second Law. We are given the mass (10 + 15 kg) and the acceleration (2 \mss). Choose the direction of motion to be the positive direction (to the right is positive).
\begin{eqnarray*}
F_{R} = ma\\
F_{\rm{applied}}+ F_{f} = ma\\
500 + F_{f} = (10 + 15) (2)\\
F_{f} = 50 - 500\\
F_{f} = - 450 N\\
\end{eqnarray*}
The frictional force is 450 N opposite to the direction of motion (to the left).\\

\westep{Find the tension in the rope}
To find the tension in the rope we need to look at one of the two crates on their own. Let's choose the 10 kg crate. Firstly, we need to draw a force diagram:

%force diagram of 10 kg crate
\begin{figure}[H]
\begin{center}
\scalebox{1} % Change this value to rescale the drawing.
{
\begin{pspicture}(0,-1.275)(8.319375,1.315)
\psframe[linewidth=0.04,dimen=outer](6.299375,0.225)(4.599375,-1.275)
\usefont{T1}{ptm}{m}{n}
\rput(5.406875,-0.535){\Large 10 kg}
\psline[linewidth=0.04cm,arrowsize=0.05291667cm 2.0,arrowlength=1.4,arrowinset=0.4]{->}(4.599375,-0.375)(2.399375,-0.375)
\psline[linewidth=0.04cm,arrowsize=0.05291667cm 2.0,arrowlength=1.4,arrowinset=0.4]{->}(6.299375,-0.375)(8.299375,-0.375)
\usefont{T1}{ptm}{m}{n}
\rput(7.329375,-0.165){Tension T}
\usefont{T1}{ptm}{m}{n}
\rput(2.5239062,-0.165){$\frac{1}{3}$ of total frictional force}
\psline[linewidth=0.04cm,arrowsize=0.05291667cm 2.0,arrowlength=1.4,arrowinset=0.4]{->}(4.799375,0.825)(6.499375,0.825)
\usefont{T1}{ptm}{m}{n}
\rput(5.936719,1.135){a = 2 \mss}
\usefont{T1}{ptm}{m}{n}
\rput(3.10125,-0.665){$F_{f}$ on 10 kg crate}
\end{pspicture}
}
\end{center}
\caption{Force diagram of 10 kg crate}
\end{figure}

The frictional force on the 10 kg block is one third of the total, therefore:\\
$F_{f} = \frac{1}{3} \times$ 450\\
$F_{f} = 150\ \rm{N}$\\

If we apply Newton's Second Law:\\
\begin{eqnarray*}
F_{R} &=& ma\\
T + F_{f} &=& (10)(2)\\
T + (-150) &=& 20\\
T &=& 170\ \rm{N}
\end{eqnarray*}

Note: If we had used the same principle and applied it to 15 kg crate, our calculations would have been the following:\\
\begin{eqnarray*}
F_{R} &=& ma\\
F_{\rm{applied}}+T + F_{f} &=& (15)(2)\\
500 + T + (-300) &=& 30\\
T &=& -170\ \rm{N}
\end{eqnarray*}
The negative answer here means that the force is in the direction opposite to the motion, in other words to the left, which is correct. However, the question asks for the magnitude of the force and your answer will be quoted as 170 N.
}
\end{wex}

%\subsubsection{Box on surface II}

\begin{wex}{Newton II - Man pulling a box}{A man is pulling a 20 kg box with a rope that makes an angle of 60$\degree$ with the horizontal. If he applies a force of 150 N and a frictional force of 15 N is present, calculate the acceleration of the box.
%man pulling box at an angle
\begin{figure}[H]
\begin{center}
\scalebox{.8} % Change this value to rescale the drawing.
{
\begin{pspicture}(0,-1.5613594)(6.03,1.5005769)
\psframe[linewidth=0.04,dimen=outer](3.0,-0.51135945)(1.0,-1.5113595)
\psline[linewidth=0.04cm](4.5,0.78864056)(4.1,-0.31135944)
\psline[linewidth=0.04cm](4.1,-0.31135944)(4.4,-1.5113595)
\psline[linewidth=0.04cm](4.4,-1.5113595)(4.6,-1.4113594)
\psline[linewidth=0.04cm](4.1,-0.31135944)(4.1,-1.1113595)
\psline[linewidth=0.04cm](4.1,-1.1113595)(3.8,-1.5113595)
\psline[linewidth=0.04cm](3.8,-1.4713595)(4.0,-1.4713595)
\psline[linewidth=0.04cm](4.4,0.48864058)(4.0,0.08864056)
\psline[linewidth=0.04cm](4.0,0.08864056)(4.6,0.18864056)
\psline[linewidth=0.04cm](4.4,0.5886406)(4.4,-0.11135944)
\psline[linewidth=0.04cm](4.4,-0.11135944)(4.7,0.08864056)
\rput{-18.434948}(-0.10306987,1.542136){\psellipse[linewidth=0.04,dimen=outer](4.7,1.0886406)(0.2,0.4)}
\pscustom[linewidth=0.08]
{
\newpath
\moveto(2.98,-0.53135943)
\lineto(3.11,-0.42135945)
\curveto(3.175,-0.36635944)(3.3,-0.25135943)(3.36,-0.19135943)
\curveto(3.42,-0.13135944)(3.58,-0.00135943)(3.68,0.06864057)
\curveto(3.78,0.13864057)(3.945,0.25364056)(4.01,0.29864055)
\curveto(4.075,0.34364057)(4.17,0.40364057)(4.2,0.41864055)
\curveto(4.23,0.43364057)(4.285,0.44864056)(4.31,0.44864056)
\curveto(4.335,0.44864056)(4.38,0.44364056)(4.4,0.43864056)
\curveto(4.42,0.43364057)(4.45,0.37864056)(4.46,0.32864055)
\curveto(4.47,0.27864057)(4.505,0.21864057)(4.53,0.20864056)
\curveto(4.555,0.19864057)(4.595,0.16864057)(4.61,0.14864056)
\curveto(4.625,0.12864056)(4.64,0.09364056)(4.64,0.07864056)
}
\psline[linewidth=0.04cm,linestyle=dotted,dotsep=0.16cm](3.0,-0.53135943)(3.96,-0.51135945)
\usefont{T1}{ptm}{m}{n}
\rput(1.9228125,-0.9813594){20 kg}
\usefont{T1}{ptm}{m}{n}
\rput(3.68125,-0.26135945){60 $\degree$}
\usefont{T1}{ptm}{m}{n}
\rput(3.3064063,0.37864056){150 N}
\psline[linewidth=0.06cm](0.36,-1.5313594)(6.0,-1.5313594)
\pscustom[linewidth=0.06]
{
\newpath
\moveto(4.64,1.0286405)
\lineto(4.64,0.9486406)
\curveto(4.64,0.90864056)(4.655,0.85864055)(4.67,0.84864056)
\curveto(4.685,0.8386406)(4.705,0.8286406)(4.72,0.8286406)
}
\psdots[dotsize=0.12](4.76,1.1686406)
\psline[linewidth=0.04cm,arrowsize=0.05291667cm 2.0,arrowlength=1.4,arrowinset=0.4]{->}(1.0,-1.3913594)(0.0,-1.3913594)
\usefont{T1}{ptm}{m}{n}
\rput(0.37640625,-1.1613594){15 N}
\end{pspicture}
}
\end{center}
% \caption{Man pulling a box}
\end{figure}
}{\westep{Draw a force diagram}
The motion is horizontal and therefore we will only consider the forces in a horizontal direction. Remember that vertical forces do not influence horizontal motion and vice versa.
%force diagram of man pulling box at angle
\begin{figure}[H]
\begin{center}
\scalebox{1} % Change this value to rescale the drawing.
{
\begin{pspicture}(0,-1.17)(4.641875,1.19)
\psframe[linewidth=0.04,dimen=outer](3.0,-0.17)(1.0,-1.17)
\psline[linewidth=0.04cm,linestyle=dotted,dotsep=0.16cm](3.0,-0.19)(4.36,-0.19)
\usefont{T1}{ptm}{m}{n}
\rput(1.9228125,-0.64){20 kg}
\usefont{T1}{ptm}{m}{n}
\rput(3.673125,0.08){60 $\degree$}
\usefont{T1}{ptm}{m}{n}
\rput(3.3064063,0.72){150 N}
\psline[linewidth=0.04cm,arrowsize=0.05291667cm 2.0,arrowlength=1.4,arrowinset=0.4]{->}(1.0,-0.45)(0.0,-0.45)
\usefont{T1}{ptm}{m}{n}
\rput(0.37640625,-0.82){15 N}
\psline[linewidth=0.04cm,arrowsize=0.05291667cm 2.0,arrowlength=1.4,arrowinset=0.4]{->}(2.98,-0.17)(4.42,1.17)
\psline[linewidth=0.04cm,linestyle=dotted,dotsep=0.16cm](4.38,1.15)(4.38,-0.17)
\usefont{T1}{ptm}{m}{n}
\rput(3.8514063,-0.42){$F_{x}$}
\end{pspicture}
}
\end{center}
% \caption{Force diagram}
\end{figure}

\westep{Calculate the horizontal component of the applied force}
The applied force is acting at an angle of 60 $\degree$ to the horizontal. We can only consider forces that are parallel to the motion. The horizontal component of the applied force needs to be calculated before we can continue:\\
\begin{eqnarray*}
F_{x} &=& 150~\rm{cos}~60\degree\\
F_{x} &=&~75\rm{N}
\end{eqnarray*}

\westep{Calculate the acceleration}
To find the acceleration we apply Newton's Second Law:\\
\begin{eqnarray*}
F_{R} &=& ma\\
F_{x}+ F_{f} &=& (20)(a)\\
75 + (-15) &=& 20a\\
a &=& 3 \emss \mbox{to the right}
\end{eqnarray*}
}
\end{wex}

\begin{wex}{Newton II - Truck and trailer}
{A 2000 kg truck pulls a 500 kg trailer with a constant acceleration. The engine of the truck produces a thrust of 10 000 N. Ignore the effect of friction.\\
\begin{enumerate}
\item Calculate the acceleration of the truck.
\item Calculate the tension in the tow bar T between the truck and the trailer, if the tow bar makes an angle of 25$\degree$ with the horizontal.
\end{enumerate}
\begin{figure}[H]
\begin{center}
\scalebox{1} % Change this value to rescale the drawing.
{
\begin{pspicture}(0,-1.8539063)(6.795,1.8539063)
\psframe[linewidth=0.04,dimen=outer](1.7,-0.05609375)(0.0,-0.75609374)
\pscircle[linewidth=0.04,dimen=outer](0.45,-0.7860938){0.25}
\pscircle[linewidth=0.04,dimen=outer](1.25,-0.7860938){0.25}
\psframe[linewidth=0.04,dimen=outer](4.1,0.16390625)(2.4,-0.73609376)
\pspolygon[linewidth=0.04](4.84,0.68390626)(5.34,-0.01609375)(5.34,-0.7160938)(4.14,-0.7160938)(4.14,0.68390626)
\pscircle[linewidth=0.04,dimen=outer](4.6,-0.73609376){0.3}
\pscircle[linewidth=0.04,dimen=outer](3.0,-0.73609376){0.3}
\psline[linewidth=0.08cm](1.68,-0.55609375)(2.42,-0.27609375)
\psline[linewidth=0.06cm,arrowsize=0.05291667cm 2.0,arrowlength=1.4,arrowinset=0.4]{->}(1.9,1.2639062)(4.1,1.2639062)
\psline[linewidth=0.04cm,linestyle=dotted,dotsep=0.16cm](1.7,-0.63609374)(2.4,-0.63609374)
\pscustom[linewidth=0.04]
{
\newpath
\moveto(2.0,-0.5360938)
\lineto(2.05,-0.73609376)
\curveto(2.075,-0.8360937)(2.025,-1.0360937)(1.95,-1.1360937)
\curveto(1.875,-1.2360938)(1.8,-1.3610938)(1.8,-1.4360938)
}
\usefont{T1}{ptm}{m}{n}
\rput(1.7670312,-1.6260937){25$\degree$}
\usefont{T1}{ptm}{m}{n}
\rput(3.1373436,1.6739062){a = ? \mss}
\usefont{T1}{ptm}{m}{n}
\rput(0.888125,-0.32609376){500 kg}
\usefont{T1}{ptm}{m}{n}
\rput(3.3828125,-0.22609375){2000 kg}
\psline[linewidth=0.06cm,arrowsize=0.05291667cm 2.0,arrowlength=1.4,arrowinset=0.4]{->}(4.6,0.16390625)(6.0,0.16390625)
\usefont{T1}{ptm}{m}{n}
\rput(6.056406,0.47390625){10 000 N}
\usefont{T1}{ptm}{m}{n}
\rput(2.0,-0.22609375){T}
\end{pspicture}
}
\end{center}
\caption{Truck pulling a trailer}
\end{figure}
}{\westep{Draw a force diagram}
Draw a force diagram indicating all the horizontal forces on the system as a whole:
%Force diagram of truck and trailer
\begin{figure}[H]
\begin{center}
\scalebox{1} % Change this value to rescale the drawing.
{
\begin{pspicture}(0,-1.05)(6.83,1.05)
\psframe[linewidth=0.04,dimen=outer](1.7,0.03)(0.0,-0.67)
\pscircle[linewidth=0.04,dimen=outer](0.45,-0.7){0.25}
\pscircle[linewidth=0.04,dimen=outer](1.25,-0.7){0.25}
\psframe[linewidth=0.04,dimen=outer](4.1,0.25)(2.4,-0.65)
\pspolygon[linewidth=0.04](4.84,0.77)(5.34,0.07)(5.34,-0.63)(4.14,-0.63)(4.14,0.77)
\pscircle[linewidth=0.04,dimen=outer](4.6,-0.65){0.3}
\pscircle[linewidth=0.04,dimen=outer](3.0,-0.65){0.3}
\psline[linewidth=0.08cm](1.68,-0.47)(2.42,-0.19)
\psline[linewidth=0.04cm,linestyle=dotted,dotsep=0.16cm](1.7,-0.55)(2.4,-0.55)
\usefont{T1}{ptm}{m}{n}
\rput(2.8828125,0.66){2500 kg}
\psline[linewidth=0.06cm,arrowsize=0.05291667cm 2.0,arrowlength=1.4,arrowinset=0.4]{->}(5.4,0.25)(6.8,0.25)
\usefont{T1}{ptm}{m}{n}
\rput(6.056406,0.56){10 000 N}
\usefont{T1}{ptm}{m}{n}
\rput(2.0,-0.14){T}
\psframe[linewidth=0.04,linestyle=dashed,dash=0.16cm 0.16cm,dimen=outer](5.4,1.05)(0.0,-1.05)
\end{pspicture}
}
\end{center}
\caption{Force diagram for truck pulling a trailer}
\end{figure}

\westep{Find the acceleration of the system}
In the absence of friction, the only force that causes the system to accelerate is the thrust of the engine. If we now apply Newton's Second Law:
\begin{eqnarray*}
F_{R} &=& ma\\
10 000 &=& (500 + 2000) a\\
a &=& 4 \emss\ \rm{to~the~right}
\end{eqnarray*}

\westep{Find the horizontal component of T}
We are asked to find the tension in the tow bar, but because the tow bar is acting at an angle, we need to find the horizontal component first. We will find the horizontal component in terms of T and then use it in the next step to find T.
%horizontal component
\scalebox{1} % Change this value to rescale the drawing.
{
\begin{pspicture}(0,-0.758125)(2.9940624,0.758125)
\psline[linewidth=0.04cm,linestyle=dashed,dash=0.16cm 0.16cm](0.5921875,-0.3403125)(2.8921876,-0.3403125)
\psline[linewidth=0.04cm,linestyle=dashed,dash=0.16cm 0.16cm](2.8921876,-0.3403125)(2.8921876,0.6596875)
\psline[linewidth=0.04cm](0.5921875,-0.3403125)(2.8921876,0.6596875)
\usefont{T1}{ptm}{m}{n}
\rput(1.4921875,0.2696875){T}
\usefont{T1}{ptm}{m}{n}
\rput(1.6676563,-0.5303125){T cos25$\degree$}
\pscustom[linewidth=0.04]
{
\newpath
\moveto(1.1921875,-0.2403125)
\lineto(1.1421875,0.0596875)
\curveto(1.1171875,0.2096875)(1.0421875,0.3846875)(0.8921875,0.4596875)
}
\usefont{T1}{ptm}{m}{n}
\rput(0.8592188,0.5696875){25$\degree$}
\end{pspicture}
}

The horizontal component is T cos~25$\degree$.\\

\westep{Find the tension in the tow bar}
To find T, we will apply Newton's Second Law:\\
\begin{eqnarray*}
F_{R} &=& ma\\
F - T\ \rm{cos}~25\degree &=& ma\\
10 000 - T\ \rm{cos}~25\degree &=& (2000)(4)\\
T\ \rm{cos}~25\degree &=& 2000\\
T &=& 2206,76 \rm{N}
\end{eqnarray*}
}
\end{wex}

\subsubsection{Object on an inclined plane}
When we place an object on a slope the force of gravity (F$_{g}$) acts straight down and not perpendicular to the slope. Due to gravity pulling straight down, the object will tend to slide down the slope with a force equal to the horizontal component of the force of gravity (F$_{g}$~sin~$\theta$). The object will 'stick' to the slope due to the frictional force between the object and the surface. As you increase the angle of the slope, the horizontal component will also increase until the frictional force is overcome and the object starts to slide down the slope.
\Tip{Do not use the abbreviation W for weight as it is used to abbreviate 'work'. Rather use the force of gravity F$_{g}$ for weight.}
The force of gravity will also tend to push an object 'into' the slope. The vertical component of this force is equal to the vertical component of the force of gravity (F$_{g}$ cos $\theta$). There is no movement in this direction as this force is balanced by the slope pushing up against the object. This ``pushing force'' is called the normal force (N) and is equal to the force required to make the component of the resultant force perpendicularly into the plane zero, F$_{g}$ cos $\theta$ in this case, but opposite in direction.

%Explaining objects on an inclined plane
\begin{figure}[H]
\begin{center}
\scalebox{1.0} % Change this value to rescale the drawing.
{
\begin{pspicture}(0,-3.6039062)(8.82375,3.6039062)
\definecolor{color114b}{rgb}{0.4,0.4,0.4}
\rput{30.623583}(1.2169431,-1.7169884){\psframe[linewidth=0.04,dimen=outer](4.5940623,1.8139062)(2.8940625,0.9139063)}
\pscircle[linewidth=0.04,dimen=outer](7.1940627,2.1139061){1.1}
\pscircle[linewidth=0.04,dimen=outer](4.4440627,1.2639062){0.15}
\pspolygon[linewidth=0.04,fillstyle=solid,fillcolor=color114b](6.4403443,1.350546)(6.5245543,1.5576903)(6.6319985,1.4657356)(6.616509,1.6651348)(6.823653,1.580925)(6.815908,1.6806246)(6.915608,1.6883695)(6.900118,1.8877687)(7.107262,1.803559)(7.091772,2.0029583)(7.390871,2.026193)(7.283427,2.1181476)(7.4828258,2.1336374)(7.5825257,2.1413825)(7.5747805,2.241082)(7.77418,2.256572)(7.6589904,2.4482262)(7.9580894,2.471461)(7.9503446,2.5711606)(8.157489,2.4869509)
\psline[linewidth=0.04cm](6.310357,1.4991902)(8.017768,2.6886222)
\psline[linewidth=0.08cm,arrowsize=0.05291667cm 2.0,arrowlength=1.4,arrowinset=0.4]{->}(3.7940626,1.4139062)(3.7940626,-0.68609375)
\psline[linewidth=0.08cm,linestyle=dashed,dash=0.16cm 0.16cm,arrowsize=0.05291667cm 2.0,arrowlength=1.4,arrowinset=0.4]{->}(3.7940626,1.4139062)(4.7940626,-0.18609375)
\psline[linewidth=0.08cm,linestyle=dashed,dash=0.16cm 0.16cm,arrowsize=0.05291667cm 2.0,arrowlength=1.4,arrowinset=0.4]{->}(3.7940626,1.4139062)(2.7940626,0.7139062)
\usefont{T1}{ptm}{m}{n}
\rput(3.8120313,-0.87609375){F$_{g}$}
\psline[linewidth=0.04cm](1.6940625,-0.38609374)(6.1940627,-0.38609374)
\usefont{T1}{ptm}{m}{n}
\rput(7.1920314,3.4239063){Surface friction}
\usefont{T1}{ptm}{m}{n}
\rput(1.9532813,1.6239063){horizontal component }
\usefont{T1}{ptm}{m}{n}
\rput(1.9295312,1.2239063){parallel to the surface}
\usefont{T1}{ptm}{m}{n}
\rput(6.1628127,0.22390625){vertical component}
\usefont{T1}{ptm}{m}{n}
\rput(6.789531,-0.17609376){perpendicular to the surface}
\psline[linewidth=0.04cm](4.3740625,1.4139062)(6.4540625,2.9139063)
\psline[linewidth=0.04cm](4.4340625,1.1339062)(7.0940623,1.0339062)
\psline[linewidth=0.04cm](1.6940625,-0.38609374)(5.6940627,2.0139062)
\usefont{T1}{ptm}{m}{n}
\rput(2.5554688,-0.17609376){$\theta$}
\usefont{T1}{ptm}{m}{n}
\rput(3.9554687,0.82390624){$\theta$}
\usefont{T1}{ptm}{m}{n}
\rput(1.4920312,0.82390624){F$_{g}$ sin $\theta$}
\usefont{T1}{ptm}{m}{n}
\rput(5.8320312,-0.57609373){F$_{g}$ cos $\theta$}
\psline[linewidth=0.04cm](1.1940625,-1.3860937)(1.1940625,-3.3860939)
\psline[linewidth=0.04cm](1.1940625,-1.3860937)(1.9940625,-2.9860938)
\psline[linewidth=0.04cm](1.9940625,-2.9860938)(1.1940625,-3.3860939)
\psline[linewidth=0.04cm,linestyle=dashed,dash=0.16cm 0.16cm](1.1940625,-1.3860937)(0.4940625,-1.7860937)
\psline[linewidth=0.04cm,linestyle=dashed,dash=0.16cm 0.16cm](0.4940625,-1.7860937)(1.1940625,-3.3860939)
\usefont{T1}{ptm}{m}{n}
\rput(1.3554688,-1.9760938){$\theta$}
\usefont{T1}{ptm}{m}{n}
\rput(2.2920313,-3.3760939){F$_{g}$ sin $\theta$}
\usefont{T1}{ptm}{m}{n}
\rput(2.3320312,-2.2760937){F$_{g}$ cos $\theta$}
\psline[linewidth=0.04cm](1.8940625,-2.7860937)(1.6940625,-2.8860939)
\psline[linewidth=0.04cm](1.6940625,-2.8860939)(1.7940625,-3.0860937)
\usefont{T1}{ptm}{m}{n}
\rput(0.91203123,-2.5760937){F$_{g}$}
\end{pspicture}
}
\end{center}
\end{figure}

% Phet Simulation on ramp forces and motion: SIYAVULA-SIMULATION:http://cnx.org/content/m38963/latest/#id6358
\simulation{Phet on forces}{VPkjb}
\begin{wex}{Newton II - Box on inclined plane}{A body of mass M is at rest on an inclined plane.
%box on inclined plane
\begin{center}
\begin{pspicture}(0,0)(2.6,2.6)
\SpecialCoor
%\psgrid[gridcolor=lightgray]
\psline(0,0)(3;30)
\psline(0,0)(2.6,0)
\rput{30}(1,0.6){\psframe(0,0)(1,1)\psline{->}(0.5,0.5)(1.5,0.5)\psline{->}(0.5,0.5)(0.5,1.5)\uput[r](0.5,1.5){N}\uput[r](1.5,0.5){F}}
\psline{->}(1.2,1.3)(1.2,0.3)
\uput[ur](0.6,0){$\theta$}
\end{pspicture}
\end{center}
What is the magnitude of the frictional force acting on the body?
\begin{enumerate}
\item[A]{Mg}
\item[B]{Mg cos $\theta$}
\item[C]{Mg sin $\theta$}
\item[D]{Mg tan $\theta$}
\end{enumerate}
}{\westep{Analyse the situation}
The question asks us to identify the frictional force. The body is said to be at rest on the plane, which means that it is not moving and therefore there is no resultant force. The frictional force must therefore be balanced by the force F up the inclined plane.
\westep{Choose the correct answer}
The frictional force is equal to the component of the weight (Mg) parallel to the surface, which is equal to Mg~sin~$\theta$.
}
\end{wex}



\begin{wex}{Newton II - Object on a slope}{A force T = 312 N is required to keep a body at rest on a frictionless inclined plane which makes an angle of 35$\degree$ with the horizontal. The forces acting on the body are shown. Calculate the magnitudes of forces P and R, giving your answers to three significant figures.
% \begin{figure}[H]
\begin{center}
\scalebox{1} % Change this value to rescale the drawing.
{
\begin{pspicture}(0,-1.811875)(4.52,1.811875)
\rput{30.623583}(0.583109,-0.96289045){\psframe[linewidth=0.04,dimen=outer](2.9,1.0334375)(1.2,0.1334375)}
\psline[linewidth=0.08cm,arrowsize=0.05291667cm 2.0,arrowlength=1.4,arrowinset=0.4]{->}(2.1,0.6334375)(2.1,-1.4665625)
\psline[linewidth=0.08cm,linestyle=dashed,dash=0.16cm 0.16cm](2.1,0.6334375)(3.1,-0.9665625)
\psline[linewidth=0.08cm,arrowsize=0.05291667cm 2.0,arrowlength=1.4,arrowinset=0.4]{->}(2.1,0.6334375)(3.2,1.3334374)
\psline[linewidth=0.04cm](0.0,-1.1665626)(4.5,-1.1665626)
\psline[linewidth=0.04cm](0.0,-1.1665626)(4.0,1.2334375)
\usefont{T1}{ptm}{m}{n}
\rput(1.1635938,-0.9565625){35$\degree$}
\usefont{T1}{ptm}{m}{n}
\rput(2.5635939,0.0434375){35$\degree$}
\psline[linewidth=0.08cm,arrowsize=0.05291667cm 2.0,arrowlength=1.4,arrowinset=0.4]{->}(2.1,0.6334375)(1.5,1.5334375)
\usefont{T1}{ptm}{m}{n}
\rput(1.4185938,1.6434375){R}
\usefont{T1}{ptm}{m}{n}
\rput(3.3,1.4434375){T}
\usefont{T1}{ptm}{m}{n}
\rput(2.1660938,-1.6565624){P}
\end{pspicture}
}
\end{center}
% \end{figure}
}{\westep{Find the magnitude of P}
We are usually asked to find the magnitude of T, but in this case T is given and we are asked to find P. We can use the same equation. T is the force that balances the component of P parallel to the plane (P$_x$) and therefore it has the same magnitude.\\
\begin{eqnarray*}
T &=& P~\rm{sin}~\theta\\
312 &=& P~\rm{sin}~35\degree\\
P &=& 544~\eN
\end{eqnarray*}
\westep{Find the magnitude of R}
R can also be determined with the use of trigonometric ratios. The tan or cos ratio can be used. We recommend that you use the tan ratio because it does not involve using the value for P (for in case you made a mistake in calculating P).
\begin{eqnarray*}
\rm{tan}~55\degree &=& \frac{R}{T}\\
\rm{tan}~55\degree &=& \frac{R}{312}\\
R &=& \rm{tan}~55\degree \times 312\\
R &=& 445,6~\eN\\
R &=& 446~\eN
\end{eqnarray*}
Note that the question asks that the answers be given to 3 significant figures. We therefore round 445,6 N up to 446 N.
}
\end{wex}

\subsubsection{Lifts and rockets}
So far we have looked at objects being pulled or pushed across a surface, in other words motion parallel to the surface the object rests on. Here we only considered forces parallel to the surface, but we can also lift objects up or let them fall. This is vertical motion where only vertical forces are being considered.\\
\\
Let us consider a 500 kg lift, with no passengers, hanging on a cable. The purpose of the cable is to pull the lift upwards so that it can reach the next floor or to let go a little so that it can move downwards to the floor below. We will look at five possible stages during the motion of the lift.\\
%Drawing of lift with different stages

{\bf{Stage 1:}}\\ The 500 kg lift is stationary at the second floor of a tall building.\\
Because the lift is stationary (not moving) there is no resultant force acting on the lift. This means that the upward forces must be balanced by the downward forces. The only force acting down is the force of gravity which is equal to (500 x 9,8 = 4900 N) in this case. The cable must therefore pull upwards with a force of 4900 N to keep the lift stationary at this point.\\
\\
{\bf{Stage 2:}}\\ The lift moves upwards at an acceleration of 1 \mss.\\
If the lift is accelerating, it means that there is a resultant force in the direction of the motion. This means that the force acting upwards is now greater than the force of gravity F$_{g}$ (down). To find the magnitude of the force applied by the cable (F$_{c}$) we can do the following calculation: (Remember to choose a direction as positive. We have chosen upwards as positive.)\\

\begin{eqnarray*}
F_{R} &=& ma\\
F_{c}+ F_{g} &=& ma\\
F_{c}+ (-4900) &=& (500)(1)\\
F_{c} &=& 5400~\rm{N~upwards}
\end{eqnarray*}

The answer makes sense as we need a bigger force upwards to cancel the effect of gravity as well as make the lift go faster.\\
\\
{\bf{Stage 3:}}\\ The lift moves at a constant velocity.\\
When the lift moves at a constant velocity, it means that all the forces are balanced and that there is no resultant force. The acceleration is zero, therefore F$_{R}$ = 0. The force acting upwards is equal to the force acting downwards, therefore F$_{c}$ = 4900 N.\\
\\
{\bf{Stage 4:}}\\ The lift slows down at a rate of 2\mss.\\
As the lift is now slowing down there is a resultant force downwards. This means that the force acting downwards is greater than the force acting upwards. To find the magnitude of the force applied by the cable (F$_{c}$) we can do the following calculation: Again we have chosen upwards as positive, which means that the acceleration will be a negative number.\\

\begin{eqnarray*}
F_{R} &=& ma\\
F_{c}+ F_{g} &=& ma\\
F_{c}+ (-4900) &=& (500)(-2)\\
F_{c} &=& 3900~\rm{N~upwards}
\end{eqnarray*}

This makes sense as we need a smaller force upwards to ensure that the resultant force is downward. The force of gravity is now greater than the upward pull of the cable and the lift will slow down.\\
\\
{\bf{Stage 5:}}\\ The cable snaps.\\
When the cable snaps, the force that used to be acting upwards is no longer present. The only force that is present would be the force of gravity. The lift will freefall and its acceleration can be calculated as follows:

\begin{eqnarray*}
F_{R} &=& ma\\
F_{c}+ F_{g} &=& ma\\
0 + (-4900) &=& (500)(a)\\
a &=& -9,8 \emss\\
a &=& 9,8 \emss \rm{downwards}
\end{eqnarray*}

\subsubsection{Rockets}
As with lifts, rockets are also examples of objects in vertical motion. The force of gravity pulls the rocket down while the thrust of the engine pushes the rocket upwards. The force that the engine exerts must overcome the force of gravity so that the rocket can accelerate upwards. The worked example below looks at the application of Newton's Second Law in launching a rocket.

\begin{wex}{Newton II - rocket}
{A rocket is launched vertically upwards into the sky at an acceleration of 20 \mss. If the mass of the rocket is 5000 kg, calculate the magnitude and direction of the thrust of the rocket's engines.
%picture of rocket
}
{\westep{Analyse what is given and what is asked}
We have the following:\\
m = 5000 kg\\
a = 20 \mss\\
F$_{g}$ = 5000 x 9,8 = 49000 N\\
We are asked to find the thrust of the rocket engine F$_{1}$.\\

\westep{Find the thrust of the engine}
We will apply Newton's Second Law:
\begin{eqnarray*}
F_{R} &=& ma\\
F_{1}+ F_{g} &=& ma\\
F_{1} + (-49000) &=& (5000)(20)\\
F_{1} &=& 149~000~\rm{N~upwards}\\
\end{eqnarray*}
}
\end{wex}

\begin{wex}{Rockets}{How do rockets accelerate in space?
\begin{center}
\begin{pspicture}(-1,0)(1,5)
%\psgrid
%Drawing a rocket is a somewhat non-trivial exercise
%This picture is derived from
%http://www.geocities.com/rocketguy_101/ogive/OgiveNoseCones.html
\psset{xunit=0.5,yunit=0.5}
\psline(-1,4)(-2,2)(2,2)(1,4)
\psframe(-0.6,1.6)(0.6,2)
\rput{90}(1,0){\rput(5,2){
\psplot{-3}{4.45}{10.025 2 exp x 2 exp sub 0.5 exp 9.975 sub}
\psplot{-3}{4.45}{10.025 2 exp x 2 exp sub 0.5 exp 9.975 sub neg 2 sub}}}
\psdot[dotsize=3pt](0,4)
\psline{<->}(0,10)(0,0)
\uput[dr](0,10){$F$}
\uput[ur](0,0){$W$}
\psline{<-}(0.6,1.8)(2,1.8)
\uput[r](2,1.8){tail nozzle}
\end{pspicture}
\end{center}
}{\begin{itemize}
\item Gas explodes inside the rocket.
\item This exploding gas exerts a force on each side of the rocket
(as shown in the picture below of the explosion chamber inside the
rocket).
\begin{center}
\begin{pspicture}(-3.5,-2.4)(3.5,1)
%\psgrid[gridcolor=lightgray]
\psline(-1,-1)(-1,1)(1,1)(1,-1)
\cnode[fillstyle=crosshatch](0,0){.4}{mycircle}
\psline[arrowscale=2]{->}(0, 0.4)(0,1)
\psline[arrowscale=2]{->}(0.4, 0)(1,0)
\psline[arrowscale=2]{->}(-0.4, 0)(-1,0)
\rput(0,-1.6){\parbox{7cm}{Note that the forces shown in this picture
are representative. With an explosion there will be forces in all
directions.}}
\end{pspicture}
\end{center}
\item Due to the symmetry of the situation, all the forces exerted
on the rocket are balanced by forces on the opposite side, except for the force
opposite the open side. This force on the upper surface is unbalanced.
\item This is therefore the resultant force acting on the rocket and it
makes the rocket accelerate forwards.
\end{itemize}}
\end{wex}


\begin{wex}{Newton II - lifts}
{A lift, mass 250 kg, is initially at rest on the ground floor of a tall building. Passengers with an unknown total mass, m, climb into the lift. The lift accelerates upwards at 1,6 \mss. The cable supporting the lift exerts a constant upward force of 7700 N. Use g = 10 \mss.
\begin{enumerate}
\item Draw a labelled force diagram indicating all the forces acting on the lift while it accelerates upwards.
\item What is the maximum mass, m, of the passengers the lift can carry in order to achieve a constant upward acceleration of 1,6 \mss.
%\item As a height of 5~m above the ground floor, the cable suddenly snaps. Show that the velocity of the lift at that instant is 4 \ms.
%\item Determine the maximum height above the ground floor reached by the lift before it begins to descend (fall downwards).
\end{enumerate}
}
{\westep{Draw a force diagram.}
\begin{figure}[H]
\begin{center}
\scalebox{1} % Change this value to rescale the drawing.
{
\begin{pspicture}(0,-2.24)(8.101875,2.24)
\psline[linewidth=0.08cm,arrowsize=0.05291667cm 2.0,arrowlength=1.4,arrowinset=0.4]{->}(3.30125,-0.9)(3.30125,-2.2)
\psline[linewidth=0.08cm,arrowsize=0.05291667cm 2.0,arrowlength=1.4,arrowinset=0.4]{->}(3.20125,0.6)(3.20125,2.2)
\psline[linewidth=0.08cm,arrowsize=0.05291667cm 2.0,arrowlength=1.4,arrowinset=0.4]{->}(3.10125,-0.9)(3.10125,-1.9)
\usefont{T1}{ptm}{m}{n}
\rput(5.7284374,-1.19){Downward force of Earth on lift}
\psframe[linewidth=0.04,dimen=outer](3.70125,0.6)(2.70125,-0.9)
\usefont{T1}{ptm}{m}{n}
\rput(5.390625,1.61){Upward force of cable on lift}
\usefont{T1}{ptm}{m}{n}
\rput(1.4364063,-1.09){Downward force of }
\usefont{T1}{ptm}{m}{n}
\rput(1.2421875,-1.49){passengers on lift}
\usefont{T1}{ptm}{m}{n}
\rput(0.6064063,-1.89){(10 x m)}
\usefont{T1}{ptm}{m}{n}
\rput(4.4464064,-1.59){(2500 N)}
\usefont{T1}{ptm}{m}{n}
\rput(4.496406,1.21){(F$_C$ = 7700 N)}
\end{pspicture}
}
\end{center}
\end{figure}

\westep{Find the mass, m.}
Let us look at the lift with its passengers as a unit. The mass of this unit will be (250 + m) kg and the force of the Earth pulling downwards (F$_g$) will be (250~+~m)~x~10 m.s$^{-2}$. If we apply Newton's Second Law to the situation we get:
\begin{eqnarray*}
F_{net} &=& ma\\
F_C - F_g &=& ma\\
7700 - (250 + m)(10) &=& (250 + m)(1,6)\\
7700 - 2500 - 10~m &=& 400 + 1,6~m\\
4800 &=& 11,6~m\\
m &=& 413,79 \ekg
\end{eqnarray*}

}
\end{wex}


%\subsubsection{Pulleys} does not seem to be in the syllabus, add later if needed, worked example of two boxes on table and pulley with mass piece over the side, two masses next to each other hanging over same pulley, also worked examples of mcq questions.

\Exercise{title}{
  \begin{enumerate}
  \item{A tug is capable of pulling a ship with a force of 100~kN. If two such tugs are pulling on one ship, they can produce any force ranging from a minimum of 0~N to a maximum of 200~kN. Give a detailed explanation of how this is possible. Use diagrams to support your result.}
  \item{A car of mass 850~kg accelerates at 2~\mss. Calculate the magnitude of the resultant force that is causing the acceleration.}
  \item{Find the force needed to accelerate a 3~kg object at 4~\mss.}
  \item{Calculate the acceleration of an object of mass 1000 kg accelerated by a force of 100 N.}
  \item{An object of mass 7 kg is accelerating at 2,5 \mss. What resultant force acts on it?}
  \item{Find the mass of an object if a force of 40~N gives it an acceleration of 2~\mss.}
  \item{Find the acceleration of a body of mass 1 000~kg that has a 150~N force acting on it.}
  \item{Find the mass of an object which is accelerated at 2~\mss\ by a force of 40~N.}
  \item{Determine the acceleration of a mass of 24~kg when a force of 6~N acts on it. What is the acceleration if the force were doubled and the mass was halved?}
  \item{A mass of 8~kg is accelerating at 5~\mss.
      \begin{enumerate}
      \item{Determine the resultant force that is causing the acceleration.}
      \item{What acceleration would be produced if we doubled the force and reduced the mass by half?}
      \end{enumerate}}
  \item{A motorcycle of mass 100~kg is accelerated by a resultant force of 500 ~N. If the motorcycle starts from rest:
      \begin{enumerate}
      \item{What is its acceleration?}
      \item{How fast will it be travelling after 20~s?}
      \item{How long will it take to reach a speed of 35~\ms?}
      \item{How far will it travel from its starting point in 15~s?}
      \end{enumerate}}
  \item{A force acting on a trolley on a frictionless horizontal plane causes an acceleration of magnitude 6 \mss. Determine the mass of the trolley.}
  \item {A force of 200 N, acting at 60$\degree$ to the horizontal, accelerates a block of mass 50 kg along a horizontal plane as shown.
      \begin{figure}[H]
        \begin{center}
          \scalebox{1}{% Change this value to rescale the drawing.
            \begin{pspicture}(0,-1.54)(6.374375,1.53)
              \psframe[linewidth=0.04,dimen=outer](3.344375,-0.49)(1.344375,-1.49)
              \psline[linewidth=0.04cm,linestyle=dotted,dotsep=0.16cm](0.484375,-0.51)(1.444375,-0.49)
              \usefont{T1}{ptm}{m}{n}
              \rput(2.2625,-0.96){50 kg}
              \usefont{T1}{ptm}{m}{n}
              \rput(0.6975,-0.1){60 $\degree$}
              \usefont{T1}{ptm}{m}{n}
              \rput(1.6079688,0.18){200 N}
              \psline[linewidth=0.06cm](0.704375,-1.51)(6.344375,-1.51)
              \psline[linewidth=0.04cm,arrowsize=0.05291667cm 2.0,arrowlength=1.4,arrowinset=0.4]{->}(0.324375,1.51)(1.364375,-0.55)
              \psline[linewidth=0.04cm,arrowsize=0.05291667cm 2.0,arrowlength=1.4,arrowinset=0.4]{->}(3.964375,-0.99)(5.584375,-0.99)
            \end{pspicture}
          }
        \end{center}
      \end{figure}
      \begin{enumerate}
      \item Calculate the component of the 200 N force that accelerates the block horizontally.
      \item If the acceleration of the block is 1,5 \mss, calculate the magnitude of the frictional force on the block.
      \item Calculate the vertical force exerted by the block on the plane.
      \end{enumerate}}
  \item{A toy rocket of mass 0,5 kg is supported vertically by placing it in a bottle. The rocket is then ignited. Calculate the force that is required to accelerate the rocket vertically upwards at 8 \mss.}
  \item{A constant force of 70 N is applied vertically to a block of mass 5 kg as shown. Calculate the acceleration of the block.
      \begin{figure}[H]
        \begin{center}
          \scalebox{1}{% Change this value to rescale the drawing.
            \begin{pspicture}(0,-1.99)(1.715,2.01)
              \psframe[linewidth=0.04,dimen=outer](1.44,0.25)(0.0,-1.99)
              \usefont{T1}{ptm}{m}{n}
              \rput(0.728125,-0.84){5 kg}
              \usefont{T1}{ptm}{m}{n}
              \rput(1.3107812,1.08){70 N}
              \psline[linewidth=0.04cm,arrowsize=0.05291667cm 2.0,arrowlength=1.4,arrowinset=0.4]{->}(0.72,0.25)(0.72,1.99)
            \end{pspicture}
          }
        \end{center}
      \end{figure}
    }
  \item{A stationary block of mass 3kg is on top of a plane inclined at 35$\degree$ to the horizontal.\\
      % \scalebox{1}  %Change this value to rescale the drawing.
      \begin{center}
        \begin{pspicture}(0,-1.32)(4.0,1.32) \psline[linewidth=0.04cm](0.0,-1.3)(3.98,-1.3) \psline[linewidth=0.04cm](3.98,1.3)(3.98,-1.3) \psline[linewidth=0.04cm](0.0,-1.28)(3.96,1.28) \usefont{T1}{ptm}{m}{n} \rput(1.0734375,-1.05){35$^{\circ}$} \psline[linewidth=0.04cm](1.08,0.22)(1.76,0.66) \usefont{T1}{ptm}{m}{n} \rput(1.5576563,0.11){3kg} \psline[linewidth=0.04cm](1.08,0.22)(1.42,-0.34) \psline[linewidth=0.04cm](1.74,0.66)(2.1,0.12) \end{pspicture} \end{center}
      \begin{enumerate} \item Draw a force diagram (not to scale). Include the weight (F$_{g}$) of the block as well as the components of the weight that are perpendicular and parallel to the inclined plane. \item Determine the values of the weight's perpendicular and parallel components (F$_{gx}$ and F$_{gy}$). \item Determine the magnitude and direction of the frictional force between the block and plane. \end{enumerate}}
  \item{A student of mass 70 kg investigates the motion of a lift. While he stands in the lift on a bathroom scale (calibrated in newton), he notes three stages of his journey.
      \begin{enumerate}
      \item For 2 s immediately after the lift starts, the scale reads 574 N.
      \item For a further 6 s it reads 700 N.
      \item For the final 2 s it reads 854 N.
      \end{enumerate}
      Answer the following questions:
      \begin{enumerate}
      \item Is the motion of the lift upward or downward? Give a reason for your answer.
      \item Write down the magnitude and the direction of the resultant force acting on the student for each of the stages I, II and III.
      \item Calculate the magnitude of the acceleration of the lift during the first 2s.
      \end{enumerate}}
  \item{A car of mass 800~kg accelerates along a level road at 4~\mss. A frictional force of 700~N opposes its motion. What force is produced by the car's engine?}
  \item {Two objects, with masses of 1 kg and 2 kg respectively, are placed on a smooth surface and connected with a piece of string. A horizontal force of 6 N is applied with the help of a spring balance to the 1~kg object. Ignoring friction, what will the force acting on the 2~kg mass, as measured by a second spring balance, be?
      \begin{figure}[H]
        \begin{center}
          \scalebox{1} % Change this value to rescale the drawing.
          {
            \begin{pspicture}(0,-0.87171876)(11.0,0.9117187)
              \psframe[linewidth=0.04,dimen=outer](5.0,0.12828125)(4.0,-0.87171876)
              \psframe[linewidth=0.04,dimen=outer](11.0,0.12828125)(9.0,-0.87171876)
              \psframe[linewidth=0.04,dimen=outer](7.9,-0.07171875)(6.3,-0.57171875)
              \psframe[linewidth=0.04,dimen=outer](7.8,-0.17171875)(6.6,-0.47171876)
              \psframe[linewidth=0.04,dimen=outer](6.3,-0.17171875)(6.0,-0.47171876)
              \psline[linewidth=0.04cm](6.0,-0.27171874)(5.0,-0.27171874)
              \psline[linewidth=0.04cm](7.9,-0.27171874)(9.0,-0.27171874)
              \psline[linewidth=0.04cm](4.0,-0.27171874)(2.9,-0.27171874)
              \psframe[linewidth=0.04,dimen=outer](2.9,-0.07171875)(1.3,-0.57171875)
              \psframe[linewidth=0.04,dimen=outer](2.8,-0.17171875)(1.6,-0.47171876)
              \psframe[linewidth=0.04,dimen=outer](1.3,-0.17171875)(1.0,-0.47171876)
              \psline[linewidth=0.04cm](1.0,-0.27171874)(0.0,-0.27171874)
              \psline[linewidth=0.04cm](1.8,-0.17171875)(1.8,-0.27171874)
              \psline[linewidth=0.04cm](2.0,-0.17171875)(2.0,-0.27171874)
              \psline[linewidth=0.04cm](2.2,-0.17171875)(2.2,-0.27171874)
              \psline[linewidth=0.04cm](2.4,-0.17171875)(2.4,-0.27171874)
              \psline[linewidth=0.04cm](2.6,-0.17171875)(2.6,-0.27171874)
              \psline[linewidth=0.04cm](6.8,-0.17171875)(6.8,-0.27171874)
              \psline[linewidth=0.04cm](7.0,-0.17171875)(7.0,-0.27171874)
              \psline[linewidth=0.04cm](7.2,-0.17171875)(7.2,-0.27171874)
              \psline[linewidth=0.04cm](7.4,-0.17171875)(7.4,-0.27171874)
              \psline[linewidth=0.04cm](7.6,-0.17171875)(7.6,-0.27171874)
              \usefont{T1}{ptm}{m}{n}
              \rput(4.495625,-0.36171874){1 kg}
              \usefont{T1}{ptm}{m}{n}
              \rput(9.912812,-0.36171874){2 kg}
              \psline[linewidth=0.04cm,arrowsize=0.05291667cm 2.0,arrowlength=1.4,arrowinset=0.4]{->}(2.5,0.5282813)(0.7,0.5282813)
              \usefont{T1}{ptm}{m}{n}
              \rput(1.5596875,0.73828125){6 N}
              \usefont{T1}{ptm}{m}{n}
              \rput(7.0645313,0.13828126){?}
            \end{pspicture}
          }
        \end{center}
      \end{figure}
    }
  \item{A rocket of mass 200~kg has a resultant force of 4000~N upwards on it.
      \begin{enumerate}
      \item{What is its acceleration in space, where it has no weight?}
      \item{What is its acceleration on the Earth, where it has weight?}
      \item{What driving force does the rocket engine need to exert on the back of the rocket in space?}
      \item{What driving force does the rocket engine need to exert on the back of the rocket on the Earth?}
      \end{enumerate}}
    
  \item{A car going at 20 \ms accelerates uniformly and comes to a stop in a distance of 20 m.
      \begin{enumerate}
      \item What is its acceleration?
      \item If the car is 1000 kg how much force do the brakes exert?
      \end{enumerate}}
  \end{enumerate}

\practiceinfo

\begin{tabular}[h]{cccccc}
(1.) 00p4 & (2.) 00p5 & (3.) 00p6 & (4.) 00p7 & (5.) 00p8 & (6.) 00p9 & (7.) 00pa & (8.) 00pb & (9.) 00pc & (10.) 00pd & (11.) 00pe & (12.) 00pf & (13.) 00pg & (14.) 00ph & (15.) 00pi & (16.) 00pj & (17.) 00pk & (18.) 00pm & (19.) 00pn & (20.) 00pp & (21.) 00pq & 
 \end{tabular}
}



\subsection{Newton's Third Law of Motion}
Newton's Third Law of Motion deals with the interaction between pairs of objects. For example, if you hold a book up against a wall you are exerting a force on the book (to keep it there) and the book is exerting a force back at you (to keep you from falling through the book). This may sound strange, but if the book was not pushing back at you, your hand would push through the book! These two forces (the force of the hand on the book (F$_{1}$) and the force of the book on the hand (F$_{2}$)) are called an action-reaction pair of forces. They have the same magnitude, but act in opposite directions and act on different objects (the one force is onto the book and the other is onto your hand). \\

There is another action-reaction pair of forces present in this situation. The book is pushing against the wall (action force) and the wall is pushing back at the book (reaction). The force of the book on the wall (F$_{3}$) and the force of the wall on the book (F$_{4}$) are shown in the diagram.

\begin{figure}[H]
\begin{center}
\scalebox{1} % Change this value to rescale the drawing.
{
\begin{pspicture}(0,-2.325)(8.51625,2.365)
\definecolor{color0b}{rgb}{0.6,0.6,0.6}
\psframe[linewidth=0.04,dimen=outer,fillstyle=solid,fillcolor=color0b](1.7940625,1.975)(1.3940625,-2.325)
\psframe[linewidth=0.04,dimen=outer](1.9940625,0.475)(1.7940625,-0.725)
\psline[linewidth=0.06cm,arrowsize=0.05291667cm 2.0,arrowlength=1.4,arrowinset=0.4]{->}(2.0140624,0.155)(0.3140625,0.155)
\usefont{T1}{ptm}{m}{n}
\rput(0.9892188,2.185){wall}
\usefont{T1}{ptm}{m}{n}
\rput(2.355625,1.385){book}
\usefont{T1}{ptm}{m}{n}
\rput(6.1321874,1.185){F$_{1}$: force of hand on book}
\pscustom[linewidth=0.06]
{
\newpath
\moveto(2.0940626,-0.585)
\lineto(2.0740626,-0.545)
\curveto(2.0640626,-0.525)(2.0490625,-0.485)(2.0440626,-0.465)
\curveto(2.0390625,-0.445)(2.0340624,-0.4)(2.0340624,-0.375)
\curveto(2.0340624,-0.35)(2.0340624,-0.3)(2.0340624,-0.275)
\curveto(2.0340624,-0.25)(2.0290625,-0.2)(2.0240624,-0.175)
\curveto(2.0190625,-0.15)(2.0140624,-0.095)(2.0140624,-0.065)
\curveto(2.0140624,-0.035)(2.0140624,0.025)(2.0140624,0.055)
\curveto(2.0140624,0.085)(2.0140624,0.14)(2.0140624,0.165)
\curveto(2.0140624,0.19)(2.0140624,0.24)(2.0140624,0.265)
\curveto(2.0140624,0.29)(2.0140624,0.34)(2.0140624,0.365)
\curveto(2.0140624,0.39)(2.0290625,0.425)(2.0440626,0.435)
\curveto(2.0590625,0.445)(2.0790625,0.435)(2.0840626,0.415)
\curveto(2.0890625,0.395)(2.1090624,0.34)(2.1240625,0.305)
\curveto(2.1390624,0.27)(2.1590624,0.19)(2.1640625,0.145)
\curveto(2.1690626,0.1)(2.1740625,0.03)(2.1740625,0.0050)
\curveto(2.1740625,-0.02)(2.1740625,-0.07)(2.1740625,-0.095)
\curveto(2.1740625,-0.12)(2.1740625,-0.18)(2.1740625,-0.215)
\curveto(2.1740625,-0.25)(2.2040625,-0.295)(2.2340624,-0.305)
\curveto(2.2640624,-0.315)(2.3040626,-0.32)(2.3140626,-0.315)
\curveto(2.3240626,-0.31)(2.3590624,-0.305)(2.3840625,-0.305)
\curveto(2.4090624,-0.305)(2.4590626,-0.305)(2.4840624,-0.305)
\curveto(2.5090625,-0.305)(2.5590625,-0.305)(2.5840626,-0.305)
}
\pscustom[linewidth=0.06]
{
\newpath
\moveto(2.0740626,-0.545)
\lineto(2.1240625,-0.565)
\curveto(2.1490624,-0.575)(2.1990626,-0.585)(2.2240624,-0.585)
\curveto(2.2490625,-0.585)(2.2990625,-0.585)(2.3240626,-0.585)
\curveto(2.3490624,-0.585)(2.3990624,-0.585)(2.4240625,-0.585)
\curveto(2.4490626,-0.585)(2.5040624,-0.585)(2.5940626,-0.585)
}
\pscustom[linewidth=0.06]
{
\newpath
\moveto(2.1740625,-0.245)
\lineto(2.1440625,-0.195)
\curveto(2.1290624,-0.17)(2.1090624,-0.125)(2.1040626,-0.105)
\curveto(2.0990624,-0.085)(2.0940626,-0.04)(2.0940626,-0.015)
\curveto(2.0940626,0.01)(2.0840626,0.055)(2.0740626,0.075)
\curveto(2.0640626,0.095)(2.0490625,0.105)(2.0340624,0.075)
}
\psline[linewidth=0.06cm,arrowsize=0.05291667cm 2.0,arrowlength=1.4,arrowinset=0.4]{->}(1.9940625,0.175)(3.5940626,0.175)
\psline[linewidth=0.02cm](2.6940625,0.275)(2.6940625,0.075)
\psline[linewidth=0.02cm](2.7940626,0.275)(2.7940626,0.075)
\psline[linewidth=0.02cm](0.9940625,0.275)(0.9940625,0.075)
\psline[linewidth=0.02cm](1.0940624,0.275)(1.0940624,0.075)
\psline[linewidth=0.02cm,arrowsize=0.05291667cm 2.0,arrowlength=1.4,arrowinset=0.4]{->}(0.9940625,1.975)(1.5940624,1.375)
\psline[linewidth=0.02cm,arrowsize=0.05291667cm 2.0,arrowlength=1.4,arrowinset=0.4]{->}(2.1940625,1.175)(1.8940625,0.375)
\psline[linewidth=0.06cm,arrowsize=0.05291667cm 2.0,arrowlength=1.4,arrowinset=0.4]{->}(1.8140625,-0.245)(0.1140625,-0.245)
\psline[linewidth=0.06cm,arrowsize=0.05291667cm 2.0,arrowlength=1.4,arrowinset=0.4]{->}(1.7940625,-0.225)(3.3940625,-0.225)
\psline[linewidth=0.02cm](0.6940625,-0.125)(0.6940625,-0.325)
\psline[linewidth=0.02cm](2.6940625,-0.125)(2.6940625,-0.325)
\psdots[dotsize=0.18](1.7940625,-0.225)
\psdots[dotsize=0.18](1.9940625,0.175)
\usefont{T1}{ptm}{m}{n}
\rput(6.1315627,0.785){F$_{2}$: force of book on hand}
\usefont{T1}{ptm}{m}{n}
\rput(6.077344,0.385){F$_{3}$: force of book on wall}
\usefont{T1}{ptm}{m}{n}
\rput(6.0721874,-0.015){F$_{4}$: force of wall on book}
\usefont{T1}{ptm}{m}{n}
\rput(1.0220313,0.385){F$_{1}$}
\usefont{T1}{ptm}{m}{n}
\rput(3.0220313,0.385){F$_{2}$}
\usefont{T1}{ptm}{m}{n}
\rput(0.6220313,-0.515){F$_{3}$}
\usefont{T1}{ptm}{m}{n}
\rput(3.3220313,-0.515){F$_{4}$}
\end{pspicture}
}
\end{center}
\caption{Newton's action-reaction pairs}
\end{figure}

\Definition{Newton's Third Law of Motion}
{If body A exerts a force on body B, then body B exerts a force of equal magnitude on body A, but in the opposite direction.}
% Khan Academy video on Newton's Third Law of motion: SIYAVULA-VIDEO:http://cnx.org/content/m38972/latest/#newton-3
\mindsetvid{Khan on newton 3}{VPkjr}\\
Newton's action-reaction pairs can be found everywhere in life where two objects interact with one another. The following worked examples will illustrate this:

\begin{wex}{Newton III - seat belt}
{Dineo is seated in the passenger seat of a car with the seat belt on. The car suddenly stops and he moves forwards until the seat belt stops him. Draw a labelled force diagram identifying two action-reaction pairs in this situation.
\begin{figure}[H]
\begin{center}
\scalebox{0.8} % Change this value to rescale the drawing.
{
\begin{pspicture}(0,-1.970762)(2.7868114,2.0027716)
\rput{5.194429}(-0.00510652,-0.08188147){\psellipse[linewidth=0.04,dimen=outer,fillstyle=solid](0.9,-0.09722832)(0.4,1.1)}
\pscustom[linewidth=0.2]
{
\newpath
\moveto(0.4,1.1027716)
\lineto(0.53636354,0.95277166)
\curveto(0.6045453,0.8777717)(0.76363647,0.7277717)(0.8545453,0.65277165)
\curveto(0.9454544,0.57777166)(1.0818182,0.40277168)(1.1272726,0.30277166)
\curveto(1.1727271,0.20277166)(1.2636365,-0.04722833)(1.3090909,-0.19722833)
\curveto(1.3545452,-0.34722832)(1.4,-0.5722283)(1.4,-0.79722834)
}
\psframe[linewidth=0.04,dimen=outer,fillstyle=solid,fillcolor=black](0.5,1.4027717)(0.0,-1.5972283)
\psframe[linewidth=0.04,dimen=outer,fillstyle=solid,fillcolor=black](2.1,-1.1972283)(0.5,-1.5972283)
\psellipse[linewidth=0.04,dimen=outer,fillstyle=solid](1.5,-0.9972283)(0.9,0.2)
\rput{25.769327}(-0.36379522,-1.2040873){\psellipse[linewidth=0.04,dimen=outer,fillstyle=solid](2.45,-1.3972284)(0.15,0.6)}
\psellipse[linewidth=0.04,dimen=outer,fillstyle=solid](1.0,1.4027717)(0.5,0.6)
\rput{-36.422283}(0.11872557,0.690611){\psellipse[linewidth=0.04,dimen=outer,fillstyle=solid](1.1089275,0.16487084)(0.6,0.16503632)}
\end{pspicture}
}
\end{center}
\end{figure}
}
{\westep{Draw a force diagram}
Start by drawing the picture. You will be using arrows to indicate the forces so make your picture large enough so that detailed labels can also be added. The picture needs to be accurate, but not artistic! Use stickmen if you have to.

\westep{Label the diagram}
Take one pair at a time and label them carefully. If there is not enough space on the drawing, then use a key on the side.
\begin{figure}[H]
\begin{center}
\scalebox{1} % Change this value to rescale the drawing.
{
\begin{pspicture}(0,-2.22)(12.425,2.2)
\rput{-12.697448}(0.02445566,0.21980275){\psellipse[linewidth=0.04,dimen=outer,fillstyle=solid](1.0,0.0)(0.4,1.1)}
\pscustom[linewidth=0.2]
{
\newpath
\moveto(0.4,1.3)
\lineto(0.57634735,1.1049185)
\curveto(0.66452086,1.007378)(0.8833988,0.8035562)(1.0141028,0.6972751)
\curveto(1.1448069,0.59099394)(1.3080151,0.37213928)(1.3405191,0.25956604)
\curveto(1.373023,0.14699249)(1.4117529,-0.12570068)(1.4179788,-0.2858203)
\curveto(1.4242047,-0.44593993)(1.3910129,-0.67737913)(1.2727604,-0.8913376)
}
\psframe[linewidth=0.04,dimen=outer,fillstyle=solid,fillcolor=black](0.5,1.4)(0.0,-1.6)
\psframe[linewidth=0.04,dimen=outer,fillstyle=solid,fillcolor=black](2.1,-1.2)(0.5,-1.6)
\psellipse[linewidth=0.04,dimen=outer,fillstyle=solid](1.5,-1.0)(0.9,0.2)
\rput{25.769327}(-0.3650002,-1.2043629){\psellipse[linewidth=0.04,dimen=outer,fillstyle=solid](2.45,-1.4)(0.15,0.6)}
\psellipse[linewidth=0.04,dimen=outer,fillstyle=solid](1.4,1.6)(0.5,0.6)
\rput{-99.886314}(1.1396315,1.282391){\psellipse[linewidth=0.04,dimen=outer,fillstyle=solid](1.1089275,0.16209917)(0.6,0.16503632)}
\psline[linewidth=0.04cm,arrowsize=0.05291667cm 2.0,arrowlength=1.4,arrowinset=0.4]{<->}(0.5,0.2)(2.4,0.2)
\psline[linewidth=0.04cm,arrowsize=0.05291667cm 2.0,arrowlength=1.4,arrowinset=0.4]{<->}(1.6,-0.1)(1.6,-2.2)
\psdots[dotsize=0.18](1.5,0.2)
\psdots[dotsize=0.18](1.6,-1.1)
\usefont{T1}{ptm}{m}{n}
\rput(7.757344,1.51){F$_{1}$: The force of Dineo on the seat belt}
\usefont{T1}{ptm}{m}{n}
\rput(2.3279688,0.41){F$_{1}$}
\usefont{T1}{ptm}{m}{n}
\rput(1.2279687,0.41){F$_{2}$}
\usefont{T1}{ptm}{m}{n}
\rput(7.750781,1.11){F$_{2}$: The force of the seat belt on Dineo}
\usefont{T1}{ptm}{m}{n}
\rput(8.439531,0.71){F$_{3}$: The force of Dineo on the seat (downwards)}
\usefont{T1}{ptm}{m}{n}
\rput(1.9279687,-0.59){F$_{4}$}
\usefont{T1}{ptm}{m}{n}
\rput(1.9279687,-1.89){F$_{3}$}
\usefont{T1}{ptm}{m}{n}
\rput(8.219531,0.31){F$_{4}$: The force of the seat on Dineo (upwards)}
\end{pspicture}
}
\end{center}
\end{figure}

}
\end{wex}

\begin{wex}{Newton III - forces in a lift}
{Tammy travels from the ground floor to the fifth floor of a hotel in a lift. Which ONE of the following statements is TRUE about the force exerted by the floor of the lift on Tammy's feet?
\begin{itemize}
\item[A] It is greater than the magnitude of Tammy's weight.
\item[B] It is equal in magnitude to the force Tammy's feet exert on the floor.
\item[C] It is equal to what it would be in  a stationary lift.
\item[D] It is greater than what it would be in a stationary lift.
\end{itemize}}
{\westep{Analyse the situation}
This is a Newton's Third Law question and not Newton II. We need to focus on the action-reaction pairs of forces and not the motion of the lift. The following diagram will show the action-reaction pairs that are present when a person is standing on a scale in a lift.

\begin{figure}[H]
\begin{center}
\scalebox{1} % Change this value to rescale the drawing.
{
\begin{pspicture}(0,-2.230716)(10.715943,2.230716)
\definecolor{color0b}{rgb}{0.6,0.6,0.6}
\rput{90.219284}(1.439524,-2.8729498){\psframe[linewidth=0.04,dimen=outer,fillstyle=solid,fillcolor=color0b](2.3507497,1.4305377)(1.9507498,-2.8694623)}
\psline[linewidth=0.06cm,arrowsize=0.05291667cm 2.0,arrowlength=1.4,arrowinset=0.4]{->}(1.8191447,-0.50072837)(1.8256509,-2.2007158)
\usefont{T1}{ptm}{m}{n}
\rput(7.110475,1.3880199){F$_{1}$: force of feet on lift (downwards)}
\psline[linewidth=0.06cm,arrowsize=0.05291667cm 2.0,arrowlength=1.4,arrowinset=0.4]{->}(1.7992214,-0.52080476)(1.7930979,1.0791836)
\psline[linewidth=0.02cm](1.6965431,0.17880741)(1.8965416,0.17957285)
\psline[linewidth=0.02cm](1.6961603,0.2788067)(1.8961588,0.27957213)
\psline[linewidth=0.02cm](1.7030493,-1.5211802)(1.9030478,-1.5204147)
\psline[linewidth=0.02cm](1.7026666,-1.4211808)(1.9026651,-1.4204154)
\psline[linewidth=0.02cm,arrowsize=0.05291667cm 2.0,arrowlength=1.4,arrowinset=0.4]{->}(0.7984632,-0.32463342)(1.1709437,-0.68198013)
\psline[linewidth=0.06cm,arrowsize=0.05291667cm 2.0,arrowlength=1.4,arrowinset=0.4]{->}(2.179907,0.500804)(2.1864135,-1.1991836)
\psline[linewidth=0.06cm,arrowsize=0.05291667cm 2.0,arrowlength=1.4,arrowinset=0.4]{->}(2.1799839,0.6007276)(2.1738603,2.2007158)
\psline[linewidth=0.02cm](2.1241946,-0.05964705)(2.324193,-0.05888161)
\psline[linewidth=0.02cm](2.0965402,1.5203383)(2.2965386,1.5211037)
\psdots[dotsize=0.18,dotangle=90.219284](2.1799839,0.58072764)
\psdots[dotsize=0.18,dotangle=90.219284](1.7992214,-0.52080476)
\usefont{T1}{ptm}{m}{n}
\rput(6.890475,0.9880199){F$_{2}$: force of lift on feet (upwards)}
\usefont{T1}{ptm}{m}{n}
\rput(7.350475,0.58801985){F$_{3}$: force of Earth on person (downwards)}
\usefont{T1}{ptm}{m}{n}
\rput(7.130475,0.18801987){F$_{4}$: force of person on Earth (upwards)}
\usefont{T1}{ptm}{m}{n}
\rput(0.5759437,-0.15198013){lift}
\psline[linewidth=0.04cm](2.0709438,0.11801987)(1.7709438,-0.5819801)
\psline[linewidth=0.04cm](2.0709438,0.11801987)(2.2709436,-0.48198012)
\psline[linewidth=0.04cm](2.2709436,-0.48198012)(2.3709438,-0.48198012)
\psline[linewidth=0.04cm](2.0709438,0.11801987)(2.0709438,1.0180199)
\psline[linewidth=0.04cm](1.7709438,0.71801984)(2.0709438,0.6180199)
\psline[linewidth=0.04cm](2.0709438,0.6180199)(2.3709438,0.81801987)
\psellipse[linewidth=0.04,dimen=outer](2.0709438,1.3180199)(0.2,0.3)
\usefont{T1}{ptm}{m}{n}
\rput(1.3989124,-1.3719801){F$_{1}$}
\usefont{T1}{ptm}{m}{n}
\rput(1.3989124,0.52801985){F$_{2}$}
\usefont{T1}{ptm}{m}{n}
\rput(2.6989124,1.8280199){F$_{4}$}
\usefont{T1}{ptm}{m}{n}
\rput(2.6989124,-0.17198013){F$_{3}$}
\end{pspicture}
}
\end{center}
\caption{Newton's action-reaction pairs in a lift}
\end{figure}

In this question statements are made about the force of the floor (lift) on Tammy's feet. This force corresponds to F$_{2}$ in our diagram. The reaction force that pairs up with this one is F$_{1}$, which is the force that Tammy's feet exerts on the floor of the lift. The magnitude of these two forces are the same, but they act in opposite directions.\\
\westep{Choose the correct answer}
It is important to analyse the question first, before looking at the answers as the answers might confuse you. Make sure that you understand the situation and know what is asked before you look at the options.\\
The correct answer is B.\\
}
\end{wex}


\begin{wex}{Newton III - book and wall}
{Tumi presses a book against a vertical wall as shown in the sketch.
\begin{enumerate}
\item Draw a labelled force diagram indicating all the forces acting on the book.
\item State, in words, Newton's Third Law of Motion.
\item Name the action-reaction pairs of forces acting in the horizontal plane.
\end{enumerate}
\begin{figure}[H]
\begin{center}
\scalebox{1} % Change this value to rescale the drawing.
{
\begin{pspicture}(0,-1.3)(1.4,1.3)
\definecolor{color381b}{rgb}{0.8,0.8,0.8}
\psframe[linewidth=0.04,dimen=outer,doubleline=true,doublesep=0.12](1.4,1.3)(1.1,-1.3)
\psframe[linewidth=0.04,dimen=outer,fillstyle=solid,fillcolor=color381b](1.1,0.6)(0.9,-0.1)
\psline[linewidth=0.04cm,arrowsize=0.05291667cm 2.0,arrowlength=1.4,arrowinset=0.4]{->}(0.0,0.3)(0.9,0.3)
\end{pspicture}
}
\end{center}
\end{figure}}
{\westep{Draw a force diagram}
A force diagram will look like this:
\begin{figure}[H]
\begin{center}
\scalebox{1} % Change this value to rescale the drawing.
{
\begin{pspicture}(0,-2.77)(11.919375,2.77)
\definecolor{color381b}{rgb}{0.8,0.8,0.8}
\psframe[linewidth=0.04,dimen=outer,fillstyle=solid,fillcolor=color381b](4.9971876,1.05)(4.2971873,-1.05)
\psline[linewidth=0.04cm,arrowsize=0.05291667cm 2.0,arrowlength=1.4,arrowinset=0.4]{->}(2.7971876,0.05)(4.2971873,0.05)
\psline[linewidth=0.04cm,arrowsize=0.05291667cm 2.0,arrowlength=1.4,arrowinset=0.4]{->}(4.5971875,-1.05)(4.5971875,-2.75)
\psline[linewidth=0.04cm,arrowsize=0.05291667cm 2.0,arrowlength=1.4,arrowinset=0.4]{->}(4.5971875,1.05)(4.5971875,2.75)
\psline[linewidth=0.04cm,arrowsize=0.05291667cm 2.0,arrowlength=1.4,arrowinset=0.4]{->}(6.3971877,0.05)(4.9971876,0.05)
\usefont{T1}{ptm}{m}{n}
\rput(7.7273436,2.06){Upwards frictional force of wall on book}
\usefont{T1}{ptm}{m}{n}
\rput(8.385157,-1.64){Downwards gravitational force of Earth on book}
\usefont{T1}{ptm}{m}{n}
\rput(6.9853125,0.36){Force of wall on book}
\usefont{T1}{ptm}{m}{n}
\rput(2.106875,0.36){Applied force on girl on book}
\end{pspicture}
}
\end{center}
\end{figure}

Note that we had to draw all the forces acting on the book and not the action-reaction pairs. None of the forces drawn are action-reaction pairs, because they all act on the same object (the book). When you label forces, be as specific as possible, including the direction of the force and both objects involved, for example, do not say gravity (which is an incomplete answer) but rather say 'Downward (direction) gravitational force of the Earth (object) on the book (object)'. \\
\westep{State Newton's Third Law}
If body A exerts a force onto body B, then body B will exert a force equal in magnitude, but opposite in direction, onto body A.\\
\westep{Name the action-reaction pairs}
The question only asks for action-reaction forces in the horizontal plane. Therefore:\\
Pair 1: Action: Applied force of the girl on the book; Reaction: The force of the book on the girl.\\
Pair 2: Action: Force of the book on the wall; Reaction: Force of the wall on the book.\\
Note that a Newton III pair will always involve the same combination of words, like 'book on wall' and 'wall on book'. The objects are 'swapped around' in naming the pairs.\\
}
\end{wex}

\begin{g_experiment}{Balloon Rocket}{
\textbf{Aim: } In this experiment for the entire class, you will use a balloon rocket to investigate Newton's Third Law. A fishing line will be used as a track and a plastic straw taped to the balloon will help attach the balloon to the track.\\
\textbf{Apparatus: } You will need the following items for this experiment:
\begin{enumerate}
\item{balloons (one for each team)}
\item{plastic straws (one for each team)}
\item{tape (cellophane or masking)}
\item{fishing line, 10 meters in length}
\item{a stopwatch - optional (a cell phone can also be used)}
\item{a measuring tape - optional}
\end{enumerate}
\textbf{Method: }
\begin{enumerate}
\item Divide into groups of at least five.
\item Attach one end of the fishing line to the blackboard with tape. Have one teammate hold the other end of the fishing line so that it is taut and roughly horizontal. The line must be held steady and \textbf{must not} be moved up or down during the experiment.
\item Have one teammate blow up a balloon and hold it shut with his or her fingers. Have another teammate tape the straw along the side of the balloon. Thread the fishing line through the straw and hold the balloon at the far end of the line.
\item Let go of the rocket and observe how the rocket moves forward.
\item Optionally, the rockets of each group can be timed to determine a winner of the fastest rocket.
\begin{enumerate}
\item Assign one teammate to time the event. The balloon should be let go when the time keeper yells ``Go!'' Observe how your rocket moves toward the blackboard.
\item Have another teammate stand right next to the blackboard and yell ``Stop!'' when the rocket hits its target. If the balloon does not make it all the way to the blackboard, ``Stop!'' should be called when the balloon stops moving. The timekeeper should record the flight time.
\item Measure the exact distance the rocket travelled. Calculate the average speed at which the balloon travelled. To do this, divide the distance travelled by the time the balloon was ``in flight.'' Fill in your results for Trial 1 in the Table below.
\item Each team should conduct two more races and complete the sections in the Table for Trials 2 and 3. Then calculate the average speed for the three trials to determine your team's race entry time.
\end{enumerate}
\end{enumerate}
\textbf{Results: }
\begin{center}
\begin{tabular}{|c|c|c|c|}\hline\
& \textbf{Distance (m)} & \textbf{Time (s)} & \textbf{Speed (m$\cdot$s$^{-1}$)}\\\hline
Trial 1 &&&\\\hline
Trial 2 &&&\\\hline
Trial 3 &&&\\\hline
&& Average: & \\\hline
\end{tabular}
\end{center}
\textbf{Conclusions: }
The winner of this race is the team with the fastest average balloon speed.
}
\end{g_experiment}

While doing the experiment, you should think about,
\begin{enumerate}
\item{What made your rocket move?}
\item{How is Newton's Third Law of Motion demonstrated by this activity?}
\item{Draw pictures using labelled arrows to show the forces acting on the inside of the balloon before it was released and after it was released.}
\end{enumerate}



\Exercise{title}{
\begin{enumerate}
\item {A fly hits the front windscreen of a moving car. Compared to the magnitude of the force the fly exerts on the windscreen, the magnitude of the force the windscreen exerts on the fly during the collision, is:
\begin{enumerate}
\item zero.
\item smaller, but not zero.
\item bigger.
\item the same.
\end{enumerate}}%answer D

%\item {A log of wood is attached to a cart by means of a light, inelastic rope. A horse pulls the cart along a rough, horizontal road with an applied force F. The total system accelerates initally with an acceleration of magnitude a (figure 1). The forces acting on the cart during the acceleration, are indicated in Figure 2.

%
%\scalebox{0.3} % Change this value to rescale the drawing.
%{
%\begin{pspicture}(0,-5.64)(10.58,5.64)
%\definecolor{color551}{rgb}{0.8,0.0,0.2}
%\definecolor{color254}{rgb}{0.4,0.4,0.4}
%\definecolor{color216}{rgb}{0.6,0.6,0.6}
%\definecolor{color520}{rgb}{0.8,0.8,0.8}
%\psbezier[linewidth=0.04](1.34,4.7)(1.34,5.24)(0.82,5.52)(1.64,4.9)
%\psbezier[linewidth=0.04](1.74,4.54)(1.92,5.1)(1.42,5.62)(2.12,4.68)
%\psbezier[linewidth=0.04,linecolor=color254](8.96,2.56)(9.22,2.92)(9.88,2.66)(9.86,1.74)(9.84,0.82)(10.56,-1.7)(10.2,-2.12)
%\psbezier[linewidth=0.04,linecolor=color254](9.34,1.78)(9.66,1.72)(9.52,0.08)(9.44,-0.36)
%\psline[linewidth=0.04cm,linecolor=color551](8.36,-5.3)(8.36,-5.3)
%\psline[linewidth=0.04cm,linecolor=color551](8.56,-5.3)(8.56,-5.3)
%\psbezier[linewidth=0.04](2.0,4.84)(2.26,4.68)(2.16,4.52)(2.38,4.36)(2.6,4.2)(2.84,4.18)(3.46,3.62)(4.08,3.06)(4.24,2.7)(4.34,2.66)(4.44,2.62)(4.78,2.52)(5.02,2.48)(5.26,2.44)(5.56,2.44)(5.68,2.46)(5.8,2.48)(6.52,2.8)(7.36,2.82)(8.2,2.84)(8.86,2.66)(9.16,2.1)(9.46,1.54)(9.32,1.4)(9.24,0.52)(9.16,-0.36)(9.76,-0.68)(9.8,-0.78)(9.84,-0.88)(9.68,-1.92)(9.74,-2.64)(9.8,-3.36)(9.78,-3.74)(9.96,-4.36)(10.14,-4.98)(9.76,-4.22)(9.82,-5.24)
%\psbezier[linewidth=0.04](7.42,0.4)(7.92,0.3)(7.72,0.56)(7.94,0.18)(8.16,-0.2)(8.28,-0.26)(8.54,-0.6)(8.8,-0.94)(9.14,-1.04)(9.28,-1.56)(9.42,-2.08)(9.52,-3.58)(9.44,-4.16)(9.36,-4.74)(9.28,-4.46)(9.12,-5.22)
%\psbezier[linewidth=0.04](4.68,-0.54)(6.88,-0.98)(7.42,-0.12)(7.7,-0.08)
%\psbezier[linewidth=0.04](7.36,-0.32)(7.84,-0.86)(7.86,-0.9)(8.1,-1.06)(8.34,-1.22)(8.24,-1.96)(8.32,-2.7)(8.4,-3.44)(8.36,-3.88)(8.32,-4.28)(8.28,-4.68)(8.1,-4.72)(8.06,-5.08)
%\psbezier[linewidth=0.04](8.9,-0.98)(8.74,-1.82)(8.822478,-2.1607041)(8.8,-3.1)(8.777521,-4.0392957)(9.02,-4.44)(8.92,-4.5)(8.82,-4.56)(8.66,-4.52)(8.8,-5.14)
%\psbezier[linewidth=0.04](4.98,-0.14)(4.66,-0.38)(4.6,-0.68)(4.58,-1.3)(4.56,-1.92)(4.58,-2.88)(4.52,-3.46)(4.46,-4.04)(4.6,-4.76)(4.6,-4.78)(4.6,-4.8)(4.66,-4.96)(4.54,-4.94)(4.42,-4.92)(4.44,-5.1)(4.44,-5.44)
%\psbezier[linewidth=0.04](3.44,0.12)(3.58,-0.56)(3.53404,-0.32143956)(3.74,-1.3)(3.94596,-2.2785604)(3.86,-1.8)(3.98,-2.52)(4.1,-3.24)(4.10836,-4.591332)(4.1,-4.7)(4.09164,-4.8086677)(4.06,-4.7)(3.76,-5.36)
%\psbezier[linewidth=0.04](2.84,0.74)(3.06,0.08)(3.3,-0.04)(3.48,-0.18)
%\psbezier[linewidth=0.04](3.7,-1.12)(3.58,-1.82)(3.4859843,-1.6000179)(3.5,-2.74)(3.5140157,-3.879982)(3.62,-4.66)(3.64,-4.78)(3.66,-4.9)(3.5,-4.82)(3.48,-4.94)(3.46,-5.06)(3.44,-5.12)(3.44,-5.32)
%\psbezier[linewidth=0.04](2.1,3.34)(2.3,2.46)(2.68,1.9)(2.62,1.76)(2.56,1.62)(2.34,1.2)(2.56,0.68)(2.78,0.16)(2.7,0.22)(2.76,-0.76)(2.82,-1.74)(3.0141456,-2.0079513)(3.04,-2.76)(3.0658543,-3.5120487)(3.14,-4.4)(3.12,-4.6)(3.1,-4.8)(2.92,-4.76)(2.7,-5.32)
%\psbezier[linewidth=0.04](2.36,3.56)(1.78,3.02)(1.38,3.2)(1.36,3.2)(1.34,3.2)(1.12,2.98)(0.96,2.86)(0.8,2.74)(0.82,2.44)(0.28,2.54)
%\psbezier[linewidth=0.04,linecolor=color254](1.28,4.7)(1.08,4.62)(0.66,4.3)(0.8,4.36)(0.94,4.42)(1.36,4.46)(1.44,4.46)
%\psbezier[linewidth=0.04](0.96,4.38)(0.96,3.66)(0.08,3.42)(0.06,3.14)(0.04,2.86)(0.12,2.86)(0.28,2.56)
%\psbezier[linewidth=0.04](0.28,2.56)(0.5,2.6)(0.68,2.74)(0.74,2.82)
%\psdots[dotsize=0.12](0.4,3.12)
%\psline[linewidth=0.04cm](0.38,3.14)(0.36,3.02)
%\psline[linewidth=0.04cm](0.04,3.14)(0.0,3.06)
%\psline[linewidth=0.04cm](0.0,3.06)(0.04,3.0)
%\psline[linewidth=0.04cm](0.4,3.1)(0.36,3.02)
%\psbezier[linewidth=0.04](1.28,3.98)(1.56,4.42)(1.68,3.86)(1.62,4.26)
%\psdots[dotsize=0.12](1.46,4.12)
%\psdots[dotsize=0.12](1.52,4.1)
%\psdots[dotsize=0.12](1.48,4.06)
%\psbezier[linewidth=0.04,linecolor=color216,linestyle=dotted,dotsep=0.16cm](7.32,2.04)(7.42,1.28)(7.54,1.04)(7.8,0.4)
%\psbezier[linewidth=0.04,linecolor=color216,linestyle=dotted,dotsep=0.16cm](3.44,0.14)(3.2,0.64)(2.8,1.38)(3.56,2.24)
%\psbezier[linewidth=0.04,linecolor=color254](2.24,4.46)(2.32,4.1)(2.86,3.48)(2.72,3.28)(2.58,3.08)(3.04,3.66)(3.04,3.64)(3.04,3.62)(3.313018,2.6812873)(3.24,2.52)(3.1669817,2.358713)(3.54,3.06)(3.54,3.06)(3.54,3.06)(3.6540635,1.8271202)(3.62,1.66)(3.5859365,1.4928797)(4.02,2.38)(4.0,2.34)(3.98,2.3)(4.38,2.24)(4.58,1.98)(4.78,1.72)(4.26,2.68)(4.24,2.7)
%\psline[linewidth=0.04cm,linecolor=color520,linestyle=dotted,dotsep=0.16cm](7.52,-0.3)(7.84,-0.6)
%\psline[linewidth=0.04cm,linecolor=color520,linestyle=dotted,dotsep=0.16cm](7.7,-0.3)(8.1,-0.74)
%\psline[linewidth=0.04cm,linecolor=color520,linestyle=dotted,dotsep=0.16cm](7.68,-0.22)(7.92,-0.82)
%\psline[linewidth=0.04cm,linecolor=color520,linestyle=dotted,dotsep=0.16cm](7.76,-0.66)(8.24,-1.02)
%\psline[linewidth=0.04cm,linecolor=color520,linestyle=dotted,dotsep=0.16cm](3.64,3.42)(3.72,2.22)
%\psline[linewidth=0.04cm,linecolor=color520,linestyle=dotted,dotsep=0.16cm](3.76,3.16)(3.84,2.36)
%\psline[linewidth=0.04cm,linecolor=color520,linestyle=dotted,dotsep=0.16cm](3.88,2.98)(3.96,2.58)
%\psline[linewidth=0.04cm,linecolor=color520,linestyle=dotted,dotsep=0.16cm](4.04,2.74)(4.16,2.44)
%\psline[linewidth=0.04cm,linecolor=color520,linestyle=dotted,dotsep=0.16cm](3.16,3.62)(3.3,3.1)
%\psline[linewidth=0.04cm,linecolor=color520,linestyle=dotted,dotsep=0.16cm](3.24,3.62)(3.44,3.24)
%\psline[linewidth=0.04cm,linecolor=color520,linestyle=dotted,dotsep=0.16cm](3.4,3.5)(3.48,3.18)
%\psline[linewidth=0.04cm,linecolor=color520,linestyle=dotted,dotsep=0.16cm](3.4,3.24)(3.36,2.86)
%\psline[linewidth=0.04cm,linecolor=color520,linestyle=dotted,dotsep=0.16cm](2.56,4.1)(2.64,3.86)
%\psline[linewidth=0.04cm,linecolor=color520,linestyle=dotted,dotsep=0.16cm](2.72,3.88)(2.82,3.72)
%\psline[linewidth=0.04cm,linecolor=color520,linestyle=dotted,dotsep=0.16cm](2.74,3.98)(3.04,3.78)
%\psline[linewidth=0.04cm,linecolor=color520,linestyle=dotted,dotsep=0.16cm](2.82,3.8)(2.8,3.62)
%\psline[linewidth=0.04cm,linecolor=color520,linestyle=dotted,dotsep=0.16cm](1.24,4.54)(1.6,4.78)
%\psline[linewidth=0.04cm,linecolor=color520,linestyle=dotted,dotsep=0.16cm](1.46,4.6)(1.62,4.68)
%\psline[linewidth=0.04cm,linecolor=color520,linestyle=dotted,dotsep=0.16cm](1.2,4.58)(1.62,4.7)
%\psline[linewidth=0.04cm,linecolor=color520,linestyle=dotted,dotsep=0.16cm](1.44,4.6)(1.56,4.84)
%\psline[linewidth=0.04cm,linecolor=color520,linestyle=dotted,dotsep=0.16cm](2.44,4.28)(2.66,3.94)
%\psline[linewidth=0.04cm,linecolor=color520,linestyle=dotted,dotsep=0.16cm](2.42,4.2)(2.88,3.88)
%\psline[linewidth=0.04cm,linecolor=color520,linestyle=dotted,dotsep=0.16cm](2.56,4.12)(3.08,3.7)
%\psline[linewidth=0.04cm,linecolor=color520,linestyle=dotted,dotsep=0.16cm](2.84,3.86)(2.84,3.62)
%\psline[linewidth=0.04cm,linecolor=color520,linestyle=dotted,dotsep=0.16cm](2.74,3.86)(2.76,3.56)
%\psline[linewidth=0.04cm,linecolor=color520,linestyle=dotted,dotsep=0.16cm](2.88,3.88)(2.88,3.58)
%\psline[linewidth=0.04cm,linecolor=color520,linestyle=dotted,dotsep=0.16cm](2.84,3.94)(3.28,3.48)
%\psline[linewidth=0.04cm,linecolor=color520,linestyle=dotted,dotsep=0.16cm](3.0,3.8)(3.36,3.48)
%\psline[linewidth=0.04cm,linecolor=color520,linestyle=dotted,dotsep=0.16cm](3.28,3.5)(3.32,2.9)
%\psline[linewidth=0.04cm,linecolor=color520,linestyle=dotted,dotsep=0.16cm](3.72,3.18)(3.86,2.54)
%\psline[linewidth=0.04cm,linecolor=color520,linestyle=dotted,dotsep=0.16cm](3.62,3.08)(3.84,2.44)
%\psline[linewidth=0.04cm,linecolor=color520,linestyle=dotted,dotsep=0.16cm](4.0,2.84)(3.7,2.12)
%\psline[linewidth=0.04cm,linecolor=color520,linestyle=dotted,dotsep=0.16cm](4.0,2.66)(3.72,2.02)
%\psline[linewidth=0.04cm,linecolor=color520,linestyle=dotted,dotsep=0.16cm](3.76,2.66)(4.24,2.28)
%\psline[linewidth=0.04cm,linecolor=color520,linestyle=dotted,dotsep=0.16cm](3.94,2.84)(4.26,2.3)
%\psline[linewidth=0.04cm,linecolor=color520,linestyle=dotted,dotsep=0.16cm](3.92,3.0)(4.16,2.52)
%\psline[linewidth=0.04cm,linecolor=color520,linestyle=dotted,dotsep=0.16cm](10.0,-0.76)(10.0,-2.02)
%\psline[linewidth=0.04cm,linecolor=color520,linestyle=dotted,dotsep=0.16cm](9.94,-0.52)(10.18,-1.72)
%\psline[linewidth=0.04cm,linecolor=color520,linestyle=dotted,dotsep=0.16cm](9.8,-0.02)(10.02,-0.6)
%\psline[linewidth=0.04cm,linecolor=color520,linestyle=dotted,dotsep=0.16cm](9.72,0.1)(9.76,-0.44)
%\psline[linewidth=0.04cm,linecolor=color520,linestyle=dotted,dotsep=0.16cm](9.68,-0.3)(9.8,-0.58)
%\psline[linewidth=0.04cm,linecolor=color520,linestyle=dotted,dotsep=0.16cm](9.64,-0.36)(9.84,-0.6)
%\psline[linewidth=0.04cm,linecolor=color520,linestyle=dotted,dotsep=0.16cm](10.0,0.22)(10.12,-0.76)
%\psline[linewidth=0.04cm,linecolor=color520,linestyle=dotted,dotsep=0.16cm](9.92,-0.06)(10.16,-1.06)
%\psline[linewidth=0.04cm,linecolor=color520,linestyle=dotted,dotsep=0.16cm](10.12,-1.0)(10.16,-1.9)
%\psline[linewidth=0.04cm,linecolor=color520,linestyle=dotted,dotsep=0.16cm](10.18,-1.56)(10.0,-1.98)
%\psline[linewidth=0.04cm,linecolor=color520,linestyle=dotted,dotsep=0.16cm](10.16,-1.8)(9.96,-2.1)
%\psline[linewidth=0.04cm,linecolor=color520,linestyle=dotted,dotsep=0.16cm](9.94,-1.16)(9.84,-1.86)
%\psline[linewidth=0.04cm,linecolor=color520,linestyle=dotted,dotsep=0.16cm](9.86,-0.7)(9.88,-1.38)
%\psline[linewidth=0.04cm,linecolor=color520,linestyle=dotted,dotsep=0.16cm](10.02,-0.44)(9.92,-1.34)
%\psline[linewidth=0.04cm,linecolor=color520,linestyle=dotted,dotsep=0.16cm](10.12,-1.16)(10.0,-1.94)
%\psline[linewidth=0.04cm,linecolor=color520,linestyle=dotted,dotsep=0.16cm](9.96,-1.54)(9.94,-1.98)
%\psline[linewidth=0.04cm,linecolor=color520,linestyle=dotted,dotsep=0.16cm](10.1,-1.46)(10.1,-1.94)
%\psline[linewidth=0.04cm,linecolor=color520,linestyle=dotted,dotsep=0.16cm](9.92,-1.48)(9.86,-1.98)
%\psline[linewidth=0.04cm,linecolor=color520,linestyle=dotted,dotsep=0.16cm](10.16,-1.72)(10.16,-2.02)
%\psline[linewidth=0.04cm,linecolor=color520,linestyle=dotted,dotsep=0.16cm](9.92,-1.96)(9.92,-2.22)
%\psline[linewidth=0.04cm,linecolor=color520,linestyle=dotted,dotsep=0.16cm](3.86,2.6)(3.68,1.98)
%\psline[linewidth=0.04cm,linecolor=color520,linestyle=dotted,dotsep=0.16cm](3.88,2.26)(3.7,1.9)
%\psline[linewidth=0.04cm,linecolor=color520,linestyle=dotted,dotsep=0.16cm](3.62,2.44)(3.7,2.06)
%\psline[linewidth=0.04cm,linecolor=color520,linestyle=dotted,dotsep=0.16cm](3.78,2.42)(3.72,1.88)
%\psline[linewidth=0.04cm,linecolor=color520,linestyle=dotted,dotsep=0.16cm](3.88,2.34)(3.72,2.04)
%\psline[linewidth=0.04cm,linecolor=color520,linestyle=dotted,dotsep=0.16cm](4.18,2.38)(4.26,2.22)
%\psline[linewidth=0.04cm,linecolor=color520,linestyle=dotted,dotsep=0.16cm](4.24,2.54)(4.42,2.2)
%\psline[linewidth=0.04cm,linecolor=color520,linestyle=dotted,dotsep=0.16cm](4.28,2.5)(4.42,2.18)
%\psline[linewidth=0.04cm,linecolor=color520,linestyle=dotted,dotsep=0.16cm](4.26,2.5)(4.24,2.18)
%\psline[linewidth=0.04cm,linecolor=color520,linestyle=dotted,dotsep=0.16cm](3.36,3.1)(3.32,2.84)
%\psline[linewidth=0.04cm,linecolor=color520,linestyle=dotted,dotsep=0.16cm](3.28,3.14)(3.28,2.82)
%\psline[linewidth=0.04cm,linecolor=color520,linestyle=dotted,dotsep=0.16cm](3.46,3.22)(3.3,2.86)
%\psline[linewidth=0.04cm,linecolor=color520,linestyle=dotted,dotsep=0.16cm](3.3,3.14)(3.3,2.84)
%\psline[linewidth=0.04cm,linecolor=color520,linestyle=dotted,dotsep=0.16cm](2.76,3.7)(2.8,3.5)
%\psline[linewidth=0.04cm,linecolor=color520,linestyle=dotted,dotsep=0.16cm](2.74,-0.06)(2.84,-0.3)
%\psline[linewidth=0.04cm,linecolor=color216,linestyle=dotted,dotsep=0.16cm](2.9,-0.18)(3.12,-0.52)
%\psline[linewidth=0.04cm,linecolor=color216,linestyle=dotted,dotsep=0.16cm](3.0,-0.44)(3.32,-0.7)
%\psline[linewidth=0.04cm,linecolor=color216,linestyle=dotted,dotsep=0.16cm](3.3,-0.5)(3.48,-0.54)
%\psline[linewidth=0.04cm,linecolor=color216,linestyle=dotted,dotsep=0.16cm](3.38,-0.66)(3.56,-0.7)
%\psline[linewidth=0.04cm,linecolor=color216,linestyle=dotted,dotsep=0.16cm](3.38,-0.78)(3.54,-0.98)
%\psline[linewidth=0.04cm,linecolor=color520,linestyle=dotted,dotsep=0.16cm](3.46,-0.7)(3.54,-1.02)
%\psline[linewidth=0.04cm,linecolor=color216,linestyle=dotted,dotsep=0.16cm](3.38,-0.84)(3.46,-1.18)
%\psline[linewidth=0.04cm,linecolor=color216,linestyle=dotted,dotsep=0.16cm](3.3,-0.82)(3.44,-1.38)
%\psline[linewidth=0.04cm,linecolor=color520,linestyle=dotted,dotsep=0.16cm](3.6,-1.02)(3.46,-1.4)
%\psline[linewidth=0.04cm,linecolor=color216,linestyle=dotted,dotsep=0.16cm](3.52,-1.0)(3.44,-1.46)
%\psline[linewidth=0.04cm,linecolor=color216,linestyle=dotted,dotsep=0.16cm](3.44,-1.06)(3.44,-1.58)
%\psline[linewidth=0.04cm,linecolor=color520,linestyle=dotted,dotsep=0.16cm](3.54,-1.26)(3.46,-1.54)
%\psline[linewidth=0.04cm,linecolor=color216,linestyle=dotted,dotsep=0.16cm](3.4,-1.14)(3.38,-1.64)
%\psline[linewidth=0.04cm,linecolor=color216,linestyle=dotted,dotsep=0.16cm](3.44,-1.56)(3.36,-1.96)
%\psline[linewidth=0.04cm,linecolor=color216,linestyle=dotted,dotsep=0.16cm](3.3,-2.68)(3.36,-2.9)
%\psline[linewidth=0.04cm,linecolor=color216,linestyle=dotted,dotsep=0.16cm](3.4,-2.58)(3.36,-3.02)
%\psline[linewidth=0.04cm,linecolor=color216,linestyle=dotted,dotsep=0.16cm](3.32,-2.76)(3.32,-3.1)
%\psline[linewidth=0.04cm,linecolor=color216,linestyle=dotted,dotsep=0.16cm](3.24,-2.94)(3.22,-3.14)
%\psline[linewidth=0.04cm,linecolor=color216,linestyle=dotted,dotsep=0.16cm](3.24,-2.86)(3.16,-3.06)
%\psline[linewidth=0.04cm,linecolor=color216,linestyle=dotted,dotsep=0.16cm](4.1,-2.5)(4.24,-2.68)
%\psline[linewidth=0.04cm,linecolor=color216,linestyle=dotted,dotsep=0.16cm](4.16,-2.5)(4.32,-2.68)
%\psline[linewidth=0.04cm,linecolor=color216,linestyle=dotted,dotsep=0.16cm](4.18,-2.7)(4.12,-2.9)
%\psline[linewidth=0.04cm,linecolor=color216,linestyle=dotted,dotsep=0.16cm](4.28,-2.74)(4.18,-3.0)
%\psline[linewidth=0.04cm,linecolor=color216,linestyle=dotted,dotsep=0.16cm](4.28,-2.76)(4.34,-3.02)
%\psline[linewidth=0.04cm,linecolor=color520,linestyle=dotted,dotsep=0.16cm](4.42,-2.74)(4.36,-3.0)
%\psline[linewidth=0.04cm,linecolor=color216,linestyle=dotted,dotsep=0.16cm](4.28,-2.78)(4.28,-3.02)
%\psline[linewidth=0.04cm,linecolor=color216,linestyle=dotted,dotsep=0.16cm](2.72,1.8)(2.82,1.42)
%\psline[linewidth=0.04cm,linecolor=color216,linestyle=dotted,dotsep=0.16cm](2.76,1.96)(2.96,1.5)
%\psline[linewidth=0.04cm,linecolor=color216,linestyle=dotted,dotsep=0.16cm](2.9,2.18)(2.74,1.64)
%\psline[linewidth=0.04cm,linecolor=color216,linestyle=dotted,dotsep=0.16cm](2.66,2.04)(2.8,1.78)
%\psline[linewidth=0.04cm,linecolor=color254,linestyle=dotted,dotsep=0.16cm](2.64,2.02)(2.8,1.5)
%\psline[linewidth=0.04cm,linecolor=color254,linestyle=dotted,dotsep=0.16cm](2.68,1.58)(2.92,1.96)
%\psline[linewidth=0.04cm,linecolor=color254,linestyle=dotted,dotsep=0.16cm](2.72,1.96)(3.0,1.78)
%\psline[linewidth=0.04cm,linecolor=color254,linestyle=dotted,dotsep=0.16cm](3.04,2.06)(3.14,1.8)
%\psline[linewidth=0.04cm,linecolor=color254,linestyle=dotted,dotsep=0.16cm](3.3,-0.76)(3.48,-0.92)
%\psline[linewidth=0.04cm,linecolor=color254,linestyle=dotted,dotsep=0.16cm](3.3,-0.52)(3.64,-0.62)
%\psline[linewidth=0.04cm,linecolor=color254,linestyle=dotted,dotsep=0.16cm](3.48,-0.76)(3.6,-1.1)
%\psline[linewidth=0.04cm,linecolor=color254,linestyle=dotted,dotsep=0.16cm](3.48,-0.5)(3.56,-0.92)
%\psline[linewidth=0.04cm,linecolor=color254,linestyle=dotted,dotsep=0.16cm](3.44,-0.9)(3.44,-1.26)
%\psline[linewidth=0.04cm,linecolor=color254,linestyle=dotted,dotsep=0.16cm](3.6,-1.02)(3.46,-1.62)
%\psline[linewidth=0.04cm,linecolor=color254,linestyle=dotted,dotsep=0.16cm](4.18,-2.74)(4.36,-2.76)
%\psline[linewidth=0.04cm,linecolor=color254,linestyle=dotted,dotsep=0.16cm](4.02,-2.38)(4.24,-3.08)
%\psline[linewidth=0.04cm,linecolor=color254,linestyle=dotted,dotsep=0.16cm](4.16,-2.62)(4.36,-2.92)
%\psline[linewidth=0.04cm,linecolor=color254,linestyle=dotted,dotsep=0.16cm](3.38,-2.68)(3.46,-3.1)
%\psline[linewidth=0.04cm,linecolor=color254,linestyle=dotted,dotsep=0.16cm](3.28,-3.08)(3.38,-3.32)
%\psline[linewidth=0.04cm,linecolor=color254,linestyle=dotted,dotsep=0.16cm](4.52,-0.42)(4.32,-0.84)
%\psline[linewidth=0.04cm,linecolor=color254,linestyle=dotted,dotsep=0.16cm](4.56,-0.5)(4.44,-0.9)
%\psline[linewidth=0.04cm,linecolor=color254,linestyle=dotted,dotsep=0.16cm](4.44,-0.5)(4.48,-0.94)
%\psline[linewidth=0.04cm,linecolor=color254,linestyle=dotted,dotsep=0.16cm](3.36,0.6)(3.56,0.1)
%\psline[linewidth=0.04cm,linecolor=color254,linestyle=dotted,dotsep=0.16cm](3.54,0.36)(3.6,-0.12)
%\psline[linewidth=0.04cm,linecolor=color254,linestyle=dotted,dotsep=0.16cm](3.32,0.44)(3.7,-0.06)
%\psline[linewidth=0.04cm,linecolor=color254,linestyle=dotted,dotsep=0.16cm](3.64,0.26)(3.6,-0.46)
%\psline[linewidth=0.04cm,linecolor=color254,linestyle=dotted,dotsep=0.16cm](3.86,0.06)(3.68,-0.6)
%\psline[linewidth=0.04cm,linecolor=color254,linestyle=dotted,dotsep=0.16cm](5.44,0.62)(5.14,0.02)
%\psline[linewidth=0.04cm,linecolor=color254,linestyle=dotted,dotsep=0.16cm](5.2,0.36)(5.2,-0.04)
%\psline[linewidth=0.04cm,linecolor=color254,linestyle=dotted,dotsep=0.16cm](5.44,1.1)(5.38,0.3)
%\psline[linewidth=0.04cm,linecolor=color254,linestyle=dotted,dotsep=0.16cm](5.62,0.66)(5.38,0.14)
%\psline[linewidth=0.04cm,linecolor=color254,linestyle=dotted,dotsep=0.16cm](5.52,0.84)(5.28,-0.06)
%\psline[linewidth=0.04cm,linecolor=color254,linestyle=dotted,dotsep=0.16cm](5.16,-0.12)(4.84,-0.44)
%\psline[linewidth=0.04cm,linecolor=color254,linestyle=dotted,dotsep=0.16cm](4.74,-0.52)(5.14,-0.38)
%\psline[linewidth=0.04cm,linecolor=color254,linestyle=dotted,dotsep=0.16cm](4.76,-0.5)(4.9,-0.18)
%\psline[linewidth=0.04cm,linecolor=color254,linestyle=dotted,dotsep=0.16cm](4.84,-0.44)(5.12,-0.38)
%\psline[linewidth=0.04cm,linecolor=color254,linestyle=dotted,dotsep=0.16cm](4.92,-0.34)(5.14,-0.2)
%\psline[linewidth=0.04cm,linecolor=color254,linestyle=dotted,dotsep=0.16cm](4.88,-0.42)(5.14,-0.42)
%\psline[linewidth=0.04cm,linecolor=color254,linestyle=dotted,dotsep=0.16cm](4.98,-0.52)(5.32,-0.52)
%\psline[linewidth=0.04cm,linecolor=color254,linestyle=dotted,dotsep=0.16cm](5.08,-0.22)(5.38,-0.18)
%\psline[linewidth=0.04cm,linecolor=color254,linestyle=dotted,dotsep=0.16cm](5.04,-0.42)(5.56,-0.02)
%\psline[linewidth=0.04cm,linecolor=color254,linestyle=dotted,dotsep=0.16cm](5.16,-0.3)(5.48,0.26)
%\psline[linewidth=0.04cm,linecolor=color254,linestyle=dotted,dotsep=0.16cm](5.16,-0.46)(5.48,0.3)
%\psline[linewidth=0.04cm,linecolor=color254,linestyle=dotted,dotsep=0.16cm](5.38,-0.22)(5.52,0.12)
%\psline[linewidth=0.04cm,linecolor=color254,linestyle=dotted,dotsep=0.16cm](5.28,-0.52)(5.52,-0.1)
%\psline[linewidth=0.04cm,linecolor=color254,linestyle=dotted,dotsep=0.16cm](5.28,-0.46)(5.44,-0.02)
%\psline[linewidth=0.04cm,linecolor=color254,linestyle=dotted,dotsep=0.16cm](5.38,-0.44)(5.62,0.06)
%\psline[linewidth=0.04cm,linecolor=color254,linestyle=dotted,dotsep=0.16cm](5.48,0.04)(5.56,0.36)
%\psline[linewidth=0.04cm,linecolor=color254,linestyle=dotted,dotsep=0.16cm](7.48,0.66)(7.2,0.2)
%\psline[linewidth=0.04cm,linecolor=color254,linestyle=dotted,dotsep=0.16cm](7.54,0.62)(7.22,0.26)
%\psline[linewidth=0.04cm,linecolor=color254,linestyle=dotted,dotsep=0.16cm](7.48,0.86)(6.98,0.3)
%\psline[linewidth=0.04cm,linecolor=color254,linestyle=dotted,dotsep=0.16cm](7.3,0.36)(7.14,0.22)
%\psline[linewidth=0.04cm,linecolor=color254,linestyle=dotted,dotsep=0.16cm](7.32,0.5)(7.04,0.22)
%\psline[linewidth=0.04cm,linecolor=color254,linestyle=dotted,dotsep=0.16cm](7.32,0.66)(7.0,0.5)
%\psline[linewidth=0.04cm,linecolor=color254,linestyle=dotted,dotsep=0.16cm](7.44,0.84)(7.12,0.66)
%\psline[linewidth=0.04cm,linecolor=color254,linestyle=dotted,dotsep=0.16cm](7.32,0.2)(6.92,-0.04)
%\psline[linewidth=0.04cm,linecolor=color254,linestyle=dotted,dotsep=0.16cm](7.04,0.26)(6.84,-0.06)
%\psline[linewidth=0.04cm,linecolor=color254,linestyle=dotted,dotsep=0.16cm](7.48,0.18)(6.88,-0.3)
%\psline[linewidth=0.04cm,linecolor=color254,linestyle=dotted,dotsep=0.16cm](7.64,0.14)(7.2,-0.12)
%\psline[linewidth=0.04cm,linecolor=color254,linestyle=dotted,dotsep=0.16cm](7.62,0.34)(7.78,0.1)
%\psline[linewidth=0.04cm,linecolor=color254,linestyle=dotted,dotsep=0.16cm](7.7,0.26)(7.8,0.04)
%\psline[linewidth=0.04cm,linecolor=color254,linestyle=dotted,dotsep=0.16cm](7.56,0.28)(7.3,-0.1)
%\psline[linewidth=0.04cm,linecolor=color254,linestyle=dotted,dotsep=0.16cm](7.36,0.28)(7.06,-0.2)
%\psline[linewidth=0.04cm,linecolor=color254,linestyle=dotted,dotsep=0.16cm](6.64,-0.3)(6.96,-0.2)
%\psline[linewidth=0.04cm,linecolor=color254,linestyle=dotted,dotsep=0.16cm](6.74,-0.06)(7.0,-0.04)
%\psline[linewidth=0.04cm,linecolor=color254,linestyle=dotted,dotsep=0.16cm](7.68,-0.44)(8.02,-0.28)
%\psline[linewidth=0.04cm,linecolor=color254,linestyle=dotted,dotsep=0.16cm](7.78,-0.6)(8.12,-0.46)
%\psline[linewidth=0.04cm,linecolor=color254,linestyle=dotted,dotsep=0.16cm](7.86,-0.44)(8.12,-0.34)
%\psline[linewidth=0.04cm,linecolor=color254,linestyle=dotted,dotsep=0.16cm](7.54,-0.42)(7.92,-0.2)
%\psline[linewidth=0.04cm,linecolor=color254,linestyle=dotted,dotsep=0.16cm](7.72,-0.5)(8.18,-0.38)
%\psline[linewidth=0.04cm,linecolor=color254,linestyle=dotted,dotsep=0.16cm](8.0,-0.82)(8.34,-0.7)
%\psline[linewidth=0.04cm,linecolor=color254,linestyle=dotted,dotsep=0.16cm](8.0,-0.6)(8.32,-0.84)
%\psline[linewidth=0.04cm,linecolor=color254,linestyle=dotted,dotsep=0.16cm](7.94,-0.3)(8.36,-0.9)
%\psline[linewidth=0.04cm,linecolor=color254,linestyle=dotted,dotsep=0.16cm](8.76,-0.94)(8.48,-1.18)
%\psline[linewidth=0.04cm,linecolor=color254,linestyle=dotted,dotsep=0.16cm](8.72,-1.02)(8.34,-1.24)
%\psline[linewidth=0.04cm,linecolor=color254,linestyle=dotted,dotsep=0.16cm](8.66,-0.74)(8.34,-0.94)
%\psline[linewidth=0.04cm,linecolor=color254,linestyle=dotted,dotsep=0.16cm](8.66,-0.9)(8.34,-1.0)
%\psline[linewidth=0.04cm,linecolor=color254,linestyle=dotted,dotsep=0.16cm](8.5,-0.7)(8.12,-0.98)
%\psline[linewidth=0.04cm,linecolor=color254,linestyle=dotted,dotsep=0.16cm](8.28,-0.66)(8.16,-0.78)
%\psline[linewidth=0.04cm,linecolor=color254,linestyle=dotted,dotsep=0.16cm](8.6,-0.82)(8.34,-0.9)
%\psline[linewidth=0.04cm,linecolor=color254,linestyle=dotted,dotsep=0.16cm](8.74,-0.98)(8.2,-0.5)
%\psline[linewidth=0.04cm,linecolor=color254,linestyle=dotted,dotsep=0.16cm](8.6,-1.1)(8.36,-0.7)
%\psline[linewidth=0.04cm,linecolor=color254,linestyle=dotted,dotsep=0.16cm](8.8,-1.0)(8.58,-1.26)
%\psline[linewidth=0.04cm,linecolor=color254,linestyle=dotted,dotsep=0.16cm](9.12,0.36)(9.16,-0.14)
%\psline[linewidth=0.04cm,linecolor=color254,linestyle=dotted,dotsep=0.16cm](9.16,0.04)(9.4,-0.52)
%\psline[linewidth=0.04cm,linecolor=color254,linestyle=dotted,dotsep=0.16cm](9.22,-0.36)(9.56,-0.74)
%\psline[linewidth=0.04cm,linecolor=color254,linestyle=dotted,dotsep=0.16cm](9.62,1.3)(9.72,0.46)
%\psline[linewidth=0.04cm,linecolor=color254,linestyle=dotted,dotsep=0.16cm](9.68,1.02)(9.8,-0.02)
%\psline[linewidth=0.04cm,linecolor=color254,linestyle=dotted,dotsep=0.16cm](9.68,0.44)(9.7,-0.36)
%\psline[linewidth=0.04cm,linecolor=color254,linestyle=dotted,dotsep=0.16cm](9.78,-0.02)(10.02,-0.82)
%\psline[linewidth=0.04cm,linecolor=color254,linestyle=dotted,dotsep=0.16cm](9.94,-0.18)(10.1,-0.86)
%\psline[linewidth=0.04cm,linecolor=color254,linestyle=dotted,dotsep=0.16cm](9.68,0.04)(9.78,-0.54)
%\psline[linewidth=0.04cm,linecolor=color254,linestyle=dotted,dotsep=0.16cm](9.84,0.36)(10.1,-0.6)
%\psline[linewidth=0.04cm,linecolor=color254,linestyle=dotted,dotsep=0.16cm](9.86,0.54)(10.1,-0.54)
%\psline[linewidth=0.04cm,linecolor=color254,linestyle=dotted,dotsep=0.16cm](9.78,0.86)(9.84,-0.44)
%\psline[linewidth=0.04cm,linecolor=color254,linestyle=dotted,dotsep=0.16cm](9.76,1.18)(9.76,0.84)
%\psline[linewidth=0.04cm,linecolor=color254,linestyle=dotted,dotsep=0.16cm](9.22,2.5)(9.2,2.06)
%\psline[linewidth=0.04cm,linecolor=color254,linestyle=dotted,dotsep=0.16cm](9.06,2.44)(9.38,2.1)
%\psline[linewidth=0.04cm,linecolor=color254,linestyle=dotted,dotsep=0.16cm](8.9,1.9)(8.98,1.24)
%\psline[linewidth=0.04cm,linecolor=color254,linestyle=dotted,dotsep=0.16cm](9.08,1.7)(8.88,1.18)
%\psline[linewidth=0.04cm,linecolor=color254,linestyle=dotted,dotsep=0.16cm](8.9,1.78)(8.82,0.92)
%\psline[linewidth=0.04cm,linecolor=color254,linestyle=dotted,dotsep=0.16cm](9.04,1.16)(8.64,0.6)
%\psline[linewidth=0.04cm,linecolor=color254,linestyle=dotted,dotsep=0.16cm](8.68,1.18)(8.8,0.62)
%\psline[linewidth=0.04cm,linecolor=color254,linestyle=dotted,dotsep=0.16cm](8.8,1.26)(8.84,0.66)
%\psline[linewidth=0.04cm,linecolor=color254,linestyle=dotted,dotsep=0.16cm](8.9,1.4)(8.84,0.82)
%\psline[linewidth=0.04cm,linecolor=color254,linestyle=dotted,dotsep=0.16cm](8.92,1.72)(8.98,1.26)
%\psline[linewidth=0.04cm,linecolor=color254,linestyle=dotted,dotsep=0.16cm](2.8,3.74)(2.8,3.4)
%\psline[linewidth=0.04cm,linecolor=color254,linestyle=dotted,dotsep=0.16cm](2.72,3.82)(2.8,3.4)
%\psline[linewidth=0.04cm,linecolor=color254,linestyle=dotted,dotsep=0.16cm](2.92,3.72)(2.8,3.32)
%\psline[linewidth=0.04cm,linecolor=color254,linestyle=dotted,dotsep=0.16cm](2.96,3.88)(2.74,3.26)
%\psline[linewidth=0.04cm,linecolor=color254,linestyle=dotted,dotsep=0.16cm](2.58,3.66)(2.8,3.26)
%\psline[linewidth=0.04cm,linecolor=color254,linestyle=dotted,dotsep=0.16cm](2.8,3.54)(2.72,3.1)
%\psline[linewidth=0.04cm,linecolor=color254,linestyle=dotted,dotsep=0.16cm](2.16,3.24)(2.56,3.48)
%\psline[linewidth=0.04cm,linecolor=color254,linestyle=dotted,dotsep=0.16cm](2.34,3.14)(2.52,3.3)
%\psline[linewidth=0.04cm,linecolor=color254,linestyle=dotted,dotsep=0.16cm](2.32,3.32)(2.42,2.92)
%\psline[linewidth=0.04cm,linecolor=color254,linestyle=dotted,dotsep=0.16cm](2.16,3.3)(2.36,2.92)
%\psline[linewidth=0.04cm,linecolor=color254,linestyle=dotted,dotsep=0.16cm](2.34,3.5)(2.26,3.18)
%\psline[linewidth=0.04cm,linecolor=color254,linestyle=dotted,dotsep=0.16cm](2.5,3.62)(2.36,3.32)
%\psline[linewidth=0.04cm,linecolor=color254,linestyle=dotted,dotsep=0.16cm](2.9,3.42)(2.9,3.1)
%\psline[linewidth=0.04cm,linecolor=color254,linestyle=dotted,dotsep=0.16cm](3.06,3.54)(2.96,3.06)
%\psline[linewidth=0.04cm,linecolor=color254,linestyle=dotted,dotsep=0.16cm](3.12,3.26)(2.98,2.98)
%\psline[linewidth=0.04cm,linecolor=color254,linestyle=dotted,dotsep=0.16cm](3.38,3.46)(3.3,3.02)
%\psline[linewidth=0.04cm,linecolor=color254,linestyle=dotted,dotsep=0.16cm](3.44,3.32)(3.68,2.98)
%\psline[linewidth=0.04cm,linecolor=color254,linestyle=dotted,dotsep=0.16cm](3.6,3.34)(3.88,2.44)
%\psline[linewidth=0.04cm,linecolor=color254,linestyle=dotted,dotsep=0.16cm](2.42,4.22)(2.76,3.62)
%\psline[linewidth=0.04cm,linecolor=color254,linestyle=dotted,dotsep=0.16cm](2.48,4.12)(2.74,3.58)
%\psline[linewidth=0.04cm,linecolor=color254,linestyle=dotted,dotsep=0.16cm](2.6,3.98)(2.5,3.66)
%\psline[linewidth=0.04cm,linecolor=color254,linestyle=dotted,dotsep=0.16cm](2.4,4.22)(2.34,3.86)
%\psline[linewidth=0.04cm,linecolor=color254,linestyle=dotted,dotsep=0.16cm](2.98,3.78)(3.2,3.02)
%\psline[linewidth=0.04cm,linecolor=color254,linestyle=dotted,dotsep=0.16cm](3.24,3.32)(3.14,2.7)
%\psline[linewidth=0.04cm,linecolor=color254,linestyle=dotted,dotsep=0.16cm](3.44,3.34)(3.22,2.62)
%\psline[linewidth=0.04cm,linecolor=color254,linestyle=dotted,dotsep=0.16cm](3.28,2.98)(3.28,2.54)
%\psline[linewidth=0.04cm,linecolor=color254,linestyle=dotted,dotsep=0.16cm](3.12,3.38)(3.14,2.52)
%\psline[linewidth=0.04cm,linecolor=color254,linestyle=dotted,dotsep=0.16cm](3.06,3.24)(3.38,2.46)
%\psline[linewidth=0.04cm,linecolor=color254,linestyle=dotted,dotsep=0.16cm](3.46,3.22)(3.76,2.22)
%\psline[linewidth=0.04cm,linecolor=color254,linestyle=dotted,dotsep=0.16cm](3.72,2.9)(3.62,2.18)
%\psline[linewidth=0.04cm,linecolor=color254,linestyle=dotted,dotsep=0.16cm](3.72,2.82)(3.56,1.78)
%\psline[linewidth=0.04cm,linecolor=color254,linestyle=dotted,dotsep=0.16cm](3.7,2.3)(3.56,1.54)
%\psline[linewidth=0.04cm,linecolor=color254,linestyle=dotted,dotsep=0.16cm](3.36,2.36)(3.64,1.66)
%\psline[linewidth=0.04cm,linecolor=color254,linestyle=dotted,dotsep=0.16cm](3.8,2.42)(3.96,2.26)
%\psline[linewidth=0.04cm,linecolor=color254,linestyle=dotted,dotsep=0.16cm](4.08,2.6)(4.08,2.14)
%\psline[linewidth=0.04cm,linecolor=color254,linestyle=dotted,dotsep=0.16cm](4.04,2.5)(4.2,1.98)
%\psline[linewidth=0.04cm,linecolor=color254,linestyle=dotted,dotsep=0.16cm](3.72,2.42)(4.12,2.14)
%\psline[linewidth=0.04cm,linecolor=color254,linestyle=dotted,dotsep=0.16cm](4.12,2.44)(4.4,2.2)
%\psline[linewidth=0.04cm,linecolor=color254,linestyle=dotted,dotsep=0.16cm](4.4,2.3)(4.68,1.82)
%\psline[linewidth=0.04cm,linecolor=color254,linestyle=dotted,dotsep=0.16cm](4.32,2.38)(4.66,1.74)
%\psline[linewidth=0.04cm,linecolor=color254,linestyle=dotted,dotsep=0.16cm](4.16,2.66)(4.36,2.04)
%\psline[linewidth=0.04cm,linecolor=color254,linestyle=dotted,dotsep=0.16cm](4.42,2.6)(4.42,2.12)
%\psline[linewidth=0.04cm,linecolor=color254,linestyle=dotted,dotsep=0.16cm](3.84,3.34)(4.36,2.46)
%\psline[linewidth=0.04cm,linecolor=color254,linestyle=dotted,dotsep=0.16cm](4.0,2.84)(4.56,2.18)
%\psline[linewidth=0.04cm,linecolor=color254,linestyle=dotted,dotsep=0.16cm](4.28,2.58)(4.92,1.8)
%\psline[linewidth=0.04cm,linecolor=color254,linestyle=dotted,dotsep=0.16cm](4.42,2.38)(4.58,1.96)
%\psline[linewidth=0.04cm,linecolor=color254,linestyle=dotted,dotsep=0.16cm](4.0,2.46)(3.92,1.74)
%\psline[linewidth=0.04cm,linecolor=color254,linestyle=dotted,dotsep=0.16cm](3.72,2.34)(3.72,1.72)
%\psline[linewidth=0.04cm,linecolor=color254,linestyle=dotted,dotsep=0.16cm](3.64,1.94)(3.68,1.48)
%\psline[linewidth=0.04cm,linecolor=color254,linestyle=dotted,dotsep=0.16cm](9.78,-0.12)(9.6,-0.54)
%\psline[linewidth=0.04cm,linecolor=color254,linestyle=dotted,dotsep=0.16cm](9.64,0.3)(9.48,-0.3)
%\psline[linewidth=0.04cm,linecolor=color254,linestyle=dotted,dotsep=0.16cm](9.62,0.14)(9.78,-0.5)
%\psline[linewidth=0.04cm,linecolor=color254,linestyle=dotted,dotsep=0.16cm](9.86,0.04)(10.26,-0.92)
%\psline[linewidth=0.04cm,linecolor=color254,linestyle=dotted,dotsep=0.16cm](10.1,-0.78)(10.4,-1.38)
%\psline[linewidth=0.04cm,linecolor=color254,linestyle=dotted,dotsep=0.16cm](10.2,-1.38)(10.44,-1.66)
%\psline[linewidth=0.04cm,linecolor=color254,linestyle=dotted,dotsep=0.16cm](10.16,-1.58)(10.44,-1.9)
%\psline[linewidth=0.04cm,linecolor=color254,linestyle=dotted,dotsep=0.16cm](10.18,-1.82)(10.36,-2.04)
%\psline[linewidth=0.04cm,linecolor=color254,linestyle=dotted,dotsep=0.16cm](10.24,-1.18)(10.4,-1.62)
%\psline[linewidth=0.04cm,linecolor=color254,linestyle=dotted,dotsep=0.16cm](10.4,-1.8)(10.4,-1.8)
%\psline[linewidth=0.04cm,linecolor=color254,linestyle=dotted,dotsep=0.16cm](10.1,-0.6)(10.2,-1.0)
%\psline[linewidth=0.04cm,linecolor=color254,linestyle=dotted,dotsep=0.16cm](9.88,-0.22)(10.36,-0.86)
%\psline[linewidth=0.04cm,linecolor=color254,linestyle=dotted,dotsep=0.16cm](9.94,0.2)(10.1,-0.06)
%\psline[linewidth=0.04cm,linecolor=color254,linestyle=dotted,dotsep=0.16cm](1.08,4.74)(1.54,4.54)
%\psline[linewidth=0.04cm,linecolor=color254,linestyle=dotted,dotsep=0.16cm](1.04,4.62)(1.38,4.52)
%\psline[linewidth=0.04cm,linecolor=color254,linestyle=dotted,dotsep=0.16cm](0.9,4.58)(1.46,4.54)
%\psline[linewidth=0.04cm,linecolor=color254,linestyle=dotted,dotsep=0.16cm](1.48,4.76)(1.54,4.5)
%\psline[linewidth=0.04cm,linecolor=color254,linestyle=dotted,dotsep=0.16cm](1.44,4.86)(1.62,4.6)
%\psline[linewidth=0.04cm,linecolor=color254,linestyle=dotted,dotsep=0.16cm](1.7,4.78)(1.56,4.5)
%\psline[linewidth=0.04cm,linecolor=color254,linestyle=dotted,dotsep=0.16cm](1.64,4.76)(1.64,4.6)
%\psline[linewidth=0.04cm,linecolor=color254,linestyle=dotted,dotsep=0.16cm](1.12,4.12)(0.84,3.88)
%\psline[linewidth=0.04cm,linecolor=color254,linestyle=dotted,dotsep=0.16cm](0.98,4.04)(0.98,4.04)
%\psline[linewidth=0.04cm,linecolor=color254,linestyle=dotted,dotsep=0.16cm](1.08,4.04)(1.08,4.04)
%\psline[linewidth=0.04cm,linecolor=color254,linestyle=dotted,dotsep=0.16cm](1.04,4.02)(0.98,4.04)
%\psline[linewidth=0.04cm,linecolor=color254,linestyle=dotted,dotsep=0.16cm](1.16,3.78)(0.72,3.4)
%\psline[linewidth=0.04cm,linecolor=color254,linestyle=dotted,dotsep=0.16cm](1.0,3.72)(0.6,3.38)
%\psline[linewidth=0.04cm,linecolor=color254,linestyle=dotted,dotsep=0.16cm](1.22,3.78)(1.46,3.7)
%\psline[linewidth=0.04cm,linecolor=color254,linestyle=dotted,dotsep=0.16cm](1.36,3.74)(1.16,3.46)
%\psline[linewidth=0.04cm,linecolor=color254,linestyle=dotted,dotsep=0.16cm](1.44,3.74)(1.2,3.34)
%\psline[linewidth=0.04cm,linecolor=color254,linestyle=dotted,dotsep=0.16cm](5.38,2.38)(5.84,2.18)
%\psline[linewidth=0.04cm,linecolor=color254,linestyle=dotted,dotsep=0.16cm](5.62,2.42)(5.78,2.12)
%\psline[linewidth=0.04cm,linecolor=color254,linestyle=dotted,dotsep=0.16cm](6.32,2.6)(5.94,2.34)
%\psline[linewidth=0.04cm,linecolor=color254,linestyle=dotted,dotsep=0.16cm](6.32,2.6)(5.92,2.38)
%\psline[linewidth=0.04cm,linecolor=color254,linestyle=dotted,dotsep=0.16cm](6.5,2.58)(6.08,2.36)
%\psline[linewidth=0.04cm,linecolor=color254,linestyle=dotted,dotsep=0.16cm](6.48,2.66)(6.16,2.54)
%\psline[linewidth=0.04cm,linecolor=color254,linestyle=dotted,dotsep=0.16cm](2.68,1.72)(2.82,1.38)
%\psline[linewidth=0.04cm,linecolor=color254,linestyle=dotted,dotsep=0.16cm](2.72,1.72)(2.8,1.24)
%\psline[linewidth=0.04cm,linecolor=color254,linestyle=dotted,dotsep=0.16cm](8.22,-5.34)(8.22,-5.34)
%\psline[linewidth=0.04cm,linecolor=color254,linestyle=dotted,dotsep=0.16cm](8.5,-5.08)(8.5,-5.08)
%\psbezier[linewidth=0.04](9.12,-5.2)(9.08,-5.36)(9.3,-5.4)(9.82,-5.26)
%\psbezier[linewidth=0.04](9.22,-4.88)(9.26,-4.86)(9.64,-4.96)(9.78,-4.98)
%\psbezier[linewidth=0.04](8.06,-5.06)(7.92,-5.36)(8.5,-5.32)(8.8,-5.14)
%\psbezier[linewidth=0.04](8.16,-4.74)(8.46,-4.76)(8.68,-4.94)(8.74,-4.88)
%\psbezier[linewidth=0.04](3.76,-5.36)(3.6,-5.62)(4.42,-5.6)(4.42,-5.4)
%\psbezier[linewidth=0.04](3.9,-5.1)(3.9384,-5.12)(4.18,-5.2)(4.4,-5.28)
%\psbezier[linewidth=0.04](2.7,-5.34)(2.54,-5.58)(3.46,-5.56)(3.44,-5.28)
%\psbezier[linewidth=0.04](2.92,-4.94)(3.18,-4.98)(3.232,-5.132)(3.44,-5.18)
%\psline[linewidth=0.04cm,linecolor=color216,linestyle=dotted,dotsep=0.16cm](8.54,-4.44)(8.7,-4.54)
%\psline[linewidth=0.04cm,linecolor=color216,linestyle=dotted,dotsep=0.16cm](8.7,-4.42)(8.64,-4.7)
%\psline[linewidth=0.04cm,linecolor=color216,linestyle=dotted,dotsep=0.16cm](8.72,-4.52)(8.46,-4.62)
%\psline[linewidth=0.04cm,linecolor=color216,linestyle=dotted,dotsep=0.16cm](8.52,-4.46)(8.68,-4.6)
%\psline[linewidth=0.04cm,linecolor=color216,linestyle=dotted,dotsep=0.16cm](8.68,-4.54)(8.44,-4.66)
%\psline[linewidth=0.04cm,linecolor=color216,linestyle=dotted,dotsep=0.16cm](9.82,-4.72)(9.48,-4.78)
%\psline[linewidth=0.04cm,linecolor=color216,linestyle=dotted,dotsep=0.16cm](9.56,-4.68)(9.8,-4.78)
%\psline[linewidth=0.04cm,linecolor=color216,linestyle=dotted,dotsep=0.16cm](9.78,-4.6)(9.4,-4.7)
%\psline[linewidth=0.04cm,linecolor=color216,linestyle=dotted,dotsep=0.16cm](8.22,-4.86)(8.18,-5.06)
%\psline[linewidth=0.04cm,linecolor=color216,linestyle=dotted,dotsep=0.16cm](8.32,-4.9)(8.22,-5.08)
%\psline[linewidth=0.04cm,linecolor=color216,linestyle=dotted,dotsep=0.16cm](8.22,-4.98)(8.6,-5.06)
%\psline[linewidth=0.04cm,linecolor=color216,linestyle=dotted,dotsep=0.16cm](8.32,-4.88)(8.56,-5.06)
%\psline[linewidth=0.04cm,linecolor=color216,linestyle=dotted,dotsep=0.16cm](8.42,-4.9)(8.24,-5.08)
%\psline[linewidth=0.04cm,linecolor=color216,linestyle=dotted,dotsep=0.16cm](8.42,-4.98)(8.26,-5.12)
%\psline[linewidth=0.04cm,linecolor=color216,linestyle=dotted,dotsep=0.16cm](9.28,-4.98)(9.3,-5.14)
%\psline[linewidth=0.04cm,linecolor=color216,linestyle=dotted,dotsep=0.16cm](9.32,-5.02)(9.68,-5.16)
%\psline[linewidth=0.04cm,linecolor=color216,linestyle=dotted,dotsep=0.16cm](9.52,-5.06)(9.24,-5.18)
%\psline[linewidth=0.04cm,linecolor=color216,linestyle=dotted,dotsep=0.16cm](9.36,-5.02)(9.2,-5.12)
%\psline[linewidth=0.04cm,linecolor=color216,linestyle=dotted,dotsep=0.16cm](9.22,-5.12)(9.66,-5.12)
%\psline[linewidth=0.04cm,linecolor=color216,linestyle=dotted,dotsep=0.16cm](2.94,-5.06)(3.24,-5.2)
%\psline[linewidth=0.04cm,linecolor=color216,linestyle=dotted,dotsep=0.16cm](3.08,-5.1)(2.9,-5.32)
%\psline[linewidth=0.04cm,linecolor=color216,linestyle=dotted,dotsep=0.16cm](2.84,-5.14)(3.16,-5.3)
%\psline[linewidth=0.04cm,linecolor=color216,linestyle=dotted,dotsep=0.16cm](3.02,-5.1)(2.82,-5.3)
%\psline[linewidth=0.04cm,linecolor=color216,linestyle=dotted,dotsep=0.16cm](3.98,-5.24)(4.18,-5.32)
%\psline[linewidth=0.04cm,linecolor=color216,linestyle=dotted,dotsep=0.16cm](4.0,-5.24)(3.98,-5.44)
%\psline[linewidth=0.04cm,linecolor=color216,linestyle=dotted,dotsep=0.16cm](3.9,-5.34)(4.22,-5.34)
%\psline[linewidth=0.04cm,linecolor=color216,linestyle=dotted,dotsep=0.16cm](4.0,-5.3)(4.34,-5.46)
%\psline[linewidth=0.04cm,linecolor=color216,linestyle=dotted,dotsep=0.16cm](4.02,-5.24)(4.34,-5.38)
%\psline[linewidth=0.04cm,linecolor=color216,linestyle=dotted,dotsep=0.16cm](4.1,-5.18)(3.82,-5.4)
%\psline[linewidth=0.04cm,linecolor=color216,linestyle=dotted,dotsep=0.16cm](4.16,-5.32)(3.84,-5.5)
%\psline[linewidth=0.04cm,linecolor=color216,linestyle=dotted,dotsep=0.16cm](2.96,-5.14)(3.24,-5.28)
%\psline[linewidth=0.04cm,linecolor=color216,linestyle=dotted,dotsep=0.16cm](3.1,-5.08)(3.42,-5.32)
%\psline[linewidth=0.04cm,linecolor=color216,linestyle=dotted,dotsep=0.16cm](9.28,-4.98)(9.56,-5.12)
%\psline[linewidth=0.04cm,linecolor=color216,linestyle=dotted,dotsep=0.16cm](8.58,-5.0)(8.58,-5.0)
%\end{pspicture}
%}

%

%

%
%\begin{figure}[H]
%\begin{center}
%\scalebox{1} % Change this value to rescale the drawing.
%{
%\begin{pspicture}(0,-2.1367188)(13.740625,2.1567187)
%\definecolor{color381b}{rgb}{0.8,0.8,0.8}
%\rput{20.449547}(0.09337744,-0.2650297){\psellipse[linewidth=0.06,dimen=outer,fillstyle=solid](0.7813525,0.12632795)(0.21990171,0.45)}
%\rput{90.0}(5.883281,-7.8167186){\psframe[linewidth=0.04,dimen=outer,fillstyle=solid,fillcolor=color381b](7.1,-0.41671875)(6.6,-1.5167187)}
%\psline[linewidth=0.04cm,arrowsize=0.05291667cm 2.0,arrowlength=1.4,arrowinset=0.4]{->}(10.3,-0.51671875)(11.0,-0.51671875)
%\psline[linewidth=0.04cm,arrowsize=0.05291667cm 2.0,arrowlength=1.4,arrowinset=0.4]{->}(10.3,-0.41671875)(10.3,-2.1167188)
%\psline[linewidth=0.04cm,arrowsize=0.05291667cm 2.0,arrowlength=1.4,arrowinset=0.4]{->}(10.3,-0.51671875)(8.1,-0.51671875)
%\usefont{T1}{ptm}{m}{n}
%\rput(11.04625,-0.70671874){Friction}
%\usefont{T1}{ptm}{m}{n}
%\rput(12.097343,-1.6067188){Force of Earth on cart}
%\psframe[linewidth=0.04,dimen=outer](5.8,0.08328125)(3.9,-0.9167187)
%\pscircle[linewidth=0.04,dimen=outer](5.35,-0.96671873){0.25}
%\pscircle[linewidth=0.04,dimen=outer](4.35,-0.96671873){0.25}
%\psline[linewidth=0.06cm](5.8,-0.81671876)(6.3,-0.81671876)
%\usefont{T1}{ptm}{m}{n}
%\rput(3.0896876,0.09328125){F}
%\usefont{T1}{ptm}{m}{n}
%\rput(6.41375,-0.20671874){rope}
%\usefont{T1}{ptm}{m}{n}
%\rput(4.764219,-0.40671876){cart}
%\usefont{T1}{ptm}{m}{n}
%\rput(6.811406,-1.0067188){log}
%\psellipse[linewidth=0.06,dimen=outer,fillstyle=solid](1.6,-0.26671875)(1.1,0.45)
%\psellipse[linewidth=0.06,dimen=outer](2.15,-0.71671873)(0.15,0.5)
%\psellipse[linewidth=0.06,dimen=outer](1.15,-0.71671873)(0.15,0.5)
%\psline[linewidth=0.06cm](3.9,-0.11671875)(1.0,-0.11671875)
%\psellipse[linewidth=0.06,dimen=outer](2.6,-0.46671876)(0.1,0.35)
%\psellipse[linewidth=0.06,dimen=outer,fillstyle=solid](0.45,0.38328126)(0.45,0.3)
%\psdots[dotsize=0.12](10.3,-0.51671875)
%\psline[linewidth=0.06cm,arrowsize=0.05291667cm 2.0,arrowlength=1.4,arrowinset=0.4]{->}(10.3,-0.41671875)(11.9,-0.41671875)
%\usefont{T1}{ptm}{m}{n}
%\rput(11.527968,-0.00671875){F$_{1}$}
%\usefont{T1}{ptm}{m}{n}
%\rput(10.027968,0.49328125){F$_{2}$}
%\psline[linewidth=0.06cm,arrowsize=0.05291667cm 2.0,arrowlength=1.4,arrowinset=0.4]{->}(10.3,-0.51671875)(10.3,1.0832813)
%\usefont{T1}{ptm}{m}{n}
%\rput(9.389688,-0.70671874){F}
%\usefont{T1}{ptm}{m}{n}
%\rput(3.956875,1.9232812){\Large Figure 1}
%\usefont{T1}{ptm}{m}{n}
%\rput(9.7760935,1.9232812){\Large Figure 2}
%\psline[linewidth=0.06cm](0.0,-1.2167188)(7.7,-1.2167188)
%\psline[linewidth=0.02cm,arrowsize=0.05291667cm 2.0,arrowlength=1.4,arrowinset=0.4]{->}(6.2,-0.31671876)(6.1,-0.81671876)
%\usefont{T1}{ptm}{m}{n}
%\rput(1.68375,0.59328127){horse}
%\end{pspicture}
%}
%\end{center}
%\end{figure}

%

%\begin{enumerate}
%\item[A] F$_{1}$: Force of log on cart;  F$_{2}$: Reaction force of Earth on cart
%\item[B] F$_{1}$: Force of log on cart;  F$_{2}$: Force of road on cart
%\item[C] F$_{1}$: Force of rope on cart; F$_{2}$: Reaction force of Earth on cart
%\item[D] F$_{1}$: Force of rope on cart; F$_{2}$: Force of road on cart
%\end{enumerate}
%}

\item {Which of the following pairs of forces correctly illustrates Newton's Third Law?
\begin{figure}[H]
\begin{center}
\scalebox{1} % Change this value to rescale the drawing.
{
\begin{pspicture}(0,-1.9640625)(13.91375,1.9640625)
\usefont{T1}{ptm}{m}{n}
\rput(1.4120313,1.3090625){A man standing still}
\usefont{T1}{ptm}{m}{n}
\rput(4.5760937,1.3090625){A crate moving at}
\usefont{T1}{ptm}{m}{n}
\rput(4.3515625,1.0090625){constant speed}
\usefont{T1}{ptm}{m}{n}
\rput(7.8235936,1.3090625){a bird flying at a con-}
\usefont{T1}{ptm}{m}{n}
\rput(8.007343,1.0290625){stant height and velocity}
\usefont{T1}{ptm}{m}{n}
\rput(11.48625,1.3090625){A book pushed }
\usefont{T1}{ptm}{m}{n}
\rput(11.357031,1.0090625){against a wall}
\psline[linewidth=0.04cm](0.2971875,-1.1009375)(2.3971875,-1.1009375)
\psellipse[linewidth=0.04,dimen=outer](0.8471875,0.2490625)(0.15,0.25)
\psline[linewidth=0.04cm](0.8371875,0.0190625)(0.8371875,-0.6609375)
\psline[linewidth=0.04cm](0.8171875,-0.6409375)(0.6971875,-1.1009375)
\psline[linewidth=0.04cm](0.6771875,-1.0609375)(0.6171875,-1.0409375)
\psline[linewidth=0.04cm](0.8371875,-0.6209375)(0.9771875,-0.9009375)
\psline[linewidth=0.04cm](0.9771875,-0.8609375)(0.9171875,-1.1009375)
\psline[linewidth=0.04cm](0.8971875,-1.0609375)(0.9971875,-1.0609375)
\psline[linewidth=0.04cm](0.8371875,-0.1009375)(1.0171875,-0.3009375)
\psline[linewidth=0.04cm](1.0171875,-0.3009375)(1.1371875,-0.0609375)
\psline[linewidth=0.04cm](0.8171875,-0.1209375)(0.6971875,-0.3609375)
\psline[linewidth=0.04cm](0.6971875,-0.3609375)(0.9971875,-0.5009375)
\psline[linewidth=0.04cm,arrowsize=0.05291667cm 2.0,arrowlength=1.4,arrowinset=0.4]{<->}(1.3171875,-0.4409375)(1.3171875,-1.8009375)
\psdots[dotsize=0.12](1.2971874,-1.1009375)
\usefont{T1}{ptm}{m}{n}
\rput(2.1003125,-0.6359375){\tiny force of floor}
\usefont{T1}{ptm}{m}{n}
\rput(2.1310937,-1.3159375){\tiny weight of man}
\usefont{T1}{ptm}{m}{n}
\rput(1.8379687,-0.8359375){\tiny on man}
\psline[linewidth=0.04cm](3.3571875,-1.0009375)(5.7371874,-1.0009375)
\psframe[linewidth=0.04,dimen=outer](5.5971875,-0.3009375)(4.6971874,-1.0009375)
\psline[linewidth=0.04cm,arrowsize=0.05291667cm 2.0,arrowlength=1.4,arrowinset=0.4]{->}(3.8971875,-0.5009375)(4.6971874,-0.5009375)
\psline[linewidth=0.04cm,arrowsize=0.05291667cm 2.0,arrowlength=1.4,arrowinset=0.4]{->}(4.6971874,-0.9609375)(3.8971875,-0.9609375)
\usefont{T1}{ptm}{m}{n}
\rput(4.219844,0.0040625){\tiny Force used to push}
\usefont{T1}{ptm}{m}{n}
\rput(3.781875,-0.2159375){\tiny the crate}
\usefont{T1}{ptm}{m}{n}
\rput(4.4482813,-1.1759375){\tiny frictional force exerted}
\usefont{T1}{ptm}{m}{n}
\rput(3.9248438,-1.4159375){\tiny by the floor}
\pscustom[linewidth=0.04]
{
\newpath
\moveto(7.2171874,-0.1009375)
\lineto(7.406955,-0.23427063)
\curveto(7.501838,-0.3009375)(7.703467,-0.3731598)(7.810211,-0.37871522)
\curveto(7.9169555,-0.38427064)(8.106723,-0.39538208)(8.189746,-0.4009375)
\curveto(8.272769,-0.40649292)(8.432885,-0.40649292)(8.509978,-0.4009375)
\curveto(8.587071,-0.39538208)(8.747188,-0.36760437)(8.830212,-0.3453821)
\curveto(8.913235,-0.32315978)(9.043698,-0.28427094)(9.091141,-0.26760438)
\curveto(9.138583,-0.2509375)(9.203815,-0.22315979)(9.257188,-0.18982638)
}
\pscustom[linewidth=0.04]
{
\newpath
\moveto(6.3771877,0.0590625)
\lineto(6.5371876,0.0090625)
\curveto(6.6171875,-0.0159375)(6.7771873,-0.0409375)(6.8571873,-0.0409375)
\curveto(6.9371877,-0.0409375)(7.0621877,-0.0409375)(7.1071873,-0.0409375)
\curveto(7.1521873,-0.0409375)(7.2021875,-0.0409375)(7.2171874,-0.0409375)
}
\pscustom[linewidth=0.04]
{
\newpath
\moveto(7.2571874,-0.0609375)
\lineto(7.1571875,-0.0809375)
\curveto(7.1071873,-0.0909375)(7.0171876,-0.1209375)(6.9771876,-0.1409375)
\curveto(6.9371877,-0.1609375)(6.8871875,-0.1859375)(6.8571873,-0.2009375)
}
\pscustom[linewidth=0.04]
{
\newpath
\moveto(6.8571873,-0.2009375)
\lineto(6.9571877,-0.1909375)
\curveto(7.0071874,-0.1859375)(7.0921874,-0.1759375)(7.1271877,-0.1709375)
\curveto(7.1621876,-0.1659375)(7.2121873,-0.1559375)(7.2571874,-0.1409375)
}
\pscustom[linewidth=0.04]
{
\newpath
\moveto(6.9371877,-0.1609375)
\lineto(7.0471873,-0.1209375)
\curveto(7.1021876,-0.1009375)(7.1771874,-0.0909375)(7.1971874,-0.1009375)
\curveto(7.2171874,-0.1109375)(7.2371874,-0.1259375)(7.2371874,-0.1409375)
}
\pscustom[linewidth=0.04]
{
\newpath
\moveto(7.1971874,-0.1009375)
\lineto(7.1571875,-0.1109375)
\curveto(7.1371875,-0.1159375)(7.0871873,-0.1259375)(7.0571876,-0.1309375)
\curveto(7.0271873,-0.1359375)(6.9821873,-0.1509375)(6.9371877,-0.1809375)
}
\pscustom[linewidth=0.04]
{
\newpath
\moveto(6.9971876,-0.1609375)
\lineto(7.0871873,-0.1609375)
\curveto(7.1321874,-0.1609375)(7.1871877,-0.1509375)(7.1971874,-0.1409375)
\curveto(7.2071877,-0.1309375)(7.2171874,-0.1159375)(7.2171874,-0.1109375)
\curveto(7.2171874,-0.1059375)(7.2171874,-0.1059375)(7.2171874,-0.1209375)
}
\psdots[dotsize=0.12](7.3971877,-0.0209375)
\pscustom[linewidth=0.04]
{
\newpath
\moveto(7.2371874,-0.0809375)
\lineto(7.2571874,-0.01275573)
\curveto(7.2671876,0.02133514)(7.3171873,0.08951706)(7.3571873,0.12360794)
\curveto(7.3971877,0.15769883)(7.4871874,0.20542602)(7.5371876,0.2190625)
\curveto(7.5871873,0.23269898)(7.7371874,0.21224426)(7.8371873,0.17815338)
\curveto(7.9371877,0.1440625)(8.107187,0.10315338)(8.177188,0.09633514)
\curveto(8.247188,0.08951691)(8.407187,0.08269882)(8.497188,0.08269882)
\curveto(8.587188,0.08269882)(8.7421875,0.09633514)(8.807187,0.10997162)
\curveto(8.872188,0.12360794)(9.007188,0.1440625)(9.077188,0.15088074)
\curveto(9.147187,0.15769897)(9.312187,0.17133515)(9.407187,0.17815338)
\curveto(9.502188,0.18497162)(9.627188,0.15769883)(9.657187,0.12360794)
\curveto(9.687187,0.08951706)(9.752188,0.03497161)(9.857187,-0.02639204)
}
\pscustom[linewidth=0.04]
{
\newpath
\moveto(9.857187,0.0)
\lineto(9.677188,-0.0409375)
\curveto(9.587188,-0.0609375)(9.437187,-0.0659375)(9.377188,-0.0509375)
\curveto(9.317187,-0.0359375)(9.212188,-0.0209375)(9.167188,-0.0209375)
\curveto(9.122188,-0.0209375)(9.042188,-0.0259375)(9.007188,-0.0309375)
\curveto(8.972187,-0.0359375)(8.782187,-0.1059375)(8.627188,-0.1709375)
\curveto(8.472187,-0.2359375)(8.2421875,-0.2609375)(8.167188,-0.2209375)
\curveto(8.092188,-0.1809375)(7.9771876,-0.1009375)(7.9371877,-0.0609375)
\curveto(7.8971877,-0.0209375)(7.8571873,0.0240625)(7.8571873,0.0390625)
}
\pscustom[linewidth=0.04]
{
\newpath
\moveto(6.3971877,0.0590625)
\lineto(6.6471877,0.1490625)
\curveto(6.7721877,0.1940625)(6.9571877,0.2490625)(7.0171876,0.2590625)
\curveto(7.0771875,0.2690625)(7.1971874,0.2790625)(7.2571874,0.2790625)
\curveto(7.3171873,0.2790625)(7.4071875,0.2640625)(7.4371877,0.2490625)
\curveto(7.4671874,0.2340625)(7.5021877,0.2190625)(7.5171876,0.2190625)
}
\pscustom[linewidth=0.04]
{
\newpath
\moveto(9.217188,-0.1809375)
\lineto(9.267187,-0.1609375)
\curveto(9.292188,-0.1509375)(9.327188,-0.1159375)(9.337188,-0.0909375)
\curveto(9.347187,-0.0659375)(9.357187,-0.0359375)(9.357187,-0.0209375)
}
\psdots[dotsize=0.12](8.077188,-0.1809375)
\usefont{T1}{ptm}{m}{n}
\rput(7.6139064,-1.6359375){\tiny The weight of the bird = }
\usefont{T1}{ptm}{m}{n}
\rput(7.508281,-1.8559375){\tiny force of Earth on bird}
\psline[linewidth=0.04cm,arrowsize=0.05291667cm 2.0,arrowlength=1.4,arrowinset=0.4]{->}(8.057187,-0.1809375)(8.057187,-1.3409375)
\usefont{T1}{ptm}{m}{n}
\rput(7.151094,-0.6759375){\tiny Weight of the bird}
\psline[linewidth=0.04cm](11.697187,0.6990625)(11.697187,-1.2609375)
\psframe[linewidth=0.04,dimen=outer](11.677188,0.2990625)(11.497188,-0.7609375)
\psline[linewidth=0.04cm,dotsize=0.07055555cm 2.0,arrowsize=0.05291667cm 2.0,arrowlength=1.4,arrowinset=0.4]{*->}(11.577188,-0.1609375)(10.557187,-0.1609375)
\psline[linewidth=0.04cm,arrowsize=0.05291667cm 2.0,arrowlength=1.4,arrowinset=0.4]{->}(11.577188,-0.1409375)(12.637188,-0.1409375)
\usefont{T1}{ptm}{m}{n}
\rput(10.501094,-0.3759375){\tiny Force of wall on book}
\usefont{T1}{ptm}{m}{n}
\rput(12.86375,-0.3759375){\tiny Force of book on wall}
\usefont{T1}{ptm}{m}{n}
\rput(1.4259375,1.7390625){\Large A}
\usefont{T1}{ptm}{m}{n}
\rput(4.6001563,1.7390625){\Large B}
\usefont{T1}{ptm}{m}{n}
\rput(7.800156,1.7390625){\Large C}
\usefont{T1}{ptm}{m}{n}
\rput(11.717343,1.7390625){\Large D}
\end{pspicture}
}
\end{center}
\end{figure}
}
\end{enumerate}
\practiceinfo

\begin{tabular}[h]{cccccc}
(1.) 01w1 & (2.) 01w2 & 
 \end{tabular}
}

\subsection{Different types of forces}

\subsubsection{Tension}
Tension is the magnitude of the force that exists in objects like ropes, chains and struts that are providing support. For example, there are tension forces in the ropes supporting a child's swing hanging from a tree.

\subsubsection{Contact and non-contact forces}
In this chapter we have come across a number of different types of forces, for example a push or a pull, tension in a string, frictional forces and the normal force. These are all examples of contact forces where there is a physical point of contact between applying the force and the object. Non-contact forces are forces that act over a distance, for example magnetic forces, electrostatic forces and gravitational forces.\\
\\
When an object is placed on a surface, two types of surface forces can be identified. Friction is a force that acts between the surface and the object and is parallel to the surface. The normal force is a force that acts between the object and the surface and is perpendicular to the surface.\\

\subsubsection{The normal force}
A 5 kg box is placed on a rough surface and a 10 N force is applied at an angle of 36,9$\degree$ to the horizontal. The box does not move. The normal force (N or F$_N$) is the force between the box and the surface acting in the vertical direction. If this force is not present the box would fall through the surface because the force of gravity pulls it downwards. The normal force therefore acts upwards. We can calculate the normal force by considering all the forces in the vertical direction. All the forces in the vertical direction must add up to zero because there is no movement in the vertical direction.
\begin{eqnarray*}
N + F_y + F_g &=& 0\\
N + 6 + (-49) &=& 0\\
N &=& 43 \eN~ \mbox{upwards}
\end{eqnarray*}

\begin{figure}[H]
\begin{center}
\scalebox{.9} % Change this value to rescale the drawing.
{
\begin{pspicture}(0,-2.47)(10.195,2.47)
\psframe[linewidth=0.04,dimen=outer](3.4,1.15)(1.4,0.05)
\psline[linewidth=0.04cm,arrowsize=0.05291667cm 2.0,arrowlength=1.4,arrowinset=0.4]{->}(3.4,1.15)(4.8,2.35)
\psline[linewidth=0.04cm,arrowsize=0.05291667cm 2.0,arrowlength=1.4,arrowinset=0.4]{->}(2.4,1.15)(2.4,2.45)
\psline[linewidth=0.04cm,arrowsize=0.05291667cm 2.0,arrowlength=1.4,arrowinset=0.4]{->}(2.4,0.05)(2.4,-2.45)
\psline[linewidth=0.04cm,arrowsize=0.05291667cm 2.0,arrowlength=1.4,arrowinset=0.4]{->}(1.4,0.55)(0.0,0.55)
\psline[linewidth=0.04cm,linestyle=dashed,dash=0.16cm 0.16cm,arrowsize=0.05291667cm 2.0,arrowlength=1.4,arrowinset=0.4]{->}(3.4,1.15)(4.9,1.15)
\psline[linewidth=0.04cm,linestyle=dashed,dash=0.16cm 0.16cm,arrowsize=0.05291667cm 2.0,arrowlength=1.4,arrowinset=0.4]{->}(4.8,1.15)(4.8,2.35)
\psline[linewidth=0.04cm](0.7,0.65)(0.7,0.45)
\psline[linewidth=0.04cm](4.1,1.25)(4.1,1.05)
\usefont{T1}{ptm}{m}{n}
\rput(2.408125,0.56){5 kg}
\usefont{T1}{ptm}{m}{n}
\rput(4.304531,-0.94){F$_g$ = 5 x 9,8 = 49 N}
\usefont{T1}{ptm}{m}{n}
\rput(2.1298437,1.86){N}
\usefont{T1}{ptm}{m}{n}
\rput(3.8364062,1.96){10 N}
\usefont{T1}{ptm}{m}{n}
\rput(6.5245314,1.86){F$_y$ = 10 sin 36,9$\degree$ = 6 N}
\usefont{T1}{ptm}{m}{n}
\rput(5.454531,0.86){F$_x$ = 10 cos 36,9$\degree$ = 8 N}
\usefont{T1}{ptm}{m}{n}
\rput(0.71796876,0.86){F$_f$}
\end{pspicture}
}
\end{center}
\caption{Friction and the normal force}
\label{friction and normal}
\end{figure}

The most interesting and illustrative normal force question, that is often asked, has to do with a scale in a lift. Using Newton's third law we can solve these problems quite easily.

When you stand on a scale to measure your weight you are pulled down by gravity. There is no acceleration downwards because there is a reaction force we call the normal force acting upwards on you. This is the force that the scale would measure. If the gravitational force were less then the reading on the scale would be less.
\vspace{.5cm}
\begin{wex}{Normal Forces 1}{A man with a mass of 100~kg stands on a scale
(measuring newtons). What is the reading on the scale?}{\westep{Identify what information is given and what is asked for} We are given the mass of the man. We know the gravitational acceleration that acts on him is 9,8~$=$~\mss .

\westep{Decide what equation to use to solve the problem}
The scale measures the normal force on the man. This is the force that
balances gravity. We can use Newton's laws to solve the problem:
\begin{equation*}
F_r=F_g+F_N
\end{equation*}
where $F_r$ is the resultant force on the man.

\westep {Firstly we determine the force on the man due to gravity}
\begin{eqnarray*}
F_g&=&mg \\
&=& (100\ekg)(9,8\emss)\\
&=& 980\ \mathrm{kg\cdot\emss} \\
&=& 980\eN\ \mathrm{downwards}
\end{eqnarray*}

\westep{Now determine the normal force acting upwards on the man}
We now know the gravitational force downwards. We know that the sum of all the forces must equal the resultant acceleration times the mass. The overall resultant acceleration of the man on the scale is $0$ - so $F_r=0$.
\begin{eqnarray*}
F_r&=&F_g+F_N \\
0 &=& -980\eN+F_N\\
F_N &=& 980\eN\ \mathrm{upwards}
\end{eqnarray*}

\westep{Quote the final answer}
The normal force is 980~N upwards. It exactly balances the gravitational force downwards so there is no net force and no acceleration on the man. The reading on the scale is 980 N.}
\end{wex}

Now we are going to add things to exactly the same problem to show how things change slightly. We will now move to a lift moving at constant velocity. Remember if velocity is constant then acceleration is zero.

\begin{wex}{Normal Forces 2}{A man with a mass of 100~kg stands on a scale (measuring newtons) inside a lift that is moving downwards at a constant velocity of 2 \ms. What is the reading on the scale?}{\westep{Identify what information is given and what is asked for}
We are given the mass of the man and the acceleration of the lift. We know the gravitational acceleration that acts on him.

\westep{Decide which equation to use to solve the problem}
Once again we can use Newton's laws. We know that the sum of all the forces must equal the resultant force, $F_r$.
\begin{equation*}
F_r=F_g+F_N
\end{equation*}

\westep{Determine the force due to gravity}
\begin{eqnarray*}
F_g&=&mg \\
&=& (100\ekg)(9,8\emss)\\
&=& 980\ \mathrm{kg\cdot\emss} \\
&=& 980\eN\ \mathrm{downwards}
\end{eqnarray*}

\westep{Now determine the normal force acting upwards on the man}
The scale measures this normal force, so once we have determined it we will know the reading on the scale. Because the lift is moving at constant velocity the overall resultant acceleration of the man on the scale is $0$. If we write out the equation:
\begin{eqnarray*}
F_r&=&F_g+F_N \\
ma&=&F_g+F_N \\
(100)(0) &=& -980\eN +F_N\\
F_N = 980\eN\ \mathrm{upwards}
\end{eqnarray*}

\westep{Quote the final answer}
The normal force is 980~N upwards. It exactly balances the gravitational force downwards so there is no net force and no acceleration on the man. The reading on the scale is 980 N.}
\end{wex}

In the previous two examples we got exactly the same result because the net acceleration on the man was zero! If the lift is accelerating downwards things are slightly different and now we will get a more interesting answer!

\begin{wex}{Normal Forces 3}{A man with a mass of 100~kg stands on a scale (measuring newtons) inside a lift that is accelerating downwards at 2 \mss. What is the reading on the scale?}
{\westep{Identify what information is given and what is asked for}
We are given the mass of the man and his resultant acceleration - this is just the acceleration of the lift. We know the gravitational acceleration also acts on him.

\westep{Decide which equation to use to solve the problem}
Once again we can use Newton's laws. We know that the sum of all the forces must equal the resultant force, $F_r$.
\begin{equation*}
F_r=F_g+F_N
\end{equation*}

\westep{Determine the force due to gravity, $F_g$}
\begin{eqnarray*}
F_g&=&mg \\
&=& (100\ekg)(9,8\emss)\\
&=& 980\ \mathrm{kg\cdot\emss} \\
&=& 980\eN\ \mathrm{downwards}
\end{eqnarray*}
\westep{Determine the resultant force, $F_r$}
The resultant force can be calculated by applying Newton's Second Law:
\begin{eqnarray*}
F_r &=& ma\\
F_r &=& (100\ekg)(-2\emss)\\
&=& -200~N\\
&=& 200~N~\mbox{down}
\end{eqnarray*}

\westep{Determine the normal force, $F_N$}
The sum of all the vertical forces is equal to the resultant force, therefore
\begin{eqnarray*}
F_r&=&F_g + F_N \\
-200\eN &=& -980\eN + F_N\\
F_N &=& 780\eN\ \mathrm{upwards}
\end{eqnarray*}

\westep{Quote the final answer}
The normal force is 780~N upwards. It balances the gravitational force downwards just enough so that the man only accelerates downwards at 2~\mss. The reading on the scale is 780 N.}
\end{wex}

\begin{wex}{Normal Forces 4}{A man with a mass of 100~kg stands on a scale (measuring newtons) inside a lift that is accelerating upwards at 4 \mss. What is the reading on the scale?}{\westep{Identify what information is given and what is asked for}
We are given the mass of the man and his resultant acceleration - this is just the acceleration of the lift. We know the gravitational acceleration also acts on him.

\westep{Decide which equation to use to solve the problem}
Once again we can use Newton's laws. We know that the sum of all the forces must equal the resultant force, $F_r$.

\begin{equation*}
F_r=F_g+F_N
\end{equation*}

\westep{Determine the force due to gravity, $F_g$}
\begin{eqnarray*}
F_g&=&mg \\
&=& (100\ekg)(9,8\emss)\\
&=& 980\ \mathrm{kg}\cdot\emss \\
&=& 980\eN\ \mathrm{downwards}
\end{eqnarray*}

\westep{Determine the resultant force, $F_r$}
The resultant force can be calculated by applying Newton's Second Law:
\begin{eqnarray*}
F_r &=& ma\\
F_r &=& (100\ekg)(4\emss)\\
&=& 400~N~\mbox{upwards}
\end{eqnarray*}

\westep{Determine the normal force, $F_N$}
The sum of all the vertical forces is equal to the resultant force, therefore
\begin{eqnarray*}
F_r&=&F_g + F_N \\
400\eN &=& -980\eN + F_N\\
F_N &=& 1380\eN\ \mathrm{upwards}
\end{eqnarray*}

\westep{Quote the final answer}
The normal force is 1380~N upwards. It balances the gravitational force downwards and then in addition applies sufficient force to accelerate the man upwards at 4\mss.  The reading on the scale is 1380 N.}
\end{wex}



\subsubsection{Friction forces}
When the surface of one object slides over the surface of another, each body exerts a frictional force on the other. For example if a book slides across a table, the table exerts a frictional force onto the book and the book exerts a frictional force onto the table (Newton's Third Law). Frictional forces act parallel to surfaces.\\

A force is not always powerful enough to make an object move, for example a small applied force might not be able to move a heavy crate. The frictional force opposing the motion of the crate is equal to the applied force but acting in the opposite direction. This frictional force is called \emph{static friction}. When we increase the applied force (push harder), the frictional force will also increase until the applied force overcomes it. This frictional force can vary from zero (when no other forces are present and the object is stationary) to a maximum that depends on the surfaces. When the applied force is greater than the maximum frictional force, the crate will move. Once the object moves, the frictional force will decrease and remain at that level, which is also dependent on the surfaces, while the objects are moving. This is called \emph{kinetic friction}. In both cases the maximum frictional force is related to the normal force and can be calculated as follows:\\
\\
For static friction: F$_f$ $\leq$ $\mu_s$ N\\
\\
Where $\mu_s$ = the coefficient of static friction\\
and N = normal force\\
\\
For kinetic friction: F$_f$ = $\mu_k$ N\\
\\
Where $\mu_k$ = the coefficient of kinetic friction\\
and N = normal force\\
\\
Remember that static friction is present when the object is not moving and kinetic friction while the object is moving. For example when you drive at constant velocity in a car on a tar road you have to keep the accelerator pushed in slightly to overcome the friction between the tar road and the wheels of the car. However, while moving at a constant velocity the wheels of the car are rolling, so this is not a case of two surfaces ``rubbing'' against each other and we are in fact looking at static friction. If you should break hard, causing the car to skid to a halt, we would be dealing with two surfaces rubbing against each other and hence kinetic friction. The higher the value for the coefficient of friction, the more 'sticky' the surface is and the lower the value, the more 'slippery' the surface is.\\

The frictional force (F$_f$) acts in the horizontal direction and can be calculated in a similar way to the normal for as long as there is no movement. If we use the same example as in Figure~\ref{friction and normal} and we choose to the rightward direction as positive,
\begin{eqnarray*}
F_f + F_x &=& 0\\
F_f + (+8) &=& 0\\
F_f &=& -8 \\
F_f &=& 8 \eN~ \mbox{to the left}
\end{eqnarray*}

\begin{wex}{Forces on a slope}{A 50 kg crate is placed on a slope that makes an angle of 30$\degree$ with the horizontal. The box does not slide down the slope. Calculate the magnitude and direction of the frictional force and the normal force present in this situation.}{
\westep{Draw a force diagram}
Draw a force diagram and fill in all the details on the diagram. This makes it easier to understand the problem.
\begin{figure}[H]
\begin{center}
\scalebox{1} % Change this value to rescale the drawing.
{
\begin{pspicture}(0,-2.6789062)(9.629063,2.6589062)
\rput{29.42486}(0.5873593,-1.4447035){\psframe[linewidth=0.04,dimen=outer](4.044689,0.9460999)(2.044689,-0.15390013)}
\psline[linewidth=0.04cm,arrowsize=0.05291667cm 2.0,arrowlength=1.4,arrowinset=0.4]{->}(2.7744842,0.87515026)(1.6940625,2.6389062)
\psline[linewidth=0.04cm,arrowsize=0.05291667cm 2.0,arrowlength=1.4,arrowinset=0.4]{->}(3.314894,-0.08295052)(3.2940626,-2.3610938)
\usefont{T1}{ptm}{m}{n}
\rput{29.42486}(0.5846268,-1.5207508){\rput(3.1709447,0.3767793){50 kg}}
\usefont{T1}{ptm}{m}{n}
\rput{29.42486}(0.95212966,-0.8962342){\rput(2.1873646,1.3575877){N}}
\usefont{T1}{ptm}{m}{n}
\rput{29.42486}(1.1931198,-1.9799297){\rput(4.3608117,1.3016655){F$_f$}}
\psline[linewidth=0.04cm](6.8940625,1.9389062)(0.4940625,-1.6610937)
\psline[linewidth=0.04cm](0.4940625,-1.6610937)(7.9940624,-1.6610937)
\psline[linewidth=0.04cm,arrowsize=0.05291667cm 2.0,arrowlength=1.4,arrowinset=0.4]{->}(4.1140623,0.57890624)(5.1940627,1.1389062)
\psline[linewidth=0.04cm,linestyle=dashed,dash=0.16cm 0.16cm,arrowsize=0.05291667cm 2.0,arrowlength=1.4,arrowinset=0.4]{->}(3.2940626,-0.06109375)(4.2940626,-1.7610937)
\psline[linewidth=0.04cm,linestyle=dashed,dash=0.16cm 0.16cm,arrowsize=0.05291667cm 2.0,arrowlength=1.4,arrowinset=0.4]{->}(4.2940626,-1.7610937)(3.2940626,-2.3610938)
\usefont{T1}{ptm}{m}{n}
\rput(5.438594,-2.4510937){F$_x$ = 490 sin 30$\degree$ = 245 N}
\usefont{T1}{ptm}{m}{n}
\rput(5.788594,-0.95109373){F$_y$ = 490 cos 30$\degree$ = 224 N}
\usefont{T1}{ptm}{m}{n}
\rput(1.6585937,-1.0510938){F$_g$ = 50 x 9,8 = 490 N}
\usefont{T1}{ptm}{m}{n}
\rput(1.4576562,-1.4510938){30$\degree$}
\usefont{T1}{ptm}{m}{n}
\rput(3.557656,-0.75109375){30$\degree$}
\psline[linewidth=0.04cm](1.9940625,1.8389063)(2.2940626,2.0389063)
\psline[linewidth=0.04cm](3.7940626,-1.2610937)(4.0940623,-1.0610938)
\psline[linewidth=0.04cm](4.4940624,0.93890625)(4.5940623,0.73890626)
\psline[linewidth=0.04cm](4.5940623,0.93890625)(4.6940627,0.73890626)
\psline[linewidth=0.04cm](3.8940625,-1.8610938)(3.9940624,-2.0610938)
\psline[linewidth=0.04cm](3.9940624,-1.8610938)(4.0940623,-2.0610938)
\end{pspicture}
}
\end{center}
\caption{Friction and the normal forces on a slope}
\end{figure}
\westep{Calculate the normal force}
The normal force acts perpendicular to the surface (and not vertically upwards). It's magnitude is equal to the component of the weight perpendicular to the slope. Therefore:
\begin{eqnarray*}
N &=& F_g~cos~30\degree\\
N &=& 490~cos~30\degree\\
N &=& 224 \eN~\mbox{perpendicular to the surface}
\end{eqnarray*}

\westep{Calculate the frictional force}
The frictional force acts parallel to the sloped surface. It's magnitude is equal to the component of the weight parallel to the slope. Therefore:
\begin{eqnarray*}
F_f &=& F_g~sin~30\degree\\
F_f &=& 490~sin~30\degree\\
F_f &=& 245 \eN~\mbox{up the slope}
\end{eqnarray*}
}
\end{wex}

We often think about friction in a negative way but very often friction is useful without us realising it. If there was no friction and you tried to prop a ladder up against a wall, it would simply slide to the ground. Rock climbers use friction to maintain their grip on cliffs. The brakes of cars would be useless if it wasn't for friction!

\begin{wex}{Coefficients of friction}{A block of wood weighing 32 N is placed on a rough, flat incline and a rope is tied to it. The tension in the rope can be increased to 8 N before the block starts to slide. A force of 4 N will keep the block moving at constant speed once it has been set in motion. Determine the coefficients of static and kinetic friction.}
{\westep{Analyse the question and determine what is asked}
The weight of the block is given (32 N) and two situations are identified: One where the block is not moving (applied force is 8 N), and one where the block is moving (applied force is 4 N). \\
We are asked to find the coefficient for static friction $\mu_s$ and kinetic friction $\mu_k$.\\
\westep{Find the coefficient of static friction}
\begin{eqnarray*}
F_f &=& \mu_s N\\
8 &=& \mu_s (32)\\
\mu_s &=& 0,25
\end{eqnarray*}
Note that the coefficient of friction does not have a unit as it shows a ratio. The value for the coefficient of friction friction can have any value up to a maximum of 0,25. When a force less than 8 N is applied, the coefficient of friction will be less than 0,25.\\

\westep{Find the coefficient of kinetic friction}
The coefficient of kinetic friction is sometimes also called the coefficient of dynamic friction. Here we look at when the block is moving:
\begin{eqnarray*}
F_f &=& \mu_k N\\
4 &=& \mu_k (32)\\
\mu_k &=& 0,125
\end{eqnarray*}
}
\end{wex}


\Exercise{title}{
\begin{enumerate}
\item {A 12 kg box is placed on a rough surface. A force of 20 N applied at an angle of 30$\degree$ to the horizontal cannot move the box. Calculate the magnitude and direction of the normal and friction forces.}
\item {A 100 kg crate is placed on a slope that makes an angle of 45$\degree$ with the horizontal. The box does not slide down the slope. Calculate the magnitude and direction of the frictional force and the normal force present in this situation.}
\item {What force T at an angle of 30$\degree$ above the horizontal, is required to drag a block weighing 20 N to the right at constant speed, if the coefficient of kinetic friction between the block and the surface is 0,20?}
\item {A block weighing 20 N rests on a horizontal surface. The coefficient of static friction between the block and the surface is 0,40 and the coefficient of dynamic friction is 0,20.
\begin{enumerate}
\item What is the magnitude of the frictional force exerted on the block while the block is at rest?
\item What will the magnitude of the frictional force be if a horizontal force of 5 N is exerted on the block?
\item What is the minimum force required to start the block moving?
\item What is the minimum force required to keep the block in motion once it has been started?
\item If the horizontal force is 10 N, determine the frictional force.
\end{enumerate}}

\item {A stationary block of mass 3kg is on top of a plane inclined at $35^{\circ}$ to the horizontal.\\ %\scalebox{1} % Change this value to rescale the drawing.
\begin{center} \begin{pspicture}(0,-1.32)(4.0,1.32) \psline[linewidth=0.04cm](0.0,-1.3)(3.98,-1.3) \psline[linewidth=0.04cm](3.98,1.3)(3.98,-1.3) \psline[linewidth=0.04cm](0.0,-1.28)(3.96,1.28)
\usefont{T1}{ptm}{m}{n} \rput(1.0734375,-1.05){35$^{\circ}$} \psline[linewidth=0.04cm](1.08,0.22)(1.76,0.66) \usefont{T1}{ptm}{m}{n} \rput(1.5576563,0.11){3kg} \psline[linewidth=0.04cm](1.08,0.22)(1.42,-0.34) \psline[linewidth=0.04cm](1.74,0.66)(2.1,0.12) \end{pspicture} \end{center} \begin{enumerate} \item Draw a force diagram (not to scale). Include the weight of the block as well as the components of the weight that are perpendicular and parallel to the inclined plane. \item Determine the values of the weight's perpendicular and parallel components. \item There exists a frictional force between the block and plane. Determine this force (magnitude and direction). \end{enumerate}}

\item {A lady injured her back when she slipped and fell in a supermarket. She holds the owner of the supermarket accountable for her medical expenses. The owner claims that the floor covering was not wet and meets the accepted standards. He therefore cannot accept responsibility. The matter eventually ends up in court. Before passing judgement, the judge approaches you, a science student, to determine whether the coefficient of static friction of the floor is a minimum of 0,5 as required. He provides you with a tile from the floor, as well as one of the shoes the lady was wearing on the day of the incident.
\begin{enumerate}
\item{Write down an expression for the coefficient of static friction.}
\item{Plan an investigation that you will perform to assist the judge in his judgement. Follow the steps outlined below to ensure that your plan meets the requirements.
\begin{enumerate}
\item{Formulate an investigation question.}
\item{Apparatus: List \emph{all} the other apparatus, except the tile and the shoe, that you will need.}
\item{A stepwise method: How will you perform the investigation? Include a relevant, labelled free body-diagram.}
\item{Results: What will you record?}
\item{Conclusion: How will you interpret the results to draw a conclusion?}
\end{enumerate}}
\end{enumerate}
}
\end{enumerate}
\practiceinfo

\begin{tabular}[h]{cccccc}
(1.) 01w3 & (2.) 01w4 & (3.) 01w5 & (4.) 01w6 & (5.) 01w7 & (6.) 01w8 &
 \end{tabular}}

\subsection{Forces in equilibrium}
At the beginning of this chapter it was mentioned that resultant forces cause objects to accelerate in a straight line. If an object is stationary or moving at constant velocity then either,
\begin{itemize}
\item{no forces are acting on the object, or}
\item{the forces acting on that object are exactly balanced.}
\end{itemize}
In other words, for stationary objects or objects moving with constant velocity, the resultant force acting on the object is zero. Additionally, if there is a perpendicular moment of force, then the object will rotate. You will learn more about moments of force later in this chapter.

Therefore, in order for an object not to move or to be in \textit{equilibrium}, the sum of the forces (resultant force) must be zero and the sum of the moments of force must be zero.

\Definition{Equilibrium}{An object in equilibrium has both the sum of the forces acting on it and the sum of the moments of the forces equal to zero.}

If a resultant force acts on an object then that object can be brought into equilibrium by applying an additional force that exactly balances this resultant. Such a force is called the {\em equilibrant} and is equal in magnitude but opposite in direction to the original resultant force acting on the object.

\Definition{Equilibrant}{The equilibrant of any number of forces is the single force required to produce equilibrium, and is equal in magnitude but opposite in direction to the resultant force.}

\begin{center}
\begin{pspicture}(-4,-2)(4,2)
\psline[arrowscale=2]{->}(0,0)(-2,-2)
\psline[arrowscale=2]{->}(0,0)(-2,2)
\psline[linestyle=dotted]{-}(-2,-2)(-4,0)
\psline[linestyle=dotted]{-}(-2,2)(-4,0)
\psline[arrowscale=2,linewidth=2pt]{->}(0,0)(-4,0)
\psline[arrowscale=2,linewidth=2pt,linestyle=dashed]{->}(0,0)(4,0)
\rput(-0.8,1.2){$F_1$}
\rput(-0.8,-1.2){$F_2$}
\rput(-2,0.3){Resultant of}
\rput(-2,-0.3){$F_1$ and $F_2$}
\rput(2,0.3){$F_3$}
\rput(2,-0.3){Equilibrant of}
\rput(2,-0.7){$F_1$ and $F_2$}
\end{pspicture}
\end{center}

In the figure the resultant of $F_1$ and $F_2$ is shown. The equilibrant of $F_1$ and $F_2$ is then the vector opposite in direction to this resultant with the same magnitude (i.e.\@{} $F_3$).

\begin{itemize}
\item{ $F_1$, $F_2$ and $F_3$ are in equilibrium}
\item{ $F_3$ is the equilibrant of $F_1$ and
$F_2$}
\item{$F_1$ and
$F_2$ are kept in equilibrium by $F_3$}
\end{itemize}

As an example of an object in equilibrium, consider an object held stationary by two ropes in the arrangement below:

\begin{center}
\begin{pspicture}(-2,-2)(5,2)
\psline[linewidth=2pt]{-}(-2,2)(4.9,2)
\psline{-}(-1.5,2)(1.02,-1)
\psline{-}(1.02,-1)(4.6,2)
\psarc{-}(-1.5,2){1.1}{-50}{0}
\psarc{-}(4.6,2){1.1}{180}{220}
\pspolygon[](0.02,-1)(2.02,-1)(2.02,-2)(0.02,-2)(0.02,-1)
\rput(-0.85,1.75){$50^{\circ}$}
\rput(3.95,1.75){$40^{\circ}$}
\rput(0.5,0.5){Rope 1}
\rput(3.75,0.5){Rope 2}
\end{pspicture}
\end{center}

Let us draw a free body diagram for the object. In the free body diagram the
object is drawn as a dot and all forces acting on the object are drawn in the correct directions starting from that dot. In this case, three forces are acting on the object.

\begin{center}
\begin{pspicture}(-2,-4)(2,2.5)
\psdot[dotsize=0.2](0,0)
\psline[arrowscale=2]{->}(0,0)(-1.92,2.3)
\psline[arrowscale=2]{->}(0,0)(1.92,1.61)
\psline[arrowscale=2]{->}(0,0)(0,-3.9)
\psline[linestyle=dashed]{-}(-1.92,2.3)(-0.5,2.3)
\psline[linestyle=dashed]{-}(0.5,1.61)(1.92,1.61)
\psarc{-}(-1.92,2.3){1.2}{-50}{0}
\psarc{-}(1.92,1.61){1.2}{180}{220}
\rput(-1.25,2.1){$50^{\circ}$}
\rput(1.25,1.41){$40^{\circ}$}
\rput(0.6,-1.95){$F_g$}
\rput(-1.4,1.2){$T_1$}
\rput(1.4,0.8){ $T_2$}
\end{pspicture}
\end{center}

Each rope exerts a force on the object in the direction of the rope away from the object. These tension forces are represented by $T_1$ and $T_2$. Since the object has mass, it is attracted towards the centre of the Earth. This weight is represented in the force diagram as $F_g$.

Since the object is stationary, the resultant force acting on the object is zero. In other words the three force vectors drawn tail-to-head form a closed triangle:

\begin{center}
\begin{pspicture}(-3,0)(0,6)
\psline[arrowscale=2]{->}(0,0)(-2.88,3.45)
\psline[arrowscale=2]{->}(-2.88,3.45)(0,5.87)
\psline[arrowscale=2]{->}(0,5.87)(0,0)
\psline[linestyle=dashed]{-}(-2.88,3.45)(-0.75,3.45)
\psline[linestyle=dashed]{-}(-2.13,5.87)(0,5.87)
\psarc{-}(-2.88,3.45){1.2}{-50}{0}
\psarc{-}(0,5.87){1.2}{180}{220}
\rput(-2.1,3.25){$50^{\circ}$}
\rput(-0.8,5.67){$40^{\circ}$}
\rput(0.8,2.93){$F_g$}
\rput(-1.95,1.8){$T_1$}
\rput(-1.9,4.75){$T_2$}
\end{pspicture}
\end{center}

\begin{wex}{Equilibrium}{A car engine of weight 2000~N is lifted by means of a chain and pulley system. The engine is initially suspended by the chain, hanging stationary. Then, the engine is pulled sideways by a mechanic, using a rope. The engine is held in such a position that the chain makes an angle of $30^{\circ}$ with the vertical. In the questions that follow, the masses of the chain and the rope can be ignored.
\begin{center}
\begin{pspicture}(-4,-1.6)(6,4)
\psline{-}(-4.5,4)(-1.5,4)
\psline{-}(-0.5,4)(2.5,4)
\psline{-}(-3,4)(-3,0)
\rput(-2.2,2){chain}
\psline[linestyle=dashed]{-}(1,4)(1,1)
\pspolygon[linewidth=1pt](-3.9,0)(-2.1,0)(-2.1,-1)(-3.9,-1)
\rput(-3,-0.5){engine}
\uput[d](-3,-1){initial}
\psline{-}(1,4)(3,0.54)
\psarc{-}(1,4){1.4}{270}{300}
\rput(1.3,3){$30^{\circ}$}
\rput(3,2){chain}
\pspolygon[linewidth=1pt](2.1,0.54)(3.9,0.54)(3.9,-0.46)(2.1,-0.46)
\rput(4.7,0.3){rope}
\rput(3,0.04){engine}
\psline{-}(3.9,0.04)(5.5,0.04)
\uput[d](3,-1){final}
\end{pspicture}
\end{center}
\begin{enumerate}
\item Draw a free body representing the forces acting on the engine
in the initial situation.
\item Determine the tension in the chain initially.
\item Draw a free body diagram representing the forces acting on the engine
in the final situation.
\item Determine the magnitude of the applied force and the tension in the chain in the final situations.
\end{enumerate}
}{\westep{Initial free body diagram for the engine}
There are only two forces acting on the engine initially: the tension in the chain, $T_{chain}$ and the weight of the engine, $F_g$.
\begin{center}
\begin{pspicture}(-1,-2)(1,2)
\psdot[dotsize=0.2](0,0)
\psline[arrowscale=2]{->}(0,0)(0,2)
\psline[arrowscale=2]{->}(0,0)(0,-2)
\rput(-0.6,1){$T_{chain}$}
\rput(-0.4,-1){$F_g$}
\end{pspicture}
\end{center}
\westep{Determine the tension in the chain}
The engine is initially stationary, which means that the resultant force on
the engine is zero. There are also no moments of force. Thus the tension in the chain exactly balances the weight of the engine. The tension in the chain is:
\begin{eqnarray*}
T_{chain} &=& F_g\\
&=& 2000\eN
\end{eqnarray*}
\westep{Final free body diagram for the engine}
There are three forces acting on the engine finally: The tension in the chain, the applied force and the weight of the engine.
\begin{center}
\begin{pspicture}(0.6,-2)(3.8,2.8)
\psline[arrowscale=2]{->}(3,-2)(0.69,2)
\psline[arrowscale=2]{->}(0.69,2)(3,2)
\psline[arrowscale=2]{->}(3,2)(3,-2)
\psline[linestyle=dotted]{-}(0.69,2)(0.69,0)
\rput(1.85,2.5){$F_{applied}$}
\rput(1.5,-0.4){$T_{chain}$}
\rput(3.5,0){$F_g$}
\psarc{-}(0.69,2){1.5}{270}{300}
\rput(1.0,0.85){$30^{\circ}$}
\pspolygon[linewidth=1pt](2.6,2)(2.6,1.6)(3,1.6)(3,2)
\psarc{-}(3,-2){1.5}{90}{120}
\rput(2.7,-0.8){$30^{\circ}$}
\end{pspicture}
\end{center}
\westep{Calculate the magnitude of the applied force and the tension in the chain in the final situation}
Since no method was specified let us calculate the magnitudes algebraically. Since the triangle formed by the three forces is a right-angle triangle this is easily done:
\begin{eqnarray*}
\frac{F_{applied}}{F_g} &=& \tan30^{\circ}\\
F_{applied} &=& (2000\eN)\tan30^{\circ}\\
&=& 1\,155\eN
\end{eqnarray*}
and
\begin{eqnarray*}
\frac{T_{chain}}{F_g} &=& \frac{1}{\cos30^{\circ}}\\
T_{chain} &=& \frac{2000\eN}{\cos30^{\circ}}\\
&=& 2\,309\eN
\end{eqnarray*}
}
\end{wex}

\Exercise{title}{
\begin{enumerate}
\item {The diagram shows an object of weight W, attached to a string. A horizontal force F is applied to the object so that the string makes an angle of $\theta$ with the vertical when the object is at rest. The force exerted by the string is T. Which one of the following expressions is incorrect?

\begin{figure}[h]
\begin{center}
\scalebox{1} % Change this value to rescale the drawing.
{
\begin{pspicture}(0,-1.72)(4.24,1.72)
\psline[linewidth=0.08cm](0.6,1.4)(4.2,1.4)
\psline[linewidth=0.04cm](2.6,1.4)(1.4,0.1)
\psline[linewidth=0.04cm,arrowsize=0.05291667cm 2.0,arrowlength=1.4,arrowinset=0.4]{->}(1.4,0.1)(1.4,-1.7)
\psline[linewidth=0.04cm,arrowsize=0.05291667cm 2.0,arrowlength=1.4,arrowinset=0.4]{->}(1.4,0.1)(0.0,0.1)
\psline[linewidth=0.04cm,linestyle=dashed,dash=0.16cm 0.16cm](2.6,1.4)(2.6,0.0)
\psdots[dotsize=0.12](1.4,0.1)
\usefont{T1}{ptm}{m}{n}
\rput(2.3614061,0.81){$\theta$}
\usefont{T1}{ptm}{m}{n}
\rput(1.8,0.91){T}
\usefont{T1}{ptm}{m}{n}
\rput(0.4896875,0.41){F}
\usefont{T1}{ptm}{m}{n}
\rput(1.6660937,-0.79){W}
\psline[linewidth=0.04cm](0.7,1.7)(1.0,1.4)
\psline[linewidth=0.04cm](1.0,1.7)(1.3,1.4)
\psline[linewidth=0.04cm](1.3,1.7)(1.6,1.4)
\psline[linewidth=0.04cm](1.6,1.7)(1.9,1.4)
\psline[linewidth=0.04cm](1.9,1.7)(2.2,1.4)
\psline[linewidth=0.04cm](2.2,1.7)(2.5,1.4)
\psline[linewidth=0.04cm](2.5,1.7)(2.8,1.4)
\psline[linewidth=0.04cm](2.8,1.7)(3.1,1.4)
\psline[linewidth=0.04cm](3.1,1.7)(3.4,1.4)
\psline[linewidth=0.04cm](3.4,1.7)(3.7,1.4)
\psline[linewidth=0.04cm](3.7,1.7)(4.0,1.4)
\end{pspicture}
}
\end{center}
\end{figure}
\begin{enumerate}
\item  F + T + W = 0
\item  W = T cos $\theta$
\item  tan $\theta$ = $\frac{F}{W}$
\item  W = T sin $\theta$
\end{enumerate}
}

\item {The point Q is in equilibrium due to three forces $F_1$, $F_2$ and $F_3$ acting on it. Which of the statements about these forces is INCORRECT?
\begin{enumerate}
\item The sum of the forces $F_1$, $F_2$ and $F_3$ is zero.
\item The three forces all lie in the same plane.
\item The resultant force of $F_1$ and $F_3$ is $F_2$.
\item The sum of the components of the forces in any direction is zero.
\end{enumerate}
\begin{center} \begin{pspicture}(0,-2.82)(4.62,2.82) \psline[linewidth=0.04cm,arrowsize=0.05291667cm 2.0,arrowlength=1.4,arrowinset=0.4]{->}(1.8,0.2)(4.6,2.8) \psline[linewidth=0.04cm,arrowsize=0.05291667cm 2.0,arrowlength=1.4,arrowinset=0.4]{->}(1.8,0.2)(1.8,-2.8) \psline[linewidth=0.04cm,arrowsize=0.05291667cm 2.0,arrowlength=1.4,arrowinset=0.4]{->}(1.8,0.2)(0.0,2.1)
\usefont{T1}{ptm}{m}{n} \rput(3.6835938,1.51){$F_2$}
\usefont{T1}{ptm}{m}{n} \rput(2.0835938,-1.39){$F_1$}
\usefont{T1}{ptm}{m}{n} \rput(2.0346875,0.11){Q}
\usefont{T1}{ptm}{m}{n} \rput(1.1835938,1.31){$F_3$} \end{pspicture} \end{center}

\item {A point is acted on by two forces in equilibrium. The forces \begin{enumerate} \item[A]{have equal magnitudes and directions.} \item[B]{have equal magnitudes but opposite directions.} \item[C]{act perpendicular to each other.} \item[D]{act in the same direction.} \end{enumerate}}

\item {A point in equilibrium is acted on by three forces. Force $F_1$ has components 15 N due south and 13 N due west. What are the components of force $F_2$?\\ %\scalebox{1} % Change this value to rescale the drawing.
\begin{center} \begin{pspicture}(0,-2.0085938)(4.005625,2.0085938) \psline[linewidth=0.04cm,linestyle=dashed,dash=0.16cm 0.16cm](1.9996876,1.7373438)(1.9996876,-1.6626563) \psline[linewidth=0.04cm,linestyle=dashed,dash=0.16cm 0.16cm](0.2996875,0.03734375)(3.6996875,0.03734375) \psline[linewidth=0.04cm,arrowsize=0.0529cm 3.17,arrowlength=1.4,arrowinset=0.0]{->}(1.9996876,0.03734375)(3.2996874,0.03734375) \psline[linewidth=0.04cm,arrowsize=0.05291667cm 3.17,arrowlength=1.4,arrowinset=0.0]{->}(1.9996876,0.03734375)(0.8996875,1.6373438) \psline[linewidth=0.04cm,arrowsize=0.05291667cm 3.17,arrowlength=1.4,arrowinset=0.0]{->}(1.9996876,0.03734375)(0.6996875,-1.2626562) \usefont{T1}{ptm}{m}{n} \rput(1.9879688,1.8373437){\footnotesize N} \usefont{T1}{ptm}{m}{n} \rput(0.13625,0.03734375){\footnotesize W} \usefont{T1}{ptm}{m}{n} \rput(1.9701562,-1.8626562){\footnotesize S} \usefont{T1}{ptm}{m}{n} \rput(3.8676562,0.03734375){\footnotesize E} \usefont{T1}{ptm}{m}{n} \rput(2.5396874,-0.16265625){\footnotesize 20 N} \usefont{T1}{ptm}{m}{n} \rput(1.5371875,1.1373438){\footnotesize F$_2$} \usefont{T1}{ptm}{m}{n} \rput(1.5371875,-0.86265624){\footnotesize F$_1$} \end{pspicture} \end{center} } \begin{enumerate} \item[A]{13 N due north and 20 due west} \item[B]{13 N due north and 13 N due west} \item[C]{15 N due north and 7 N due west} \item[D]{15 N due north and 13 N due east} \end{enumerate}
}

\item{\begin{enumerate}
\item Define the term 'equilibrant'.
\item Two tugs, one with a pull of 2500 N and the other with a pull of 3 000 N are used to tow an oil drilling 	platform. The angle between the two cables is 30~$\degree$. Determine, either by scale diagram or by calculation (a clearly labelled rough sketch must be given), the equilibrant of the two forces.
\end{enumerate}}

\item {A 10 kg block is held motionless by a force F on a frictionless plane, which is inclined at an angle of 50$\degree$ to the horizontal, as shown below:
\begin{figure}[H]
\begin{center}
\scalebox{1} % Change this value to rescale the drawing.
{
\begin{pspicture}(0,-1.4392188)(4.52,1.4592187)
\rput{30.623583}(0.45440704,-0.9981278){\psframe[linewidth=0.04,dimen=outer](2.9,0.78078127)(1.2,-0.11921875)}
\psline[linewidth=0.08cm,arrowsize=0.05291667cm 2.0,arrowlength=1.4,arrowinset=0.4]{->}(2.8,0.68078125)(3.9,1.3807813)
\psline[linewidth=0.04cm](0.0,-1.4192188)(4.5,-1.4192188)
\psline[linewidth=0.04cm](0.0,-1.4192188)(4.0,0.98078126)
\usefont{T1}{ptm}{m}{n}
\rput(1.1623437,-1.2092187){50$\degree$}
\usefont{T1}{ptm}{m}{n}
\rput(3.1896875,1.2907813){F}
\usefont{T1}{ptm}{m}{n}
\rput(2.085625,0.39078125){10 kg}
\end{pspicture}
}
\end{center}
\end{figure}
\begin{enumerate}
\item Draw a force diagram (not a triangle) indicating all the forces acting on the block.
\item Calculate the magnitude of force F.  Include a labelled diagram showing a triangle of forces in your answer.
\end{enumerate}}

\item {A rope of negligible mass is strung between two vertical struts. A mass M of weight W hangs from the rope through a hook fixed at point Y \begin{enumerate} \item Draw a vector diagram, plotted head to tail, of the forces acting at point Y. Label each force and show the size of each angle. \item Where will the tension be greatest? Part P or Q? Motivate your answer. \item When the tension in the rope is greater than 600N it will break. What is the maximum mass that the above set up can support? \end{enumerate} \begin{center} %\scalebox{1} % Change this value to rescale the drawing.
\begin{pspicture}(0,-1.63)(4.9071875,1.61) \psframe[linewidth=0.04,dimen=outer](0.661875,1.61)(0.481875,-1.31) \psframe[linewidth=0.04,dimen=outer](4.481875,1.59)(4.301875,-1.31) \psline[linewidth=0.04cm](0.641875,1.39)(1.821875,-0.05) \psline[linewidth=0.04cm](1.821875,-0.05)(4.301875,1.43) \psline[linewidth=0.04cm](1.701875,0.07)(1.841875,0.17) \psline[linewidth=0.04cm](1.821875,0.17)(1.941875,0.01) \psdots[dotsize=0.12](1.801875,-0.05) \psline[linewidth=0.04cm](1.781875,-0.07)(1.781875,-0.43) \psframe[linewidth=0.04,dimen=outer](2.061875,-0.41)(1.481875,-0.99) \psline[linewidth=0.04cm,arrowsize=0.05291667cm 2.0,arrowlength=1.4,arrowinset=0.4]{->}(1.761875,-1.13)(1.761875,-1.61)
\usefont{T1}{ptm}{m}{n} \rput(2.0154688,-1.44){W}
\usefont{T1}{ptm}{m}{n} \rput(1.7434375,-0.76){M}
\usefont{T1}{ptm}{m}{n} \rput(1.6501563,-0.1){Y}
\usefont{T1}{ptm}{m}{n} \rput(1.2225,0.96){P}
\usefont{T1}{ptm}{m}{n} \rput(3.40125,1.12){Q}
\usefont{T1}{ptm}{m}{n} \rput(0.98546875,0.66){$30^{\circ}$}
\usefont{T1}{ptm}{m}{n} \rput(3.8054688,0.7){$60^{\circ}$} \end{pspicture} \end{center}
}

\item {An object of weight $w$ is supported by two cables attached to the ceiling and wall as shown. The tensions in the two cables are $T_1$ and $T_2$ respectively. Tension $T_1 = 1200$~N. Determine the tension $T_2$ and weight $w$ of the object by accurate construction and measurement or calculation.\\

\scalebox{0.75} % Change this value to rescale the drawing.
{\begin{pspicture}(0,-2.03)(6.4809375,2.03) \psline[linewidth=0.04cm](0.4809375,2.01)(0.4809375,-2.01) \psline[linewidth=0.04cm](0.4809375,2.01)(6.4609375,2.01) \psframe[linewidth=0.04,dimen=outer](3.9409375,-1.3730845)(3.0009375,-1.9700994) \psline[linewidth=0.04cm](3.4809375,-0.77606964)(3.4809375,-1.3929851) \psline[linewidth=0.04cm,arrowsize=0.05291667cm 2.0,arrowlength=1.4,arrowinset=0.4]{->}(3.4809375,-0.79597014)(5.5809374,2.01) \psline[linewidth=0.04cm,arrowsize=0.05291667cm 2.0,arrowlength=1.4,arrowinset=0.4]{->}(3.4809375,-0.79)(0.5009375,0.27) \usefont{T1}{ptm}{m}{n} \rput(4.872031,0.5){T$_1$} \usefont{T1}{ptm}{m}{n} \rput(1.7120312,-0.44){T$_2$} \usefont{T1}{ptm}{m}{n} \rput(3.4745312,-1.68){w} \usefont{T1}{ptm}{m}{n} \rput(5.114375,1.78){45$^\circ$} \usefont{T1}{ptm}{m}{n} \rput(0.8640625,-0.14){70$^\circ$} \end{pspicture} }
}


\item {A rope is tied at two points which are 70 cm apart from each other, on the same horizontal line.  The total length of rope is 1 m, and the maximum tension it can withstand is 1000 N.  Find the largest mass (m), in kg, that can be carried at the midpoint of the rope, without breaking the rope.  Include a labelled diagram showing the triangle of forces in your answer.
\begin{figure}[h]
\begin{center}
\scalebox{1} % Change this value to rescale the drawing.
{
\begin{pspicture}(0,-1.0417187)(8.02,1.0617187)
\psline[linewidth=0.04cm](0.0,0.5782812)(2.0,0.5782812)
\psline[linewidth=0.04cm](2.0,0.5782812)(2.0,-1.0217187)
\psline[linewidth=0.04cm](8.0,0.5782812)(6.0,0.5782812)
\psline[linewidth=0.04cm](6.0,0.5782812)(6.0,-1.0217187)
\psline[linewidth=0.04cm](2.0,0.5782812)(4.0,0.17828125)
\psline[linewidth=0.04cm](4.0,0.17828125)(6.0,0.5782812)
\psline[linewidth=0.04cm](4.0,0.17828125)(4.0,-0.42171875)
\psframe[linewidth=0.04,dimen=outer](4.4,-0.42171875)(3.6,-1.0217187)
\usefont{T1}{ptm}{m}{n}
\rput(4.0376563,-0.71171874){m}
\psline[linewidth=0.04cm,arrowsize=0.05291667cm 2.0,arrowlength=1.4,arrowinset=0.4]{->}(3.8,0.67828125)(2.0,0.67828125)
\psline[linewidth=0.04cm,arrowsize=0.05291667cm 2.0,arrowlength=1.4,arrowinset=0.4]{->}(3.7,0.67828125)(6.0,0.67828125)
\usefont{T1}{ptm}{m}{n}
\rput(3.9425,0.8882812){70 cm}
\end{pspicture}
}
\end{center}
\end{figure}
}

\end{enumerate}
\practiceinfo

\begin{tabular}[h]{cccccc}
(1.) 01w9 & (2.) 01wa & (3.) 01wb & (4.) 01wc & (5.) 01wd & (6.) 01we & (7.) 01wf & (8.) 01wg & (9.) 01wh & 
 \end{tabular}
}
\section{Forces between Masses}

In Grade 10, you saw that gravitational fields exert forces on masses in the field. A field is a region of space in which an object experiences a force. The strength of a field is defined by a field strength. For example, the gravitational field strength, $g$, on or near the surface of the Earth has a value that is approximately 9,8 \mss.

The force exerted by a field of strength $g$ on an object of mass $m$ is given by:
\equ{F=m\cdot g}{eq:p:m:fmi11:weight}
This can be re-written in terms of $g$ as:
\nequ{g=\frac{F}{m}}
This means that $g$ can be understood to be a measure of force exerted per unit mass.

The force defined in Equation~\ref{eq:p:m:fmi11:weight} is known as weight.

Objects in a gravitational field exert forces on each other without touching. The gravitational force is an example of a non-contact force.

Gravity is a force and therefore must be described by a vector - so remember that gravity has both magnitude and direction.

%\pagebreak[4]
\subsection{Newton's Law of Universal Gravitation}

\Definition{Newton's Law of Universal Gravitation}{Every point mass attracts every other point mass by a force directed along the line connecting the two. This force is proportional to the product of the masses and inversely proportional to the square of the distance between them.}

The magnitude of the attractive gravitational force between the two point masses, $F$ is given by:
\equ{F = G \frac{m_1 m_2}{r^2}}{eq:newtongravitation}
where:
$G$ is the gravitational constant, $m_1$ is the mass of the first point mass, $m_2$ is the mass of the second point mass and $r$ is the distance between the two point masses.

Assuming SI units, $F$ is measured in newtons (N), $m_1$ and $m_2$ in kilograms (kg), $r$ in meters (m), and the constant $G$ is approximately equal to $6,67 \times 10^{-11} N\cdot m^2\cdot~kg^{-2}$. Remember that this is a force of attraction.

For example, consider a man of mass 80~kg standing 10 m from a woman with a mass of 65~kg. The attractive gravitational force between them would be:
\begin{eqnarray*}
F &=& G \frac{m_1 m_2}{r^2}\\
&=& (6,67 \times 10^{-11}\eN\cdot m^2\cdot~kg^{-2})(\frac{(80 kg)(65 kg)}{(10 m)^2})\\
&=&3,47 \times 10^{-9}\eN
\end{eqnarray*}
If the man and woman move to 1 m apart, then the force is:
\begin{eqnarray*}
F &=& G \frac{m_1 m_2}{r^2}\\
&=& (6,67 \times 10^{-11}\eN\cdot m^2\cdot~kg^{-2})(\frac{(80 kg)(65 kg)}{(1 m)^2})\\
&=&3,47 \times 10^{-7}\eN
\end{eqnarray*}

As you can see, these forces are very small.

Now consider the gravitational force between the Earth and the Moon. The mass of the Earth is $5,98\times 10^{24}$~kg, the mass of the Moon is $7,35\times 10^{22}$~kg and the Earth and Moon are $3,8 \times 10^{8}$~m apart. The gravitational force between the Earth and Moon is:
\begin{eqnarray*}
F &=& G \frac{m_1 m_2}{r^2}\\
&=& (6,67 \times 10^{-11}\eN\cdot m^2\cdot~kg^{-2})(\frac{(5,98\times 10^{24} kg)(7,35\times 10^{22} kg)}{(0,38 \times 10^{9} m)^2})\\
&=&2,03\times 10^{20}\eN
\end{eqnarray*}

From this example you can see that the force is very large.

These two examples demonstrate that the greater the masses, the greater the force between them. The $1/r^2$ factor tells us that the distance between the two bodies plays a role as well. The closer two bodies are, the
stronger the gravitational force between them is. We feel the gravitational attraction of the Earth most at the surface since that is the closest we can get to it, but if we were in outer-space, we would barely feel the effect of the Earth's gravity!

Remember that
\begin{equation}
\label{eqn_Fma_again}
F=m\cdot a
\end{equation}
which means that every object on Earth feels the same gravitational acceleration! That means whether you drop a pen or a book (from the same height), they will both take the same length of time to hit the ground... in fact they will be head to head for the entire fall if you drop them at the
same time. We can show this easily by using the two equations above (Equations~\ref{eq:newtongravitation} and \ref{eqn_Fma_again}). The force between the Earth (which has the mass $m_e$) and an object of mass $m_o$ is
\begin{equation}
\label{eqn_gravex1_1}
F=\frac{G m_o m_e}{r^2}
\end{equation}
and the acceleration of an object of mass $m_o$ (in terms of the force acting on it) is
\begin{equation}
\label{eqn_gravex1_2}
a_o=\frac{F}{m_o}
\end{equation}
So we substitute equation (\ref{eqn_gravex1_1}) into Equation~(\ref{eqn_gravex1_2}), and we find that
\begin{equation}
\label{eqn_gravex1_3}
a_o=\frac{G m_e}{r^2}
\end{equation}
Since it doesn't matter what $m_o$ is, this tells us that the acceleration on a body (due to the Earth's gravity) does not depend on the mass of the body. Thus all objects experience the same gravitational acceleration. The force on different bodies will be different but the acceleration will be the same. Due to the fact that this acceleration caused by gravity is the same on all objects we label it differently, instead of using $a$ we use $g$ which we call the gravitational acceleration.

\subsection{Comparative Problems}
Comparative problems involve calculation of something in terms of something else that we know. For example, if you weigh 490 N on Earth and the gravitational acceleration on Venus is 0,903 that of the gravitational acceleration on the Earth, then you would weigh 0,903~x~490~N~=~442,5~N on Venus.

\subsubsection{Principles for answering comparative problems}
\begin{itemize}
\item Write out equations and calculate all quantities for the given situation
\item Write out all relationships between variable from first and
second case
\item Write out second case
\item Substitute all first case variables into second case
\item Write second case in terms of first case
\end{itemize}

\begin{wex}{Comparative Problem 1}{A man has a mass of 70~kg. The planet Zirgon is the same size as the Earth but has twice the mass of the Earth. What would the man weigh on Zirgon, if the gravitational acceleration on Earth is 9,8~\mss?}{
\westep{Determine what information has been given}
The following has been provided:
\begin{itemize}
\item the mass of the man, $m$
\item the mass of the planet Zirgon ($m_Z$) in terms of the mass of the Earth ($m_E$), $m_Z=2m_E$
\item the radius of the planet Zirgon ($r_Z$) in terms of the radius of the Earth ($r_E$), $r_Z=r_E$
\end{itemize}

\westep{Determine how to approach the problem}
We are required to determine the man's weight on Zirgon ($w_Z$). We can do this by using:
\begin{equation*}
w = mg = G\frac{m_1\cdot m_2}{r^2}
\end{equation*}
to calculate the weight of the man on Earth and then use this value to determine the weight of the man on Zirgon.

\westep{Situation on Earth}
\begin{eqnarray*}
w_E &=& mg_E = G\frac{m_E \cdot m}{r_E^2}\\
&=&(70\ekg)(9,8\emss)\\
&=& 686\eN
\end{eqnarray*}

\westep{Situation on Zirgon in terms of situation on Earth}
Write the equation for the gravitational force on Zirgon and then substitute the values for $m_Z$ and $r_Z$, in terms of the values for the Earth.
\begin{eqnarray*}
w_Z = mg_Z &=& G\frac{m_Z\cdot m}{r_Z^2}\\
&=&G\frac{2m_E\cdot m}{r_E^2}\\
&=&2 (G\frac{m_E\cdot m}{r_E^2})\\
&=&2 w_E\\
&=&2(686\eN)\\
&=&1\,372\eN
\end{eqnarray*}

\westep{Quote the final answer}
The man weighs 1 372 N on Zirgon.}
\end{wex}

\begin{wex}{Comparative Problem 2}{A man has a mass of 70~kg. On the planet Beeble how much will he weigh if Beeble has mass half of that of the Earth and a radius one quarter that of the Earth. Gravitational acceleration on Earth is 9,8~\mss.}
{\westep{Determine what information has been given}
The following has been provided:
\begin{itemize}
\item the mass of the man on Earth, $m$
\item the mass of the planet Beeble ($m_B$) in terms of the mass of the Earth ($m_E$), $m_B=\frac{1}{2}m_E$
\item the radius of the planet Beeble ($r_B$) in terms of the radius of the Earth ($r_E$), $r_B=\frac{1}{4}r_E$
\end{itemize}

\westep{Determine how to approach the problem}
We are required to determine the man's weight on Beeble ($w_B$). We can do this by using:
\begin{equation}
w = mg = G\frac{m_1\cdot m_2}{r^2}
\end{equation}
to calculate the weight of the man on Earth and then use this value to determine the weight of the man on Beeble.

\westep{Situation on Earth}
\begin{eqnarray*}
w_E &=& mg_E = G\frac{m_E \cdot m}{r_E^2}\\
&=&(70\ekg)(9,8\emss)\\
&=& 686\eN
\end{eqnarray*}

\westep{Situation on Beeble in terms of situation on Earth}
Write the equation for the gravitational force on Beeble and then substitute the values for $m_B$ and $r_B$, in terms of the values for the Earth.
\begin{eqnarray*}
w_B = mg_B &=& G\frac{m_B\cdot m}{r_B^2}\\
&=&G\frac{\frac{1}{2}m_E\cdot m}{(\frac{1}{4}r_E)^2}\\
&=&8 (G\frac{m_E\cdot m}{r_E^2})\\
&=&8 w_E\\
&=&8(686\eN)\\
&=&5\,488\eN
\end{eqnarray*}

\westep{Quote the final answer}
The man weighs 5 488 N on Beeble.}
\end{wex}

%Falling Bodies - left out here as it is done in grade 12.

%Badly written - rewrite!! Not in syllabus
%\Extension{Terminal Velocity}{Physics is all about being simple - all we do is look at the world around us and notice how it really works. It is the one thing everyone is qualified to do - we spend most of our time when we are really young experimenting to find out how things work.

%Take a book - wave it in the air - change the angle and direction. what happens of course there is resistance. different angles make it greater - the faster the book moves the greater it is. The bigger the area of the book moving in the direction of motion the greater the force.

%So we know that air resistance exists! it is a force. So what happens when an object falls? of course there is air resistance - or drag as it is normally called. There is an approximate formula for the drag force as well. The important thing to realise is that when the drag force and the gravitational force are equal for a falling body there is no net force acting on it - which means no net acceleration. That does not mean it does not move - but it means that its speed does not change.

%It falls at a constant velocity! This velocity is called terminal velocity.}

%Not in syllabus - too little said here, must write much more or leave out.
%\Extension{Drag force}{The actual force of air resistance is quite complicated. Experiment by moving a book through the air with the face of the book and then the side of the book forward, you will agree that the area of the book makes a difference as to how much you must work in order to move the book at the same speed in both cases. This is why racing cars are slim-lined in design, and not shaped like a big box!

%Get a plastic container lid (or anything waterproof) swing it around in air and then try to swing it around under water. The density of the water is much larger than the air, making you have to work harder at swinging the lid in water. This is why boats and submarines are a lot slower than aeroplanes! So we know that density, area and speed all play a role in the drag force. The expression we use for drag force is \begin{equation} D=\frac{1}{2}C\rho Av^2 \end{equation} where $C$ is a constant which depends on the object and fluid interactions, $\rho$ is the density, $A$ is the area and $v$ is the velocity.}


\Exercise{title}{
\begin{enumerate}
\item {Two objects of mass 2m and 3m respectively exert a force F on each other
when they are a certain distance apart. What will be the force between two objects situated the same distance apart but having a mass of 5m and 6m respectively?
\begin{enumerate}
\item  0,2 F
\item  1,2 F
\item  2,2 F
\item  5 F
\end{enumerate}
}

\item {As the distance of an object above the surface of the Earth is greatly increased, the
weight of the object would
\begin{enumerate}
\item  increase
\item  decrease
\item  increase and then suddenly decrease
\item  remain the same
\end{enumerate}
}

\item {A satellite circles around the Earth at a height where the gravitational force is a
factor 4 less than at the surface of the Earth. If the Earth's radius is R, then the height of the satellite above the surface is:
\begin{enumerate}
\item  R
\item  2 R
\item  4 R
\item  16 R
\end{enumerate}
}

\item {A satellite experiences a force F when at the surface of the Earth. What will be the force on the satellite if it orbits at a height equal to the diameter of the Earth:
\begin{enumerate}
\item $\frac{1}{F}$
\item  $\frac{1}{2}$ $F$
\item  $\frac{1}{3}$ $F$
\item  $\frac{1}{9}$ $F$
\end{enumerate}}

\item {The weight of a rock lying on surface of the Moon is W. The radius of the Moon is R. On planet Alpha, the same rock has weight 8W. If the radius of planet Alpha is half that of the Moon, and the mass of the Moon is M, then the mass, in kg, of planet Alpha is:
\begin{enumerate}
\item  $\frac{M}{2}$
\item  $\frac{M}{4}$
\item  2 M
\item  4 M
\end{enumerate}	}

\item {Consider the symbols of the two physical quantities $g$ and $G$ used in Physics.
\begin{enumerate}
\item Name the physical quantities represented by $g$ and $G$.
\item Derive a formula for calculating $g$ near the Earth's surface using Newton's Law of Universal Gravitation. M and R represent the mass and radius of the Earth respectively.
\end{enumerate}
}

\item {Two spheres of mass 800g and 500g respectively are situated so that their centres are 200 cm apart. Calculate the gravitational force between them.}

\item {Two spheres of mass 2 kg and 3 kg respectively are situated so that the gravitational force between them is 2,5 x 10$^{-8}$ N. Calculate the distance between them.}

\item {Two identical spheres are placed 10 cm apart. A force of 1,6675 x 10$^{-9}$ N exists between them. Find the masses of the spheres.}

\item {Halley's comet, of approximate mass 1 x 10$^{15}$ kg was 1,3 x 10$^8$ km from the Earth,
at its point of closest approach during its last sighting in 1986.
\begin{enumerate}
\item Name the force through which the Earth and the comet interact.
\item Is the magnitude of the force experienced by the comet the same, greater than or less than the force experienced by the Earth? Explain.
\item Does the acceleration of the comet increase, decrease or remain the same as it moves closer to the Earth? Explain.
\item If the mass of the Earth is 6 x 10$^{24}$ kg, calculate the magnitude of the force exerted by the Earth on Halley's comet at its point of closest approach.
\end{enumerate}
}

\end{enumerate}
\practiceinfo

\begin{tabular}[h]{cccccc}
(1.) 01wi & (2.) 01wj & (3.) 01wk & (4.) 01wm & (5.) 01wn & (6.) 01wp & (7.) 01wq & (8.) 01wr & (9.) 01ws & (10.) 01wt & 
 \end{tabular}
}

\section{Momentum and Impulse}

Momentum is a physical quantity which is closely related to forces. Momentum is a property which applies to moving objects.

\Definition{Momentum}{Momentum is the tendency of an object to continue to move in its direction of travel. Momentum is calculated from the product of the mass and velocity of an object.}

The momentum (symbol $p$) of an object of mass $m$ moving at velocity $v$ is:
\nequ{p=m \cdot v}

According to this equation, momentum is related to both the mass and velocity of an object. A small car travelling at the same velocity as a big truck will have a smaller momentum than the truck. The smaller the mass, the smaller the velocity.\\
A car travelling at 120~\kph will have a bigger momentum than the same car travelling at 60 \kph. Momentum is also related to velocity; the smaller the velocity, the smaller the momentum.\\
Different objects can also have the same momentum, for example a car travelling slowly can have the same momentum as a motor cycle travelling relatively fast. We can easily demonstrate this.

 Consider a car of mass 1~000~kg with a velocity of 8 \ms (about 30 \kph). The momentum of the car is therefore
\begin{eqnarray*}
p&=&m \cdot v\\
&=& (1000 \ekg)(8 \ems)\\
&=& 8000 \ekg\cdot\ems
\end{eqnarray*}

Now consider a motor cycle of mass 250~kg travelling at 32 \ms\ (about 115 \kph). The momentum of the motor cycle is:
\begin{eqnarray*}
p&=&m \cdot v\\
&=& (250 \ekg)(32 \ems)\\
&=& 8000 \ekg \cdot \ems
\end{eqnarray*}

Even though the motor cycle is considerably lighter than the car, the fact that the motor cycle is travelling much faster than the car means that the momentum of both vehicles is the same.\\
\\
From the calculations above, you are able to derive the unit for momentum as kg$\cdot$\ms.\\
Momentum is also vector quantity, because it is the product of a scalar ($m$) with a vector ($v$).\\This means that whenever we calculate the momentum of an object, we need to include the direction of the momentum.
% Khan Academy video on momentum: SIYAVULA-VIDEO:http://cnx.org/content/m38980/latest/#momentum-1
\mindsetvid{Khan on momentum}{VPkki}
\begin{wex}{Momentum of a Soccer Ball}{A soccer ball of mass 420 g is kicked at 20 \ms\ towards the goal post. Calculate the momentum of the ball.}{
\westep{Identify what information is given and what is asked for}
The question explicitly gives
\begin{itemize}
\item the mass of the ball, and
\item the velocity of the ball
\end{itemize}
The mass of the ball must be converted to SI units.

\nequ{420\ \mathrm{g}=0,42\ekg}

We are asked to calculate the momentum of the ball. From the definition of momentum,
\nequ{p=m\cdot v}
we see that we need the mass and velocity of the ball, which we are given.

\westep{Do the calculation}
We calculate the magnitude of the momentum of the ball,
\begin{eqnarray*}
p&=& m \cdot v \\
&=& (0,42 \ekg)(20 \ems)\\
&=& 8,4\ekg \cdot \ems
\end{eqnarray*}
\westep{Quote the final answer}
We quote the answer with the direction of motion included,
$p$ = 8,4~kg$\cdot$\ms\ in the direction of the goal post.}
\end{wex}

\begin{wex}{Momentum of a cricket ball}{A cricket ball of mass 160 g is bowled at 40 \ms\ towards a batsman. Calculate the momentum of the cricket ball.}{\westep{Identify what information is given and what is asked for}
The question explicitly gives
\begin{itemize}
\item the mass of the ball ($m$ = 160 g = 0,16 kg), and
\item the velocity of the ball ($v$ = 40 \ms)
\end{itemize}
To calculate the momentum we will use \nequ{p = m\cdot v}

\westep{Do the calculation}
\begin{eqnarray*}
p&=& m \cdot v \\
&=& (0,16 \ekg)(40 \ems)\\
&=& 6,4\ekg \cdot \ems \\
&=& 6,4\ekg \cdot \ems \mbox{in the direction of the batsman}
\end{eqnarray*}
}
\end{wex}

\begin{wex}{Momentum of the Moon}{The Moon is 384~400 km away from the Earth and orbits the Earth in 27,3 days. If the Moon has a mass of 7,35 x 10$^{22}$kg, what is the magnitude of its momentum if we assume a circular orbit?}{\westep{Identify what information is given and what is asked for} The question explicitly gives
\begin{itemize}
\item the mass of the Moon (m = 7,35 x 10$^{22}$ kg)
\item the distance to the Moon (384 400 km = 384 400 000 m = 3,844 x 10$^8$ m)
\item the time for one orbit of the Moon (27,3 days = 27,3 x 24 x 60 x 60 = 2,36 x 10$^6$ s)
\end{itemize}

We are asked to calculate only the magnitude of the momentum of the Moon (i.e.\@{} we do not need to specify a direction). In order to do this we require the mass and the magnitude of the velocity of the Moon, since
\nequ{p = m \cdot v}

\westep{Find the magnitude of the velocity of the Moon}
The magnitude of the average velocity is the same as the speed. Therefore:
\nequ{s = \frac{d}{\Delta t}}

We are given the time the Moon takes for one orbit but not how far it travels in that time. However, we can work this out from the distance to the Moon and the fact that the Moon has a circular orbit. Using the equation for the circumference, $C$, of a circle in terms of its radius, we can determine the distance travelled by the Moon in one
orbit:
\begin{eqnarray*}
C&=&2\pi r\\
&=&2\pi (3,844\times10^8 \mathrm{m})\\
&=& 2,42\times10^{9}\ \mathrm{m}
\end{eqnarray*}

Combining the distance travelled by the Moon in an orbit and the time
taken by the Moon to complete one orbit, we can determine the
magnitude of the Moon's velocity or speed,
\begin{eqnarray*}
s &=& \frac{d}{\Delta t}\\
&=& \frac{C}{T}\\
&=& \frac{2,42 \times 10^9 m}{2,36 \times 10^6 s}\\
&=& 1,02\times 10^{3}\ems.
\end{eqnarray*}
\westep{Finally calculate the momentum and quote the answer}
The magnitude of the Moon's momentum is:
\begin{eqnarray*}
p &=& m \cdot v\\
&=& (7,35\times 10^{22} \ekg)(1,02\times 10^{3} \ems)\\
&=& 7,50 \times 10^{25}\ekg \cdot \ems
\end{eqnarray*}}
\end{wex}

\subsection{Vector Nature of Momentum}
As we have said, momentum is a vector quantity. Since momentum is a vector, the techniques of vector addition discussed in Chapter~\ref{chap:vectors} must be used to calculate the total momentum of a system.

\begin{wex}{Calculating the Total Momentum of a System}{Two billiard balls roll towards each other. They each have a mass of 0,3~kg. Ball 1 is moving at $v_1=1\ems$ to the right, while ball 2 is moving at $v_2=0,8\ems$ to the left. Calculate the total momentum of the system.}{
\westep{Identify what information is given and what is asked for} The question
explicitly gives
\begin{itemize}
\item the mass of each ball,
\item the velocity of ball 1, $v_1$, and
\item the velocity of ball 2, $v_2$,
\end{itemize}
all in the correct units!

We are asked to calculate the {\bf total momentum of
the system}. In this example our system consists of two balls. To find
the total momentum we must determine the momentum of each ball and add them.

\nequ{p_{total} = p_1 + p_2}

Since ball 1 is moving to the right, its momentum is in this direction,
while the second ball's momentum is directed towards the left.

\begin{center}
\begin{pspicture}(-3,-1)(3,1)%\psgrid
\psline{-}(-2.8,-0.5)(2.8,-0.5)
\pscircle(-2,0.){0.5}
\psline{->}(-1.5,0.)(-0.5,0.)
\rput(-1,-0.25){$v_1$}
\rput(-2,0.){$m_1$}
\pscircle(2,0.){0.5}
\psline{->}(1.5,0)(0.7,0.)
\rput(1.1,-0.25){$v_2$}
\rput(2,0.){$m_2$}
\end{pspicture}
\end{center}

Thus, we are required to find the sum of two vectors acting along the
same straight line. The algebraic method of vector addition introduced
in Chapter~\ref{chap:vectors} can thus be used.\\

\westep{Choose a frame of reference} Let us choose right as the
positive direction, then obviously left is negative.\\

\westep{Calculate the momentum} The total momentum of the system is then the sum of the two momenta taking the
directions of the velocities into account. Ball 1 is travelling
at 1 \ms to the right or +1 \ms. Ball 2 is travelling at 0,8 \ms to the left or -0,8 \ms. Thus,
\begin{eqnarray*}
p_{total} &=& m_1 v_1 + m_2 v_2\\
&=& (0,3 \ekg)(+1 \ems) + (0,3 \ekg)(-0,8 \ems)\\
&=& (+0,3 \ekg\cdot\ems) + (-0,24 \ekg\cdot\ems) \\
&=& +0,06\ekg \cdot \ems\\
&=& 0,06\ekg \cdot \ems \mbox{to the right}
\end{eqnarray*}
In the last step the direction was added in words. Since the result
in the second last line is positive, the total momentum of the system
is in the positive direction (i.e.\@{} to the right).}
\end{wex}
\pagebreak
\Exercise{title}{
\begin{enumerate}
\item{\begin{enumerate}
\item{The fastest recorded delivery for a cricket ball is 161,3 \kph, bowled by Shoaib Akhtar of Pakistan during a match against England in the 2003 Cricket World Cup, held in South Africa. Calculate the ball's momentum if it has a mass of 160 g.}
\item{The fastest tennis service by a man is 246,2 \kph by Andy Roddick of the United States of America during a match in London in 2004. Calculate the ball's momentum if it has a mass of 58 g.}
\item{The fastest server in the women's game is Venus Williams of the United States of America, who recorded a serve of 205 \kph during a match in Switzerland in 1998. Calculate the ball's momentum if it has a mass of 58 g.}
\item{If you had a choice of facing Shoaib, Andy or Venus and didn't want to get hurt, who would you choose based on the momentum of each ball.}
\end{enumerate}}
\item{Two golf balls roll towards each other. They each have a mass of 100 g. Ball 1 is moving at $v_1$ = 2,4 \ms to the right, while ball 2 is moving at $v_2$ = 3 \ms to the left. Calculate the total momentum of the system.}
\item{Two motor cycles are involved in a head on collision. Motorcycle A has a mass of 200 kg and was travelling at 120 \kph south. Motor cycle B has a mass of 250 kg and was travelling north at 100 \kph. A and B are about to collide. Calculate the momentum of the system before the collision takes place.}
\end{enumerate}
\practiceinfo

\begin{tabular}[h]{cccccc}
(1.) 01wu & (2.) 01wv & (3.) 01ww & 
\end{tabular}
}

\subsection{Change in Momentum}

Let us consider a tennis ball (mass = 0,1 kg) that is dropped at an initial velocity of 5 \ms and bounces back at a final velocity of 3 \ms. As the ball approaches the floor it has a momentum that we call the momentum before the collision. When it moves away from the floor it has a different momentum called the momentum after the collision. The bounce on the floor can be thought of as a collision taking place where the floor exerts a force on the tennis ball to change its momentum.\\

The momentum before the bounce can be calculated as follows:\\
Because momentum and velocity are vectors, we have to choose a direction as positive. For this example we choose the initial direction of motion as positive, in other words, downwards is positive.
\begin{eqnarray*}
p_{i} &=& m \cdot v_{i}\\
&=& (0,1 \ekg)(+5 \ems)\\
&=& 0,5\ \ekg\cdot\ems \mbox{downwards}
\end{eqnarray*}

When the tennis ball bounces back it changes direction. The final velocity will thus have a negative value.
The momentum after the bounce can be calculated as follows:\\
\begin{eqnarray*}
p_{f} &=& m \cdot v_{f}\\
&=& (0,1 \ekg)(-3 \ems)\\
&=& -0,3 \ekg\cdot\ems\\
&=& 0,3\ \ekg\cdot\ems \mbox{upwards}
\end{eqnarray*}

Now let us look at what happens to the momentum of the tennis ball. The momentum changes during this bounce. We can calculate the change in momentum as follows:\\
Again we have to choose a direction as positive and we will stick to our initial choice as downwards is positive. This means that the final momentum will have a negative number. \\
\begin{eqnarray*}
\Delta p &=& p_{f} - p_{i}\\
&=& m \cdot v_{f} - m \cdot v_{i}\\
&=& (-0,3 \ekg) - (0,5 \ems)\\
&=& -0,8 \ekg\cdot\ems\\
&=& 0,8\ \ekg\cdot\ems \mbox{upwards}
\end{eqnarray*}

You will notice that this number is bigger than the previous momenta calculated. This is should be the case as the ball needed to be stopped and then given momentum to bounce back. \\

\begin{wex}{Change in Momentum}{A rubber ball of mass 0,8~kg is dropped and strikes the floor with an initial velocity of 6 \ms. It bounces back with a final velocity of 4 \ms. Calculate the change in the momentum of the rubber ball caused by the floor.
\begin{figure}[H]
\begin{center}
\scalebox{1} % Change this value to rescale the drawing.
{
\begin{pspicture}(0,-2.23)(6.12,2.21)
\psline[linewidth=0.04cm](0.0,-2.21)(6.1,-2.21)
\pscircle[linewidth=0.04,dimen=outer](2.15,1.86){0.35}
\pscircle[linewidth=0.04,linestyle=dashed,dash=0.16cm 0.16cm,dimen=outer](3.39,0.02){0.35}
\psline[linewidth=0.04cm,arrowsize=0.05291667cm 2.0,arrowlength=1.4,arrowinset=0.4]{->}(2.14,1.39)(2.16,-1.63)
\psline[linewidth=0.04cm,linestyle=dashed,dash=0.16cm 0.16cm,arrowsize=0.05291667cm 2.0,arrowlength=1.4,arrowinset=0.4]{->}(3.38,-1.89)(3.38,-0.51)
\usefont{T1}{ptm}{m}{n}
\rput(1.5459375,0.42){6 \ms}
\usefont{T1}{ptm}{m}{n}
\rput(4.0909376,-0.98){4 \ms}
\usefont{T1}{ptm}{m}{n}
\rput(3.4551563,1.88){m = 0,8 kg}
\end{pspicture}
}
\end{center}
\end{figure}
}
{\westep{Identify the information given and what is asked}
The question explicitly gives
\begin{itemize}
\item the ball's mass (m = 0,8 kg),
\item the ball's initial velocity (v$_i$ = 6 \ms), and
\item the ball's final velocity (v$_f$ = 4 \ms)
\end{itemize}
all in the correct units.\\

We are asked to calculate the change in momentum of the ball,
\nequ{\Delta p = mv_f - mv_i}
We have everything we need to find $\Delta p$. Since the initial momentum is directed downwards and the final momentum is in the upward direction, we can use the algebraic method of subtraction discussed in the vectors chapter.\\

\westep{Choose a frame of reference}
Let us choose down as the positive direction.\\

\westep{Do the calculation and quote the answer}
\begin{eqnarray*}
\Delta p &=& mv_f - mv_i\\
&=& (0,8 \ekg)(-4 \ems)-(0,8 \ekg)(+6 \ems)\\
&=& (-3,2 \ekg\cdot\ems)-(4,8 \ekg\cdot\ems)\\
&=& -8 \\
&=& 8\ \ekg\cdot\ems\ \mbox{upwards}
\end{eqnarray*}
}
\end{wex}

\Exercise{title}{
\begin{enumerate}
\item {Which expression accurately describes the change of momentum of an object?
\begin{enumerate}
\item  $\frac{F}{m}$
\item  $\frac{F}{t}$
\item  $F \cdot m$
\item  $F \cdot t$
\end{enumerate}}
\item A child drops a ball of mass 100 g. The ball strikes the ground with a velocity of 5~\ms and rebounds with a velocity of 4 \ms. Calculate the change of momentum of the ball.
\item A 700 kg truck is travelling north at a velocity of 40 \kph when it is approached by a 500 kg car travelling south at a velocity of 100 \kph. Calculate the total momentum of the system.
\end{enumerate}
\practiceinfo

\begin{tabular}[h]{cccccc}
(1.) 01wx & (2.) 01wy & (3.) 01wz & 
\end{tabular}
}

\subsection{Newton's Second Law revisited}
You have learned about Newton's Second Law of motion earlier in this chapter.
Newton's Second Law describes the relationship between the motion of an object and the net force on the object. We said that the motion of an object, and therefore its momentum, can only change when a resultant force is acting on it. We can therefore say that because a net force causes an object to move, it also causes its momentum to change. We can now define Newton's Second Law of motion in terms of momentum.\\

\Definition{Newton's Second Law of Motion (N2)}{The net or resultant force acting on an object is equal to the rate of change of momentum.}

Mathematically, Newton's Second Law can be stated as:
\nequ{F_{net}=\frac{\Delta p}{\Delta t}}

\subsection{Impulse}
Impulse is the product of the net force and the time interval for which the force acts. Impulse is defined as:
\equ{\mathrm{Impulse}=F\cdot \Delta t}{eq:impulse}
However, from Newton's Second Law, we know that
\begin{eqnarray*}
F&=&\frac{\Delta p}{\Delta t}\\
\therefore\quad F \cdot \Delta t\ &=& \Delta p\\
&=&\mathrm{Impulse}
\end{eqnarray*}
Therefore,
\nequ{\mathrm{Impulse}=\Delta p}

Impulse is equal to the change in momentum of an object. From this equation we see, that for a given change in momentum, $F_{net}\Delta t$ is fixed. Thus, if $F_{net}$ is reduced, $\Delta t$ must be increased (i.e.\@ a smaller resultant force must be applied for longer to bring about the same change in momentum). Alternatively if $\Delta t$ is reduced (i.e.\@{} the resultant force is applied for a shorter period) then the resultant force must be increased to bring about the same change in momentum.\\

\begin{wex}{Impulse and Change in momentum}{A 150 N resultant force acts on a 300 kg trailer. Calculate how long it takes this force to change the trailer's velocity from 2 \ms to 6 \ms in the same direction. Assume that the forces acts to the right.}
{\westep{Identify what information is given and what is asked for}
The question explicitly gives
\begin{itemize}
\item{the trailer's mass as 300 kg,}
\item{the trailer's initial velocity as 2 \ms to the right,}
\item{the trailer's final velocity as 6 \ms to the right, and}
\item{the resultant force acting on the object}
\end{itemize}
all in the correct units!

We are asked to calculate the time taken $\Delta t$ to accelerate the trailer from the 2 to 6 \ms. From the Law of Momentum,
\begin{eqnarray*}
{F}_{net}\Delta t &=& \Delta p \\
&=& mv_f-mv_i\\
&=& m(v_f - v_i).
\end{eqnarray*}

Thus we have everything we need to find $\Delta t$!\\

\westep{Choose a frame of reference}
Choose right as the positive direction.\\

\westep{Do the calculation and quote the final answer}
\begin{eqnarray*}
F_{net}\Delta t &=& m(v_f - v_i)\\
(+150 \eN)\Delta t &=& (300 \ekg)((+6 \ems)-(+2 \ems))\\
(+150 \eN)\Delta t &=& (300 \ekg)(+4 \ems)\\
\Delta t &=& \frac{(300 \ekg)(+4 \ems)}{+150 \eN}\\
\Delta t &=& 8\es
\end{eqnarray*}
It takes 8~s for the force to change the object's velocity from 2 \ms to the right to 6 \ms to the right.}
\end{wex}

\begin{wex}{Impulsive cricketers!}{A cricket ball weighing 156~g is moving at 54~\kph towards a batsman. It is hit by the batsman back towards the bowler at 36~\kph. Calculate
\begin{enumerate}
\item the ball's impulse, and
\item the average force exerted by the bat if the ball is in contact with the bat for 0,13~s.
\end{enumerate}}{\westep{Identify what information is given and what is asked for}
The question explicitly gives
\begin{itemize}
\item the ball's mass,
\item the ball's initial velocity,
\item the ball's final velocity, and
\item the time of contact between bat and ball
\end{itemize}

We are asked to calculate the impulse
\begin{equation*}
\mathrm{Impulse} = \Delta p = F_{net}\Delta t
\end{equation*}
Since we do not have the force exerted by the bat on the ball (F$_{net}$), we have to calculate the impulse from the change in momentum of the ball. Now, since
\begin{eqnarray*}
\Delta p &=& p_{f}-p_{i}\\
&=& mv_f - mv_i,
\end{eqnarray*}
we need the ball's mass, initial velocity and final velocity, which we are given.\\
\westep{Convert to S.I. units}
Firstly let us change units for the mass
\begin{eqnarray*}
1000\ \mathrm{g} &=& 1\ekg\\
\mbox{So, }1\ \mathrm{g} &=& \frac{1}{1000}\ \mathrm{kg}\\
\therefore 156 \times 1\ \mathrm{g} &=& 156 \times \frac{1}{1000}\ \mathrm{kg}\\
&=& 0,156~\ekg
\end{eqnarray*}
Next we change units for the velocity
\begin{eqnarray*}
1\ \mathrm{km\cdot h^{-1}} &=& \frac{1000 \emm}{3\ 600\es}\\
\therefore 54 \times 1\ \mathrm{km\cdot h^{-1}} &=& 54 \times \frac{1\,000 \emm}{3\,600\es}\\
&=&15\ems
\end{eqnarray*}

Similarly, 36 \kph = 10 \ms.\\

\westep{Choose a frame of reference}
Let us choose the direction from the batsman to the bowler as the positive direction. Then the initial velocity of the ball is $v_i$ = -15 \ms, while the final velocity of the ball is $v_f$ = 10 \ms.

\westep{Calculate the momentum}
Now we calculate the change in momentum,
\begin{eqnarray*}
p &=& p_{f} - p_{i}\\
&=& mv_f - mv_i\\
&=& m(v_f - v_i)\\
&=&(0,156 \ekg)((+10 \ems)-(-15 \ems))\\
&=& +3,9 \ekg\cdot\ems\\
&=& 3,9 \ekg\cdot\ems \mbox{in the direction from batsman to bowler}
\end{eqnarray*}

\westep{Determine the impulse}
Finally since impulse is just the change in momentum of the ball,
\begin{eqnarray*}
\mathrm{Impulse} &=& \Delta p\\
&=& 3,9 \ekg\cdot\ems \mbox{in the direction from batsman to bowler}
\end{eqnarray*}

\westep{Determine the average force exerted by the bat}
\begin{equation*}
\mathrm{Impulse} = F_{net}\Delta t = \Delta p
\end{equation*}
We are given $\Delta t$ and we have calculated the impulse of the ball.
\begin{eqnarray*}
F_{net}\Delta t &=& \mathrm{Impulse}\\
F_{net}(0,13 \es) &=& +3,9 \eN\cdot\es\\
F_{net} &=& \frac{+3,9\eN\cdot\es}{0,13\es}\\
&=& +30 \eN\\
&=& 30\eN\quad \mbox{in the direction from batsman to bowler}
\end{eqnarray*}}
\end{wex}

\Exercise{title}{
\begin{enumerate}
\item {Which one of the following is NOT a unit of impulse?
\begin{enumerate}
\item $N \cdot s$
\item  $kg \cdot \ems$
\item  $J \cdot \ems$
\item  $J \cdot m^{-1} \cdot s$
\end{enumerate}}
\item {A toy car of mass 1 kg moves eastwards with a speed of 2 \ms.  It collides head-on with a toy train.  The train has a mass of 2 kg and is moving at a speed of 1,5 \ms westwards.  The car rebounds (bounces back) at 3,4 \ms and the train rebounds at 1,2 \ms.
\begin{enumerate}
\item Calculate the change in momentum for each toy.
\item Determine the impulse for each toy.
\item Determine the duration of the collision if the magnitude of the force exerted by each toy is 8 N.
\end{enumerate}
}
\item {A bullet of mass 20 g strikes a target at 300 \ms and exits at 200 \ms.  The tip of the bullet takes 0,0001s to pass through the target.  Determine:
\begin{enumerate}
\item the change of momentum of the bullet.
\item the impulse of the bullet.
\item the magnitude of the force experienced by the bullet.
\end{enumerate}}

\item {A bullet of mass 20 g strikes a target at 300 \ms.  Determine under which circumstances the bullet experiences the greatest change in momentum, and hence impulse:
\begin{enumerate}
\item When the bullet exits the target at 200 \ms.
\item When the bullet stops in the target.
\item When the bullet rebounds at 200 \ms.
\end{enumerate}}

\item {A ball with a mass of 200 g strikes a wall at right angles at a velocity of 12 \ms and rebounds at a velocity of 9 \ms.
\begin{enumerate}
\item Calculate the change in the momentum of the ball.
\item What is the impulse of the wall on the ball?
\item Calculate the magnitude of the force exerted by the wall on the ball if the collision takes 0,02s.
\end{enumerate}}

\item {If the ball in the previous problem is replaced with a piece of clay of 200 g which is thrown against the wall with the same velocity, but then sticks to the wall, calculate:
\begin{enumerate}
\item The impulse of the clay on the wall.
\item The force exerted by the clay on the wall if it is in contact with the wall for 0,5 s before it comes to rest.
\end{enumerate}}
\end{enumerate}
\practiceinfo

\begin{tabular}[h]{cccccc}
(1.) 01x0 & (2.) 01x1 & (3.) 01x2 & (4.) 01x3 & (5.) 01x4 & (6.) 01x5 & 
\end{tabular}
}
\subsection{Conservation of Momentum}
In the absence of an external force acting on a system, momentum is conserved.

\Definition{Conservation of Linear Momentum}{The total linear momentum of an isolated system is constant. An isolated system has no forces acting on it from the outside.}

This means that in an isolated system the total momentum before a collision or explosion is equal to the total momentum after the collision or explosion.

Consider a simple collision of two billiard balls. The balls are rolling on a frictionless surface and the system is isolated. So, we can apply conservation of momentum. The first ball has a mass $m_1$ and an initial velocity $v_{i1}$. The second ball has a mass $m_2$ and moves towards the first ball with an initial velocity $v_{i2}$. This
situation is shown in Figure~\ref{fig_mom1}.

\begin{figure}[!htbp]
\begin{center}
\begin{pspicture}(-3,0)(3,1)
%\psgrid[gridcolor=lightgray]
\psline[linewidth=2pt](-3,0)(3,0)
\pscircle(-2,0.5){0.5}
\pscircle(2,0.5){0.5}
\psline{->}(-1.5,0.5)(-1,0.5)
\psline{<-}(1,0.5)(1.5,0.5)
\uput[u](-1,0.5){$v_{i1}$}
\uput[u](1,0.5){$v_{i2}$}
\rput(-2,0.5){$m_1$}
\rput(2,0.5){$m_2$}
\end{pspicture}
\end{center}
\caption{Before the collision.}
\label{fig_mom1}
\end{figure}

The total momentum of the system before the collision, $p_i$ is:
\nequ{p_{i}=m_1v_{i1} + m_2v_{i2}}

After the two balls collide and move away they each have a different momentum. If the first ball has a final velocity of $v_{f1}$ and the second ball has a final velocity of $v_{f2}$ then we have the situation shown in Figure~\ref{fig_mom2}.

\begin{figure}[!htbp]
\begin{center}
\begin{pspicture}(-3,0)(3,1)
%\psgrid[gridcolor=lightgray]
\psline[linewidth=2pt](-3,0)(3,0)
\rput(0.5,0){\pscircle(-2,0.5){0.5}
\rput(-2,0.5){$m_1$}}
\rput(-0.5,0){\pscircle(2,0.5){0.5}
\rput(2,0.5){$m_2$}}
\psline{<-}(-2.5,0.5)(-2,0.5)
\psline{->}(2,0.5)(2.5,0.5)
\uput[u](-2.5,0.5){$v_{f1}$}
\uput[u](2.5,0.5){$v_{f2}$}
\end{pspicture}
\end{center}
\caption{After the collision.}
\label{fig_mom2}
\end{figure}

The total momentum of the system after the collision, $p_f$ is:
\nequ{p_{f}=m_1v_{f1} + m_2v_{f2}}

This system of two balls is isolated since there are no external forces acting on the balls. Therefore, by the principle of conservation of linear momentum, the total momentum before the collision is equal to the total momentum after the collision. This gives the equation for the conservation of momentum in a collision of two objects,

\begin{center}
\psshadowbox{
\begin{tabular}{rl}
\multicolumn{2}{c}{$p_{i}=p_{f}$}\\
\multicolumn{2}{c}{$m_1v_{i1} + m_2v_{i2} = m_1v_{f1} + m_2v_{f2}$}\\
\\
$m_1$&: mass of object 1 (kg)\\
$m_2$&: mass of object 2 (kg)\\
$v_{i1}$&: initial velocity of object 1 (\ms + direction)\\
$v_{i2}$&: initial velocity of object 2 (\ms - direction)\\
$v_{f1}$&: final velocity of object 1 (\ms - direction)\\
$v_{f2}$&: final velocity of object 2 (\ms + direction)\\
\end{tabular}}
\end{center}

This equation is always true - momentum is always conserved in collisions.

\begin{wex}{Conservation of Momentum 1}
{A toy car of mass 1 kg moves westwards with a speed of 2 \ms.  It collides head-on with a toy train.  The train has a mass of 1,5 kg and is moving at a speed of 1,5 \ms eastwards.  If the car rebounds at 2,05 \ms, calculate the velocity of the train.}
{\westep{Draw rough sketch of the situation}
\begin{figure}[H]
\begin{center}
\scalebox{1} % Change this value to rescale the drawing.
{
\begin{pspicture}(0,-1.8492187)(9.664687,1.8692187)
\usefont{T1}{ptm}{m}{n}
\rput(7.8928127,-0.71921873){1 kg}
\usefont{T1}{ptm}{m}{n}
\rput(0.68,1.6807812){BEFORE}
\usefont{T1}{ptm}{m}{n}
\rput(4.099375,1.6807812){v$_{i1}$ = 1,5 \ms}
\usefont{T1}{ptm}{m}{n}
\rput(4.069375,-1.4192188){v$_{f1}$ = ? \ms}
\usefont{T1}{ptm}{m}{n}
\rput(8.109375,-1.4192188){v$_{f2}$ = 2,05 \ms}
\usefont{T1}{ptm}{m}{n}
\rput(7.859375,1.6807812){v$_{i2}$ = 2 \ms}
\usefont{T1}{ptm}{m}{n}
\rput(0.5675,-1.4192188){AFTER}
\psframe[linewidth=0.04,dimen=outer](4.1971874,0.27078125)(3.2971876,-0.22921875)
\psline[linewidth=0.04cm](4.3971877,0.67078125)(4.3971877,-0.22921875)
\psline[linewidth=0.04cm](4.3971877,-0.22921875)(5.6971874,-0.22921875)
\psline[linewidth=0.04cm](5.6971874,-0.22921875)(5.6971874,0.27078125)
\psline[linewidth=0.04cm](5.6971874,0.27078125)(5.1971874,0.27078125)
\psline[linewidth=0.04cm](5.1971874,0.27078125)(5.1971874,0.67078125)
\psline[linewidth=0.04cm](5.1971874,0.67078125)(4.3971877,0.67078125)
\pscircle[linewidth=0.04,dimen=outer](3.5971875,-0.22921875){0.1}
\pscircle[linewidth=0.04,dimen=outer](3.8971875,-0.22921875){0.1}
\pscircle[linewidth=0.04,dimen=outer](4.7971873,-0.12921876){0.2}
\pscircle[linewidth=0.04,dimen=outer](5.3971877,-0.22921875){0.1}
\psframe[linewidth=0.04,dimen=outer](4.3971877,-0.02921875)(4.1971874,-0.12921876)
\psframe[linewidth=0.04,dimen=outer](3.0971875,0.27078125)(2.1971874,-0.22921875)
\pscircle[linewidth=0.04,dimen=outer](2.4971876,-0.22921875){0.1}
\pscircle[linewidth=0.04,dimen=outer](2.7971876,-0.22921875){0.1}
\psframe[linewidth=0.04,dimen=outer](3.2971876,-0.02921875)(3.0971875,-0.12921876)
\psline[linewidth=0.04cm,arrowsize=0.05291667cm 2.0,arrowlength=1.4,arrowinset=0.4]{->}(3.2971876,1.3707813)(5.1971874,1.3707813)
\psline[linewidth=0.04cm](7.0971875,-0.12921876)(8.797188,-0.12921876)
\psline[linewidth=0.04cm](8.797188,-0.12921876)(8.797188,0.27078125)
\psline[linewidth=0.04cm](8.797188,0.27078125)(8.397187,0.27078125)
\psline[linewidth=0.04cm](8.397187,0.27078125)(8.297188,0.67078125)
\psline[linewidth=0.04cm](8.297188,0.67078125)(7.6971874,0.67078125)
\psline[linewidth=0.04cm](7.6971874,0.67078125)(7.4971876,0.27078125)
\psline[linewidth=0.04cm](7.4971876,0.27078125)(7.0971875,0.27078125)
\psline[linewidth=0.04cm](7.0971875,0.27078125)(7.0971875,-0.12921876)
\pscircle[linewidth=0.04,dimen=outer](7.4971876,-0.12921876){0.2}
\pscircle[linewidth=0.04,dimen=outer](8.397187,-0.12921876){0.2}
\psline[linewidth=0.04cm,arrowsize=0.05291667cm 2.0,arrowlength=1.4,arrowinset=0.4]{->}(8.597187,1.3707813)(7.1971874,1.3707813)
\usefont{T1}{ptm}{m}{n}
\rput(3.7328124,-0.71921873){1,5 kg}
\psline[linewidth=0.04cm,arrowsize=0.05291667cm 2.0,arrowlength=1.4,arrowinset=0.4]{->}(7.1971874,-1.8292187)(8.797188,-1.8292187)
\end{pspicture}
}
\end{center}
\end{figure}

\westep{Choose a frame of reference}
We will choose to the east as positive.\\

\westep{Apply the Law of Conservation of momentum}
\begin{eqnarray*}
p_{i}&=& p_{f}\\
m_1v_{i1} + m_2v_{i2} &=& m_1v_{f1} + m_2v_{f2}\\
(1,5 \ekg)(+1,5 \ems) + (2 \ekg)(-2 \ems) &=& (1,5 \ekg)(v_{f1}) + (2 \ekg)(2,05 \ems)\\
2,25 \ekg\cdot\ems- 4\ekg\cdot\ems - 4,1\ekg\cdot\ems &=& (1,5\ekg) \ v_{f1}\\
5,85 \ekg\cdot\ems &=& (1,5\ekg)\ v_{f1}\\
v_{f1} &=& 3,9 \ems \mbox{eastwards}
\end{eqnarray*}
}
\end{wex}

\begin{wex}{Conservation of Momentum 2}
{A helicopter flies at a speed of 275 \ms. The pilot fires a missile forward out of a gun barrel at a speed of 700 \ms. The respective masses of the helicopter and the missile are 5000 kg and 50 kg. Calculate the new speed of the helicopter immediately after the missile had been fired.}
{\westep{Draw rough sketch of the situation}
\begin{figure}[H]
\begin{center}
\scalebox{1} % Change this value to rescale the drawing.
{
\begin{pspicture}(0,-1.8039062)(11.524688,1.8039062)
\psellipse[linewidth=0.04,dimen=outer](5.1471877,0.31390625)(1.05,0.6)
\psframe[linewidth=0.04,dimen=outer](6.2971873,-0.68609375)(4.0971875,-0.7860938)
\psframe[linewidth=0.04,dimen=outer](5.0971875,0.01390625)(4.9971876,-0.7860938)
\psframe[linewidth=0.04,dimen=outer](6.3971877,-0.48609376)(4.2971873,-0.5860937)
\psframe[linewidth=0.04,dimen=outer](5.3971877,-0.18609375)(5.2971873,-0.48609376)
\psframe[linewidth=0.04,dimen=outer](5.7971873,-0.7860938)(4.5971875,-1.0860938)
\psellipse[linewidth=0.04,dimen=outer](5.1971874,1.1139063)(1.3,0.1)
\psellipse[linewidth=0.04,dimen=outer](7.5971875,0.51390624)(0.1,0.5)
\psline[linewidth=0.04cm](5.8971877,0.41390625)(7.7971873,0.61390626)
\psline[linewidth=0.04cm](5.8971877,0.21390624)(7.8971877,0.41390625)
\psellipse[linewidth=0.04,dimen=outer](5.1971874,0.9639062)(0.1,0.25)
\psline[linewidth=0.04cm,arrowsize=0.05291667cm 2.0,arrowlength=1.4,arrowinset=0.4]{->}(4.7971873,-0.88609374)(1.7971874,-0.88609374)
\usefont{T1}{ptm}{m}{n}
\rput(5.2971873,1.6239063){helicopter}
\usefont{T1}{ptm}{m}{n}
\rput(5.2704687,-1.1760937){missile}
\usefont{T1}{ptm}{m}{n}
\rput(5.0753126,0.32390624){5000 kg}
\usefont{T1}{ptm}{m}{n}
\rput(5.1953125,-1.5760938){50 kg}
\usefont{T1}{ptm}{m}{n}
\rput(9.28,1.3239063){BEFORE}
\usefont{T1}{ptm}{m}{n}
\rput(10.039375,0.82390624){v$_{i1}$ = 275 \ms}
\usefont{T1}{ptm}{m}{n}
\rput(1.269375,0.82390624){v$_{f1}$ = ? \ms}
\usefont{T1}{ptm}{m}{n}
\rput(1.459375,0.32390624){v$_{f2}$ = 700 \ms}
\usefont{T1}{ptm}{m}{n}
\rput(10.039375,0.32390624){v$_{i2}$ = 275 \ms}
\usefont{T1}{ptm}{m}{n}
\rput(0.5675,1.3239063){AFTER}
\end{pspicture}
}
\end{center}
\caption{helicopter and missile}
\end{figure}
\westep{Analyse the question and list what is given}
$m_1$ = 5000 kg\\
$m_2$ = 50 kg\\
$v_{i1}$ = $v_{i2}$ = 275 \ms\\
$v_{f1}$ = ?\\
$v_{f2}$ = 700 \ms\\

\westep{Apply the Law of Conservation of momentum}
The helicopter and missile are connected initially and move at the same velocity. We will therefore combine their masses and change the momentum equation as follows:
\begin{eqnarray*}
p_{i}&=& p_{f}\\
(m_1 + m_2)v_{i} &=& m_1v_{f1} + m_2v_{f2}\\
(5000\ekg + 50\ekg)(275 \ems) &=& (5000 \ekg)(v_{f1}) + (50 \ekg)(700 \ems)\\
1388750\ekg\cdot\ems -35000\ekg\cdot\ems &=& (5000 \ekg)(v_{f1})\\
v_{f1} &=& 270,75 \ems
\end{eqnarray*}
Note that speed is asked and not velocity, therefore no direction is included in the answer.
}
\end{wex}

\begin{wex}{Conservation of Momentum 3}
{A bullet of mass 50 g travelling horizontally at 500 \ms strikes a stationary wooden block of mass 2 kg resting on a smooth horizontal surface. The bullet goes through the block and comes out on the other side at 200 \ms. Calculate the speed of the block after the bullet has come out the other side.}
{\westep{Draw rough sketch of the situation}
\begin{figure}[H]
\begin{center}
\scalebox{1} % Change this value to rescale the drawing.
{
\begin{pspicture}(0,-2.1907814)(8.664375,2.1907814)
\usefont{T1}{ptm}{m}{n}
\rput(4.3896875,-0.20296875){2 kg}
\usefont{T1}{ptm}{m}{n}
\rput(2.9396875,1.9970312){BEFORE}
\usefont{T1}{ptm}{m}{n}
\rput(1.4390625,-0.90296876){v$_{i1}$ = 500 \ms}
\usefont{T1}{ptm}{m}{n}
\rput(7.1590624,-1.4229687){v$_{f1}$ = 200 \ms}
\usefont{T1}{ptm}{m}{n}
\rput(6.9290624,-1.9629687){v$_{f2}$ = ? \ms}
\usefont{T1}{ptm}{m}{n}
\rput(4.2678127,0.8570312){v$_{i2}$ = 0 \ms (stationary)}
\usefont{T1}{ptm}{m}{n}
\rput(7.5471873,2.0170312){AFTER}
\usefont{T1}{ptm}{m}{n}
\rput(1.785,-0.44296876){50 g = 0,05 kg}
\psframe[linewidth=0.04,dimen=outer](5.436875,0.38703126)(3.336875,-0.81296873)
\psframe[linewidth=0.04,dimen=outer](1.4081801,0.14703125)(0.996875,-0.03680646)
\rput{-88.25549}(1.2851367,1.43653){\psarc[linewidth=0.04](1.3830425,0.05582807){0.04964515}{0.0}{180.0}
\psline[linewidth=0.04](1.4326876,0.05582807)(1.3333974,0.05582807)}
\psline[linewidth=0.04cm,arrowsize=0.05291667cm 2.0,arrowlength=1.4,arrowinset=0.4]{->}(1.576875,0.04703125)(2.556875,0.04703125)
\psframe[linewidth=0.04,linestyle=dashed,dash=0.16cm 0.16cm,dimen=outer](6.22818,0.14703125)(5.816875,-0.03680646)
\rput{-88.25549}(5.9584026,6.254296){\psarc[linewidth=0.04,linestyle=dashed,dash=0.16cm 0.16cm](6.2030425,0.05582807){0.04964515}{0.0}{180.0}
\psline[linewidth=0.04,linestyle=dashed,dash=0.16cm 0.16cm](6.2526875,0.05582807)(6.1533976,0.05582807)}
\psline[linewidth=0.04cm,linestyle=dashed,dash=0.16cm 0.16cm,arrowsize=0.05291667cm 2.0,arrowlength=1.4,arrowinset=0.4]{->}(6.396875,0.04703125)(7.376875,0.04703125)
\end{pspicture}
}
\end{center}
\end{figure}
\westep{Choose a frame of reference}
We will choose to the right as positive.\\

\westep{Apply the Law of Conservation of momentum}
\begin{eqnarray*}
p_{i}&=& p_{f}\\
m_1v_{i1} + m_2v_{i2} &=& m_1v_{f1} + m_2v_{f2}\\
(0,05 \ekg)(+500 \ems) + (2 \ekg)(0 \ems) &=& (0,05 \ekg)(+200 \ems) + (2 \ekg)(v_{f2})\\
25 + 0 - 10 &=& 2\ v_{f2}\\
v_{f2} &=& 7,5 \ems \mbox{in the same direction as the bullet}
\end{eqnarray*}
}
\end{wex}

\subsection{Physics in Action: Impulse}
A very important application of impulse is improving safety and reducing injuries. In many cases, an object needs to be brought to rest from a certain initial velocity. This means there is a certain specified change in momentum. If the time during which the momentum changes can be increased then the force that must be applied will be less and so it will cause less damage. This is the principle behind arrestor beds for trucks, airbags, and bending your knees when you jump off a chair and land on the ground.

\subsubsection{Air-Bags in Motor Vehicles}
Air bags are used in motor vehicles because they are able to reduce the effect of the force experienced by a person during an accident. Air bags extend the time required to stop the momentum of the driver and passenger. During a collision, the motion of the driver and passenger carries them towards the windshield. If they are stopped by a collision with the windshield, it would result in a large force exerted over a short time in order to bring them to a stop. If instead of hitting the windshield, the driver and passenger hit an air bag, then the time of the impact is increased. Increasing the time of the impact results in a decrease in the force.

\subsubsection{Padding as Protection During Sports}
The same principle explains why wicket keepers in cricket use padded gloves or why there are padded mats in gymnastics. In cricket, when the wicket keeper catches the ball, the padding is slightly compressible, thus reducing the effect of the force on the wicket keepers hands. Similarly, if a gymnast falls, the padding compresses and reduces the effect of the force on the gymnast's body.

\subsubsection{Arrestor Beds for Trucks}
An arrestor bed is a patch of ground that is softer than the road. Trucks use these when they have to make an emergency stop. When a trucks reaches an arrestor bed the time interval over which the momentum is changed is increased. This decreases the force and causes the truck to slow down.

\subsubsection{Follow-Through in Sports}
In sports where rackets and bats are used, like tennis, cricket, squash, badminton and baseball, the hitter is often encouraged to follow-through when striking the ball. High speed films of the collisions between bats/rackets and balls have shown that following through increases the time over which the collision between the racket/bat and ball occurs. This increase in the time of the collision causes an increase in the velocity change of the ball. This means that a hitter can cause the ball to leave the racket/bat faster by following through. In these sports, returning the ball with a higher velocity often increases the chances of success.

\subsubsection{Crumple Zones in Cars}
Another safety application of trying to reduce the force experienced is in crumple zones in cars. When two cars have a collision, two things can happen:
\begin{enumerate}
\item the cars bounce off each other, or
\item the cars crumple together.
\end{enumerate}

Which situation is more dangerous for the occupants of the cars? When cars bounce off each other, or rebound, there is a larger change in momentum and therefore a larger impulse. A larger impulse means that a greater force is experienced by the occupants of the cars. When cars crumple together, there is a smaller change in momentum and therefore a smaller impulse. The smaller impulse means that the occupants of the cars experience a smaller force. Car manufacturers use this idea and design crumple zones into cars, such that the car has a greater chance of crumpling than rebounding in a collision. Also, when the car crumples, the change in the car's momentum happens over a longer time. Both these effects result in a smaller force on the occupants of the car, thereby increasing their chances of survival.

\Activity{type}{Egg Throw}{This activity demonstrates the effect of impulse and how it is used to improve safety. Have two learners hold up a bed sheet or large piece of fabric. Then toss an egg at the sheet. The egg should not break, because the collision between the egg and the bed sheet lasts over an extended period of time since the bed sheet has some give in it. By increasing the time of the collision, the force of the impact is minimised. Take care to aim at the sheet, because if you miss the sheet, you will definitely break the egg and have to clean up the mess!}

\Exercise{title}{
\begin{enumerate}
\item{A canon, mass 500 kg, fires a shell, mass 1 kg, horizontally to the right at 500 \ms. What is the magnitude and direction of the initial recoil velocity of the canon?}
\item{A trolley of mass 1 kg is moving with a speed of 3 \ms. A block of wood, mass 0,5 kg, is dropped vertically into the trolley. Immediately after the collision, the speed of the trolley and block is 2 \ms. By way of calculation, show whether momentum is conserved in the collision.}
\item{A 7200 kg empty railway truck is stationary. A fertiliser firm loads 10800 kg fertiliser into the truck. A second, identical, empty truck is moving at 10 \ms when it collides with the loaded truck.
\begin{enumerate}
\item If the empty truck stops completely immediately after the collision, use a conservation law to calculate the velocity of the loaded truck immediately after the collision.
\item Calculate the distance that the loaded truck moves after collision, if a constant frictional force of 24 kN acts on the truck.
\end{enumerate}}
\item{A child drops a squash ball of mass 0,05 kg. The ball strikes the ground with a velocity of 4 \ms and rebounds with a velocity of 3 \ms. Does the law of conservation of momentum apply to this situation? Explain.}
\item{A bullet of mass 50 g travelling horizontally at 600 \ms strikes a stationary wooden block of mass 2 kg resting on a smooth horizontal surface. The bullet gets stuck in the block.
\begin{enumerate}
\item Name and state the principle which can be applied to find the speed of the block-and-bullet system after the bullet entered the block.
\item Calculate the speed of the bullet-and-block system immediately after impact.
\item If the time of impact was 5 x 10$^{-4}$ seconds, calculate the force that the bullet exerts on the block during impact.
\end{enumerate}}
\end{enumerate}
\practiceinfo
\begin{tabular}[h]{cccccc}
(1.) 01x6 & (2.) 01x7 & (3.) 01x8 & (4.) 01x9 & (5.) 01xa &
\end{tabular}
}


%\pagebreak[4]
\section{Torque and Levers}
\subsection{Torque}

This chapter has dealt with forces and how they lead to motion in a straight line. In this section, we examine how forces lead to rotational motion.

When an object is fixed or supported at one point and a force acts on it a distance away from the support, it tends to make the object turn. The moment of force or \textit{torque} (symbol, $\tau$ read \textit{tau}) is defined as the product of the distance from the support or pivot ($r$) and the component of force perpendicular to the object, $F_{\perp}$.

\equ{\tau=F_{\perp} \cdot r}{eq:fmig11:torque}

Torque can be seen as a rotational force. The unit of torque is N$\cdot$m and torque is a vector quantity. Some examples of where torque arises are shown in Figures~\ref{fig:fmig11:torque:seesaw}, \ref{fig:fmig11:torque:propellor} and \ref{fig:fmig11:torque:spanner}.

\begin{figure}[H]
\begin{center}
\begin{pspicture}(-5,0)(5,2)
%\psgrid[gridcolor=lightgray]
\SpecialCoor
\pspolygon[fillcolor=lightgray,fillstyle=solid](-0.25,0)(0,1)(0.25,0)
\psline[linewidth=4pt](-4,1)(4,1)
\psline[linewidth=2pt]{->}(3,2)(3,1)
\uput[dr](3,2){$F$}
\pcline[offset=8pt]{|->}(0,1)(3,1)
\lput*{:U}{$r$}
\psarc[arrows=<-](2,1){1}{-90}{-10}
\uput[ul](2,0){$\tau$}
\end{pspicture}
\caption{The force exerted on one side of a see-saw causes it to swing.}
\label{fig:fmig11:torque:seesaw}
\end{center}
\end{figure}

\begin{figure}[H]
\begin{center}
\begin{pspicture}(0.4,-0.6)(9.6,2)
%\psgrid[gridcolor=lightgray]
\rput(5.5,0){\pscurve[xunit=0.4,yunit=0.4,linewidth=0.5pt,fillstyle=solid,fillcolor=lightgray]%
(0,0)(0.82,0.1)(2.8,0.345)(5.37,0.55)(7.84,0.582)(8.75,0.507)(9.43,0.36)(9.98,0.1)(10,0)(9.98,-0.1)(9.43,-0.36)(8.75,-0.507)(7.84,-0.582)(5.37,-0.55)(2.8,-0.345)(0.82,-0.1)(0,0)}
\rput{180}(4.5,0){\pscurve[xunit=0.4,yunit=0.4,linewidth=0.5pt,fillstyle=solid,fillcolor=lightgray]%
(0,0)(0.82,0.1)(2.8,0.345)(5.37,0.55)(7.84,0.582)(8.75,0.507)(9.43,0.36)(9.98,0.1)(10,0)(9.98,-0.1)(9.43,-0.36)(8.75,-0.507)(7.84,-0.582)(5.37,-0.55)(2.8,-0.345)(0.82,-0.1)(0,0)}
\pscircle[linewidth=0.5pt,fillstyle=solid,fillcolor=lightgray](5,0){0.5}
\psline[linewidth=2pt]{->}(9,2)(9,0.25)
\uput[dr](9,2){$F$}

\rput(5,-1){\pcline[offset=0.6]{|->}(0,1)(4,1)
\lput*[fillcolor=lightgray,fillstyle=solid]{:U}{$r$}
\psarc[arrows=<-](3,1){1}{-90}{-10}
\uput[ul](3,0){$\tau$}
}
\end{pspicture}
\caption{The force exerted on the edge of a propeller causes the propeller to spin.}
\label{fig:fmig11:torque:propellor}
\end{center}
\end{figure}
\Tip{Loosening a bolt: if you are trying to loosen (or tighten) a bolt, apply the force on the spanner further away from the bolt, as this results in a greater torque to the bolt making it easier to loosen.}
\begin{figure}[H]
\begin{center}
\begin{pspicture}(0,-2)(6.6,2)
%\psgrid[gridcolor=lightgray]
\SpecialCoor
%bolt
\rput(0.4,0){\pspolygon[fillstyle=solid,fillcolor=lightgray](0,0)(0.5,0.87)(1.5,0.87)(2,0)(1.5,-0.87)(0.5,-0.87)}

%spanner
\psline[linewidth=1.5pt,xunit=1.5,yunit=1.5](0.5,0.87)(1.5,0.87)(2,0)(1.5,-0.87)(0.5,-0.87)
\rput(0.4,0){\psline[linewidth=1.5pt](0.5,0.87)(1.5,0.87)(2,0)(1.5,-0.87)(0.5,-0.87)}
\psline[linewidth=1.5pt](0.9,0.87)(0.75,1.3)
\psline[linewidth=1.5pt](0.9,-0.87)(0.75,-1.3)
\psarc[linewidth=1.5pt](6,0){0.5}{-90}{90}
\pscircle[linewidth=1.5pt](6,0){0.25}
\psline[linewidth=1.5pt](6,-0.5)(2.7,-0.5)
\psline[linewidth=1.5pt](6,0.5)(2.7,0.5)

%force
\psline[linewidth=2pt]{->}(6,2)(6,0.5)
\uput[dr](6,2){$F$}

\pcline[offset=0pt]{|->}(1.4,0)(6,0)
\lput*{:U}{$r$}
\psarc[arrows=<-](5,-0.5){1}{-90}{-10}
\uput[ul](5,-1.5){$\tau$}

\end{pspicture}
\caption{The force exerted on a spanner helps to loosen the bolt.}
\label{fig:fmig11:torque:spanner}
\end{center}
\end{figure}

For example in Figure~\ref{fig:fmig11:torque:spanner}, if a force $F$ of 10~N is applied perpendicularly to the spanner at a distance $r$ of 0,3~m from the centre of the bolt, then the torque applied to the bolt is:
\begin{eqnarray*}
\tau&=&F_{\perp} \cdot r\\
&=&(10\,\eN)(0,3\emm)\\
&=&3\,\mathrm{N\cdot m}
\end{eqnarray*}
If the force of 10~N is now applied at a distance of 0,15~m from the centre of the bolt, then the torque is:
\begin{eqnarray*}
\tau&=&F_{\perp} \cdot r\\
&=&(10\,\eN)(0,15\emm)\\
&=&1,5\,\mathrm{N\cdot m}
\end{eqnarray*}
This shows that there is less torque when the force is applied closer to the bolt than further away.



\Tip{Any component of a force exerted parallel to an object will not cause the object to turn. Only perpendicular components cause turning.}



\begin{wex}{Merry-go-round}{Several children are playing in the park. One child pushes the merry-go-round with a force of 50~N. The diameter of the merry-go-round is 3,0~m. What torque does the child apply if the force is applied perpendicularly at point A?
\begin{center}
\begin{pspicture}(-2,-2)(2,2)
%\psgrid
\pscircle(0,0){1.5}
\uput[l](-1.5,0){A}
\psline{<->}(-1.5,0)(1.5,0)
\pcline[offset=8pt,linestyle=none](-1.5,0)(1.5,0)
\lput{:U}{diameter = 3~m}
\psline[linewidth=2pt]{->}(-1.5,2)(-1.5,0)
\uput[l](-1.5,2){$F$}
\end{pspicture}
\end{center}}{
\westep{Identify what has been given}
The following has been given:
\begin{itemize}
\item{the force applied, $F$~=~50~N}
\item{the diameter of the merry-go-round, $2r$~=~3~m, therefore $r$~=~1,5~m.}
\end{itemize}
The quantities are in SI units.

\westep{Decide how to approach the problem}
We are required to determine the torque applied to the merry-go-round. We can do this by using:
\nequ{\tau=F_{\perp}\cdot r}
We are given $F_{\perp}$ and we are given the diameter of the merry-go-round. Therefore, $r$ = 1,5~m.

\westep{Solve the problem}
\begin{eqnarray*}
\tau&=&F_{\perp}\cdot r\\
&=&(50\eN)(1,5\emm)\\
&=&75\,\mathrm{N\cdot m}
\end{eqnarray*}

\westep{Write the final answer}
75~$\mathrm{N\cdot m}$ of torque is applied to the merry-go-round.
}
\end{wex}
\Tip{Torques: the direction of a torque is either clockwise or anticlockwise. When torques are added, choose one direction as positive and the opposite direction as negative. If equal clockwise and anticlockwise torques are applied to an object, they will cancel out and there will be no net turning effect.}
\begin{wex}{Flat tyre}{Kevin is helping his dad replace the flat tyre on the car. Kevin has been asked to undo all the wheel nuts. Kevin holds the spanner at the same distance for all nuts, but applies the force at two angles (90\deg\ and 60\deg). If Kevin applies a force of 60~N, at a distance of 0,3~m away from the nut, which angle is the best to use?
Prove your answer by means of calculations.
\begin{center}
\psset{xunit=0.70,yunit=0.70}
\begin{pspicture}(0,-2)(14,2)
%\psgrid[gridcolor=lightgray]
\SpecialCoor
%bolt
\rput(0.4,0){\pspolygon[fillstyle=solid,fillcolor=lightgray](0,0)(0.5,0.87)(1.5,0.87)(2,0)(1.5,-0.87)(0.5,-0.87)}
%spanner
\psline[linewidth=1.5pt,xunit=1.5,yunit=1.5](0.5,0.87)(1.5,0.87)(2,0)(1.5,-0.87)(0.5,-0.87)
\rput(0.4,0){\psline[linewidth=1.5pt](0.5,0.87)(1.5,0.87)(2,0)(1.5,-0.87)(0.5,-0.87)}
\psline[linewidth=1.5pt](0.9,0.87)(0.75,1.3)
\psline[linewidth=1.5pt](0.9,-0.87)(0.75,-1.3)
\psarc[linewidth=1.5pt](6,0){0.35}{-90}{90}
\pscircle[linewidth=1.5pt](6,0){0.172}
\psline[linewidth=1.5pt](6,-0.5)(2.7,-0.5)
\psline[linewidth=1.5pt](6,0.5)(2.7,0.5)
%force
\psline[linewidth=2pt]{->}(6,2)(6,0.5)
\uput[dr](6,2){$F$}
\pcline[offset=0pt]{|->}(1.4,0)(6,0)
\lput*{:U}{$r$}
\rput(7,0){
%bolt
\rput(0.4,0){\pspolygon[fillstyle=solid,fillcolor=lightgray](0,0)(0.5,0.87)(1.5,0.87)(2,0)(1.5,-0.87)(0.5,-0.87)}
%spanner
\psline[linewidth=1.5pt,xunit=1.5,yunit=1.5](0.5,0.87)(1.5,0.87)(2,0)(1.5,-0.87)(0.5,-0.87)
\rput(0.4,0){\psline[linewidth=1.5pt](0.5,0.87)(1.5,0.87)(2,0)(1.5,-0.87)(0.5,-0.87)}
\psline[linewidth=1.5pt](0.9,0.87)(0.75,1.3)
\psline[linewidth=1.5pt](0.9,-0.87)(0.75,-1.3)
\psarc[linewidth=1.5pt](6,0){0.35}{-90}{90}
\pscircle[linewidth=1.5pt](6,0){0.172}
\psline[linewidth=1.5pt](6,-0.5)(2.7,-0.5)
\psline[linewidth=1.5pt](6,0.5)(2.7,0.5)
%force
\psline[linewidth=2pt]{->}(5.25,1.89)(6,0.5)
\uput[dl](5.25,1.89){$F$}
\psarc{<->}(6,0.5){0.75}{120}{180}
\uput[ul](5.9,0.5){\small{$60^{\circ}$}}
\psline[linestyle=dashed]{->}(6,1.89)(6,0.5)
\uput[dr](6,1.89){$F_{\perp}$}
\pcline[offset=0pt]{|->}(1.4,0)(6,0)
\lput*{:U}{$r$}
}
\end{pspicture}
\end{center}
}{\westep{Identify what has been given}
The following has been given:
\begin{itemize}
\item{the force applied, $F$~=~60~N}
\item{the angles at which the force is applied, $\theta=90^{\circ}$ and $\theta=60^{\circ}$}
\item{the distance from the centre of the nut at which the force is applied, $r$~=~0,3~m}
\end{itemize}
The quantities are in SI units.\\

\westep{Decide how to approach the problem}
We are required to determine which angle is more better to use. This means that we must find which angle gives the higher torque. We can use
\nequ{\tau=F_{\perp}\cdot r}
to determine the torque. We are given $F$ for each situation. $F_{\perp}=F\sin\theta$ and we are given $\theta$. We are also given the distance away from the nut, at which the force is applied.\\

\westep{Solve the problem for $\theta=90^{\circ}$}
$F_{\perp}=F$
\begin{eqnarray*}
\tau&=&F_{\perp}\cdot r\\
&=&(60\eN)(0,3\emm)\\
&=&18\,\mathrm{N\cdot m}
\end{eqnarray*}

\westep{Solve the problem for $\theta=60^{\circ}$}
\begin{eqnarray*}
\tau&=&F_{\perp}\cdot r\\
&=&F\sin\theta \cdot r\\
&=&(60\eN)\sin(\theta)(0,3\emm)\\
&=&15,6\,\mathrm{N\cdot m}
\end{eqnarray*}

\westep{Write the final answer}
The torque from the perpendicular force is greater than the torque from the force applied at $60^{\circ}$. Therefore, the best angle is $90^{\circ}$.}
\end{wex}

\subsection{Mechanical Advantage and Levers}
We can use our knowledge about the moments of forces (torque) to determine whether situations are balanced. For example two mass pieces are placed on a seesaw as shown in Figure~\ref{fig:fmig11:torque:balance}. The one mass is 3 kg and the other is 6 kg. The masses are placed at distances of 2 m and 1 m (respectively) from the pivot. By looking at the clockwise and anti-clockwise moments, we can determine whether the seesaw will pivot (move) or not. If the sum of the clockwise and anti-clockwise moments is zero, the seesaw is in equilibrium (i.e.\@{} balanced).

\begin{figure}[H]
\begin{center}
\begin{pspicture}(-5,0)(5,3.6)
%\psgrid[gridcolor=lightgray]
\SpecialCoor
\rput(1,0){\pspolygon[fillcolor=lightgray,fillstyle=solid](-0.25,0)(0,1)(0.25,0)}
\psframe(-3,1)(-2,2)
\psframe(3,1)(4,3)
\psline[linewidth=4pt](-3,1)(4,1)
\pcline[offset=8pt]{|-|}(-2.5,3)(1,3)
\lput*{:U}{2 m}
\pcline[offset=8pt]{|-|}(1,3)(3.5,3)
\lput*{:U}{1 m}
\rput(-2.5,1.5){3 kg}
\rput(3.5,2){6 kg}
\psline{->}(-2.5,1)(-2.5,0)
\uput[r](-2.5,0.5){$F_1$}
\psline{->}(3.5,1)(3.5,0)
\uput[r](3.5,0.5){$F_2$}
\end{pspicture}
\caption{The moments of force are balanced.}
\label{fig:fmig11:torque:balance}
\end{center}
\end{figure}

The clockwise moment can be calculated as follows:
\begin{eqnarray*}
\tau_1 &=& F_{\perp}\cdot r\\
\tau_1 &=& (6 \ekg)(9,8 \emss)(1 m)\\
\tau_1 &=& 58,8 N \cdot m~\mbox{clockwise}
\end{eqnarray*}


The anti-clockwise moment can be calculated as follows:
\begin{eqnarray*}
\tau_2 &=& F_{\perp}\cdot r\\
\tau_2 &=& (3 \ekg)(9,8 \emss)(2 m)\\
\tau_2 &=& 58,8 N \cdot m~ \mbox{anti-clockwise}
\end{eqnarray*}

The sum of the moments of force will be zero:

The resultant moment is zero as the clockwise and anti-clockwise moments of force are in opposite directions and therefore cancel each other. \\

As we see in Figure~\ref{fig:fmig11:torque:balance}, we can use different distances away from a pivot to balance two different forces. This principle is applied to a lever to make lifting a heavy object much easier.

\Definition{Lever}{A lever is a rigid object that is used with an appropriate fulcrum or pivot point to multiply the mechanical force that can be applied to another object.}

\begin{figure}[h]
\begin{center}
\begin{pspicture}(-5,0)(5,3.6)
%\psgrid[gridcolor=lightgray]
\SpecialCoor
\rput(1,0){\pspolygon[fillcolor=lightgray,fillstyle=solid](-0.25,0)(0,1)(0.25,0)}
\rput{-10}(0,0.2){\psline[linewidth=4pt](-3,1)(4,1)}
\psline[linewidth=1pt]{->}(-2.8,2.8)(-2.8,1.8)
\psline[linewidth=2pt]{<-}(4,2.7)(4,0.7)
\uput[u](-2.8,2.8){effort}
\uput[u](4,2.7){load}
\end{pspicture}
\caption{A lever is used to put in a small effort to get out a large load.}
\label{fig:fmig11:torque:mechanicaladvantage}
\end{center}
\end{figure}

\begin{IFact}{Archimedes reputedly said: \textit{Give me a lever long enough and a fulcrum on which to place it, and I shall move the world.}}\end{IFact}

The concept of getting out more than the effort is termed mechanical advantage, and is one example of the principle of moments. The lever allows one to apply a smaller force over a greater distance. For instance to lift a certain unit of weight with a lever with an effort of half a unit we need a distance from the fulcrum in the effort's side to be twice the distance of the weight's side. It also means that to lift the weight 1 meter we need to push the lever for 2 meters. The amount of work done is always the same and independent of the dimensions of the lever (in an ideal lever). The lever only allows to trade force for distance.

Ideally, this means that the mechanical advantage of a system is the ratio of the force that performs the work (output or load) to the applied force (input or effort), assuming there is no friction in the system. In reality, the mechanical advantage will be less than the ideal value by an amount determined by the amount of friction.

\begin{equation*}
\mathrm{mechanical~advantage}=\frac{\mathrm{load}}{\mathrm{effort}}
\end{equation*}

For example, you want to raise an object of mass 100~kg. If the pivot is placed as shown in Figure~\ref{fig:fmig11:torque:mechanicaladvantage:example}, what is the mechanical advantage of the lever?

\begin{figure}[htbp]
\begin{center}
\begin{pspicture}(-5,0)(5,3.6)
%\psgrid[gridcolor=lightgray]
\SpecialCoor
\pcline{|-|}(-2.8,3)(1,3)
\lput*{:U}{1~m}
\pcline{|-|}(1,3)(4,3)
\lput*{:U}{0.5~m}
\rput(1,0){\pspolygon[fillcolor=lightgray,fillstyle=solid](-0.25,0)(0,1)(0.25,0)}
\rput{-10}(0,0.2){\psline[linewidth=4pt](-3,1)(4,1)}
\psline[linewidth=1pt]{->}(-2.8,2.8)(-2.8,1.8)
\psline[linewidth=2pt]{<-}(4,2.7)(4,0.7)
\uput[dl](-2.8,2.8){$F$?}
\uput[dr](4,2.7){100~kg}
\end{pspicture}
\caption{A lever is used to put in a small effort to get out a large load.}
\label{fig:fmig11:torque:mechanicaladvantage:example}
\end{center}
\end{figure}

In order to calculate mechanical advantage, we need to determine the load and effort.

\Tip{Effort is the input force and load is the output force.}

The load is easy to calculate, it is simply the weight of the 100~kg object.
\nequ{F_{load}=m \cdot g = 100\ekg \cdot 9,8\emss = 980\eN}

The effort is found by balancing torques.
\begin{eqnarray*}
F_{load}\cdot r_{load} &=& F_{effort}\cdot r_{effort} \\
980\eN \cdot 0.5\emm &=&F_{effort} \cdot 1\emm\\
F_{effort}&=&\frac{980\eN \cdot 0.5\emm}{1\emm}\\
&=&490\eN
\end{eqnarray*}

The mechanical advantage is:
\begin{eqnarray*}
\mathrm{mechanical~advantage}&=&\frac{\mathrm{load}}{\mathrm{effort}}\\
&=&\frac{980\eN}{490\eN}\\
&=&2
\end{eqnarray*}

Since mechanical advantage is a ratio, it does not have any units.

\Extension{Pulleys}{Pulleys change the direction of a tension force on a flexible material, e.g.\@{} a rope or cable. In addition, pulleys can be ``added together'' to create mechanical advantage, by having the flexible material looped over several pulleys in turn. More loops and pulleys increases the mechanical advantage.}

\subsection{Classes of levers}

{\bf{Class 1 levers}}\\
In a class 1 lever the fulcrum is between the effort and the load. Examples of class 1 levers are the seesaw, crowbar and equal-arm balance. The mechanical advantage of a class 1 lever can be increased by moving the fulcrum closer to the load.

\begin{figure}[H]
\begin{center}
\scalebox{1} % Change this value to rescale the drawing.
{
\begin{pspicture}(0,-1.5953125)(6.773361,1.5953125)
\psline[linewidth=0.08cm](0.0,0.12842709)(6.7333612,0.1496875)
\pstriangle[linewidth=0.08,dimen=outer](3.6233609,-1.1503125)(1.22,1.34)
\psline[linewidth=0.04cm,arrowsize=0.05291667cm 2.0,arrowlength=1.4,arrowinset=0.4]{->}(0.03336104,0.1496875)(0.01336104,-1.2703125)
\usefont{T1}{ptm}{m}{n}
\rput(0.52695477,-0.5403125){effort}
\usefont{T1}{ptm}{m}{n}
\rput(6.2285175,0.7596875){load}
\usefont{T1}{ptm}{m}{n}
\rput(3.5857048,-1.4403125){fulcrum}
\psline[linewidth=0.04cm,arrowsize=0.05291667cm 2.0,arrowlength=1.4,arrowinset=0.4]{<-}(6.7333612,1.5496875)(6.7133613,0.1296875)
\end{pspicture}
}
\end{center}
\caption{Class 1 levers}
\end{figure}

{\bf{Class 2 levers}}\\
In class 2 levers the fulcrum is at the one end of the bar, with the load closer to the fulcrum and the effort on the other end of bar. The mechanical advantage of this type of lever can be increased by increasing the length of the bar. A bottle opener or wheel barrow are examples of class 2 levers.
\begin{figure}[H]
\begin{center}
\scalebox{1} % Change this value to rescale the drawing.
{
\begin{pspicture}(0,-1.5925)(7.36,1.5725)
\psline[linewidth=0.08cm](0.5866389,0.1312396)(7.32,0.1525)
\pstriangle[linewidth=0.08,dimen=outer](0.61,-1.1475)(1.22,1.34)
\psline[linewidth=0.04cm,arrowsize=0.05291667cm 2.0,arrowlength=1.4,arrowinset=0.4]{->}(3.72,0.1525)(3.7,-1.2675)
\usefont{T1}{ptm}{m}{n}
\rput(6.713594,0.8625){effort}
\usefont{T1}{ptm}{m}{n}
\rput(4.215156,-0.5375){load}
\usefont{T1}{ptm}{m}{n}
\rput(0.57234377,-1.4375){fulcrum}
\psline[linewidth=0.04cm,arrowsize=0.05291667cm 2.0,arrowlength=1.4,arrowinset=0.4]{<-}(7.32,1.5525)(7.3,0.1325)
\end{pspicture}
}
\end{center}
\caption{Class 2 levers}
\end{figure}


{\bf{Class 3 levers}}\\
In class 3 levers the fulcrum is also at the end of the bar, but the effort is between the fulcrum and the load. An example of this type of lever is the human arm.
\begin{figure}[H]
\begin{center}
\scalebox{1} % Change this value to rescale the drawing.
{
\begin{pspicture}(0,-1.5925)(7.36,1.5725)
\psline[linewidth=0.08cm](0.5866389,0.1312396)(7.32,0.1525)
\pstriangle[linewidth=0.08,dimen=outer](0.61,-1.1475)(1.22,1.34)
\psline[linewidth=0.04cm,arrowsize=0.05291667cm 2.0,arrowlength=1.4,arrowinset=0.4]{->}(7.32,0.1525)(7.3,-1.2675)
\usefont{T1}{ptm}{m}{n}
\rput(4.313594,0.6625){effort}
\usefont{T1}{ptm}{m}{n}
\rput(6.8151565,-0.5375){load}
\usefont{T1}{ptm}{m}{n}
\rput(0.57234377,-1.4375){fulcrum}
\psline[linewidth=0.04cm,arrowsize=0.05291667cm 2.0,arrowlength=1.4,arrowinset=0.4]{<-}(3.72,1.5525)(3.7,0.1325)
\end{pspicture}
}
\end{center}
\caption{Class 3 levers}
\end{figure}

\Exercise{title}{
\begin{enumerate}
\item {Riyaad applies a force of 120 N on a spanner to undo a nut.
\begin{enumerate}
\item Calculate the moment of the force if he applies the force 0,15 m from the bolt.
\item The nut does not turn, so Riyaad moves his hand to the end of the spanner and applies the same force 0,2 m away from the bolt. Now the nut begins to move. Calculate the force. Is it bigger or smaller than before?
\item Once the nuts starts to turn, the moment needed to turn it is less than it was to start it turning. It is now 20 N$\cdot$m. Calculate the new moment of force that Riyaad now needs to apply 0,2 m away from the nut.
\end{enumerate}}

\item {Calculate the clockwise and anticlockwise moments of force in the figure below to see if the see-saw is balanced.
\begin{figure}[H]
\begin{center}
\scalebox{1} % Change this value to rescale the drawing.
{
\begin{pspicture}(0,-1.48)(7.115,1.48)
\psline[linewidth=0.08cm,arrowsize=0.05291667cm 2.0,arrowlength=1.4,arrowinset=0.4]{-}(0.0,-0.04)(6.98,-0.02)
\pstriangle[linewidth=0.08,dimen=outer](3.51,-1.06)(1.1,1.06)
\psline[linewidth=0.04cm,arrowsize=0.05291667cm 2.0,arrowlength=1.4,arrowinset=0.4]{->}(0.98,-0.04)(0.96,-1.46)
\psline[linewidth=0.04cm,arrowsize=0.05291667cm 2.0,arrowlength=1.4,arrowinset=0.4]{->}(5.98,-0.06)(6.0,-1.46)
\psline[linewidth=0.04cm,arrowsize=0.05291667cm 2.0,arrowlength=1.4,arrowinset=0.4]{->}(1.0,0.7)(3.5,0.7)
\psline[linewidth=0.04cm,arrowsize=0.05291667cm 2.0,arrowlength=1.4,arrowinset=0.4]{->}(5.98,0.7)(3.5,0.7)
\psline[linewidth=0.04cm](3.48,1.46)(3.48,-0.04)
\psdots[dotsize=0.2](0.98,0.7)
\psdots[dotsize=0.2](5.98,0.72)
\psline[linewidth=0.02cm](1.0,0.66)(1.0,-0.12)
\psline[linewidth=0.02cm](0.94,0.68)(0.94,0.0)
\psline[linewidth=0.02cm](6.0,0.7)(6.0,-0.06)
\psline[linewidth=0.02cm](5.94,0.68)(5.94,-0.04)
\usefont{T1}{ptm}{m}{n}
\rput(1.778125,1.07){1,5 m}
\usefont{T1}{ptm}{m}{n}
\rput(4.851875,1.07){3 m}
\usefont{T1}{ptm}{m}{n}
\rput(1.5203125,-0.73){900 N}
\usefont{T1}{ptm}{m}{n}
\rput(6.6246877,-0.73){450 N}
\end{pspicture}
}
\end{center}
\end{figure}
}
\item {Jeffrey uses a force of 390 N to lift a load of 130 kg.
\begin{figure}[H]
\begin{center}
\scalebox{1} % Change this value to rescale the drawing.
{
\begin{pspicture}(0,-1.21)(7.2333612,1.22)
\psline[linewidth=0.08cm](0.0,1.1787395)(6.886722,0.04126041)
\pstriangle[linewidth=0.08,dimen=outer](3.4633608,-1.2)(1.1,1.84)
\psline[linewidth=0.04cm,arrowsize=0.05291667cm 2.0,arrowlength=1.4,arrowinset=0.4]{->}(0.03336104,1.2)(0.01336104,-0.22)
\usefont{T1}{ptm}{m}{n}
\rput(0.6735173,0.31){390 N}
\usefont{T1}{ptm}{m}{n}
\rput(5.908986,-0.79){130 kg}
\psframe[linewidth=0.04,dimen=outer](7.2333612,-0.4)(6.433361,-1.2)
\psline[linewidth=0.02cm](6.833361,0.1)(6.633361,-0.4)
\psline[linewidth=0.02cm](6.633361,-0.4)(6.633361,-1.2)
\psline[linewidth=0.02cm](6.833361,0.0)(7.033361,-0.4)
\psline[linewidth=0.02cm](7.033361,-0.4)(7.033361,-1.2)
\psdots[dotsize=0.12](6.833361,0.0)
\end{pspicture}
}
\end{center}
\end{figure}
\begin{enumerate}
\item Calculate the mechanical advantage of the lever that he is using.
\item What type of lever is he using? Give a reason for your answer.
\item If the force is applied 1 m from the pivot, calculate the maximum distance between the pivot and the load.
\end{enumerate}}

\item {A crowbar is used to lift a box of weight 400 N. The box is placed 75 cm from the pivot. A crow bar is a class 1 lever.
\begin{enumerate}
\item Why is a crowbar a class 1 lever? Draw a diagram to explain your answer.
\item What force F needs to be applied at a distance of 1,25 m from the pivot to balance the crowbar?
\item If force F was applied at a distance of 2 m, what would the magnitude of F be?
\end{enumerate}}

\item {A wheelbarrow is used to carry a load of 200 N. The load is 40 cm from the pivot and the force F is applied at a distance of 1,2 m from the pivot.
\begin{enumerate}
\item What type of lever is a wheelbarrow?
\item Calculate the force F that needs to be applied to lift the load.
\end{enumerate}}

\item {The bolts holding a car wheel in place is tightened to a torque of 90 N $\cdot$ m. The mechanic has two spanners to undo the bolts, one with a length of 20 cm and one with a length of 30 cm. Which spanner should he use? Give a reason for your answer by showing calculations and explaining them.}

\end{enumerate}
\practiceinfo

\begin{tabular}[h]{cccccc}
 (1.) 01xb & (2.) 01xc & (3.) 01xd & (4.) 01xe & (5.) 01xf & (6.) 01xg
\end{tabular}

}
\summary{VPkkv}
\begin{description}
\item[Newton's First Law] Every object will remain at rest or in
uniform motion in a straight line unless it is made to change its
state by the action of an \emph{unbalanced force}.
\item[Newton's Second Law] The resultant force acting on a body
will cause the body to accelerate in the direction of the
resultant force The acceleration of the body is directly proportional to the magnitude of
the resultant force and inversely proportional to the mass of the object.
\item[Newton's Third Law] If body A exerts a force on body B then body B
will exert an equal but opposite force on body A.
\item[Newton's Law of Universal Gravitation] Every body in the universe exerts a force on every other body. The force is directly proportional to the product of the masses of the bodies and inversely proportional to the square of the distance between them.
\item[Equilibrium] Objects at rest or moving with constant velocity
are in {\em equilibrium} and have a {\em zero} {\em resultant force}.
\item[Equilibrant] The {\em equilibrant} of any number of forces is
the single force required to produce equilibrium.
\item[Triangle Law for Forces in Equilibrium] Three forces in
equilibrium can be represented in magnitude and direction by the
three sides of a triangle taken in order.
\item[Momentum] The {\em momentum} of an object is defined as its
mass multiplied by its velocity.
\item[Momentum of a System] The {\em total momentum of a system}
is the sum of the momenta of each of the objects in the system.
\item[Principle of Conservation of Linear Momentum:] `The total linear
momentum of an isolated system is constant' or `In an isolated
system the total momentum before a collision (or explosion) is equal
to the total momentum after the collision (or explosion)'.
\item[Law of Momentum:] The applied resultant force acting on an
object is equal to the rate of change of the object's momentum and
this force is in the direction of the change in momentum.
\end{description}

% Presentation in summary: SIYAVULA-PRESENTATION:http://cnx.org/content/m38991/latest/#slidesharefigure

\begin{eocexercises}{}
\subsubsection{Forces and Newton's Laws}
\begin{enumerate}
\item{[SC 2003/11] A constant, resultant force acts on a body which can move freely in a straight line. Which physical quantity will remain constant?
\begin{enumerate}
\item{acceleration}
\item{velocity}
\item{momentum}
\item{kinetic energy}
\end{enumerate}}

\item{[SC 2005/11 SG1] Two forces, 10~N and 15~N, act at an angle at the same point.

\begin{center}
\begin{pspicture}(-2,0)(2,2.4)
%\psgrid
\SpecialCoor
\psline{->}(0,0)({2;150})
\psline{->}(0,0)({3;50})
\uput[dl]({2;150}){10 N}
\uput[dr]({3;50}){15 N}
\end{pspicture}
\end{center}

Which of the following \textbf{cannot} be the resultant of these two forces?
\begin{enumerate}
\item {2 N}
\item {5 N}
\item {8 N}
\item {20 N}
\end{enumerate}}

\item{A concrete block weighing 250 N is at rest on an inclined surface at an angle of 20$^\circ$. The magnitude of the normal force, in newtons, is
\begin{enumerate}
\item 250
\item 250 cos 20$^\circ$
\item 250 sin 20$^\circ$
\item 2500 cos 20$^\circ$
\end{enumerate}} %Answer:B

\item {A 30 kg box sits on a flat frictionless surface. Two forces of 200 N each are applied to the box as shown in the diagram. Which statement best describes the motion of the box? \begin{enumerate} 
\item The box is lifted off the surface.
\item The box moves to the right.
\item The box does not move.
\item The box moves to the left. \end{enumerate}
\begin{center}
\scalebox{0.8} % Change this value to rescale the drawing.
{
\begin{pspicture}(0,-2.62)(8.42,2.615)
\psframe[linewidth=0.03,dimen=outer](5.88,0.54)(2.74,-2.6)
\psline[linewidth=0.04cm](0.62,-2.58)(8.4,-2.6)
\psline[linewidth=0.04cm,arrowsize=0.05291667cm 2.0,arrowlength=1.4,arrowinset=0.4]{->}(2.76,-1.4)(0.0,-1.4)
\psline[linewidth=0.03cm,arrowsize=0.05291667cm 2.0,arrowlength=1.4,arrowinset=0.4]{->}(5.84,0.5)(8.3,2.6)
\usefont{T1}{ptm}{m}{n}
\rput(4.3076563,-1.07){30kg}
\psline[linewidth=0.04cm,linestyle=dotted,dotsep=0.16cm](5.86,0.52)(8.3,0.58)
\usefont{T1}{ptm}{m}{n}
\rput(7.1735935,0.79){$30^{\circ}$}
\usefont{T1}{ptm}{m}{n}
\rput(1.73125,-1.13){200N}
\usefont{T1}{ptm}{m}{n}
\rput(6.39125,1.61){200N} \end{pspicture}}
\end{center}}

\item {A concrete block weighing 200 N is at rest on an inclined surface at an angle of 20$^\circ$. The normal reaction, in newtons, is \begin{enumerate} \item 200 \item 200 cos 20$^\circ$ \item 200 sin 20$^\circ$ \item 2000 cos 20$^\circ$ \end{enumerate}
}

\item{[SC 2003/11]A box, mass $m$, is at rest on a rough horizontal surface. A force of constant magnitude $F$ is then applied on the box at an angle of 60\deg  to the horizontal, as shown.
\begin{center}
\begin{pspicture}(-3,-0.4)(3,2)
\SpecialCoor
%\psgrid[gridcolor=lightgray]
\psline[linewidth=2pt](-3,0)(3,0)
\multido{\n=-2.8+0.1}{59}
{\rput(\n,-0.2){\psline(-0.2,0)(0,0.2)}}
\psframe[linewidth=1pt](-0.5,0)(0.5,1)
\psline(-2,1)(-0.5,1)
\psline[linewidth=2pt]{->}(-0.5,1)({2;120})
\uput[l](-1,1.73){$F$}
\uput[l](-2,1){A}
\uput[r](0.5,1){B}
\rput(0,0.5){$m$}
\uput[ul](-0.7,1){\small{$60^\circ$}}
\uput[ur](-3,0){rough surface}
\end{pspicture}
\end{center}
If the box has a uniform horizontal acceleration of magnitude, $a$, the frictional force acting on the box is $\ldots$
\begin{enumerate}
\item {$F\cos 60^\circ -ma$ in the direction of A}
\item {$F\cos 60^\circ -ma$ in the direction of B}
\item {$F\sin 60^\circ -ma$ in the direction of A}
\item {$F\sin 60^\circ -ma$ in the direction of B}
\end{enumerate}}

\item{[SC 2002/11 SG] Thabo stands in a train carriage which is moving eastwards. The train suddenly brakes. Thabo continues to move eastwards due to the effect of:
\begin{enumerate}
\item his inertia.
\item the inertia of the train.
\item the braking force on him.
\item a resultant force acting on him.
\end{enumerate}}

\item{[SC 2002/11 HG1] A body slides down a frictionless inclined plane. Which one of the following physical quantities will remain constant throughout the motion?
\begin{enumerate}
\item velocity
\item momentum
\item acceleration
\item kinetic energy
\end{enumerate}}

\item{[SC 2002/11 HG1] A body moving at a \emph{CONSTANT VELOCITY} on a horizontal plane, has a number of unequal forces acting on it. Which one of the following statements is TRUE?

\begin{enumerate}
\item At least two of the forces must be acting in the same direction.
\item The resultant of the forces is zero.
\item Friction between the body and the plane causes a resultant force.
\item The vector sum of the forces causes a resultant force which acts in the direction of motion.
\end{enumerate}}

\item{[IEB 2005/11 HG] Two masses of $m$ and $2m$ respectively are connected by an elastic band on a frictionless surface. The masses are pulled in opposite directions by two forces each of magnitude $F$, stretching the elastic band and holding the masses stationary.

\begin{center}
\begin{pspicture}(0,0)(10,2)
\SpecialCoor
%\psgrid[gridcolor=lightgray]
\psline[linewidth=2pt](0,0)(10,0)
\uput[u](0,1){$F$}
\psline{<-}(0,1)(1,1)
\psframe(1,0)(3,2)
\rput(2,1){$m$}
\psline(3,1)(3.5,1)
\psframe(3.5,0.75)(6.5,1.25)
\rput(5,1){elastic band}
\psline(6.5,1)(7,1)
\psframe(7,0)(9,2)
\rput(8,1){$2m$}
\psline{->}(9,1)(10,1)
\uput[u](10,1){$F$}
\end{pspicture}
\end{center}
Which of the following gives the magnitude of the tension in the elastic band?
\begin{enumerate}
\item {zero}
\item {$\frac{1}{2}F$}
\item {$F$}
\item {$2F$}
\end{enumerate}}

\item{[IEB 2005/11 HG] A rocket takes off from its launching pad, accelerating up into the air.

\begin{center}
\begin{pspicture}(-1,0)(1,5)
%\psgrid
%Drawing a rocket is a somewhat non-trivial exercise
%This picture is derived from
%http://www.geocities.com/rocketguy_101/ogive/OgiveNoseCones.html
\psset{xunit=0.5,yunit=0.5}
\psline(-1,4)(-2,2)(2,2)(1,4)
\psframe(-0.6,1.6)(0.6,2)
\rput{90}(1,0){\rput(5,2){
\psplot{-3}{4.45}{10.025 2 exp x 2 exp sub 0.5 exp 9.975 sub}
\psplot{-3}{4.45}{10.025 2 exp x 2 exp sub 0.5 exp 9.975 sub neg 2 sub}}}
\psdot[dotsize=3pt](0,4)
\psline{<->}(0,10)(0,0)
\uput[dr](0,10){$\vec{F}$}
\uput[ur](0,0){$\vec{W}$}
\psline{<-}(0.6,1.8)(2,1.8)
\uput[r](2,1.8){tail nozzle}
\end{pspicture}
\end{center}
The rocket accelerates because the magnitude of the upward force, $F$ is greater than the magnitude of the rocket's weight, $W$. Which of the following statements \textbf{best} describes how force $F$ arises?
\begin{enumerate}
\item {$F$ is the force of the air acting on the base of the rocket.}
\item {$F$ is the force of the rocket's gas jet \textit{pushing down} on the air.}
\item {$F$ is the force of the rocket's gas jet \textit{pushing down} on the ground.}
\item {$F$ is the reaction to the force that the rocket exerts on the gases which escape out through the tail nozzle.}
\end{enumerate}}

\item{[SC 2001/11 HG1]
A box of mass 20~kg rests on a smooth horizontal surface. What will happen to the box if two forces each of magnitude 200~N are applied simultaneously to the box as shown in the diagram.
\begin{center}
\begin{pspicture}(0,0)(5,2)
\SpecialCoor
%\psgrid[gridcolor=lightgray]
\psline[linewidth=2pt](0,0)(5,0)
\psframe(2.5,0)(3.5,1)
\rput(3,0.5){20~kg}
\psline{->}(2.5,0.75)(1.5,0.75)
\uput[u](1.5,0.75){200~N}
\rput(3.5,1){\psline{->}(0,0)(1;30)\psline(0,0)(1;0)\rput(0.8,0.2){30\deg}\uput[u](1;30){200~N}}
\end{pspicture}
\end{center}
The box will:
\begin{enumerate}
\item {be lifted off the surface.}
\item {move to the left.}
\item {move to the right.}
\item {remain at rest.}
\end{enumerate}}

\item{[SC 2001/11 HG1]
A 2~kg mass is suspended from spring balance X, while a 3~kg mass is suspended from spring balance Y. Balance X is in turn suspended from the 3~kg mass. Ignore the weights of the two spring balances.

\begin{center}
\begin{pspicture}(1,-1)(3,3)
%\begin{pspicture}(1,1)(2,2)
%\psgrid[gridcolor=gray]
\def\springbalance{\psframe(0,0.25)(0.25,1)\pscircle(0.125,0.125){0.125}

\psline(0.125,1)(0.125,1.5)}
\psline(1,3)(3,3)
\rput(1.875,1.5){\springbalance}
\uput[ur](2.175,2){Y}
\psframe(1.5,1)(2.5,1.5)
\rput(2,1.25){3~kg}

\rput(0,-2){\rput(1.875,1.5){\springbalance}
\uput[ur](2.175,2){X}
\psframe(1.5,1)(2.5,1.5)
\rput(2,1.25){2~kg}}
\end{pspicture}
\end{center}
The readings (in N)  on balances X and Y are as follows:
\begin{center}
\begin{tabular}{|c|l|l|}\hline
&\textbf{X}&\textbf{Y}\\\hline
(A)&20&30\\\hline
(B)&20&50\\\hline
(C)&25&25\\\hline
(D)&50&50\\\hline
\end{tabular}
\end{center}
}

\item{[SC 2002/03 HG1]
$P$ and $Q$ are two forces of equal magnitude applied simultaneously to a body at X.
\begin{center}
\begin{pspicture}(-1,-0.4)(2.4,2)
%\psgrid[gridcolor=lightgray]
\psline{->}(0,0)(2;0)\uput[r](2;0){P}
\psline{->}(0,0)(2;120)\uput[r](2;120){Q}
\psarc(0,0){0.5}{0}{120}\uput[r](0.1;60){$\theta$}
\uput[d](0,0){X}
\end{pspicture}
\end{center}
As the angle $\theta$ between the forces is \textbf{decreased} from 180\deg\ to 0\deg, the magnitude of the resultant of the two forces will
\begin{enumerate}
\item {initially increase and then decrease.}
\item {initially decrease and then increase.}
\item {increase only.}
\item {decrease only.}
\end{enumerate}}

\item{[SC 2002/03 HG1]
The graph below shows the velocity-time graph for a moving object:
\begin{center}
\begin{pspicture}(0,0)(3,3)
%\psgrid[gridcolor=lightgray]
\psset{unit=0.5}
\psline{<->}(0,0)(0,4.5)\uput[u](0,4.5){$v$}
\psline{->}(0,2.5)(5,2.5)\uput[r](5,2.5){$t$}
\psline(0,4)(4,0)
\end{pspicture}
\end{center}
Which of the following graphs could best represent the relationship between the resultant force applied to the object and time?

\begin{center}
\begin{tabular}{cccc}
\begin{pspicture}(0,0)(2,2)
%\psgrid[gridcolor=lightgray]
\psline(0,0)(0,2)\uput[l](0,2){F}
\psline(0,1)(2,1)\uput[r](1.7,0.8){t}
\psplot{0}{1.3}{x 2 exp neg 2 add}
\end{pspicture}
&
\begin{pspicture}(0,0)(2,2)
%\psgrid[gridcolor=lightgray]
\psline(0,0)(0,2)\uput[l](0,2){F}
\psline(0,1)(2,1)\uput[r](1.7,0.8){t}
\psplot{0}{1.3}{0.5}
\end{pspicture}
&
\begin{pspicture}(0,0)(2,2)
%\psgrid[gridcolor=lightgray]
\psline(0,0)(0,2)\uput[l](0,2){F}
\psline(0,1)(2,1)\uput[r](1.7,0.8){t}
\psline(0,0.5)(0.75,0.5)(0.75,1.5)(1.5,1.5)
\end{pspicture}
&
\begin{pspicture}(0,0)(2,2)
%\psgrid[gridcolor=lightgray]
\psline(0,0)(0,2)\uput[l](0,2){F}
\psline(0,1)(2,1)\uput[r](1.7,0.8){t}
\psline(0,1.5)(1.5,0.2)
\end{pspicture}
\\
(a)&(b)&(c)&(d)\\
\end{tabular}
\end{center}}

\item{[SC 2002/03 HG1]
Two blocks each of mass 8~kg are in contact with each other and are accelerated along a frictionless surface by a force of 80~N as shown in the diagram. The force which block Q will exert on block P is equal to ...
\begin{center}
\begin{pspicture}(0,0)(3.2,1.6)
%\psgrid[gridcolor=lightgray]
\psline[linewidth=2pt](0,0)(3.2,0)
\psframe(1,0)(2,1)\rput(1.5,0.5){8~kg}\uput[u](1.5,1){Q}
\psframe(2,0)(3,1)\rput(2.5,0.5){8~kg}\uput[u](2.5,1){P}
\psline{->}(0,0.5)(1,0.5)
\uput[u](0.5,0.5){80~N}
\end{pspicture}
\end{center}
\begin{enumerate}
\item {0~N}
\item {40~N}
\item {60~N}
\item {80~N}
\end{enumerate}}

\item{[SC 2002/03 HG1]
Three 1~kg mass pieces are placed on top of a 2~kg trolley. When a force of magnitude $F$ is applied to the trolley, it experiences an acceleration $a$.
\begin{center}
\begin{pspicture}(0,0.2)(4,2.6)
%\psgrid[gridcolor=lightgray]
\psline[linewidth=2pt](0,0.25)(4,0.25)
\psframe(1,.5)(3,1.5)\rput(2,1){2~kg}
\rput(1.5,0.5){\pscircle(0,0){0.25}}

\rput(2.5,0.5){\pscircle(0,0){0.25}}
\rput(1,1.5){\psframe(0,0)(1,0.5)\rput(0.5,0.25){1~kg}}
\rput(2,1.5){\psframe(0,0)(1,0.5)\rput(0.5,0.25){1~kg}}
\rput(1.5,2){\psframe(0,0)(1,0.5)\rput(0.5,0.25){1~kg}}
\psline{->}(3,1)(4,1)\uput[u](4,1){$F$}
\end{pspicture}
\end{center}
If one of the 1~kg mass pieces falls off while $F$ is still being applied, the trolley will accelerate at ...
\begin{enumerate}
\item {$\frac{1}{5}a$}
\item {$\frac{4}{5}a$}
\item {$\frac{5}{4}a$}
\item {$5a$}
\end{enumerate}}

\item{[IEB 2004/11 HG1] A car moves along a horizontal road at constant velocity. Which of the following statements is true?
\begin{enumerate}
\item {The car is not in equilibrium.}
\item {There are no forces acting on the car.}
\item {There is zero resultant force.}
\item {There is no frictional force.}
\end{enumerate}
}

\item{[IEB 2004/11 HG1] A crane lifts a load vertically upwards at constant speed. The upward force exerted on the load is $F$. Which of the following statements is correct?
\begin{enumerate}
\item {The acceleration of the load is 9,8 m.s$^{-2}$ downwards.}
\item {The resultant force on the load is F.}
\item {The load has a weight equal in magnitude to F.}
\item {The forces of the crane on the load, and the weight of the load, are an example of a Newton's third law 'action-reaction' pair.}
\end{enumerate}
}

\item{[IEB 2004/11 HG1] A body of mass $M$ is at rest on a smooth horizontal surface with two forces applied to it as in the diagram below. Force $F_1$ is equal to $Mg$. The force $F_1$ is applied to the right at an angle $\theta$ to the horizontal, and a force of $F_2$ is applied horizontally to the left.

\begin{figure}[H]
\begin{center}
\scalebox{1} % Change this value to rescale the drawing.
{
\begin{pspicture}(0,-1.3)(7.0,1.33)
\definecolor{color1b}{rgb}{0.8,0.8,0.8}
\psframe[linewidth=0.04,dimen=outer,fillstyle=solid,fillcolor=color1b](7.0,-1.0)(0.0,-1.3)
\psframe[linewidth=0.04,dimen=outer](4.5,0.4)(3.0,-1.0)
\psline[linewidth=0.1cm](3.0,-0.5)(3.0,-0.5)
\psline[linewidth=0.06cm,arrowsize=0.05291667cm 2.0,arrowlength=1.4,arrowinset=0.4]{<-}(4.5,0.0)(6.3,0.0)
\usefont{T1}{ptm}{m}{n}
\rput(3.7532814,-0.29){M}
\usefont{T1}{ptm}{m}{n}
\rput(5.447969,0.21){F$_2$}
\psline[linewidth=0.06cm,arrowsize=0.05291667cm 2.0,arrowlength=1.4,arrowinset=0.4]{->}(1.64,1.3)(3.04,0.3)
\psline[linewidth=0.02cm,linestyle=dashed,dash=0.16cm 0.16cm](1.1,0.3)(3.0,0.3)
\usefont{T1}{ptm}{m}{n}
\rput(2.2614062,0.51){$\theta$}
\usefont{T1}{ptm}{m}{n}
\rput(2.96375,1.07){F$_1$=Mg}
\end{pspicture}
}
\end{center}
\end{figure}
How is the body affected when the angle $\theta$ is increased?
\begin{enumerate}
\item {It remains at rest.}
\item {It lifts up off the surface, and accelerates towards the right.}
\item {It lifts up off the surface, and accelerates towards the left.}
\item {It accelerates to the left, moving along the smooth horizontal surface.}
\end{enumerate}
}

\item{[IEB 2003/11 HG1] Which of the following statements correctly explains why a passenger in a car, who is not restrained by the seat belt, continues to move forward when the brakes are applied suddenly?
\begin{enumerate}
\item {The braking force applied to the car exerts an equal and opposite force on the passenger.}
\item {A forward force (called inertia) acts on the passenger.}
\item {A resultant forward force acts on the passenger.}
\item {A zero resultant force acts on the passenger.}
\end{enumerate}}

\item{[IEB 2004/11 HG1]

A rocket (mass 20 000 kg) accelerates from rest to 40 \ms in the first 1,6 seconds of its journey upwards into space.\\
The rocket's propulsion system consists of exhaust gases, which are pushed out of an outlet at its base.
\begin{enumerate}
\item{Explain, with reference to the appropriate law of Newton, how the escaping exhaust gases exert an upwards force (thrust) on the rocket.}
\item{What is the magnitude of the total thrust exerted on the rocket during the first 1,6~s?}
\item{An astronaut of mass 80 kg is carried in the space capsule. Determine the resultant force acting on him during the first 1,6 s.}
\item{Explain why the astronaut, seated in his chair, feels ``heavier'' while the rocket is launched.}
\end{enumerate}}

\item{[IEB 2003/11 HG1 - Sports Car]

\begin{enumerate}
\item{State Newton's Second Law of Motion.}
\item{A sports car (mass 1 000 kg) is able to accelerate uniformly from rest to 30 \ms in a minimum time of 6 s.
\begin{enumerate}
\item{Calculate the magnitude of the acceleration of the car.}
\item{What is the magnitude of the resultant force acting on the car during these 6 s?}
\end{enumerate}}

\item{The magnitude of the force that the wheels of the vehicle exert on the road surface as it accelerates is 7500 N. What is the magnitude of the retarding forces acting on this car?}

\item{By reference to a suitable Law of Motion, explain why a headrest is important in a car with such a rapid acceleration.}
\end{enumerate}}

\item{[IEB 2005/11 HG1]
A child (mass 18 kg) is strapped in his car seat as the car moves to the right at constant velocity along a straight level road. A tool box rests on the seat beside him.

\begin{figure}[H]
\begin{center}
\scalebox{1} % Change this value to rescale the drawing.
{
\begin{pspicture}(0,-2.070762)(2.7868114,2.1027718)
\rput{-12.697448}(0.04582672,0.21742497){\psellipse[linewidth=0.04,dimen=outer,fillstyle=solid](1.0,-0.09722832)(0.4,1.1)}
\pscustom[linewidth=0.2]
{
\newpath
\moveto(0.4,1.2027717)
\lineto(0.57634735,1.0076902)
\curveto(0.66452086,0.9101496)(0.8833988,0.70632786)(1.0141028,0.60004675)
\curveto(1.1448069,0.49376562)(1.3080151,0.27491096)(1.3405191,0.1623377)
\curveto(1.373023,0.04976416)(1.4117529,-0.22292902)(1.4179788,-0.38304865)
\curveto(1.4242047,-0.54316825)(1.3910129,-0.7746075)(1.2727604,-0.9885659)
}
\psframe[linewidth=0.04,dimen=outer,fillstyle=solid,fillcolor=black](0.5,1.3027717)(0.0,-1.6972283)
\psframe[linewidth=0.04,dimen=outer,fillstyle=solid,fillcolor=black](2.1,-1.2972283)(0.5,-1.6972283)
\psellipse[linewidth=0.04,dimen=outer,fillstyle=solid](1.5,-1.0972283)(0.9,0.2)
\rput{25.769327}(-0.40727013,-1.2140322){\psellipse[linewidth=0.04,dimen=outer,fillstyle=solid](2.45,-1.4972284)(0.15,0.6)}
\psellipse[linewidth=0.04,dimen=outer,fillstyle=solid](1.4,1.5027716)(0.5,0.6)
\rput{-99.886314}(1.235416,1.1684691){\psellipse[linewidth=0.04,dimen=outer,fillstyle=solid](1.1089275,0.06487083)(0.6,0.16503632)}
\psframe[linewidth=0.02,dimen=outer,fillstyle=solid](1.5,-0.6972283)(0.0,-1.6972283)
\usefont{T1}{ptm}{m}{n}
\rput(0.73609376,-1.1872283){tool box}
\end{pspicture}
}
\end{center}
\end{figure}

The driver brakes suddenly, bringing the car rapidly to a halt. There is negligible friction between the car seat and the box.

\begin{enumerate}
\item{Draw a labelled free-body diagram of \textbf{the forces acting on the child} during the time that the car is being braked.}
\item{Draw a labelled free-body diagram of \textbf{the forces acting on the box} during the time that the car is being braked.}
\item{What is the rate of change of the child's momentum as the car is braked to a standstill from a speed of 72 km.h$^{-1}$ in 4 s.}\\
\\
\textit{Modern cars are designed with safety features (besides seat belts) to protect drivers and passengers during collisions e.g.\@{} the crumple zones on the car's body. Rather than remaining rigid during a collision, the crumple zones allow the car's body to collapse steadily.}
\item{State Newton's second law of motion.}
\item{Explain how the crumple zone on a car reduces the force of impact on it during a collision.}
\end{enumerate}}

\item{[SC 2003/11 HG1]The total mass of a lift together with its load is 1 200 kg. It is moving downwards at a constant velocity of 9 \ms.

\begin{center}
\begin{pspicture}(0,0)(3,3)
\psframe(0,0)(2,2)
\psline[linewidth=3pt](1,3)(1,2)
\psline{->}(2.5,2.5)(2.5,1.5)
\uput[u](3.5,1.5){9 \ms}
\rput(1,1){1 200 kg}
\end{pspicture}
\end{center}

\begin{enumerate}
\item{What will be the magnitude of the force exerted by the cable on the lift while it is moving downwards at constant velocity? Give an explanation for your answer.}\\
\textit{The lift is now uniformly brought to rest over a distance of 18 m.}

\item{Calculate the magnitude of the acceleration of the lift.}
\item{Calculate the magnitude of the force exerted by the cable while the lift is being brought to rest.}
\end{enumerate}}

\item{A driving force of 800~N acts on a car of mass 600~kg.
\begin{enumerate}
\item{Calculate the car's acceleration.}
\item{Calculate the car's speed after 20 s.}
\item{Calculate the new acceleration if a frictional force of 50~N starts to act on the car after 20~s.}
\item{Calculate the speed of the car after another 20 s (i.e.\@{} a total of 40 s after the start).}
\end{enumerate}}

\item{[IEB 2002/11 HG1 - A Crate on an Inclined Plane]

Elephants are being moved from the Kruger National Park to the Eastern Cape. They are loaded into crates that are pulled up a ramp (an inclined plane) on frictionless rollers.\\
\\
The diagram shows a crate being held stationary on the ramp by means of a rope parallel to the ramp. The tension in the rope is 5 000 N.

\begin{figure}[H]
\begin{center}
\scalebox{1} % Change this value to rescale the drawing.
{
\begin{pspicture}(0,-1.0843273)(5.03,1.0909185)
\psline[linewidth=0.06cm](4.0,-0.04651485)(0.0,-1.0465149)
\psline[linewidth=0.06cm](0.0,-1.0465149)(5.0,-1.0465149)
\rput{14.333483}(0.04287787,-0.48082444){\psframe[linewidth=0.04,dimen=outer](2.7834227,0.3939155)(1.0834227,-0.53373486)}
\usefont{T1}{ptm}{m}{n}
\rput(2.0898438,-0.8565149){15$\degree$}
\usefont{T1}{ptm}{m}{n}
\rput(1.95,-0.07651485){Elephants}
\psline[linewidth=0.04cm,arrowsize=0.05291667cm 2.0,arrowlength=1.4,arrowinset=0.4]{->}(2.7,0.27348515)(4.04,0.67348516)
\usefont{T1}{ptm}{m}{n}
\rput{14.632455}(0.31923503,-0.90039885){\rput(3.6489062,0.7834851){5000 N}}
\end{pspicture}
}
\end{center}
\end{figure}


\begin{enumerate}
\item{Explain how one can deduce the following: ``The forces acting on the crate are in equilibrium''.}
\item{Draw a labelled free-body diagram of the forces acting on the crane and elephant. (Regard the crate and elephant as one object, and represent them as a dot. Also show the relevant angles between the forces.)}
\item{The crate has a mass of 800 kg. Determine the mass of the elephant.}
\item{The crate is now pulled up the ramp at a constant speed. How does the crate being pulled up the ramp at a constant speed affect the forces acting on the crate and elephant? Justify your answer, mentioning any law or principle that applies to this situation.}
\end{enumerate}}

\item{[IEB 2002/11 HG1 - Car in Tow]

Car A is towing Car B with a light tow rope. The cars move along a straight, horizontal road.

\begin{enumerate}
\item{Write down a statement of Newton's Second Law of Motion (in words).}
\item{As they start off, Car A exerts a forwards force of 600 N at its end of the tow rope. The force of friction on Car B when it starts to move is 200 N. The mass of Car B is 1 200 kg. Calculate the acceleration of Car B.}
\item{After a while, the cars travel at constant velocity. The force exerted on the tow rope is now 300 N while the force of friction on Car B increases. What is the magnitude and direction of the force of friction on Car B now?}
\item{Towing with a rope is very dangerous. A solid bar should be used in preference to a tow rope. This is especially true should Car A suddenly apply brakes. What would be the advantage of the solid bar over the tow rope in such a situation?}
\item{The mass of Car A is also 1 200 kg. Car A and Car B are now joined by a solid tow bar and the total braking force is 9 600 N. Over what distance could the cars stop from a velocity of 20 \ms?}
\end{enumerate}}

\item{[IEB 2001/11 HG1] - \textbf{Testing the Brakes of a Car}\\
\\
A braking test is carried out on a car travelling at 20 \ms. A braking distance of 30 m is measured when a braking force of 6 000 N is applied to stop the car.
\begin{enumerate}
\item{Calculate the acceleration of the car when a braking force of 6 000 N is applied.}
\item{\textbf{Show} that the mass of this car is 900 kg.}
\item{How long (in s) does it take for this car to stop from 20 \ms under the braking action described above?}
%\item{Calculate the rate of energy dissipation of this car when it is braked to a halt from 20 m.s$^{-1}$.} Only done in grade 12.
\item{ A trailer of mass 600 kg is attached to the car and the braking test is repeated from 20 \ms using the same braking force of 6 000 N. How much longer will it take to stop the car with the trailer in tow?}
\end{enumerate}}

\item{[IEB 2001/11 HG1] A rocket takes off from its launching pad, accelerating up into the air. Which of the following statements best describes the reason for the upward acceleration of the rocket?

\begin{enumerate}
\item {The force that the atmosphere (air) exerts underneath the rocket is greater than the weight of the rocket.}
\item {The force that the ground exerts on the rocket is greater than the weight of the rocket.}
\item {The force that the rocket exerts on the escaping gases is less than the weight of the rocket.}
\item {The force that the escaping gases exerts on the rocket is greater than the weight of the rocket.}
\end{enumerate}}

\item{[IEB 2005/11 HG] A box is held stationary on a smooth plane that is inclined at angle $\theta$ to the horizontal.

\begin{center}
\begin{pspicture}(0,0)(2.6,2.6)
\SpecialCoor
%\psgrid[gridcolor=lightgray]
\psline(0,0)(3;30)
\psline(0,0)(2.6,0)
\rput{30}(1,0.6){\psframe(0,0)(1,1)\psline{->}(0.5,0.5)(1.5,0.5)\psline{->}(0.5,0.5)(0.5,1.5)\uput[r](0.5,1.5){$N$}\uput[r](1.5,0.5){$F$}}
\psline{->}(1.2,1.3)(1.2,0.3)
\uput[r](1.2,0.3){$w$}
\uput[ur](0.6,0){$\theta$}
\end{pspicture}
\end{center}
$F$ is the force exerted by a rope on the box. $w$ is the weight of the box and $N$ is the normal force of the plane on the box. Which of the following statements is correct?
\begin{enumerate}
\item {$\tan \theta =\frac{F}{w}$}
\item {$\tan \theta =\frac{F}{N}$}
\item {$\cos \theta =\frac{F}{w}$}
\item {$\sin \theta =\frac{N}{w}$}
\end{enumerate}}

% --------------------

%forces in equilibrium
\item{[SC 2001/11 HG1]
As a result of three forces $F_1$, $F_2$ and $F_3$ acting on it, an object at point P is in equilibrium.

\begin{center}
\begin{pspicture}(-2,-2)(2,2)
\SpecialCoor
%\psgrid[gridcolor=lightgray]
\psline{->}(0,0)(1.5;30)
\psline{->}(0,0)(1.5;150)
\psline{->}(0,0)(1.5;270)
\uput[r](1.5;30){$F_2$}
\uput[l](1.5;150){$F_1$}
\uput[l](1.5;270){$F_3$}
\end{pspicture}
\end{center}
Which of the following statements is \textbf{not true} with reference to the three forces?
\begin{enumerate}
\item{The resultant of forces $F_1$, $F_2$ and $F_3$ is zero.}
\item{Forces $F_1$, $F_2$ and $F_3$ lie in the same plane.}
\item{Forces $F_3$ is the resultant of forces $F_1$ and $F_2$.}
\item{The sum of the components of all the forces in any chosen direction is zero.}
\end{enumerate}}

\item {A block of mass M is held stationary by a rope of negligible mass. The block rests on a frictionless plane which is inclined at $30^{\circ}$ to the horizontal. \\ %\scalebox{1} % Change this value to rescale the drawing.
\begin{center} \begin{pspicture}(0,-1.32)(4.0,1.32) \psline[linewidth=0.04cm](0.0,-1.3)(3.98,-1.3) \psline[linewidth=0.04cm](3.98,1.3)(3.98,-1.3) \psline[linewidth=0.04cm](0.0,-1.28)(3.96,1.28) \usefont{T1}{ptm}{m}{n} \rput(1.064375,-1.05){30$^{\circ}$} \psline[linewidth=0.04cm](1.08,0.22)(1.76,0.66) \usefont{T1}{ptm}{m}{n} \rput(1.5615625,0.13){M} \psline[linewidth=0.04cm](1.08,0.22)(1.42,-0.34) \psline[linewidth=0.04cm](1.74,0.66)(2.1,0.12) \psline[linewidth=0.04cm](2.7,1.3)(3.06,0.72) \psline[linewidth=0.04cm](2.86,1.0)(1.94,0.4) \end{pspicture} \end{center} \begin{enumerate} \item Draw a labelled force diagram which shows all the forces acting on the block. \item Resolve the force due to gravity into components that are parallel and perpendicular to the plane. \item Calculate the weight of the block when the force in the rope is 8N. \end{enumerate}
}

\item{[SC 2003/11] A heavy box, mass $m$, is lifted by means of a rope R which passes over a pulley fixed to a pole. A second rope S, tied to rope R at point P, exerts a horizontal force and pulls the box to the right. After lifting the box to a certain height, the box is held stationary as shown in the sketch below. Ignore the masses of the ropes. The tension in rope R is 5 850 N.

\begin{center}
\begin{pspicture}(0,0)(9,5)
\SpecialCoor
\psline[linewidth=2pt](0,0)(9,0) %ground
\psframe(2,0)(2.25,4.5)
\pscircle(2.125,4.75){0.25}
\pscircle(2.125,4.75){0.1}
\psline[linewidth=3pt](2,1)(0,3.7)
\psline(0,3.7)(2.125,5)
\psline(2.125,5)(6,3)(8.5,3)
\psline(6,3)(6,2)
\psframe(5,1)(7,2)
\psline[linestyle=dashed](6,3)(6,5)
\uput[ur](6,3){P}
\uput[u](8,3){rope S}
\uput[ul](6.1,3.2){$70^\circ$}
\uput[l](1.3,2){strut}
\uput[ur](4,4){rope R}
\rput(6,1.5){box}
\end{pspicture}
\end{center}

\begin{enumerate}
\item{Draw a diagram (with labels) of all the forces acting at the point P, when P is in equilibrium.}
\item{By resolving the force exerted by rope R into components, calculate the $\ldots$}
\begin{enumerate}
\item{magnitude of the force exerted by rope S.}
\item{mass, m, of the box.}
\end{enumerate}
\item{Will the tension in rope R, increase, decrease or remain the same if rope S is pulled further to the right (the length of rope R remains the same)? Give a reason for your choice.}
\end{enumerate}}

\item {A tow truck attempts to tow a broken down car of mass 400 kg. The coefficient of static friction is 0,60 and the coefficient of kinetic (dynamic) friction is 0,4. A rope connects the tow truck to the car. Calculate the force required:
\begin{enumerate}
\item to just move the car if the rope is parallel to the road.
\item to keep the car moving at constant speed if the rope is parallel to the road.
\item to just move the car if the rope makes an angle of 30$\degree$ to the road.
\item to keep the car moving at constant speed if the rope makes an angle of 30$\degree$ to the road.
\end{enumerate}}
%This question needs to be checked to make sure enough info is given

\item {A crate weighing 2000 N is to be lowered at constant speed down skids 4 m long, from a truck 2 m high.
\begin{enumerate}
\item If the coefficient of sliding friction between the crate and the skids is 0,30, will the crate need to be pulled down or held back?
\item How great is the force needed parallel to the skids?
\end{enumerate}}

\item {Block A in the figures below weighs 4 N and block B weighs 8 N. The coefficient of kinetic friction between all surfaces is 0,25. Find the force P necessary to drag block B to the left at constant speed if
\begin{enumerate}
\item A rests on B and moves with it
\item A is held at rest
\item A and B are connected by a light flexible cord passing around a fixed frictionless pulley
\end{enumerate}
\begin{figure}[H]
\begin{center}
\scalebox{1} % Change this value to rescale the drawing.
{
\begin{pspicture}(0,-1.4284375)(13.06,1.4284375)
\psline[linewidth=0.08cm](2.98,1.3884375)(2.98,-0.6115625)
\psline[linewidth=0.08cm](3.0,-0.6315625)(0.0,-0.6315625)
\psline[linewidth=0.08cm](5.02,-0.6315625)(8.02,-0.6315625)
\psline[linewidth=0.08cm](7.98,-0.6115625)(7.98,1.3884375)
\psline[linewidth=0.08cm](10.02,-0.6515625)(13.02,-0.6515625)
\psline[linewidth=0.08cm](12.98,-0.6115625)(12.98,1.3884375)
\psframe[linewidth=0.04,dimen=outer](1.98,-0.0115625)(0.98,-0.6115625)
\psframe[linewidth=0.04,dimen=outer](1.78,0.3884375)(1.18,-0.0115625)
\psframe[linewidth=0.04,dimen=outer](6.98,-0.0115625)(5.98,-0.6115625)
\psframe[linewidth=0.04,dimen=outer](6.78,0.3884375)(6.18,-0.0115625)
\psframe[linewidth=0.04,dimen=outer](11.98,-0.0115625)(10.98,-0.6115625)
\psframe[linewidth=0.04,dimen=outer](11.78,0.3884375)(11.18,-0.0115625)
\psline[linewidth=0.04cm,arrowsize=0.05291667cm 2.0,arrowlength=1.4,arrowinset=0.4]{->}(0.98,-0.2115625)(0.08,-0.2115625)
\psline[linewidth=0.04cm,arrowsize=0.05291667cm 2.0,arrowlength=1.4,arrowinset=0.4]{->}(5.98,-0.2115625)(5.08,-0.2115625)
\psline[linewidth=0.04cm,arrowsize=0.05291667cm 2.0,arrowlength=1.4,arrowinset=0.4]{->}(10.98,-0.2115625)(10.08,-0.2115625)
\usefont{T1}{ptm}{m}{n}
\rput(1.5065625,0.6984375){A}
\usefont{T1}{ptm}{m}{n}
\rput(1.4871875,-0.4015625){B}
\usefont{T1}{ptm}{m}{n}
\rput(6.4065623,0.6984375){A}
\usefont{T1}{ptm}{m}{n}
\rput(6.4871874,-0.4015625){B}
\usefont{T1}{ptm}{m}{n}
\rput(11.506562,0.6984375){A}
\usefont{T1}{ptm}{m}{n}
\rput(11.487187,-0.4015625){B}
\psline[linewidth=0.04cm](6.78,0.1884375)(7.98,0.1884375)
\pscircle[linewidth=0.04,dimen=outer](12.68,-0.0115625){0.2}
\psline[linewidth=0.04cm](11.78,0.1884375)(12.68,0.1884375)
\psline[linewidth=0.04cm](12.68,-0.2115625)(11.98,-0.2115625)
\psline[linewidth=0.08cm](12.68,-0.0115625)(12.98,-0.0115625)
\usefont{T1}{ptm}{m}{n}
\rput(0.47078124,0.0984375){P}
\usefont{T1}{ptm}{m}{n}
\rput(5.4707813,0.0984375){P}
\usefont{T1}{ptm}{m}{n}
\rput(10.470781,0.0984375){P}
\usefont{T1}{ptm}{m}{n}
\rput(1.4451562,-1.2015625){(a)}
\usefont{T1}{ptm}{m}{n}
\rput(6.4951563,-1.2015625){(b)}
\usefont{T1}{ptm}{m}{n}
\rput(11.485156,-1.2015625){(c)}
\end{pspicture}
}
\end{center}
\end{figure}
}
\end{enumerate}


\subsubsection{Gravitation}
\begin{enumerate}
\item{[SC 2003/11]An object attracts another with a gravitational force $F$. If the distance between the centres of the two objects is now decreased to a third ($\frac{1}{3}$) of the original distance, the force of attraction that the one object would exert on the other would become$\ldots$
\begin{enumerate}
\item {$\frac{1}{9}F$}
\item {$\frac{1}{3}F$}
\item {$3F$}
\item {$9F$}
\end{enumerate}}

\item{[SC 2003/11] An object is dropped from a height of 1 km above the Earth. If air resistance is ignored, the acceleration of the object is dependent on the $\dots$
\begin{enumerate}
\item {mass of the object}
\item {radius of the earth}
\item {mass of the earth}
\item {weight of the object}
\end{enumerate}}


\item {A man has a mass of 70 kg on Earth. He is walking on a new planet that has a mass four times that of the Earth and the radius is the same as that of the Earth ( M$_E$ = 6 x 10$^{24}$ kg, r$_E$ = 6 x 10$^6$ m )
\begin{enumerate}
\item Calculate the force between the man and the Earth.
\item What is the man's mass on the new planet?
\item Would his weight be bigger or smaller on the new planet? Explain how you arrived at your answer.
\end{enumerate}
}

\item {Calculate the distance between two objects, 5000 kg and 6 x 10$^{12}$ kg respectively, if the magnitude of the force between them is 3 x 10$^{?8}$ N.	}

\item {Calculate the mass of the Moon given that an object weighing 80 N on the Moon has a weight of 480 N on Earth and the radius of the Moon is 1,6 x 10$^{16}$ m.}

\item {The following information was obtained from a free-fall experiment to determine the value of $g$ with a pendulum.\\
Average falling distance between marks = 920 mm\\
Time taken for 40 swings = 70 s\\
Use the data to calculate the value of $g$.}	%[9,61 m.s-2]

\item {An astronaut in a satellite 1600 km above the Earth experiences gravitational force of the magnitude of 700 N on Earth.  The Earth's radius is 6400 km.  Calculate
\begin{enumerate}
\item The magnitude of the gravitational force which the astronaut experiences in the satellite.
\item The magnitude of the gravitational force on an object in the satellite which weighs 300 N on Earth.
\end{enumerate}}

\item {An astronaut of mass 70 kg on Earth lands on a planet which has half the Earth's radius and twice its mass.  Calculate the magnitude of the force of gravity which is exerted on him on the planet.}

\item {Calculate the magnitude of the gravitational force of attraction between two spheres of lead with a mass of 10 kg and 6 kg respectively if they are placed 50 mm apart.}

\item {The gravitational force between two objects is 1200 N.  What is the gravitational force between the objects if the mass of each is doubled and the distance between them halved?	}

\item {Calculate the gravitational force between the Sun with a mass of 2 x 10$^{30}$ kg and the Earth with a mass of 6 x 10$^{24}$ kg if the distance between them is 1,4 x 10$^8$ km.	}

\item {How does the gravitational force of attraction between two objects change when
\begin{enumerate}
\item the mass of each object is doubled.
\item the distance between the centres of the objects is doubled.
\item the mass of one object is halved, and the distance between the centres of the objects is halved.
\end{enumerate}}

\item {Read each of the following statements and say whether you agree or not. Give reasons for your answer and rewrite the statement if necessary:
\begin{enumerate}
\item The gravitational acceleration g is constant.
\item The weight of an object is independent of its mass.
\item G is dependent on the mass of the object that is being accelerated.
\end{enumerate}}

\item {An astronaut weighs 750 N on the surface of the Earth.
\begin{enumerate}
\item What will his weight be on the surface of Saturn, which has a mass 100 times greater than the Earth, and a radius 5 times greater than the Earth?
\item What is his mass on Saturn?
\end{enumerate}
}

\item {A piece of space garbage is at rest at a height 3 times the Earth's radius above the Earth's surface.  Determine its acceleration due to gravity. Assume the Earth's mass is 6,0 x 10$^{24}$ kg and the Earth's radius is 6400 km.}

\item {Your mass is 60 kg in Paris at ground level.  How much less would you weigh after taking a lift to the top of the Eiffel Tower, which is 405 m high? Assume the Earth's mass is 6,0~x~10$^{24}$ kg and the Earth's radius is 6400 km.}

\item { \begin{enumerate}
\item{State Newton's Law of Universal Gravitation.}
\item{Use Newton's Law of Universal Gravitation to determine the magnitude of the acceleration due to 	gravity on the Moon.\\
The mass of the Moon is 7,40 $\times$ 10$^{22}$ kg.\\
The radius of the Moon is 1,74 $\times$ 10$^6$ m.\\}
\item{Will an astronaut, kitted out in his space suit, jump higher on the Moon or on the Earth? Give a reason for your answer.
%Justify your answer by comparing the kinetic energy, the gravitational potential energy and the maximum height reached in both situations.%energy not done in grade 11
}
\end{enumerate}
}

\end{enumerate}


\subsubsection{Momentum}
\begin{enumerate}
\item{[SC 2003/11]A projectile is fired vertically upwards from the ground. At the highest point of its motion, the projectile explodes and separates into two pieces of equal mass. If one of the pieces is projected vertically upwards after the explosion, the second piece will $\ldots$
\begin{enumerate}
\item {drop to the ground at zero initial speed.}
\item {be projected downwards at the same initial speed at the first piece.}
\item {be projected upwards at the same initial speed as the first piece.}
\item {be projected downwards at twice the initial speed as the first piece.}
\end{enumerate}}

\item{[IEB 2004/11 HG1] A ball hits a wall horizontally with a speed of 15 \ms. It rebounds horizontally with a speed of 8 \ms. Which of the following statements about the system of the ball and the wall is \textbf{true}?
\begin{enumerate}
\item {The total linear momentum of the system is not conserved during this collision.}
\item {The law of conservation of energy does not apply to this system.}
\item {The change in momentum of the wall is equal to the change in momentum of the ball.}
\item {Energy is transferred from the ball to the wall.}
\end{enumerate}}

\item{[IEB 2001/11 HG1] A block of mass M collides with a stationary block of mass 2M. The two blocks move off together with a velocity of v. What is the velocity of the block of mass M immediately \textbf{before} it collides with the block of mass 2M?
\begin{enumerate}
\item {v}
\item {2v}
\item {3v}
\item {4v}
\end{enumerate}}

\item{[IEB 2003/11 HG1] A cricket ball and a tennis ball move horizontally towards you with the \underline{same momentum}. A cricket ball has greater mass than a tennis ball. You apply the same force in stopping each ball.\\
How does the time taken to stop each ball compare?
\begin{enumerate}
\item {It will take longer to stop the cricket ball.}
\item {It will take longer to stop the tennis ball.}
\item {It will take the same time to stop each of the balls.}
\item {One cannot say how long without knowing the kind of collision the ball has when stopping.}
\end{enumerate}}

\item{[IEB 2004/11 HG1] Two identical billiard balls collide head-on with each other. The first ball hits the second ball with a speed of V, and the second ball hits the first ball with a speed of 2V. After the collision, the first ball moves off in the opposite direction with a speed of 2V. Which expression correctly gives the speed of the second ball after the collision?
\begin{enumerate}
\item {V}
\item {2V}
\item {3V}
\item {4V}
\end{enumerate}}

\item{[SC 2002/11 HG1] Which one of the following physical quantities is the same as the rate of change of momentum?
\begin{enumerate}
\item resultant force
\item work
\item power
\item impulse
\end{enumerate}}

\item{[IEB 2005/11 HG] Cart X moves along a smooth track with momentum $p$. A resultant force $F$ applied to the cart stops it in time $t$. Another cart Y has only half the mass of X, but it has the same momentum $p$.

\begin{center}
\begin{pspicture}(0,0)(6,1.6)
\SpecialCoor
%\psgrid[gridcolor=lightgray]
\def\cart{\psframe(0,0.4)(1,1)\pscircle(0.2,0.2){0.2}\pscircle(0.8,0.2){0.2}}
\psline[linewidth=2pt](0,0)(6,0) %ground

%Cart X
\rput(0,0){\cart}
\uput[u](0.5,1){X}
\rput(0.5,0.7){$2m$}
\psline{->}(1,1)(1.5,1)
\uput[r](1.5,1){$p$}
\psline{<-}(1,0.5)(2,0.5)
\uput[r](2,0.5){$F$}

%Cart Y
\rput(3.5,0){
\rput(0,0){\cart}
\uput[u](0.5,1){Y}
\rput(0.5,0.7){$m$}
\psline{->}(1,1)(1.5,1)
\uput[r](1.5,1){$p$}
\psline{<-}(1,0.5)(2,0.5)
\uput[r](2,0.5){$F$}
}
\end{pspicture}
\end{center}
In what time will cart Y be brought to rest when the same resultant force $F$ acts on it?
\begin{enumerate}
\item {$\frac{1}{2}t$}
\item {$t$}
\item {$2t$}
\item {$4t$}
\end{enumerate}}

\item{[SC 2002/03 HG1]
A ball with mass $m$ strikes a wall perpendicularly with a speed, $v$. If it rebounds in the opposite direction with the same speed, $v$, the magnitude of the change in momentum will be ...
\begin{enumerate}
\item {$2mv$}
\item {$mv$}
\item {$\frac{1}{2}mv$}
\item {$0~mv$}
\end{enumerate}}

\item{Show that impulse and momentum have the same units.}
\item{A golf club exerts an average force of 3 kN on a ball of mass 0,06 kg. If the golf club is in contact with the golf ball for 5 x 10$^{-4}$ seconds, calculate
\begin{enumerate}
\item the change in the momentum of the golf ball.
\item the velocity of the golf ball as it leaves the club.
\end{enumerate}}
\item{During a game of hockey, a player strikes a stationary ball of mass 150 g. The graph below shows how the force of the ball varies with the time.
\begin{figure}[H]
\begin{center}
\scalebox{1} % Change this value to rescale the drawing.
{
\begin{pspicture}(0,-3.0284376)(8.459063,3.0084374)
\psline[linewidth=0.04cm,arrowsize=0.05291667cm 2.0,arrowlength=1.4,arrowinset=0.4]{->}(1.4940625,-2.5115626)(1.4940625,2.9884374)
\psline[linewidth=0.04cm,arrowsize=0.05291667cm 2.0,arrowlength=1.4,arrowinset=0.4]{->}(1.4940625,-2.5115626)(8.094063,-2.5115626)
\psline[linewidth=0.04cm](2.4940624,-2.4115624)(2.4940624,-2.6115625)
\psline[linewidth=0.04cm](3.4940624,-2.4115624)(3.4940624,-2.6115625)
\psline[linewidth=0.04cm](4.4940624,-2.4115624)(4.4940624,-2.6115625)
\psline[linewidth=0.04cm](5.4940624,-2.4115624)(5.4940624,-2.6115625)
\psline[linewidth=0.04cm](6.4940624,-2.4115624)(6.4940624,-2.6115625)
\usefont{T1}{ptm}{m}{n}
\rput(2.4954689,-2.8015625){0,1}
\usefont{T1}{ptm}{m}{n}
\rput(3.51,-2.8015625){0,2}
\usefont{T1}{ptm}{m}{n}
\rput(4.5025,-2.8015625){0,3}
\usefont{T1}{ptm}{m}{n}
\rput(5.51125,-2.8015625){0,4}
\usefont{T1}{ptm}{m}{n}
\rput(6.505625,-2.8015625){0,5}
\usefont{T1}{ptm}{m}{n}
\rput(1.0590625,-1.5015625){50}
\usefont{T1}{ptm}{m}{n}
\rput(1.0365624,-0.5015625){100}
\usefont{T1}{ptm}{m}{n}
\rput(1.0365624,0.4984375){150}
\usefont{T1}{ptm}{m}{n}
\rput(1.05375,1.4984375){200}
\psline[linewidth=0.04cm](1.3940625,-1.5115625)(1.5940624,-1.5115625)
\psline[linewidth=0.04cm](1.3940625,-0.5115625)(1.5940624,-0.5115625)
\psline[linewidth=0.04cm](1.3940625,0.4884375)(1.5940624,0.4884375)
\psline[linewidth=0.04cm](1.3940625,1.4884375)(1.5940624,1.4884375)
\psline[linewidth=0.04cm](1.4940625,-2.5115626)(4.0540624,0.4684375)
\psline[linewidth=0.04cm](4.0340624,0.4884375)(6.4340625,-2.5115626)
\psline[linewidth=0.02cm](1.4940625,-0.5115625)(7.0940623,-0.5115625)
\psline[linewidth=0.02cm](1.4940625,0.4884375)(7.0940623,0.4884375)
\psline[linewidth=0.02cm](1.3940625,1.4884375)(7.0940623,1.4884375)
\psline[linewidth=0.02cm](2.4940624,1.4884375)(2.4940624,-2.5115626)
\psline[linewidth=0.02cm](3.4940624,1.4884375)(3.4940624,-2.4115624)
\psline[linewidth=0.02cm](4.4940624,1.4884375)(4.4940624,-2.5115626)
\psline[linewidth=0.02cm](5.4940624,1.4884375)(5.4940624,-2.5115626)
\psline[linewidth=0.02cm](6.4940624,1.4884375)(6.4940624,-2.5115626)
\psline[linewidth=0.02cm](1.4940625,-1.5115625)(7.0940623,-1.5115625)
\usefont{T1}{ptm}{m}{n}
\rput(7.8010936,-2.8015625){Time (s)}
\usefont{T1}{ptm}{m}{n}
\rput(0.69359374,2.3984375){Force (N)}
\end{pspicture}
}
\end{center}
\end{figure}

\begin{enumerate}
\item What does the area under this graph represent?
\item Calculate the speed at which the ball leaves the hockey stick.
\item The same player hits a practise ball of the same mass, but which is made from a softer material. The hit is such that the ball moves off with the same speed as before. How will the \textbf{area}, the \textbf{height} and the \textbf{base} of the triangle that forms the graph, compare with that of the original ball?
\end{enumerate}}

\item{The fronts of modern cars are deliberately designed in such a way that in case of a head-on collision, the front would crumple. Why is it desirable that the front of the car should crumple?}

\item{A ball of mass 100 g strikes a wall horizontally at 10 \ms and rebounds at 8 \ms. It is in contact with the wall for 0,01 s.
\begin{enumerate}
\item Calculate the average force exerted by the wall on the ball.
\item Consider a lump of putty also of mass 100 g which strikes the wall at 10 \ms and comes to rest in 0,01 s against the surface. Explain qualitatively (no numbers) whether the force exerted on the putty will be less than, greater than of the same as the force exerted on the ball by the wall. Do not do any calculations.
\end{enumerate}}

\item{Shaun swings his cricket bat and hits a stationary cricket ball vertically upwards so that it rises to a height of 11,25 m above the ground. The ball has a mass of 125 g. Determine
\begin{enumerate}
\item the speed with which the ball left the bat.
\item the impulse exerted by the bat on the ball.
\item the impulse exerted by the ball on the bat.
\item for how long the ball is in the air.
\end{enumerate}}

\item{A glass plate is mounted horizontally 1,05 m above the ground. An iron ball of mass 0,4 kg is released from rest and falls a distance of 1,25 m before striking the glass plate and breaking it. The total time taken from release to hitting the ground is recorded as 0,80 s. Assume that the time taken to break the plate is negligible.

\begin{figure}[H]
\begin{center}
\scalebox{1} % Change this value to rescale the drawing.
{
\begin{pspicture}(0,-2.19)(4.72,2.17)
\psdiamond[linewidth=0.08,dimen=outer](2.35,-0.18)(2.35,0.39)
\pscircle[linewidth=0.04,dimen=outer](2.34,1.97){0.2}
\psline[linewidth=0.04cm,arrowsize=0.05291667cm 2.0,arrowlength=1.4,arrowinset=0.4]{<->}(2.32,1.77)(2.32,-0.15)
\psline[linewidth=0.04cm,arrowsize=0.05291667cm 2.0,arrowlength=1.4,arrowinset=0.4]{->}(2.28,-0.51)(2.28,-2.17)
\psline[linewidth=0.04cm](0.1,-2.13)(4.7,-2.13)
\psline[linewidth=0.04cm,linestyle=dashed,dash=0.16cm 0.16cm](1.9,-0.13)(2.7,-0.13)
\usefont{T1}{ptm}{m}{n}
\rput(2.988125,1.18){1,25 m}
\usefont{T1}{ptm}{m}{n}
\rput(2.988125,-1.12){1,05 m}
\psline[linewidth=0.04cm,linestyle=dashed,dash=0.16cm 0.16cm](2.0,1.77)(2.8,1.77)
\end{pspicture}
}
\end{center}
\end{figure}

\begin{enumerate}
\item Calculate the speed at which the ball strikes the glass plate.
\item Show that the speed of the ball immediately after breaking the plate is 2,0 \ms.
\item Calculate the magnitude and give the direction of the change of momentum which the ball experiences during its contact with the glass plate.
\item Give the magnitude and direction of the impulse which the glass plate experiences when the ball hits it.
\end{enumerate}}

\item{[SC 2004/11 HG1]A cricket ball, mass 175 g is thrown directly towards a player at a velocity of 12 \ms. It is hit back in the opposite direction with a velocity of 30 \ms. The ball is in contact with the bat for a period of 0,05 s.
\begin{enumerate}
\item{Calculate the impulse of the ball.}
\item{Calculate the magnitude of the force exerted by the bat on the ball.}
\end{enumerate}}

\item{[IEB 2005/11 HG1] A ball bounces to a vertical height of 0,9 m when it is dropped from a height of 1,8 m. It rebounds immediately after it strikes the ground, and the effects of air resistance are negligible.

\begin{center}
\begin{pspicture}(0,0)(4,4)
\def\tennisball{\pscircle(0,0){.25}\psarc(-0.2,0){0.15}{-87}{87}\psarc(0.2,0){0.15}{87}{-87}}
%		\psgrid[gridcolor=lightgray]
\psline(0,0)(4,0)
\rput(2,3.6){\tennisball}
\rput(2,1.8){\psset{dash= 2pt 2pt,linestyle=dashed}\tennisball}
\psline[linestyle=dashed]{<->}(1,0)(1,3.6)
\psline[linestyle=dashed]{<->}(3,0)(3,1.8)
\uput[l](1,1.8){1,8~m}
\uput[r](3,0.9){0,9~m}
\end{pspicture}
\end{center}

\begin{enumerate}
\item{How long (in s) does it take for the ball to hit the ground after it has been dropped?}
\item{At what speed does the ball strike the ground?}
\item{At what speed does the ball rebound from the ground?}
\item{How long (in s) does the ball take to reach its maximum height after the bounce?}
\item{Draw a velocity-time graph for the motion of the ball from the time it is dropped to the time when it rebounds to 0,9 m. Clearly, show the following on the graph:
\begin{enumerate}
\item{the time when the ball hits the ground}
\item{the time when it reaches 0,9 m}
\item{the velocity of the ball when it hits the ground, and}
\item{the velocity of the ball when it rebounds from the ground.}
\end{enumerate}}
\end{enumerate}}


\item{[SC 2002/11 HG1] In a railway shunting yard, a locomotive of mass 4~000~kg, travelling due east at a velocity of 1,5 \ms, collides with a stationary goods wagon of mass 3~000~kg in an attempt to couple with it. The coupling fails and instead the goods wagon moves due east with a velocity of 2,8 \ms.

\begin{enumerate}
\item \label{q.7.1} Calculate the magnitude and direction of the velocity of the locomotive immediately after collision.
\item Name and state in words the law you used to answer question (\ref{q.7.1})
\end{enumerate}}

\item{[SC 2005/11 SG1] A combination of trolley A (fitted with a spring) of mass 1~kg, and trolley B of mass 2~kg, moves to the right at 3 \ms\ along a frictionless, horizontal surface. The spring is kept compressed between the two trolleys.

\begin{center}
\begin{pspicture}(0,-0.6)(8.2,2.6)

%\psgrid
\SpecialCoor
\psline[linewidth=2pt](0,0)(8.2,0) %ground
\psframe(2,0.25)(4,1.25)
\uput[dr](2,1.25){A}
\rput(3,0.75){1~kg}
\pscircle[fillcolor=white,fillstyle=solid](2.5,0.25){0.25}
\pscircle[fillcolor=white,fillstyle=solid](3.5,0.25){0.25}
\rput(2.2,0){\psframe(2,0.25)(4,1.75)
\uput[dr](2,1.75){B}
\rput(3,1){2~kg}
\pscircle[fillcolor=white,fillstyle=solid](2.5,0.25){0.25}
\pscircle[fillcolor=white,fillstyle=solid](3.5,0.25){0.25}}
\pscoil[coilarm=0.01cm,coilwidth=0.2cm,coilheight=0.5](4,1.15)(4.2,1.15)
\psline{->}(3.6,2)(4.6,2)
\uput[u](4.1,2){3 \ms}
\uput[d](4,0){Before}
\end{pspicture}
\end{center}

While the combination of the two trolleys is moving at 3 \ms\ , the spring is released and when it has expanded completely, the 2~kg trolley is then moving to the right at 4,7 \ms\ as shown below.

\begin{center}
\begin{pspicture}(0,-0.6)(8.2,2.6)
%\psgrid
\SpecialCoor
\psline[linewidth=2pt](0,0)(8.2,0) %ground

\rput(-1,0){\psframe(2,0.25)(4,1.25)
\uput[dr](2,1.25){A}
\rput(3,0.75){1~kg}
\pscircle[fillcolor=white,fillstyle=solid](2.5,0.25){0.25}
\pscircle[fillcolor=white,fillstyle=solid](3.5,0.25){0.25}}

\rput(3.2,0){\psframe(2,0.25)(4,1.75)
\uput[dr](2,1.75){B}
\rput(3,1){2~kg}
\pscircle[fillcolor=white,fillstyle=solid](2.5,0.25){0.25}
\pscircle[fillcolor=white,fillstyle=solid](3.5,0.25){0.25}}
\pscoil[coilarm=0.01cm,coilwidth=0.2cm,coilheight=1](3,1.15)(3.5,1.15)
\psline{->}(5.7,2)(6.7,2)
\uput[u](6.2,2){4,7 \ms}
\uput[d](4,0){After}
\end{pspicture}
\end{center}

\begin{enumerate}
\item State, in words, the principle of \emph{conservation of linear momentum}.
\item Calculate the magnitude and direction of the velocity of the 1~kg trolley immediately after the spring has expanded completely.
\end{enumerate}}

\item{[IEB 2002/11 HG1] A ball bounces back from the ground. Which of the following statements is true of this event?
\begin{enumerate}
\item{The magnitude of the change in momentum of the ball is equal to the magnitude of the change in momentum of the Earth.}
\item{The magnitude of the impulse experienced by the ball is greater than the magnitude of the impulse experienced by the Earth.}
\item{The speed of the ball before the collision will always be equal to the speed of the ball after the collision.}
\item{Only the ball experiences a change in momentum during this event.}
\end{enumerate}}

\item{[SC 2002/11 SG] A boy is standing in a small stationary boat. He throws his schoolbag, mass 2~kg, horizontally towards the jetty with a velocity of 5 \ms. The \emph{combined mass} of the boy and the boat is 50~kg.
\begin{enumerate}
\item Calculate the magnitude of the horizontal momentum of the bag immediately after the boy has thrown it.
\item Calculate the velocity (magnitude and direction) of the \emph{boat-and-boy} immediately after the bag is thrown.
\end{enumerate}}
\end{enumerate}


\subsubsection{Torque and levers}
\begin{enumerate}

\item {State whether each of the following statements are true or false. If the statement is false, rewrite the statement correcting it.
\begin{enumerate}
\item The torque tells us what the turning effect of a force is.
\item To increase the mechanical advantage of a spanner you need to move the effort closer to the load.
\item A class 2 lever has the effort between the fulcrum and the load.
\item An object will be in equilibrium if the clockwise moment and the anticlockwise moments are equal.
\item Mechanical advantage is a measure of the difference between the load and the effort.
\item The force times the perpendicular distance is called the mechanical advantage.
\end{enumerate}}

\item {Study the diagram below and determine whether the seesaw is balanced. Show all your calculations.
\begin{figure}[H]
\begin{center}
\scalebox{1} % Change this value to rescale the drawing.
{
\begin{pspicture}(0,-1.48)(7.94,1.48)
\psline[linewidth=0.08cm](0.0,-0.04)(7.9,-0.04)
\pstriangle[linewidth=0.08,dimen=outer](3.93,-1.06)(1.1,1.06)
\psline[linewidth=0.04cm,arrowsize=0.05291667cm 2.0,arrowlength=1.4,arrowinset=0.4]{->}(0.9,-0.04)(0.88,-1.46)
\psline[linewidth=0.04cm,arrowsize=0.05291667cm 2.0,arrowlength=1.4,arrowinset=0.4]{->}(6.9,-0.06)(6.92,-1.46)
\psline[linewidth=0.04cm,arrowsize=0.05291667cm 2.0,arrowlength=1.4,arrowinset=0.4]{->}(1.42,0.7)(3.92,0.7)
\psline[linewidth=0.04cm,arrowsize=0.05291667cm 2.0,arrowlength=1.4,arrowinset=0.4]{->}(6.4,0.7)(3.92,0.7)
\psline[linewidth=0.04cm](3.9,1.46)(3.9,-0.04)
\usefont{T1}{ptm}{m}{n}
\rput(2.498125,1.07){1,2 m}
\usefont{T1}{ptm}{m}{n}
\rput(5.2753124,1.07){2 m}
\usefont{T1}{ptm}{m}{n}
\rput(0.608125,0.67){5 kg}
\usefont{T1}{ptm}{m}{n}
\rput(7.009375,0.57){3 kg}
\psframe[linewidth=0.04,dimen=outer](1.4,1.26)(0.0,-0.04)
\psframe[linewidth=0.04,dimen=outer](7.8,1.26)(6.4,-0.04)
\end{pspicture}
}
\end{center}
\end{figure}}
\item {Two children are playing on a seesaw.  Tumi has a weight of 200 N and Thandi weighs 240 N. Tumi is sitting at a distance of 1,2 m from the pivot.
\begin{enumerate}
\item What type of lever is a seesaw?
\item Calculate the moment of the force that Tumi exerts on the seesaw.
\item At what distance from the pivot should Thandi sit to balance the seesaw?
\end{enumerate}}
\item {An applied force of 25 N is needed to open the cap of a glass bottle using a bottle opener. The distance between the applied force and the fulcrum is 10 cm and the distance between the load and the fulcrum is 1 cm.
\begin{enumerate}
\item What type of lever is a bottle opener? Give a reason for your answer.
\item Calculate the mechanical advantage of the bottle opener.
\item Calculate the force that the bottle cap is exerting.
\end{enumerate}}

\item {Determine the force needed to lift the 20 kg load in the wheelbarrow in the diagram below.
\begin{figure}[h]
\begin{center}
\scalebox{1} % Change this value to rescale the drawing.
{
\begin{pspicture}(0,-1.415)(4.103445,1.435)
\pspolygon[linewidth=0.04](0.14,0.585)(1.54,-0.515)(2.34,-0.515)(2.64,0.585)
\rput{22.28843}(0.25648692,-0.83200556){\psframe[linewidth=0.04,dimen=outer](4.24,0.285)(0.24,0.185)}
\psline[linewidth=0.08cm](0.44,0.485)(0.44,-0.515)
\psline[linewidth=0.08cm](2.6,0.365)(3.14,-0.875)
\pscircle[linewidth=0.08,dimen=outer](0.45,-0.505){0.45}
\pscustom[linewidth=0.04]
{
\newpath
\moveto(0.82,0.625)
\lineto(1.07,0.775)
\curveto(1.195,0.85)(1.385,0.925)(1.45,0.925)
\curveto(1.515,0.925)(1.675,0.92)(1.77,0.915)
\curveto(1.865,0.91)(2.01,0.855)(2.06,0.805)
\curveto(2.11,0.755)(2.165,0.675)(2.17,0.645)
\curveto(2.175,0.615)(2.18,0.59)(2.18,0.595)
\curveto(2.18,0.6)(2.17,0.61)(2.14,0.625)
}
\usefont{T1}{ptm}{m}{n}
\rput(1.7628125,1.255){20 kg}
\psline[linewidth=0.04cm,linestyle=dashed,dash=0.16cm 0.16cm](1.8,0.085)(1.82,-1.395)
\psline[linewidth=0.04cm,linestyle=dashed,dash=0.16cm 0.16cm](0.44,-0.515)(0.44,-1.355)
\psline[linewidth=0.04cm,linestyle=dashed,dash=0.16cm 0.16cm](4.06,0.965)(4.06,-1.355)
\usefont{T1}{ptm}{m}{n}
\rput(1.140625,-1.165){50 cm}
\usefont{T1}{ptm}{m}{n}
\rput(2.9425,-1.165){75 cm}
\psline[linewidth=0.02cm,arrowsize=0.05291667cm 2.0,arrowlength=1.4,arrowinset=0.4]{<->}(0.44,-1.355)(1.82,-1.355)
\psline[linewidth=0.02cm,arrowsize=0.05291667cm 2.0,arrowlength=1.4,arrowinset=0.4]{<->}(1.9,-1.355)(3.98,-1.355)
\end{pspicture}
}
\end{center}
\end{figure}
}
\item {A body builder picks up a weight of 50 N using his right hand. The distance between the body builder's hand and his elbow is 45 cm. The distance between his elbow and where his muscles are attached to his forearm is 5 cm.
\begin{enumerate}
\item What type of lever is the human arm? Explain your answer using a diagram.
\item Determine the force his muscles need to apply to hold the weight steady.
\end{enumerate}}

\end{enumerate}
\practiceinfo

\textbf{Force and Newton's laws}\\
\begin{tabular}[h]{cccccc}
(1.) 00p4 & (2.) 00p5 & (3.) 00p6 & (4.) 00p7 & (5.) 00p8 & (6.) 00p9 & (7.) 00pa & (8.) 00pb & (9.) 00pc & (10.) 00pd & (11.) 00pe & (12.) 00pf & (13.) 00pg & (14.) 00ph & (15.) 00pi & (16.) 00pj & (17.) 00pk & (18.) 00pm & (19.) 00pn & (20.) 00pp & (21.) 00pq & (22.) 00pr & (23.) 00ps & (24.) 00pt & (25.) 00pu & (26.) 00pv & (27.) 00pw & (28.) 00px & (29.) 00py & (30.) 00pz & (31.) 00q0 & (32.) 00q1 & (33.) 00q2 & (34.) 00q3 & (35.) 00q4 & (36.) 00q5 & (37.) 00q6 & 
 \end{tabular}\\


\textbf{Gravitation}\\
\begin{tabular}[h]{cccccc}
(1.) 00q7 & (2.) 00q8 & (3.) 00q9 & (4.) 00qa & (5.) 00qb & (6.) 00qc & (7.) 00qd & (8.) 00qe & (9.) 00qf & (10.) 00qg & (11.) 00qh & (12.) 00qi & (13.) 00qj & (14.) 00qk & (15.) 00qm & (16.) 00qn & (17.) 00qp &
\end{tabular}\\

\textbf{Gravitation}\\
\begin{tabular}[h]{cccccc}
 (1.) 00qq & (2.) 00qr & (3.) 00qs & (4.) 00qt & (5.) 00qu & (6.) 00qv & (7.) 00qw & (8.) 00qx & (9.) 00qy & (10.) 00qz & (11.) 00r0 & (12.) 00r1 & (13.) 00r2 & (14.) 00r3 & (15.) 00r4 & (16.) 00r5 & (17.) 00r6 & (18.) 00r7 & (19.) 00r8 & (20.) 00r9 & (21.) 00ra & 
\end{tabular}\\

\textbf{Torque and levers}\\
\begin{tabular}[h]{cccccc}
(1.) 00rb & (2.) 00rc & (3.) 00rd & (4.) 00re & (5.) 00rf & (6.) 00rg &
\end{tabular}
\end{eocexercises}