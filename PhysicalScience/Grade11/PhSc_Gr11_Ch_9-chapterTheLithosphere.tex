\chapter{The Lithosphere}
\label{chap:lith}

%\nts{Need map showing major mining areas in SA}\\

% CHILD SECTION START

\section{Introduction}
\label{sec:lith:intro}

If we were to cut the Earth in half we would see that our planet is made up of a number of layers, namely the \textbf{core} at the centre (seperated into the inner and outer core), the \textbf{mantle}, the \textbf{upper mantle}, the outer \textbf{crust} and the \textbf{atmosphere} (figure \ref{fig:earth section}). The core is made up mostly of iron. The mantle, which lies between the core and the crust, consists of molten rock, called \textbf{magma} which moves continuously because of convection currents. The crust is the thin, hard outer layer that 'floats' on the magma of the mantle. It is the upper part of the mantle and the crust that make up the \textbf{lithosphere} ('lith' means 'types of stone' and 'sphere' refers to the round shape of the earth). Together, the lithosphere, hydrosphere and atmosphere make up the world as we know it. \\

\begin{figure}[h]
\begin{center}
% \usepackage{pst-plot} % For axes
\scalebox{0.6} % Change this value to rescale the drawing.
{
\begin{pspicture}(0,-4.08)(23.94,9.181)
\pscircle[linewidth=0.04,dimen=outer](4.63,1.41){0.71}
\pscircle[linewidth=0.04,dimen=outer](4.63,1.39){1.59}
\pscircle[linewidth=0.04,dimen=outer](4.63,1.43){2.65}
\pscircle[linewidth=0.04,dimen=outer](4.63,1.43){3.51}
\pscircle[linewidth=0.04,dimen=outer](4.63,1.43){3.79}
\pscircle[linewidth=0.04,linestyle=dashed,dash=0.16cm 0.16cm,dimen=outer](4.65,1.39){4.65}
\psline[linewidth=0.04cm](10.44,4.18)(13.48,-2.46)
\psline[linewidth=0.04cm](13.44,-2.46)(16.44,4.34)
\psline[linewidth=0.04cm](13.52,-2.46)(16.32,-1.54)
\psline[linewidth=0.04cm](16.36,-1.54)(19.36,4.78)
\psline[linewidth=0.04cm](14.04,-1.22)(16.92,-0.42)
\psline[linewidth=0.04cm](14.4,-0.42)(17.24,0.34)
\psline[linewidth=0.04cm](15.28,1.62)(18.24,2.3)
\psline[linewidth=0.04cm](16.0,3.18)(18.92,3.7)
\psline[linewidth=0.04cm](16.12,3.46)(19.0,3.98)
\psline[linewidth=0.04cm](16.12,3.42)(16.04,3.42)
\psline[linewidth=0.04cm](15.48,-1.42)(19.64,-1.3)
\psline[linewidth=0.04cm](16.72,-0.18)(19.48,-0.1)
\psline[linewidth=0.04cm](17.08,1.5)(19.52,1.62)
\psline[linewidth=0.04cm](17.72,3.06)(19.56,3.1)
\psline[linewidth=0.04cm](18.48,3.74)(19.56,3.78)
\psline[linewidth=0.04cm](18.52,4.26)(19.56,4.3)
\usefont{T1}{ptm}{m}{n}
\rput(20.710938,-1.27){Inner Core}
\usefont{T1}{ptm}{m}{n}
\rput(20.808437,-0.03){Outer Core}
\usefont{T1}{ptm}{m}{n}
\rput(20.520937,1.69){Mantle}
\usefont{T1}{ptm}{m}{n}
\rput(20.931875,3.17){Upper Mantle}
\usefont{T1}{ptm}{m}{n}
\rput(20.321095,3.81){Crust}
\usefont{T1}{ptm}{m}{n}
\rput(20.836094,4.33){Atmosphere}
\psarc[linewidth=0.04](13.48,-1.7){0.68}{45.0}{135.0}
\psarc[linewidth=0.04](13.48,-1.58){1.52}{54.462322}{125.53768}
\psarc[linewidth=0.04](13.46,0.0){2.42}{42.455196}{139.28915}
\psarc[linewidth=0.04](13.54,0.32){3.78}{49.66686}{132.47388}
\psarc[linewidth=0.04](13.5,0.32){4.02}{51.241913}{131.53177}
\psarc[linewidth=0.04,linestyle=dashed,dash=0.16cm 0.16cm](13.56,0.78){4.6}{50.355824}{132.13759}
\psline[linewidth=0.04cm,linestyle=dashed,dash=0.16cm 0.16cm](16.48,4.3)(19.32,4.78)
\psarc[linewidth=0.04,linestyle=dashed,dash=0.16cm 0.16cm](15.14,0.76){5.78}{42.81836}{143.88066}
\psline[linewidth=0.04cm](4.6,1.34)(0.96,5.74)
\psline[linewidth=0.04cm](4.6,1.3)(4.96,6.9)
\psarc[linewidth=0.24200001,arrowsize=0.05291667cm 2.0,arrowlength=1.4,arrowinset=0.4]{<-<}(7.1,4.24){4.82}{24.968987}{155.90797}
\end{pspicture}
}
\caption{A cross-section through the earth to show its different layers}
\label{fig:earth section}
\end{center}
\end{figure}


\Definition{Lithosphere}{The lithosphere is the solid outermost shell of our planet. The lithosphere includes the crust and the upper part of the mantle, and is made up of material from both the continents and the oceans on the Earth's surface.}


In grade 10 have focused on the hydrosphere and the atmosphere. The lithosphere is also very important, not only because it is the surface on which we live, but also because humans gain many valuable resources from this part of the planet.


% CHILD SECTION END



% CHILD SECTION START

\section{The chemistry of the earth's crust}

The crust is made up of about 80 elements, which occur in over 2000 different compounds and minerals. However, most of the mass of the material in the crust is made up of only 8 of these elements. These are oxygen (O), silica (Si), aluminium (Al), iron (Fe), calcium (Ca), sodium (Na), potassium (K) and magnesium (Mg). These metal elements are seldom found in their pure form, but are usually part of other more complex \textbf{minerals}. A mineral is a compound that is formed through geological processes, which give it a particular structure. A mineral could be a pure element, but more often minerals are made up of many different elements combined. \textit{Quartz} is just one example. It is a mineral that is made up of silicon and oxygen. Some more examples are shown in table \ref{tab:minerals}.

\Definition{Mineral}{Minerals are natural compounds formed through geological processes. The term 'mineral' includes both the material's chemical composition and its structure. Minerals range in composition from pure elements to complex compounds.}

\begin{table}[h]
\begin{center}
\caption{Table showing examples of minerals and their chemistry}
\label{tab:minerals}
\begin{tabular}{|l|p{4cm}|p{6cm}|}\hline
\textbf{Mineral} & \textbf{Chemistry} & \textbf{Comments}\\\hline
Quartz & SiO$_{2}$ (silicon dioxide) & Quartz is used for glass, in electrical components, optical lenses and in building stone \\\hline
Gold & Au (pure element) or AuTe$_{2}$ (Calaverite, a gold mineral) & Gold is often found in a group of minerals called the \textit{tellurides}. Calaverite is a mineral that belongs to this group, and is the most common gold-bearing mineral. Gold has an affinity for tellurium (Te). \\\hline
Hematite & Fe$_{2}$O$_{3}$ (iron oxide) & Iron usually occurs in iron oxide minerals or as an alloy of iron and nickel. \\\hline
Orthoclase & KAlSi$_{3}$O$_{8}$ (potassium aluminium silicate) & Orthoclase belongs to the \textit{feldspar} group of minerals. \\\hline
Copper & Cu (pure element) or Cu$_{2}$(CO$_{3}$)(OH)$_{2}$ (malachite or copper carbonate hydroxide) & copper can be mined as a pure element or as a mineral such as malachite. \\\hline
\end{tabular}
\end{center}
\end{table}

A \textbf{rock} is a combination of one or more minerals. \textit{Granite} for example, is a rock that is made up of minerals such as SiO$_{2}$, Al$_{2}$O$_{3}$, CaO, K$_{2}$O, Na$_{2}$O and others. There are three different types of rocks, \textbf{igneous}, \textbf{sedimentary} and \textbf{metamorphic}. Igneous rocks (e.g. granite, basalt) are formed when magma is brought to the earth's surface as lava, and then solidifies. Sedimentary rocks (e.g. sandstone, limestone) form when rock fragments, organic matter or other sediment particles are deposited and then compacted over time until they solidify. Metamorphic rock is formed when any other rock types are subjected to intense heat and pressure over a period of time. Examples include slate and marble.\\


Many of the elements that are of interest to us (e.g. gold, iron, copper), are unevenly distributed in the lithosphere. In places where these elements are abundant, it is profitable to extract them (e.g. through mining) for economic purposes. If their concentration is very low, then the cost of extraction becomes more than the money that would be made if they were sold. Rocks that contain valuable minerals are called \textbf{ores}. As humans, we are particularly interested in the ores that contain metal elements, and also in those minerals that can be used to produce energy.

\Definition{Ore}{An ore is a volume of rock that contains minerals which make it valuable for mining.}

\begin{IFact}{
A \textbf{gemstone} (also sometimes called a \textbf{gem} or \textbf{semi-precious stone}), is a highly attractive and valuable piece of mineral which, when cut and polished, is used in jewelry and other adornments. Examples of gemstones are amethyst, diamond, cat's eye and sapphire.
}
\end{IFact}

\Exercise{Rocks and minerals\\}{
\begin{enumerate}
\item{Where are most of the earth's minerals concentrated?}
\item{Explain the difference between a \textit{mineral}, a \textit{rock} and an \textit{ore}.}
\item{Carry out your own research to find out which elements are found in the following minerals:
\begin{enumerate}
\item{gypsum}
\item{calcite}
\item{topaz}
\end{enumerate}
}
\item{Which minerals are found in the following rocks?}
\begin{enumerate}
\item{basalt}
\item{sandstone}
\item{marble}
\end{enumerate}
\end{enumerate}
\practiceinfo

\begin{tabular}[h]{cccccc}
(1.) aaa & (2.) aaa & (3.) aaa & (4.) aaa & 
 \end{tabular}
}



% CHILD SECTION END



% CHILD SECTION START

\section{A brief history of mineral use}

Many of the minerals that are important to humans are \textbf{metals} such as gold, aluminium, copper and iron. Throughout history, metals have played a very important role in making jewelery, tools, weapons and later machinery and other forms of technology. We have become so used to having metals around us that we seldom stop to think what life might have been like before metals were discovered. During the \textbf{Stone Age} for example, \textbf{stones} were used to make tools. Slivers of stone were cut from a rock and then sharpened. In Africa, some of the stone tools that have been found date back to 2.5 million years ago! \\

It was the discovery of \textit{metals} that led to some huge advances in agriculture, warfare, transport and even cookery. One of the first metals to be discovered was \textbf{gold}, probably because of its beautiful shiny appearance. Gold was used mostly to make jewelery because it was too soft to make harder tools. Later, \textbf{copper} became an important metal because it could be hammered into shape, and it also lasted a lot longer than the stone that had previously been used in knives, cooking utensils and weapons. Copper can also be melted and then put into a mould to re-shape it. This is known as \textbf{casting}. \\

At about the time that copper was in widespread use, it was discovered that if certain kinds of stones are heated to high enough temperatures, liquid metals flow from them. These rocks are \textbf{ores}, and contain the metal minerals inside them. The process of heating mineral ores in this way is called \textbf{smelting}. It was also found that ores do not only occur at the earth's surface, but also deep \textit{below} it. This discovery led to the beginning of \textbf{mining}.\\

But humans' explorations into the world of metals did not end here! In some areas, the ores of \textit{iron} and \textit{tin} were found close together. The cast alloy of these two metals is \textbf{bronze}. Bronze is a very useful metal because it produces a sharper edge than copper. Another important discovery was that of \textbf{iron}. Iron is the most abundant metal at the earth's surface but it is more difficult to work with than copper or tin. It is very difficult to extract iron from its ore because it has an extremely high melting point, and only specially designed furnaces are able to produce the temperatures that are needed. An important discovery was that if iron is heated in a furnace with \textit{charcoal}, some of the carbon in the charcoal is transferred to the iron, making the metal even harder. If this hot metal has its temperature reduced very suddenly, it becomes even harder and produces \textbf{steel}. Today, steel is a very important part of industry and construction.\\


\begin{IFact}{
Originally it was believed that much of Africa's knowledge of metals and their uses was from the Middle East. But this may not be the case. More recent studies have shown that iron was used far earlier than it would have been if knowledge of this metal had started in the Middle East. Many metal technologies may in fact have developed independently in Africa and in many African countries, metals have an extremely important place in society. In Nigeria's Yoruba country for example, iron has divine status because it is used to make instruments for survival. 'Ogun', the God of Iron, is seen as the protector of the kingdom.
}
\end{IFact}




% CHILD SECTION END



% CHILD SECTION START

\section{Energy resources and their uses}
\label{sec:mining:energy}

Apart from minerals and ores, the products of the lithosphere are also important in meeting our energy needs.\\

\textbf{Coal} is one of the most important fuels that is used in the production of electricity. Coal is formed from organic material when plants and animals decompose, leaving behind organic remains that accumulate and become compacted over millions of years under sedimentary rock. The layers of compact organic material that can be found between sedimentary layers, are coal. When coal is burned, a large amount of heat energy is released, which is used to produce electricity. South Africa is the world's sixth largest coal producer, with Mpumalanga contributing about 83\% of our total production. Other areas in which coal is produced, include the Free State, Limpopo and KwaZulu-Natal. One of the problems with coal however, is that it is a non-renewable resource, meaning that once all resources have been used up, it cannot simply be produced again. Burning coal also produces large quantities of greenhouse gases, which may play a role in global warming. At present, ESKOM, the South African government's electric power producer, is the coal industry's main customer.\\

Another element that is found in the crust, and which helps to meet our energy needs, is \textbf{uranium}. Uranium produces energy through the process of \textit{nuclear fission}. Neutrons are aimed at the nucleii of the uranium atoms in order to split them. When the nucleus of a uranium atom is split, a large amount of energy is released as heat. This heat is used to produce steam, which turns turbines to generate electricity. Uranium is produced as a by-product of gold in some mines in the Witwatersrand, and as a by-product in some copper mines, for example in Phalaborwa. This type of nuclear power is relatively environmentally friendly since it produces low gas emissions. However, the process does produce small amounts of radioactive wastes , which must be carefully disposed of in order to prevent contamination.\\

\textbf{Oil} is another product of the lithosphere which is critical in meeting our fuel needs. While most of South Africa's oil is imported and then processed at a refinery in either Durban, Cape Town or Sasolburg, some is extracted from coal. The technology behind this type of extraction has largely been developed by SASOL (South African Coal, Oil and Gas Corporation). Oil, like coal, is organic in origin and normally forms from organic deposits on the ocean floor. Oil requires unique geological and geochemical conditions in order to be produced. Part of this process involves the burial of organic-rich sediments under extremely high temperatures and pressures. The oil that is produced is then pushed out into nearby sedimentary layers. Oil will then move upwards until it is trapped by an impermeable rock layer. It accumulates here, and can then be accessed by oil rigs and other advanced equipment.



% CHILD SECTION END

\Activity{Research}{Mining Areas}{Using any reference resources you have available, try to find a map of the mining regions of South Africa.}

% CHILD SECTION START

\section{Mining and Mineral Processing: Gold}

\subsection{Introduction}

Gold was discovered in South Africa in the late 1800's and since then has played a very important role in South Africa's \emph{history} and \emph{economy}. Its discovery brought many foreigners into South Africa, who were lured by the promises of wealth. They set up small mining villages, which later grew into larger settlements, towns and cities. One of the first of these settlements was the beginning of present-day Johannesburg, also known as 'Egoli' or 'Place of Gold'.\\

Most of South Africa's gold is concentrated in the 'Golden Arc', which stretches from Johannesburg to Welkom. Geologists believe that, millions of years ago, this area was a massive inland lake. Rivers feeding into this lake brought sand, silt, pebbles and fine particles of gold and deposited them over a long period of time. Eventually these deposits accumulated and became compacted to form gold-bearing sedimentary rock or \textbf{gold reefs}. It is because of this complex, but unique, set of circumstances that South Africa's gold deposits are so concentrated in that area. In other countries like Zimbabwe, gold occurs in smaller 'pockets', which are scattered over a much greater area.

\subsection{Mining the Gold}

A number of different techniques can be used to mine gold. The three most common methods in South Africa are \textbf{panning}, \textbf{open cast} and \textbf{shaft} mining.

\begin{enumerate}

\item{\textbf{Panning}

Panning for gold is a manual technique that is used to sort gold from other sediments. Wide, shallow pans are filled with sand and gravel (often from river beds) that may contain gold. Water is added and the pans are shaken so that the gold is sorted from the rock and other materials. Because gold is much more dense, it settles to the bottom of the pan. \textbf{Pilgrim's Rest} in Mpumalanga, was the first site for gold panning in South Africa.
}

\item{\textbf{Open cast mining}

This is a form of surface mining. Surface layers of rock and sediments are removed so that the deeper gold rich layers can be reached. This type of mining is not suitable if the gold is buried very deep below the surface.
}

\item{\textbf{Shaft mining}

South Africa's thin but extensive gold reefs slope at an angle underneath the ground, and this means that some deposits are very deep and often difficult to reach.  Shaft mining is needed to reach the gold ore. After the initial drilling, blasting and equipping of a mine shaft, tunnels are built leading outwards from the main shaft so that the gold reef can be reached. Shaft mining is a dangerous operation, and roof supports are needed so that the rock does not collapse. There are also problems of the intense heat and high pressure below the surface which make shaft mining very complex, dangerous and expensive. A diagram illustrating open cast and shaft mining is shown in figure \ref{fig:gold mining}.
}

\begin{figure}[h]
\begin{center}
% \usepackage{pst-plot} % For axes
\scalebox{1} % Change this value to rescale the drawing.
{
\begin{pspicture}(0,-5.183)(13.84,5.197)
\psline[linewidth=0.04cm](10.84,-2.263)(2.12,-2.243)
\pscustom[linewidth=0.04]
{
\newpath
\moveto(0.54,0.737)
\lineto(0.63,0.737)
\curveto(0.675,0.737)(0.745,0.737)(0.77,0.737)
\curveto(0.795,0.737)(0.855,0.737)(0.89,0.737)
\curveto(0.925,0.737)(1.085,0.717)(1.21,0.697)
\curveto(1.335,0.677)(1.505,0.652)(1.55,0.647)
\curveto(1.595,0.642)(1.675,0.632)(1.71,0.627)
\curveto(1.745,0.622)(1.805,0.617)(1.83,0.617)
\curveto(1.855,0.617)(1.905,0.607)(1.93,0.597)
\curveto(1.955,0.587)(2.015,0.572)(2.05,0.567)
\curveto(2.085,0.562)(2.16,0.552)(2.2,0.547)
\curveto(2.24,0.542)(2.315,0.532)(2.35,0.527)
\curveto(2.385,0.522)(2.455,0.502)(2.49,0.487)
\curveto(2.525,0.472)(2.605,0.447)(2.65,0.437)
\curveto(2.695,0.427)(2.79,0.407)(2.84,0.397)
\curveto(2.89,0.387)(2.975,0.372)(3.01,0.367)
\curveto(3.045,0.362)(3.105,0.352)(3.13,0.347)
\curveto(3.155,0.342)(3.21,0.322)(3.24,0.307)
\curveto(3.27,0.292)(3.335,0.262)(3.37,0.247)
\curveto(3.405,0.232)(3.48,0.197)(3.52,0.177)
\curveto(3.56,0.157)(3.625,0.122)(3.65,0.107)
\curveto(3.675,0.092)(3.725,0.057)(3.75,0.037)
\curveto(3.775,0.017)(3.84,-0.038)(3.88,-0.073)
\curveto(3.92,-0.108)(3.985,-0.153)(4.01,-0.163)
\curveto(4.035,-0.173)(4.085,-0.198)(4.11,-0.213)
\curveto(4.135,-0.228)(4.19,-0.253)(4.22,-0.263)
\curveto(4.25,-0.273)(4.295,-0.298)(4.31,-0.313)
\curveto(4.325,-0.328)(4.36,-0.368)(4.38,-0.393)
\curveto(4.4,-0.418)(4.435,-0.468)(4.45,-0.493)
\curveto(4.465,-0.518)(4.5,-0.568)(4.52,-0.593)
\curveto(4.54,-0.618)(4.58,-0.663)(4.6,-0.683)
\curveto(4.62,-0.703)(4.66,-0.738)(4.68,-0.753)
\curveto(4.7,-0.768)(4.745,-0.793)(4.77,-0.803)
\curveto(4.795,-0.813)(4.85,-0.833)(4.88,-0.843)
\curveto(4.91,-0.853)(4.96,-0.868)(4.98,-0.873)
\curveto(5.0,-0.878)(5.045,-0.883)(5.07,-0.883)
\curveto(5.095,-0.883)(5.15,-0.883)(5.18,-0.883)
\curveto(5.21,-0.883)(5.28,-0.883)(5.32,-0.883)
\curveto(5.36,-0.883)(5.435,-0.873)(5.47,-0.863)
\curveto(5.505,-0.853)(5.575,-0.838)(5.61,-0.833)
\curveto(5.645,-0.828)(5.705,-0.813)(5.73,-0.803)
\curveto(5.755,-0.793)(5.795,-0.773)(5.81,-0.763)
\curveto(5.825,-0.753)(5.865,-0.728)(5.89,-0.713)
\curveto(5.915,-0.698)(5.965,-0.658)(5.99,-0.633)
\curveto(6.015,-0.608)(6.055,-0.553)(6.07,-0.523)
\curveto(6.085,-0.493)(6.11,-0.443)(6.12,-0.423)
\curveto(6.13,-0.403)(6.15,-0.348)(6.16,-0.313)
\curveto(6.17,-0.278)(6.195,-0.213)(6.21,-0.183)
\curveto(6.225,-0.153)(6.255,-0.103)(6.27,-0.083)
\curveto(6.285,-0.063)(6.31,-0.023)(6.32,-0.0029999996)
\curveto(6.33,0.017)(6.36,0.052)(6.38,0.067)
\curveto(6.4,0.082)(6.44,0.107)(6.46,0.117)
\curveto(6.48,0.127)(6.525,0.147)(6.55,0.157)
\curveto(6.575,0.167)(6.61,0.192)(6.62,0.207)
\curveto(6.63,0.222)(6.66,0.252)(6.68,0.267)
\curveto(6.7,0.282)(6.745,0.307)(6.77,0.317)
\curveto(6.795,0.327)(6.845,0.337)(6.87,0.337)
\curveto(6.895,0.337)(6.95,0.342)(6.98,0.347)
\curveto(7.01,0.352)(7.065,0.362)(7.09,0.367)
\curveto(7.115,0.372)(7.16,0.392)(7.18,0.407)
\curveto(7.2,0.422)(7.25,0.457)(7.28,0.477)
\curveto(7.31,0.497)(7.36,0.527)(7.38,0.537)
\curveto(7.4,0.547)(7.445,0.567)(7.47,0.577)
\curveto(7.495,0.587)(7.55,0.597)(7.58,0.597)
\curveto(7.61,0.597)(7.685,0.602)(7.73,0.607)
\curveto(7.775,0.612)(7.85,0.627)(7.88,0.637)
\curveto(7.91,0.647)(7.965,0.657)(7.99,0.657)
\curveto(8.015,0.657)(8.07,0.657)(8.1,0.657)
\curveto(8.13,0.657)(8.2,0.657)(8.24,0.657)
\curveto(8.28,0.657)(8.355,0.652)(8.39,0.647)
\curveto(8.425,0.642)(8.485,0.637)(8.51,0.637)
\curveto(8.535,0.637)(8.59,0.627)(8.62,0.617)
\curveto(8.65,0.607)(8.71,0.592)(8.74,0.587)
\curveto(8.77,0.582)(8.825,0.577)(8.85,0.577)
\curveto(8.875,0.577)(8.925,0.577)(8.95,0.577)
\curveto(8.975,0.577)(9.03,0.577)(9.06,0.577)
\curveto(9.09,0.577)(9.145,0.582)(9.17,0.587)
\curveto(9.195,0.592)(9.245,0.602)(9.27,0.607)
\curveto(9.295,0.612)(9.345,0.617)(9.37,0.617)
\curveto(9.395,0.617)(9.45,0.617)(9.48,0.617)
\curveto(9.51,0.617)(9.565,0.622)(9.59,0.627)
\curveto(9.615,0.632)(9.67,0.637)(9.7,0.637)
\curveto(9.73,0.637)(9.785,0.637)(9.81,0.637)
\curveto(9.835,0.637)(9.89,0.637)(9.92,0.637)
\curveto(9.95,0.637)(10.005,0.637)(10.03,0.637)
\curveto(10.055,0.637)(10.12,0.637)(10.16,0.637)
\curveto(10.2,0.637)(10.27,0.637)(10.3,0.637)
\curveto(10.33,0.637)(10.395,0.637)(10.43,0.637)
\curveto(10.465,0.637)(10.525,0.637)(10.55,0.637)
\curveto(10.575,0.637)(10.62,0.642)(10.64,0.647)
\curveto(10.66,0.652)(10.7,0.657)(10.76,0.657)
}
\pscustom[linewidth=0.04]
{
\newpath
\moveto(4.32,-0.343)
\lineto(4.24,-0.363)
\curveto(4.2,-0.373)(4.135,-0.383)(4.11,-0.383)
\curveto(4.085,-0.383)(4.035,-0.388)(4.01,-0.393)
\curveto(3.985,-0.398)(3.93,-0.408)(3.9,-0.413)
\curveto(3.87,-0.418)(3.81,-0.428)(3.78,-0.433)
\curveto(3.75,-0.438)(3.69,-0.443)(3.66,-0.443)
\curveto(3.63,-0.443)(3.57,-0.448)(3.54,-0.453)
\curveto(3.51,-0.458)(3.45,-0.463)(3.42,-0.463)
\curveto(3.39,-0.463)(3.335,-0.463)(3.31,-0.463)
\curveto(3.285,-0.463)(3.235,-0.458)(3.21,-0.453)
\curveto(3.185,-0.448)(3.135,-0.443)(3.11,-0.443)
\curveto(3.085,-0.443)(3.04,-0.433)(3.02,-0.423)
\curveto(3.0,-0.413)(2.95,-0.388)(2.92,-0.373)
\curveto(2.89,-0.358)(2.835,-0.333)(2.81,-0.323)
\curveto(2.785,-0.313)(2.735,-0.303)(2.71,-0.303)
\curveto(2.685,-0.303)(2.64,-0.298)(2.62,-0.293)
\curveto(2.6,-0.288)(2.555,-0.283)(2.53,-0.283)
\curveto(2.505,-0.283)(2.46,-0.273)(2.44,-0.263)
\curveto(2.42,-0.253)(2.375,-0.243)(2.35,-0.243)
\curveto(2.325,-0.243)(2.275,-0.238)(2.25,-0.233)
\curveto(2.225,-0.228)(2.175,-0.218)(2.15,-0.213)
\curveto(2.125,-0.208)(2.075,-0.203)(2.05,-0.203)
\curveto(2.025,-0.203)(1.975,-0.208)(1.95,-0.213)
\curveto(1.925,-0.218)(1.88,-0.233)(1.86,-0.243)
\curveto(1.84,-0.253)(1.8,-0.268)(1.78,-0.273)
\curveto(1.76,-0.278)(1.715,-0.288)(1.69,-0.293)
\curveto(1.665,-0.298)(1.615,-0.313)(1.59,-0.323)
\curveto(1.565,-0.333)(1.515,-0.348)(1.49,-0.353)
\curveto(1.465,-0.358)(1.41,-0.363)(1.38,-0.363)
\curveto(1.35,-0.363)(1.29,-0.368)(1.26,-0.373)
\curveto(1.23,-0.378)(1.175,-0.388)(1.15,-0.393)
\curveto(1.125,-0.398)(1.075,-0.408)(1.05,-0.413)
\curveto(1.025,-0.418)(0.975,-0.423)(0.95,-0.423)
\curveto(0.925,-0.423)(0.875,-0.428)(0.85,-0.433)
\curveto(0.825,-0.438)(0.775,-0.438)(0.75,-0.433)
\curveto(0.725,-0.428)(0.675,-0.413)(0.65,-0.403)
\curveto(0.625,-0.393)(0.57,-0.368)(0.54,-0.353)
}
\pscustom[linewidth=0.04]
{
\newpath
\moveto(4.6,-0.703)
\lineto(4.52,-0.703)
\curveto(4.48,-0.703)(4.415,-0.708)(4.39,-0.713)
\curveto(4.365,-0.718)(4.32,-0.728)(4.3,-0.733)
\curveto(4.28,-0.738)(4.235,-0.748)(4.21,-0.753)
\curveto(4.185,-0.758)(4.135,-0.768)(4.11,-0.773)
\curveto(4.085,-0.778)(4.035,-0.788)(4.01,-0.793)
\curveto(3.985,-0.798)(3.935,-0.803)(3.91,-0.803)
\curveto(3.885,-0.803)(3.835,-0.808)(3.81,-0.813)
\curveto(3.785,-0.818)(3.735,-0.828)(3.71,-0.833)
\curveto(3.685,-0.838)(3.635,-0.843)(3.61,-0.843)
\curveto(3.585,-0.843)(3.535,-0.843)(3.51,-0.843)
\curveto(3.485,-0.843)(3.415,-0.843)(3.37,-0.843)
\curveto(3.325,-0.843)(3.255,-0.843)(3.23,-0.843)
\curveto(3.205,-0.843)(3.145,-0.843)(3.11,-0.843)
\curveto(3.075,-0.843)(3.01,-0.843)(2.98,-0.843)
\curveto(2.95,-0.843)(2.895,-0.838)(2.87,-0.833)
\curveto(2.845,-0.828)(2.8,-0.818)(2.78,-0.813)
\curveto(2.76,-0.808)(2.715,-0.798)(2.69,-0.793)
\curveto(2.665,-0.788)(2.615,-0.778)(2.59,-0.773)
\curveto(2.565,-0.768)(2.515,-0.763)(2.49,-0.763)
\curveto(2.465,-0.763)(2.415,-0.758)(2.39,-0.753)
\curveto(2.365,-0.748)(2.32,-0.738)(2.3,-0.733)
\curveto(2.28,-0.728)(2.24,-0.718)(2.22,-0.713)
\curveto(2.2,-0.708)(2.16,-0.698)(2.14,-0.693)
\curveto(2.12,-0.688)(2.07,-0.683)(2.04,-0.683)
\curveto(2.01,-0.683)(1.95,-0.678)(1.92,-0.673)
\curveto(1.89,-0.668)(1.835,-0.663)(1.81,-0.663)
\curveto(1.785,-0.663)(1.735,-0.663)(1.71,-0.663)
\curveto(1.685,-0.663)(1.635,-0.663)(1.61,-0.663)
\curveto(1.585,-0.663)(1.535,-0.663)(1.51,-0.663)
\curveto(1.485,-0.663)(1.44,-0.668)(1.42,-0.673)
\curveto(1.4,-0.678)(1.355,-0.693)(1.33,-0.703)
\curveto(1.305,-0.713)(1.255,-0.733)(1.23,-0.743)
\curveto(1.205,-0.753)(1.16,-0.773)(1.14,-0.783)
\curveto(1.12,-0.793)(1.075,-0.808)(1.05,-0.813)
\curveto(1.025,-0.818)(0.975,-0.828)(0.95,-0.833)
\curveto(0.925,-0.838)(0.87,-0.843)(0.84,-0.843)
\curveto(0.81,-0.843)(0.755,-0.848)(0.73,-0.853)
\curveto(0.705,-0.858)(0.655,-0.868)(0.63,-0.873)
\curveto(0.605,-0.878)(0.555,-0.888)(0.53,-0.893)
\curveto(0.505,-0.898)(0.465,-0.908)(0.42,-0.923)
}
\pscustom[linewidth=0.04]
{
\newpath
\moveto(6.18,-0.183)
\lineto(6.25,-0.193)
\curveto(6.285,-0.198)(6.335,-0.213)(6.35,-0.223)
\curveto(6.365,-0.233)(6.405,-0.248)(6.43,-0.253)
\curveto(6.455,-0.258)(6.51,-0.268)(6.54,-0.273)
\curveto(6.57,-0.278)(6.625,-0.288)(6.65,-0.293)
\curveto(6.675,-0.298)(6.73,-0.308)(6.76,-0.313)
\curveto(6.79,-0.318)(6.88,-0.323)(6.94,-0.323)
\curveto(7.0,-0.323)(7.105,-0.323)(7.15,-0.323)
\curveto(7.195,-0.323)(7.265,-0.323)(7.29,-0.323)
\curveto(7.315,-0.323)(7.365,-0.318)(7.39,-0.313)
\curveto(7.415,-0.308)(7.47,-0.298)(7.5,-0.293)
\curveto(7.53,-0.288)(7.59,-0.283)(7.62,-0.283)
\curveto(7.65,-0.283)(7.71,-0.278)(7.74,-0.273)
\curveto(7.77,-0.268)(7.825,-0.258)(7.85,-0.253)
\curveto(7.875,-0.248)(7.925,-0.238)(7.95,-0.233)
\curveto(7.975,-0.228)(8.02,-0.218)(8.04,-0.213)
\curveto(8.06,-0.208)(8.105,-0.203)(8.13,-0.203)
\curveto(8.155,-0.203)(8.215,-0.203)(8.25,-0.203)
\curveto(8.285,-0.203)(8.35,-0.203)(8.38,-0.203)
\curveto(8.41,-0.203)(8.465,-0.203)(8.49,-0.203)
\curveto(8.515,-0.203)(8.575,-0.203)(8.61,-0.203)
\curveto(8.645,-0.203)(8.715,-0.203)(8.75,-0.203)
\curveto(8.785,-0.203)(8.85,-0.203)(8.88,-0.203)
\curveto(8.91,-0.203)(8.97,-0.203)(9.0,-0.203)
\curveto(9.03,-0.203)(9.085,-0.203)(9.11,-0.203)
\curveto(9.135,-0.203)(9.19,-0.203)(9.22,-0.203)
\curveto(9.25,-0.203)(9.305,-0.203)(9.33,-0.203)
\curveto(9.355,-0.203)(9.405,-0.203)(9.43,-0.203)
\curveto(9.455,-0.203)(9.505,-0.203)(9.53,-0.203)
\curveto(9.555,-0.203)(9.61,-0.203)(9.64,-0.203)
\curveto(9.67,-0.203)(9.725,-0.203)(9.75,-0.203)
\curveto(9.775,-0.203)(9.825,-0.203)(9.85,-0.203)
\curveto(9.875,-0.203)(9.93,-0.208)(9.96,-0.213)
\curveto(9.99,-0.218)(10.045,-0.223)(10.07,-0.223)
\curveto(10.095,-0.223)(10.145,-0.233)(10.17,-0.243)
\curveto(10.195,-0.253)(10.245,-0.263)(10.27,-0.263)
\curveto(10.295,-0.263)(10.35,-0.268)(10.38,-0.273)
\curveto(10.41,-0.278)(10.465,-0.283)(10.49,-0.283)
\curveto(10.515,-0.283)(10.565,-0.283)(10.59,-0.283)
\curveto(10.615,-0.283)(10.665,-0.283)(10.69,-0.283)
\curveto(10.715,-0.283)(10.75,-0.283)(10.78,-0.283)
}
\pscustom[linewidth=0.04]
{
\newpath
\moveto(6.02,-0.583)
\lineto(6.07,-0.593)
\curveto(6.095,-0.598)(6.15,-0.608)(6.18,-0.613)
\curveto(6.21,-0.618)(6.265,-0.628)(6.29,-0.633)
\curveto(6.315,-0.638)(6.365,-0.643)(6.39,-0.643)
\curveto(6.415,-0.643)(6.47,-0.643)(6.5,-0.643)
\curveto(6.53,-0.643)(6.585,-0.643)(6.61,-0.643)
\curveto(6.635,-0.643)(6.685,-0.643)(6.71,-0.643)
\curveto(6.735,-0.643)(6.785,-0.643)(6.81,-0.643)
\curveto(6.835,-0.643)(6.885,-0.643)(6.91,-0.643)
\curveto(6.935,-0.643)(6.985,-0.648)(7.01,-0.653)
\curveto(7.035,-0.658)(7.08,-0.668)(7.1,-0.673)
\curveto(7.12,-0.678)(7.165,-0.688)(7.19,-0.693)
\curveto(7.215,-0.698)(7.265,-0.708)(7.29,-0.713)
\curveto(7.315,-0.718)(7.37,-0.723)(7.4,-0.723)
\curveto(7.43,-0.723)(7.5,-0.723)(7.54,-0.723)
\curveto(7.58,-0.723)(7.655,-0.723)(7.69,-0.723)
\curveto(7.725,-0.723)(7.79,-0.723)(7.82,-0.723)
\curveto(7.85,-0.723)(7.91,-0.723)(7.94,-0.723)
\curveto(7.97,-0.723)(8.095,-0.713)(8.19,-0.703)
\curveto(8.285,-0.693)(8.405,-0.683)(8.43,-0.683)
\curveto(8.455,-0.683)(8.51,-0.683)(8.54,-0.683)
\curveto(8.57,-0.683)(8.625,-0.683)(8.65,-0.683)
\curveto(8.675,-0.683)(8.74,-0.683)(8.78,-0.683)
\curveto(8.82,-0.683)(8.9,-0.678)(8.94,-0.673)
\curveto(8.98,-0.668)(9.08,-0.663)(9.14,-0.663)
\curveto(9.2,-0.663)(9.285,-0.663)(9.31,-0.663)
\curveto(9.335,-0.663)(9.385,-0.663)(9.41,-0.663)
\curveto(9.435,-0.663)(9.49,-0.663)(9.52,-0.663)
\curveto(9.55,-0.663)(9.605,-0.663)(9.63,-0.663)
\curveto(9.655,-0.663)(9.71,-0.663)(9.74,-0.663)
\curveto(9.77,-0.663)(9.825,-0.663)(9.85,-0.663)
\curveto(9.875,-0.663)(9.925,-0.663)(9.95,-0.663)
\curveto(9.975,-0.663)(10.025,-0.663)(10.05,-0.663)
\curveto(10.075,-0.663)(10.125,-0.663)(10.15,-0.663)
\curveto(10.175,-0.663)(10.225,-0.663)(10.25,-0.663)
\curveto(10.275,-0.663)(10.325,-0.663)(10.35,-0.663)
\curveto(10.375,-0.663)(10.425,-0.663)(10.45,-0.663)
\curveto(10.475,-0.663)(10.525,-0.663)(10.55,-0.663)
\curveto(10.575,-0.663)(10.625,-0.663)(10.65,-0.663)
\curveto(10.675,-0.663)(10.725,-0.663)(10.75,-0.663)
\curveto(10.775,-0.663)(10.81,-0.653)(10.84,-0.623)
}
\pscustom[linewidth=0.04]
{
\newpath
\moveto(0.6017364,-1.168705)
\lineto(0.7419229,-1.1756153)
\curveto(0.812016,-1.1790706)(0.92181915,-1.1873366)(0.96152896,-1.1921474)
\curveto(1.0012383,-1.1969582)(1.0707306,-1.2053767)(1.1005127,-1.2089851)
\curveto(1.1302948,-1.2125936)(1.199787,-1.2210121)(1.2394965,-1.2258229)
\curveto(1.2792059,-1.2306336)(1.3635888,-1.2408564)(1.4082624,-1.2462684)
\curveto(1.4529358,-1.2516804)(1.5224277,-1.2600996)(1.5472461,-1.2631062)
\curveto(1.5720649,-1.2661128)(1.6217017,-1.272126)(1.6465204,-1.2751325)
\curveto(1.6713392,-1.2781391)(1.720976,-1.2841536)(1.7457947,-1.2871602)
\curveto(1.7706131,-1.2901667)(1.8196487,-1.3011439)(1.843866,-1.3091145)
\curveto(1.8680832,-1.3170844)(1.9220825,-1.3286629)(1.9518646,-1.3322712)
\curveto(1.9816468,-1.3358797)(2.0362475,-1.3424946)(2.0610664,-1.3455012)
\curveto(2.085885,-1.3485078)(2.1299567,-1.3588837)(2.1492102,-1.3662525)
\curveto(2.1684637,-1.3736213)(2.2305865,-1.4012941)(2.2734559,-1.4215974)
\curveto(2.3163252,-1.4419001)(2.3840132,-1.4652107)(2.4088318,-1.4682173)
\curveto(2.43365,-1.4712238)(2.4826856,-1.4822011)(2.5069032,-1.4901717)
\curveto(2.5311205,-1.4981416)(2.5845184,-1.514684)(2.6136994,-1.5232557)
\curveto(2.6428802,-1.5318276)(2.696278,-1.5483699)(2.7204957,-1.5563399)
\curveto(2.744713,-1.5643104)(2.7981102,-1.5808527)(2.8272913,-1.5894246)
\curveto(2.8564723,-1.5979964)(2.910472,-1.6095747)(2.9352906,-1.6125813)
\curveto(2.9601088,-1.615588)(3.0147095,-1.6222029)(3.0444922,-1.6258113)
\curveto(3.0742743,-1.6294197)(3.1338384,-1.6366353)(3.1636207,-1.6402436)
\curveto(3.1934032,-1.643852)(3.2480042,-1.650467)(3.2728224,-1.6534736)
\curveto(3.297641,-1.6564802)(3.3472779,-1.6624935)(3.3720965,-1.6655)
\curveto(3.3969152,-1.6685066)(3.446552,-1.6745198)(3.471371,-1.6775264)
\curveto(3.4961889,-1.6805329)(3.5607178,-1.6883509)(3.6004272,-1.6931617)
\curveto(3.6401367,-1.6979725)(3.7145922,-1.7069929)(3.7493384,-1.7112019)
\curveto(3.7840846,-1.7154115)(3.848612,-1.7232295)(3.8783948,-1.7268373)
\curveto(3.908177,-1.730445)(3.967741,-1.7376618)(3.9975233,-1.7412697)
\curveto(4.027306,-1.7448775)(4.0918336,-1.7526954)(4.1265798,-1.7569051)
\curveto(4.1613255,-1.761114)(4.230818,-1.7695332)(4.265564,-1.7737422)
\curveto(4.30031,-1.7779511)(4.3747654,-1.7869716)(4.414475,-1.7917824)
\curveto(4.4541845,-1.7965932)(4.5286403,-1.8056135)(4.5633864,-1.8098226)
\curveto(4.598132,-1.8140321)(4.6582985,-1.8162849)(4.683718,-1.8143281)
\curveto(4.7091384,-1.8123714)(4.7699046,-1.8096602)(4.805252,-1.8089057)
\curveto(4.8405995,-1.8081514)(4.911294,-1.8066425)(4.9466414,-1.8058882)
\curveto(4.9819884,-1.8051338)(5.0427556,-1.8024226)(5.068176,-1.8004658)
\curveto(5.0935955,-1.798509)(5.143834,-1.7995589)(5.1686525,-1.8025655)
\curveto(5.193471,-1.805572)(5.273492,-1.8102297)(5.328694,-1.8118806)
\curveto(5.383896,-1.8135316)(5.4645176,-1.8132259)(5.489938,-1.811269)
\curveto(5.5153575,-1.8093122)(5.565596,-1.8103621)(5.590415,-1.8133687)
\curveto(5.615233,-1.8163753)(5.669834,-1.8229902)(5.699617,-1.826598)
\curveto(5.7293987,-1.8302058)(5.7988906,-1.838625)(5.8385997,-1.8434358)
\curveto(5.8783092,-1.8482466)(5.9478016,-1.8566657)(5.9775844,-1.8602736)
\curveto(6.007366,-1.8638813)(6.0718946,-1.8716993)(6.106641,-1.875909)
\curveto(6.1413865,-1.8801179)(6.210878,-1.8885372)(6.2456236,-1.8927461)
\curveto(6.2803698,-1.8969551)(6.344898,-1.9047725)(6.37468,-1.9083809)
\curveto(6.404463,-1.9119892)(6.4689903,-1.9198066)(6.5037365,-1.9240156)
\curveto(6.5384827,-1.9282253)(6.6179023,-1.9378468)(6.6625757,-1.9432588)
\curveto(6.7072487,-1.9486713)(6.7866673,-1.958293)(6.8214135,-1.962502)
\curveto(6.8561597,-1.9667115)(6.925652,-1.9751308)(6.9603977,-1.9793403)
\curveto(6.995144,-1.9835494)(7.0596714,-1.9913667)(7.089454,-1.9949751)
\curveto(7.1192365,-1.9985828)(7.173837,-2.0051973)(7.198656,-2.0082037)
\curveto(7.223474,-2.0112104)(7.2780747,-2.0178254)(7.3078575,-2.0214338)
\curveto(7.33764,-2.0250423)(7.402168,-2.0328596)(7.436914,-2.0370686)
\curveto(7.47166,-2.0412781)(7.5560427,-2.051501)(7.6056786,-2.0575147)
\curveto(7.6553164,-2.0635285)(7.7446632,-2.0743525)(7.7843723,-2.0791633)
\curveto(7.824082,-2.0839741)(7.893574,-2.0923927)(7.923357,-2.0960004)
\curveto(7.95314,-2.099609)(8.007739,-2.1062238)(8.032558,-2.1092305)
\curveto(8.057377,-2.112237)(8.111978,-2.1188521)(8.14176,-2.12246)
\curveto(8.171543,-2.1260676)(8.231106,-2.1332843)(8.260889,-2.1368923)
\curveto(8.290671,-2.1405)(8.354597,-2.1532815)(8.388742,-2.1624544)
\curveto(8.422887,-2.171628)(8.501704,-2.1862128)(8.546378,-2.1916256)
\curveto(8.591052,-2.1970375)(8.670471,-2.206659)(8.705217,-2.2108686)
\curveto(8.739964,-2.2150776)(8.804492,-2.222895)(8.834273,-2.2265034)
\curveto(8.864056,-2.2301118)(8.918656,-2.2367263)(8.943474,-2.2397327)
\curveto(8.968293,-2.2427394)(9.027858,-2.2499557)(9.062604,-2.2541645)
\curveto(9.09735,-2.2583742)(9.166241,-2.2717566)(9.200386,-2.2809298)
\curveto(9.23453,-2.2901032)(9.298457,-2.302884)(9.3282385,-2.3064923)
\curveto(9.35802,-2.3101008)(9.422549,-2.3179183)(9.457294,-2.322127)
\curveto(9.492041,-2.3263369)(9.551605,-2.333553)(9.576424,-2.3365595)
\curveto(9.601243,-2.3395662)(9.655842,-2.3461812)(9.685625,-2.3497896)
\curveto(9.715407,-2.3533978)(9.779937,-2.3612154)(9.814683,-2.3654244)
\curveto(9.849429,-2.369634)(9.908391,-2.381814)(9.932608,-2.3897846)
\curveto(9.956825,-2.3977547)(10.015789,-2.4099348)(10.050534,-2.4141438)
\curveto(10.08528,-2.4183533)(10.14981,-2.4261708)(10.179591,-2.4297786)
\curveto(10.2093725,-2.433387)(10.263973,-2.440002)(10.288792,-2.4430087)
\curveto(10.313611,-2.4460151)(10.363247,-2.4520283)(10.388066,-2.455035)
\curveto(10.412886,-2.4580414)(10.462522,-2.464056)(10.487341,-2.4670625)
\curveto(10.512159,-2.4700692)(10.56676,-2.476684)(10.596543,-2.4802918)
\curveto(10.626326,-2.4838996)(10.680925,-2.4905148)(10.705745,-2.4935212)
\curveto(10.730563,-2.496528)(10.760946,-2.4951727)(10.777641,-2.4820855)
}
\psdots[dotsize=0.12](0.52,-0.583)
\psdots[dotsize=0.12](0.74,-0.743)
\psdots[dotsize=0.12](0.84,-0.563)
\psdots[dotsize=0.12](0.94,-0.683)
\psdots[dotsize=0.12](1.1,-0.543)
\psdots[dotsize=0.12](1.14,-0.663)
\psdots[dotsize=0.12](1.34,-0.483)
\psdots[dotsize=0.12](1.46,-0.583)
\psdots[dotsize=0.12](1.62,-0.403)
\psdots[dotsize=0.12](1.78,-0.563)
\psdots[dotsize=0.12](1.86,-0.383)
\psdots[dotsize=0.12](2.08,-0.303)
\psdots[dotsize=0.12](2.0,-0.443)
\psdots[dotsize=0.12](1.96,-0.583)
\psdots[dotsize=0.12](2.18,-0.643)
\psdots[dotsize=0.12](2.16,-0.463)
\psdots[dotsize=0.12](2.32,-0.383)
\psdots[dotsize=0.12](2.48,-0.383)
\psdots[dotsize=0.12](2.44,-0.583)
\psdots[dotsize=0.12](2.6,-0.543)
\psdots[dotsize=0.12](2.76,-0.423)
\psdots[dotsize=0.12](2.9,-0.483)
\psdots[dotsize=0.12](2.8,-0.603)
\psdots[dotsize=0.12](2.68,-0.683)
\psdots[dotsize=0.12](2.88,-0.743)
\psdots[dotsize=0.12](2.98,-0.643)
\psdots[dotsize=0.12](3.08,-0.543)
\psdots[dotsize=0.12](3.3,-0.543)
\psdots[dotsize=0.12](3.14,-0.643)
\psdots[dotsize=0.12](3.26,-0.743)
\psdots[dotsize=0.12](3.46,-0.663)
\psdots[dotsize=0.12](3.6,-0.503)
\psdots[dotsize=0.12](3.6,-0.703)
\psdots[dotsize=0.12](3.7,-0.603)
\psdots[dotsize=0.12](3.82,-0.743)
\psdots[dotsize=0.12](3.9,-0.583)
\psdots[dotsize=0.12](3.98,-0.483)
\psdots[dotsize=0.12](4.04,-0.623)
\psdots[dotsize=0.12](4.28,-0.423)
\psdots[dotsize=0.12](4.24,-0.583)
\psdots[dotsize=0.12](4.38,-0.603)
\psdots[dotsize=0.12](6.16,-0.503)
\psdots[dotsize=0.12](6.38,-0.323)
\psdots[dotsize=0.12](6.56,-0.383)
\psdots[dotsize=0.12](6.54,-0.563)
\psdots[dotsize=0.12](6.76,-0.523)
\psdots[dotsize=0.12](6.86,-0.403)
\psdots[dotsize=0.12](7.0,-0.563)
\psdots[dotsize=0.12](7.1,-0.423)
\psdots[dotsize=0.12](7.18,-0.583)
\psdots[dotsize=0.12](7.34,-0.443)
\psdots[dotsize=0.12](7.62,-0.363)
\psdots[dotsize=0.12](7.52,-0.463)
\psdots[dotsize=0.12](7.4,-0.583)
\psdots[dotsize=0.12](7.54,-0.623)
\psdots[dotsize=0.12](7.78,-0.643)
\psdots[dotsize=0.12](8.0,-0.643)
\psdots[dotsize=0.12](7.8,-0.483)
\psdots[dotsize=0.12](7.94,-0.323)
\psdots[dotsize=0.12](8.0,-0.463)
\psdots[dotsize=0.12](8.14,-0.303)
\psdots[dotsize=0.12](8.4,-0.323)
\psdots[dotsize=0.12](8.26,-0.443)
\psdots[dotsize=0.12](8.18,-0.603)
\psdots[dotsize=0.12](8.42,-0.583)
\psdots[dotsize=0.12](8.66,-0.603)
\psdots[dotsize=0.12](8.52,-0.403)
\psdots[dotsize=0.12](8.74,-0.303)
\psdots[dotsize=0.12](8.76,-0.423)
\psdots[dotsize=0.12](8.96,-0.303)
\psdots[dotsize=0.12](8.94,-0.543)
\psdots[dotsize=0.12](9.2,-0.583)
\psdots[dotsize=0.12](9.06,-0.403)
\psdots[dotsize=0.12](9.28,-0.343)
\psdots[dotsize=0.12](9.44,-0.563)
\psdots[dotsize=0.12](9.44,-0.423)
\psdots[dotsize=0.12](9.58,-0.303)
\psdots[dotsize=0.12](9.66,-0.443)
\psdots[dotsize=0.12](9.8,-0.583)
\psdots[dotsize=0.12](9.78,-0.403)
\psdots[dotsize=0.12](9.9,-0.303)
\psdots[dotsize=0.12](9.92,-0.463)
\psdots[dotsize=0.12](10.12,-0.603)
\psdots[dotsize=0.12](10.1,-0.403)
\psdots[dotsize=0.12](10.32,-0.363)
\psdots[dotsize=0.12](10.28,-0.503)
\psdots[dotsize=0.12](10.42,-0.543)
\psdots[dotsize=0.12](10.62,-0.383)
\psdots[dotsize=0.12](10.72,-0.583)
\psdots[dotsize=0.12,dotangle=-6.907633](0.6889838,-1.2800063)
\psdots[dotsize=0.12,dotangle=-6.907633](0.6252203,-1.4737438)
\psdots[dotsize=0.12,dotangle=-6.907633](0.8165524,-1.5573621)
\psdots[dotsize=0.12,dotangle=-6.907633](0.818346,-1.3762633)
\psdots[dotsize=0.12,dotangle=-6.907633](1.0873039,-1.650602)
\psdots[dotsize=0.12,dotangle=-6.907633](1.029533,-1.462287)
\psdots[dotsize=0.12,dotangle=-6.907633](1.1853545,-1.3401408)
\psdots[dotsize=0.12,dotangle=-6.907633](1.297879,-1.5753816)
\psdots[dotsize=0.12,dotangle=-6.907633](1.379092,-1.4039043)
\psdots[dotsize=0.12,dotangle=-6.907633](1.5313262,-1.6439558)
\psdots[dotsize=0.12,dotangle=-6.907633](1.43988,-1.7336085)
\psdots[dotsize=0.12,dotangle=-6.907633](1.6907766,-1.8244429)
\psdots[dotsize=0.12,dotangle=-6.907633](1.5427413,-1.3834378)
\psdots[dotsize=0.12,dotangle=-6.907633](1.689553,-1.5019549)
\psdots[dotsize=0.12,dotangle=-6.907633](1.8429691,-1.3996636)
\psdots[dotsize=0.12,dotangle=-6.907633](1.7298745,-1.6680096)
\psdots[dotsize=0.12,dotangle=-6.907633](1.8808851,-1.5855732)
\psdots[dotsize=0.12,dotangle=-6.907633](2.0144465,-1.4808766)
\psdots[dotsize=0.12,dotangle=-6.907633](2.138386,-1.4555992)
\psdots[dotsize=0.12,dotangle=-6.907633](2.077028,-1.6294819)
\psdots[dotsize=0.12,dotangle=-6.907633](1.9458722,-1.7143236)
\psdots[dotsize=0.12,dotangle=-6.907633](2.0108593,-1.8430742)
\psdots[dotsize=0.12,dotangle=-6.907633](2.2292624,-1.8695334)
\psdots[dotsize=0.12,dotangle=-6.907633](2.2557216,-1.6511303)
\psdots[dotsize=0.12,dotangle=-6.907633](2.4265869,-1.5710992)
\psdots[dotsize=0.12,dotangle=-6.907633](2.4049385,-1.7497927)
\psdots[dotsize=0.12,dotangle=-6.907633](2.4548814,-1.836428)
\psdots[dotsize=0.12,dotangle=-6.907633](2.6089094,-1.8953809)
\psdots[dotsize=0.12,dotangle=-6.907633](2.5535438,-1.687211)
\psdots[dotsize=0.12,dotangle=-6.907633](2.7743523,-1.6938154)
\psdots[dotsize=0.12,dotangle=-6.907633](2.7352545,-1.8502486)
\psdots[dotsize=0.12,dotangle=-6.907633](2.8844717,-1.948911)
\psdots[dotsize=0.12,dotangle=-6.907633](3.0102048,-1.7425348)
\psdots[dotsize=0.12,dotangle=-6.907633](3.0854251,-1.95311)
\psdots[dotsize=0.12,dotangle=-6.907633](3.2141757,-1.8881229)
\psdots[dotsize=0.12,dotangle=-6.907633](3.3080273,-1.7786155)
\psdots[dotsize=0.12,dotangle=-6.907633](3.350754,-1.9248154)
\psdots[dotsize=0.12,dotangle=-6.907633](3.4157412,-2.053566)
\psdots[dotsize=0.12,dotangle=-6.907633](3.5168087,-1.8844941)
\psdots[dotsize=0.12,dotangle=-6.907633](3.6852689,-1.8243178)
\psdots[dotsize=0.12,dotangle=-6.907633](3.6636205,-2.0030112)
\psdots[dotsize=0.12,dotangle=-6.907633](3.5673635,-2.1323736)
\psdots[dotsize=0.12,dotangle=-6.907633](3.7635064,-2.1762822)
\psdots[dotsize=0.12,dotangle=-6.907633](3.8026044,-2.0198488)
\psdots[dotsize=0.12,dotangle=-6.907633](3.861557,-1.865821)
\psdots[dotsize=0.12,dotangle=-6.907633](4.0029464,-1.8628039)
\psdots[dotsize=0.12,dotangle=-6.907633](4.318218,-1.9211448)
\psdots[dotsize=0.12,dotangle=-6.907633](4.156974,-1.9217566)
\psdots[dotsize=0.12,dotangle=-6.907633](3.9271557,-2.1558156)
\psdots[dotsize=0.12,dotangle=-6.907633](4.1509814,-2.3038092)
\psdots[dotsize=0.12,dotangle=-6.907633](4.19489,-2.1076663)
\psdots[dotsize=0.12,dotangle=-6.907633](4.4499855,-1.997547)
\psdots[dotsize=0.12,dotangle=-6.907633](4.6009965,-1.9151105)
\psdots[dotsize=0.12,dotangle=-6.907633](4.4632363,-2.2207608)
\psdots[dotsize=0.12,dotangle=-6.907633](4.3314686,-2.1443586)
\psdots[dotsize=0.12,dotangle=-6.907633](4.317036,-2.2634876)
\psdots[dotsize=0.12,dotangle=-6.907633](4.674423,-2.3067846)
\psdots[dotsize=0.12,dotangle=-6.907633](4.636507,-2.120875)
\psdots[dotsize=0.12,dotangle=-6.907633](4.77969,-1.936759)
\psdots[dotsize=0.12,dotangle=-6.907633](4.9433393,-1.9162923)
\psdots[dotsize=0.12,dotangle=-6.907633](4.881981,-2.0901752)
\psdots[dotsize=0.12,dotangle=-6.907633](4.904853,-2.2339697)
\psdots[dotsize=0.12,dotangle=-6.907633](5.0835466,-2.255618)
\psdots[dotsize=0.12,dotangle=-6.907633](5.0408196,-2.1094182)
\psdots[dotsize=0.12,dotangle=-6.907633](5.139482,-1.960201)
\psdots[dotsize=0.12,dotangle=-6.907633](5.3055367,-1.9198796)
\psdots[dotsize=0.12,dotangle=-6.907633](5.561244,-1.9710044)
\psdots[dotsize=0.12,dotangle=-6.907633](5.3308144,-2.0438192)
\psdots[dotsize=0.12,dotangle=-6.907633](5.4752207,-2.1821914)
\psdots[dotsize=0.12,dotangle=-6.907633](5.2670507,-2.2375567)
\psdots[dotsize=0.12,dotangle=-6.907633](5.673769,-2.2062452)
\psdots[dotsize=0.12,dotangle=-6.907633](5.912027,-2.2351098)
\psdots[dotsize=0.12,dotangle=-6.907633](6.1352406,-2.2218592)
\psdots[dotsize=0.12,dotangle=-6.907633](6.336194,-2.2260582)
\psdots[dotsize=0.12,dotangle=-6.907633](6.614162,-2.2597337)
\psdots[dotsize=0.12,dotangle=-6.907633](6.909579,-2.315669)
\psdots[dotsize=0.12,dotangle=-6.907633](5.8217626,-1.9824195)
\psdots[dotsize=0.12,dotangle=-6.907633](5.8470397,-2.1063592)
\psdots[dotsize=0.12,dotangle=-6.907633](6.1123686,-2.0780647)
\psdots[dotsize=0.12,dotangle=-6.907633](6.3554373,-2.0672197)
\psdots[dotsize=0.12,dotangle=-6.907633](6.5118704,-2.1063175)
\psdots[dotsize=0.12,dotangle=-6.907633](6.7675776,-2.1574423)
\psdots[dotsize=0.12,dotangle=-6.907633](7.0106463,-2.1465971)
\psdots[dotsize=0.12,dotangle=-6.907633](7.0756335,-2.2753477)
\psdots[dotsize=0.12,dotangle=-6.907633](7.2091947,-2.170651)
\psdots[dotsize=0.12,dotangle=-6.907633](7.269371,-2.339111)
\psdots[dotsize=0.12,dotangle=-6.907633](7.3830776,-2.2320092)
\psdots[dotsize=0.12,dotangle=-6.907633](7.5936527,-2.1567888)
\psdots[dotsize=0.12,dotangle=-6.907633](7.5124397,-2.3282661)
\psdots[dotsize=0.12,dotangle=-6.907633](7.398733,-2.4353683)
\psdots[dotsize=0.12,dotangle=-6.907633](6.6628814,-2.023881)
\psdots[dotsize=0.12,dotangle=-6.907633](7.5924706,-2.4991317)
\psdots[dotsize=0.12,dotangle=-6.907633](7.6490183,-2.3649585)
\psdots[dotsize=0.12,dotangle=-6.907633](7.8517656,-2.1880589)
\psdots[dotsize=0.12,dotangle=-6.907633](8.169442,-2.2265449)
\psdots[dotsize=0.12,dotangle=-6.907633](8.020838,-2.2891266)
\psdots[dotsize=0.12,dotangle=-6.907633](7.9570737,-2.4828641)
\psdots[dotsize=0.12,dotangle=-6.907633](7.817478,-2.3047824)
\psdots[dotsize=0.12,dotangle=-6.907633](7.7934246,-2.5033307)
\psdots[dotsize=0.12,dotangle=-6.907633](8.106291,-2.5815265)
\psdots[dotsize=0.12,dotangle=-6.907633](8.352377,-2.7120705)
\psdots[dotsize=0.12,dotangle=-6.907633](8.242257,-2.456975)
\psdots[dotsize=0.12,dotangle=-6.907633](8.440194,-2.3197846)
\psdots[dotsize=0.12,dotangle=-6.907633](8.423356,-2.4587686)
\psdots[dotsize=0.12,dotangle=-6.907633](8.665813,-2.2866795)
\psdots[dotsize=0.12,dotangle=-6.907633](8.92152,-2.3378043)
\psdots[dotsize=0.12,dotangle=-6.907633](8.830687,-2.5887008)
\psdots[dotsize=0.12,dotangle=-6.907633](8.635155,-2.706036)
\psdots[dotsize=0.12,dotangle=-6.907633](8.621904,-2.4828224)
\psdots[dotsize=0.12,dotangle=-6.907633](8.873413,-2.7349007)
\psdots[dotsize=0.12,dotangle=-6.907633](9.03164,-2.5928998)
\psdots[dotsize=0.12,dotangle=-6.907633](8.735611,-2.3757203)
\psdots[dotsize=0.12,dotangle=-6.907633](9.299375,-2.5447505)
\psdots[dotsize=0.12,dotangle=-6.907633](9.530416,-2.6331797)
\psdots[dotsize=0.12,dotangle=-6.907633](10.550228,-2.6962895)
\psdots[dotsize=0.12,dotangle=-6.907633](10.319186,-2.6078606)
\psdots[dotsize=0.12,dotangle=-6.907633](9.956988,-2.6042733)
\psdots[dotsize=0.12,dotangle=-6.907633](9.701282,-2.5531485)
\psdots[dotsize=0.12,dotangle=-6.907633](9.830032,-2.4881616)
\psdots[dotsize=0.12,dotangle=-6.907633](9.512355,-2.4496753)
\psdots[dotsize=0.12,dotangle=-6.907633](9.245233,-2.6590686)






\psdots[dotsize=0.12,dotangle=-6.907633](9.097809,-2.3793075)
\psdots[dotsize=0.12,dotangle=-6.907633](10.701239,-2.6138532)
\psdots[dotsize=0.12,dotangle=-6.907633](10.085739,-2.5392864)
\psdots[dotsize=0.12,dotangle=-6.907633](6.025121,-1.9667637)
\psdots[dotsize=0.12,dotangle=-6.907633](9.805978,-2.6867096)
\psdots[dotsize=0.12,dotangle=-6.907633](1.2190715,-1.727004)
\psdots[dotsize=0.12,dotangle=-6.907633](0.9844008,-1.3359419)
\psdots[dotsize=0.12,dotangle=-6.907633](0.88153946,-1.6861126)
\psdots[dotsize=0.12,dotangle=-6.907633](0.58131164,-1.6698867)
\psdots[dotsize=0.12](6.32,-0.503)
\psline[linewidth=0.04cm](10.76,0.657)(10.76,-0.083)
\psline[linewidth=0.04cm](10.8,-0.923)(10.82,-2.243)
\psline[linewidth=0.04cm](10.84,-3.223)(10.86,-5.083)
\psline[linewidth=0.04cm](11.78,0.797)(11.86,-5.163)
\pscustom[linewidth=0.04]
{
\newpath
\moveto(11.76,0.777)
\lineto(11.82,0.807)
\curveto(11.85,0.822)(11.91,0.847)(11.94,0.857)
\curveto(11.97,0.867)(12.02,0.882)(12.04,0.887)
\curveto(12.06,0.892)(12.115,0.897)(12.15,0.897)
\curveto(12.185,0.897)(12.27,0.902)(12.32,0.907)
\curveto(12.37,0.912)(12.485,0.907)(12.55,0.897)
\curveto(12.615,0.887)(12.74,0.842)(12.8,0.807)
\curveto(12.86,0.772)(12.985,0.687)(13.05,0.637)
\curveto(13.115,0.587)(13.2,0.507)(13.22,0.477)
\curveto(13.24,0.447)(13.275,0.397)(13.29,0.377)
\curveto(13.305,0.357)(13.335,0.322)(13.35,0.307)
\curveto(13.365,0.292)(13.4,0.292)(13.42,0.307)
\curveto(13.44,0.322)(13.475,0.347)(13.49,0.357)
\curveto(13.505,0.367)(13.535,0.397)(13.55,0.417)
\curveto(13.565,0.437)(13.595,0.467)(13.61,0.477)
\curveto(13.625,0.487)(13.655,0.507)(13.67,0.517)
\curveto(13.685,0.527)(13.725,0.537)(13.75,0.537)
\curveto(13.775,0.537)(13.805,0.542)(13.82,0.557)
}
\psline[linewidth=0.04cm](10.74,-0.042999998)(6.22,-0.163)
\psline[linewidth=0.04cm](10.8,-0.963)(6.12,-1.043)
\psline[linewidth=0.04cm](6.12,-1.043)(6.12,-0.563)
\pscustom[linewidth=0.04]
{
\newpath
\moveto(0.51453096,-1.7222352)
\lineto(0.60387754,-1.7330592)
\curveto(0.64855105,-1.7384712)(0.7478253,-1.7504976)(0.80242616,-1.7571125)
\curveto(0.857027,-1.7637275)(0.9414099,-1.7739509)(0.971192,-1.7775587)
\curveto(1.0009742,-1.7811666)(1.0500097,-1.7921443)(1.0692633,-1.7995131)
\curveto(1.0885168,-1.8068818)(1.1325891,-1.8172578)(1.1574075,-1.8202645)
\curveto(1.182226,-1.823271)(1.2312618,-1.8342488)(1.2554791,-1.8422188)
\curveto(1.279696,-1.8501893)(1.3336957,-1.8617684)(1.3634778,-1.8653761)
\curveto(1.3932599,-1.8689839)(1.4478607,-1.8755989)(1.4726795,-1.8786055)
\curveto(1.4974979,-1.8816121)(1.547135,-1.8876264)(1.5719534,-1.8906331)
\curveto(1.5967718,-1.8936397)(1.6513727,-1.9002546)(1.6811551,-1.9038625)
\curveto(1.7109375,-1.9074702)(1.7804291,-1.9158894)(1.8201388,-1.9207002)
\curveto(1.8598486,-1.925511)(1.9343042,-1.9345307)(1.9690499,-1.9387398)
\curveto(2.0037959,-1.9429493)(2.068324,-1.9507673)(2.0981064,-1.9543751)
\curveto(2.1278887,-1.9579829)(2.187453,-1.9651997)(2.2172353,-1.9688075)
\curveto(2.2470179,-1.9724153)(2.31651,-1.9808345)(2.3562195,-1.9856453)
\curveto(2.3959289,-1.9904561)(2.460457,-1.9982735)(2.485276,-2.00128)
\curveto(2.510094,-2.0042865)(2.564695,-2.0109017)(2.5944777,-2.01451)
\curveto(2.6242597,-2.0181184)(2.67886,-2.0247328)(2.7036786,-2.0277393)
\curveto(2.7284973,-2.030746)(2.793025,-2.0385633)(2.832735,-2.043374)
\curveto(2.872445,-2.0481849)(2.9419367,-2.0566034)(2.9717188,-2.060212)
\curveto(3.0015008,-2.0638204)(3.0561018,-2.0704353)(3.0809205,-2.073442)
\curveto(3.1057386,-2.0764484)(3.155376,-2.0824616)(3.1801941,-2.0854683)
\curveto(3.2050128,-2.0884748)(3.2633746,-2.1056185)(3.2969177,-2.1197553)
\curveto(3.3304615,-2.1338923)(3.3888233,-2.1510358)(3.4136415,-2.1540425)
\curveto(3.43846,-2.1570492)(3.4874964,-2.168027)(3.5117133,-2.1759968)
\curveto(3.5359302,-2.1839674)(3.5794013,-2.1993074)(3.5986547,-2.2066762)
\curveto(3.6179078,-2.2140448)(3.6564147,-2.2287836)(3.6756682,-2.2361524)
\curveto(3.694922,-2.2435212)(3.7389941,-2.2538972)(3.7638123,-2.256904)
\curveto(3.7886305,-2.2599103)(3.837666,-2.270888)(3.8618836,-2.2788582)
\curveto(3.886101,-2.2868288)(3.939499,-2.3033712)(3.96868,-2.3119428)
\curveto(3.9978607,-2.3205147)(4.0518603,-2.332093)(4.0766783,-2.3350997)
\curveto(4.101497,-2.3381062)(4.1610622,-2.345323)(4.195808,-2.349532)
\curveto(4.230554,-2.3537416)(4.300045,-2.3621607)(4.334791,-2.3663704)
\curveto(4.3695374,-2.3705792)(4.429102,-2.3777955)(4.453921,-2.3808022)
\curveto(4.478739,-2.3838086)(4.5283766,-2.3898225)(4.5531945,-2.3928292)
\curveto(4.5780125,-2.3958356)(4.633816,-2.3925228)(4.664801,-2.3862038)
\curveto(4.695786,-2.3798847)(4.752191,-2.3716085)(4.7776113,-2.3696516)
\curveto(4.803031,-2.3676949)(4.8588343,-2.3643818)(4.8892183,-2.3630257)
\curveto(4.919602,-2.36167)(4.974804,-2.363321)(4.999623,-2.3663275)
\curveto(5.024441,-2.3693342)(5.1050634,-2.369029)(5.1608667,-2.365716)
\curveto(5.21667,-2.3624036)(5.318349,-2.3545759)(5.364225,-2.3500605)
\curveto(5.4101014,-2.345545)(5.4764333,-2.3384712)(5.4968896,-2.3359132)
\curveto(5.5173464,-2.3333552)(5.583077,-2.3312452)(5.628351,-2.331694)
\curveto(5.673626,-2.3321419)(5.7393565,-2.3300326)(5.7598133,-2.3274739)
\curveto(5.7802696,-2.3249152)(5.8255444,-2.3253634)(5.850363,-2.3283699)
\curveto(5.875181,-2.3313766)(5.92542,-2.3324263)(5.95084,-2.3304694)
\curveto(5.9762597,-2.3285127)(6.0370264,-2.3258016)(6.072374,-2.325047)
\curveto(6.107721,-2.3242927)(6.1784153,-2.3227847)(6.2137628,-2.32203)
\curveto(6.24911,-2.3212757)(6.3148413,-2.3191657)(6.3452244,-2.3178103)
\curveto(6.375608,-2.3164546)(6.4314117,-2.3131416)(6.4568315,-2.311185)
\curveto(6.4822516,-2.309228)(6.5324903,-2.310278)(6.557308,-2.3132844)
\curveto(6.582127,-2.316291)(6.636728,-2.322906)(6.66651,-2.3265145)
\curveto(6.696293,-2.3301227)(6.7552557,-2.3423023)(6.7844367,-2.3508742)
\curveto(6.8136177,-2.3594458)(6.8769436,-2.3771906)(6.9110875,-2.3863642)
\curveto(6.945232,-2.3955371)(7.018485,-2.4144843)(7.0575933,-2.4242592)
\curveto(7.0967016,-2.4340339)(7.1687517,-2.4629085)(7.2016935,-2.4820087)
\curveto(7.234636,-2.5011091)(7.2961574,-2.5337458)(7.324736,-2.547281)
\curveto(7.353316,-2.560816)(7.4110765,-2.5829237)(7.4402575,-2.5914955)
\curveto(7.4694386,-2.6000674)(7.5432935,-2.614051)(7.5879664,-2.6194637)
\curveto(7.632639,-2.6248755)(7.7219872,-2.6356995)(7.76666,-2.6411116)
\curveto(7.811333,-2.6465242)(7.890752,-2.6561458)(7.925498,-2.6603549)
\curveto(7.960244,-2.6645644)(8.019208,-2.6767445)(8.043426,-2.684715)
\curveto(8.067641,-2.6926851)(8.116677,-2.7036622)(8.141497,-2.7066689)
\curveto(8.166315,-2.7096753)(8.225277,-2.721855)(8.259421,-2.7310286)
\curveto(8.293567,-2.7402015)(8.3618555,-2.7585475)(8.396001,-2.7677205)
\curveto(8.430145,-2.776894)(8.498434,-2.79524)(8.532578,-2.8044136)
\curveto(8.566724,-2.8135865)(8.635013,-2.8319325)(8.669158,-2.8411055)
\curveto(8.703302,-2.850279)(8.777156,-2.8642628)(8.816865,-2.8690736)
\curveto(8.856575,-2.8738844)(8.926067,-2.882303)(8.95585,-2.8859115)
\curveto(8.985632,-2.8895197)(9.040232,-2.8961349)(9.065051,-2.8991413)
\curveto(9.0898695,-2.902148)(9.154397,-2.9099653)(9.194106,-2.914776)
\curveto(9.233816,-2.919587)(9.323164,-2.9304109)(9.3728,-2.9364247)
\curveto(9.422438,-2.9424384)(9.506822,-2.9526606)(9.541568,-2.9568703)
\curveto(9.576313,-2.9610798)(9.635877,-2.9682953)(9.660696,-2.971302)
\curveto(9.685515,-2.9743085)(9.745078,-2.9815254)(9.779824,-2.9857345)
\curveto(9.81457,-2.989944)(9.903917,-3.000768)(9.958518,-3.0073824)
\curveto(10.013119,-3.0139973)(10.112392,-3.0260248)(10.157066,-3.0314374)
\curveto(10.20174,-3.0368502)(10.271232,-3.045268)(10.29605,-3.0482748)
\curveto(10.320869,-3.0512812)(10.380433,-3.0584981)(10.415179,-3.062707)
\curveto(10.449925,-3.066916)(10.52002,-3.0703719)(10.5553665,-3.0696175)
\curveto(10.590713,-3.0688632)(10.642756,-3.0550215)(10.659451,-3.0419343)
\curveto(10.676145,-3.0288472)(10.703369,-3.0119991)(10.734956,-3.0007164)
}
\psdots[dotsize=0.12,dotangle=-6.907633](9.144164,-2.8281405)
\psdots[dotsize=0.12,dotangle=-6.907633](9.411899,-2.7799914)
\psdots[dotsize=0.12,dotangle=-6.907633](9.727171,-2.8383322)
\psdots[dotsize=0.12,dotangle=-6.907633](10.195859,-2.7943819)
\psdots[dotsize=0.12,dotangle=-6.907633](10.484059,-2.9098818)
\psdots[dotsize=0.12,dotangle=-6.907633](10.285511,-2.885828)
\psdots[dotsize=0.12,dotangle=-6.907633](9.908269,-2.8401258)
\psdots[dotsize=0.12,dotangle=-6.907633](9.543666,-2.8563933)
\psdots[dotsize=0.12,dotangle=-6.907633](10.67478,-2.832256)
\psdots[dotsize=0.12,dotangle=-6.907633](10.042442,-2.8966732)
\psdots[dotsize=0.12,dotangle=-6.907633](10.401623,-2.758871)
\psdots[dotsize=0.12,dotangle=-6.907633](10.620638,-2.9465744)
\psline[linewidth=0.04cm](10.82,-3.263)(2.16,-3.203)
\psline[linewidth=0.04cm](2.14,-2.243)(2.16,-3.243)
\pscircle[linewidth=0.04,dimen=inner](1.14,-0.042999998){0.3}
\pscircle[linewidth=0.04,dimen=inner](5.18,-0.563){0.28}
\pscircle[linewidth=0.04,dimen=inner](4.16,-1.483){0.24}
\pscircle[linewidth=0.04,dimen=inner](11.35,-1.473){0.25}
\pscircle[linewidth=0.04,dimen=inner](6.21,-2.813){0.29}
\usefont{T1}{ptm}{m}{n}
\rput(1.0895313,-0.073){1}
\usefont{T1}{ptm}{m}{n}
\rput(5.177969,-0.573){2}
\usefont{T1}{ptm}{m}{n}
\rput(4.1357813,-1.513){3}
\usefont{T1}{ptm}{m}{n}
\rput(11.332188,-1.493){4}
\usefont{T1}{ptm}{m}{n}
\rput(6.170625,-2.813){5}
\pscircle[linewidth=0.04,dimen=inner](0.37,3.867){0.27}
\usefont{T1}{ptm}{m}{n}
\rput(0.32953125,3.867){1}
\pscircle[linewidth=0.04,dimen=inner](0.37,3.207){0.27}
\usefont{T1}{ptm}{m}{n}
\rput(0.33796874,3.207){2}
\pscircle[linewidth=0.04,dimen=inner](0.36,2.537){0.28}
\usefont{T1}{ptm}{m}{n}
\rput(0.33578125,2.527){3}
\pscircle[linewidth=0.04,dimen=inner](0.37,1.887){0.27}
\usefont{T1}{ptm}{m}{n}
\rput(0.3521875,1.867){4}
\pscircle[linewidth=0.04,dimen=inner](0.39,1.247){0.29}
\usefont{T1}{ptm}{m}{n}
\rput(0.370625,1.207){5}
\usefont{T1}{ptm}{m}{n}
\rput(3.3284376,3.847){Gold seam close to surface}
\usefont{T1}{ptm}{m}{n}
\rput(4.3825,3.187){Open cast mining at shallow gold seam}
\usefont{T1}{ptm}{m}{n}
\rput(2.6892188,2.607){Sloping gold seams}
\usefont{T1}{ptm}{m}{n}
\rput(4.364375,1.307){Tunnel from main shaft to access gold}
\usefont{T1}{ptm}{m}{n}
\rput(3.1135938,1.907){Main underground shaft}
\psframe[linewidth=0.04,dimen=inner](0.8,5.157)(0.0,4.397)
\psdots[dotsize=0.092](0.1,5.037)
\psdots[dotsize=0.092](0.34,5.057)
\psdots[dotsize=0.092](0.2,4.837)
\psdots[dotsize=0.092](0.44,4.717)
\psdots[dotsize=0.092](0.4,4.977)
\psdots[dotsize=0.092](0.58,5.077)
\psdots[dotsize=0.092](0.58,4.817)
\psdots[dotsize=0.092](0.66,4.977)
\psdots[dotsize=0.092](0.64,4.637)
\psdots[dotsize=0.092](0.48,4.517)
\psdots[dotsize=0.092](0.66,4.517)
\psdots[dotsize=0.092](0.1,4.677)
\psdots[dotsize=0.092](0.28,4.737)
\psdots[dotsize=0.092](0.24,4.517)
\psdots[dotsize=0.092](0.08,4.477)
\psdots[dotsize=0.092](0.68,4.797)
\psdots[dotsize=0.092](0.4,4.817)
\psdots[dotsize=0.092](0.2,4.977)
\psdots[dotsize=0.092](0.72,5.117)
\psdots[dotsize=0.092](0.32,4.437)
\psdots[dotsize=0.092](0.32,4.637)
\usefont{T1}{ptm}{m}{n}
\rput(2.16875,4.747){Gold deposits}
\end{pspicture}
}
\caption{Diagram showing open cast and shaft mining}
\label{fig:gold mining}
\end{center}
\end{figure}


\end{enumerate}

\subsection{Processing the gold ore}

For every ton of ore that is mined, only a very small amount of gold is extracted. A number of different methods can be used to separate gold from its ore, but one of the more common methods is called \textbf{gold cyanidation}.\\

In the process of gold cyanidation, the ore is crushed and then cyanide (CN$^{-}$) solution is added so that the gold particles are chemically dissolved from the ore. In this step of the process, gold is oxidised. Zinc dust is then added to the cyanide solution. The zinc takes the place of the gold, so that the gold is precipitated out of the solution. This process is shown in figure \ref{fig:gold:processing}.

\begin{figure}[h]
\begin{center}
\begin{pspicture}(-7,-2.5)(7,7)
%\psgrid[gridcolor=lightgray]
\rput(0,6){\textbf{STEP 1} - The ore is crushed until it is in fine pieces}
\psline[linewidth=1pt,arrows=->](0,5.6)(0,4.6)
\rput(0,4){\textbf{STEP 2} - A sodium cyanide (NaCN) solution is mixed with the finely ground rock}
\rput(0,3.5){\rm${4Au + 8NaCN + O_{2} + 2H_{2}O \rightarrow 4NaAu(CN)_{2} + 4NaOH}$}
\rput(0,3){Gold is \textbf{oxidised}.}
\psline[linewidth=1pt,arrows=->](0,2.5)(0,2)
\rput(0,1.5){\textbf{STEP 3} - The gold-bearing solution can then be separated from the}
\rput(0,1){remaining solid ore through filtration}
\psline[linewidth=1pt,arrows=->](0,0.5)(0,-0.5)
\rput(0,-1){\textbf{STEP 4} - Zinc is added. Zinc replaces the gold in the gold-cyanide solution.}
\rput(0,-1.5){The gold is precipitated from the solution.}
\rput(0,-2){This is the \textbf{reduction} part of the reaction.}
\end{pspicture}
\caption{Flow diagram showing how gold is processed}
\label{fig:gold:processing}
\end{center}
\end{figure}

\begin{IFact}{
Another method that is used to process gold is called the 'carbon-in-pulp' (CIP) method. This method makes use of the high affinity that activated carbon has for gold, and there are three stages to the process. The first stage involves the \emph{absorption} of gold in the cyanide solution to carbon. In the \emph{elution} stage, gold is removed from the carbon into an alkaline cyanide solution. In the final stage, \emph{electro-winning} is used to remove gold from the solution through a process of electrolysis. Gold that has been removed is deposited on steel wool electrodes. The carbon is then treated so that it can be re-used.}
\end{IFact}

\subsection{Characteristics and uses of gold}
Gold has a number of uses because of its varied and unique characteristics. Below is a list of some of these characteristics that have made gold such a valuable metal:
\begin{itemize}

\item \emph{Shiny}

Gold's beautiful appearance has made it one of the favourite metals for use in jewelery.

\item \emph{Durable}

Gold does not tarnish or corrode easily, and therefore does not deteriorate in quality. It is sometimes used in dentistry to make the crowns for teeth.

\item \emph{Malleable and ductile}

Since gold can be bent and twisted into shape, as well as being flattened into very thin sheets, it is very useful in fine wires and to produce sheets of gold.

\item \emph{Good conductor}

Gold is a good conductor of electricity and is therefore used in transistors, computer circuits and telephone exchanges.

\item \emph{Heat ray reflector}

Because gold reflects heat very effectively, it is used in space suits and in vehicles. It is also used in the protective outer coating of artificial satellites. One of the more unusual applications of gold is its use in firefighting, where a thin layer of gold is placed in the masks of the firefighters to protect them from the heat.

\end{itemize}

\Activity{Case Study}{Dropping like a gold balloon\\}{
Read the article below, which has been adapted from one that appeared in the \textit{Financial Mail} on 15th April 2005 and then answer the questions that follow.\\

\begin{quote}{
As recently as 1980, South Africa accounted for over 70\% of world gold production. In 2004, that figure was a dismal 14\%. Chamber of Mines figures showed that SA's annual gold production last year slipped to its lowest level since 1931.

Chamber economist Roger Baxter says the 'precipitous' fall in production was caused by the dual impact of the fall in the rand gold price due to the strong rand, and the continued upward rise in costs. Many of these costs, laments Baxter, are 'costs we do not have control over'. These include water, transport, steel and labour costs, which rose by 13\% on average in 2004.

He provides a breakdown of the cost components faced by mines:
\begin{itemize}
\item{Water prices have risen by 10\% per year for the past 3 years}
\item{Steel prices have increased by double-digit rates for each of the past 3 years}
\item{Spoornet's tariffs rose 35\% in 2003 and 16.5\% in 2004}
\item{Labour costs, which make up 50\% of production costs, rose above inflation in 2003 and 2004}
\end{itemize}
}

At these costs, and at current rand gold prices, about 10 mines, employing 90 000 people, are marginal or loss-making, says Baxter.
\end{quote}

\begin{enumerate}
\item{Refer to the table below showing SA's gold production in tons between 1980 and 2004.
\begin{center}
\begin{tabular}{|p{2cm}|p{2cm}|}\hline
\textbf{Year} & \textbf{Production (t)}\\\hline
1980 & 675 \\\hline
1985 & 660 \\\hline
1990 & 600 \\\hline
1995 & 525 \\\hline
2000 & 425 \\\hline
2004 & 340 \\\hline
\end{tabular}
\end{center}

Draw a line graph to illustrate these statistics.}

\item{What percentage did South Africa's gold production contribute towards global production in:
\begin{enumerate}
\item{1980}
\item{2004}
\end{enumerate}
}
\item{Outline two reasons for this drop in gold production.}
\item{Briefly explain how the increased cost of resources such as water contributes towards declining profitability in gold mines.}
\item{Suggest a reason why the cost of \textit{steel} might affect gold production.}
\item{Suggest what impact a decrease in gold production is likely to have on...
\begin{enumerate}
\item{South Africa's economy}
\item{mine employees}
\end{enumerate}
}
\item{Find out what the current price of gold is. Discuss why you think gold is so expensive.}

\end{enumerate}
}

\subsection{Environmental impacts of gold mining}
However, despite the incredible value of gold and its usefulness in a variety of applications, all mining has an environmental cost. The following are just a few of the environmental impacts of gold mining:

\begin{itemize}
\item{\textit{Resource consumption}

Gold mining needs large amounts of electricity and water.
}

\item{\textit{Poisoned water}

Acid from gold processing can leach into nearby water systems such as rivers, causing damage to animals and plants, as well as humans that may rely on that water for drinking. The disposal of other toxic waste (e.g. cyanide) can also have a devastating effect on biodiversity.
}

\item{\textit{Solid waste}

This applies particularly to open pit mines, where large amounts of soil and rock must be displaced in order to access the gold reserves. Processing the gold ore also leaves solid waste behind.
}

\item{\textit{Air pollution}

Dust from open pit mines, as well as harmful gases such as sulfur dioxide and nitrogen dioxide which are released from the furnaces, contribute to air pollution.
}

\item{\textit{Threaten natural areas}

Mining activities often encroach on protected areas and threaten biodiversity in their operation areas.
}
\end{itemize}

\Activity{Discussion}{Mine rehabilitation\\}{

There is a growing emphasis on the need to rehabilitate old mine sites that are no longer in use. If it is too difficult to restore the site to what it was before, then a new type of land use might be decided for that area. Any mine rehabilitation programme should aim to achieve the following:\\

\begin{itemize}
\item{ensure that the site is safe and stable}
\item{remove pollutants that are contaminating the site}
\item{restore the biodiversity that was there before mining started}
\item{restore waterways to what they were before mining\\}
\end{itemize}

There are different ways to achieve these goals. Plants for example, can be used to remove metals from polluted soils and water, and can also help to stabilise the soil so that other vegetation can grow. Land contouring can help to restore drainage in the area.\\

\textbf{Discussion:}\\

\textit{In groups of 3-4, discuss the following questions:}\\

\begin{enumerate}
\item{What are the main benefits of mine rehabilitation?}
\item{What are some of the difficulties that may be experienced in trying to rehabilitate a mine site?}
\item{Suggest some creative ideas that could be used to encourage mining companies to rehabilitate old sites.}
\item{One rehabilitation project that has received a lot of publicity is the rehabilitation of dunes that were mined for titanium by Richards Bay Minerals (RBM). As a group, carry out your own research to try to find out more about this project.}
\begin{itemize}
\item{What actions did RBM take to rehabilitate the dunes?}
\item{Was the project successful?}
\item{What were some of the challenges faced?}
\end{itemize}
\end{enumerate}
}

\Exercise{Gold mining\\}{
Mapungubwe in the Limpopo Province is evidence of gold mining in South Africa as early as 1200. Today, South Africa is a world leader in the technology of gold mining. The following flow diagram illustrates some of the most important steps in the recovery of gold from its ore.

\begin{center}
\begin{pspicture}(-6,-1.4)(6,0.8)
\psframe(-5.8,-1)(-3.3,0.5)
\rput(-4.5,0){Gold}
\rput(-4.5,-0.3){bearing ore}

\psframe(-2.8,-1)(-0.3,0.5)
\rput(-1.5,0){NaAu(CN)$_{2}$}

\psframe(0.2,-1)(2.7,0.5)
\rput(1.5,0){Gold}
\rput(1.5,-0.3){precipitate}

\psframe(3.2,-1)(5.7,0.5)
\rput(4.5,0){Pure gold}

\psline[arrows=->,linewidth=2pt](-3.3,-0.25)(-2.8,-0.25)
\rput(-3,0){\textbf{A}}
\psline[arrows=->,linewidth=2pt](-0.3,-0.25)(0.2,-0.25)
\rput(0,0){\textbf{B}}
\psline[arrows=->,linewidth=2pt](2.7,-0.25)(3.2,-0.25)
\rput(3,0){\textbf{C}}

\end{pspicture}
\end{center}

\begin{enumerate}
\item{Name the process indicated by A.}
\item{During process A, gold is extracted from the ore. Is gold oxidised or reduced during this process?}
\item{Use oxidation numbers to explain your answer to the question above.}
\item{Name the chemical substance that is used in process B.}
\item{During smelting (illustrated by C in the diagram), gold is sent into a calcining furnace. Briefly explain the importance of this process taking place in the furnace.}
\item{The recovery of gold can have a negative impact on water in our country, if not managed properly. State at least one negative influence that the recovery of gold can have on water resources and how it will impact on humans and the environment.}
\end{enumerate}
\practiceinfo

\begin{tabular}[h]{cccccc}
(1.) aaa & (2.) aaa & (3.) aaa & (4.) aaa & (5.) aaa & (6.) aaa & 
 \end{tabular}
}



% CHILD SECTION END



% CHILD SECTION START

\section{Mining and mineral processing: Iron}
\label{sec:mining:iron}

Iron is one of the most abundant metals on Earth. Its concentration is highest in the core, and lower in the crust. It is extracted from \textbf{iron ore} and is almost never found in its elemental form. Iron ores are usually rich in \textbf{iron oxide} minerals and may vary in colour from dark grey to rusty red. Iron is usually found in minerals such as magnetite (Fe$_{3}$O$_{4}$) and hematite (Fe$_{2}$O$_{3}$). Iron ore also contains other elements, which have to be removed in various ways. These include silica (Si), phosphorus (P), aluminium (Al) and sulfur (S).

\subsection{Iron mining and iron ore processing}

One of the more common methods of mining for iron ore is \textbf{open cast mining}. Open cast mining is used when the iron ore is found near the surface. Once the ore has been removed, it needs to be crushed into fine particles before it can be processed further.\\

As mentioned earlier, iron is commonly found in the form of \textbf{iron oxides}. To create pure iron, the ore must be \textbf{smelted} to remove the oxygen.

\Definition{Smelting}{
Smelting is a method used to extract a metal from its ore and then purify it.
}

Smelting usually involves heating the ore and also adding a reducing agent (e.g. carbon) so that the metal can be freed from its ore. The bonds between iron and oxygen are very strong, and therefore it is important to use an element that will form stronger bonds with oxygen that the iron. This is why carbon is used. In fact, carbon monoxide is the main ingredient that is needed to strip oxygen from iron. These reactions take place in a \textbf{blast furnace}.\\

A blast furnace is a huge steel container many metres high and lined with heat-resistant material. In the furnace the solid raw materials, i.e. iron ore, carbon (in the form of 'coke', a type of coal) and a flux (e.g. limestone) are fed into the top of the furnace and a blast of heated air is forced into the furnace from the bottom. Temperatures in a blast furnace can reach 2000$^{\degree}$C. A simple diagram of a blast furnace is shown in figure \ref{fig:blast furnace}. The equations for the reactions that take place are shown in the flow diagram below. \\

\begin{center}

STEP 1: Production of carbon monoxide\\

\rm${C + O_{2} \rightarrow CO_{2}}$

\rm${CO_{2} + C \rightarrow 2CO}$
\end{center}

\begin{center}
STEP 2: Reduction of iron oxides takes place in a number of stages to produce iron.\\

\rm${3Fe_{2}O_{3} + CO \rightarrow 2Fe_{3}O_{4} + CO_{2}}$

\rm${Fe_{3}O_{4} + CO \rightarrow 3FeO + CO_{2}}$

\rm${FeO + CO \rightarrow Fe + CO_{2}}$
\end{center}

\begin{center}
STEP 3: Fluxing

The flux is used to melt impurities in the ore. A common flux is limestone (CaCO$_{3}$). Common impurities are silica, phosphorus (makes steel brittle), aluminium and sulfur (produces SO$_{2}$ gases during smelting and interferes with the smellting process).\\


\rm${CaCO_{3} \rightarrow CaO + CO_{2}}$

\rm${CaO + SiO_{2} \rightarrow CaSiO_{3}}$
\end{center}

In step 3, the calcium carbonate breaks down into calcium oxide and carbon dioxide. The calcium oxide then reacts with silicon dioxide (the impurity) to form a \textbf{slag}. In this case the slag is the CaSiO$_{3}$. The slag melts in the furnace, whereas the silicon dioxide would not have, and floats on the more dense iron. This can then be separated and removed.

\begin{figure}[h]
\begin{center}
% \usepackage{pst-plot} % For axes
\scalebox{1} % Change this value to rescale the drawing.
{
\begin{pspicture}(0,-5.3301563)(12.657051,5.3501563)
\psline[linewidth=0.04cm](7.256426,1.6698438)(6.256426,-2.2901564)
\psline[linewidth=0.04cm](6.276426,-2.2901564)(5.756426,-2.5901563)
\psline[linewidth=0.04cm](6.356426,-2.6101563)(5.856426,-2.8701563)
\pscustom[linewidth=0.04]
{
\newpath
\moveto(6.9164257,-3.8301563)
\lineto(6.853569,-3.8301563)
\curveto(6.8221397,-3.8301563)(6.717378,-3.8301563)(6.644045,-3.8301563)
\curveto(6.5707116,-3.8301563)(6.49214,-3.8301563)(6.4764256,-3.8301563)
}
\pscustom[linewidth=0.04]
{
\newpath
\moveto(6.4564257,-4.1701565)
\lineto(6.536426,-4.1701565)
\curveto(6.576426,-4.1701565)(6.6735687,-4.1701565)(6.730712,-4.1701565)
\curveto(6.787854,-4.1701565)(6.873568,-4.1701565)(6.90214,-4.1701565)
\curveto(6.9307117,-4.1701565)(6.953568,-4.1701565)(6.9364257,-4.1701565)
}
\psline[linewidth=0.04cm](6.9564257,-4.150156)(6.9564257,-5.3101563)
\pscustom[linewidth=0.04]
{
\newpath
\moveto(6.336426,-2.6301563)
\lineto(6.4564257,-2.7601562)
\curveto(6.516426,-2.8251562)(6.631426,-2.9851563)(6.6864257,-3.0801563)
\curveto(6.741426,-3.1751564)(6.816426,-3.3551562)(6.836426,-3.4401562)
\curveto(6.856426,-3.5251563)(6.881426,-3.6601562)(6.8964257,-3.8101563)
}
\psline[linewidth=0.04cm](7.296426,1.6698438)(6.756426,1.9498438)
\psline[linewidth=0.04cm](6.937431,2.2187123)(7.495421,1.9209751)
\pscustom[linewidth=0.04]
{
\newpath
\moveto(7.5002637,1.8958292)
\lineto(7.5439095,1.9770392)
\curveto(7.5657325,2.017644)(7.6290884,2.1171985)(7.670621,2.1761477)
\curveto(7.7121534,2.235097)(7.8138733,2.3602097)(7.8740597,2.426373)
\curveto(7.9342456,2.4925363)(8.046786,2.5896685)(8.09914,2.6206372)
}
\psline[linewidth=0.04cm](9.39914,1.6898438)(10.39914,-2.2701561)
\psline[linewidth=0.04cm](10.37914,-2.2701561)(10.89914,-2.5701563)
\psline[linewidth=0.04cm](10.29914,-2.5901563)(10.79914,-2.8501563)
\pscustom[linewidth=0.04]
{
\newpath
\moveto(9.75914,-4.630156)
\lineto(9.821997,-4.630156)
\curveto(9.853425,-4.630156)(9.958187,-4.630156)(10.031521,-4.630156)
\curveto(10.104854,-4.630156)(10.183425,-4.630156)(10.199141,-4.630156)
}
\pscustom[linewidth=0.04]
{
\newpath
\moveto(10.219141,-4.970156)
\lineto(10.139141,-4.970156)
\curveto(10.099141,-4.970156)(10.001997,-4.970156)(9.944855,-4.970156)
\curveto(9.887712,-4.970156)(9.801997,-4.970156)(9.773426,-4.970156)
\curveto(9.744855,-4.970156)(9.721997,-4.970156)(9.7391405,-4.970156)
}
\psline[linewidth=0.04cm](9.756426,-4.990156)(9.75914,-5.2901564)
\pscustom[linewidth=0.04]
{
\newpath
\moveto(10.31914,-2.6101563)
\lineto(10.199141,-2.832699)
\curveto(10.139141,-2.9439697)(10.02414,-3.2178686)(9.969141,-3.3804955)
\curveto(9.914141,-3.5431225)(9.83914,-3.851258)(9.81914,-3.9967663)
\curveto(9.79914,-4.142275)(9.77414,-4.3733764)(9.75914,-4.630156)
}
\psline[linewidth=0.04cm](9.35914,1.6898438)(9.89914,1.9698437)
\psline[linewidth=0.04cm](9.718135,2.2387123)(9.160145,1.9409751)
\pscustom[linewidth=0.04]
{
\newpath
\moveto(9.155302,1.9158292)
\lineto(9.111656,1.9970392)
\curveto(9.089834,2.0376441)(9.026479,2.1371984)(8.984945,2.1961477)
\curveto(8.943412,2.2550972)(8.841693,2.3802097)(8.781507,2.446373)
\curveto(8.721319,2.5125363)(8.608779,2.6096685)(8.556426,2.6406372)
}
\psline[linewidth=0.04cm](6.9364257,-5.3101563)(9.776426,-5.3101563)
\rput{237.22333}(18.706892,6.0239577){\psarc[linewidth=0.1,arrowsize=0.05291667cm 2.0,arrowlength=1.4,arrowinset=0.4]{<-}(10.996426,-2.0901563){0.64}{0.0}{117.58224}}
\rput{311.693}(3.4347844,3.4790485){\psarc[linewidth=0.1,arrowsize=0.05291667cm 2.0,arrowlength=1.4,arrowinset=0.4]{->}(5.596426,-2.0901563){0.64}{242.41776}{0.0}}
\psline[linewidth=0.1cm,arrowsize=0.05291667cm 2.0,arrowlength=1.4,arrowinset=0.4]{<-}(5.536426,-4.010156)(6.796426,-3.9901562)
\psline[linewidth=0.1cm,arrowsize=0.05291667cm 2.0,arrowlength=1.4,arrowinset=0.4]{<-}(6.0736995,2.4623232)(7.179152,1.8573643)
\psline[linewidth=0.1cm,arrowsize=0.05291667cm 2.0,arrowlength=1.4,arrowinset=0.4]{<-}(10.419152,2.4423232)(9.3137,1.8373643)
\psline[linewidth=0.1cm,arrowsize=0.05291667cm 2.0,arrowlength=1.4,arrowinset=0.4]{<-}(11.056426,-4.8301563)(9.796426,-4.8101563)
\psline[linewidth=0.1cm,arrowsize=0.05291667cm 2.0,arrowlength=1.4,arrowinset=0.4]{<-}(8.3322935,2.4897916)(8.316426,4.2098436)
\psline[linewidth=0.1cm,arrowsize=0.05291667cm 2.0,arrowlength=1.4,arrowinset=0.4]{<-}(8.435275,2.4910934)(8.977577,3.6285942)
\psline[linewidth=0.1cm,arrowsize=0.05291667cm 2.0,arrowlength=1.4,arrowinset=0.4]{<-}(8.217577,2.4910934)(7.675275,3.6285942)
\psline[linewidth=0.04cm](6.9364257,-4.090156)(9.796426,-4.070156)
\psline[linewidth=0.04cm](6.696426,-3.0901563)(10.1164255,-3.0701563)
\pscustom[linewidth=0.04]
{
\newpath
\moveto(6.9164257,-3.3301563)
\lineto(7.096426,-3.2801561)
\curveto(7.1864257,-3.2551563)(7.346426,-3.2301562)(7.4164257,-3.2301562)
}
\pscustom[linewidth=0.04]
{
\newpath
\moveto(7.8964257,-3.2501562)
\lineto(7.716426,-3.3501563)
\curveto(7.6264257,-3.4001563)(7.446426,-3.4751563)(7.356426,-3.5001562)
\curveto(7.266426,-3.5251563)(7.131426,-3.5501564)(6.9964256,-3.5501564)
}
\pscustom[linewidth=0.04]
{
\newpath
\moveto(8.496426,-3.2301562)
\lineto(8.236425,-3.4201562)
\curveto(8.106426,-3.5151563)(7.801426,-3.6451561)(7.6264257,-3.6801562)
\curveto(7.451426,-3.7151563)(7.211426,-3.7651563)(7.016426,-3.8101563)
}
\pscustom[linewidth=0.04]
{
\newpath
\moveto(7.316426,-3.9901562)
\lineto(7.526426,-3.9201562)
\curveto(7.631426,-3.8851562)(7.8764257,-3.8451562)(8.016426,-3.8401563)
\curveto(8.156425,-3.8351562)(8.341426,-3.8351562)(8.476426,-3.8501563)
}
\pscustom[linewidth=0.04]
{
\newpath
\moveto(8.296426,-3.6501563)
\lineto(8.456426,-3.5701563)
\curveto(8.536426,-3.5301561)(8.736425,-3.4601562)(8.856426,-3.4301562)
\curveto(8.976426,-3.4001563)(9.171426,-3.4001563)(9.396426,-3.4901562)
}
\pscustom[linewidth=0.04]
{
\newpath
\moveto(8.996426,-3.7101562)
\lineto(9.266426,-3.6901562)
\curveto(9.401426,-3.6801562)(9.626426,-3.5801563)(9.716426,-3.4901562)
\curveto(9.806426,-3.4001563)(9.906425,-3.3001564)(9.936426,-3.2701561)
}
\pscustom[linewidth=0.04]
{
\newpath
\moveto(8.976426,-3.2501562)
\lineto(9.146426,-3.2501562)
\curveto(9.231426,-3.2501562)(9.381426,-3.2701561)(9.446425,-3.2901564)
\curveto(9.511426,-3.3101563)(9.581426,-3.3351562)(9.596426,-3.3501563)
}
\pscustom[linewidth=0.04]
{
\newpath
\moveto(8.916426,-3.9101562)
\lineto(9.106426,-3.9201562)
\curveto(9.201426,-3.9251564)(9.381426,-3.9151564)(9.466426,-3.9001563)
\curveto(9.551426,-3.8851562)(9.676426,-3.8451562)(9.716426,-3.8201563)
\curveto(9.756426,-3.7951562)(9.806426,-3.7551563)(9.816426,-3.7401562)
}
\pscustom[linewidth=0.04]
{
\newpath
\moveto(8.456426,-3.9701562)
\lineto(8.576426,-3.9301562)
\curveto(8.636426,-3.9101562)(8.731426,-3.8401563)(8.766426,-3.7901564)
\curveto(8.801426,-3.7401562)(8.841426,-3.6901562)(8.856426,-3.6901562)
}
\pscustom[linewidth=0.04]
{
\newpath
\moveto(7.076426,-3.9501562)
\lineto(6.986426,-3.9801562)
\curveto(6.941426,-3.9951563)(6.881426,-4.0201564)(6.836426,-4.050156)
}
\pscustom[linewidth=0.04]
{
\newpath
\moveto(7.076426,-4.4101562)
\lineto(7.216426,-4.3301563)
\curveto(7.286426,-4.2901564)(7.446426,-4.255156)(7.536426,-4.260156)
\curveto(7.6264257,-4.2651563)(7.786426,-4.3101563)(7.856426,-4.3501563)
\curveto(7.926426,-4.3901563)(8.111425,-4.4601564)(8.226426,-4.490156)
\curveto(8.341426,-4.5201564)(8.566426,-4.5001564)(8.676426,-4.450156)
\curveto(8.786426,-4.400156)(8.931426,-4.3151565)(8.966426,-4.280156)
\curveto(9.001426,-4.2451563)(9.101426,-4.215156)(9.166426,-4.220156)
\curveto(9.231426,-4.2251563)(9.3664255,-4.2701564)(9.436426,-4.3101563)
\curveto(9.506426,-4.3501563)(9.596426,-4.425156)(9.6164255,-4.4601564)
\curveto(9.636426,-4.4951563)(9.666426,-4.5451565)(9.676426,-4.5601563)
}
\pscustom[linewidth=0.04]
{
\newpath
\moveto(7.036426,-4.7901564)
\lineto(7.2064257,-4.6701565)
\curveto(7.2914257,-4.610156)(7.4764256,-4.5601563)(7.576426,-4.570156)
\curveto(7.676426,-4.5801563)(7.856426,-4.6651564)(7.9364257,-4.740156)
\curveto(8.016426,-4.8151565)(8.201426,-4.900156)(8.306426,-4.9101562)
\curveto(8.411426,-4.9201565)(8.626426,-4.845156)(8.736425,-4.760156)
\curveto(8.846426,-4.675156)(9.021426,-4.575156)(9.086426,-4.5601563)
\curveto(9.151426,-4.5451565)(9.301426,-4.575156)(9.386426,-4.6201563)
\curveto(9.471426,-4.6651564)(9.581426,-4.755156)(9.606426,-4.800156)
\curveto(9.631426,-4.845156)(9.681426,-4.8951564)(9.756426,-4.9101562)
}
\pscustom[linewidth=0.04]
{
\newpath
\moveto(7.036426,-5.130156)
\lineto(7.1864257,-4.990156)
\curveto(7.261426,-4.9201565)(7.401426,-4.8301563)(7.466426,-4.8101563)
\curveto(7.531426,-4.7901564)(7.6864257,-4.8351564)(7.776426,-4.900156)
\curveto(7.866426,-4.965156)(8.026426,-5.070156)(8.096426,-5.110156)
\curveto(8.166426,-5.150156)(8.326426,-5.1851563)(8.416426,-5.180156)
\curveto(8.506426,-5.175156)(8.676426,-5.1201563)(8.756426,-5.070156)
\curveto(8.836426,-5.0201564)(8.966426,-4.950156)(9.016426,-4.930156)
\curveto(9.066426,-4.9101562)(9.191426,-4.9201565)(9.266426,-4.950156)
\curveto(9.341426,-4.9801564)(9.461426,-5.055156)(9.506426,-5.1001563)
\curveto(9.551426,-5.1451564)(9.621426,-5.195156)(9.696425,-5.2101564)
}
\pscustom[linewidth=0.04]
{
\newpath
\moveto(7.236426,-5.1901565)
\lineto(7.326426,-5.130156)
\curveto(7.3714256,-5.1001563)(7.471426,-5.0801563)(7.526426,-5.090156)
\curveto(7.5814257,-5.1001563)(7.691426,-5.130156)(7.7464256,-5.150156)
\curveto(7.801426,-5.1701565)(7.866426,-5.1901565)(7.8964257,-5.1901565)
}
\pscustom[linewidth=0.04]
{
\newpath
\moveto(8.816426,-5.2501564)
\lineto(8.956426,-5.1901565)
\curveto(9.026426,-5.1601562)(9.136426,-5.1451564)(9.176426,-5.1601562)
\curveto(9.216426,-5.175156)(9.281425,-5.2051563)(9.356426,-5.2501564)
}
\pscustom[linewidth=0.04]
{
\newpath
\moveto(8.016426,-4.2301564)
\lineto(8.106426,-4.260156)
\curveto(8.151426,-4.275156)(8.266426,-4.300156)(8.336426,-4.3101563)
\curveto(8.406425,-4.320156)(8.511426,-4.3101563)(8.546426,-4.2901564)
\curveto(8.581426,-4.2701564)(8.631426,-4.235156)(8.676426,-4.1901565)
}
\pscustom[linewidth=0.04]
{
\newpath
\moveto(9.516426,-4.1701565)
\lineto(9.586426,-4.1901565)
\curveto(9.621426,-4.200156)(9.671426,-4.2301564)(9.716426,-4.2901564)
}
\usefont{T1}{ptm}{m}{n}
\rput(5.184395,1.0598438){iron}
\usefont{T1}{ptm}{m}{n}
\rput(11.707363,-1.7201562){air}
\usefont{T1}{ptm}{m}{n}
\rput(12.189395,-2.0401564){blast}
\usefont{T1}{ptm}{m}{n}
\rput(10.662989,2.7798438){waste}
\usefont{T1}{ptm}{m}{n}
\rput(10.984707,2.4798439){gases}
\usefont{T1}{ptm}{m}{n}
\rput(9.676739,3.8198438){limestone}
\usefont{T1}{ptm}{m}{n}
\rput(7.2643948,3.7598438){coke}
\usefont{T1}{ptm}{m}{n}
\rput(4.9143944,-4.0001564){molten}
\usefont{T1}{ptm}{m}{n}
\rput{91.50117}(-0.66477084,-1.0555471){\rput(0.16423836,-0.84015626){Step 1}}
\usefont{T1}{ptm}{m}{n}
\rput{89.71741}(1.6050956,1.0038431){\rput(0.2806446,1.3198438){Step 2}}
\usefont{T1}{ptm}{m}{n}
\rput(11.534394,-4.5401564){molten}
\usefont{T1}{ptm}{m}{n}
\rput(4.9586134,-4.300156){slag}
\usefont{T1}{ptm}{m}{n}
\rput(11.464395,-4.880156){iron}
\usefont{T1}{ptm}{m}{n}
\rput(4.647363,-2.0201561){air}
\usefont{T1}{ptm}{m}{n}
\rput(4.329395,-2.3201563){blast}
\usefont{T1}{ptm}{m}{n}
\rput(5.484707,2.3398438){gases}
\usefont{T1}{ptm}{m}{n}
\rput(5.662988,2.6998436){waste}
\usefont{T1}{ptm}{m}{n}
\rput(8.244394,5.159844){iron}
\usefont{T1}{ptm}{m}{n}
\rput(8.2943945,4.8398438){oxide}
\usefont{T1}{ptm}{m}{n}
\rput(7.7420506,-1.4001563){carbon}
\usefont{T1}{ptm}{m}{n}
\rput(9.194395,-1.4001563){dioxide}
\usefont{T1}{ptm}{m}{n}
\rput(8.287363,-1.7601563){forms}
\usefont{T1}{ptm}{m}{n}
\rput(8.322051,0.19984375){carbon}
\usefont{T1}{ptm}{m}{n}
\rput(7.756738,-0.26015624){monoxide}
\usefont{T1}{ptm}{m}{n}
\rput(9.267364,-0.26015624){forms}
\usefont{T1}{ptm}{m}{n}
\rput(8.304395,1.5598438){iron}
\usefont{T1}{ptm}{m}{n}
\rput(8.327363,1.1798438){forms}
\usefont{T1}{ptm}{m}{n}
\rput(3.1220508,-1.2001562){carbon}
\usefont{T1}{ptm}{m}{n}
\rput(3.1004884,-1.4401562){+}
\usefont{T1}{ptm}{m}{n}
\rput(3.182051,-1.6201563){oxygen}
\psline[linewidth=0.06cm,arrowsize=0.05291667cm 2.0,arrowlength=1.4,arrowinset=0.4]{<-}(4.776426,-1.3901563)(3.7964258,-1.3701563)
\usefont{T1}{ptm}{m}{n}
\rput(5.442051,-1.1801562){carbon}
\usefont{T1}{ptm}{m}{n}
\rput(5.5143948,-1.5201563){dioxide}
\usefont{T1}{ptm}{m}{n}
\rput(1.6220509,0.15984374){carbon}
\usefont{T1}{ptm}{m}{n}
\rput(3.1543946,0.15984374){dioxide}
\usefont{T1}{ptm}{m}{n}
\rput(2.3620508,-0.40015626){carbon}
\usefont{T1}{ptm}{m}{n}
\rput(2.4004884,-0.08015625){+}
\usefont{T1}{ptm}{m}{n}
\rput(5.342051,0.15984374){carbon}
\usefont{T1}{ptm}{m}{n}
\rput(5.3967385,-0.24015625){monoxide}
\usefont{T1}{ptm}{m}{n}
\rput(2.3820508,2.0198438){carbon}
\usefont{T1}{ptm}{m}{n}
\rput(2.4367383,1.6198437){monoxide}
\usefont{T1}{ptm}{m}{n}
\rput(4.462051,1.5998437){carbon}
\usefont{T1}{ptm}{m}{n}
\rput(5.8943944,1.5998437){dioxide}
\usefont{T1}{ptm}{m}{n}
\rput(5.140488,1.3598437){+}
\psline[linewidth=0.06cm,arrowsize=0.05291667cm 2.0,arrowlength=1.4,arrowinset=0.4]{<-}(4.616426,1.1898438)(3.476426,1.1898438)
\psline[linewidth=0.06cm,arrowsize=0.05291667cm 2.0,arrowlength=1.4,arrowinset=0.4]{<-}(4.4564257,-0.19015625)(3.2364259,-0.19015625)
\psline[linewidth=0.1cm,dotsize=0.07055555cm 2.0]{-**}(0.59642583,1.3098438)(1.6164259,1.3098438)
\psline[linewidth=0.1cm,dotsize=0.07055555cm 2.0]{-**}(6.336426,1.2498437)(7.576426,1.2498437)
\psline[linewidth=0.1cm,dotsize=0.07055555cm 2.0]{-**}(0.47642586,-0.17015626)(1.2164259,-0.17015626)
\psline[linewidth=0.1cm,dotsize=0.07055555cm 2.0]{-**}(0.47642586,-1.5301563)(2.196426,-1.5301563)
\psline[linewidth=0.1cm](0.49642587,-0.13015625)(0.47642586,-1.5901562)
\psline[linewidth=0.1cm,dotsize=0.07055555cm 2.0]{-**}(6.236426,0.04984375)(7.276426,0.04984375)
\psline[linewidth=0.1cm,dotsize=0.07055555cm 2.0]{-**}(6.196426,-1.4701562)(7.116426,-1.4701562)
\usefont{T1}{ptm}{m}{n}
\rput(8.284863,4.4398437){(haematite)}
\psline[linewidth=0.16cm,arrowsize=0.05291667cm 2.0,arrowlength=1.4,arrowinset=0.4]{<-}(8.316426,-0.43015626)(8.296426,-1.1901562)
\psline[linewidth=0.16cm,arrowsize=0.05291667cm 2.0,arrowlength=1.4,arrowinset=0.4]{<-}(8.316426,1.0098437)(8.316426,0.44984376)
\psbezier[linewidth=0.04](8.412824,1.9086435)(8.571933,2.6926615)(9.41556,1.4194393)(9.280028,0.6510441)
\psbezier[linewidth=0.04](7.2402534,-1.6578374)(6.6797194,-2.1944404)(7.328253,-3.2701561)(7.776426,-2.8301563)
\psbezier[linewidth=0.04](7.459749,0.049658842)(7.021253,0.71877825)(7.4146056,1.359148)(7.853102,0.69002867)
\psbezier[linewidth=0.04](8.996426,-2.4101562)(8.996426,-3.2101562)(9.956426,-2.3901563)(9.956426,-1.5901562)
\psbezier[linewidth=0.04](9.396426,-0.5101563)(9.696425,-1.0974618)(9.396074,-1.5101563)(9.061253,-0.95354635)
\psbezier[linewidth=0.04](7.432822,-1.0704511)(6.9199033,-1.2119691)(6.7934175,-0.7855624)(7.3026414,-0.5994883)
\psbezier[linewidth=0.04](8.614286,-2.1141438)(8.896426,-2.7358274)(8.445101,-3.105397)(8.11353,-2.5095587)
\rput{-74.25834}(5.536593,12.852634){\psarc[linewidth=0.04](11.256426,2.7698438){0.5}{349.69516}{233.1301}}
\rput{62.216858}(5.8972445,-2.9184053){\psarc[linewidth=0.04](5.366766,3.427159){0.30801213}{271.97495}{180.0}}
\rput{150.16873}(10.600184,2.8552308){\psarc[linewidth=0.04](4.919816,2.8394094){0.25428185}{272.93204}{180.0}}
\rput{-87.8947}(9.107019,13.307641){\psarc[linewidth=0.04](11.456426,1.9298438){0.14}{279.4623}{180.0}}
\rput{73.14646}(10.4659395,-7.7064753){\psarc[linewidth=0.04](10.426426,3.1998436){0.27}{271.97495}{180.0}}
\rput{15.593642}(1.0204523,-1.3928704){\psarc[linewidth=0.04](5.596426,3.0298438){0.14}{279.4623}{180.0}}
\rput{-119.127205}(14.54583,12.661449){\psarc[linewidth=0.04](10.9925375,2.0575178){0.2929101}{279.4623}{180.0}}
\rput{-14.260295}(-0.56463444,1.6110893){\psarc[linewidth=0.04](6.157355,3.0624402){0.16360788}{279.4623}{180.0}}
\usefont{T1}{ptm}{m}{n}
\rput(2.4404883,1.2998438){+}
\usefont{T1}{ptm}{m}{n}
\rput(1.9443946,1.0798438){iron}
\usefont{T1}{ptm}{m}{n}
\rput(3.0143945,1.0798438){oxide}
\end{pspicture}
}
\caption{A blast furnace, showing the reactions that take place to produce iron}
\label{fig:blast furnace}
\end{center}
\end{figure}


\subsection{Types of iron}

Iron is the most used of all the metals. Its combination of low cost and high strength make it very important in applications such as industry, automobiles, the hulls of large ships and in the structural components of buildings. Some of the different forms that iron can take include:

\begin{itemize}

\item{\textbf{Pig iron} is raw iron and is the direct product when iron ore and coke are smelted. It has between 4\% and 5\% carbon and contains varying amounts of contaminants such as sulfur, silicon and phosphorus. Pig iron is an intermediate step between iron ore, cast iron and steel.}

\item{\textbf{Wrought iron} is commercially pure iron and contains less than 0.2\% carbon. It is tough, malleable and ductile. Wrought iron does not rust quickly when it is used outdoors. It has mostly been replaced by mild steel for 'wrought iron' gates and blacksmithing. Mild steel does not have the same corrosion resistance as true wrought iron, but is cheaper and more widely available.}

\item{\textbf{Steel} is an alloy made mostly of iron, but also containing a small amount of carbon. Elements other than carbon can also be used to make alloy steels. These include manganese and tungsten. By varying the amounts of the alloy elements in the steel, the following characteristics can be altered: hardness, elasticity, ductility and tensile strength.
}

\item{\textbf{Corrugated iron} is actually sheets of galvanised steel that have been rolled to give them a corrugated pattern. Corrugated iron is a common building material.}
\end{itemize}

One problem with iron and steel is that pure iron and most of its alloys rust. These products therefore need to be protected from water and oxygen, and this is done through painting, galvanisation and plastic coating.

\begin{IFact}{
Iron is also a very important element in all living organisms. One important role that iron plays is that it is a component of the protein \textbf{haemoglobin} which is the protein in blood. It is the iron in the haemoglobin that helps to attract and hold oxygen so that this important gas can be transported around the body in the blood, to where it is needed.
}
\end{IFact}

\subsection{Iron in South Africa}

The primary steel industry is an important part of the South African economy and it generates a great deal of foreign exchange.

\begin{itemize}
\item{About 40 million tons of iron ore is mined annually in South Africa. Approximately 15 million tons are consumed locally, and the remaining 25 million tons are exported.}
\item{South Africa is ranked about 20th in the world in terms of its crude steel production.}
\item{South Africa is the largest steel producer in Africa.}
\item{South Africa exports crude steel and finished steel products, and a lot is also used locally.}
\item{Some of the products that are manufactured in South Africa include reinforcing bars, railway track material, wire rod, plates and steel coils and sheets.}
\end{itemize}

\Exercise{Iron\\}{
Iron is usually extracted from heamatite (iron(III)oxide). Iron ore is mixed with limestone and coke in a blast furnace to produce the metal. The following incomplete word equations describe the extraction process:\\

\begin{itemize} \item[A] coke + oxygen $\rightarrow$ gasX\\ \item[B] gasX + coke $\rightarrow$ gasY\\ \item[C] iron(III)oxide + gasY $\rightarrow$ iron + gasX \\
\end{itemize} \begin{enumerate} \item Name the gases X and Y.
\item Write a balanced chemical equation for reaction C.
\item What is the function of gas Y in reaction C?
\item Why is limestone added to the reaction mixture?
\item Briefly describe the impact that the mining of iron has on the economy and the environment in our country.
\end{enumerate}

(DoE Exemplar Paper, Grade 11, 2007)

\practiceinfo

\begin{tabular}[h]{cccccc}
(1.) aaa & (2.) aaa & (3.) aaa & (4.) aaa & (5.) aaa & 
 \end{tabular}
}



% CHILD SECTION END



% CHILD SECTION START

\section{Mining and mineral processing: Phosphates}

A phosphate is a salt of \textbf{phosphoric acid} (H$_{3}$PO$_{4}$). Phosphates are the naturally occurring form of the element phosphorus. Phosphorus is seldom found in its pure elemental form, and \textbf{phosphate} therefore refers to a rock or ore that contains phosphate ions. The chemical formula for the phosphate ion is PO$_{4}^{3-}$.

\subsection{Mining phosphates}

Phosphate is found in beds in sedimentary rock, and has to be quarried to access the ore. A quarry is a type of open pit mine that is used to extract ore. In South Africa, the main phosphate producer is at the Palaborwa alkaline igneous complex, which produces about 3 million tons of ore per year. The ore is crushed into a powder and is then treated with sulfuric acid to form a superphosphate (Ca(H$_{2}$PO$_{4}$)$_{2}$), which is then used as a fertilizer. In the equation below, the phosphate mineral is calcium phosphate (Ca$_{3}$(PO$_{4}$)$_{2}$.

\begin{center}
\rm${Ca_{10}(PO_{4})_{6}F_{2} + 7H_{2}SO_{4} + 3H_{2}O \rightarrow 3Ca(H_{2}PO_{4})_{2}H_{2}O + 7CaSO_{4}}$
\end{center}

Alternatively, the ore can be treated with concentrated phosphoric acid (which forms a triple superphosphate), in which case the reaction looks like this:

\begin{center}
\rm${Ca_{10}(PO_{4})_{6}F_{2} + 14H_{3}PO_{4} + 10H_{2}O \rightarrow 10Ca(H_{2}PO_{4})_{2}H_{2}O + 2HF}$
\end{center}

\subsection{Uses of phosphates}

Phosphates are mostly used in \textbf{agriculture}. Phosphates are one of the three main nutrients needed by plants, and they are therefore an important component of \textbf{fertilisers} to stimulate plant growth.

\begin{IFact}{Exploring the lithosphere for minerals is not a random process! \textbf{Geologists} help to piece together a picture of what past environments might have been like, so that predictions can be made about where minerals might have a high concentration. \textbf{Geophysicists} measure gravity, magnetics and the electrical properties of rocks to try to make similar predictions, while \textbf{geochemists} sample the soils at the earth's surface to get an idea of what lies beneath them. You can see now what an important role scientists play in mineral exploration!
}
\end{IFact}

\Exercise{Phosphates}{
Rock phosphate [Ca$_{10}$(PO$_4$)$_6$F$_2$], mined from open pit mines at Phalaborwa, is an important raw material in the production of fertilisers. The following two reactions are used to transform rock phosphate into water soluble phosphates:

\begin{itemize}
\item[A:] Ca$_{10}$(PO$_4$)$_6$F$_2$ + 7X + 3H$_2$O $\rightarrow$ 3Ca(H$_2$PO$_4$)$_2$H$_2$O + 2HF + 7CaSO$_4$\\
\item[B:] Ca$_{10}$(PO$_4$)$_6$F$_2$ + 14Y + 10H$_2$O $\rightarrow$ 10Ca(H$_2$PO$_4$)$_2$H$_2$O + 2HF\\
\end{itemize}

\begin{enumerate}
\item Identify the acids represented by X and Y.
\item Despite similar molecular formulae, the products Ca(H$_2$PO$_4$)$_2$ formed in the two reactions have different common names. Write down the names for each of these products in reactions A and B.
\item Refer to the products in reactions A and B and write down TWO advantages of reaction B over reaction A.
\item Why is rock phosphate unsuitable as fertiliser?
\item State ONE advantage and ONE disadvantage of phosphate mining.
\end{enumerate}

(DoE Exemplar Paper, Grade 11, 2007)
\practiceinfo

\begin{tabular}[h]{cccccc}
(1.) aaa & (2.) aaa & (3.) aaa & (4.) aaa & (5.) aaa & 
 \end{tabular}
}

\Activity{Case Study}{Controversy on the Wild Coast - Titanium mining\\}{
\textit{Read the extract below, which has been adapted from an article that appeared in the} Mail and Guardian \textit{on 4th May 2007, and then answer the questions that follow.\\}

\begin{quote}{A potentially violent backlash looms in Pondoland over efforts by an Australian company to persuade villagers to back controversial plans to mine an environmentally sensitive strip of the Wild Coast. The mining will take place in the Xolobeni dunes, south of Port Edward. The application has outraged environmental groups, largely because the proposed mining areas form part of the Pondoland centre of endemism, which has more species than the United Kingdom, some of which are endemic and facing extinction.\\

Exploratory drilling revealed Xolobeni has the world's 10th largest titanium deposit, worth about R11 billion. The amount of money that will be spent over the mine's 22 years, including a smelter, is estimated at R1.4 billion. The Australian mining company predicts that 570 direct jobs will be created.\\

But at least two communities fiercely oppose the mining plans. Some opponents are former miners who fear Gauteng's mine dumps and compounds will be replicated on the Wild Coast. Others are employees of failed ecotourism ventures, who blame the mining company for their situation. Many are suspicious of outsiders. The villagers have also complained that some of the structures within the mining company are controlled by business leaders with political connections, who are in it for their own gain. Intimidation of people who oppose the mining has also been alleged. Headman Mandoda Ndovela was shot dead after his outspoken criticism of the mining.\\

Mzamo Dlamini, a youth living in one of the villages that will be affected by the mining, said 10\% of the Amadiba 'who were promised riches by the mining company' support mining. 'The rest say if people bring those machines, we will fight.'
}

\end{quote}

\begin{enumerate}
\item{Explain what the following words means:}
\begin{enumerate}
\item{endemic}
\item{smelter}
\item{ecotourism}
\end{enumerate}
\item{What kinds of 'riches' do you think the Amadiba people have been promised by the mining company?}
\item{In two columns, list the potential \textbf{advantages} and \textbf{disadvantages} of mining in this area.}
\item{Imagine that you were one of the villagers in this area. Write down \textit{three questions} that you would want the mining company to answer before you made a decision about whether to oppose the mining or not. Share your ideas with the rest of the class.}
\item{Imagine that you are an environmentalist. What would your main concerns be about the proposed mining project? Again share your answers with the rest of the class.}
\end{enumerate}

}



% CHILD SECTION END



% CHILD SECTION START

\section{Energy resources and their uses: Coal}
\label{sec:mining:energy}

The products of the lithosphere are also important in meeting our \textbf{energy needs}. Coal is one product that is used to produce energy. In South Africa, coal is particularly important because most of our electricity is generated using coal as a fuel.  South Africa is the world's sixth largest coal producer, with Mpumalanga contributing about 83\% of our total production. Other areas in which coal is produced, include the Free State, Limpopo and KwaZulu-Natal. One of the problems with coal however, is that it is a \textbf{non-renewable resource}, meaning that once all resources have been used up, it cannot simply be produced again. Burning coal also produces large quantities of greenhouse gases, which may play a role in global warming. At present, ESKOM, the South African government's electric power producer, is the coal industry's main customer.\\


\subsection{The formation of coal}

Coal is what is known as a \textbf{fossil fuel}. A fossil fuel is a \textit{hydrocarbon} that has been formed from organic material such as the remains of plants and animals. When plants and animals decompose, they leave behind organic remains that accumulate and become compacted over millions of years under sedimentary rock. Over time, the \textit{heat} and \textit{pressure} in these parts of the earth's crust also increases, and coal is formed. When coal is burned, a large amount of heat energy is released, which is used to produce electricity. \textbf{Oil} is also a fossil fuel and is formed in a similar way.

\Definition{Fossil Fuel}{A fossil fuel is a hydrocarbon that is formed from the fossilised remains of dead plants and animals that have been under conditions of intense heat and pressure for millions of years.}

\subsection{How coal is removed from the ground}

Coal can be removed from the crust in a number of different ways. The most common methods used are \textit{strip mining}, \textit{open cast mining} and \textit{underground mining}.

\begin{enumerate}
\item{\textbf{Strip mining}

Strip mining is a form of surface mining that is used when the coal reserves are very shallow. The \textit{overburden} (overlying sediment) is removed so that the coal seams can be reached. These sediments are replaced once the mining is finished, and in many cases, attempts are made to \textit{rehabilitate} the area.
}

\item{\textbf{Open cast mining}

Open cast mining is also a form of surface mining, but here the coal deposits are too deep to be reached using strip mining. One of the environmental impacts of open cast mining is that the overburden is dumped somewhere else away from the mine, and this leaves a huge pit in the ground.
}

\item{\textbf{Underground mining}

Undergound mining is normally used when the coal seams are amuch deeper, usually at a depth greater than 40 m. As with shaft mining for gold, the problem with underground mining is that it is very dangerous, and there is a very real chance that the ground could collapse during the mining if it is not supported. One way to limit the danger is to use \textit{pillar support} methods, where some of the ground is left unmined so that it forms pillars to support the roof. All the other surfaces underground will be mined. Using another method called \textit{longwalling}, the roof is allowed to collapse as the mined-out area moves along. In South Africa, only a small percentage of coal is mined in this way.
}
\end{enumerate}

\subsection{The uses of coal}

Although in South Africa, the main use of coal is to produce electricity, it can also be used for other purposes.

\begin{enumerate}
\item{\textbf{Electricity}

In order to generate electricity, solid coal must be crushed and then burned in a furnace with a boiler. A lot of steam is produced and this is used to spin turbines which then generate electricity.
}

\item{\textbf{Gasification}

If coal is broken down and subjected to very high temperatures and pressures, it forms a \textit{synthesis gas}, which is a mix of carbon dioxide and hydrogen gases. This is very important in the \textit{chemical industry} (this will be discussed in Grade 12).
}

\item{\textbf{Liquid fuels}

Coal can also be changed into liquid fuels like petrol and diesel using the \textbf{Fischer-Tropsch process}. In fact, South Africa is one of the leaders in this technology (refer to Grade 12). The only problem is that producing liquid fuels from coal, rather than refining petroleum that has been drilled, releases much greater amounts of carbon dioxide into the atmosphere, and this contributes further towards global warming.
}
\end{enumerate}

\subsection{Coal and the South African economy}

In South Africa, the coal industry is second only to the gold industry. More than this, South Africa is one of the world's top coal \textit{exporters}, and also one of the top \textit{producers}. Of the coal that is produced, most is used locally to produce electricity and the rest is used in industry and domestically.\\

The problem with coal however, is that it is a \textbf{non-renewable resource} which means that once all the coal deposits have been mined, there will be none left. Coal takes such a long time to form, and requires such specific environmental conditions, that it would be impossible for coal to re-form at a rate that would keep up with humankind's current consumption. It is therefore very important that South Africa, and other countries that rely on coal, start to look for alternative energy resources.

\subsection{The environmental impacts of coal mining}

There are a number of environmental impacts associated with coal mining.

\begin{itemize}
\item{\textit{Visual impact and landscape scars}

Coal mining leaves some very visible scars on the landscape, and destroys biodiversity (e.g. plants, animals). During strip mining and open cast mining, the visual impact is particularly bad, although this is partly reduced by rehabilitation in some cases.
}

\item{\textit{Spontaneous combustion and atmospheric pollution}

Coal that is left in mine dumps may spontaneously combust, producing large amounts of sulfurous smoke which contributes towards atmospheric pollution.}

\item{\textit{Acid formation}

Waste products from coal mining have a high concentration of sulfur compounds. When these compounds are exposed to water and oxygen, sulfuric acid is formed. If this acid washes into nearby water systems, it can cause a lot of damage to the ecosystem. Acid can also leach into soils and alter its acidity. This in turn affects what will be able to grow there.
}

\item{\textit{Global warming}

As was discussed earlier, burning coal to generate electricity produces \textit{carbon dioxide} and \textit{nitrogen oxides} which contribute towards global warming (refer to chapter \ref{chap:atmosphere}). Another gas that causes problems is \textit{methane}. All coal contains methane, and deeper coal contains the most methane. As a greenhouse gas, methane is about twenty times more potent than carbon dioxide.
}
\end{itemize}

\begin{IFact}{It is easy to see how mining, and many other activities including industry and vehicle transport, contribute towards Global Warming. It was for this reason that South Africa joined the \textbf{Carbon Sequestration Leadership Forum (CSLF)}. The forum is an international climate change initiative that focuses on developing cost effective technologies to separate and capture carbon dioxide from the atmosphere so that it can be stored in some way. The CSLF also aims to make these technologies as widely available as possible.
}
\end{IFact}

\Exercise{Coal in South Africa\\}{
The following advertisement appeared in a local paper:\\

\begin{center}
\scalebox{0.5} % Change this value to rescale the drawing.
{
\begin{pspicture}(0,-7.0)(22.06,7.0)
\psframe[linewidth=0.04,dimen=outer](22.06,7.0)(0.0,-7.0)
\usefont{T1}{ptm}{m}{n}
\rput(10.87625,5.6){\Huge ENERGY STARTS WITH SOUTH AFRICAN COAL}
\usefont{T1}{ptm}{m}{n}
\rput(10.9245,-5.76){\Huge Coal SOUTH AFRICA}
\psframe[linewidth=0.04,dimen=outer](7.02,4.38)(2.02,4.2)
\psframe[linewidth=0.04,dimen=outer](3.2,4.2)(3.0,-4.62)
\psframe[linewidth=0.04,dimen=outer](6.0,4.2)(5.8,-4.62)
\psline[linewidth=0.04cm](3.18,3.9)(5.82,1.18)
\psline[linewidth=0.04cm](3.2,3.68)(5.8,0.98)
\psline[linewidth=0.04cm](5.8,3.88)(4.5,2.58)
\psline[linewidth=0.04cm](5.8,3.68)(4.6,2.48)
\psline[linewidth=0.04cm](4.4,2.48)(3.2,1.18)
\psline[linewidth=0.04cm](4.5,2.38)(3.2,0.98)
\psellipse[linewidth=0.04,dimen=outer](2.32,4.06)(0.2,0.06)
\psellipse[linewidth=0.04,dimen=outer](2.32,3.94)(0.2,0.06)
\psellipse[linewidth=0.04,dimen=outer](2.32,3.82)(0.2,0.06)
\psellipse[linewidth=0.04,dimen=outer](2.32,3.7)(0.2,0.06)
\psellipse[linewidth=0.04,dimen=outer](2.32,3.58)(0.2,0.06)
\psframe[linewidth=0.04,dimen=outer](2.38,4.22)(2.28,4.1)
\psframe[linewidth=0.04,dimen=outer](2.38,3.54)(2.28,3.42)
\psellipse[linewidth=0.04,dimen=outer](4.44,4.06)(0.2,0.06)
\psellipse[linewidth=0.04,dimen=outer](4.44,3.94)(0.2,0.06)
\psellipse[linewidth=0.04,dimen=outer](4.44,3.82)(0.2,0.06)
\psellipse[linewidth=0.04,dimen=outer](4.44,3.7)(0.2,0.06)
\psellipse[linewidth=0.04,dimen=outer](4.44,3.58)(0.2,0.06)
\psframe[linewidth=0.04,dimen=outer](4.5,4.22)(4.4,4.1)
\psframe[linewidth=0.04,dimen=outer](4.5,3.54)(4.4,3.42)
\psellipse[linewidth=0.04,dimen=outer](6.68,4.06)(0.2,0.06)
\psellipse[linewidth=0.04,dimen=outer](6.68,3.94)(0.2,0.06)
\psellipse[linewidth=0.04,dimen=outer](6.68,3.82)(0.2,0.06)
\psellipse[linewidth=0.04,dimen=outer](6.68,3.7)(0.2,0.06)
\psellipse[linewidth=0.04,dimen=outer](6.68,3.58)(0.2,0.06)
\psframe[linewidth=0.04,dimen=outer](6.74,4.22)(6.64,4.1)
\psframe[linewidth=0.04,dimen=outer](6.74,3.54)(6.64,3.42)
\psline[linewidth=0.03cm](2.3,3.38)(1.2,2.98)
\psline[linewidth=0.03cm](4.5,3.38)(1.2,2.28)
\psline[linewidth=0.03cm](6.7,3.38)(1.2,1.58)
\psline[linewidth=0.03cm](6.8,3.44)(10.22,3.92)
\psline[linewidth=0.03cm](4.5,3.38)(5.4,3.48)
\psline[linewidth=0.03cm](9.06,3.94)(6.9,3.68)
\psline[linewidth=0.03cm](6.5,3.6)(6.0,3.54)
\psline[linewidth=0.03cm](5.66,3.48)(5.76,3.5)
\psline[linewidth=0.03cm](2.42,3.4)(3.0,3.48)
\psline[linewidth=0.03cm](3.14,3.48)(3.3,3.52)
\psline[linewidth=0.03cm](3.48,3.58)(4.28,3.64)
\psline[linewidth=0.03cm](4.64,3.7)(5.7,3.82)
\psline[linewidth=0.03cm](6.0,3.84)(6.5,3.88)
\psline[linewidth=0.03cm](6.86,3.88)(7.46,3.94)
\psdots[dotsize=0.12](1.84,3.18)
\psdots[dotsize=0.12](1.78,3.1)
\psdots[dotsize=0.12](1.92,3.18)
\psdots[dotsize=0.12](4.0,3.2)
\psdots[dotsize=0.12](4.1,3.18)
\psdots[dotsize=0.12](3.96,3.16)
\psdots[dotsize=0.12](3.96,3.12)
\psdots[dotsize=0.12](6.06,3.16)
\psdots[dotsize=0.12](6.14,3.12)
\psdots[dotsize=0.12](5.98,3.08)
\psdots[dotsize=0.12](2.98,3.48)
\psdots[dotsize=0.12](5.16,3.46)
\psdots[dotsize=0.12](5.1,3.4)
\psdots[dotsize=0.12](5.22,3.46)
\psdots[dotsize=0.12](7.38,3.52)
\psdots[dotsize=0.12](7.3,3.46)
\psdots[dotsize=0.12](7.48,3.5)
\usefont{T1}{ptm}{m}{n}
\rput{24.10856}(1.724999,-4.530826){\rput(11.415065,1.725853){\LARGE No additives}}
\usefont{T1}{ptm}{m}{n}
\rput{24.10856}(1.5106086,-4.678201){\rput(11.652934,1.2187845){\LARGE No preservatives}}
\usefont{T1}{ptm}{m}{n}
\rput{24.10856}(1.3008013,-4.8148274){\rput(11.867928,0.64469594){\LARGE No artificial colouring}}
\usefont{T1}{ptm}{m}{n}
\rput{24.10856}(1.1123062,-4.979809){\rput(12.159968,0.0649246){\LARGE Best before 2300}}
\usefont{T1}{ptm}{m}{n}
\rput(14.189375,-2.86){\LARGE Coal is as old as the hills and}
\usefont{T1}{ptm}{m}{n}
\rput(12.3359375,-3.46){\LARGE just as natural.}
\usefont{T1}{ptm}{m}{n}
\rput(13.995313,-4.28){\LARGE It's like juice from real fruit.}
\psbezier[linewidth=0.04](17.02,3.6741111)(17.42,3.774111)(18.82,3.774111)(19.22,3.6741111)(19.62,3.5741112)(19.62,1.4741111)(19.22,1.374111)(18.82,1.2741112)(17.42,1.2741112)(17.02,1.374111)(16.62,1.4741111)(16.62,3.5741112)(17.02,3.6741111)
\psbezier[linewidth=0.04](17.12,1.374111)(17.12,0.97411114)(17.32,-1.625889)(17.32,-1.625889)(17.32,-1.625889)(18.92,-1.625889)(18.92,-1.625889)(18.92,-1.625889)(19.12,0.97411114)(19.12,1.374111)
\psframe[linewidth=0.04,dimen=outer](19.12,-1.625889)(17.12,-1.9258889)
\psframe[linewidth=0.04,dimen=outer](18.76,4.634111)(18.62,3.734111)
\psline[linewidth=0.04cm](19.5,3.1341112)(19.64,3.054111)
\psline[linewidth=0.04cm](19.64,3.054111)(19.52,2.714111)
\psline[linewidth=0.04cm](19.56,2.794111)(19.6,2.5941112)
\psline[linewidth=0.04cm](19.6,2.5941112)(19.54,2.474111)
\psdots[dotsize=0.06](19.6,2.5941112)
\psdots[dotsize=0.06](19.64,3.054111)
\psdots[dotsize=0.06](19.5,3.1341112)
\psdots[dotsize=0.06](19.54,2.474111)
\psbezier[linewidth=0.04,doubleline=true,doublesep=0.08](18.68,3.754111)(18.68,4.374111)(19.240385,4.781829)(19.58,4.594111)(19.919617,4.406393)(20.08,3.514111)(20.02,2.954111)(19.96,2.3941112)(19.5,0.6341111)(19.5,-0.4058889)(19.5,-1.4458889)(20.34,-1.6058888)(20.42,-0.76588887)(20.5,0.07411111)(20.46,0.5541111)(20.42,0.8941111)(20.38,1.2341111)(20.26,2.5941112)(19.64,2.9141111)
\psbezier[linewidth=0.04](19.659061,2.9155788)(19.616848,2.7784832)(19.548103,2.8015883)(19.587189,2.9208252)(19.626274,3.040062)(19.701275,3.0526748)(19.659061,2.9155788)
\psline[linewidth=0.02cm](19.92,2.754111)(20.02,2.5941112)
\psline[linewidth=0.02cm](19.9,2.6141112)(20.0,2.434111)
\psbezier[linewidth=0.04](17.22,3.3741112)(17.42,3.474111)(18.82,3.474111)(19.02,3.3741112)(19.22,3.274111)(19.24,1.7741112)(19.02,1.6741111)(18.8,1.5741111)(17.42,1.5741111)(17.22,1.6741111)(17.02,1.7741112)(17.02,3.274111)(17.22,3.3741112)
\psframe[linewidth=0.04,dimen=outer](18.72,3.0741112)(18.12,2.9141111)
\psframe[linewidth=0.04,dimen=outer](18.72,2.8741112)(18.12,2.714111)
\psframe[linewidth=0.04,dimen=outer](18.72,2.6741111)(18.12,2.514111)
\psdots[dotsize=0.12](18.22,2.994111)
\psdots[dotsize=0.12](18.32,2.994111)
\psdots[dotsize=0.12](18.2,2.794111)
\psdots[dotsize=0.12](18.3,2.794111)
\psdots[dotsize=0.12](18.4,2.794111)
\psdots[dotsize=0.12](18.2,2.5941112)
\psdots[dotsize=0.12](18.3,2.5941112)
\end{pspicture}
}
\end{center}

\begin{enumerate}
\item{\textit{"Coal is as old as the hills, and just as natural."} Is this statement TRUE? Motivate your answer by referring to how coal is formed.}
\item{Coal is a non-renewable energy source. Quote a statement from the advertisement that gives an indication that coal is non-renewable. Give a reason for your choice.}
\item{Is coal actually a healthy source of energy? Motivate your answer by referring to all influences that coal and coal mining have on both humans and the environment.}
\item{Why is coal used as a primary energy source in South Africa?}
\end{enumerate}

(DoE Exemplar Paper 2, Grade 11, 2007)
\practiceinfo

\begin{tabular}[h]{cccccc}
(1.) aaa & (2.) aaa & (3.) aaa & (4.) aaa & 
 \end{tabular}
}



% CHILD SECTION END



% CHILD SECTION START

\section{Energy resources and their uses: Oil}
\label{sec:oil}

\textbf{Oil} is another product of the lithosphere which is very important in meeting our fuel needs.

\subsection{How oil is formed}

Oil is formed in a very similar way to coal, except that the organic material is laid down in \textbf{oceans}. Organisms such as zooplankton and algae settle to the ocean floor and become buried under layers of mud. Over time, as these layers of sediment accumulate and the heat and pressure also increase, the organic material changes to a waxy material called \textit{kerogen}. Eventually, with continuing heat and pressure, liquid and gas \textbf{hydrocarbons} are formed. These hydrocarbons are lighter than rock and therefore move upwards through the rock layers before being trapped by an impermeable layer. Here the oil will slowly accumulate until there is enough that it can be accessed by oil rigs and other equipment. \textbf{Crude oil} or \textbf{petroleum}, is actually a mixture of hydrocarbons (mostly alkanes) of different lengths, ranging from 5 carbons to 18 carbons in the hydrocarbon chain. If the mixture contains mostly short hydrocarbons, then it is a gas called \textbf{natural gas}. As the hydrocarbon chains in the mixture become longer, the product becomes more and more solid. Coal is made up of the longest hydrocarbons. For more information on hydrocarbons, refer to Grade 12.

\subsection{Extracting oil}

When enough oil has accumulated in a well, it becomes economically viable to try to extract it either through \textbf{drilling} or \textbf{pumping}. If the pressure in the oil reservoir is high, the oil is forced naturally to the surface. This is known as \textit{primary recovery} of oil. If the pressure is low, then pumps must be used to extract it. This is known as \textit{secondary recovery}. When the oil is very difficult to extract, steam injection into the reservoir can be used to heat the oil, reduce its viscosity and make it easier to extract.\\

While most of South Africa's oil is imported and then processed at a refinery in either Durban, Cape Town or Sasolburg, some is extracted from coal, as discussed in section \ref{sec:mining:energy}.


\subsection{Other oil products}

Oil can also be used to make a variety of different products. You will find more information on this in Grade 12.

\begin{itemize}
\item{\textit{Fractional distillation}

Fractional distillation is the separation of a mixture into the parts that make it up. In oil refineries, crude oil is separated into useful products such as asphalt, diesel, fuel oil, gasoline, kerosine, liquid petroleum gas (LPG) and tar, to name just a few.
}

\item{\textit{Cracking}

There are two types of cracking, \textit{steam cracking} and \textit{hydrocracking}. Cracking is used to change heavy hydrocarbons such as petroleum into lighter hydrocarbons such as fuels (LPG and gasoline), plastics (ethylene) and other products that are needed to make fuel gas (propylene).
}
\end{itemize}

\subsection{The environmental impacts of oil extraction and use}

Some of the key environmental impacts associated with the extraction and use of oil are as follows:

\begin{itemize}
\item{\textit{Pollution}

Exploring the oceans for oil, and the actual drilling process, can result in major pollution.}

\item{\textit{Ecosystem impacts}

Dredging the ocean floors for oil can disrupt seabed ecosystems.
}

\item{\textit{Global warming}

Burning oil as a fuel source produces carbon dioxide, which contributes towards global warming.
}
\end{itemize}


% CHILD SECTION END



% CHILD SECTION START

\section{Alternative energy resources}
\label{sec:mining:alternative energy}

As the world's population increases, so does the demand for energy. As we have already mentioned, many of our energy resources are \textbf{non-renewable} and will soon run out. In addition, many of the fuels that we use produce large amounts of greenhouse gases, which can contribute towards global warming. If we are to maintain the quality and health of our planet, and also meet our growing need for energy, we will need to investigate alternative energy resources. In this next section, we are going to take a closer look at some of these possible alternatives. Many of these options are very controversial, and may have both pros and cons.

\begin{itemize}

\item{\textit{Nuclear power}}

Another element that is found in the crust, and which helps to meet our energy needs, is \textbf{uranium}. Uranium produces energy through the process of \textit{nuclear fission} (chapter \ref{chap:an}). Neutrons are aimed at the nucleii of the uranium atoms in order to split them. When the nucleus of a uranium atom is split, a large amount of energy is released as heat. This heat is used to produce steam, which turns turbines to generate electricity. Uranium is produced as a by-product of gold in some mines in the Witwatersrand, and as a by-product in some copper mines, for example in Palaborwa. Many people regard this type of nuclear power as relatively environmentally friendly because it doesn't produce a lot of greenhouse gases. However, generating nuclear power does produce radioactive wastes, which must be carefully disposed of in order to prevent contamination. There are also concerns around leaking of nuclear materials.

\item{\textit{Natural gas}

Natural gas is formed in a similar way to oil and is often located above oil deposits in the earth's crust. 'Natural gas' refers to a hydrocarbon gas, composed mostly of methane. It is highly combustible and produces low emissions.\\

In June 2002, construction began on a pipeline that would stretch for 865 km between Mozambique and South Africa. Mozambique has large sources of under-utilised natural gas and so an agreement was reached between SASOL and the South African and Mozambican governments to build the pipeline, which would transport natural gas from Mozambique to South Africa. The benefits of natural gas include the fact that it is a clean-burning fossil fuel and few by-products are emitted as pollutants. It is also an economical and efficient energy source as the gas can easily be piped directly to a customer's facility.}

\item{\textit{Biofuels}

In many parts of the world, ethanol is currently being used as a substitute for crude petroleum. Ethanol can be produced through the fermentation of sugar-containing products such as sugar cane. One of the problems with this however, is the vast areas of land that are needed to cultivate the necessary crops. Crops such as maize can also be used in the process.
In South Africa, a company called 'Ethanol Africa' has been set up by commercial farmers to convert their surplus maize into environmentally-friendly biofuel, and plans are underway to establish ethanol plants in some of the maize-producing provinces.}

\item{\textit{Hydropower}

Hydropower produces energy from the action of falling water. As water falls from a height, the energy is used to turn turbines which produce electricity. However, for hydropower to be effective, a large dam is needed to store water. The building of a dam comes with its own set of problems such as the expense of construction, as well as the social and environmental impacts of relocating people (if the area is populated),and disrupting a natural river course.}

\item{\textit{Solar energy}

Solar energy is energy from the sun. The sun's radiation is trapped in solar panels and is then converted into electricity. While this process is environmentally friendly, and solar energy is a renewable resource, the supply of radiation is not constant (think for example of cloudy days, and nights), and the production of electricity is not efficient. Solar energy can however meet small energy needs such as the direct heating of homes.}

\item{\textit{Geothermal energy}

This type of energy comes from the natural heat below the Earth's surface. If hot underground steam can be tapped and brought to the surface, it has the potential to produce electricity.}
\end{itemize}

\Activity{Discussion}{Using energy wisely\\}{
The massive power cuts or 'load shedding' that South Africans began to experience at the beginning of 2008, were a dramatic wake-up call to the growing energy crisis that the country faces. \\

There are alternative energy sources available, but they will take years to become functional, and many of them have their own problems. Another way to look at the problem, is to put the focus on reducing how much energy is \textit{used} rather than focusing only on ways to meet the growing demand.\\

\begin{enumerate}
\item{In your groups, discuss ways that each of the following groups of people could save energy.}
\begin{enumerate}
\item{industries}
\item{domestic users}
\item{farmers}
\end{enumerate}
\item{Discuss creative incentives that could be used to encourage each of these groups to reduce their energy consumption.}
\end{enumerate}
}

\summary{aaa}

\begin{itemize}
\item{The \textbf{lithosphere} is the solid, outermost part of our planet and contains many important metal elements such as gold and iron, as well as products that are needed to produce energy.}
\item{These elements are seldom found in their pure form, but rather as \textbf{minerals} in \textbf{rocks}.}
\item{A \textbf{mineral} is formed through geological processes. It can either be a pure element (e.g. gold) or may consist of a number of different elements e.g. the gold-bearing mineral calaverite (AuTe$_{2}$).}
\item{A \textbf{rock} is an aggregate of a number of different minerals.}
\item{An \textbf{ore} is rock that contains minerals which make it valuable for mining.}
\item{Minerals have been used throughout \textbf{history}. As new metals and minerals were discovered, important growth took place in industry, agriculture and technology.}
\item{\textbf{Gold} is one of the most important metals in the history of South Africa. It was the discovery of gold that led to an influx of fortune-seeking foreigners, and a growth in mining villages and towns.}
\item{Most of South Africa's gold is concentrated in the 'Golden Arc' in the area between Johannesburg and Welkom.}
\item{Three methods are used to obtain gold from the lithosphere: \textbf{panning}, \textbf{open cast mining} and \textbf{shaft mining}.}
\item{Gold ore must be processed so that the metal can be removed. One way to process the ore after it has been crushed is a method called \textbf{gold cyanidation}. A cyanide solution is added to the crushed ore so that a gold-cyanide solution is formed. Zinc is then added to this solution so that the gold is precipitated out.}
\item{Gold has a number of important \textbf{characteristics} which make it a useful metal for jewelery and other applications. The metal is shiny, durable, malleable, ductile, is a good conductor of electricity and is also a good heat reflector.}
\item{Gold mining has a number of \textbf{environmental impacts}, which include resource consumption, air pollution, poisoned water, solid waste and the destruction of biodiversity in natural areas.}
\item{\textbf{Mine rehabilitation} is one way of restoring old mine sites to what they were like before.}
\item{\textbf{Iron} is another important metal and is used in industry, furniture and building materials.}
\item{Iron is usually found in minerals such as \textbf{iron oxides}. These minerals must be processed to remove the metal.}
\item{When iron ore is \textbf{processed}, a blast furnace is used. The iron ore, carbon and a flux are added to the top of the furnace and hot air is blasted into the bottom of the furnace. A number of reactions occur in the furnace to finally remove the iron from its ore. Iron oxides are reduced by carbon monoxide to produce iron.}
\item{Iron can occur in a number of forms, depending on its level of purity and carbon content. It can also occur in an \textbf{alloy} e.g. steel.}
\item{\textbf{Phosphates} are found in sedimentary rock, which must be quarried to access the ore.}
\item{Phosphates react with phosphoric acid or sulfuric acid to produce a \textbf{superphosphate} (Ca(H$_{2}$PO$_{4}$)$_{2}$), which is an important component in \textbf{fertilisers}.}
\item{The products of the lithosphere are also important in meeting \textbf{energy needs}. \textbf{Coal} and \textbf{oil} can be extracted from the lithosphere for this purpose.}
\item{Coal and oil are both \textbf{fossil fuels}. A fossil fuel is a hydrocarbon that has been formed from the fossilsed remains of plants and animals that have been under conditions of high heat and pressure over a long period of time.}
\item{Coal and oil are \textbf{non-renewable resources}, meaning that once they have been used up, no more can be produced.}
\item{Coal can be removed from the ground using \textbf{strip mining}, \textbf{open cast mining} or \textbf{underground mining}. }
\item{Coal is burned to produce energy, which can be used to generate \textbf{electricity}. Coal can also be used to produce \textbf{liquid fuels} or a \textbf{syngas} which can be converted into other useful products for the chemical industry.}
\item{Some of the \textbf{environmental impacts} associated with coal mining include landscape scars, spontaneous combustion, acid formation and global warming.}
\item{Oil is also a fossil fuel but it forms in the \textbf{oceans}. It can extracted using either \textbf{pumping} or \textbf{drilling}, depending on the pressure of the oil.}
\item{\textbf{Fractional distillation} of oil can be used to make products such as diesel, gasoline and liquid petroleum gas.}
\item{\textbf{Cracking} can be used to convert heavy hydrocarbons to light hydrocarbons.}
\item{The \textbf{environmental impacts} of oil extraction and use are similar to those for coal.}
\item{In view of the number of environmental impacts associated with the extraction and use of coal and oil, other \textbf{alternative energy sources} should be considered. These include nuclear power, biofuels, hydropower and a number of others. All of these alternatives have their own advantages and disadvantages.}
\end{itemize}

\begin{eocexercises}{}
\begin{enumerate}
\item{Give one word to describe each of the following phrases:}
\begin{enumerate}
\item{earth's crust together with the upper layer of the mantle}
\item{a mineral containing silica and oxygen}
\item{an alloy of iron and tin}
\item{a manual technique used to sort gold from other sediments}
\end{enumerate}

\item{For each of the following questions, choose the \textit{one correct answer} from the list provided.}

\begin{enumerate}
\item{One of the main reasons that South Africa's gold industry has been so economically viable is that...}
\begin{enumerate}
\item{gold panning can be used as an additional method to extract gold}
\item{open cast mining can be used to extract gold reserves}
\item{South Africa's geological history is such that its gold reserves are concentrated in large reefs}
\item{South Africa has large amounts of water to use in mining}
\end{enumerate}

\item{The complete list of reactants in an iron blast furnace is...}
\begin{enumerate}
\item{carbon and oxygen}
\item{coal, oxygen, iron ore and limestone}
\item{carbon, oxygen and iron ore}
\item{coal, air, iron ore and slag}
\end{enumerate}

\end{enumerate}

\item{\begin{quote}{
\textbf{More profits, more poisons}\\

In the last three decades, gold miners have made use of \textit{cyanidation} to recover gold from the ore. Over 99\% of gold from ore can be extracted in this way. It allows miners to obtain gold flakes - too small for the eye to see. Gold can also be extracted from the waste of old operations which sometimes leave as much as a third of the gold behind.\\

The left-over cyanide can be re-used, but is more often stored in a pond behind a dam or even dumped directly into a local river. A teaspoonful of 2\% solution of cyanide can kill a human adult.\\

Mining companies insist that cyanide breaks down when exposed to sunlight and oxygen which render it harmless. They also point to scientific studies that show that cyanide swallowed by fish will not 'bio-accumulate', which means it does not pose a risk to anyone who eats the fish. In practice, cyanide solution that seeps into the ground will not break down because of the absence of sunlight. If the cyanide solution is very acidic, it could turn into cyanide gas, which is toxic to fish. On the other hand, if the solution is alkaline the cyanide does not break down.\\

There are no reported cases of human death from cyanide spills. If you don't see corpses, everything is okay.
}
\end{quote}
}
\begin{enumerate}
\item{What is \textit{cyanidation}?}
\item{What type of chemical reaction takes place during this process: precipitation, acid-base or redox?}
\item{Is the pH of the solution after cyanidation greater than, less than or equal to 7?}
\item{How is solid gold recovered from this solution?}
\item{Refer to cyanidation and discuss the meaning of the heading of this extract: \textit{More profits, more poisons}.}

(DoE Grade 11 Paper 2, 2007)
\end{enumerate}
\end{enumerate}

\practiceinfo

\begin{tabular}[h]{cccccc}
(1.) aaa & (2.) aaa & (3.) aaa & 
 \end{tabular}
\end{eocexercises}



% CHILD SECTION END



% CHILD SECTION END



% CHILD SECTION START

