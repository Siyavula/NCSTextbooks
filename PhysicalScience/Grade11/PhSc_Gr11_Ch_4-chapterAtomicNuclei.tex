\chapter{Atomic Nuclei}
\label{chap:an}
Veritasium video on radiation: http://cnx.org/content/m38946/latest/#radiation
\textbf{Nuclear physics} is the branch of physics which deals with the \textbf{nucleus} of the atom. Within this field, some scientists focus their attention on looking at the \textit{particles} inside the nucleus and understanding how they interact, while others classify and interpret the \textit{properties} of nuclei. This detailed knowledge of the nucleus makes it possible for \textit{technological advances} to be made. In this next chapter, we are going to touch on each of these different areas within the field of nuclear physics.


% CHILD SECTION START

\section{Nuclear structure and stability}
\label{sec:an:ns}

You will remember from an earlier chapter that an atom is made up of different types of particles: protons (positive charge) neutrons (neutral) and electrons (negative charge). The nucleus is the part of the atom that contains the protons and the neutrons, while the electrons are found in energy orbitals around the nucleus. The protons and neutrons together are called \textbf{nucleons}. It is the nucleus that makes up most of an atom's \textit{atomic mass}, because an electron has a very small mass when compared with a proton or a neutron.\\

Within the nucleus, there are different forces which act between the particles. The \textbf{strong nuclear force} is the force between two or more nucleons, and this force binds protons and neutrons together inside the nucleus. The \textbf{electromagnetic force} causes the repulsion between like-charged (positive) protons. In a way then, these forces are trying to produce opposite effects in the nucleus. The strong nuclear force acts to hold all the protons and neutrons close together, while the electromagnetic force acts to push protons further apart. In atoms where the nuclei are small, the strong nuclear force overpowers the electromagnetic force. However, as the nucleus gets bigger (in elements with a higher number of nucleons), the electromagnetic force becomes greater than the strong nuclear force. In these nuclei, it becomes possible for particles and energy to be ejected from the nucleus. These nuclei are called \textbf{unstable}. The particles and energy that a nucleus releases are referred to as \textbf{radiation}, and the atom is said to be \textbf{radioactive}. We are going to look at these concepts in more detail in the next few sections.



% CHILD SECTION END



% CHILD SECTION START

\section{The Discovery of Radiation}
\label{sec:an:td}

Radioactivity was first discovered in 1896 by a French scientist called Henri Becquerel while he was working on phosphorescent materials. He wrapped a photgraphic plate in black paper and placed various phosphorescent substances on it. When he used uranium salts he noticed that the film blackened even if it was kept in a dark room. He eventually concluded that some rays must be coming out of the uranium crystals to produce this effect and that these rays were able to pass through the paper. \\

His observations were taken further by the Polish scientist Marie Curie and her husband Pierre, who increased our knowledge of radioactive elements. In 1903, Henri, Marie and Pierre were awarded the Nobel Prize in Physics for their work on radioactive elements. This award made Marie the first woman ever to receive a Nobel Prize. Marie Curie and her husband went on to discover two new elements, which they named \textbf{polonium} (Po) after Marie's home country, and \textbf{radium} (Ra) after its highly radioactive characteristics. For these dicoveries, Marie was awarded a Nobel Prize in Chemistry in 1911, making her one of very few people to receive two Nobel Prizes. \\

\begin{IFact}
{Marie Curie died in 1934 from aplastic anemia, which was almost certainly partly due to her massive exposure to radiation during her lifetime. Most of her work was carried out in a shed without safety measures, and she was known to carry test tubes full of radioactive isotopes in her pockets and to store them in her desk drawers. By the end of her life, not only was she very ill, but her hands had become badly deformed due to their constant exposure to radiation. Unfortunately it was only later in her life that the full dangers of radiation were understood. In fact, because of their high levels of radioactivity, her papers from the 1890's are considered too dangerous to handle. Even her cookbook is highly radioactive. These documents are kept in lead-lined boxes, and those who wish to consult them must wear protective clothing. }
\end{IFact}


% CHILD SECTION END



% CHILD SECTION START

\section{Radioactivity and Types of Radiation}
\label{sec:an:r}

In section \ref{sec:an:ns}, we discussed that when a nucleus is unstable it can emit particles and energy. This process is called \textbf{radioactive decay}.

\Definition{Radioactive decay}{Radioactive decay is the process in which an unstable atomic nucleus loses energy by emitting particles or electromagnetic waves. These emitted particles or electromagnetic waves are called \textbf{radiation}.
}

When a nucleus undergoes radioactive decay, it emits radiation and the nucleus is said to be radioactive. We are exposed to small amounts of radiation all the time. Even the rocks around us emit radiation! However some elements are far more radioactive than others. Even within a single element, there may be some isotopes that are more radioactive than others simply because they contain a larger number of neutrons. These radioactive isotopes are called \textbf{radioisotopes}.\\

Radiation can be emitted in different forms. There are three main types of radiation: alpha, beta and gamma
radiation. These are shown in figure \ref{fig:atomicnuclei:radiationtypes}, and are described below.\\

\begin{figure}[!h]
\begin{center}
\begin{pspicture}(-1,0)(8,3)
%\psgrid[gridcolor=gray]

\def\water{\psset{unit=0.25}
\pscircle(0,0){2}
\rput{150}{\psarc[fillcolor=white,fillstyle=solid](-1.5,1){1.5}{30}{260}
\psarc[fillcolor=white,fillstyle=solid](1.5,1){1.5}{280}{150}
\rput(-1.5,1){\pscurve(1.5;30)(-1;142.5)(1.5;260)}
\rput(1.5,1){\pscurve(1.5;150)(-1;37.5)(1.5;280)}}\psset{unit=1}}

\rput(2,1){\pspolygon(0,0)(0,2)(1,1.5)(1,-0.5)\uput[d](0.5,-0.5){paper}}

\rput(4,1){\pspolygon[fillstyle=solid,fillcolor=lightgray](0,0)(0,2)(1,1.5)(1,-0.5)\pspolygon[fillstyle=solid,fillcolor=lightgray](1,-0.5)(1.2,-0.5)(1.2,1.5)(1,1.5)\pspolygon[fillstyle=solid,fillcolor=lightgray](1.2,1.5)(1,1.5)(0,2)(0.2,2)\uput[d](0.5,-0.5){aluminium}}

\rput(6,1){\pspolygon[fillstyle=solid,fillcolor=gray](0,0)(0,2)(1,1.5)(1,-0.5)\pspolygon[fillstyle=solid,fillcolor=gray](1,-0.5)(1.8,-0.5)(1.8,1.5)(1,1.5)\pspolygon[fillstyle=solid,fillcolor=gray](1.8,1.5)(1,1.5)(0,2)(0.8,2)\uput[d](0.5,-0.5){lead}}

\rput(0,2.6){\uput[l](0,0){alpha ($\alpha$)}\psline(0,0)(2.6,0)}

\rput(0,2){\uput[l](0,0){beta ($\beta$)}\psline(0,0)(4.6,0)}

\rput(0,1.4){\uput[l](0,0){gamma ($\gamma$)}\psplot[xunit=0.01111,plotpoints=300]{0}{585}{x 10 mul sin 0.05 mul}}

\end{pspicture}
\caption{Types of radiation}
\label{fig:atomicnuclei:radiationtypes}
\end{center}
\end{figure}


\subsection{Alpha ($\alpha$) particles and alpha decay}

An alpha particle is made up of two protons and two neutrons bound together. This type of radiation has a \textit{positive charge}. An alpha particle is sometimes represented using the chemical symbol $He^{2+}$, because it has the same structure as a Helium atom (two neutrons and two protons) ,but without the two electrons to balance the positive charge of the protons, hence the overall charge of +2. Alpha particles have a relatively low penetration power. Penetration power describes how easily the particles can pass through another material. Because alpha particles have a \textit{low} penetration power, it means that even something as thin as a piece of paper, or the outside layer of the human skin, will absorb these particles so that they can't penetrate any further.\\

Alpha decay occurs in nuclei that contain too many protons, which results in strong repulsion forces between these positively charged particles. As a result of these repulsive forces, the nucleus emits an $\alpha$ particle. This can be seen in the decay of Americium (Am) to Neptunium (Np).\\

Example:

\begin{center}
$\rm{^{241}_{95}Am \rightarrow ^{237}_{93}Np + \alpha particle}$
\end{center}

Let's take a closer look at what has happened during this reaction. Americium (Z = 95; A = 241) undergoes $\alpha$ decay and releases one alpha particle (i.e. 2 protons and 2 neutrons). The atom now has only 93 protons (Z = 93). On the periodic table, the element which has 93 protons (Z = 93) is called Neptunium. Therefore, the Americium atom has become a Neptunium atom. The atomic mass of the neptunium atom is 237 (A = 237) because 4 nucleons (2 protons and 2 neutrons) were emitted from the atom of Americium.

Phet simulation on alpha decay: SIYAVULA-SIMULATION:http://cnx.org/content/m38928/latest/#id63458
\subsection{Beta ($\beta$) particles and beta decay}
In nuclear physics, $\beta$ decay is a type of radioactive decay in which a $\beta$ particle (an electron or a positron) is emitted. In the case of electron emission, it is referred to as beta minus ($\beta$-), while in the case of a positron emission as beta plus ($\beta$+).

An electron and positron have identical physical characteristics except for opposite charge.

In certain types of radioactive nuclei that have too many neutrons, a neutron may be converted into a proton, an electron and another particle called a neutrino. The high energy electrons that are released in this way are the $\beta$ - particles. This process can occur for an isolated neutron.

In $\beta$+ decay, energy is used to convert a proton into a neutron(n), a
positron (e+) and a neutrino ($\nu$e):

\begin{equation*}
\mbox{energy} + p \rightarrow n + e^+ + {\nu}e
\end{equation*}

So, unlike $\beta$-, $\beta$+ decay cannot occur in isolation, because it requires energy, the mass of the neutron being greater than the mass of the
proton. $\beta$+ decay can only happen inside nuclei when the value of  the binding energy of the mother nucleus is less than that of the daughter
nucleus. The difference between these energies goes into the reaction of
converting a proton into a neutron, a positron and a neutrino and into
the kinetic energy of these particles.
\\

The diagram below shows what happens during $\beta$ decay:

\begin{figure}[!h]
\begin{pspicture}(-8,-5)(9,3)
%\psgrid[gridcolor=lightgray]
\psellipse(-3,0)(1.5,1.5)
\psellipse(3,0)(1.5,1.5)

\psellipse(-3.5,0)(0.4,0.4)
\psellipse(-3.1,0.6)(0.4,0.4)
\psellipse*(-2.7,0)(0.4,0.4)

\psellipse(3.5,0)(0.4,0.4)
\psellipse*(3.1,0.6)(0.4,0.4)
\psellipse*(2.7,0)(0.4,0.4)

\psline[arrows=->](-0.8,0)(0.8,0)

\psline[arrows=->,linestyle=dashed,dash=3pt 2pt](5,0.5)(6.5,1.5)
\psline[arrows=->,linestyle=dashed,dash=3pt 2pt](5,-0.5)(6.5,-1.5)

\rput(7.5,-1.9){neutrino $(\bar{\nu})$}
\rput(7.5,1.9){electron ($\beta$ particle)}
\rput(-3,-2.3){\textbf{Hydrogen-3}}
\rput(3,-2.3){\textbf{Helium-3}}

\psellipse*(-4,-3.5)(0.4,0.4)
\rput(-2,-3.5){= one proton}
\psellipse(-4,-4.5)(0.4,0.4)
\rput(-2,-4.5){= one neutron}

\psline(-4.5,0)(-5.5,0)
\rput(-7,0){Atomic nucleus}
\rput(-0.2,1.8){One of the neutrons from H-3}
\rput(-0.2,1.4){is converted to}
\rput(-0.2,1){a proton}
\rput(7,-3){An electron and a}
\rput(7,-3.4){neutrino are released}
\end{pspicture}
\caption{$\beta$ decay in a hydrogen atom}
\label{fig:beta decay}
\end{figure}

During beta decay, the number of neutrons in the atom decreases by one, and the number of protons increases by one. Since the number of protons before and after the decay is different, the atom has changed into a different element. In figure \ref{fig:beta decay}, Hydrogen has become Helium. The beta decay of the Hydrogen-3 atom can be represented as follows:\\

\begin{center}
$\rm{^{3}_{1}H \rightarrow ^{3}_{2}He + \beta particle + \bar{\nu}}$
\end{center}

\begin{IFact}{
When scientists added up all the energy from the neutrons, protons and electrons involved in $\beta$-decays, they noticed that there was always some energy missing. We know that energy is always conserved, which led Wolfgang Pauli in 1930 to come up with the idea that another particle, which was not detected yet, also had to be involved in the decay. He called this particle the neutrino (Italian for "little neutral one"), because he knew it had to be neutral, have little or no mass, and interact only very weakly, making it very hard to find experimentally! The neutrino was finally identified experimentally about 25 years after Pauli first thought of it.\\

Due to the radioactive processes inside the sun, each 1 $cm^{2}$ patch of the earth receives 70 billion (70$\times 10^{9}$) neutrinos each second! Luckily neutrinos only interact very weakly so they do not harm our bodies when billions of them pass through us every second.
}
\end{IFact}
Phet simulation on beta decay: SIYAVULA-SIMULATION:http://cnx.org/content/m38928/latest/#id6345
\subsection{Gamma ($\gamma$) rays and gamma decay}

When particles inside the nucleus collide during radioactive decay, energy is released. This energy can leave the nucleus in the form of waves of electromagnetic energy called gamma rays. Gamma radiation is part of the electromagnetic spectrum, just like visible light. However, unlike visible light, humans cannot see gamma rays because they are at a much higher frequency and a higher energy. Gamma radiation has no mass or charge. This type of radiation is able to penetrate most common substances, including metals. Only substance with high atomic masses (like lead) and high densities (like concrete or granite) are effective at absorbing gamma rays.\\

Gamma decay occurs if the nucleus is at a very high an energy state. Since gamma rays are part of the electromagnetic spectrum, they can be thought of as waves \emph{or} particles. Therefore in gamma decay, we can think of a ray or a particle (called a photon) being released. The atomic number and atomic mass remain unchanged.

\begin{figure}[!h]
\begin{pspicture}(-5,-3)(9,3)
%\psgrid[gridcolor=lightgray]
\psellipse(-3,0)(1.5,1.5)
\psellipse(3,0)(1.5,1.5)

\psellipse*(-3.5,0)(0.4,0.4)
\psellipse*(-3.1,0.6)(0.4,0.4)
\psellipse(-2.7,0)(0.4,0.4)

\psellipse(3.5,0)(0.4,0.4)
\psellipse*(3.1,0.6)(0.4,0.4)
\psellipse*(2.7,0)(0.4,0.4)

\psline[arrows=->](-0.8,0)(0.8,0)

\psline[arrows=->,linestyle=dashed,dash=3pt 2pt](5,0.5)(6.5,1.5)

\rput(7.5,1.9){photon ($\gamma$ particle)}
\rput(-3,-2.3){\textbf{Helium-3}}
\rput(3,-2.3){\textbf{Helium-3}}
\end{pspicture}
\caption{$\gamma$ decay in a helium atom}
\end{figure}

Table \ref{tab:radiation type summary} summarises and compares the three types of radioactive decay that have been discussed.

\begin{table}[h]
\begin{center}
\caption{A comparison of alpha, beta and gamma decay}
\label{tab:radiation type summary}
\begin{tabular}{|l|l|p{2cm}|p{2cm}|}\hline
\textbf{Type of decay} & \textbf{Particle/ray released} & \textbf{Change in element} & \textbf{Penetration power} \\\hline
Alpha ($\alpha$) & $\alpha$ particle (2 protons and 2 neutrons) & Yes & Low \\\hline
Beta ($\beta$) & $\beta$ particle (electron)  &  Yes & Medium \\\hline
Gamma ($\gamma$) & $\gamma$ ray (electromagnetic energy) & No & High \\\hline
\end{tabular}
\end{center}
\end{table}
Khan Academy video on types of decay: SIYAVULA-VIDEO:http://cnx.org/content/m38928/latest/#circuits-1
\begin{wex}{Radioactive decay\\}{The isotope $^{241}_{95}$Am undergoes radioactive decay and loses two alpha particles.

\begin{enumerate}
\item{Write the chemical formula of the element that is produced as a result of the decay.}
\item{Write an equation for this decay process.\\}
\end{enumerate}
}

{\westep{Work out the number of protons and/or neutrons that the radioisotope loses during radioactive decay}

One $\alpha$ particle consists of two protons and two neutrons. Since two $\alpha$ particles are released, the total number of protons lost is four and the total number of neutrons lost is also four.\\
}

{\westep{Calculate the atomic number (Z) and atomic mass number (A) of the element that is formed.}

\begin{equation*}
Z = 95 - 4
= 91
\end{equation*}

\begin{equation*}
A = 241 - 8
= 233\\
\end{equation*}
}

{\westep{Refer to the periodic table to see which element has the atomic number that you have calculated. }
The element that has Z = 91 is Protactinium (Pa).\\
}

{\westep{Write the symbol for the element that has formed as a result of radioactive decay.}
\begin{center}
$^{233}_{91}$Pa \\
\end{center}
}

{\westep{Write an equation for the decay process.}
\begin{center}
\rm${^{241}_{95}Am \rightarrow ^{233}_{91}Pa}$ + 4 protons + 4 neutrons
\end{center}
}
\end{wex}

\Activity{Discussion}{Radiation\\}{
In groups of 3-4, discuss the following questions:\\

\begin{itemize}
\item{Which of the three types of radiation is most dangerous to living creatures (including humans!)}
\item{What can happen to people if they are exposed to high levels of radiation?}
\item{What can be done to protect yourself from radiation (Hint: Think of what the radiologist does when you go for an X-ray)?}
\end{itemize}
}

\Exercise{Radiation and radioactive elements\\}{
\begin{enumerate}
\item{There are two main forces inside an atomic nucleus:
\begin{enumerate}
\item{Name these two forces.}
\item{Explain why atoms that contain a greater number of nucleons are more likely to be radioactive.}
\end{enumerate}
}
\item{The isotope $^{241}_{95}$Am undergoes radioactive decay and loses three alpha particles.
\begin{enumerate}
\item{Write the chemical formula of the element that is produced as a result of the decay.}
\item{How many nucleons does this element contain?}
\end{enumerate}
}

\item{Complete the following equation:
\begin{center}
\rm${^{210}_{82}Pb \rightarrow}$ (alpha decay)
\end{center}
}

\item{Radium-228 decays by emitting a beta particle. Write an equation for this decay process.}

\item{Describe how gamma decay differs from alpha and beta decay.}

\end{enumerate}
\practiceinfo

\begin{tabular}[h]{cccccc}
(1.) aaa & (2.) aaa & (3.) aaa & (4.) aaa & (5.) aaa & 
 \end{tabular}
}


% CHILD SECTION END



% CHILD SECTION START

\section{Sources of radiation}
\label{sec:an:sources}

The sources of radiation can be either \textbf{natural} or \textbf{man-made}.

\subsection{Natural background radiation}
\begin{itemize}
\item{\textit{Cosmic radiation}

The Earth, and all living things on it, are constantly bombarded by radiation from space. Charged particles from the sun and stars interact with the Earth's atmosphere and magnetic field to produce a shower of radiation, which is mostly beta and gamma radiation. The amount of cosmic radiation varies in different parts of the world because of differences in elevation and also the effects of the Earth's magnetic field.
}
\item{\textit{Terrestrial Radiation}

Radioactive material is found throughout nature. It occurs naturally in the soil, water, and vegetation. The major isotopes that are of concern are uranium and the decay products of uranium, such as thorium, radium, and radon. Low levels of uranium, thorium, and their decay products are found everywhere. Some of these materials are ingested (taken in) with food and water, while others are breathed in. The dose of radiation from terrestrial sources varies in different parts of the world.
}
\end{itemize}


\begin{IFact}{
Cosmic and terrestrial radiation are not the only natural sources. All people have radioactive potassium-40, carbon-14, lead-210 and other isotopes inside their bodies from birth.
}
\end{IFact}

\subsection{Man-made sources of radiation}

Although all living things are exposed to natural background radiation, there are other sources of radiation. Some of these will affect most members of the public, while others will only affect those people who are exposed to radiation through their work.

\begin{itemize}
\item{\textit{Members of the Public}

Man-made radiation sources that affect members of the public include televisions, tobacco (polonium-210), combustible fuels, smoke detectors (americium), luminous watches (tritium) and building materials. By far, the most significant source of man-made radiation exposure to the public is from medical procedures, such as diagnostic x-rays, nuclear medicine, and radiation therapy. Some of the major isotopes involved are I-131, Tc-99m, Co-60, Ir-192, and Cs-137. The production of nuclear fuel using uranium is also a source of radiation for the public, as is fallout from nuclear weapons testing or use.
}
\item{\textit{Individuals who are exposed through their work}

Any people who work in the following environments are exposed to radiation at some time: radiology (X-ray) departments, nuclear power plants, nuclear medicine departments, high-energy physics, x-ray crystallography (study of crystal structure) and radiation oncology (the study of cancer) departments. Some of the isotopes that are of concern are cobalt-60, cesium-137, and americium-241}
\end{itemize}



\begin{IFact}{
Radiation therapy (or radiotherapy) uses ionising radiation as part of cancer treatment to control malignant cells. In cancer, a malignant cell is one that divides very rapidly to produce many more cells. These groups of dividing cells can form a growth or \textbf{tumour}. The malignant cells in the tumour can take nutrition away from other healthy body cells, causing them to die, or can increase the pressure in parts of the body because of the space that they take up. Radiation therapy uses radiation to try to target these malignant cells and kill them. However, the radiation can also damage other, healthy cells in the body. To stop this from happening, shaped radiation beams are aimed from several angles to intersect at the tumour, so that the radiation dose here is much higher than in the surrounding, healthy tissue. But even doing this doesn't protect all the healthy cells, and that is why people have side-effects to this treatment. \\

Note that radiation therapy is different from chemotherapy, which uses \textit{chemicals}, rather than radiation, to destroy malignant cells. Generally, the side effects of chemotherapy are greater because the treatment is not as localised as it is with radiation therapy. The chemicals travel throughout the body, affecting many healthy cells.
}
\end{IFact}
\


% CHILD SECTION END



% CHILD SECTION START

\section{The 'half-life' of an element}
\label{sec:an:halflife}

\Definition{Half-life}{The half-life of an element is the time it takes for half the atoms of a radioisotope to decay into other atoms.}

Table \ref{tab:an:r} gives some examples of the half-lives of different elements.\\

\begin{table}[!h]
\begin{center}
\caption{Table showing the half-life of a number of elements}

\begin{tabular}{|l|l|l|}\hline
\textbf{Radioisotope} & \textbf{Chemical symbol} & \textbf{Half-life}\\\hline\hline
Polonium-212 & Po-212 & 0.16 seconds\\\hline
Sodium-24 & Na-24 & 15 hours\\\hline
Strontium-90 & Sr-90 & 28 days\\\hline
Cobalt-60 & Co-60 & 5.3 years\\\hline
Caesium-137 & Cs-137 & 30 years\\\hline
Carbon-14 & C-14 & 5 760 years\\\hline
Calcium-41 & Ca-41 & 100 000 years\\\hline
Beryllium-10 & Be-10 & 2 700 000 years\\\hline
Uranium-235 & U-235 & 7.1 billion years\\\hline\hline
\end{tabular}
\label{tab:an:r}

\end{center}
\end{table}

So, in the case of Sr-90, it will take 28 days for half of the atoms to decay into other atoms. It will take another 28 days for half of the remaining atoms to decay. Let's assume that we have a sample of strontium that weighs 8g. After the first 28 days there will be:

\begin{center}
1/2 x 8 = 4 g Sr-90 left
\end{center}

After 56 days, there will be:
\begin{center}
1/2 x 4 g = 2 g Sr-90 left
\end{center}

After 84 days, there will be:
\begin{center}
1/2 x 2 g = 1 g Sr-90 left
\end{center}

If we convert these amounts to a \textit{fraction} of the original sample, then after 28 days 1/2 of the sample remains undecayed. After 56 days 1/4 is undecayed and after 84 days, 1/8 and so on.
Khan Academy video on half-life: SIYAVULA-VIDEO:http://cnx.org/content/m38943/latest/#slidesharefigure
\Activity{Group work}{Understanding half-life\\}{

\textit{Work in groups of 4-5}

\textbf{You will need:}

16 sheets of A4 paper per group, scissors, 2 boxes per group, a marking pen and timer/stopwatch.\\

\textbf{What to do:\\}

\begin{itemize}
\item{Your group should have two boxes. Label one 'decayed' and the other 'radioactive'.}
\item{Take the A4 pages and cut each into 4 pieces of the same size. You should now have 64 pieces of paper. Stack these neatly and place them in the 'radioactive' box. The paper is going to represent some radioactive material.}
\item{Set the timer for one minute. After one minute, remove half the sheets of paper from the radioactive box and put them in the 'decayed' box.}
\item{Set the timer for another minute and repeat the previous step, again removing half the pieces of paper that are left in the radioactive box and putting them in the decayed box.}
\item{Repeat this process until 8 minutes have passed. You may need to start cutting your pieces of paper into even smaller pieces as you progress.\\}
\end{itemize}

\textbf{Questions:}

\begin{enumerate}
\item{How many pages were left in the radioactive box after...
\begin{enumerate}
\item{1 minute}
\item{3 minutes}
\item{5 minutes}
\end{enumerate}
}
\item{What percentage (\%) of the pages were left in the radioactive box after...
\begin{enumerate}
\item{2 minutes}
\item{4 minutes}
\end{enumerate}
}
\item{After how many minutes is there 1/128 of radioactive material remaining?}
\item{What is the half-life of the 'radioactive' material in this exercise?}
\end{enumerate}
}


\begin{wex}{Half-life 1\\}{A 100 g sample of Cs-137 is allowed to decay. Calculate the mass of Cs-137 that will be left after 90 years\\}

{\westep{You need to know the half-life of Cs-137}
The half-life of Cs-137 is 30 years.\\
}
{\westep{Determine how many times the quantity of sample will be halved in 90 years.}
If the half-life of Cs-137 is 30 years, and the sample is left to decay for 90 years, then the number of times the quantity of sample will be halved is 90/30 = 3.\\
}
{\westep{Calculate the quantity that will be left by halving the mass of Cs-137 three times}
1. After 30 years, the mass left is 100 g $\times$ 1/2 = 50 g

2. After 60 years, the mass left is 50 g $\times$ 1/2 = 25 g

3. After 90 years, the mass left is 25 g $\times$ 1/2 = 12.5 g\\

Note that a quicker way to do this calculation is as follows:

Mass left after 90 years = (1/2)$^{3}$ $\times$ 100 g = 12.5 g (The exponent is the number of times the quantity is halved)

}
\end{wex}

\begin{wex}{Half-life 2\\}{An 80 g sample of Po-212 decays until only 10 g is left. How long did it take for this decay to take place?\\}

{\westep{Calculate the fraction of the original sample that is left after decay}
Fraction remaining = 10 g/80 g = 1/8\\
}

{\westep{Calculate how many half-life periods of decay (x) must have taken place for 1/8 of the original sample to be left}

\begin{equation*}
(\frac{1}{2})^{x} = \frac{1}{8}
\end{equation*}

Therefore, x = 3\\
}

{\westep{Use the half-life of Po-212 to calculate how long the sample was left to decay}
The half-life of Po-212 is 0.16 seconds. Therefore if there were three periods of decay, then the total time is 0.16 $\times$ 3. The time that the sample was left to decay is 0.48 seconds.
}

\end{wex}




\Exercise{Looking at half life\\}{
\begin{enumerate}
\item{Imagine that you have 100 g of Na-24.
\begin{enumerate}
\item{What is the half life of Na-24?}
\item{How much of this isotope will be left after 45 hours?}
\item{What percentage of the original sample will be left after 60 hours?}
\end{enumerate}
}
\item{A sample of Sr-90 is allowed to decay. After 84 days, 10 g of the sample remains.
\begin{enumerate}
\item{What is the half life of Sr-90?}
\item{How much Sr-90 was in the original sample?}
\item{How much Sr-90 will be left after 112 days?}
\end{enumerate}
}
\end{enumerate}
\practiceinfo

\begin{tabular}[h]{cccccc}
(1.) aaa & (2.) aaa & 
 \end{tabular}
}



% CHILD SECTION END



% CHILD SECTION START

\section{The Dangers of Radiation}
\label{sec:an:dangers}

Natural radiation comes from a variety of sources such as the rocks, sun and from space. However, when we are exposed to large amounts of radiation, this can cause damage to cells. $\gamma$ radiation is particularly dangerous because it is able to penetrate the body, unlike $\alpha$ and $\beta$ particles whose penetration power is less. Some of the dangers of radiation are listed below:

\begin{itemize}
\item{\textbf{Damage to cells}}

Radiation is able to penetrate the body, and also to penetrate the membranes of the cells within our bodies, causing massive damage. \textit{Radiation poisoning} occurs when a person is exposed to large amounts of this type of radiation. Radiation poisoning damages tissues within the body, causing symptoms such as diarrhoea, vomiting, loss of hair and convulsions.

\item{\textbf{Genetic abnormalities}}

When radiation penetrates cell membranes, it can damage chromosomes within the nucleus of the cell. The chromosomes contain all the genetic information for that person. If the chromosomes are changed, this may lead to genetic abnormalities in any children that are born to the person who has been exposed to radiation. Long after the nuclear disaster of Chernobyl in Russia in 1986, babies were born with defects such as missing limbs and abnormal growths.

\item{\textbf{Cancer}}

Small amounts of radiation can cause cancers such as leukemia (cancer of the blood)

\end{itemize}


% CHILD SECTION END



% CHILD SECTION START

\section{The Uses of Radiation}
\label{sec:an:uses}

However, despite the many dangers of radiation, it does have many powerful uses, some of which are listed below:

\begin{itemize}
\item{\textbf{Medical Field}}

Radioactive \textit{chemical tracers} emitting $\gamma$ rays can give information about a person's internal anatomy and the functioning of specific organs. The radioactive material may be injected into the patient, from where it will target specific areas such as bones or tumours. As the material decays and releases radiation, this can be seen using a special type of camera or other instrument. The radioactive material that is used for this purpose must have a short half-life so that the radiation can be detected quickly and also so that the material is quickly removed from the patient's body.
Using radioactive materials for this purpose can mean that a tumour or cancer may be diagnosed long before these would have been detected using other methods such as X-rays.

Radiation may also be used to sterilise medical equipment.

\Activity{Research Project}{The medical uses of radioisotopes\\}{

Carry out your own research to find out more about the radioisotopes that are used to diagnose diseases in the following parts of the body:

\begin{itemize}
\item{thyroid gland}
\item{kidneys}
\item{brain\\}
\end{itemize}

In each case, try to find out...

\begin{enumerate}
\item{which radioisotope is used}
\item{what the sources of this radioisotope are}
\item{how the radioisotope enters the patient's body and how it is monitored}
\end{enumerate}
}

\item{\textbf{Biochemistry and Genetics}}

Radioisotopes may be used as tracers to label molecules so that chemical processes such as DNA replication or amino acid transport can be traced.

\item{\textbf{Food preservation}}

Irradiation of food can stop vegetables or plants from sprouting after they have been harvested. It also kills bacteria and parasites, and controls the ripening of fruits.

\item{\textbf{Environment}}

Radioisotopes can be used to trace and analyse pollutants.

\item{\textbf{Archaeology and Carbon dating}}

Natural radioisotopes such as C-14 can be used to determine the age of organic remains. All living organisms (e.g. trees, humans) contain carbon. Carbon is taken in by plants and trees through the process of photosynthesis in the form of carbon dioxide and is then converted into organic molecules. When animals feed on plants, they also obtain carbon through these organic compounds. Some of the carbon in carbon dioxide is the radioactive C-14, while the rest is a non-radioactive form of carbon. When an organism dies, no more carbon is taken in and the amount of C-14 in the body stops increasing. From this point onwards, C-14 begins its radioactive decay which reduces the amount of C-14 in the body. When scientists uncover remains, they are able to estimate the age of the remains by seeing how much C-14 is left in the body relative to the amount of non-radioactive carbon. The less C-14 there is, the older the remains because radioactive decay must have been taking place for a long time. Because scientists know the exact rate of decay of C-14, they can calculate a relatively accurate estimate of the age of the remains. Carbon dating has been an important tool in building up historical records.

\Activity{Case Study}{Using radiocarbon dating\\}{

Radiocarbon dating has played an important role in uncovering many aspects of South Africa's history. Read the following extract from an article that appeared in Afrol news on 10th February 2007 and then answer the questions that follow.\\

\begin{quote}
The world famous rock art in South Africa's uKhahlamba-Drakensberg, a World Heritage Site, is three times older than previously thought, archaeologists conclude in a new study. The more than 40,000 paintings were made by the San people some 3000 years ago, a new analysis had shown.

Previous work on the age of the rock art in uKhahlamba-Drakensberg concluded it is less than 1,000 years old. But the new study - headed by a South African archaeologist leading a team from the University of Newcastle upon Tyne (UK) and Australian National University in Canberra - estimates the panels were created up to 3,000 years ago. They used the latest radio-carbon dating technology.

The findings, published in the current edition of the academic journal 'South African Humanities', have "major implications for our understanding of how the rock artists lived and the social changes that were taking place over the last three millennia," according to a press release from the British university. \\
\end{quote}

\textbf{Questions:}

\begin{enumerate}
\item{What is the half-life of carbon-14?}
\item{In the news article, what role did radiocarbon dating play in increasing our knowledge of South Africa's history?}
\item{Radiocarbon dating can also be used to analyse the remains of once-living organisms. Imagine that a set of bones are found between layers of sediment and rock in a remote area. A group of archaeologists carries out a series of tests to try to estimate the age of the bones. They calculate that the bones are approximately 23 040 years old.

What percentage of the original carbon-14 must have been left in the bones for them to arrive at this estimate?}
\end{enumerate}
}

\end{itemize}



% CHILD SECTION END



% CHILD SECTION START

\section{Nuclear Fission}
\label{sec:an:nfiss}

\textbf{Nuclear fission} is a process where the nucleus of an atom is split into two or more smaller nuclei, known as \textit{fission products}. The fission of heavy elements is an \textbf{exothermic reaction} and huge amounts of energy are released in the process. This energy can be used to produce \textit{nuclear power} or to make \textit{nuclear weapons}, both of which we will discuss a little later.

\Definition{Nuclear fission}{The splitting of an atomic nucleus into smaller nuclei}

Below is a diagram showing the nuclear fission of Uranium-235. An atom of Uranium-235 is bombarded with a neutron to initiate the fission process. This neutron is absorbed by Uranium-235, to become Uranium-236. Uranium-236 is highly unstable and breaks down into a number of lighter elements, releasing energy in the process. Free neutrons are also produced during this process, and these are then available to bombard other fissionable elements. This process is known as a \textbf{fission chain reaction}, and occurs when one nuclear reaction starts off another, which then also starts off another one so that there is a rapid increase in the number of nuclear reactions that are taking place.

\begin{figure}[!h]
\begin{pspicture}(-7,-4)(7,4)
\SpecialCoor
%\psgrid[gridcolor=lightgray]
\pscircle(-6.5,0){0.25}
\psline[linestyle=dotted,arrows=->](-6.25,0)(-5.5,0)
\rput(-6.5,-0.5){neutron}
\pscircle(-4.5,0){1}
\rput(-4.5,0){U-235}
\psline[linestyle=solid,arrows=->](-3.5,0)(-2.5,0)
\pscircle(-1.5,0){1}
\rput(-1.5,0){U-236}
\rput(-1.5,2){\small{\parbox[l]{3cm}{Neutron is absorbed by the nucleus of the U-235 atom to form U-236}}}
\psline[linestyle=solid,arrows=->,linewidth=5pt](-0.5,0)(0.5,0)
\pspolygon(1,-1)(2,0.2)(1,0)(1.8,0.6)(1,2)(2,1)(2.5,2.5)(2.8,1)(4,2)(3.2,0.5)(4,0)(3,0.2)(4,-1)(3,-0.5)(2.5,-2)(2,-0.5)
\psline[linestyle=dotted,arrows=->](2,-3)({5;290})
\pscircle[fillstyle=solid,fillcolor=white](2,-3){0.5}
\pscircle(3,4){0.75}
\pscircle({3;70}){0.25}
\pscircle({5;0}){0.25}
\pscircle({5;-30}){0.25}
\pszigzag[coilarm=0.5,linearc=0.1,coilwidth=0.5]{->}(4,1.5)(5,3)
\pszigzag[coilarm=0.5,linearc=0.1,coilwidth=0.5]{->}(4.5,-1)(6,-2.5)
\rput(5.5,3.5){\small{\parbox[l]{3cm}{Massive release of energy during nuclear fission}}}
\rput(-0.25,-3){\small{\parbox[l]{3cm}{U-236 splits into lighter elements called \textit{fission products} and free neutrons}}}
\rput(7,0){\small{\parbox[l]{3cm}{The elements and number of neutrons produced in the process, is random.}}}
\psdot[dotsize=3pt](0,0)
\end{pspicture}
\end{figure}
Phet simulation on nuclear fission: SIYAVULA-SIMULATION:http://cnx.org/content/m38951/latest/#id6845


\subsection{The Atomic bomb - an abuse of nuclear fission}
\label{subsec:an:nfiss:bomb}

A nuclear chain reaction can happen very quickly, releasing vast amounts of energy in the process. In 1939, it was discovered that Uranium could undergo nuclear fission. In fact, it was uranium that was used in the first atomic bomb. The bomb contained large amounts of Uranium-235, enough to start a runaway nuclear fission chain reaction. Because the process was uncontrolled, the energy from the fission reactions was released in a matter of \textit{seconds}, resulting in the massive explosion of that first bomb. Since then, more atomic bombs have been detonated, causing massive destruction and loss of life.


\Activity{Discussion}{Nuclear weapons testing - an ongoing issue\\}{
Read the article below which has been adapted from one that appeared in 'The Globe' in Washington on 10th October 2006, and then answer the questions that follow.\\
\begin{quote}
US officials and arms control specialists warned yesterday that North Korea's test of a small nuclear device could start an arms race in the region and threaten the landmark global treaty designed nearly four decades ago to halt the spread of nuclear weapons. US officials expressed concern that North Korea's neighbors, including Japan, Taiwan, and South Korea, could eventually decide to develop weapons of their own. They also fear that North Korea's moves could embolden Iran, and that this in turn could encourage Saudi Arabia or other neighbours in the volatile Middle East to one day seek nuclear deterrents, analysts say.

North Korea is the first country to conduct a nuclear test after pulling out of the Nuclear Nonproliferation Treaty. The treaty, which was created in 1968, now includes 185 nations (nearly every country in the world). Under the treaty, the five declared nuclear powers at the time (United States, the Soviet Union, France, China, and Great Britain) agreed to reduce their supplies of nuclear weapons. The treaty has also helped to limit the number of new nuclear weapons nations.

But there have also been serious setbacks. India and Pakistan, which never signed the treaty, became new nuclear powers, shocking the world with test explosions in 1998. The current issue of nuclear weapons testing in North Korea, is another such setback and a blow to the treaty.\\
\end{quote}


\textbf{Group discussion questions:\\}

\begin{enumerate}
\item{Discuss what is meant by an 'arms race' and a 'treaty'.}
\item{Do you think it is important to have such treaties in place to control the testing and use of nuclear weapons? Explain your answer.}
\item{Discuss some of the reasons why countries might not agree to be part of a nuclear weapons treaty.}
\item{How would you feel if South Africa decided to develop its own nuclear weapons?}
\end{enumerate}
}

\subsection{Nuclear power - harnessing energy}
\label{subsec:an:nfiss:power}

However, nuclear fission can also be carried out in a controlled way in a \textit{nuclear reactor}. A nuclear reactor is a piece of equpiment where nuclear chain reactions can be started in a controlled and sustained way. This is different from a nuclear \textit{explosion} where the chain reaction occurs in seconds. The most important use of nuclear reactors at the moment is to produce \textbf{electrical power}, and most of these nuclear reactors use nuclear fission. A \textbf{nuclear fuel} is a chemical isotope that can keep a fission chain reaction going. The most common isotopes that are used are Uranium-235 and Plutonium-239. The amount of free energy that is in nuclear fuels is far greater than the energy in a similar amount of other fuels such as gasoline. In many countries, nuclear power is seen as a relatively environmentally friendly alternative to fossil fuels, which release large amounts of greenhouse gases, and are also non-renewable resources. However, one of the concerns around the use of nuclear power is the production of \textit{nuclear waste}, which contains radioactive chemical elements.

\Activity{Debate}{Nuclear Power\\}{
The use of nuclear power as a source of energy has been a subject of much debate. There are many advantages of nuclear power over other energy sources. These include the large amount of energy that can be produced at a small plant, little atmospheric pollution and the small quantity of waste. However there are also disadvantages. These include the expense of maintaining nuclear power stations, the huge impact that an accident could have as well as the disposal of dangerous nuclear waste.\\

Use these ideas as a starting point for a class debate.\\

\begin{center}
\textbf{Nuclear power - An energy alternative or environmental hazard?}
\end{center}

Your teacher will divide the class into teams. Some of the teams will be 'pro' nuclear power while the others will be 'anti' nuclear power.
}



% CHILD SECTION END



% CHILD SECTION START

\section{Nuclear Fusion}
\label{sec:an:nfus}

\textbf{Nuclear fusion} is the joining together of the nuclei of two atoms to form a heavier nucleus. If the atoms involved are small, this process is accompanied by the release of energy. It is the nuclear fusion of elements that causes stars to shine and hydrogen bombs to explode. As with nuclear \textit{fission} then, there are both positive and negative uses of nuclear fusion.

\Definition{Nuclear fusion}{The joining together of the nuclei of two atoms to form a larger nucleus.}

You will remember that nuclei naturally repel one another because of the electrostatic force between their positively charged protons. So, in order to bring two nuclei together, a lot of energy must be supplied if fusion is to take place. If two nuclei can be brought close enough together however, the electrostatic force is overwhelmed by the more powerful strong nuclear force which only operates over short distances. If this happens, nuclear fusion can take place. Inside the cores of stars, the temperature is high enough for hydrogen fusion to take place but scientists have so far been unsuccessful in making the process work in the laboratory. One of the huge advantages of nuclear fusion, if it could be made to work in the laboratory, is that it is a relatively environmentally friendly source of energy. The helium that is produced is not radioactive or poisonous and does not carry the dangers of nuclear fission.\\



% CHILD SECTION END



% CHILD SECTION START

\section{Nucleosynthesis}
An astronomer named Edwin Hubble discovered in the 1920's that the universe is expanding. He measured that far-away galaxies are moving away from the earth at great speed, and the further away they are, the faster they are moving.

\Extension{What are galaxies?}{
Galaxies are huge clusters of stars and matter in the universe. The earth is part of the Milky Way galaxy which is shaped like a very large spiral. Astronomers can measure the light coming from distant galaxies using telescopes. Edwin Hubble was also able to measure the velocities of galaxies.}

These observations led people to see that the universe is expanding. It also led to the 'Big Bang' hypothesis. The 'Big Bang' hypothesis is an idea about how the universe may have started. According to this theory, the universe started off at the beginning of time as a point which then exploded and expanded into the universe we live in today. This happened between 10 and 14 billion years ago.\\

Just after the Big Bang, when the universe was only $10^{-43}$s old, it was very hot and was made up of quarks and leptons (an example of a lepton is the electron). As the universe expanded, ($\sim10^{-2}$s) and cooled, the quarks started binding together to form protons and neutrons (together called \emph{nucleons}).

\subsection{Age of Nucleosynthesis (225 s - $10^{3}$ s)}
About 225 s after the Big Bang, the protons and neutrons started binding together to form simple \emph{nuclei}. The process of forming nuclei is called \emph{nucleosynthesis}. When a proton and a neutron bind together, they form the \emph{deuteron}. The deuteron is like a hydrogen nucleus (which is just a proton) with a neutron added to it so it can be written as $^{2}\rm{H}$. Using protons and neutrons as building blocks, more nuclei can be formed as shown below. For example, the Helium-4 nucleus (also called an \emph{alpha particle}) can be formed in the following ways:

\begin{eqnarray*}
^{2}\rm{H} + n & \rightarrow & ^{3}\rm{H} \\
\rm{deuteron} + \rm{neutron} & \rightarrow & \rm{triton}
\end{eqnarray*}

\centerline{\emph{then:}}

\begin{eqnarray*}
^{3}\rm{H} + p & \rightarrow & ^{4}\rm{He} \\
\rm{triton} + \rm{proton} & \rightarrow & \rm{Helium}\rm{4} \ \rm{(alpha} \ \rm{particle)}
\end{eqnarray*}

or

\begin{eqnarray*}
^{2}\rm{H} + p & \rightarrow & ^{3}\rm{He} \\
\rm{deuteron} + \rm{proton} & \rightarrow & \rm{Helium}\rm{3}
\end{eqnarray*}

\centerline{\emph{then:}}

\begin{eqnarray*}
^{3}\rm{He} + n & \rightarrow & ^{4}\rm{He} \\
\rm{Helium3} + \rm{neutron} & \rightarrow & \rm{Helium}\rm{4} \ \rm{(alpha} \ \rm{particle)}
\end{eqnarray*}

Some $^{7}\rm{Li}$ nuclei could also have been formed by the fusion of $^{4}\rm{He}$ and $^{3}\rm{H}$.

\subsection{Age of Ions ($10^{3}$ s - $10^{13}$ s)}
However, at this time the universe was still very hot and the electrons still had too much energy to become bound to the alpha particles to form helium \emph{atoms}. Also, the nuclei with mass numbers greater than 4 (i.e. greater than $^{4}\rm{He}$) are very short-lived and would have decayed almost immediately after being formed. Therefore, the universe moved through a stage called the Age of Ions when it consisted of free positively charged $\rm{H}^{+}$ ions and $^{4}\rm{He}$ ions, and negatively charged electrons not yet bound into atoms.

\subsection{Age of Atoms ($10^{13}$ s - $10^{15}$ s)}
As the universe expanded further, it cooled down until the electrons were able to bind to the hydrogen and helium nuclei to form hydrogen and helium atoms. Earlier, during the Age of Ions, both the hydrogen and helium ions were positively charged which meant that they repelled each other (electrostatically). During the Age of Atoms, the hydrogen and helium along with the electrons, were in the form of atoms which are electrically neutral and so they no longer repelled each other and instead pulled together under gravity to form clouds of gas, which evetually formed stars.

\subsection{Age of Stars and Galaxies (the universe today)}
Inside the core of stars, the densities and temperatures are high enough for fusion reactions to occur. Most of the heavier nuclei that exist today were formed inside stars from thermonuclear reactions! (It's interesting to think that the atoms that we are made of were actually manufactured inside stars!). Since stars are mostly composed of hydrogen, the first stage of thermonuclear reactions inside stars involves hydrogen and is called \textbf{hydrogen burning}. The process has three steps and results in four hydrogen atoms being formed into a helium atom with (among other things) two photons (light!) being released. \\

The next stage is \textbf{helium burning} which results in the formation of carbon. All these reactions release a large amount of energy and heat the star which causes heavier and heavier nuclei to fuse into nuclei with higher and higher atomic numbers. The process stops with the formation of $^{56}\rm{Fe}$, which is the most strongly bound nucleus. To make heavier nuclei, even higher energies are needed than is possible inside normal stars. These nuclei are most likely formed when huge amounts of energy are released, for example when stars explode (an exploding star is called a \textbf{supernova}). This is also how all the nuclei formed inside stars get "recycled" in the universe to become part of new stars and planets.

\summary{aaa}

\begin{itemize}
\item{\textbf{Nuclear physics} is the branch of physics that deals with the nucleus of an atom.}
\item{There are two forces between the particles of the nucleus. The \textbf{strong nuclear force} is an attractive force between the neutrons and the \textbf{electromagnetic force} is the repulsive force between like-charged protons.}
\item{In atoms with large nuclei, the electromagnetic force becomes greater than the strong nuclear force and particles or energy may be released from the nucleus.}
\item{\textbf{Radioactive decay} occurs when an unstable atomic nucleus loses energy by emitting particles or electromagnetic waves.}
\item{The particles and energy released are called \textbf{radiation} and the atom is said to be \textbf{radioactive}.}
\item{Radioactive isotopes are called \textbf{radioisotopes}.}
\item{Radioactivity was first discovered by Henri Becquerel, Marie Curie and her husband Pierre.}
\item{There are three types of radiation from radioactive decay: \textbf{alpha ($\alpha$)}, \textbf{beta ($\beta$)} and \textbf{gamma ($\gamma$)} radiation.}
\item{During \textbf{alpha decay}, an alpha particle is released. An alpha particle consists of two protons and two neutrons bound together. Alpha radiation has low penetration power.}
\item{During \textbf{beta decay}, a beta particle is released. During beta decay, a neutron is converted to a proton, an electron and a neutrino. A beta particle is the electron that is released. Beta radiation has greater penetration power than alpha radiation.}
\item{During \textbf{gamma decay}, electromagnetic energy is released as gamma rays. Gamma radiation has the highest penetration power of the three radiation types.}
\item{There are many \textbf{sources of radiation}. Some sources are natural and others are man-made.}
\item{Natural sources of radiation include cosmic and terrestrial radiation.}
\item{Man-made sources of radiation include televisions, smoke detectors, X-rays and radiation therapy.}
\item{The \textbf{half-life} of an element is the time it takes for half the atoms of a radioisotope to decay into other atoms.}
\item{Radiation can be very damaging. Some of the \textbf{negative impacts of radiation} exposure include damage to cells, genetic abnormalities and cancer.}
\item{However, radiation can also have many \textbf{positive uses}. These include use in the medical field (e.g. chemical tracers), biochemistry and genetics, use in food preservation, the environment and in archaeology.}
\item{\textbf{Nuclear fission} is the splitting of an atomic nucleus into smaller fission products. Nuclear fission produces large amounts of energy, which can be used to produce nuclear power, and to make nuclear weapons.}
\item{\textbf{Nuclear fusion} is the joining together of the nuclei of two atoms to form a heavier nucleus. In stars, fusion reactions involve the joining of hydrogen atoms to form helium atoms.}
\item{\textbf{Nucleosynthesis} is the process of forming nuclei. This was very important in helping to form the universe as we know it.}
\end{itemize}


\begin{eocexercises}{}

\begin{enumerate}
\item{Explain each of the following terms:}
\begin{enumerate}
\item{electromagnetic force}
\item{radioactive decay}
\item{radiocarbon dating}
\end{enumerate}

\item{For each of the following questions, choose the \textbf{one correct answer}:}

\begin{enumerate}
\item{The part of the atom that undergoes radioactive decay is the...
\begin{enumerate}
\item{neutrons}
\item{nucleus}
\item{electrons}
\item{entire atom}
\end{enumerate}
}
\item{The radioisotope Po-212 undergoes alpha decay. Which of the following statements is \textbf{true}?
\begin{enumerate}
\item{The number of protons in the element remains unchanged.}
\item{The number of nucleons after decay is 212.}
\item{The number of protons in the element after decay is 82.}
\item{The end product after decay is Po-208.}
\end{enumerate}
}
\end{enumerate}


\item{20 g of sodium-24 undergoes radoactive decay. Calculate the percentage of the original sample that remains after 60 hours.}

\item{Nuclear physics can be controversial. Many people argue that studying the nucleus has led to devastation and huge loss of life. Others would argue that the benefits of nuclear physics far outweigh the negative things that have come from it.}
\begin{enumerate}
\item{Outline some of the ways in which nuclear physics has been used in negative ways.}
\item{Outline some of the benefits that have come from nuclear physics.}
\end{enumerate}
\end{enumerate}

\practiceinfo

\begin{tabular}[h]{cccccc}
(1.) aaa & (2.) aaa & (3.) aaa & (4.) aaa & 
 \end{tabular}
\end{eocexercises}


% CHILD SECTION END



% CHILD SECTION END



% CHILD SECTION START

