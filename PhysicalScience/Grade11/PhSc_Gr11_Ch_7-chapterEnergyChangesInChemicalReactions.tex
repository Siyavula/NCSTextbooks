\chapter{Energy Changes In Chemical Reactions}
\label{chap:energychanges}

All chemical reactions involve energy changes. In some reactions, we are able to observe these energy changes as either an increase or a decrease in the overall energy of the system.

\chapterstartvid{VPjkd}

% CHILD SECTION START

\section{What causes the energy changes in chemical reactions?}

When a chemical reaction occurs, bonds in the reactants \textit{break}, while new bonds \textit{form} in the product. The following example explains this.
Hydrogen reacts with oxygen to form water, according to the following equation:

\begin{equation*}
2\text{H}_{2} + \text{O}_{2} \rightarrow 2\text{H}_{2}\text{O}
% \label{eqnfirst:ec:ec}
\end{equation*}

In this reaction, the bond between the two hydrogen atoms in the H$_{2}$ molecule will \textit{break}, as will the bond between the oxygen atoms in the O$_{2}$ molecule. New bonds will \textit{form} between the two hydrogen atoms and the single oxygen atom in the water molecule that is formed as the product.

For bonds to \textit{break}, energy must be \textit{absorbed}. When new bonds \textit{form}, energy is \textit{released}. The energy that is needed to break a bond is called the \textbf{bond energy} or \textbf{bond dissociation energy}. Bond energies are measured in units of kJ.mol$^{-1}$.

\Definition{Bond energy}{Bond energy is a measure of bond strength in a chemical bond. It is the amount of energy (in kJ.mol$^{-1}$) that is needed to break the chemical bond between two atoms.}



% CHILD SECTION END



% CHILD SECTION START

\section{Exothermic and endothermic reactions}
\label{sec:energychanges:exoendo}

In some reactions, the energy that must be \textit{absorbed} to break the bonds in the reactants, is less than
the total energy that is \textit{released} when new bonds are formed. This means that in the overall reaction, energy is \textit{released} as either heat or light. This type of reaction is called an \textbf{exothermic} reaction. Another way of describing an exothermic reaction is that it is one in which the energy of the product is less than the energy of the reactants, because energy has been released during the reaction. We can represent this using the following general formula:

\begin{center}
$\rm{Reactants \rightarrow Product + Energy}$
\end{center}
\Definition{Exothermic reaction}{An exothermic reaction is one that releases energy in the form of heat or light.}

In other reactions,the energy that must be \textit{absorbed} to break the bonds in the reactants, is more than
the total energy that is \textit{released} when new bonds are formed. This means that in the overall reaction, energy must be \textit{absorbed} from the surroundings. This type of reaction is known as an \textbf{endothermic} reaction. Another way of describing an endothermic reaction is that it is one in which the energy of the product is greater than the energy of the reactants, because energy has been absorbed during the reaction. This can be represented by the following formula:

\begin{center}
$\rm{Reactants + Energy \rightarrow Product}$
\end{center}
\Definition{Endothermic reaction}{An endothermic reaction is one that absorbs energy in the form of heat or light.}

The difference in energy (E) between the reactants and the products is known as the \textbf{heat of the reaction}. It is also sometimes referred to as the \textbf{enthalpy change} of the system.

\begin{g_experiment}{Endothermic and exothermic reactions 1}{
\textbf{Apparatus and materials: } You will need citric acid, sodium bicarbonate, a glass beaker, the lid of a margarine container, thermometer, glass stirring rod and a pair of scissors. Note that citric acid is found in citrus fruits such as lemons. Sodium bicarbonate is actually bicarbonate of soda (baking soda), the baking ingredient that helps cakes to rise.\\
\textbf{Method: }
\begin{enumerate}
\item{Cut a piece of plastic from the margarine container lid that will be big enough to cover the top of the beaker. Cut a small hole in the centre of this piece of plastic and place the thermometer through it.}
\item{Pour some citric acid (H$_{3}$C$_{6}$H$_{5}$O$_{7}$) into the glass beaker, cover the beaker with its 'lid' and record the temperature of the solution.}
\item{Stir in the sodium bicarbonate (NaHCO$_{3}$), then cover the beaker again.}
\item{Immediately record the temperature, and then take a temperature reading every two minutes after that. Record your results in a table like the one below.}
\end{enumerate}
\begin{center}
\begin{tabular}{|l|c|c|c|c|}\hline
\textbf{Time} (mins) & 0 & 2 & 4 & 6 \\\hline
\textbf{Temperature} ($^{0}$C) & & & &  \\\hline
\end{tabular}
\end{center}

The equation for the reaction that takes place is:

\begin{center}
$\text{H}_{3}\text{C}_{6}\text{H}_{5}\text{O}_{7}\text{(aq)} + 3\text{NaHCO}_{3}\text{(s)} \rightarrow 3\text{CO}_{2}\text{(g)} + 3\text{H}_{2}\text{O}(\ell) + \text{Na}_{3}\text{C}_{6}\text{H}_{5}\text{O}_{7}\text{(aq)}$
\end{center}

\textbf{Results: }
\begin{itemize}
\item{Plot your temperature results on a graph of temperature against time. What happens to the temperature during this reaction?}
\item{Is this an exothermic or an endothermic reaction?}
\item{Why was it important to keep the beaker covered with a lid?}
\item{Do you think a glass beaker is the best thing to use for this experiment? Explain your answer.}
\item{Suggest another container that could have been used and give reasons for your choice. It might help you to look back to Grade 10 for some ideas!}
\end{itemize}
}
\end{g_experiment}
\begin{g_experiment}{Endothermic and exothermic reactions 2}{

\textbf{Apparatus and materials: } Vinegar, steel wool, thermometer, glass beaker and plastic lid (from previous demonstration).\\
\textbf{Method: }
\begin{enumerate}
\item{Put the thermometer through the plastic lid, cover the beaker and record the temperature in the empty beaker. You will need to leave the thermometer in the beaker for about 5 minutes in order to get an accurate reading.}
\item{Take the thermometer out of the jar.}
\item{Soak a piece of steel wool in vinegar for about a minute. The vinegar removes the protective coating from the steel wool so that the metal is exposed to oxygen.}
\item{After the steel wool has been in the vinegar, remove it and squeeze out any vinegar that is still on the wool. Wrap the steel wool around the thermometer and place it (still wrapped round the thermometer) back into the jar. The jar is automatically sealed when you do this because the thermometer is through the top of the lid.}
\item{Leave the steel wool in the beaker for about 5 minutes and then record the temperature. Record your observations.\\}
\end{enumerate}
\textbf{Results: } You should notice that the temperature \textit{increases} when the steel wool is wrapped around the thermometer. \\
\textbf{Conclusion: } The reaction between oxygen and the exposed metal in the steel wool, is \textbf{exothermic}, which means that energy is released and the temperature increases.
}
\end{g_experiment}
\pagebreak
\section{The heat of reaction}

The \textbf{heat of the reaction} is represented by the symbol $\Delta H$, where:

\begin{eqnarray*}
\Delta H = E_{prod} - E_{react}
\end{eqnarray*}

\begin{itemize}
\item{In an \textit{exothermic} reaction, $\Delta H$ is less than zero because the energy of the reactants is greater than the energy of the product. For example,\\

$\text{H}_{2} + \text{Cl}_{2} \rightarrow 2\text{HCl} \hspace{.5cm} \Delta H = -183 \text{ kJ}$
}
\item{In an \textit{endothermic} reaction, $\Delta H$ is greater than zero because the energy of the reactants is less than the energy of the product. For example, \\

$\text{C} + \text{H}_{2}\text{O} \rightarrow \text{CO} + \tex{H}_{2}} \hspace{.5cm} \Delta H = +131 \text{ kJ}$
}
\end{itemize}

Some of the information relating to exothermic and endothermic reactions is summarised in table \ref{tab:energy}.

\Definition{Enthalpy}{Enthalpy is the heat content of a chemical system for a given pressure, and is given the symbol 'H'.}

\begin{table}[h]
\begin{center}
\begin{tabular}{|p{4cm}|p{4cm}|p{4cm}|}\hline
\textbf{Type of reaction} & \textbf{Exothermic} & \textbf{Endothermic} \\\hline
\textbf{Energy absorbed or released} & Released & Absorbed \\\hline
\textbf{Relative energy of reactants and products} & Energy of reactants greater than energy of product & Energy of reactants less than energy of product \\\hline
\textbf{Sign of $\Delta$H} & Negative & Positive \\\hline
\end{tabular}
\caption{A comparison of exothermic and endothermic reactions}
\label{tab:energy}
\end{center}
\end{table}

\textbf{Writing equations using $\Delta$H}\\
There are two ways to write the heat of the reaction in an equation.
For the exothermic reaction $\text{C(s)} + \text{O}_{2}\text{(g)} \rightarrow \text{CO}_{2}\text{(g)}$, we can write:\\
$\text{C(s)} + \text{O}_{2}\text{(g)} \rightarrow \text{CO}_{2}\text{(g)} \hspace{.5cm} \Delta H = -393 \text{ kJ} \cdot \text{mol}^{-1}$ or\\
$\text{C(s)} + \text{O}_{2}\text{(g)} \rightarrow \text{CO}_{2}\text{(g)} + 393 \text{ kJ} \cdot \text{mol}^{-1}$\\
For the endothermic reaction, $\text{C(s)} + \text{H}_{2}\text{O(g)} \rightarrow \text{H}_{2}\text{(g)} + \text{CO(g)}$, we can write:\\
$\text{C(s)} + \text{H}_{2}\text{O(g)} \rightarrow \text{H}_{2}\text{(g)} + \text{CO(g)} \hspace{.5cm} \Delta H = +131 \text{ kJ} \cdot \text{mol}^{-1}$ or\\
$\text{C(s)} + \text{H}_{2}\text{O(g)} + 131 \text{kJ} \cdot \text{mol}^{-1} \rightarrow \text{CO} + \text{H}_{2}$ \\
The \textbf{units} for $\Delta H$ are $\text{kJ} \cdot \text{mol}^{-1}$.  In other words, the $\Delta H$ value gives the amount of energy that is absorbed or released per mole of product that is formed. Units can also be written as kJ, which then gives the total amount of energy that is released or absorbed when the product forms.

\Activity{Investigation}{Endothermic and exothermic reactions\\}{

\textbf{Apparatus and materials:\\}

Approximately 2 g each of calcium chloride (CaCl$_{2}$), sodium hydroxide (NaOH), potassium nitrate (KNO$_{3}$) and barium chloride (BaCl$_{2}$); concentrated sulphuric acid (H$_{2}$SO$_{4}$) (Be Careful, this can cause serious burns) ; 5 test tubes; thermometer.\\

\textbf{Method:\\}

\begin{enumerate}
\item{Dissolve about 1 g of each of the following substances in 5-10 cm$^{3}$ of water in a test tube: CaCl$_{2}$, NaOH, KNO$_{3}$ and BaCl$_{2}$.}
\item{Observe whether the reaction is endothermic or exothermic, either by feeling whether the side of the test tube gets hot or cold, or using a thermometer.}
\item{Dilute 3 cm$^{3}$ of concentrated H$_{2}$SO$_{4}$ in 10 cm$^{3}$ of water in the fifth test tube and observe whether the temperature changes.}
\item{Wait a few minutes and then carefully add NaOH to the H$_{2}$SO$_{4}$. Observe any energy changes.}
\item{Record which of the above reactions are endothermic and which are exothermic.\\}
\end{enumerate}

\textbf{Results:\\}
\begin{itemize}
\item{When BaCl$_{2}$ and KNO$_{3}$ dissolve in water, they take in heat from the surroundings. The dissolution of these salts is \textbf{endothermic}.}
\item{When CaCl$_{2}$ and NaOH dissolve in water, heat is released. The process is \textbf{exothermic}.}
\item{The reaction of H$_{2}$SO$_{4}$ and NaOH is also \textbf{exothermic}.\\}
\end{itemize}
}


\section{Examples of endothermic and exothermic reactions}

There are many examples of endothermic and exothermic reactions that occur around us all the time. The following are just a few examples.

\begin{enumerate}
\item{\textbf{Endothermic reactions}

\begin{itemize}
\item{\textbf{Photosynthesis}}

Photosynthesis is the chemical reaction that takes place in plants, which uses energy from the sun to change carbon dioxide and water into food that the plant needs to survive, and which other organisms (such as humans and other animals) can eat so that they too can survive. The equation for this reaction is:

\begin{center}
$\rm{6CO_{2} + 12H_{2}O + energy \rightarrow C_{6}H_{12}O_{6} + 6O_{2} + 6H_{2}O}$
\end{center}

Photosynthesis is an endothermic reaction because it will not happen without an external source of energy, which in this case is sunlight. \\

\item{\textbf{The thermal decomposition of limestone}}

In industry, the breakdown of limestone into quicklime and carbon dioxide is very important. Quicklime can be used to make steel from iron and also to neutralise soils that are too acid. However, the limestone must be heated in a kiln at a temperature of over $900^{\circ}C$ before the decomposition reaction will take place. The equation for the reaction is shown below:

\begin{center}
$\rm{CaCO_{3} \rightarrow  CaO +  CO_{2}}$
\end{center}

\end{itemize}
}

\item{\textbf{Exothermic reactions}

\begin{itemize}
\item{\textbf{Combustion reactions}} - The burning of fuel is an example of a combustion reaction, and we as humans rely heavily on this process for our energy requirements. The following equations describe the combustion of a hydrocarbon such as \textit{methane} (CH$_{4}$):

\begin{center}
$\rm{Fuel + Oxygen \rightarrow Heat + Water + Carbon ~Dioxide}$

$\rm{CH_{4} + 2O_{2} \rightarrow Heat + 2H_{2}O + CO_{2}}$
\end{center}

This is why we burn fuels for energy, because the chemical changes that take place during the reaction release huge amounts of energy, which we then use for things like power and electricity. You should also note that \textit{carbon dioxide} is produced during this reaction. Later we will discuss some of the negative impacts of $CO_{2}$ on the environment. The chemical reaction that takes place when fuels burn therefore has both positive and negative consequences.\\
\begin{IFact}{
Lightsticks or glowsticks are used by divers, campers, and for decoration and fun. A lightstick is a plastic tube with a glass vial inside it. To activate a lightstick, you bend the plastic stick, which breaks the glass vial. This allows the chemicals that are inside the glass to mix with the chemicals in the plastic tube. These two chemicals react and release energy. Another part of a lightstick is a fluorescent dye which changes this energy into light, causing the lightstick to glow!
}
\end{IFact}
\item{\textbf{Respiration}}

Respiration is the chemical reaction that happens in our bodies to produce energy for our cells. The equation below describes what happens during this reaction:

\begin{center}
$\rm{C_{6}H_{12}O_{6} + 6O_{2} \rightarrow 6CO_{2} + 6H_{2}O + energy}$
\end{center}

In the reaction above, glucose (a type of carbohydrate in the food we eat) reacts with oxygen from the air that we breathe in, to form carbon dioxide (which we breathe out), water and energy. The energy that is produced allows the cell to carry out its functions efficiently. Can you see now why you are always told that you must eat food to get energy? It is not the food itself that provides you with energy, but the exothermic reaction that takes place when compounds within the food react with the oxygen you have breathed in!
\end{itemize}
}
\end{enumerate}
\pagebreak
\Exercise{Endothermic and exothermic reactions}{
\begin{enumerate}
\item{In each of the following reactions, say whether the reaction is endothermic or exothermic, and give a reason for your answer.}

\begin{enumerate}
\item{$\text{H}_{2} + \text{I}_{2} \rightarrow 2\text{HI} + 21 \text{ kJ} \cdot \text{mol}^{-1}$}
\item{$\text{CH}_{4} + 2\text{O}_{2} \rightarrow \text{CO}_{2} + 2\text{H}_{2}\text{O} \hspace{.5cm} \Delta H = -802 \text{ kJ} \cdot \text{mol}^{-1}$}
\item{The following reaction takes place in a flask:

$\rm{Ba(OH)_{2}.8H_{2}O + 2NH_{4}NO_{3} \rightarrow Ba(NO_{3})_{2} + 2NH_{3} + 10H_{2}O}$

Within a few minutes, the temperature of the flask drops by approximately $20 \degree C$.}
\item{$2\text{Na} + \text{Cl}_{2} \rightarrow 2\text{NaCl} \hspace{.5cm} \Delta H = -411 \text{ kJ} \cdot \text{mol}^{-1}$}
\item{$\rm{C + O_{2} \rightarrow CO_{2}}$}
\end{enumerate}

\item{For each of the following descriptions, say whether the process is endothermic or exothermic and give a reason for your answer.}
\begin{enumerate}
\item{evaporation}
\item{the combustion reaction in a car engine}
\item{bomb explosions}
\item{melting ice}
\item{digestion of food}
\item{condensation}
\end{enumerate}
\end{enumerate}
\practiceinfo

\begin{tabular}[h]{cccccc}
(1.) 00zc & (2.) 00zd & 
 \end{tabular}
}

\section{Spontaneous and non-spontaneous reactions}


\Activity{Demonstration}{Spontaneous and non-spontaneous reactions}{

\textbf{Apparatus and materials:}

A length of magnesium ribbon, thick copper wire and a Bunsen burner.

\scalebox{.8} % Change this value to rescale the drawing.
{
\begin{pspicture}(0,-4.051816)(6.8179855,4.0171604)
\psframe[linewidth=0.053000003,dimen=middle](3.5,-3.7753158)(1.0,-4.0253158)
\psframe[linewidth=0.053000003,dimen=middle](2.75,-1.7753159)(1.75,-2.7753158)
\psframe[linewidth=0.053000003,dimen=middle](2.5,0.22468415)(2.0,-1.7753159)
\psline[linewidth=0.053000003](2.5,-2.7753158)(2.5,-3.5253158)(3.5,-3.7753158)
\psline[linewidth=0.053000003](1.0,-3.7753158)(2.0,-3.5253158)(2.0,-3.2753158)
\psline[linewidth=0.053000003cm](0.0,-3.2753158)(2.0,-3.2753158)
\psline[linewidth=0.053000003cm](0.0,-3.0253158)(2.0,-3.0253158)
\psellipse[linewidth=0.053000003,dimen=middle](2.0,-3.1503158)(0.1,0.125)
\psframe[linewidth=0.053000003,dimen=middle,fillstyle=solid](2.0,-3.0253158)(0.0,-3.2753158)
\psline[linewidth=0.053000003cm](0.0,-3.2753158)(2.0,-3.2753158)
\psline[linewidth=0.053000003](0.0,-3.0253158)(2.0,-3.0253158)(2.0,-2.7753158)
\psbezier[linewidth=0.04](2.1167896,0.2643312)(1.5723941,0.5315021)(1.9143941,1.3061602)(1.9683942,1.3911602)(2.0223942,1.4761603)(2.2956908,2.0846148)(2.5640426,2.2308874)(2.8323941,2.3771603)(2.334416,1.9391412)(2.49385,1.5532057)(2.653284,1.1672703)(2.6611853,-0.002839753)(2.1167896,0.2643312)
\psbezier[linewidth=0.04](2.1171942,0.7171602)(1.8723942,0.057160247)(2.592394,0.072160244)(2.3835943,0.75466025)(2.1747942,1.4371603)(2.3619943,1.3771602)(2.1171942,0.7171602)
\psline[linewidth=0.04cm](2.4811106,0.70769924)(3.5611105,1.7876992)
\rput{-50.467102}(-2.0705352,2.8619668){\psframe[linewidth=0.04,dimen=outer](2.4011106,3.7676992)(1.6011106,3.4876995)}
\psline[linewidth=0.04cm](2.3011105,3.3676991)(2.4411106,3.187699)
\psline[linewidth=0.04cm](2.4411106,3.187699)(2.7011108,3.147699)
\psline[linewidth=0.04cm](2.7011108,3.147699)(3.3411105,2.5076993)
\psline[linewidth=0.04cm](2.1811104,3.2676992)(2.3011105,3.1276991)
\psline[linewidth=0.04cm](2.3011105,3.1276991)(2.3211105,2.947699)
\psline[linewidth=0.04cm](2.3211105,2.947699)(2.3411107,2.8876991)
\psline[linewidth=0.04cm](2.3411107,2.8876991)(2.8611104,2.2676992)
\psline[linewidth=0.04cm](2.8611104,2.2676992)(3.6211104,1.7676991)
\psline[linewidth=0.04cm](3.6211104,1.7676991)(3.8011105,1.9876994)
\psline[linewidth=0.04cm](3.8011105,1.9876994)(3.1811106,2.2676992)
\psline[linewidth=0.04cm](3.1811106,2.2676992)(2.4811106,2.9876995)
\psline[linewidth=0.04cm](2.4811106,2.9876995)(2.3611107,3.0476992)
\psline[linewidth=0.04cm](2.4811106,2.9876995)(2.6811106,2.947699)
\psline[linewidth=0.04cm](2.6811106,2.947699)(3.1611106,2.427699)
\psline[linewidth=0.04cm](3.4811106,1.4476992)(3.7011106,1.6676992)
\psline[linewidth=0.04cm](2.1211104,0.7876992)(3.2811105,1.9676993)
\psline[linewidth=0.04cm](3.4411106,2.1276991)(4.281111,3.0276992)
\psline[linewidth=0.04cm](4.281111,3.0276992)(4.4611106,2.8676991)
\psline[linewidth=0.04cm](4.4611106,2.8676991)(3.7211106,2.0476992)
\psline[linewidth=0.04cm](3.5011106,1.8076992)(2.4011104,0.74769926)
\psline[linewidth=0.04cm](2.4011104,0.74769926)(2.1011107,0.80769926)
\psline[linewidth=0.04cm](2.4011104,0.7276992)(2.5011106,0.7276992)
\psline[linewidth=0.04cm](3.7411106,2.0076993)(4.4811106,2.7876992)
\psline[linewidth=0.04cm](4.496177,2.7771604)(4.4161773,2.89716)
\psline[linewidth=0.04cm](3.6811106,1.6476992)(3.5011106,1.7276994)
\psline[linewidth=0.04cm](3.5161772,1.4571602)(3.336177,1.5571604)
\psline[linewidth=0.04cm](3.116177,1.7971601)(3.136177,2.0771604)
\psline[linewidth=0.04cm](3.136177,2.4571602)(3.1411104,2.3076992)
\psline[linewidth=0.04cm](3.3211105,2.5476992)(3.2961771,2.2171602)
\usefont{T1}{ptm}{m}{n}
\rput(6.005017,2.6526992){\large magnesium}
\usefont{T1}{ptm}{m}{n}
\rput(5.8214235,2.312699){\large ribbon}
\psline[linewidth=0.04cm](4.0411105,2.5676992)(5.0411105,2.5676992)
\end{pspicture} 
}


\textbf{Method:\\}

\begin{enumerate}
\item{Scrape the length of magnesium ribbon and copper wire clean.}
\item{Heat each piece of metal over the Bunsen burner, in a non-luminous flame. Do Not look directly at the flame. Observe whether any chemical reaction takes place.}
\item{Remove the metals from the flame and observe whether the reaction stops. If the reaction stops, return the metal to the Bunsen flame and continue to heat it.\\}
\end{enumerate}

\textbf{Results:\\}

\begin{itemize}
\item{Did any reaction take place before the metals were heated?}
\item{Did either of the reactions continue after they were removed from the flame?}
\item{Write a balanced equation for each of the chemical reactions that takes place.}
\end{itemize}
}

In the demonstration above, the reaction between magnesium and oxygen, and the reaction between copper and oxygen are both \textbf{non-spontaneous}. Before the metals were held over the Bunsen burner, no reaction was observed.  They need energy to \textit{initiate} the reaction. After the reaction has started, it may then carry on spontaneously. This is what happened when the magnesium reacted with oxygen. Even after the magnesium was removed from the flame, the reaction continued. Other reactions will not carry on unless there is a constant addition of energy. This was the case when copper reacted with oxygen. As soon as the copper was removed from the flame, the reaction stopped.\\

Now try carefully adding a solution of dilute sulphuric acid to a solution of sodium hydroxide. What do you observe? This is an example of a \textbf{spontaneous reaction} because the reaction takes place without any energy being added.

\Definition{Spontaneous reaction}{
A spontaneous reaction is a physical or chemical change that occurs without the addition of energy.
}


% CHILD SECTION END



% CHILD SECTION START

\section{Activation energy and the activated complex}
\label{sec:energychanges:activation}

From the demonstrations of spontaneous and non-spontaneous reactions, it should be clear that most reactions will not take place until the system has some minimum amount of energy added to it. This energy is called the \textbf{activation energy}. \textbf{Activation energy} is the 'threshold energy' or the energy that must be overcome in order for a chemical reaction to occur.

\Definition{Activation energy}{Activation energy or 'threshold energy' is the energy that must be overcome in order for a chemical reaction to occur.}

It is possible to draw an energy diagram to show the energy changes that take place during a particular reaction. Let's consider an example:

\begin{eqnarray*}
\text{H}_{2}\text{(g)} + \text{F}_{2}\text{(g)} \rightarrow 2\text{HF(g)}
\end{eqnarray*}

\begin{figure}[h]
\begin{center}
\begin{pspicture}(-1,-0.5)(10,5.5)
%  \psgrid(0,0)(0,0)(10,5.5)
\psline{->}(0,0)(10,0)
\psline{->}(0,0)(0,5.5)
\pscurve[showpoints=false](0,2.3)(1.5,2.3)(2,2.3)(2.4,2.36)
(4.25,5)(6.6,1.1)(7,1)(7.5,1)(9,1)
\psline[linestyle=dotted](0,2.3)(4.5,2.3)
\psline[linestyle=dotted](3.5,1)(9,1)
\psline[linestyle=dotted](4.1,5)(1.75,5)
\psline{<->}(4,2.3)(4,1)
\psline{<->}(2,5)(2,2.3)
\rput[bl](3.85,5.1){[H$_2$F$_2$] (\small activated complex)}
\rput[b](8,1.1){2HF}
\rput[t](8,0.9){\small products}
\rput[b](1,2.4){$\rm H_2+F_2$}
\rput[t](1,2.2){\small reactants}
\rput[rb](1.9,4){\small activation}
\rput[rt](1.7,3.9){\small energy}
\rput[rb](3.85,1.55){$\rm \Delta H=-268$}
\rput[rt](3.85,1.5){\small kJ.mol$^{-1}$}
\rput[r](4.45,-0.5){Time}
\psline{->}(4.5,-0.5)(5.3,-0.5)
\rput[r]{90}(-0.5,3.65){Potential energy}
\psline{->}(-0.5,3.75)(-0.5,4.75)
\end{pspicture}
\caption{The energy changes that take place during an exothermic reaction}
\label{fig:energychanges:exothermic}
\end{center}
\end{figure}

The reaction between $\text{H}_{2}\text{(g)}$ and $\text{F}_{2}\text{(g)}$ (figure \ref{fig:energychanges:exothermic}) needs energy in order to proceed, and this is the activation energy. Once the reaction has started, an in-between, temporary state is reached where the two reactants combine to give $\text{H}_{2}\text{F}_{2}$. This state is sometimes called a \textbf{transition state} and the energy that is needed to reach this state is equal to the activation energy for the reaction. The compound that is formed in this transition state is called the \textbf{activated complex}. The transition state lasts for only a very short time, after which either the original bonds reform, or the bonds are broken and a new product forms. In this example, the final product is HF and it has a lower energy than the reactants. The reaction is exothermic and $\Delta H$ is negative.

\Definition{Activated complex}{The activated complex is a transitional structure in a chemical reaction that results from the effective collisions between reactant molecules, and which remains while old bonds break and new bonds form.}

\Tip{Enzymes and activation energy\\

An enzyme is a catalyst that helps to speed up the rate of a reaction by lowering the activation energy of a reaction. There are many enzymes in the human body, without which lots of important reactions would never take place. Cellular respiration is one example of a reaction that is catalysed by enzymes. You will learn more about catalysts in Grade 12.
}

In endothermic reactions, the final products have a higher energy than the reactants. An energy diagram is shown below (figure \ref{fig:energychanges:endothermic}) for the endothermic reaction $\text{XY} + \text{Z} \rightarrow \text{X} + \text{YZ}$. In this example, the activated complex has the formula XYZ. Notice that the activation energy for the endothermic reaction is much greater than for the exothermic reaction.

\begin{figure}[h]
\begin{center}
\begin{pspicture}(-1,0)(10,6)
% \psgrid(0,0)(0,0)(10,5.5)
\psline{->}(0,0)(10,0)
\psline{->}(0,0)(0,5.5)
\pscurve[showpoints=false](0,1)(1.5,1)(1.75,1)(2,1)(2.4,1.1)
(4.5,5)(6.6,2.36)(7,2.3)(7.25,2.3)(7.5,2.3)(9,2.3)
\psline[linestyle=dotted](0,1)(5,1)
\psline[linestyle=dotted](4,2.3)(9,2.3)
\psline{<->}(4.5,2.3)(4.5,1)
\rput[b](4.5,5.1){[XYZ]}
\rput[b](8,2.4){$\rm X+YZ$}
\rput[t](8,2.2){\small products}
\rput[b](1,1.1){$\rm XY+Z$}
\rput[t](1,0.9){\small reactants}
\rput[lb](4.65,1.55){$\rm \Delta H>0$}
\rput[r](4.45,-0.5){Time}
\psline{->}(4.5,-0.5)(5.3,-0.5)
\rput[r]{90}(-0.5,3.65){Potential energy}
\psline{->}(-0.5,3.75)(-0.5,4.75)
\psline{<->}(2,1)(2,5)
\rput(1,3){\small activation}
\rput(1,2.6){\small energy}
\end{pspicture}
\end{center}
\caption{The energy changes that take place during an endothermic reaction}
\label{fig:energychanges:endothermic}
\end{figure}

\begin{IFact}{
The reaction between H and F was considered by NASA (National Aeronautics and Space Administration) as a fuel system for rocket boosters because of the energy that is released during this exothermic reaction.
}
\end{IFact}



\Exercise{Energy and reactions\\}{

\begin{enumerate}
\item{
Carbon reacts with water according to the following equation:

\begin{center}
$\rm{C + H_{2}O \rightarrow CO + H_{2}}$ $\Delta$H $>$ 0
\end{center}

\begin{enumerate}
\item{Is this reaction endothermic or exothermic?}
\item{Give a reason for your answer.}
\end{enumerate}
}

\item{
Refer to the graph below and then answer the questions that follow:

\begin{center}
\begin{pspicture}(-1,-1)(10,6)
% \psgrid(0,0)(0,0)(10,5.5)
\psline{->}(0,0)(10,0)
\psline{->}(0,0)(0,5.5)
\pscurve[showpoints=false](0,1)(1.5,1)(1.75,1)(2,1)(2.4,1.1)
(4.5,5)(6.6,2.36)(7,2.3)(7.25,2.3)(7.5,2.3)(9,2.3)
\psline[linestyle=dotted](0,1)(5,1)
\psline[linestyle=dotted](4,2.3)(9,2.3)
\rput[r](4.45,-0.5){Time}
\psline{->}(4.5,-0.5)(5.3,-0.5)
\rput[r]{90}(-0.5,3.65){Potential energy}
\psline{->}(-0.5,3.75)(-0.5,4.75)
\end{pspicture}
\end{center}

\begin{enumerate}
\item{What is the energy of the reactants?}
\item{What is the energy of the products?}
\item{Calculate $\Delta$H.}
\item{What is the activation energy for this reaction?}
\end{enumerate}
}

\end{enumerate}
\practiceinfo

\begin{tabular}[h]{cccccc}
(1.) 00ze & (2.) 00zf & 
 \end{tabular}
}


\summary{VPjlm}

\begin{itemize}
\item{When a reaction occurs, some bonds break and new bonds form. These changes involve \textbf{energy}.}
\item{When bonds break, energy is \textbf{absorbed} and when new bonds form, energy is \textbf{released}.}
\item{The \textbf{bond energy} is the amount of energy that is needed to break the chemical bond between two atoms.}
\item{If the energy that is needed to \textit{break} the bonds is greater than the energy that is released when new bonds \textit{form}, then the reaction is \textbf{endothermic}. The energy of the product is greater than the energy of the reactants.}
\item{If the energy that is needed to \textit{break} the bonds is less than the energy that is released when new bonds \textit{form}, then the reaction is \textbf{exothermic}. The energy of the product is less than the energy of the reactants.}
\item{An endothermic reaction is one that \textbf{absorbs} energy in the form of heat, while an exothermic reaction is one that \textbf{releases} energy in the form of heat and light.}
\item{The difference in energy between the reactants and the product is called the \textbf{heat of reaction} and has the symbol $\Delta$H.}
\item{In an endothermic reaction, $\Delta$H is a positive number, and in an exothermic reaction, $\Delta$H will be negative.}
\item{Photosynthesis, evaporation and the thermal decomposition of limestone, are all examples of endothermic reactions.}
\item{Combustion reactions and respiration are both examples of exothermic reactions.}
\item{A reaction which proceeds without additional energy being added, is called a \textbf{spontaneous reaction}.}
\item{Reactions where energy must be \textit{continuously supplied} for the reaction to continue, are called \textbf{non-spontaneous} reactions.}
\item{In any reaction, some minimum energy must be overcome before the reaction will proceed. This is called the \textbf{activation energy} of the reaction.}
\item{The \textbf{activated complex} is the transitional product that is formed during a chemical reaction while old bonds break and new bonds form.}
\end{itemize}

\begin{eocexercises}{}

\begin{enumerate}
\item{For each of the following, say whether the statement is \textbf{true} or \textbf{false}. If it is false, give a reason for your answer.
\begin{enumerate}
\item{Energy is released in all chemical reactions.}
\item{The condensation of water vapour is an example of an endothermic reaction.}
\item{In an exothermic reaction $\Delta$H is less than zero.}
\item{All non-spontaneous reactions are endothermic.}
\end{enumerate}
}

\item{For the following reaction:

\begin{center}
$\rm{A + B \rightarrow AB}$ $\Delta H = -129$ kJ.mol$^{-1}$
\end{center}
}
\begin{enumerate}
\item{The energy of the reactants is less than the energy of the product.}
\item{The energy of the product is less than the energy of the reactants.}
\item{The reaction is non-spontaneous.}
\item{The overall energy of the system increases during the reaction.}
\end{enumerate}

\item{Consider the following chemical equilibrium:

\begin{center}
$\rm{2NO_{2} \rightleftharpoons N_{2}O_{4}}$  $\Delta H < 0$
\end{center}

Which one of the following graphs best represents the changes in potential energy that take place during the production of N$_{2}$O$_{4}$?}\\

\scalebox{.8}{
\begin{pspicture}(0,-2.5748436)(15.06,2.5348437)
\psline[linewidth=0.04cm](0.02,2.5148437)(0.02,-0.48515624)
\psline[linewidth=0.04cm](0.02,-0.48515624)(3.02,-0.48515624)
\psline[linewidth=0.04cm](4.02,2.5148437)(4.02,-0.48515624)
\psline[linewidth=0.04cm](4.02,-0.48515624)(7.02,-0.48515624)
\psline[linewidth=0.04cm](12.04,2.4748437)(12.04,-0.52515626)
\psline[linewidth=0.04cm](12.04,-0.52515626)(15.04,-0.52515626)
\psline[linewidth=0.04cm](8.04,2.4948437)(8.04,-0.5051563)
\psline[linewidth=0.04cm](8.04,-0.5051563)(11.04,-0.5051563)
\psline[linewidth=0.04cm](0.0,0.21484375)(0.62,0.21484375)
\psline[linewidth=0.04cm](2.14,1.5948437)(2.74,1.5948437)
\psline[linewidth=0.04cm](4.0,0.59484375)(4.64,0.59484375)
\psline[linewidth=0.04cm](6.16,-0.08515625)(6.9,-0.08515625)
\psline[linewidth=0.04cm](8.04,0.51484376)(8.78,0.51484376)
\psline[linewidth=0.04cm](10.06,0.51484376)(10.88,0.51484376)
\psline[linewidth=0.04cm](14.52,0.53484374)(15.02,0.53484374)
\psbezier[linewidth=0.04](0.62,0.21484375)(0.98,0.25484374)(0.98,2.2748437)(1.52,2.2948437)(2.06,2.3148437)(1.88,1.5348438)(2.16,1.5948437)
\psbezier[linewidth=0.04](4.62,0.59484375)(4.9,0.6148437)(5.06,1.2548437)(5.48,1.1948438)(5.9,1.1348437)(5.96,-0.14515625)(6.18,-0.08515625)
\psbezier[linewidth=0.04](8.76,0.51484376)(9.0,0.51484376)(9.12,1.4748437)(9.5,1.4748437)(9.88,1.4748437)(9.84,0.47484374)(10.1,0.51484376)
\psbezier[linewidth=0.04](12.04,2.0348437)(12.34,2.0748436)(13.26,1.6348437)(13.38,1.4748437)(13.5,1.3148438)(14.08,0.51484376)(14.52,0.53484374)
%\usefont{T1}{ptm}{m}{n}
\rput(13.539687,-0.86015624){\small (iv)}
%\usefont{T1}{ptm}{m}{n}
\rput(9.389688,-0.80015624){\small (iii)}
%\usefont{T1}{ptm}{m}{n}
\rput(5.4396877,-0.86015624){\small (ii)}
%\usefont{T1}{ptm}{m}{n}
\rput(1.5496875,-0.84015626){\small (i)}
%%\usefont{T1}{ptm}{m}{n}
\end{pspicture}}

\item{The cellular respiration reaction is catalysed by enzymes. The equation for the reaction is:

\begin{center}
$\rm{C_{6}H_{12}O_{6} + 6O_{2} \rightarrow 6CO_{2} + 6H_{2}O}$
\end{center}

The change in potential energy during this reaction is shown below:

\begin{center}
\begin{pspicture}(-1,-0.5)(10,5.5)
%  \psgrid(0,0)(0,0)(10,5.5)
\psline{->}(0,0)(10,0)
\psline{->}(0,0)(0,5.5)
\pscurve[showpoints=false](0,2.3)(1.5,2.3)(2,2.3)(2.4,2.36)
(4.25,5)(6.6,1.1)(7,1)(7.5,1)(9,1)
\psline[linestyle=dotted](0,2.3)(4.5,2.3)
\psline[linestyle=dotted](3.5,1)(9,1)
\psline[linestyle=dotted](4.1,5)(1.75,5)
\psline{<->}(4,2.3)(4,1)
\psline{<->}(2,5)(2,2.3)


\rput[t](8.1,0.9){\small 6CO$_{2}$ + 6H$_{2}$O}

\rput[t](1,2.2){\small C$_{6}$H$_{12}$O$_{6}$ + 6O$_{2}$}
\rput[rb](1.9,4){\small activation}
\rput[rt](1.7,3.9){\small energy}
\rput[rb](3.85,1.55){$\Delta$H}

\rput[r](4.45,-0.5){Time}
\psline{->}(4.5,-0.5)(5.3,-0.5)
\rput[r]{90}(-0.5,3.65){Potential energy}
\psline{->}(-0.5,3.75)(-0.5,4.75)
\end{pspicture}
\end{center}

\begin{enumerate}
\item{Will the value of $\Delta$H be positive or negative? Give a reason for your answer.}
\item{Explain what is meant by 'activation energy'.}
\item{What role do enzymes play in this reaction?}
\item{Glucose is one of the reactants in cellular respiration. What important chemical reaction produces glucose?}
\item{Is the reaction in your answer above an endothermic or an exothermic one? Explain your answer.}
\item{Explain why proper nutrition and regular exercise are important in maintaining a healthy body.}
\end{enumerate}



}
\end{enumerate}

\practiceinfo

\begin{tabular}[h]{cccccc}
(1.) 00zg & (2.) 00zh & (3.) 00zi & (4.) 01y7 &
 \end{tabular}
\end{eocexercises}


% CHILD SECTION END



% CHILD SECTION END



% CHILD SECTION START

