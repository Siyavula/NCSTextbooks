\chapter{Intermolecular Forces}
\label{chap:intermolecular}
In the previous chapter, we discussed the different forces that exist \textit{between atoms} (intramolecular forces). When atoms are joined to one another they form molecules, and these molecules in turn have forces that bind them together. These forces are known as \textbf{intermolecular forces}, and we are going to look at them in more detail in this next section.\\
\chapterstartvid{VPhgb}
\Definition{Intermolecular forces}{Intermolecular forces are forces that act between stable molecules.}

You will also recall from the previous chapter, that we can describe molecules as being either \textbf{polar} or \textbf{non-polar}. A polar molecule is one in which there is a difference in electronegativity between the atoms in the molecule, such that the shared electron pair spends more time close to the atom that attracts it more strongly. The result is that one end of the molecule will have a slightly positive charge ($\delta^{+}$), and the other end will have a slightly negative charge ($\delta^{+}$). The molecule is said to be a \textbf{dipole}. However, it is important to remember that just because the bonds within a molecule are polar, the molecule itself may not necessarily be polar. The shape of the molecule may also affect its polarity. A few examples are shown in table \ref{tab:molecule polarity examples} to refresh your memory!

\begin{table}[H]
\begin{center}
\caption{Polarity in molecules with different atomic bonds and molecular shapes}
\label{tab:molecule polarity examples}
\begin{tabular}{|p{2.5cm}|p{1.5cm}|p{1.5cm}|p{4cm}|p{2cm}|}\hline
\textbf{Molecule} & \textbf{Chemical formula} & \textbf{Bond between atoms} & \textbf{Shape of molecule} & \textbf{Polarity of molecule} \\\hline
Hydrogen & H$_{2}$ & Covalent &
\begin{center}
\begin{pspicture}(1,1.8)(5,2.2)
%\psgrid[gridcolor=lightgray]
\rput(3,2.5){Linear molecule}
\rput(2,2){\textbf{H}}
\rput(4,2){\textbf{H}}
\psline(2.5,2)(3.5,2)
\end{pspicture}
\end{center}
& Non-polar \\\hline

Hydrogen chloride & HCl & Polar covalent &
\begin{center}
\begin{pspicture}(1,1.8)(5,2.2)
%\psgrid[gridcolor=lightgray]
\rput(3,2.5){Linear molecule}
\rput(2,2){\textbf{H$^{\delta^{+}}$}}
\rput(4,2){\textbf{Cl$^{\delta^{-}}$}}
\psline(2.5,2)(3.5,2)
\end{pspicture}
\end{center}
& Polar \\\hline

Carbon tetrafluoride & CF$_{4}$ & Polar covalent &
\begin{center}
\scalebox{.8}{
\begin{pspicture}(0,1)(5,4.3)
%\psgrid[gridcolor=lightgray]
\rput(2.5,4.7){Tetrahedral molecule}
\rput(2.5,2.5){\textbf{$C^{\delta^{+}}$}}
\psline(2.5,3)(2.5,4)
\psline(3,2.5)(4,2.5)
\psline(2.5,2)(2.5,1)
\psline(2,2.5)(1,2.5)
\rput(2.5,4.2){\textbf{$F^{\delta^{-}}$}}
\rput(4.4,2.5){\textbf{$F^{\delta^{-}}$}}
\rput(2.5,0.7){\textbf{$F^{\delta^{-}}$}}
\rput(0.7,2.5){\textbf{$F^{\delta^{-}}$}}
\end{pspicture}}
\end{center}
& Non-polar \\\hline
\end{tabular}
\end{center}
\end{table}

\section{Types of Intermolecular Forces}
\label{sec:intermolecular:types}

It is important to be able to recognise whether the molecules in a substance are polar or non-polar because this will determine what type of intermolecular forces there are. This is important in explaining the properties of the substance.

\begin{enumerate}
\item{\textbf{Van der Waals forces}}

These intermolecular forces are named after a Dutch physicist called Johannes van der Waals (1837 -1923), who recognised that there were weak attractive and repulsive forces between the molecules of a gas, and that these forces caused gases to deviate from 'ideal gas' behaviour. Van der Waals forces are \textit{weak} intermolecular forces, and can be divided into three types:

\begin{enumerate}
\item{\textit{Dipole-dipole forces}

Figure \ref{fig:dipole} shows a simplified dipole molecule, with one end slightly positive and the other slightly negative.

\begin{figure}[H]
\begin{center}
\scalebox{.8}{
\begin{pspicture}(0,1)(4,3)
%\psgrid[gridcolor=lightgray]
\psellipse(2,2)(2.5,1)
\rput(0,2){\textbf{$\delta^{+}$}}
\rput(4,2){\textbf{$\delta^{-}$}}
\end{pspicture}}
\caption{A simplified diagram of a dipole molecule}
\label{fig:dipole}
\end{center}
\end{figure}

When one dipole molecule comes into contact with another dipole molecule, the positive pole of the one molecule will be attracted to the negative pole of the other, and the molecules will be held together in this way (figure \ref{fig:dipole-dipole}). Examples of materials/substances that are held together by dipole-dipole forces are HCl, FeS, KBr, SO$_{2}$ and NO$_{2}$.

\begin{figure}[H]
\begin{center}
\scalebox{.8}{
\begin{pspicture}(0,0)(9,4)
%\psgrid[gridcolor=lightgray]
\psellipse(2,2)(2.5,1)
\rput(0,2){\textbf{$\delta^{+}$}}
\rput(4,2){\textbf{$\delta^{-}$}}
\psellipse(7,2)(2.5,1)
\rput(5,2){\textbf{$\delta^{+}$}}
\rput(9,2){\textbf{$\delta^{-}$}}
\end{pspicture}}
\end{center}
\caption{Two dipole molecules are held together by the attractive force between their oppositely charged poles}
\label{fig:dipole-dipole}
\end{figure}
}

\item{\textit{Ion-dipole forces}

As the name suggests, this type of intermolecular force exists between an ion and a dipole molecule. You will remember that an \textit{ion} is a charged atom, and this will be attracted to one of the charged ends of the polar molecule. A positive ion will be attracted to the negative pole of the polar molecule, while a negative ion will be attracted to the positive pole of the polar molecule. This can be seen when sodium chloride (NaCl) dissolves in water. The positive sodium ion (Na$^{+}$) will be attracted to the slightly negative oxygen atoms in the water molecule, while the negative chloride ion (Cl$^{-}$) is attracted to the slightly positive hydrogen atom. These intermolecular forces weaken the ionic bonds between the sodium and chloride ions so that the sodium chloride dissolves in the water (figure \ref{fig:ion-dipole}).

\begin{figure}[H]
\begin{center}
\scalebox{.8}{
\begin{pspicture}(-4,-4)(4,4)
%\psgrid[gridcolor=lightgray]
\psellipse(0,0)(0.5,0.5)
\psellipse(1.5,0)(1,0.5)
\psellipse(0,-1.5)(0.5,1)
\psellipse(-1.5,0)(1,0.5)
\psellipse(0,1.5)(0.5,1)
\psellipse(-3,0)(0.5,0.5)
\psellipse(0,3)(0.5,0.5)
\psellipse(3,0)(0.5,0.5)
\psellipse(0,-3)(0.5,0.5)

\rput(0,0){\textbf{$\mbox{Na}^{+}$}}
\rput(3,0){\textbf{$\mbox{Cl}^{-}$}}
\rput(-3,0){\textbf{$\mbox{Cl}^{-}$}}
\rput(0,3){\textbf{$\mbox{Cl}^{-}$}}
\rput(0,-3){\textbf{$\mbox{Cl}^{-}$}}

\rput(1.5,0){\textbf{$\mbox{H}_{2}\mbox{O}$}}
\rput(-1.5,0){\textbf{$\mbox{H}_{2}\mbox{O}$}}
\rput(0,1.5){\textbf{$\mbox{H}_{2}\mbox{O}$}}
\rput(0,-1.5){\textbf{$\mbox{H}_{2}\mbox{O}$}}

\rput(0.8,0){$\delta^{-}$}
\rput(-0.8,0){$\delta^{-}$}
\rput(0,0.8){$\delta^{-}$}
\rput(0,-0.8){$\delta^{-}$}

\rput(2.2,0){$\delta^{+}$}
\rput(-2.2,0){$\delta^{+}$}
\rput(0,2.2){$\delta^{+}$}
\rput(0,-2.2){$\delta^{+}$}
\end{pspicture}}
\caption{Ion-dipole forces in a sodium chloride solution}
\label{fig:ion-dipole}
\end{center}
\end{figure}
}

\item{\textit{London forces}}

These intermolecular forces are also sometimes called 'dipole- induced dipole' or 'momentary dipole' forces. Not all molecules are polar, and yet we know that there are also intermolecular forces between non-polar molecules such as carbon dioxide. In non-polar molecules the electronic charge is evenly distributed but it is possible that at a particular moment in time, the electrons might not be evenly distributed. The molecule will have a \textit{temporary dipole}. In other words, each end of the molecules has a slight charge, either positive or negative. When this happens, molecules that are next to each other attract each other very weakly. These London forces are found in the halogens (e.g. F$_{2}$ and I$_{2}$), the noble gases (e.g. Ne and Ar) and in other non-polar molecules such as carbon dioxide and methane.

\end{enumerate}

\item{\textbf{Hydrogen bonds}

As the name implies, this type of intermolecular bond involves a hydrogen atom. The hydrogen must be attached to another atom that is strongly electronegative, such as oxygen, nitrogen or fluorine. Water molecules for example, are held together by hydrogen bonds between the hydrogen atom of one molecule and the oxygen atom of another (figure \ref{fig:hydrogen bonds}). Hydrogen bonds are stronger than van der Waals forces. It is important to note however, that both van der Waals forces and hydrogen bonds are weaker than the covalent and ionic bonds that exist between \textit{atoms}.

\begin{figure}[H]
\begin{center}
\begin{pspicture}(-1,-1)(8,4)
\SpecialCoor
%\psgrid%[gridcolor=lightgray]

\def\water{\psset{unit=0.25}
\pscircle(0,0){2}
\rput{150}{\psarc[fillcolor=white,fillstyle=solid](-1.5,1){1.5}{30}{260}
\psarc[fillcolor=white,fillstyle=solid](1.5,1){1.5}{280}{150}
\rput(-1.5,1){\pscurve(1.5;30)(-1;142.5)(1.5;260)}
\rput(1.5,1){\pscurve(1.5;150)(-1;37.5)(1.5;280)}}\psset{unit=1}}

\def\h20{
\pnode(1;217.5){RO}\pnode(0,0){H}\pnode(1;322.5){LO}
\psline(RO)(H)
\psline(LO)(H)
\rput*(H){O}
\rput*(LO){H}
\rput*(RO){H}}

\pnode(0,0){a}
\pnode(1,2){b}
\pnode(2,0){c}
\pnode(3,2){d}
\rput(a){\water}
\rput(b){\water}
\rput(c){\water}
\rput(d){\water}

\psline[linestyle=dashed](a)(0.5,2)
\psline[linestyle=dashed](b)(1.5,0)
\psline[linestyle=dashed](c)(2.5,2)

\psline[linestyle=dashed](5,0)(5,1)
\psline[linestyle=dashed](5.8,1.6)(5.8,2.6)

\rput(5,0){\h20}
\rput(5.8,1.6){\h20}
\rput(6.6,3.2){\h20}

\psline[linestyle=dashed](-1,3.4)(0,3.4)
\uput[r](0,3.4){hydrogen bonds}
\psline[linestyle=solid](-1,3.0)(0,3.0)
\uput[r](0,3.0){atomic bonds}
\psline(0.6,2.1)(1,2.2)
\psline(1.2,2.1)(1.2,1.6)
\psline(2.6,2.1)(3,2.2)
\psline(3.2,2)(3.2,1.6)
\psline(-0.4,0)(0,0.2)
\psline(0.2,0)(0.3,-0.4)
\psline(1.6,0.2)(2,0.2)
\psline(2.2,0)(2.3,-0.5)
\end{pspicture}
\caption{Two representations showing the hydrogen bonds between water molecules: space-filling model and structural formula.}
\label{fig:hydrogen bonds}
\end{center}
\end{figure}
}
\end{enumerate}

\Exercise{Types of intermolecular forces}{

\begin{enumerate}
\item{Complete the following table by placing a tick to show which type of intermolecular force occurs in each substance:

\begin{tabular}{|l|p{1.8cm}|p{1.8cm}|p{1.8cm}|p{1.8cm}|}\hline
\textbf{Formula} & \textbf{Dipole-dipole} & \textbf{Momentary dipole} & \textbf{Ion-dipole} & \textbf{hydrogen bond} \\\hline
HCl & & & & \\\hline
CO$_{2}$ & & & & \\\hline
I$_{2}$ & & & & \\\hline
H$_{2}$O & & & & \\\hline
KI(aq) & & & & \\\hline
NH$_{3}$ & & & & \\\hline
\end{tabular}
In which of the substances are the intermolecular forces:
\begin{enumerate}
\item{strongest}
\item{weakest}
\end{enumerate}}
\end{enumerate}
\practiceinfo

\begin{tabular}[h]{cccccc}
(1.) 00wf &  & 
 \end{tabular}
}

\section{Understanding intermolecular forces}
\label{sec:intermolecular:understanding}

The types of intermolecular forces that occur in a substance will affect its properties, such as its \textbf{phase}, \textbf{melting point} and \textbf{boiling point}. You should remember, if you think back to the kinetic theory of matter, that the \textit{phase} of a substance is determined by how strong the forces are between its particles. The weaker the forces, the more likely the substance is to exist as a gas because the particles are able to move far apart since they are not held together very strongly. If the forces are very strong, the particles are held closely together in a solid structure. Remember also that the \textit{temperature} of a material affects the energy of its particles. The more energy the particles have, the more likely they are to be able to overcome the forces that are holding them together. This can cause a change in phase.
\vspace{-.5cm}
\Definition{Boiling point}{The temperature at which a material will change from a liquid to a gas.}
\Definition{Melting point}{The temperature at which a material will change from a solid to a liquid.}
Now look at the data in table \ref{tab:intermolecular:mpbp}.

\begin{table}[h]
\begin{center}
\begin{tabular}{|c|c|c|c|}\hline
\textbf{Formula} & \textbf{Formula mass} & \textbf{Melting point} ($^{\circ}$C) & \textbf{Boiling point} ($^{\circ}$C) at 1 atm \\\hline
He & 4 & -270 & -269 \\\hline
Ne & 20 & -249 & -246 \\\hline
Ar & 40 & -189 & -186 \\\hline
F$_{2}$ & 38 & -220 & -188 \\\hline
Cl$_{2}$ & 71 & -101 & -35 \\\hline
Br$_{2}$ & 160 & -7 & 58 \\\hline
NH$_{3}$ & 17 & -78 & -33 \\\hline
H$_{2}$O & 18 & 0 & 100 \\\hline
HF & 20 & -83 & 20 \\\hline
\end{tabular}
\caption{Melting point and boiling point of a number of chemical substances}
\label{tab:intermolecular:mpbp}
\end{center}
\end{table}

The melting point and boiling point of a substance, give us information about the \textit{phase} of the substance at room temperature, and the \textit{strength of the intermolecular forces}. The examples below will help to explain this.

\textbf{Example 1:} Fluorine (F$_{2}$)\\

\textit{Phase at room temperature}\\

Fluorine (F$_{2}$) has a melting point of $-220^{\circ}$C and a boiling point of $-188^{\circ}$C. This means that for any temperature that is greater than $-188^{\circ}$C, fluorine will be a gas. At temperatures below $-220^{\circ}$C, fluorine would be a solid, and at any temperature in between these two, fluorine will be a liquid. So, at room temperature ($25^{\circ}$), fluorine exists as a gas.\\

\textit{Strength of intermolecular forces}\\

What does this information tell us about the intermolecular forces in fluorine? In fluorine, these forces must be very weak for it to exist as a gas at room temperature. Only at temperatures below $-188^{\circ}$C will the molecules have a low enough energy that they will come close enough to each other for forces of attraction to act between the molecules. The intermolecular forces in fluorine are very weak \textbf{van der Waals} forces because the molecules are \textit{non-polar}.\\

\textbf{Example 2:} Hydrogen fluoride (HF)\\

\textit{Phase at room temperature}\\

For temperatures below $-83^{\circ}$C, hydrogen fluoride is a solid. Between $-83^{\circ}$C and $20^{\circ}$C, it exists as a liquid, and if the temperature is increased above $20^{\circ}$C, it will become a gas. So on a cool day, a sample of hydrogen fluoride is a liquid and on a warm day it will be a gas!\\

\textit{Strength of intermolecular forces}\\

What does this tell us about the intermolecular forces in hydrogen fluoride? The forces are stronger than those in fluorine, because more energy is needed for hydrogen fluoride to change into the gaseous phase. In other words, more energy is needed for the intermolecular forces to be overcome so that the molecules can move further apart. Intermolecular forces will exist between the hydrogen atom of one molecule of hydrogen fluoride and the fluorine atom of another molecule of hydrogen fluoride. These are \textbf{hydrogen bonds}, which are stronger than van der Waals forces.\\

\textbf{Now look at water.} What do you notice? Luckily for us, water behaves quite differently from the other molecules shown in the table. Imagine if water were like ammonia (NH$_{3}$), which is a gas above a temperature of $-33^{\circ} \text{C}$! There would be no liquid water on the planet, and that would mean that no life would be able to survive here. The hydrogen bonds in water are particularly strong and this gives water unique qualities when compared to other molecules with hydrogen bonds. This will be discussed more in chapter \ref{chap:globalcycles}. You should also note that the strength of the intermolecular forces increases with an increase in formula mass. This can be seen by the increasing melting and boiling points of substances as formula mass increases.

\Exercise{Applying your knowledge of intermolecular forces}{
Refer to the data in table \ref{tab:intermolecular:mpbp} and then use your knowledge of different types of intermolecular forces to explain the following statements: \vspace{-.5cm}
\begin{enumerate}[noitemsep]
\item{The boiling point of F$_{2}$ is much lower than the boiling point of NH$_{3}$}
\item{At room temperature, many elements exist naturally as gases}
\item{The boiling point of HF is higher than the boiling point of Cl$_{2}$}
\item{The boiling point of water is much higher than HF, even though they both contain hydrogen bonds}
\end{enumerate}
% Automatically inserted shortcodes - number to insert 4
 \practiceinfo
\begin{tabular}[h]{cccccc}
% Question 1
(1.)	01ht	&
\end{tabular}
% Automatically inserted shortcodes - number inserted 4
}

\section{Intermolecular forces in liquids}
\label{sec:intermolecular:liquids}

Intermolecular forces affect a number of properties in liquids. We will use the example of water to explain this.

\begin{itemize}
\item{\textbf{Surface tension}

You may have noticed how some insects are able to walk across a water surface, and how leaves float in water. This is because of surface tension. In water, each molecule is held to the surrounding molecules by strong hydrogen bonds. Molecules in the centre of the liquid are completely surrounded by other molecules, so these forces are exerted in all directions. However, molecules at the surface do not have any water molecules above them to pull them upwards. Because they are only pulled sideways and downwards, the water molecules at the surface are held more closely together. This is called \textbf{surface tension}.

\begin{figure}[!h]
\begin{center}
\begin{pspicture}(0,0)(3.2,4)
%\psgrid[gridcolor=gray]
\psset{unit=2}
\psline[fillstyle=solid,fillcolor=lightgray,linearc=7pt](0,1)(0,0)(1.5,0)(1.5,1)
\rput(0,0.81){\psline[xunit=1,fillstyle=solid,fillcolor=white,linearc=7pt,linestyle=none](0,1)(0,0)(1.5,0)(1.5,1)}

\multirput(0.15,0.1)(0.2,0){7}{\pscircle(0,0){0.1}}
\multirput(0.25,0.3)(0.2,0){6}{\pscircle(0,0){0.1}}
\multirput(0.15,0.5)(0.2,0){7}{\pscircle(0,0){0.1}}
\multirput(0.25,0.7)(0.2,0){6}{\pscircle(0,0){0.1}}
\rput(0,0){\beaker}

\rput(0.45,0.3){\degrees[1.2]
\multido{\n=0+0.2}{6}{
\rput{\n}{\psline{<->}(0,0)(0.25,0)}
}}

\rput(1.05,0.3){\degrees[1.2]
\multido{\n=0+0.2}{6}{
\rput{\n}{\psline{<->}(0,0)(0.25,0)}
}}
\multirput(0.2,0.7)(0.4,0){3}{\psline{<->}(0,0)(0.25,0)}
\psline{<-}(1.05,0.3)(2,0.3)
\uput[r](2,0.3){\parbox{3.1cm}{For molecules in the centre of the liquid, the intermolecular forces act in all directions.}}
\psline{<-}(0.15,0.7)(-.3,0.7)
\uput[l](-.3,0.7){\parbox{3cm}{For molecules at the surface there are no upward forces, so the molecules are closer together.}}
\end{pspicture}
\end{center}
\caption{Surface tension in a liquid}
\end{figure}

\item{\textbf{Capillarity}}

\Activity{Investigation}{Capillarity\\}{
Half fill a beaker with water and hold a hollow glass tube in the centre as shown below. Mark the level of the water in the glass tube, and look carefully at the shape of the air-water interface in the tube. What do you notice?

\begin{center}
\begin{pspicture}(0,0)(3.2,5.4)
%\psgrid[gridcolor=gray]
\psset{unit=2}
\psline[fillstyle=solid,fillcolor=lightgray,linearc=7pt](0,1)(0,0)(1.5,0)(1.5,1)
\rput(0,0.7){\psline[xunit=0.367,fillstyle=solid,fillcolor=white,linearc=7pt,linestyle=none](0,1)(0,0)(1.5,0)(1.5,1)}
\rput(0.95,0.7){\psline[xunit=0.367,fillstyle=solid,fillcolor=white,linearc=7pt,linestyle=none](0,1)(0,0)(1.5,0)(1.5,1)}
\psframe[fillstyle=solid,fillcolor=lightgray,linestyle=none](0.55,1)(0.95,1.2)
\rput(0.55,1){\psline[xunit=0.267,fillstyle=solid,fillcolor=white,linearc=7pt,linestyle=none](0,1)(0,0)(1.5,0)(1.5,1)}
\rput(0,0){\beaker}
\rput(0.55,0.5){
\psarc(0.2,2){0.2}{0}{180}
\psline(0,0)(0,2)
\psline(0.4,0)(0.4,2)}
\uput[ur](0,0){water}
\psline{<-}(0.75,1)(2,1)
\uput[r](2,1){meniscus}
\end{pspicture}
\end{center}

At the air-water interface, you will notice a \textbf{meniscus}, where the water appears to dip in the centre. In the glass tube, the attractive forces between the glass and the water are stronger than the intermolecular forces between the water molecules. This causes the water to be held more closely to the glass, and a meniscus forms. The forces between the glass and the water also mean that the water can be 'pulled up' higher when it is in the tube than when it is in the beaker. Capillarity is the surface tension that occurs in liquids that are inside narrow tubes.
}}

\item{\textbf{Evaporation}
\begin{IFact}{\textbf{Transpiration in plants} - Did you know that plants also 'sweat'? In plants, this is called \textit{transpiration}, and a plant will lose water through pores in the leaf surface called \textit{stomata}. Although this water loss is important in the survival of a plant, if a plant loses too much water, it will die. Plants that live in very hot, dry places such as deserts, must be specially adapted to reduce the amount of water that transpires (evaporates) from their leaf surface. Desert plants have some amazing adaptations to deal with this problem! Some have hairs on their leaves, which reflect sunlight so that the temperature is not as high as it would be, while others have a thin waxy layer covering their leaves, which reduces water loss. Some plants are even able to close their stomata during the day when temperatures (and therefore transpiration) are highest.}
\end{IFact}

In a liquid, each particle has kinetic energy, but some particles will have more energy than others. We therefore refer to the \textit{average} kinetic energy of the molecules when we describe the liquid. When the liquid is heated, those particles which have the highest energy will be able to overcome the intermolecular forces holding them in the liquid phase, and will become a gas. This is called \textbf{evaporation}. Evaporation occurs when a liquid changes to a gas. The stronger the intermolecular forces in a liquid, the higher the temperature of the molecules will have to be for it to become a gas. You should note that a liquid doesn't necessarily have to reach boiling point before evaporation can occur. Evaporation can take place all the time. You will see this if you leave a glass of water outside in the sun. Over a period of time, the water level slowly drops.

What happens then to the molecules of water that remain in the liquid? Remember that it was the molecules with the highest energy that left the liquid. This means that the average kinetic energy of the remaining molecules will decrease, and so will the \textit{temperature} of the liquid.\\

A similar process takes place when a person sweats during exercise. When you exercise, your body temperature increases and you begin to release moisture (sweat) through the pores in your skin. The sweat quickly evaporates and causes the temperature of your skin to drop. This helps to keep your body temperature at a level that is suitable for it to function properly.}
\end{itemize}
In the same way that intermolecular forces affect the properties of liquids, they also affect the properties of solids. For example, the stronger the intermolecular forces between the particles that make up the solid, the \textit{harder} the solid is likely to be, and the higher its \textit{melting point} is likely to be.

\summary{VPhgi}

\begin{itemize}
\item{\textbf{Intermolecular forces} are the forces that act between stable molecules.}
\item{The \textbf{type} of intermolecular force in a substance, will depend on the \textbf{nature of the molecules}.}
\item{\textbf{Polar molecules} have an unequal distribution of charge, meaning that one part of the molecule is slightly positive and the other part is slightly negative. \textbf{Non-polar molecules} have an equal distribution of charge.}
\item{There are three types of \textbf{Van der Waal's forces}. These are dipole-dipole, ion-dipole and London forces (momentary dipole).}
\item{\textbf{Dipole-dipole} forces exist between two \textbf{polar molecules}, for example between two molecules of hydrogen chloride.}
\item{\textbf{Ion-dipole} forces exist between \textbf{ions and dipole molecules}. The ion is attracted to the part of the molecule that has an opposite charge to its own. An example of this is when an ionic solid such as sodium chloride dissolves in water.}
\item{\textbf{Momentary dipole} forces occur between two \textbf{non-polar molecules}, where at some point there is an unequal distribution of charge in the molecule. For example, there are London forces between two molecules of carbon dioxide.}
\item{\textbf{Hydrogen bonds} occur between \textbf{hydrogen atoms} and other \textbf{atoms that have a high electronegativity} such as oxygen, nitrogen and fluorine. The hydrogen atom in one molecule will be attracted to the nitrogen atom in another molecule, for example. There are hydrogen bonds between water molecules and between ammonia molecules.}
\item{Intermolecular forces affect the \textbf{properties} of substances. For example, the stronger the intermolecular forces, the higher the \textbf{melting point} of that substance, and the more likely that substance is to exist as a solid or liquid. Its \textbf{boiling point} will also be higher.}
\item{In \textbf{liquids}, properties such as \textbf{surface tension}, \textbf{capillarity} and \textbf{evaporation} are the result of intermolecular forces.}
\end{itemize}

\begin{eocexercises}{}
\begin{enumerate}
\item{Give one word or term for each of the following descriptions:}
\begin{enumerate}
\item{The tendency of an atom in a molecule to attract bonding electrons.}
\item{A molecule that has an unequal distribution of charge.}
\item{A charged atom.}
\end{enumerate}

\item{The following table gives the melting points of various hydrides:}
\begin{center}
\begin{tabular}{|c|c|}\hline
\textbf{Hydride} & \textbf{Melting point} ($^{\circ}$C) \\\hline
HI & $-50$ \\\hline
NH$_{3}$ & $-78$ \\\hline
H$_{2}$S & $-83$ \\\hline
CH$_{4}$ & $-184$ \\\hline
\end{tabular}
\end{center}

In which of these hydrides does hydrogen bonding occur?
\begin{enumerate}
\item{HI only}
\item{NH$_{3}$ only}
\item{HI and NH$_{3}$ only}
\item{HI, NH$_{3}$ and H$_{2}$S}
\end{enumerate}
(IEB Paper 2, 2003)

\item{Refer to the list of substances below:}
\begin{center}
HCl, Cl$_{2}$, H$_{2}$O, NH$_{3}$, N$_{2}$, HF
\end{center}
Select the true statement from the list below:
\begin{enumerate}
\item{NH$_{3}$ is a non-polar molecule}
\item{The melting point of NH$_{3}$ will be higher than for Cl$_{2}$}
\item{Ion-dipole forces exist between molecules of HF}
\item{At room temperature N$_{2}$ is usually a liquid}
\end{enumerate}

\item{The respective boiling points for four chemical substances are given below:

Hydrogen sulphide $-60^{\circ}$C

Ammonia $-33^{\circ}$C

Hydrogen fluoride $20^{\circ}$C

Water $100^{\circ}$C

\begin{enumerate}
\item{Which one of the substances exhibits the strongest forces of attraction between its molecules in the liquid state?}
\item{Give the name of the force responsible for the relatively high boiling points of ammonia and water and explain how this force originates.}
\item{The shapes of the molecules of hydrogen sulfide and water are similar, yet their boiling points differ. Explain.}
\end{enumerate}

}
(IEB Paper 2, 2002)
\end{enumerate}

\practiceinfo

\begin{tabular}[h]{cccccc}
(1.) 00wh & (2.) 00wi & (3.) 00wj & (4.) aaa
 \end{tabular}
\end{eocexercises}


% CHILD SECTION END



% CHILD SECTION START

