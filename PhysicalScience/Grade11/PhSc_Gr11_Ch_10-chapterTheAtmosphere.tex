\chapter{The Atmosphere}
\label{chap:atmosphere}

Our Earth is truly an amazing planet! Not only is it exactly the right distance from the sun to have temperatures that will support life, but it is also one of the only planets in our solar system to have liquid water on its surface. In addition, our Earth has an atmosphere that has just the right composition to allow life to exist. The \textbf{atmosphere} is the layer of gases that surrounds the Earth. We may not always be aware of them, but without these gases, life on Earth would definitely not be possible. The atmosphere provides the gases that animals and plants need for respiration (breathing) and photosynthesis (the production of food), it helps to keep temperatures on Earth constant and also protects us from the sun's harmful radiation. \\

In this chapter, we are going to take a closer look at the chemistry of the Earth's atmosphere and at some of the human activities that threaten the delicate balance that exists in this part of our planet.

\chapterstartvideo{VPkdh}
\section{The composition of the atmosphere}
\label{sec:atmos:comp}

Earth's atmosphere is made up of a mixture of gases. Two important gases are nitrogen and oxygen, which make up about 78.1\% and 20.9\% of the atmosphere respectively. A third gas, argon, contributes about 0.9\%, and a number of other gases such as carbon dioxide, methane, water vapour, helium and ozone make up the remaining 0.1\%. In an earlier chapter, we discussed the importance of nitrogen as a component of proteins, the building blocks of life. Similarly, oxygen is essential for life because it is the gas we need for respiration. We will discuss the importance of some of the other gases later in this chapter.

\IFact{
The Earth's early atmosphere was very different from what it is today. When the Earth formed around 4.5 billion years ago, there was probably no atmosphere. Some scientists believe that the earliest atmosphere contained gases such as water vapour, carbon dioxide, nitrogen and sulphur which were released from inside the planet as a result of volcanic activity.}
%  Many scientists also believe that the first stage in the evolution of life, around 4 billion years ago, needed an oxygen-free environment. At a later stage, these primitive forms of plant life began to release small amounts of oxygen into the atmosphere as a product of photosynthesis. During photosynthesis, plants use carbon dioxide, water and sunlight to produce simple sugars. Oxygen is also released in the process.
% \begin{center}
% $\rm{6CO_{2} + 6H_{2}O}$ + sunlight \rm${\rightarrow C_{6}H_{12}O_{6} + 6O_{2}}$
% \end{center}
% This build-up of oxygen in the atmosphere eventually led to the formation of the ozone layer, which helped to filter the sun's harmful UV radiation so that plants were able to flourish in different environments. As plants became more widespread and photosythesis increased, so did the production of oxygen. The increase in the amount of oxygen in the atmosphere allowed more forms of life to develope on Earth.

If you have ever climbed to a very high altitude (altitude means the 'height' in the atmosphere), you will have noticed that it becomes very difficult to breathe, and many climbers suffer from 'altitude sickness' before they reach their destination. This is because the density of gases becomes less as you move higher in the atmosphere. It is \textbf{gravity} that holds the atmosphere close to the Earth. As you move higher, this force weakens slightly and so the gas particles become more spread out. In effect, when you are at a high altitude, the gases in the atmosphere haven't changed, but there are fewer oxygen molecules in the same amount of air that you are able to breathe.

\Definition{Earth's atmosphere}{
The Earth's atmosphere is a layer of gases that surround the planet, and which are held there by the Earth's gravity.} 
The atmosphere contains roughly 78.1\% nitrogen, 20.9\% oxygen, 0.9\% argon, 0.038\% carbon dioxide, trace amounts of other gases, and a variable amount of water vapour. This mixture of gases is commonly known as air. The atmosphere protects life on Earth by absorbing ultraviolet solar radiation and reducing temperature extremes between day and night.


\section{The structure of the atmosphere}
\label{sec:atmos:structure}

The Earth's atmosphere is divided into different layers, each with its own particular characteristics (Figure~\ref{fig:atmos:structure}).\\

\begin{figure}[H]
\begin{center}
\scalebox{.8}{
\begin{pspicture}(0,-0.5)(13,13)
%\psgrid[gridcolor=lightgray]
\rput(1,0){\psline(0,0)(0,12)
\psline(0,0)(12,0)
\rput(-0.25,0){0}
\rput(-0.25,1){10} \rput(-0.25,2){20} \rput(-0.25,3){30} \rput(-0.25,4){40} \rput(-0.25,5){50} \rput(-0.25,6){60} \rput(-0.25,7){70} \rput(-0.25,8){80} \rput(-0.25,9){90} \rput(-0.25,10){100} \rput(-0.25,11){110} \rput(-0.25,12){120}
\rput(0,-0.25){-100} \rput(1,-0.25){-90} \rput(2,-0.25){-80} \rput(3,-0.25){-70} \rput(4,-0.25){-60} \rput(5,-0.25){-50} \rput(6,-0.25){-40} \rput(7,-0.25){-30} \rput(8,-0.25){-20} \rput(9,-0.25){-10} \rput(10,-0.25){0} \rput(11,-0.25){10} \rput(12,-0.25){20}
\rput(-2,6){\textbf{Height (km)}}
\rput(6,-0.6){\textbf{Temperature ($\degree$C)}}
\psline(11.5,0)(5,1)
\psline(5,1)(5,2)
\psline(5,2)(6,3.5)
\psline(6,3.5)(10,4.5)
\psline(10,4.5)(10,5.2)
\psline(10,5.2)(8,6.2)
\psline(8,6.2)(1,8)
\psline(1,8)(1,9)
\psline(1,9)(3,10)
\psline(3,10)(7,11)
\psline(7,11)(8,11.1)
\psline{<->}(12,0.1)(12,2)
\psline{<->}(12,2.1)(12,5)
\psline{<->}(12,5.1)(12,9)
\psline{<->}(12,9.1)(12,11)
\rput(13.5,1){Troposphere}
\rput(13.5,3.5){Stratosphere}
\rput(13.5,6.5){Mesosphere}
\rput(13.5,10){Thermosphere}}
\end{pspicture}}
\end{center}
\caption{A generalised diagram showing the structure of the atmosphere and the changing temperatures up to a height of 110 km}
\label{fig:atmos:structure}
\end{figure}

\subsection{The troposphere}

The \textbf{troposphere} is the lowest level in the atmosphere, and it is the part in which we live. The troposphere varies in thickness, and extends from the ground to a height of about 7 km at the poles and about 18 km at the equator. An important characteristic of the troposphere is that its temperature \textit{decreases} with an increase in altitude. In other words, as you climb higher, it will get colder. 
You will have noticed this if you have climbed a mountain, or if you have moved from a city at a high altitude to one which is lower; the average temperature is often lower where the altitude is higher. 
\IFact{Aeroplanes fly just above the troposphere to avoid all this turbulence. If you have ever been in an aeroplane and have looked out the window once you are well into the flight, you will have noticed that you are actually flying above the level of the clouds. Clouds and weather occur in the troposphere, whereas the stratosphere has very stable atmospheric conditions and very little turbulence.}
This is because the troposphere is heated from the 'bottom up'. In other words, places that are closer to sea level will be warmer than those at higher altitudes. The heating of the atmosphere will be discussed in more detail later in this chapter.\\

The word troposphere comes from the Greek \textit{tropos}, meaning \textit{turning} or \textit{mixing}. The troposphere is the most turbulent (or agitated) part of the atmosphere and is the part where our \textbf{weather} takes place. Weather is the state of the air at a particular place and time e.g.\@{} if it is warm or cold, wet or dry, and how cloudy or windy it is. 

\subsection{The stratosphere}

Above the troposphere is another layer called the \textbf{stratosphere}. The stratosphere extends from altitudes of 18 to 50 km.  \\

The stratosphere is different from the troposphere because its temperature \textit{increases} as altitude increases. This is because the stratosphere absorbs solar radiation directly, meaning that the upper layers closer to the sun will be warmer. The upper layers of the stratosphere are also warmer because of the presence of the \textbf{ozone layer}. Ozone (O$_{3}$) is formed when solar radiation splits an oxygen molecule (O$_{2}$) into two atoms of oxygen. Each individual atom is then able to combine with an oxygen molecule to form ozone. The two reactions are shown below:


\begin{equation*}
\text{O}_{2} \rightarrow \text{O} + \text{O}
\end{equation*}

\begin{equation*}
\text{O} + \text{O}_{2} \rightarrow \text{O}_{3}
\end{equation*}



The change from one type of molecule to another produces energy, and this contributes to higher temperatures in the upper part of the stratosphere. An important function of the ozone layer is to absorb UV radiation and reduce the amount of harmful radiation that reaches the Earth's surface.
\Extension{CFCs and the ozone layer}{You may have heard people talking about 'the hole in the ozone layer'. What do they mean by this and do we need to be worried about it? \\
Most of the Earth's ozone is found in the stratosphere and this limits the amount of UV radiation that reaches the Earth. However, human activities have once again disrupted the chemistry of the atmosphere. Chlorofluorocarbons (CFCs) are compounds found in aerosol cans, fridges and airconditioners. In aerosol cans, it is the CFCs that cause the substance within the can to be sprayed outwards. The negative side of CFCs is that, when they are released into the atmosphere, they break down ozone molecules so that the ozone is no longer able to protect us as much from UV rays. The 'ozone hole' is actually a thinning of the ozone layer approximately above Antarctica. Let's take a closer look at the chemical reactions that are involved in breaking down ozone:
\begin{enumerate}
\item{When CFCs react with UV radiation, a carbon-chlorine bond in the chlorofluorocarbon breaks and a new compound is formed, with a chlorine atom.
\begin{equation*}
\text{CFCl}_{3} + \text{UV light} \rightarrow \text{CFCl}_{2}^{-} + \text{Cl}^{+}
\end{equation*}
}
\item{The single chlorine atom reacts with ozone to form a molecule of chlorine monoxide and oxygen gas. In the process, ozone is destroyed.
\begin{equation*}
\text{Cl}^{-} + \text{O}_{3} \rightarrow \text{ClO} + \text{O}_{2}
\end{equation*}
}
\item{The chlorine monoxide then reacts with a free oxygen atom (UV radiation breaks O$_{2}$ down into single oxygen atoms) to form oxygen gas and a single chlorine atom.
\begin{equation*}
\text{ClO} + \text{O} \rightarrow \text{Cl} + \text{O}_{2}
\end{equation*}
}
\item{The chlorine atom is then free to attack more ozone molecules, and the process continues. A single CFC molecule can destroy 100 000 ozone molecules.}
\end{enumerate}
One observed consequence of ozone depletion is an increase in the incidence of skin cancer in affected areas because there is more UV radiation reaching Earth's surface. CFC replacements are now being used to reduce emissions, and scientists are trying to find ways to restore ozone levels in the atmosphere.}

\subsection{The mesosphere}

The mesosphere is located about 50-80 km above Earth's surface. Within this layer, temperature decreases with increasing altitude. Temperatures in the upper mesosphere can fall as low as -100$\degree$C in some areas. Millions of meteors burn up daily in the mesosphere because of collisions with the gas particles that are present in this layer. This leads to a high concentration of iron and other metal atoms.

\subsection{The thermosphere}

The thermosphere exists at altitudes above 80 km. In this part of the atmosphere, UV and shorter X-Ray radiation from the sun cause neutral gas atoms to be \textit{ionised}. At these radiation frequencies, photons from the solar radiation are able to dislodge electrons from neutral atoms and molecules during a collision. A \textit{plasma} is formed, which consists of negative free electrons and positive ions. The part of the atmosphere that is ionised by solar radiation is called the \textbf{ionosphere}. At the same time that ionisation takes place however, an opposing process called recombination also begins. Some of the free electrons are drawn to the positive ions, and combine again with them if they are in close enough contact. Since the gas density increases at lower altitudes, the recombination process occurs more often here because the gas molecules and ions are closer together. The ionisation process produces energy which means that the upper parts of the thermosphere, which are dominated by ionisation, have a higher temperature than the lower layers where recombination takes place. Overall, temperature in the thermosphere increases with an increase in altitude.\\
\IFact{The ionosphere is also home to the \textbf{aurorae}. Aurorae are caused by the collision of charged particles (e.g.\@{} electrons) with atoms in the Earth's upper atmosphere. Charged particles are energised and so, when they collide with atoms, the atoms also become energised. Shortly afterwards, the atoms emit the energy they have gained, as light. Often these emissions are from oxygen atoms, resulting in a greenish glow (wavelength 557.7 nm) and, at lower energy levels or higher altitudes, a dark red glow (wavelength 630 nm). Many other colours can also be observed. For example, emissions from atomic nitrogen are blue, and emissions from molecular nitrogen are purple. Aurorae emit visible light (as described above), and also infra-red, ultraviolet and x-rays, which can be observed with special instruments.}
\Extension{The ionosphere and radio waves}{The ionosphere is of practical importance because it allows \textbf{radio waves} to be transmitted. A radio wave is a type of electromagnetic radiation that humans use to transmit information without wires. When using high-frequency bands, the ionosphere is used to reflect the transmitted radio beam. When a radio wave reaches the ionosphere, the electric field in the wave causes the electrons in the ionosphere to start oscillating at the same frequency as the radio wave. Some of the radio wave energy is given up to this mechanical oscillation. The oscillating electron will then either recombine with a positive ion, or will re-radiate the original wave energy back downward again. The beam returns to the Earth's surface, and may then be reflected back into the ionosphere for a second time.}


\Exercise{The composition of the atmosphere\\}{
\begin{enumerate}
\item{Complete the following summary table by providing the missing information for each layer in the atmosphere.}

\begin{tabular}{|p{2cm}|p{2cm}|p{2.5cm}|p{3cm}|}\hline
\textbf{Atmospheric layer} & \textbf{Height (km)} & \textbf{Gas composition} & \textbf{General characteristics} \\\hline
Troposphere & 0-18 & & Turbulent; part of atmosphere where weather occurs \\\hline
& & & Ozone reduces harmful radiation reaching Earth \\\hline
Mesosphere & & & High concentration of metal atoms\\\hline
& more than 80 km & & \\\hline
\end{tabular}

\item{
Use your knowledge of the atmosphere to explain the following statements:
\begin{enumerate}
\item{Athletes who live in coastal areas need to acclimatise (adjust to higher altitude) if they are competing at high altitudes.}
\item{Higher incidences of skin cancer have been recorded in areas where the ozone layer in the atmosphere is thin.}
\item{During a flight, turbulence generally decreases above a certain altitude.}
\end{enumerate}
}

\end{enumerate}
\practiceinfo

\begin{tabular}[h]{cccccc}
(1.) 00mt & (2.) 00mu & 
 \end{tabular}
}

\section{Greenhouse gases and global warming}
\label{sec:atmosphere:greenhouse}

\subsection{The heating of the atmosphere}

As we mentioned earlier, the distance of the Earth from the sun is not the only reason that temperatures on Earth are within a range that is suitable to support life. The composition of the atmosphere is also critically important. \\

The Earth receives electromagnetic energy from the sun in the \textit{visible spectrum}. There are also small amounts of infrared and ultraviolet radiation in this incoming solar energy. Most of the radiation is \textit{shortwave} radiation, and it passes easily through the atmosphere towards the Earth's surface, with some being reflected before reaching the surface. At the surface, some of the energy is absorbed, and this heats up the Earth's surface. But the situation is a little more complex than this. \\

A large amount of the sun's energy is re-radiated from the surface back into the atmosphere as \textbf{infrared} radiation, which is invisible to humans. As this radiation passes through the atmosphere, some of it is absorbed by \textbf{greenhouse gases} such as carbon dioxide, water vapour and methane. These gases are very important because they re-emit the energy back towards the surface. By doing this, they help to warm the lower layers of the atmosphere even further. It is this 're-emission' of heat by greenhouse gases, combined with surface heating and other processes (e.g.\@{} conduction and convection) that maintain temperatures at exactly the right level to support life. Without the presence of greenhouse gases, most of the sun's energy would be lost and the Earth would be a lot colder than it is! A simplified diagram of the heating of the atmosphere is shown in Figure~\ref{fig:heating the Earth}.


\begin{figure}[H]
\begin{center}
\begin{pspicture}(-5,-5)(5,5)
%psgrid

%the sun
\psarc(-5,5){3.0}{270}{0}
\rput[l](-4,4){sun}
%the Earth
\psarc(6,-6){8.0}{100}{165}
\rput[l](2,-2){Earth's surface}
%the atmosphere
\psarc(6,-6){6.5}{100}{165}
\rput[l](4.4,1){atmosphere}
%solar radiation
\rput[l]{-45}(-3,3){
\psCoil[coilaspect=2,coilheight=1.33,coilwidth=.5]{60}{3220}
\pscurve{->}(5.9,0.22)(6,0.22)(6.1,0.07)(6.25,0)
}
\rput[l](-4.2,1.2){\begin{tabular}{l}Incoming \\short-wave \\solar radiation\end{tabular}}

%infrared
\rput[l]{-45}(-0.2,1.6){
\psCoil[coilaspect=2,coilheight=2.5,coilwidth=.5]{-180}{945}
\psline{->}(-0.6,-0.25)(-0.8,-0.25)
}
\rput[l](-0.6,2.2){\begin{tabular}{l}Outgoing long-wave \\infrared radiation  \end{tabular}}

%reflected 1
\rput{-20}(-0.2,-1.6){
\psCoil[coilaspect=2,coilheight=2.5,coilwidth=.5]{-135}{360}
}
\rput{290}(-0.9,-1.7){
\psCoil[coilaspect=2,coilheight=2.5,coilwidth=.5]{0}{620}
\pscurve{->}(2.1,-0.05)(2.2,0)(2.25,0)
}
\rput[l](-4,-1.9){\begin{tabular}{l}infrared radiation \\is absorbed and \\re-emitted by \\greenhouse gases \\in the atmosphere\end{tabular}}

\rput{-40}(5,2.3){
\rput{-20}(-0.2,-1.6){
\psCoil[coilaspect=2,coilheight=2.5,coilwidth=.5]{-135}{330}
}
\rput{290}(-0.9,-1.7){
\psCoil[coilaspect=2,coilheight=2.5,coilwidth=.5]{0}{640}
}
}
\pscurve{->}(4.4,0.4)(4.45,0.35)(4.5,0.3)

\end{pspicture}
\caption{The heating of the Earth's atmosphere}
\label{fig:heating the Earth}
\end{center}
\end{figure}


\subsection{The greenhouse gases and global warming}

Many of the greenhouse gases occur naturally in small quantities in the atmosphere. However, human activities have greatly increased their concentration, and this has led to a lot of concern about the impact that this could have in \textit{increasing} global temperatures. This phenomenon is known as \textbf{global warming}. Because the natural concentrations of these gases are low, even a small increase in their concentration as a result of human emissions, could have a big effect on temperature. But before we go on, let's look at where some of these human gas emissions come from.

\begin{itemize}

\item{\textbf{Carbon dioxide} (CO$_{2}$)

Carbon dioxide enters the atmosphere through the burning of fossil fuels (oil, natural gas, and coal), solid waste, trees and wood products, and also as a result of other chemical reactions (e.g.\@{} the manufacture of cement). Carbon dioxide can also be \textit{removed} from the atmosphere when it is absorbed by plants during photosynthesis.}

\item{\textbf{Methane} (CH$_{4}$)

Methane is emitted when coal, natural gas and oil are produced and transported. Methane emissions can also come from livestock and other agricultural practises and from the decay of organic waste.}

\item{\textbf{Nitrous oxide} (N$_{2}$O)

Nitrous oxide is emitted by agriculture and industry, and when fossil fuels and solid waste are burned.}

\item{\textbf{Fluorinated gases} (e.g.\@{} hydrofluorocarbons, perfluorocarbons, and sulphur hexafluoride)

These gases are all \textit{synthetic}, in other words they are man-made. They are emitted from a variety of industrial processes. Fluorinated gases are sometimes used in the place of other ozone-depleting substances (e.g.\@{} CFCs). These are very powerful greenhouse gases, and are sometimes referred to as High Global Warming Potential gases ('High GWP gases').}

\end{itemize}

\textbf{Overpopulation} is a major problem in reducing greenhouse gas emissions, and in slowing down global warming. As populations grow, their demands on resources (e.g.\@{} energy) increase, and so does their production of greenhouse gases.\\

\Extension{\large{Ice core drilling - Taking a look at Earth's past climate}}{

Global warming is a very controversial issue. While many people are convinced that the increase in average global temperatures is directly related to the increase in atmospheric concentrations of carbon dioxide, others argue that the climatic changes we are seeing are part of a natural pattern. One way in which scientists are able to understand what is happening at present, is to understand the Earth's \textit{past} atmosphere, and the factors that affected its temperature.\\

So how, you may be asking, do we know what the Earth's \textit{past} climate was like? One method that is used is \textbf{ice core drilling}. Antarctica is the coldest continent on Earth, and because of this there is very little melting of ice that takes place. Over thousands of years, ice has accumulated in layers and has become more and more compacted as new ice is added. This is partly why Antarctica is also on average one of the \textit{highest} continents! On average, the ice sheet that covers Antarctica is 2500 m thick, and at its deepest location, is 4700 m thick.\\

As the snow is deposited on top of the ice sheet each year, it traps different chemicals and impurities which are dissolved in the ice. The ice and impurities hold information about the Earth's environment and climate at the time that the ice was deposited. Drilling an ice core from the surface down, is like taking a journey back in time. The deeper into the ice you venture, the older the layer of ice. By analysing the gases and oxygen isotopes that are present (along with many other techniques) in the ice at various points in the Earth's history, scientists can start to piece together a picture of what the Earth's climate must have been like.


\scalebox{0.8}{
\begin{minipage}{\textwidth}
\begin{center}
\begin{pspicture}(-3,-1)(2,7)
%\psgrid[gridcolor=lightgray]
\psline(-2,0)(-2,5)
\psline(1,0)(1,5)
\psellipse(-0.5,5)(1.5,0.5)
\psellipse(-0.5,0)(1.5,0.5)
\psline(-0.5,3.5)(2,3.5)
\psline(-0.5,1)(2,1)
\rput(3.8,3.5){Top layers are the most}
\rput(3.8,3.2){recently deposited}
\rput(3.8,1){Bottom layers are}
\rput(3.8,0.7){the oldest}
\psline[arrows=<-](-2.5,0)(-2.5,5)
\rput(-4,2.5){Increasing age}
\end{pspicture}
\end{center}
\end{minipage}
}


One of the most well known ice cores was the one drilled at a Russian station called \textbf{Vostok} in central Antarctica. So far, data has been gathered for dates as far back as 160 000 years!
}

\Activity{Case Study}{Looking at past climatic trends\\}{

\textit{Make sure that you have read the 'extension box' on ice core drilling before you try this activity.}\\

The values in the table below were extrapolated from data obtained by scientists studying the Vostok ice core. 'Local temperature change' means by how much the temperature at that time was different from what it is today. For example, if the local temperature change 160 000 years ago was -9$\degree$C, this means that atmospheric temperatures at that time were 9$\degree$C \textit{lower} than what they are today. 'ppm' means 'parts per million' and is a unit of measurement for gas concentrations.\\

\begin{center}
\begin{tabular}{|p{3cm}|p{3cm}|p{1.5cm}|}\hline
\textbf{Years before present ($\times$ 1000)} & \textbf{Local temperature change ($\degree$C)} & \textbf{Carbon dioxide (ppm)}\\\hline
160 & -9 & 190 \\\hline
150 & -10 & 205 \\\hline
140 & -10 & 240 \\\hline
130 & -3 & 280 \\\hline
120 & +1 & 278 \\\hline
110 & -4 & 240 \\\hline
100 & -8  & 225 \\\hline
90 & -5  & 230 \\\hline
80 & -6 & 220 \\\hline
70 & -8 & 250 \\\hline
60 & -9 & 190 \\\hline
50 & -7 & 220 \\\hline
40 & -8 & 180 \\\hline
30 & -7 & 225 \\\hline
20 & -9 & 200 \\\hline
10 & -2 & 260 \\\hline
0 (1850) & -0.5 & 280 \\\hline
Present &  & 371 \\\hline
\end{tabular}
\end{center}

\textbf{Questions}

\begin{enumerate}
\item{On the same set of axes, draw graphs to show how temperature and carbon dioxide concentrations have changed over the last 160 000 years. Hint: 'Years before present' will go on the x-axis, and should be given \textit{negative} values.}
\item{Compare the graphs that you have drawn. What do you notice?}
\item{Is there a relationship between temperature and the atmospheric concentration of carbon dioxide?}
\item{Do these graphs \textit{prove} that temperature changes are determined by the concentration of gases such as carbon dioxide in the atmosphere? Explain your answer.}
\item{What other factors might you need to consider when analysing climatic trends?}
\end{enumerate}
}
\vfill
\clearpage


\subsection{The consequences of global warming}

\Activity{Group Discussion}{The impacts of global warming\\}{

\textit{In groups of 3-4, read the following extracts and then answer the questions that follow.}\\

\begin{quote}
\textbf{By 2050 Warming to Doom Million Species, Study Says}

By 2050, rising temperatures exacerbated by human-induced belches of carbon dioxide and other greenhouse gases could send more than a million of Earth's land-dwelling plants and animals down the road to extinction, according to a recent study. "Climate change now represents at least as great a threat to the number of species surviving on Earth as habitat-destruction and modification," said Chris Thomas, a conservation biologist at the University of Leeds in the United Kingdom.

The researchers worked independently in six biodiversity-rich regions around the world, from Australia to South Africa, plugging field data on species distribution and regional climate into computer models that simulated the ways species' ranges are expected to move in response to temperature and climate changes. According to the researchers' collective results, the predicted range of climate change by 2050 will place 15 to 35 percent of the 1 103 species studied at risk of extinction.

\textit{National Geographic News, 12 July 2004}
\end{quote}


\begin{quote}
\textbf{Global Warming May Dry Up Africa's Rivers, Study Suggests}

Many climate scientists already predict that less rain will fall annually in parts of Africa within 50 years due to global warming. Now new research suggests that even a small decrease in rainfall on the continent could cause a drastic reduction in river water, the lifeblood for rural populations in Africa.

A decrease in water availability could occur across about 25 percent of the continent, according to the new study. Hardest hit would be areas in northwestern and southern Africa, with some of the most serious effects striking large areas of Botswana and South Africa.

To predict future rainfall, the scientists compared 21 of what they consider to be the best climate change models developed by research teams around the world. On average, the models forecast a 10 to 20\% drop in rainfall in northwestern and southern Africa by 2070. With a 20\% decrease, Cape Town would be left with just 42\% of its river water, and "Botswana would completely dry up," de Wit said. In parts of northern Africa, river water levels would drop below 50\%.

Less river water would have serious implications not just for people but for the many animal species whose habitats rely on regular water supplies.

\textit{National Geographic News, 3 March 2006}
\end{quote}

\textbf{Discussion questions}

\begin{enumerate}
\item{What is meant by 'biodiversity'?}
\item{Explain why global warming is likely to cause a \textit{loss of biodiversity}.}
\item{Why do you think a loss of biodiversity is of such concern to conservationists?}
\item{Suggest some plant or animal species in South Africa that you think might be particularly vulnerable to extinction if temperatures were to rise significantly. Explain why you chose these species.}
\item{In what way do people, animals and plants rely on river water?}
\item{What effect do you think a 50\% drop in river water level in some parts of Africa would have on the people living in these countries?}
\item{Discuss some of the other likely impacts of global warming that we can expect (e.g.\@{} sea level rise, melting of polar ice caps, changes in ocean currents).}
\end{enumerate}
}

\subsection{Taking action to combat global warming}

Global warming is a major concern at present. A number of organisations, panels and research bodies have been working to gather accurate and relevant information so that a true picture of our current situation can be painted. One important organisation that you may have heard of is the \textbf{Intergovernmental Panel on Climate Change} (IPCC). The IPCC was established in 1988 by two United Nations organisations, the World Meteorological Organisation (WMO) and the United Nations Environment Programme (UNEP), to evaluate the risk of climate change brought on by humans. You may also have heard of the \textbf{Kyoto Protocol}, which will be discussed a little later.\\

\Activity{Group Discussion}{World carbon dioxide emissions\\}{The data in the table below shows carbon dioxide emissions from the consumption of fossil fuels (in million metric tons of carbon dioxide).\\

\begin{center}
\begin{tabular}{|l|c|c|c|c|c|c|}\hline
\textbf{Region or Country} & \textbf{1980} & \textbf{1985} & \textbf{1990} & \textbf{1995} & \textbf{2000} & \textbf{2004}\\\hline
United States & 4754 & 4585 & 5013 & 5292 & 5815 & 5912 \\\hline
Brazil & 186 & 187 & 222 & 288 & 345 & 336 \\\hline
France & 487 & 394 & 368 & 372 & 399 & 405 \\\hline
UK & 608 & 588 & 598 & 555 & 551 & 579 \\\hline
Saudi Arabia & 175 & 179 & 207 & 233 & 288 & 365 \\\hline
Botswana & 1.26	& 1.45 & 2.68 & 3.44 & 4.16 & 3.83 \\\hline
South Africa & 234 & 298 & 295 & 344 & 378 & 429 \\\hline
India & 299 & 439 & 588 & 867 & 1000 & 1112\\\hline
World Total & 18333 & 19412 & 21426 & 22033 & 23851 & 27043\\\hline
\end{tabular}
\end{center}

\textbf{Questions}

\begin{enumerate}
\item{Redraw the table and use a coloured pen to highlight those countries that are 'developed' and those that are 'developing'.}
\item{Explain why CO$_{2}$ emissions are so much higher in developed countries than in developing countries.}
\item{How does South Africa compare to the other developing countries, and also to the developed countries?}
\end{enumerate}

Carbon dioxide emissions are a major problem worldwide. The \textbf{Kyoto Protocol} was signed in Kyoto, Japan in December 1997. Its main objective was to reduce global greenhouse gas emissions by encouraging countries to become signatories to the guidelines that had been laid out in the protocol. These guidelines set targets for the world's major producers to reduce their emissions within a certain time. However, some of the worst contributors to greenhouse gas emissions (e.g.\@{} USA) were not prepared to sign the protocol, partly because of the potential effect this would have on the country's economy, which relies on industry and other 'high emission' activities.\\

\textbf{Panel discussion}\\

Form groups with 5 people in each. Each person in the group must adopt one of the following roles during the discussion:
\begin{itemize}
\item{the owner of a large industry}
\item{an environmental scientist}
\item{an economist}
\item{a politician}
\item{a chairperson for the discussion}
\end{itemize}

In your group, you are going to discuss some of the economic and environmental implications for a country that decides to  sign the Kyoto Protocol. Each person will have the opportunity to express the view of the character they have adopted. You may ask questions of the other people, or challenge their ideas, provided that you ask permission from the chairperson first.
}

\summary{VPkds}

\begin{itemize}[noitemsep]
\item{The \textbf{atmosphere} is the layer of gases that the surrounds Earth. These gases are important in sustaining life, regulating temperature and protecting the Earth from harmful radiation.}
\item{The gases that make up the atmosphere are nitrogen, oxygen, carbon dioxide, argon and others e.g.\@{} water vapour, methane.}
\item{There are four layer in the atmosphere, each with their own characteristics.}
\item{The \textbf{troposphere} is the lowest layer and here, temperature decreases with an increase in altitude. The troposphere is where weather occurs.}
\item{The next layer is the \textbf{stratosphere} where temperature increases with an increase in altitude because of the presence of ozone in this layer, and the direct heating from the sun.}
\item{The depletion of the ozone layer is largely because of CFCs, which break down ozone through a series of chemical reactions.}
\item{The \textbf{mesosphere} is characterised by very cold temperatures and meteor collisions. The mesosphere contains high concentrations of metal atoms.}
\item{In the \textbf{thermosphere}, neutral atoms are ionised by UV and X-ray radiation from the sun. Temperature increases with an increase in altitude because of the energy that is released during this ionisation process, which occurs mostly in the upper thermosphere. }
\item{The thermosphere is also known as the \textbf{ionosphere}, and is the part of the atmosphere where radio waves can be transmitted.}
\item{The \textbf{auroras} are bright coloured skies that occur when charged particles collide with atoms in the upper atmosphere. Depending on the type of atom, energy is released as light at different wavelengths.}
\item{The Earth is heated by radiation from the sun. Incoming radiation has a short wavelength and some is absorbed directly by the Earth's surface. However, a large amount of energy is re-radiated as longwave infrared radiation.}
\item{\textbf{Greenhouse gases} such as carbon dioxide, water vapour and methane absorb infrared radiation and re-emit it back towards the Earth's surface. In this way, the bottom layers of the atmosphere are kept much warmer than they would be if all the infrared radiation was lost.}
\item{Human activities such as the burning of fossil fuels, increase the concentration of greenhouse gases in the atmosphere and may contribute towards \textbf{global warming}.}
\item{Some of the impacts of global warming include changing climate patterns, rising sea levels and a loss of biodiversity, to name a few. Interventions are needed to reduce this phenomenon.}
\end{itemize}

\begin{eocexercises}{}

\begin{enumerate}

\item{
The atmosphere is a relatively thin layer of gases which support life and provide protection to living organisms. The force of gravity holds the atmosphere against the Earth. The diagram below shows the temperatures associated with the various layers that make up the atmosphere and the altitude (height) from the Earth's surface.

\begin{center}
\scalebox{.8}{
\psset{unit=0.7}
\begin{pspicture}(0,-1)(13,13)
%\psgrid[gridcolor=lightgray]
\rput(1,0){\psline(0,0)(0,12)
\psline(0,0)(12,0)
\rput(-0.25,0){0}
\rput(-0.25,1){10} \rput(-0.25,2){20} \rput(-0.25,3){30} \rput(-0.25,4){40} \rput(-0.25,5){50} \rput(-0.25,6){60} \rput(-0.25,7){70} \rput(-0.25,8){80} \rput(-0.25,9){90} \rput(-0.25,10){100} \rput(-0.25,11){110} \rput(-0.25,12){120}
\rput(0,-0.25){-100} \rput(1,-0.25){-90} \rput(2,-0.25){-80} \rput(3,-0.25){-70} \rput(4,-0.25){-60} \rput(5,-0.25){-50} \rput(6,-0.25){-40} \rput(7,-0.25){-30} \rput(8,-0.25){-20} \rput(9,-0.25){-10} \rput(10,-0.25){0} \rput(11,-0.25){10} \rput(12,-0.25){20}
\rput(-2,6){\textbf{Height (km)}}
\rput(6,-0.8){\textbf{Temperature ($\degree$C)}}
\psline(11.5,0)(5,1)
\psline(5,1)(5,2)
\psline(5,2)(6,3.5)
\psline(6,3.5)(10,4.5)
\psline(10,4.5)(10,5.2)
\psline(10,5.2)(8,6.2)
\psline(8,6.2)(1,8)
\psline(1,8)(1,9)
\psline(1,9)(3,10)
\psline(3,10)(7,11)
\psline(7,11)(8,11.1)
\psline{<->}(12,0.1)(12,2)
\psline{<->}(12,2.1)(12,4)
\psline{<->}(12,4.1)(12,8)
\psline{<->}(12,8.1)(12,11)
\rput(13,1){A}
\rput(13,3.5){B}
\rput(13,6.5){C}
\rput(13,10){D}}
\end{pspicture}}
\end{center}

\begin{enumerate}
\item{Write down the names of the layers A, B and D of the atmosphere.}
\item{In which one of the layers of the atmosphere is ozone found?}
\item{Give an explanation for the decrease in temperature as altitude increases in layer A.}
\item{In layer B, there is a steady increase in temperature as the altitude increases. Write down an explanation for this trend.}
\end{enumerate}
}

\item{
\begin{quote}
\textbf{Planet Earth in Danger}\\ It is now accepted that greenhouse gases are to blame for Planet Earth getting warmer. The increase in the number of sudden floods in Asia and droughts in Africa; the rising sea level and increasing average temperatures are global concerns. Without natural greenhouse gases,like carbon dioxide and water vapour,life on Earth is not possible. However, the increase in levels of carbon dioxide in the atmosphere since the Industrial Revolution is of great concern. Greater disasters are to come, which will create millions of climate refugees. It is our duty to take action for the sake of future generations who will pay dearly for the wait-and-see attitude of the current generation. Urgent action to reduce waste is needed. Global warming is a global challenge and calls for a global response now, not later.\\ (Adapted from a speech by the French President, Jacques Chirac)\\

\end{quote}

\begin{enumerate}
\item How do greenhouse gases, such as carbon dioxide, heat up the Earth's surface?\\
\item Draw a Lewis structure for the carbon dioxide molecule\\
\item The chemical bonds within the carbon dioxide molecule are polar. Support this statement by performing a calculation using the table of electronegativities.\\
\item Classify the carbon dioxide molecule as polar or non-polar. Give a reason for your answer.\\
\item Suggest ONE way in which YOU can help to reduce the emissions of greenhouse gases.\\ \end{enumerate}

}

\item{Plants need carbon dioxide (CO$_{2}$) to manufacture food. However, the engines of motor vehicles cause too much carbon dioxide to be released into the atmosphere.}
\begin{enumerate}
\item{State the possible consequence of having too much carbon dioxide in the atmosphere.}
\item{Explain \textbf{two} possible effects on humans if the amount of carbon dioxide in the atmosphere becomes too low.}
\end{enumerate}

(DoE Exemplar Paper Grade 11, 2007)
\end{enumerate}

\practiceinfo

\begin{tabular}[h]{cccccc}
(1.) 00mv & (2.) 00mw & (3.) 00mx & 
 \end{tabular}
\end{eocexercises}
