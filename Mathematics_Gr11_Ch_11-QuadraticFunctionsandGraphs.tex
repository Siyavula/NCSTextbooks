\chapter{Quadratic Functions and Graphs - Grade 11}
\label{m:fg:q11}

\section{Introduction}
In Grade 10, you studied graphs of many different forms. In this chapter, you will learn a little more about the graphs of functions.

%\begin{syllabus}
%\item Demonstrate the ability to work with various types of functions.
%\item Recognise relationships between variables in terms of numerical, graphical, verbal and symbolic representations and convert flexibly between these representations (tables, graphs, words and formulae).
%\item Generate as many graphs as necessary, initially by means of point-by-point plotting, supported by available technology, to make and test conjectures and hence to generalise the effects of the parameters $a$ and $q$ on the graphs of functions including:
%\begin{eqnarray*}
%y=a(x+p)^2+q\\
%\end{eqnarray*}
%\item Identify characteristics as listed below and hence use applicable characteristics to sketch graphs of functions including those listed above:
%\begin{itemize}
%\item domain and range;
%\item intercepts with the axes;
%\item turning points, minima and maxima;
%\item asymptotes;
%\item shape and symmetry;
%\item average gradient (average rate of change);
%\item intervals on which the function increases/decreases;
%\item the discrete or continuous nature of the graph.
%\end{itemize}
%\end{syllabus}

\section{Functions of the Form $y=a(x+p)^2+q$}
This form of the quadratic function is slightly more complex than the form studied in Grade 10, $y=ax^2+q$. The general shape and position of the graph of the function of the form $f(x)=a(x+p)^2+q$ is shown in Table~\ref{fig:mf:g:parabola}.

\begin{figure}[!ht]
\begin{center}
\begin{pspicture}(-5,-3)(5,3)
%\psgrid
\psaxes(0,0)(-5,-3)(5,3)
\psset{yunit=0.5}
\psplot[plotstyle=curve,arrows=<->]{-5}{1}{x 2 add 2 exp 2 sub}
\end{pspicture}
\caption{Graph of $f(x)=\frac{1}{2}(x+2)^2 - 1$}
 \label{fig:mf:g:parabola}
\end{center}
\end{figure}

\Activity{Investigation}{Functions of the Form $y=a(x+p)^2+q$}{
\begin{enumerate}
\item{On the same set of axes, plot the following graphs:
\begin{enumerate}
\item{$a(x)=(x-2)^2$}
\item{$b(x)=(x-1)^2$}
\item{$c(x)=x^2$}
\item{$d(x)=(x+1)^2$}
\item{$e(x)=(x+2)^2$}
\end{enumerate}
Use your results to deduce the effect of $p$.}
\item{On the same set of axes, plot the following graphs:
\begin{enumerate}
\item{$f(x)=(x-2)^2+1$}
\item{$g(x)=(x-1)^2+1$}
\item{$h(x)=x^2+1$}
\item{$j(x)=(x+1)^2+1$}
\item{$k(x)=(x+2)^2+1$}
\end{enumerate}
Use your results to deduce the effect of $q$.}
\item{Following the general method of the above activities, choose your own values of $p$ and $q$ to plot 5 different graphs (on the same set of axes) of $y=a(x+p)^2+q$ to deduce the effect of $a$.}
\end{enumerate}}

From your graphs, you should have found that $a$ affects whether the graph makes a smile or a frown. If $a<0$, the graph makes a frown and if $a>0$ then the graph makes a smile. This was shown in Grade 10.%Figure~\ref{fig:mf:g:parabola10a}.

You should have also found that the value of $q$ affects whether the turning point of the graph is above the $x$-axis ($q<0$) or below the $x$-axis ($q>0$).

You should have also found that the value of $p$ affects whether the turning point is to the left of the $y$-axis ($p>0$) or to the right of the $y$-axis ($p<0$).

These different properties are summarised in Table~\ref{tab:mf:graphs:summarypar}. The axes of symmetry for each graph is shown as a dashed line.

\begin{table}[htb]
\begin{center}
\caption{Table summarising general shapes and positions of functions of the form $y=a(x+p)^2+q$. The axes of symmetry are shown as dashed lines.}
\label{tab:mf:graphs:summarypar}
\begin{tabular}{|c|c|c||c|c|}\hline
&\multicolumn{2}{c||}{$p<0$}&\multicolumn{2}{c|}{$p>0$}\\\hline
& $a>0$&$a<0$& $a>0$&$a<0$\\\hline\hline
$q \ge 0$&
\begin{pspicture}(-1.2,-1.2)(1.2,1.2)
\psset{yunit=0.25,xunit=0.25}
\psaxes[arrows=<->,dx=0,Dx=10,dy=0,Dy=10](0,0)(-4,-4)(4,4)
\psplot[plotstyle=curve,arrows=<->]{-0.6}{2.6}{x 1 sub 2 exp 1 add}
\psline[linestyle=dashed](1,-3.5)(1,3.5)
\end{pspicture}
&
\begin{pspicture}(-1.2,-1.2)(1.2,1.2)
\psset{yunit=0.25,xunit=0.25}
\psaxes[arrows=<->,dx=0,Dx=10,dy=0,Dy=10](0,0)(-4,-4)(4,4)
\psplot[plotstyle=curve,arrows=<->]{-0.6}{2.6}{x 1 sub 2 exp neg 1 add}
\psline[linestyle=dashed](1,-3.5)(1,3.5)
\end{pspicture}
&
\begin{pspicture}(-1.2,-1.2)(1.2,1.2)
\psset{yunit=0.25,xunit=0.25}
\psaxes[arrows=<->,dx=0,Dx=10,dy=0,Dy=10](0,0)(-4,-4)(4,4)
\psplot[plotstyle=curve,arrows=<->]{-2.6}{0.6}{x 1 add 2 exp 1 add}
\psline[linestyle=dashed](-1,-3.5)(-1,3.5)
\end{pspicture}
&
\begin{pspicture}(-1.2,-1.2)(1.2,1.2)
\psset{yunit=0.25,xunit=0.25}
\psaxes[arrows=<->,dx=0,Dx=10,dy=0,Dy=10](0,0)(-4,-4)(4,4)
\psplot[plotstyle=curve,arrows=<->]{-2.6}{0.6}{x 1 add 2 exp neg 1 add}
\psline[linestyle=dashed](-1,-3.5)(-1,3.5)
\end{pspicture}\\\hline
$q\le 0$&
\begin{pspicture}(-1.2,-1.2)(1.2,1.2)
\psset{yunit=0.25,xunit=0.25}
\psaxes[arrows=<->,dx=0,Dx=10,dy=0,Dy=10](0,0)(-4,-4)(4,4)
\psplot[plotstyle=curve,arrows=<->]{-0.6}{2.6}{x 1 sub 2 exp 1 sub}
\psline[linestyle=dashed](1,-3.5)(1,3.5)
\end{pspicture}
&
\begin{pspicture}(-1.2,-1.2)(1.2,1.2)
\psset{yunit=0.25,xunit=0.25}
\psaxes[arrows=<->,dx=0,Dx=10,dy=0,Dy=10](0,0)(-4,-4)(4,4)
\psplot[plotstyle=curve,arrows=<->]{-0.6}{2.6}{x 1 sub 2 exp neg 1 sub}
\psline[linestyle=dashed](1,-3.5)(1,3.5)
\end{pspicture}
&
\begin{pspicture}(-1.2,-1.2)(1.2,1.2)
\psset{yunit=0.25,xunit=0.25}
\psaxes[arrows=<->,dx=0,Dx=10,dy=0,Dy=10](0,0)(-4,-4)(4,4)
\psplot[plotstyle=curve,arrows=<->]{-2.6}{0.6}{x 1 add 2 exp 1 sub}
\psline[linestyle=dashed](-1,-3.5)(-1,3.5)
\end{pspicture}
&
\begin{pspicture}(-1.2,-1.2)(1.2,1.2)
\psset{yunit=0.25,xunit=0.25}
\psaxes[arrows=<->,dx=0,Dx=10,dy=0,Dy=10](0,0)(-4,-4)(4,4)
\psplot[plotstyle=curve,arrows=<->]{-2.6}{0.6}{x 1 add 2 exp neg 1 sub}
\psline[linestyle=dashed](-1,-3.5)(-1,3.5)
\end{pspicture}\\\hline
\end{tabular}
\end{center}
\end{table}
Phet simulation for graphing: SIYAVULA-SIMULATION:http://cnx.org/content/m30843/latest/#graphing
\subsection{Domain and Range}
For $f(x)=a(x+p)^2+q$, the domain is $\{x:x\in\mathbb{R}\}$ because there is no value of $x \in \mathbb{R}$ for which $f(x)$ is undefined.

The range of $f(x)=a(x+p)^2+q$ depends on whether the value for $a$ is positive or negative. We will consider these two cases separately.

If $a>0$ then we have:
\begin{eqnarray*}
(x+p)^2&\ge& 0 \quad \mbox{(The square of an expression is always positive)} \\
a(x+p)^2&\ge& 0 \quad \mbox{(Multiplication by a positive number maintains the nature of the inequality)} \\
a(x+p)^2 + q &\ge& q\\
f(x) &\ge& q
\end{eqnarray*}
This tells us that for all values of $x$, $f(x)$ is always greater than or equal to $q$. Therefore if $a>0$, the range of $f(x)=a(x+p)^2+q$ is $\{f(x):f(x)\in[q,\infty)\}$.

Similarly, it can be shown that if $a<0$ that the range of $f(x)=a(x+p)^2+q$ is $\{f(x):f(x)\in(-\infty,q]\}$. This is left as an exercise.

For example, the domain of $g(x)=(x-1)^2 + 2$ is $\{x:x\in\mathbb{R}\}$ because there is no value of $x \in \mathbb{R}$ for which $g(x)$ is undefined. The range of $g(x)$ can be calculated as follows:
\begin{eqnarray*}
(x-p)^2&\ge& 0\\
(x+p)^2+2&\ge& 2\\
g(x) &\ge& 2
\end{eqnarray*}
Therefore the range is $\{g(x):g(x)\in[2,\infty)\}$.

\Exercise{Domain and Range}{
\begin{enumerate}
\item{Given the function $f(x)= (x - 4)^2 - 1$. Give the range of $f(x)$.}
\item{What is the domain of the equation $y = 2x^2 -5x-18$ ?}
\end{enumerate}}

\subsection{Intercepts}
For functions of the form, $y=a(x+p)^2+q$, the details of calculating the intercepts with the $x$ and $y$ axes is given.

The $y$-intercept is calculated as follows:
\begin{eqnarray}
y&=&a(x+p)^2+q\\
y_{int}&=&a(0+p)^2+q\\
&=&ap^2+q
\end{eqnarray}

If $p=0$, then $y_{int}=q$.

For example, the $y$-intercept of $g(x)=(x-1)^2 + 2$ is given by setting $x=0$ to get:
\begin{eqnarray*}
g(x)&=&(x-1)^2 + 2\\
y_{int}&=&(0-1)^2 + 2\\
&=&(-1)^2 + 2\\
&=&1 + 2\\
&=&3\\
\end{eqnarray*}

The $x$-intercepts are calculated as follows:
\begin{eqnarray}
y&=&a(x+p)^2+q\\
0&=&a(x_{int}+p)^2+q\\
a(x_{int}+p)^2&=&-q\\
x_{int}+p&=&\pm \sqrt{-\frac{q}{a}}\\
\label{eq:xintpar}
x_{int}&=&\pm \sqrt{-\frac{q}{a}}-p
\end{eqnarray}
However, (\ref{eq:xintpar}) is only valid if $-\frac{q}{a}>0$ which means that either $q<0$ or $a<0$ but not both. This is consistent with what we expect, since if $q>0$ and $a>0$ then $-\frac{q}{a}$ is negative and in this case the graph lies above the $x$-axis and therefore does not intersect the $x$-axis. If however, $q>0$ and $a<0$, then $-\frac{q}{a}$ is positive and the graph is hat shaped with turning point above the $x$-axis and should have two $x$-intercepts. Similarly, if $q<0$ and $a>0$ then $-\frac{q}{a}$ is also positive, and the graph should intersect with the $x$-axis twice.

For example, the $x$-intercepts of $g(x)=(x-1)^2 + 2$ are given by setting $y=0$ to get:
\begin{eqnarray*}
g(x)&=&(x-1)^2 + 2\\
0&=&(x_{int}-1)^2 + 2\\
-2&=&(x_{int}-1)^2
\end{eqnarray*}
which has no real solutions. Therefore, the graph of $g(x)=(x-1)^2 + 2$ does not have any $x$-intercepts.

\Exercise{Intercepts}{
\begin{enumerate}
\item{Find the x- and y-intercepts of the function 	$f(x) = (x -4)^2 -1$.}
\item{Find the intercepts with both axes of the graph of $f(x)=x^2 - 6x + 8$.}
\item{Given: $f (x) = -x^2 + 4x - 3$. Calculate the x- and y-intercepts of the graph of $f$.}
\end{enumerate}}

\subsection{Turning Points}
The turning point of the function of the form $f(x)=a(x+p)^2+q$ is given by examining the range of the function. We know that if $a>0$ then the range of $f(x)=a(x+p)^2+q$ is $\{f(x):f(x)\in[q,\infty)\}$ and if $a<0$ then the range of $f(x)=a(x+p)^2+q$ is $\{f(x):f(x)\in(-\infty,q]\}$.

So, if $a>0$, then the lowest value that $f(x)$ can take on is $q$. Solving for the value of $x$ at which $f(x)=q$ gives:
\begin{eqnarray*}
q&=&a(x+p)^2+q\\
0&=&a(x+p)^2\\
0&=&(x+p)^2\\
0&=&x+p\\
x&=&-p
\end{eqnarray*}
$\therefore$ $x=-p$ at $f(x)=q$. The co-ordinates of the (minimal) turning point is therefore $(-p,q)$.

Similarly, if $a<0$, then the highest value that $f(x)$ can take on is $q$ and the co-ordinates of the (maximal) turning point is $(-p,q)$.

\Exercise{Turning Points}{
\begin{enumerate}
\item{Determine the turning point of the graph of $f(x)=x^2 - 6x + 8$ .}
\item{Given: $f (x) = -x^2 + 4x - 3$. Calculate the co-ordinates of the turning point of $f$.}
\item{Find the turning point of the following function by completing the square:\\
$y = \frac{1}{2}(x+2)^2 - 1$.}
\end{enumerate}}

\subsection{Axes of Symmetry}
There is only one axis of symmetry for the function of the form $f(x)=a(x+p)^2+q$. This axis passes through the turning point and is parallel to the $y$-axis. Since the $x$-coordinate of the turning point is $x=-p$, this is the axis of symmetry.

\Exercise{Axes of Symmetry}{
\begin{enumerate}
\item{Find the equation of the axis of symmetry of the graph $y = 2x^2 - 5x - 18$.}
\item{Write down the equation of the axis of symmetry of the graph of
$y = 3(x-2)^2+1$.}
\item{Write down an example of an equation of a parabola where the y-axis is the axis of symmetry.}
\end{enumerate}}

\subsection{Sketching Graphs of the Form $f(x)=a(x+p)^2+q$}
In order to sketch graphs of the form $f(x)=a(x+p)^2+q$, we need to determine five characteristics:
\begin{enumerate}
\item{sign of $a$}
\item{domain and range}
\item{turning point}
\item{$y$-intercept}
\item{$x$-intercept}
\end{enumerate}

For example, sketch the graph of $g(x)=-\frac{1}{2}(x+1)^2-3$. Mark the intercepts, turning point and axis of symmetry.

Firstly, we determine that $a<0$. This means that the graph will have a maximal turning point.

The domain of the graph is $\{x:x\in\mathbb{R}\}$ because $f(x)$ is defined for all $x\in \mathbb{R}$. The range of the graph is determined as follows:
\begin{eqnarray*}
(x+1)^2 &\ge& 0\\
-\frac{1}{2}(x+1)^2 &\le& 0\\
-\frac{1}{2}(x+1)^2-3 &\le& -3\\
\therefore f(x) &\le& -3
\end{eqnarray*}

Therefore the range of the graph is $\{f(x):f(x)\in(-\infty,-3]\}$.

Using the fact that the maximum value that $f(x)$ achieves is -3, then the $y$-coordinate of the turning point is -3. The $x$-coordinate is determined as follows:
\begin{eqnarray*}
-\frac{1}{2}(x+1)^2-3 &=& -3\\
-\frac{1}{2}(x+1)^2-3+3 &=& 0\\
-\frac{1}{2}(x+1)^2 &=& 0\\
\mbox{Divide both sides by $-\frac{1}{2}$:} \quad (x+1)^2 &=& 0\\
\mbox{Take square root of both sides:} \quad x+1 &=& 0\\
\therefore \quad x&=&-1
\end{eqnarray*}

The coordinates of the turning point are: $(-1,-3)$.

The $y$-intercept is obtained by setting $x=0$. This gives:
\begin{eqnarray*}
y_{int} &=&-\frac{1}{2}(0+1)^2-3\\
&=&-\frac{1}{2}(1)-3\\
&=&-\frac{1}{2}-3\\
&=&-\frac{1}{2}-3\\
&=&-\frac{7}{2}
\end{eqnarray*}

The $x$-intercept is obtained by setting $y=0$. This gives:
\begin{eqnarray*}
0 &=&-\frac{1}{2}(x_{int}+1)^2-3\\
3 &=&-\frac{1}{2}(x_{int}+1)^2\\
-3 \cdot 2 &=&(x_{int}+1)^2\\
-6 &=&(x_{int}+1)^2
\end{eqnarray*}
which has no real solutions. Therefore, there are no $x$-intercepts.

We also know that the axis of symmetry is parallel to the $y$-axis and passes through the turning point.

\begin{figure}[!ht]
\begin{center}
\begin{pspicture}(-5,-6)(5,1)
%\psgrid
\psset{yunit=0.75,xunit=0.75}
\psaxes[arrows=<->](0,0)(-5,-8)(5,1)
\psplot[plotstyle=curve,arrows=<->]{-4}{2}{x 1 add 2 exp 0.5 mul neg 3 sub}
\psdots(-1,-3)(0,-3.5)
\psline[linestyle=dashed](-1,-8)(-1,1)
\uput[r](0,-3.5){(0,-3.5)}
\uput[ul](-1,-3){(-1,-3)}
\end{pspicture}
\caption{Graph of the function $f(x)=-\frac{1}{2}(x+1)^2-3$}
\label{fig:mf:g:sketchexample}
\end{center}
\end{figure}

Khan Academy video on graphing quadratics:SIYAVULA-VIDEO:http://cnx.org/content/m30843/latest/#quadratics-1 
\Exercise{Sketching the Parabola}{
\begin{enumerate}
\item{Draw the graph of $y=3(x-2)^2+1$ showing all the intercepts with the axes as well as the coordinates of the turning point.}
\item{Draw a neat sketch graph of the function defined by $y = ax^2 + bx + c$ if $a > 0$; $b < 0$; $b^2 = 4ac$.}
\end{enumerate}}

\subsection{Writing an equation of a shifted parabola}
Given a parabola with equation $y = x^2 - 2x - 3$.  The graph of the parabola is shifted one unit to the right.  Or else the y-axis shifts one unit to the left i.e. $x$ becomes $x-1$. Therefore the new equation will become:\\
\begin{eqnarray*}
y &=& (x-1)^2 - 2(x - 1) - 3\\
&=& x^2 - 2x + 1 - 2x + 2 - 3\\
&=& x^2 -4x
\end{eqnarray*}
If the given parabola is shifted 3 units down i.e. $y$ becomes $y+3$. The new equation will be:\\
(Notice the x-axis then moves 3 units upwards)
\begin{eqnarray*}
y + 3 &=& x^2 - 2x - 3\\
y&=&x^2 - 2x - 6
\end{eqnarray*}


\section{End of Chapter Exercises}
\begin{enumerate}
\item Show that if $a<0$, then the range of $f(x)=a(x+p)^2+q$ is $\{f(x):f(x)\in(-\infty,q]\}$.
\item If (2,7) is the turning point of $f(x)=-2x^2-4ax+k$, find the values of the constants $a$ and $k$.
\item The graph in the figure is represented by the equation $f(x)=ax^2+bx$. The coordinates of the turning point are (3,9). Show that $a=-1$ and $b=6$.

\begin{center}
\begin{pspicture}(-1,-1)(4,5)
%\psgrid
\psset{xunit=0.5,yunit=0.5}
\psaxes[arrows=<->,Dy=20,Dx=20](0,0)(-2,-2)(8,10)
\psplot[plotstyle=curve,arrows=<->]{-0.25}{6.25}{x 2 exp neg 6 x mul add}
\psdots(3,9)
\uput[r](3,9){(3,9)}
\end{pspicture}
\end{center}

\item Given: $y = x^2 - 2x + 3$. Give the equation of the new graph originating if:
\begin{enumerate}
\item The graph of $f$ is moved three units to the left.
\item The $x$-axis is moved down three units.
\end{enumerate}
\item A parabola with turning point (-1,-4) is shifted vertically by 4 units upwards.  What are the coordinates of the turning point of the shifted parabola?

\end{enumerate}




% CHILD SECTION END



% CHILD SECTION START

