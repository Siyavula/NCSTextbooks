\chapter{Functions and Graphs}
\label{m:fg12}

\section{Introduction}
\label{m:fg12:i}
In Grades 10 and 11 you have learnt about linear functions and quadratic functions as well as the hyperbolic functions and exponential functions and many more.  In Grade 12 you are expected to demonstrate the ability to work with
various types of functions and  relations including  the inverses of some functions and generate graphs of the
inverse relations of functions, in particular the  inverses of:
\nequ{y = ax + q}
\nequ{y = ax^2}
\nequ{y = ax; a>0}.

\section{Definition of a Function}
A \textit{function} is a relation for which there is only one value of $y$ corresponding to any value of $x$. We sometimes write $y=f(x)$, which is notation meaning '$y$ is a function of $x$'. This definition makes complete sense when compared to our real world examples --- each person has only one height, so height is a function of people; on each day, in a specific town, there is only one average temperature.

However, some very common  mathematical constructions are not functions. For example, consider the relation $x^2+y^2=4$. This relation describes a circle of radius $2$ centred at the origin, as in Figure \ref{fig:mt:g:vlt}. If we let $x=0$, we see that $y^2=4$ and thus either $y=2$ or $y=-2$. Since there are two $y$ values which are possible for the same $x$ value, the relation $x^2+y^2=4$ is \textbf{not} the graph a function.

There is a simple test to check if a relation is a function, by looking at its graph. This test is called the \textit{vertical line test}. If it is possible to draw any vertical line (a line of constant $x$) which crosses the graph of the relation more than once, then the relation is \textbf{not} a function. If more than one intersection point exists, then the intersections correspond to multiple values of $y$ for a single value of $x$.

We can see this with our previous example of the circle by looking at its graph again in Figure \ref{fig:mt:g:vlt}.

\begin{figure}[!ht]
\begin{center}
\begin{pspicture}(-3.5,-3.5)(3.5,3.5)
% FIXED. (FIXME: axes labels intersect the circle. )
{\psaxes[labels=none]{<->}(0,0)(-3,-3)(3,3)}
\rput(0.3,1){$1$}
\rput(-1, -0.3){$-1$}
\rput(2.2, -0.3){$2$}
\rput(-2.3, -0.3){$-2$}
% \rput(0.3,-1){$-1$}
\rput(0.3, 2.2){$2$}
\rput(0.2,2.8){$y$}
\rput(2.8,-.2){$x$}
\pscircle(0,0){2}
\psline[linestyle=dashed](1,2.5)(1,-2.5)
\psdots(1,1.732)(1,-1.732)
\end{pspicture}
\caption{Graph of $x^2+y^2=4$}
\label{fig:mt:g:vlt}
\end{center}
\end{figure}
We see that we can draw a  vertical line, for example the dotted line in the drawing, which cuts the circle more than once. Therefore this is \textbf{not} a function.


\Exercise{}{
\begin{enumerate}
\item{State whether each of the following equations are functions or not:
\begin{enumerate}
\item{$x+y = 4$}
\item{$y=\frac{x}{4}$}
\item{$y=2^x$}
\item{$x^2 + y^2 = 4$}
\end{enumerate}
}
\item{The table gives the average per capita income, $d$, in a region of the country as a function of $u$, the percentage
unemployed. Write down the equation to show that income is a function of the persent unemployed.
\begin{table}[htbp]
\begin{center}
\begin{tabular}{|l|c|c|c|c|}\hline
$u$&$1$&$2$&$3$&$4$\\\hline
$d$&$22500$&$22000$&$21500$&$21000$\\\hline
\end{tabular}
\end{center}
\end{table}
}
\end{enumerate}
}
\section{Notation Used for Functions}
In Grade 10 you were introduced to the notation used to name a function.  In a function $y=f(x)$, $y$ is called the \textit{dependent variable}, because the value of $y$ depends on what you choose as $x$. We say $x$ is the \textit{independent variable}, since we can choose $x$ to be any number. Similarly if $g(t) = 2t + 1$, then $t$ is the independent variable and $g$ is the function name.
If $f(x) = 3x-5$ and you are ask to determine $f(3)$, then you have to work out the value for $f(x)$ when $x = 3$.
For example,
\begin{eqnarray*}
f(x) &=& 3x-5\\
f(3) &=& 3(3)-5\\
&=&4
\end{eqnarray*}

\section{Graphs of Inverse Functions}
\label{mf:inverses}

%\begin{syllabus}
%\item Demonstrate the ability to work with various types of functions and inverses.
%\item Demonstrate knowledge of the formal definition of a function.
%\item Investigate and generate graphs of the inverse relations of functions, in particular the inverses of:
%\begin{eqnarray*}
%y=ax+q\\
%y=ax^2\\
%y=a^x \mbox{ ;$a>0$}
%\end{eqnarray*}
%\item Determine which inverses are functions and how the domain of the original function needs to be restricted so that the inverse is also a function.
%\item Identify characteristics as listed below and hence use applicable characteristics to sketch graphs of the inverses of the functions listed above:
%\begin{itemize}
%\item domain and range;
%\item intercepts with the axes;
%\item turning points, minima and maxima;
%\item asymptotes;
%\item shape and symmetry;
%\item average gradient (average rate of change);
%\item intervals on which the function increases/decreases.
%\end{itemize}
%\end{syllabus}

In earlier grades, you studied various types of functions and understood the effect of various parameters in the general equation. In this section, we will consider \textit{inverse functions}.

An inverse function is a function which does the reverse of a given function. More formally, if $f$ is a function with domain $X$, then $f^{-1}$ is its inverse function if and only if for every $x \in X$ we have:
\begin{equation}
f^{-1}(f(x))=x
\end{equation}

A simple way to think about this is that a function, say $y=f(x)$, gives you a $y$-value if you substitute an $x$-value into $f(x)$. The inverse function tells you tells you which $x$-value was used to get a particular $y$-value when you substitue the $y$-value into $f^{-1}(y)$. There are some things which can complicate this - for example, for the $\sin$ function there are many $x$-values that give you a peak as the function oscillates. This means that the inverse of the $\sin$ function would be tricky to define because if you substitute the peak $y$-value into it you won't know which of the $x$-values was used to get the peak.

\begin{eqnarray*}
 y&=&f(x) \quad \mbox{we have a function} \\
 y_1 &=& f(x_1) \quad \mbox{we subsititute a specific $x$-value into the function to get a specific $y$-value}\\
 &&\mbox{consider the inverse function}\\
 x&=&f^{-1}(y)\\
 x&=&f^{-1}(y)\quad\mbox{substituting the specific $y$-value into the inverse should return the specific $x$-value}\\
 &=&f^{-1}(y_1)\\
  &=&x_1
\end{eqnarray*}

This works both ways, if we don't have any complications like in the case of the $\sin$ function, so we can write:

\begin{equation}
f^{-1}(f(x))=f(f^{-1}(x))=x
\end{equation}

For example, if the function $x \rightarrow 3x + 2$ is given, then its inverse function is $x \rightarrow \dfrac{(x-2)}{3}$. This is usually written as:
\begin{eqnarray}
f &\colon& x \to 3x+2\\
f^{-1} &\colon& x \to \dfrac{(x-2)}{3}
\end{eqnarray}

The superscript "$-1$" is not an exponent. 

If a function $f$ has an inverse then $f$ is said to be invertible.

If $f$ is a real-valued function, then for $f$ to have a valid inverse, it must pass the \textbf{horizontal line test}, that is a horizontal line $y = k$ placed anywhere on the graph of $f$ must pass through $f$ exactly once for all real $k$.

It is possible to work around this condition, by defining a multi-valued function as an inverse.

If one represents the function $f$ graphically in a $xy$-coordinate system, the inverse function of the equation of a straight line, $f^{-1}$, is the reflection of the graph of $f$ across the line $y = x$.

Algebraically, one computes the inverse function of $f$ by solving the equation
\nequ{y = f(x)}
for $x$, and then exchanging $y$ and $x$ to get
\nequ{y = f^{- 1}(x)}
Khan Academy video on introduction to inverse functions:SIYAVULA-VIDEO:http://cnx.org/content/m39282/latest/#inverses-1

\subsection{Inverse Function of $y=ax+q$}
The inverse function of $y=ax+q$ is determined by solving for $x$ as:
\begin{eqnarray}
y&=&ax+q\\
ax&=&y-q\\
x&=&\frac{y-q}{a}\\
&=&\frac{1}{a}y-\frac{q}{a}
\end{eqnarray}

Therefore the inverse of $y=ax+q$ is $y=\frac{1}{a}x-\frac{q}{a}$.

The inverse function of a straight line is also a straight line, except for the case where the straight line is a perfectly horizontal line, in which case the inverse is undefined.

For example, the straight line equation given by $y=2x-3$ has as inverse the function, $y=\frac{1}{2}x+\frac{3}{2}$. The graphs of these functions are shown in Figure \ref{fig:mf:inverses:straight}. It can be seen that the two graphs are reflections of each other across the line $y=x$.

\begin{figure}[htb]
\begin{center}
\pspicture(-4,-4)(4,4)
\psaxes{<->}(0,0)(-4,-4)(4,4)
\psplot[arrows=<->]{-0.5}{3.4}{2 x mul 3 sub}
\psplot[arrows=<->]{-3.75}{3.75}{0.5 x mul 1.5 add}
\psplot[linestyle=dashed]{-3.75}{3.75}{x}
\uput[l](-1,1.3){$f^{-1}(x)=\frac{1}{2}x+\frac{3}{2}$}
\uput[r](1,-1.3){$f(x)=2x-3$}
\endpspicture
\caption{The graphs of the function $f(x)=2x-3$ and its inverse $f^{-1}(x)=\frac{1}{2}x+\frac{3}{2}$. The line $y=x$ is shown as a dashed line.}
\label{fig:mf:inverses:straight}
\end{center}
\end{figure}

\subsubsection{Domain and Range}
We have seen that the domain of a function of the form $y=ax+q$ is $\{x:x\in\mathbb{R}\}$ and the range is $\{y:y\in\mathbb{R}\}$. Since the inverse function of a straight line is also a straight line, the inverse function will have the same domain and range as the original function.

\subsubsection{Intercepts}
The general form of the inverse function of the form $y=ax+q$ is $y=\frac{1}{a}x-\frac{q}{a}$.

By setting $x=0$ we have that the $y$-intercept is $y_{int}=-\frac{q}{a}$. Similarly, by setting $y=0$ we have that the $x$-intercept is $x_{int}=q$.

It is interesting to note that if $f(x)=ax+q$, then $f^{-1}(x)=\frac{1}{a}x-\frac{q}{a}$ and the $y$-intercept of $f(x)$ is the $x$-intercept of $f^{-1}(x)$ and the $x$-intercept of $f(x)$ is the $y$-intercept of $f^{-1}(x)$.

\Exercise{}{
\begin{enumerate}
\item{Given $f(x)=2x-3$, find $f^{-1}(x)$}
\item{Consider the function $f(x) = 3x - 7$.
\begin{enumerate}
\item{Is the relation a function?}
\item{If it is a function, identify the domain and range.}
\end{enumerate}}
\item{Sketch the graph of the function $f(x)= 3x - 1$ and its inverse on the same set of axes.}
\item{The inverse of a function is $f^{-1}(x)= 2x - 4$, what is the function $f(x)$?}
\end{enumerate}
}
\subsection{Inverse Function of $y=ax^2$}
The inverse relation, possibly a function, of $y=ax^2$ is determined by solving for $x$ as:
\begin{eqnarray}
y&=&ax^2\\
x^2&=&\frac{y}{a}\\
x&=&\pm\sqrt{\frac{y}{a}}
\end{eqnarray}

\begin{figure}[htb]
\begin{center}
\pspicture(-4,-4)(4,4)
\psaxes{<->}(0,0)(-4,-4)(4,4)
\psplot[arrows=<->]{-2}{2}{x 2 exp}
\psplot[arrows=->]{0}{4}{x 0.5 exp}
\psplot[arrows=->]{0}{4}{x 0.5 exp neg}
\psplot[linestyle=dashed]{-3.75}{3.75}{x}
\uput[u](2,4){$f(x)=x^2$}
\uput[r](4,2){$f^{-1}(x)=\sqrt{x}$}
\uput[r](4,-2){$f^{-1}(x)=-\sqrt{x}$}
\endpspicture
\caption{The function $f(x)=x^2$ and its inverse $f^{-1}(x)=\pm\sqrt{x}$. The line $y=x$ is shown as a dashed line.}
\label{fig:mf:inverses:quadratic}
\end{center}
\end{figure}

We see that the inverse relation of $y=ax^2$ is not a function because it fails the vertical line test. If we draw a vertical line through the graph of $f^{-1}(x)=\pm\sqrt{x}$, the line intersects the graph more than once. There has to be a restriction on the domain of a parabola for the inverse to also be a function.  Consider the function $f(x) = -x^2 + 9$. The inverse of $f$ can be found by witing $f(y) = x$.  Then
\begin{eqnarray*}
x&=& -y^2 + 9 \\
y^2 &=& 9-x\\
y&=& \pm \sqrt{9-x}
\end{eqnarray*}
If we restrict the domain of $f(x)$ to be $x \ge 0$, then $\sqrt{9-x}$ is a function. If the restriction on the domain of $f$ is $x\le 0$ then $-\sqrt{9-x}$ would be a function, inverse to $f$.

Khan Academy video on inverse functions, example 2: http://cnx.org/content/m39282/latest/#inverses-2
Khan Academy video on inverse functions, example 3: http://cnx.org/content/m39282/latest/#inverses-3

\Exercise{}{
\begin{enumerate}
\item{The graph of $f^{-1}$ is shown. Find the equation of $f$, given that the graph of $f$ is a parabola. (Do not simplify your answer) 
\begin{center}
\pspicture(-4,-1)(4,4)
%\psgrid
\psaxes[Dx=10,Dy=20]{<->}(0,0)(-4,-1)(4,4)
\psplot[arrows=<-]{-4}{3}{3 x sub 0.5 mul 0.5 exp 1 add}
\psplot[arrows=<-]{-4}{3}{3 x sub 0.5 mul 0.5 exp neg 1 add}
\uput[r](1,2.2){$f^{-1}$}
\uput[r](3,1){$(3;1)$}
\uput[d](1,0){$(1;0)$}
\psdots(3,1)(1,0)
\endpspicture
\end{center}}
\item{$f(x) = 2x^2$.
\begin{enumerate}
\item{Draw the graph of $f$ and state its domain and range.}
\item{Find $f^{-1}$ and, if it exists, state the domain and range.}
\item{What must the domain of $f$ be, so that $f^{-1}$ is a function ?}
\end{enumerate}}
\item{Sketch the graph of $x=-\sqrt{10-y^2}$. Label a point on the graph other than the intercepts with the axes.} 
\item{\begin{enumerate}
\item{Sketch the graph of $y=x^2$ labelling a point other than the origin on your graph.}
\item{Find the equation of the inverse of the above graph in the form $y=\ldots$}
\item{Now sketch the graph of $y=\sqrt{x}$.}
\item{The tangent to the graph of $y=\sqrt{x}$ at the point $A(9;3)$ intersects the $x$-axis at $B$. Find the equation of this tangent and hence or otherwise prove that the $y$-axis bisects the straight line $AB$.}
\end{enumerate}
}
\item{Given: $g(x)=-1+\sqrt{x}$, find the inverse of $g(x)$ in the form $g^{-1}(x)=\ldots$}
\end{enumerate}
}
\subsection{Inverse Function of $y=a^x$}
The inverse function of $y=a^x$ is determined by solving for $x$ as follows:
\begin{eqnarray}
y&=&a^x\\
\log(y)&=&\log(a^x)\\
&=&x\log(a)\\
\therefore \quad x&=&\frac{\log(y)}{\log(a)}
\end{eqnarray}
The inverse of $y = 10^x$ is $x = \log(y)$ Therefore, if $f(x) = 10^x$, then $f(x)^{-1} = \log(x)$.

\begin{figure}[htb]
\begin{center}
\pspicture(-4,-4)(4,4)
\psaxes{<->}(0,0)(-4,-4)(4,4)
\psplot[arrows=<->]{.0003}{4}{x log}
\psplot[arrows=<->]{-3.5}{0.6}{10 x exp}
\psplot[linestyle=dashed]{-3.75}{3.75}{x}
\uput[r](0.6,4){$f(x)=10^x$}
\uput[r](4,0.6){$f^{-1}(x)=\log(x)$}
\endpspicture
\caption{The function $f(x)=10^x$ and its inverse $f^{-1}(x)=\log(x)$. The line $y=x$ is shown as a dashed line.}
\label{fig:mf:inverses:exponential}
\end{center}
\end{figure}
The exponential function and the logarithmic function are inverses of each other; the graph of the one is the graph of the other, reflected in the line $y = x$.
The domain of the function is equal to the range of the inverse. The range of the function is equal to the domain of the inverse.

\Exercise{}{
\begin{enumerate}
\item{Given that $f(x)=(\frac{1}{5})^x$, sketch the graphs of $f$ and $f^{-1}$ on the same system of axes indicating a point on each graph (other than the intercepts) and showing clearly which is $f$ and which is $f^{-1}$.}
\item{Given that $f(x)=4^{-x}$,
\begin{enumerate}
\item{Sketch the graphs of $f$ and $f^{-1}$ on the same system of axes indicating a point on each graph (other than the intercepts) and showing clearly which is $f$ and which is $f^{-1}$.}
\item{Write $f^{-1}$ in the form $y=\ldots$}
\end{enumerate}}
\item{Given $g(x) = -1 + \sqrt{x}$, find the inverse of $g(x)$ in the form $g^{-1}(x) = \ldots$}
\item{\begin{enumerate}
\item{Sketch the graph of $y=x^2$, labeling a point other than the origin on your graph.}
\item{Find the equation of the inverse of the above graph in the form $y=\ldots$}
\item{Now, sketch $y = \sqrt{x}$.}
\item{The tangent to the graph of $y = \sqrt{x}$ at the point $A(9; 3)$ intersects the $x$-axis at $B$. Find the equation of this tangent, and hence, or otherwise, prove that the $y$-axis bisects the straight line $AB$.}
\end{enumerate}}

\end{enumerate}
}
\begin{eocexercises}{}
\begin{enumerate}

\item{Sketch the graph of $x=-\sqrt{10-y^2}$. Is this graph a function?  Verify your answer.}
\item{$f(x) = \dfrac{1}{x-5}$, 
\begin{enumerate}
\item{determine the $y$-intercept of $f(x)$}
\item{determine $x$ if $f(x) = -1$.}
\end{enumerate}}
\item{Below, you are given three graphs and five equations.
\begin{center}
\begin{pspicture}(-1,-0.6)(7,6)
%\psgrid[gridlabels=10pt,gridlabelcolor=black]
\psset{xunit=0.5, yunit=0.5}
\psaxes[labels=none]{->}(0, 0)(-1,0)(2.5, 10)
\uput[r](2.5, 0){$x$}
\uput[u](0, 10){$y$}

\pscurve(-1, 9)(-0, 3)(1, 1)(2.3, 0.24)
\uput[u](0, 11){\textbf{Graph 1}}

\rput(7,7){
\psaxes[labels=none]{->}(0, 0)(-1, -7)(3, 3)
\uput[r](3, 0){$x$}
\uput[u](0, 3){$y$}
\pscurve(0.03, -3.192)(0.05, -2.727)(0.07, -2.421)(0.09,-2.192)(0.2, -1.465)(0.22, -1.378)(0.3, -1.096)(0.4, -0.834)(0.5,-0.631)(0.6, -0.465)(0.7, -0.325)(0.8, -0.203)(0.9, -0.096)(1,0)(1.1, 0.087)(1.2, 0.166)(1.3, 0.239)(1.4, 0.306)(1.5,0.369)(1.6, 0.428)(1.7, 0.483)(1.8, 0.535)(1.9, 0.584)(2,0.631)(2.1, 0.675)(2.2, 0.718)(2.3, 0.758)(2.4, 0.797)(2.5,0.834)(2.6, 0.87)(2.7, 0.904)(2.8, 0.937)(2.9, 0.969)
\uput[u](0, 4){\textbf{Graph 2}}
}

\rput(16,7){
\psaxes[labels=none]{->}(0, 0)(-3, -4)(3, 3)
\uput[r](3, 0){$x$}
\uput[u](0, 3){$y$}
\pscurve(-0.05, -2.727)(-0.075, -2.358)(-0.1, -2.096)(-0.2,-1.465)(-0.3, -1.096)(-0.4, -0.834)(-0.5, -0.631)(-0.6,-0.465)(-0.7, -0.325)(-0.8, -0.203)(-0.9, -0.096)(-1, 0.000)(-1.1,0.087)(-1.2, 0.166)(-1.3, 0.239)(-1.4, 0.306)(-1.5, 0.369)(-1.6,0.428)(-1.7, 0.483)(-1.8, 0.535)(-3,1)
\uput[u](0, 4){\textbf{Graph 3}}
}
\end{pspicture}
\end{center}

\begin{enumerate}
\item{$y=\log_3x$}
\item{$y=-\log_3x$}
\item{$y=\log_3(-x)$}
\item{$y=3^{-x}$}
\item{$y=3^x$}
\end{enumerate}
Write the equation that best matches each graph.}

% \item{The graph of $y=f(x)$ is shown in the diagram below.
% 
% \begin{center}
% \psset{unit=10mm}
% \begin{pspicture}(-4, -2.5)(5, 5)
% %\psgrid[gridlabels=10pt,gridlabelcolor=black]
% \psaxes[labels=none, ticks=all]{<->}(0, 0)(-4.5, -2.5)(5,3.5)
% \uput[r](5, 0){$x$}
% \uput[u](0, 3.5){$y$}
% 
% \psline(-4, -2)(0, -2)(2, 0)(4, 2)
% \psframe(-0.3, -2)(0, -1.7)
% 
% \uput[r](0, -2){$-2$}
% \uput[d](2, 0){$2$}
% \uput[r](4, 2){$f(x)$}
% \end{pspicture}
% \end{center}
% 
% \begin{enumerate}
% \item{Find the value of $x$ such that $f(x)=0$.}
% \item{Evaluate $f(3) + f(-1)$.}
% \end{enumerate}}

\item{Given $g(x) = -1 + \sqrt{x}$, find the inverse of $g(x)$ in the form $g^{-1}(x) = \ldots$}
\item{Consider the equation $h(x) = 3 ^x$
\begin{enumerate}
\item{Write down the inverse in the form $h^{-1}(x) =\ldots$}
\item{Sketch the graphs of $h(x)$ and $h^{-1}(x)$ on the same set of axes, labelling the intercepts with the axes.}
\item{For which values of $x$ is $h^{-1}(x)$ undefined ?}
\end{enumerate}
}
\item{\begin{enumerate}

\item{Sketch the graph of $y=2x^2+1$, labelling a point other than the origin on your graph.}
\item{Find the equation of the inverse of the above graph in the form $y=\ldots$}
\item{Now, sketch $y = \sqrt{x}$.}
\item{The tangent to the graph of $y = \sqrt{x}$ at the point $A(9; 3)$ intersects the $x$-axis at $B$. Find the equation of this tangent, and hence, or otherwise, prove that the $y$-axis bisects the straight line $AB$.}
\end{enumerate}}

\end{enumerate}



% CHILD SECTION END 



% CHILD SECTION START 
\practiceinfo
\end{eocexercises}
