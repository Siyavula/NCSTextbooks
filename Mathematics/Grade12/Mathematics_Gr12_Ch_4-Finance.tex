\chapter{Finance}
\label{m:f12}

\section{Introduction}
In earlier grades simple interest and compound interest were studied, together with the concept of depreciation. Nominal and effective interest rates were also described. Since this chapter expands on earlier work, it would be best if you revised the work done in Grades 10 and 11.%Chapters~\ref{m:f10} and \ref{m:f11}.

If you master the techniques in this chapter, when you start working and earning you will be able to apply the techniques in this chapter to critically assess how to invest your money. And when you are looking at applying for a bond from a bank to buy a home, you will confidently be able to get out the calculator and work out with amazement how much you could actually save by making additional repayments. Indeed, this chapter will provide you with the fundamental concepts you will need to confidently manage your finances and with some successful investing, sit back on your yacht and enjoy the millionaire lifestyle.

\chapterstartvideo{aaa}

\section{Finding the Length of the Investment or Loan}
\label{sec:m:f12:term}

%\begin{syllabus}
%\item Calculate the value of $n$ in the formula $A = P(1 + i)^n$
%\end{syllabus}

In Grade 11, we used the Compound Interest formula $A = P(1 + i)^n$ to determine the term of the investment or loan, by trial and error. Remember that $P$ is the initial amount, $A$ is the current amount, $i$ is the interest rate and $n$ is the number of time units (number of months or years). So if we invest an amount and know what the interest rate is, then we can work out how long it will take for the money to grow to the required amount.

Now that you have learnt about logarithms, you are ready to work out the proper algebraic solution. If you need to remind yourself how logarithms work, go to Chapter~\ref{m:f12:logarithms} (on Page~\pageref{m:f12:logarithms}).

The basic finance equation is:
\begin{equation*}
A = P \cdot (1+i)^n
\end{equation*}

If you don't know what $A$, $P$, $i$ and $n$ represent, then you should definitely revise the work from Grade 10 and 11.

Solving for $n$:
\begin{eqnarray*}
A &=& P(1+i)^n\\
(1+i)^n &=& (A/P)\\
\log((1+i)^n)&=&\log(A/P)\\
n\log(1+i)&=&\log(A/P)\\
n &=& \log(A/P) / \log(1+i)
\end{eqnarray*}

Remember, you do not have to memorise this formula. It is very easy to derive any time you need it. It is simply a matter of writing down what you have, deciding what you need, and solving for that variable.

\begin{wex}{Term of Investment - Logarithms}{Suppose we invested R$3~500$ into a savings account which pays $7,5\%$ compound interest. After an unknown period of time our account is worth R$4~044,69$. For how long did we invest the money? How does this compare with the trial and error answer from Chapters~\ref{m:f11}.}{
\westep{Determine what is given and what is required}
\begin{itemize}
\item{$P$=R$3~500$}
\item{$i$=$7,5$\%}
\item{$A$=R$4~044,69$}
\end{itemize}
We are required to find $n$.
\westep{Determine how to approach the problem}
We know that:
\begin{eqnarray*}
A &=& P(1+i)^n\\
(1+i)^n &=& (A/P)\\
\log((1+i)^n)&=&\log(A/P)\\
n\log(1+i)&=&\log(A/P)\\
n &=& \log(A/P) / \log(1+i)
\end{eqnarray*}

\westep{Solve the problem}
\begin{eqnarray*}
n &=& \log(A/P) / \log(1+i)\\
&=&\frac{\log(\frac{4~044,69}{3~500})}{\log(1+0.075)}\qquad \mbox{Remember that: }7.5\%=\frac{7.5}{100}=0.075\\
&=&2.0
\end{eqnarray*}
\westep{Write final answer}
The R$3~500$ was invested for 2 years.}
\end{wex}

\section{Series of Payments}
%\begin{syllabus}
%\item Apply knowledge of geometric series to solving annuity, bond repayment and sinking fund problems, with or without the use of the formulae:
%\begin{eqnarray*}
%F = \frac{x[(1+i)^n -1]}{i}\\
%P = \frac{x[1- (1+i)^{-n}]}{i}
%\end{eqnarray*}
%\end{syllabus}

By this stage, you know how to do calculations such as `If I want R$1~000$ in three years' time, how much do I need to invest now at $10\%$?'

What if we extend this as follows: `If I want to draw R$1~000$ next year, R$1~000$ the next year and R$1~000$ after three years ... how much do I need to initially put into a bank account earning $10\%$ p.a. to be able to afford to be able to do this?'

The obvious way of working that out is to work out how much you need now to afford the payments individually and sum them. We'll work out how much is needed now to afford the payment of R$1~000$ in a year $(= $ {R}$1~000\times (1,10)^{-1} = ${R}$909,09)$, the amount needed now for the following year's R$1~000 (= ${R}$1~000\times (1,10)^{-2} = $ {R}$826,45)$ and the amount needed now for the R$1~000$ after three years $(= ${R}$1~000\times (1,10)^{-3} =${R}$751,31)$. Adding these together gives you the amount needed to afford all three payments and you get R$2~486,85$.

So, if you put R$2~486,85$ into a $10\%$ bank account now, you will be able to draw out R$1~000$ in a year, R$1~000$ a year after that, and R$1~000$ a year after that - and your bank account balance will decrease to R$0$. You would have had exactly the right amount of money to do that (obviously!).

You can check this as follows:

\begin{tabular}{lll}
Amount at Time 0 (i.e. Now) &&= R$2~486,85$\\
Amount at Time 1 (i.e. a year later) &= R$2~486,85(1+10\%)$ &= R$2~735,54$\\
Amount after withdrawing R$1~000$ &= R$2~735,54 ~-$ R$1~000$ &= R$1~735,54$\\
Amount at Time 2 (i.e. a year later) &= R$1~735,54(1+10\%)$ &= R$1~909,09$\\
Amount after withdrawing R$1~00$0 &= R$1~909,09 ~-$ R$1~000$ &= R$909,09$\\
Amount at Time 3 (i.e. a year later) &= R$909,09(1+10\%)$ &= R$1~000$\\
Amount after withdrawing R$1~000$ &= R$1~000 ~-$ R$1~000$ &= R$0$\\
\end{tabular}

Perfect! Of course, for only three years, that was not too bad. But what if I asked you how much you needed to put into a bank account now, to be able to afford R$100$ a month for the next 15 years. If you used the above approach you would still get the right answer, but it would take you weeks!

There is - I'm sure you guessed - an easier way! This section will focus on describing how to work with:
\begin{itemize}
\item{\textbf{annuities} - a fixed sum payable each year or each month, either to provide a pre-determined sum at the end of a number of years or months (referred to as a future value annuity) or a fixed amount paid each year or each month to repay (amortise) a loan (referred to as a present value annuity).}
\item{\textbf{bond repayments} - a fixed sum payable at regular intervals to pay off a loan. This is an example of a present value annuity.}
\item{\textbf{sinking funds} - an accounting term for cash set aside for a particular purpose and invested so that the correct amount of money will be available when it is needed. This is an example of a future value annuity.}
\end{itemize}

\subsection{Sequences and Series}
\label{m:f:ss}
Before we progress, you need to go back and read Chapter~\ref{mp:s} (from Page~\pageref{mp:s}) to revise sequences and series.

In summary, if you have a series of $n$ terms in total which looks like this:
\begin{equation*}
a + ar + ar^2 + ... + ar^{n-1} = a [ 1 + r + r^2 + ... r^{n-1} ]
\end{equation*}

this can be simplified as:
\begin{eqnarray*}
\frac{a (r^n - 1)}{r-1} &\mbox{useful when $r>1$}\\
\frac{a (1 - r^n)}{1-r} &\mbox{useful when $0\le r <1$}
\end{eqnarray*}

\subsection{Present Values of a Series of Payments}\label{presentvalues}
So having reviewed the mathematics of sequences and series, you might be wondering how this is meant to have any practical purpose! Given that we are in the finance section, you would be right to guess that there must be some financial use to all this. Here is an example which happens in many people's lives - so you know you are learning something practical.

Let us say you would like to buy a property for R$300~000$, so you go to the bank to apply for a mortgage bond. The bank wants it to be repaid by annually payments for the next 20 years, starting at end of this year. They will charge you $15\%$ interest per annum. At the end of the 20 years the bank would have received back the total amount you borrowed together with all the interest they have earned from lending you the money. You would obviously want to work out what the annual repayment is going to be!

Let $X$ be the annual repayment, $i$ is the interest rate, and $M$ is the amount of the mortgage bond you will be taking out.

Time lines are particularly useful tools for visualizing the series of payments for calculations, and we can represent these payments on a time line as:

\begin{figure}[htbp]
\begin{center}
\begin{pspicture}(0,0)(10,3)
\psline(0,1.5)(4,1.5) %solid horizontal line
\psline(6,1.5)(10,1.5) %solid horizontal line
\psline[linestyle=dashed](4,1.5)(6,1.5) %dashed horizontal line

\psline[arrows=->](0,1.25)(0,1.75) %arrow at 0
\psline[arrows=->](1,1.25)(1,1.75) %arrow at 1
\psline[arrows=->](2,1.25)(2,1.75) %arrow at 2
\psline[arrows=->](8,1.25)(8,1.75) %arrow at 8
\psline[arrows=->](9,1.25)(9,1.75) %arrow at 9
\psline[arrows=->](10,1.25)(10,1.75) %arrow at 10

\psline(3,1.4)(3,1.6) %vertical line at 3
\psline(7,1.4)(7,1.6) %vertical line at 7

\uput[d](0,1.25){$0$}
\uput[d](1,1.25){$1$}
\uput[d](2,1.25){$2$}
\uput[d](8,1.25){$18$}
\uput[d](9,1.25){$19$}
\uput[d](10,1.25){$20$}

\uput[u](1,1.75){$X$}
\uput[u](2,1.75){$X$}
\uput[u](8,1.75){$X$}
\uput[u](9,1.75){$X$}
\uput[u](10,1.75){$X$}

\uput[r](10.5,1.75){Cash Flows}
\uput[r](10.5,1.25){Time}
\end{pspicture}
\caption{Time line for an annuity (in arrears) of $X$ for $n$ periods.}
\end{center}
\end{figure}

The present value of all the payments (which includes interest) must equate to the (present) value of the mortgage loan amount.

Mathematically, you can write this as:
\begin{equation*}
M = X(1+i)^{-1} + X(1+i)^{-2} + X(1+i)^{-3} + \cdots + X(1+i)^{-20}
\end{equation*}

The painful way of solving this problem would be to do the calculation for each of the terms above - which is $20$ different calculations. Not only would you probably get bored along the way, but you are also likely to make a mistake.

Naturally, there is a simpler way of doing this! You can rewrite the above equation as follows:

\begin{eqnarray*}
M = X[v^1 + v^2 + v^3 + ... + v^{20}]\\
\mbox{where $v = (1+i)^{-1} = 1/(1+i)$}
\end{eqnarray*}

Of course, you do not have to use the method of substitution to solve this. We just find this a useful method because you can get rid of the negative exponents - which can be quite confusing! As an exercise - to show you are a real financial whizz - try to solve this without substitution. It is actually quite easy.

Now, the item in square brackets is the sum of a geometric sequence, as discussion in section~\ref{mp:s}. This can be re-written as follows, using what we know from Chapter~\ref{mp:s} of this text book:
\begin{eqnarray*}
v^1 + v^2 + v^3 + ... + v^n &=& v(1 + v + v^2 + ... + v^{n-1})\\
&=& v\left(\frac{1 - v^n}{1-v}\right)\\
&=& \frac{1 - v^n}{1/v-1}\\
&=& \frac{1-(1+i)^{-n}}{i}\\
\end{eqnarray*}

Note that we took out a common factor of $v$ before using the formula for the geometric sequence.

So we can write:
\begin{equation*}
M = X \biggl[\frac{(1-(1+i)^{-n})}{i}\biggr]
\end{equation*}

This can be re-written:
\begin{equation*}
X = \frac{M}{[\frac{(1-(1+i)^{-n})}{i}]} = \frac{Mi}{1-(1+i)^{-n}}
\end{equation*}

So, this formula is useful if you know the amount of the mortgage bond you need and want to work out the repayment, or if you know how big a repayment you can afford and want to see what property you can buy.

For example, if I want to buy a house for R$300~000$ over 20 years, and the bank is going to charge me $15\%$ per annum on the outstanding balance, then the annual repayment is:
\begin{eqnarray*}
X &=& \frac{Mi}{1-(1+i)^{-n}}\\
&=& \frac{\mbox{R}300\,000 \times 0,15}{1-(1+0,15)^{-20}}\\
&=& \mbox{R}47\,928,44
\end{eqnarray*}
This means, each year for the next 20 years, I need to pay the bank R$47~928,44$ per year before I have paid off the mortgage bond. 

On the other hand, if I know I will have only R$30~000$ per year to repay my bond, then how big a house can I buy? That is easy ....

\begin{eqnarray*}
M &=& X \biggl[\frac{(1-(1+i)^{-n})}{i}\biggr]\\
&=& \mbox{R}30~000 \biggl[\frac{(1-(1,15)^{-20})}{0,15}\biggr]\\
&=& \mbox{R}187~779,94
\end{eqnarray*}

So, for R$30~000$ a year for 20 years, I can afford to buy a house of R$187~800$ (rounded to the nearest hundred).

The bad news is that R$187~800$ does not come close to the R$300~000$ you wanted to pay! The good news is that you do not have to memorise this formula. In fact , when you answer questions like this in an exam, you will be expected to start from the beginning - writing out the opening equation in full, showing that it is the sum of a geometric sequence, deriving the answer, and then coming up with the correct numerical answer.

\begin{wex}{Monthly mortgage repayments}{Sam is looking to buy his first flat, and has R$15~000$ in cash savings which he will use as a deposit. He has viewed a flat which is on the market for R$250~000$, and he would like to work out how much the monthly repayments would be. He will be taking out a 30 year mortgage with monthly repayments. The annual interest rate is $11\%$.}
{\westep{Determine what is given and what is needed}
The following is given:
\begin{itemize}
\item{Deposit amount = R$15~000$}
\item{Price of flat = R$250~000$}
\item{interest rate, $i=11\%$}
\end{itemize}
We are required to find the monthly repayment for a 30-year mortgage.

\westep{Determine how to approach the problem}
We know that:
\nequ{X = \frac{M}{[\frac{(1-(1+i)^{-n})}{i}]}}
In order to use this equation, we need to calculate $M$, the amount of the mortgage bond, which is the purchase price of property less the deposit which Sam pays upfront.
\begin{eqnarray*}
M &=& \mbox{R}250~000 - \mbox{R}15~000\\
&=& \mbox{R}235~000
\end{eqnarray*}

Now because we are considering monthly repayments, but we have been given an annual interest rate, we need to convert this to a monthly interest rate, $i_{12}$. (If you are not clear on this, go back and revise section \ref{m:f11:nominal}.)

\begin{eqnarray*}
(1+ i_{12})^{12} &=& (1+i)\\
(1 + i_{12})^{12} &=& 1,11\\
i_{12} &=& 0,873459\%
\end{eqnarray*}

We know that the mortgage bond is for 30 years, which equates to 360 months.

\westep{Solve the problem}
Now it is easy, we can just plug the numbers in the formula, but do not forget that you can always deduce the formula from first principles as well!

\begin{eqnarray*}
X &=& \frac{M}{[\frac{(1-(1+i)^{-n})}{i}]}\\
&=& \frac{\mbox{R}235~000}{[\frac{(1-(1.00876459)^{-360})}{0,008734594}]}\\
&=& \mbox{R}2~146,39
\end{eqnarray*}

\westep{Write the final answer}
That means that to buy a flat for R$250~000$, after Sam pays a R$15~000$ deposit, he will make repayments to the bank each month for the next 30 years equal to R$2~146,39$.}
\end{wex}

\begin{wex}{Monthly mortgage repayments}{You are considering purchasing a flat for R$200~000$ and the bank's mortgage rate is currently $9\%$ per annum payable monthly. You have savings of R$10~000$ which you intend to use for a deposit. How much would your monthly mortgage payment be if you were considering a mortgage over 20 years.}{
\westep{Determine what is given and what is required}
The following is given:
\begin{itemize}
\item{Deposit amount = R$10~000$}
\item{Price of flat = R$200~000$}
\item{Interest rate, $i=9\%$}
\end{itemize}
We are required to find the monthly repayment for a 20-year mortgage.

\westep{Determine how to approach the problem}
We are considering monthly mortgage repayments, so it makes sense to use months as our time period.

The interest rate was quoted as $9\%$ per annum payable monthly, which means that the monthly effective rate $= \frac{9\%}{12} = 0,75\%$ per month. Once we have converted 20 years into 240 months, we are ready to do the calculations!

First we need to calculate $M$, the amount of the mortgage bond, which is the purchase price of property minus the deposit which Sam pays up-front.
\begin{eqnarray*}
M &=& \mbox{R}200\:~000 - \mbox{R}10\:~000\\
&=& \mbox{R}190\:~000
\end{eqnarray*}

The present value of our mortgage payments $X$ (which includes interest), must equate to the present mortgage amount
\begin{eqnarray*}
M = X\times (1 + 0,75\%)^{-1}&+& \\
X\times (1 + 0,75\%)^{-2}&+& \\
X\times (1 + 0,75\%)^{-3}&+& \\
X\times (1 + 0,75\%)^{-4} &+& \ldots \\
X\times (1 + 0,75\%)^{-239}+ X\times (1 + 0,75\%)^{-240}
\end{eqnarray*}

But it is clearly much easier to use our formula than work out $240$ factors and add them all up!

\westep{Solve the problem}
\begin{eqnarray*}
X \times \frac{1 - (1 + 0,75\%)^{-240}}{0,75\%} &=& \mbox{R}190~000\\
X \times 111,14495 &=& \mbox{R}190~000\\
X &=& \mbox{R}1~709,48
\end{eqnarray*}

\westep{Write the final answer}
So to repay a R$190~000$ mortgage over 20 years, at $9\%$ interest payable monthly, will cost you R$1~709,48$ per month for 240 months.}
\end{wex}

\subsubsection{Show Me the Money…}
Now that you've done the calculations for the worked example and know what the monthly repayments are, you can work out some surprising figures. For example, R$1~709,48$ per month for 240 months makes for a total of R$410~275,20$ ($=$R$1~709,48 \times 240$). That is more than double the amount that you borrowed! This seems like a lot. However, now that you've studied the effects of time (and interest) on money, you should know that this amount is somewhat meaningless. The value of money is dependant on its timing.

Nonetheless, you might not be particularly happy to sit back for 20 years making your R$1~709,48$ mortgage payment every month knowing that half the money you are paying are going toward interest. But there is a way to avoid those heavy interest charges. It can be done for less than R$300$ extra every month...

So our payment is now R$2~000$. The interest rate is still $9\%$ per annum payable monthly ($0,75\%$ per month), and our principal amount borrowed is R$190~000$. Making this higher repayment amount every month, how long will it take to pay off the mortgage?

The present value of the stream of payments must be equal to R$190~000$ (the present value of the borrowed amount). So we need to solve for $n$ in:
\begin{eqnarray*}
\mbox{R}2~000 \times [1 - (1 + 0,75\%)^{-n}]/0,75\% &=& \mbox{R}190~000\\
1 - (1 + 0,75\%)^{-n} &=& (\frac{190~000\times 0,75\%}{2~000})\\
\log(1+0,75\%)^{-n} &=& \log[(1 - \frac{190~000\times 0,0075}{2~000}]\\
-n\times \log(1+0,75\%) &=& \log[(1 - \frac{190~000\times 0,0075}{2~000}]\\
-n \times 0,007472 &=& -1,2465\\
n &=& 166,8 \mbox{ months}\\
& =& 13,9 \mbox{ years}
\end{eqnarray*}
So the mortgage will be completely repaid in less than 14 years, and you would have made a total payment of $166,8\times$ R$2~000 =$ R$333~600$.

Can you see what is happened? Making regular payments of R$2~000$ instead of the required R$1~709,48$, you will have saved R$76~675,20$ ($=$ R$410~275,20 -$ R$333~600$) in interest, and yet you have only paid an additional amount of R$290,52$ for 166,8 months, or R$48~458,74$. You surely know by now that the difference between the additional R$48~458,74$ that you have paid and the R$76~675,20$ interest that you have saved is attributable to, yes, you have got it, compound interest!

\subsection{Future Value of a Series of Payments}
In the same way that when we have a single payment, we can calculate a present value or a future value - we can also do that when we have a series of payments.

In the above section, we had a few payments, and we wanted to know what they are worth now - so we calculated present values. But the other possible situation is that we want to look at the future value of a series of payments.

Maybe you want to save up for a car, which will cost R$45~000$ - and you would like to buy it in 2 years time. You have a savings account which pays interest of $12\%$ per annum. You need to work out how much to put into your bank account now, and then again each month for 2 years, until you are ready to buy the car.

Can you see the difference between this example and the ones at the start of the chapter where we were only making a single payment into the bank account - whereas now we are making a series of payments into the same account? This is a sinking fund.

So, using our usual notation, let us write out the answer. Make sure you agree how we come up with this. Because we are making monthly payments, everything needs to be in months. So let $A$ be the closing balance you need to buy a car, $P$ is how much you need to pay into the bank account each month, and $i_{12}$ is the monthly interest rate. (Careful - because $12\%$ is the annual interest rate, so we will need to work out later what the monthly interest rate is!)

\begin{equation*}
A = P(1+i_{12})^{24} + P(1+i_{12})^{23} + ... + P(1+i_{12})^1
\end{equation*}

Here are some important points to remember when deriving this formula:
\begin{enumerate}
\item{We are calculating future values, so in this example we use $(1+i_{12})^n$ and not $(1+i_{12})^{-n}$. Check back to the start of the chapter if this is not obvious to you by now.}
\item{If you draw a timeline you will see that the time between the first payment and when you buy the car is 24 months, which is why we use $24$ in the first exponent.}
\item{Again, looking at the timeline, you can see that the $24^{th}$ payment is being made one month before you buy the car - which is why the last exponent is a $1$.}
\item{Always check that you have got the right number of payments in the equation. Check right now that you agree that there are $24$ terms in the formula above.}
\end{enumerate}

So, now that we have the right starting point, let us simplify this equation:
\begin{eqnarray*}
A &=& P[(1+i_{12})^{24} + (1+i_{12})^{23} + \ldots + (1+i_{12})^{1}]\\
&=& P [ X^{24} + X^{23} + \ldots + X^1] \mbox{ using $X=(1+i_{12})$}
\end{eqnarray*}

Note that this time $X$ has a positive exponent not a negative exponent, because we are doing future values. This is not a rule you have to memorise - you can see from the equation what the obvious choice of $X$ should be.

Let us re-order the terms:
\begin{equation*}
A = P [ X^1 + X^2 + \ldots + X^{24}] = P \cdot X [1 + X + X^2 + \ldots + X^{23}]
\end{equation*}

This is just another sum of a geometric sequence, which as you know can be simplified as:
\begin{eqnarray*}
A &=& P \cdot X [X^n - 1] / ((1+i_{12})-1)\\
&=& P \cdot X [X^n - 1] / i_{12}
\end{eqnarray*}

So if we want to use our numbers, we know that $A = $R$45~000$, $n=24$ (because we are looking at monthly payments, so there are 24 months involved) and $i = 12\%$ per annum.

BUT (and it is a big but) we need a monthly interest rate. Do not forget that the trick is to keep the time periods and the interest rates in the same units - so if we have monthly payments, make sure you use a monthly interest rate! Using the formula from Grade 11%
%Section \ref{m:f11:nominal}
, we know that $(1+i) = (1+i_{12})^{12}$. So we can show that $i_{12} = 0,0094888 = 0,94888 \%$. 

Therefore,
\begin{eqnarray*}
45~000 &=& P (1,0094888) [(1,0094888)^{24} - 1] / 0,0094888\\
P&=&1 662,67
\end{eqnarray*}

This means you need to invest R$166~267$ each month into that bank account to be able to pay for your car in 2 years time.

%This is wrong, but I'm not sure how to fix it. Since it is just an alternative way of doing something I removed it, but it would be worth checking it for future editions of the book.
%There is another way of looking at this too - in terms of present values. We know that we need an amount of R45~000 in 24 months time, and at a monthly interest rate of 0,94888\%, the present value of this amount is R35~873,72449. Now the question is what monthly amount at 0,94888\% interest over 24 month has a present value of R35~873,72449? We have seen this before - it is just like the mortgage questions! So let us go ahead and see if we get to the same answer…

%\begin{eqnarray*}
%P &=& M / [(1-(1+i)^{-n}) / i]\\
%&=& \rm{R}35~873,72449 [(1-(1,0094888)^{-24}) / 0,0094888]\\
%&=& \rm{R}1~662,67
%\end{eqnarray*}

\Exercise{}{
\begin{enumerate}
\item{You have taken out a mortgage bond for R$875~000$ to buy a flat. The bond is for 30 years and the interest rate is $12\%$ per annum payable monthly.
\begin{enumerate}
\item What is the monthly repayment on the bond?
\item How much interest will be paid in total over the 30 years?
\end{enumerate}}
\item{How much money must be invested now to obtain regular annuity payments of R$5~500$ per month for five years?  The money is invested at $11,1\%$ p.a., compounded monthly. (Answer to the nearest hundred rand.)}
\end{enumerate}

% Automatically inserted shortcodes - number to insert 2
\par \practiceinfo
\par \begin{tabular}[h]{cccccc}
% Question 1
(1.)	01e1	&
% Question 2
(2.)	01e2	&
\end{tabular}}
% Automatically inserted shortcodes - number inserted 2
\section{Investments and Loans}
\label{s:investmentandloans}

By now, you should be well equipped to perform calculations with compound interest. This section aims to allow you to use these valuable skills to critically analyse investment and loan options that you will come across in your later life. This way, you will be able to make informed decisions on options presented to you.

At this stage, you should understand the mathematical theory behind compound interest. However, the numerical implications of compound interest are often subtle and far from obvious.

Recall the example `Show Me the Money' in Section \ref{presentvalues}. For an extra payment of R$29~052$ a month, we could have paid off our loan in less than 14 years instead of 20 years. This provides a good illustration of the long term effect of compound interest that is often surprising. In the following section, we'll aim to explain the reason for the drastic reduction in the time it takes to repay the loan.

\subsection{Loan Schedules}
\label{ss:loadschedules}

So far, we have been working out loan repayment amounts by taking all the payments and discounting them back to the present time. We are not considering the repayments individually. Think about the time you make a repayment to the bank. There are numerous questions that could be raised: how much do you still owe them? Since you are paying off the loan, surely you must owe them less money, but how much less? We know that we'll be paying interest on the money we still owe the bank. When exactly do we pay interest? How much interest are we paying?

The answer to these questions lie in something called the load schedule.

We will continue to use the earlier example. There is a loan amount of R$190~000$. We are paying it off over 20 years at an interest of $9\%$ per annum payable monthly. We worked out that the repayments should be R$1~709,48$.

Consider the first payment of R$1~709,48$ one month into the loan. First, we can work out how much interest we owe the bank at this moment. We borrowed R$190~000$ a month ago, so we should owe:

\begin{eqnarray*}
I &=& M \times i_{12}\\
&=& \mbox{R}190~000 \times 0,75\%\\
&=& \mbox{R}1~425\\
\end{eqnarray*}

We are paying them R$1~425$ in interest. We call this the interest component of the repayment. We are only paying off R$1~709,48 -$ R$1~425 = $R$284,48$ of what we owe! This is called the capital component. That means we still owe R$190~000 -$ R$284,48 =$ R$189~715,52$. This is called the capital outstanding. Let's see what happens at the end of the second month. The amount of interest we need to pay is the interest on the capital outstanding.

\begin{eqnarray*}
I &=& M \times i_{12}\\
&=& \mbox{R}189~715,52 \times 0,75\%\\
&=& \mbox{R}1~422,87\\
\end{eqnarray*}

Since we don't owe the bank as much as we did last time, we also owe a little less interest. The capital component of the repayment is now R$1~709,48 -$ R$1~422,87 =$ R$286,61$. The capital outstanding will be R$189~715,52 -$ R$286,61 =$ R$189~428,91$. This way, we can break each of our repayments down into an interest part and the part that goes towards paying off the loan.

This is a simple and repetitive process. Table \ref{tb:lschedule1} is a table showing the breakdown of the first 12 payments. This is called a loan schedule.

\begin{table}
\begin{center}
\begin{tabular}{|l|lr@{,}l|lr@{,}l|lr@{,}l|lr@{,}l|}
\hline
Time&\multicolumn{3}{l|}{Repayment}&\multicolumn{3}{p{2cm}|}{Interest Component}&\multicolumn{3}{p{2cm}|}{Capital Component}&\multicolumn{3}{p{2.5cm}|}{Capital Outstanding}\\
\hline
\hline
$0$&\multicolumn{3}{l|}{}&\multicolumn{3}{r|}{}&\multicolumn{3}{r|}{}&R&$190~000$&$00$\\
1&R&$1~709$&$48$&R&$1~425$&$00$&R&$284$&$48$&R&$189~715$&$52$\\
$2$&R&$1~709$&$48$&R&$1~422$&$87$&R&$286$&$61$&R&$189~428$&$91$\\
$3$&R&$1~709$&$48$&R&$1~420$&$72$&R&$288$&$76$&R&$189~140$&$14$\\
$4$&R&$1~709$&$48$&R&$1~418$&$55$&R&$290$&$93$&R&$188~849$&$21$\\
$5$&R&$1~709$&$48$&R&$1~416$&$37$&R&$293$&$11$&R&$188~556$&$10$\\
$6$&R&$1~709$&$48$&R&$1~414$&$17$&R&$295$&$31$&R&$188~260$&$79$\\
$7$&R&$1~709$&$48$&R&$1~411$&$96$&R&$297$&$52$&R&$187~963$&$27$\\
$8$&R&$1~709$&$48$&R&$1~409$&$72$&R&$299$&$76$&R&$187~663$&$51$\\
$9$&R&$1~709$&$48$&R&$1~407$&$48$&R&$302$&$00$&R&$187~361$&$51$\\
$10$&R&$1~709$&$48$&R&$1~405$&$21$&R&$304$&$27$&R&$187~057$&$24$\\
$11$&R&$1~709$&$48$&R&$1~402$&$93$&R&$306$&$55$&R&$186~750$&$69$\\
$12$&R&$1~709$&$48$&R&$1~400$&$63$&R&$308$&$85$&R&$186~441$&$84$\\

\hline
\end{tabular}
\end{center}
\caption{A loan schedule with repayments of R$1~709,48$ per month.}
\label{tb:lschedule1}
\end{table}

Now, let's see the same thing again, but with R$2~000$ being repaid each year. We expect the numbers to change. However, how much will they change by? As before, we owe R$1~425$ in interest in interest. After one month. However, we are paying R$2~000$ this time. That leaves R$575$ that goes towards paying off the capital outstanding, reducing it to R$189~425$. By the end of the second month, the interest owed is R$1~420,69$ (That's R$189~425\times i_{12}$). Our R$2~000$ pays for that interest, and reduces the capital amount owed by R$2~000 -$ R$1~420,69 =$ R$579,31$. This reduces the amount outstanding to R$188~845,69$.

Doing the same calculations as before yields a new loan schedule shown in Table \ref{tb:lschedule2}.

\begin{table}
\begin{center}
\begin{tabular}{|l|lr@{,}l|lr@{,}l|lr@{,}l|lr@{,}l|}
\hline
Time&\multicolumn{3}{l|}{Repayment}&\multicolumn{3}{p{2cm}|}{Interest Component}&\multicolumn{3}{p{2cm}|}{Capital Component}&\multicolumn{3}{p{2.5cm}|}{Capital Outstanding}\\
\hline
\hline
$0$&\multicolumn{3}{l|}{}&\multicolumn{3}{r|}{}&\multicolumn{3}{r|}{}&R&$190~000$&$00$\\
$1$&R&$2~000$&$00$&R&$1~425$&$00$&R&$575$&$00$&R&$189~425$&$00$\\
$2$&R&$2~000$&$00$&R&$1~420$&$69$&R&$579$&$31$&R&$188~845$&$69$\\
$3$&R&$2~000$&$00$&R&$1~416$&$34$&R&$583$&$66$&R&$188~262$&$03$\\
$4$&R&$2~000$&$00$&R&$1~411$&$97$&R&$588$&$03$&R&$187~674$&$00$\\
$5$&R&$2~000$&$00$&R&$1~407$&$55$&R&$592$&$45$&R&$187~081$&$55$\\
$6$&R&$2~000$&$00$&R&$1~403$&$11$&R&$596$&$89$&R&$186~484$&$66$\\
$7$&R&$2~000$&$00$&R&$1~398$&$63$&R&$601$&$37$&R&$185~883$&$30$\\
$8$&R&$2~000$&$00$&R&$1~394$&$12$&R&$605$&$88$&R&$185~277$&$42$\\
$9$&R&$2~000$&$00$&R&$1~389$&$58$&R&$610$&$42$&R&$184~667$&$00$\\
$10$&R&$2~000$&$00$&R&$1~385$&$00$&R&$615$&$00$&R&$184~052$&$00$\\
$11$&R&$2~000$&$00$&R&$1~380$&$39$&R&$619$&$61$&R&$183~432$&$39$\\
$12$&R&$2~000$&$00$&R&$1~375$&$74$&R&$624$&$26$&R&$182~808$&$14$\\
\hline
\end{tabular}
\end{center}
\caption{A loan schedule with repayments of R$2~000$ per month.}
\label{tb:lschedule2}
\end{table}

The important numbers to notice is the ``Capital Component" column. Note that when we are paying off R$2~000$ a month as compared to R$1~709,48$ a month, this column more than double. In the beginning of paying off a loan, very little of our money is used to pay off the capital outstanding. Therefore, even a small increase in repayment amounts can significantly increase the speed at which we are paying off the capital.

What's more, look at the amount we are still owing after one year (i.e. at time $12$). When we were paying R$1~709,48$ a month, we still owe R$186~441,84$. However, if we increase the repayments to R$2~000$ a month, the amount outstanding decreases by over R$3~000$ to R$182~808,14$. This means we would have paid off over R$7~000$ in our first year instead of less than R$4~000$. This increased speed at which we are paying off the capital portion of the loan is what allows us to pay off the whole loan in around 14 years instead of the original $20$. Note however, the effect of paying R$2~000$ instead of R$1~709,48$ is more significant in the beginning of the loan than near the end of the loan.

It is noted that in this instance, by paying slightly more than what the bank would ask you to pay, you can pay off a loan a lot quicker. The natural question to ask here is: why are banks asking us to pay the lower amount for much longer then? Are they trying to cheat us out of our money?

There is no simple answer to this. Banks provide a service to us in return for a fee, so they are out to make a profit. However, they need to be careful not to cheat their customers for fear that they'll simply use another bank. The central issue here is one of scale. For us, the changes involved appear big. We are paying off our loan 6 years earlier by paying just a bit more a month. To a bank, however, it doesn't matter much either way. In all likelihood, it doesn't affect their profit margins one bit!

Remember that the bank calculates repayment amounts using the same methods as we've been learning. They decide on the correct repayment amounts for a given interest rate and set of terms. Smaller repayment amounts will make the bank more money, because it will take you longer to pay off the loan and more interest will acumulate. Larger repayment amounts mean that you will pay off the loan faster, so you will accumulate less interest i.e. the bank will make less money off of you. It's a simple matter of less money now or more money later. Banks generally use a 20 year repayment period by default.

Learning about financial mathematics enables you to duplicate these calculations for yourself. This way, you can decide what's best for you. You can decide how much you want to repay each month and you'll know of its effects. A bank wouldn't care much either way, so you should pick something that suits you.

\begin{wex}{Monthly Payments}{Stefan and Marna want to buy a house that costs R $1~200~000$. Their parents offer to put down a $20\%$ payment towards the cost of the house.  They need to get a moratage for the balance.  What are their monthly repayments if the term of the home loan is 30 years and the interest is $7,5\%$, compounded monthly?}
{
\westep{Determine how much money they need to borrow}
R$ 1~200~00 - $R$ 240~000 = $R$ 960~000$
\westep{Determine how to approach the problem}
Use the formula: \\
\nequ{P = \dfrac{x[1-(1+i)^{-n}]}{i}}\\
Where \\
$P = $R$960~000$\\
$n = 30 \times 12 = 360 \mbox{ months}$\\
$i = 0,075 \div 12 = 0,00625$
\westep{Solve the problem}
\begin{eqnarray*}
\mbox{R}960~000&=& \dfrac{x[1-(1 + 0,00625)^{-360}]}{0,00625}\\
&=&x(143,0176273)\\
x&=&\mbox{R} 6~712,46
\end{eqnarray*}
\westep{Write the final answer}
The monthly repayments $= $R$6~712,46$
}
\end{wex}

\Exercise{}{
\begin{enumerate}
\item{A property costs R$1~800~000$.  Calculate the monthly repayments if the interest rate is $14\%$ p.a. compounded monthly and the loan must be paid off in 20 years time.}
\item{A loan of R $4~200$ is to be returned in two equal annual instalments.  If the rate of interest of $10\%$ per annum, compounded annually, calculate the amount of each instalment.}
\end{enumerate}

% Automatically inserted shortcodes - number to insert 2
\par \practiceinfo
\par \begin{tabular}[h]{cccccc}
% Question 1
(1.)	01e3	&
% Question 2
(2.)	01e4	&
\end{tabular}}
% Automatically inserted shortcodes - number inserted 2

\subsection{Calculating Capital Outstanding}
\label{ss:capitaloutstanding}

As defined in Section \ref{ss:loadschedules}, capital outstanding is the amount we still owe the people we borrowed money from at a given moment in time. We also saw how we can calculate this using loan schedules. However, there is a significant disadvantage to this method: it is very time consuming. For example, in order to calculate how much capital is still outstanding at time $12$ using the loan schedule, we'll have to first calculate how much capital is outstanding at time $1$ through to $11$ as well. This is already quite a bit more work than we'd like to do. Can you imagine calculating the amount outstanding after 10 years (time $120$)?

Fortunately, there is an easier method. However, it is not immediately clear why this works, so let's take some time to examine the concept.

\subsubsection{Prospective Method for Capital Outstanding}

Let's say that after a certain number of years, just after we made a repayment, we still owe amount $Y$. What do we know about $Y$? We know that using the loan schedule, we can calculate what it equals to, but that is a lot of repetitive work. We also know that $Y$ is the amount that we are still going to pay off. In other words, all the repayments we are still going to make in the future will exactly pay off $Y$. This is true because in the end, after all the repayments, we won't be owing anything.

Therefore, the present value of all outstanding future payments equal the present amount outstanding. This is the prospective method for calculating capital outstanding.

Let's return to a previous example. Recall the case where we were trying to repay a loan of R$200~000$ over 20 years. A R$10~000$ deposit was put down, so the amount being payed off was R$190~000$. At an interest rate of $9\%$ compounded monthly, the monthly repayment was R$1~709,48$. In Table \ref{tb:lschedule1}, we can see that after 12 months, the amount outstanding was R$186~441,84$. Let's try to work this out using the the prospective method.

After time $12$, there are still $19 \times 12 = 228$ repayments left of R$1~709,48$ each. The present value is:

\begin{eqnarray*}
n &=& 228\\
i &=& 0,75\%\\
Y &=&R1~709,48 \times \frac{1-1,0075^{-228}}{0,0075}\\
&=& R186~441,92\\
\end{eqnarray*}

Oops! This seems to be almost right, but not quite. We should have got R$186~441,84$. We are $8$ cents out. However, this is in fact not a mistake. Remember that when we worked out the monthly repayments, we rounded to the nearest cents and arrived at R$1~709,48$. This was because one cannot make a payment for a fraction of a cent. Therefore, the rounding off error was carried through. That's why the two figures don't match exactly. In financial mathematics, this is largely unavoidable.


\section{Formula Sheet}
As an easy reference, here are the key formulae that we derived and used during this chapter. While memorising them is nice (there are not many), it is the application that is useful. Financial experts are not paid a salary in order to recite formulae, they are paid a salary to use the right methods to solve financial problems.

\subsection{Definitions}
\begin{tabular}{ll}
$P$ &Principal (the amount of money at the starting point of the calculation)\\
$i$ &interest rate, normally the effective rate per annum\\
$n$ &period for which the investment is made\\
$iT$ &the interest rate paid $T$ times per annum, i.e. $iT = \frac{\mbox{Nominal Interest Rate}}{T}$
\end{tabular}

\subsection{Equations}
\begin{equation*}
\left.\begin{array}{l}
\mbox{Present Value - simple}\\
\mbox{Future Value - simple} \\
\mbox{Solve for $i$}\\
\mbox{Solve for $n$}\\
\end{array}\right\}= P (1 + i \cdot n)
\end{equation*}

\begin{equation*}
\left.\begin{array}{l}
\mbox{Present Value - compound}\\
\mbox{Future Value - compound} \\
\mbox{Solve for $i$}\\
\mbox{Solve for $n$}\\
\end{array}\right\}= P (1+i)^n
\end{equation*}

\Tip{Always keep the interest and the time period in the same units of time (e.g. both in years, or both in months etc.).}

\begin{eocexercises}{}
\begin{enumerate}
\item{Thabo is about to invest his R$8~500$ bonus in a special banking product which will pay $1\%$ per annum for 1 month, then $2\%$ per annum for the next 2 months, then $3\%$ per annum for the next 3 months, $4\%$ per annum for the next 4 months, and $0\%$ for the rest of the year. The bank is going to charge him R$100$ to set up the account. How much can he expect to get back at the end of the period?}

\item{A special bank account pays simple interest of $8\%$ per annum. Calculate the opening balance required to generate a closing balance of R$5~000$ after 2 years.}

\item{A different bank account pays compound interest of $8\%$ per annum. Calculate the opening balance required to generate a closing balance of R$5~000$ after 2 years.}

\item{Which of the two answers above is lower, and why?}

\item{7 months after an initial deposit, the value of a bank account which pays compound interest of $7,5\%$ per annum is R$3~650,81$. What was the value of the initial deposit?}

%This question does not make sense, please look at it for future editions
%\item{Suppose you invest R500 this year compounded at interest rate $i$ for a year in Bank T. In the following year you invest the accumulation that you received for another year at the same interest rate and on the third year, you invested the accumulation you received at the same interest rate too. If $P$ represents the present value (R500), find a pattern for this investment. [Hint: find a formula]}

\item{Thabani and Lungelo are both using UKZN Bank for their saving. Suppose Lungelo makes a deposit of $X$ today at interest rate of $i$ for six years. Thabani makes a deposit of $3X$ at an interest rate of $0,05\%$. Thabani made his deposit 3 years after Lungelo made his first deposit. If after 6 years, their investments are equal, calculate the value of $i$ and find $X$. If the sum of their investment is R$20~000$, use the value of $X$ to find out how much Thabani earned in 6 years.}

\item{Sipho invests R$500$ at an interest rate of $\log(1,12)$ for 5 years. Themba, Sipho's sister invested R$200$ at interest rate $i$ for 10 years on the same date that her brother made his first deposit. If after 5 years, Themba's accumulation equals Sipho's, find the interest rate $i$ and find out whether Themba will be able to buy her favorite cell phone after 10 years which costs R$2~000$.}

%This question doesn't make sense either. Do you meane she invested R20000 at 5% for 6 years?
%\item{Moira deposits R20 000 in her saving account for 2 years at an interest rate of 0.05. After 2 years, she invested her accumulation for another 2 years, at the same interest rate. After 4 years, she invested her accumulation for which she got for another 2 years at an interest rate of 5 \%. After 6 years she chose to buy a car which costs R26 000. Her husband, Robert invested the same amount at interest rate of 5 \% for 6 years.
%\begin{enumerate}
%\item{Without using any numbers, find a pattern for Moira's investment?}
%\item{How Moira's investment differ from Robert's?}
%\end{enumerate}}

\item{Calculate the real cost of a loan of R$10~000$ for 5 years at $5\%$ capitalised monthly. Repeat this for the case where it is capitalised half yearly i.e. Every 6 months.}
\item{Determine how long, in years, it will take for the value of a motor vehicle to decrease
to $25\%$ of its original value if the rate of depreciation, based on the reducing-balance
method, is $21\%$ per annum.}

%This question contradicts itself and is unclear
%\item{Andr\'{e} and Thoko, decided to invest their winnings (amounting to R10 000) from their science project. They decided to divide their winnings according to the following: Because Andr\'{e} was the head of the project and he spent more time on it, Andr\'{e} got 65,2 \% of the winnings and Thoko got 34,8\%. So, Thoko decided to invest only 0,5 \% of the share of her sum and Andr\'{e}  decided to invest 1,5 \% of the share of his sum. When they calculated how much each contributed in the investment, Thoko had 25 \% and Andr\'{e} had 75 \% share. They planned to invest their money for 20 years but, as a result of Thoko finding a job in Australia 7 years after their initial investment. They both decided to take whatever value was there and split it according to their initial investment(in terms of percentages). Find how much each will get after 7 years, if the interest rate is equal to the percentage that Thoko invested (NOT the money but the percentage).}

\end{enumerate}



% CHILD SECTION END 



% CHILD SECTION START 
% Automatically inserted shortcodes - number to insert 12
\par \practiceinfo
\par \begin{tabular}[h]{cccccc}
% Question 1
(1.)	01e5	&
% Question 2
(2.)	01e6	&
% Question 3
(3.)	01e7	&
% Question 4
(4.)	01e8	&
% Question 5
(5.)	01e9	&
% Question 6
(6.)	01ea	\\ % End row of shortcodes
% Question 7
(7.)	01eb	&
% Question 8
(8.)	01ec	&
% Question 9
(9.)	01ed	&

\end{tabular}
% Automatically inserted shortcodes - number inserted 12
\end{eocexercises}
