\chapter{Gradient at a Point - Grade 11}
\label{m:gradient11}

%\begin{syllabus}
%\item The learner must be able to investigate numerically the average gradient between two points on a curve and develop an intuitive understanding of the concept of the gradient of a curve at a point.
%\end{syllabus}

\section{Introduction}
In Grade 10, we investigated the idea of \textit{average gradient} and saw that the gradient of some functions varied over different intervals. In Grade 11, we further look at the idea of average gradient, and are introduced to the idea of a gradient of a curve at a point.

\section{Average Gradient}
We saw that the average gradient between two points on a curve is the gradient of the straight line passing through the two points.

\begin{figure}[htbp]
\begin{center}
\begin{pspicture}(-2,-1.5)(2,3.5)
%\psgrid[gridcolor=gray]
\psset{yunit=0.5,xunit=1}
\psaxes[labels=none,ticks=none]{<->}(0,0)(-3,-3)(3,7)
\psplot[linecolor=gray]{-3}{3}{x 2 exp 2 sub}
\uput[dl](0,7){$y$}
\uput[dl](3,0){$x$}
\pnode(-3,7){A}
\pnode(-1,-1){C}
\psdots(A)(C)
\psline(A)(C)
\uput[r](-3,7){$A(-3;7)$}
\uput[l](-1,-1){$C(-1;-1)$}
\end{pspicture}
\caption{The average gradient between two points on a curve is the gradient of the straight line that passes through the points.}
\label{fig:gradient11:strlines}
\end{center}
\end{figure}

What happens to the gradient if we fix the position of one point and move the second point closer to the fixed point?

\Activity{Investigation}{Gradient at a Single Point on a Curve}
{
The curve shown below is defined by $y=-2x^2-5$. Point $B$ is fixed at coordinates
$(0;-5)$. The position of point $A$ varies. Complete the table below by calculating
the $y$-coordinates of point $A$ for the given $x$-coordinates and then calculate
the average gradient between points $A$ and $B$.
\begin{center}
\begin{minipage}{\textwidth}
\begin{multicols}{2}
\begin{tabular}{|r|r|r|}\hline
$x_A$&$y_A$&average gradient\\\hline\hline
$-2$&&\\\hline
$-1.5$&&\\\hline
$-1$&&\\\hline
$-0.5$&&\\\hline
$0$&&\\\hline
$0.5$&&\\\hline
$1$&&\\\hline
$1.5$&&\\\hline
$2$&&\\\hline
\hline
\end{tabular}

\begin{pspicture}(-2,-5)(2,0)
%\psgrid[gridcolor=gray]
\psset{yunit=0.25,xunit=0.5}
\psaxes[labels=none,ticks=none]{<->}(0,0)(-3,-20)(3,1)
\psplot{-2.5}{2.5}{x 2 exp -2 mul 5 sub}
\uput[dl](0,0){$y$}
\uput[dl](3,0){$x$}
\psdots(!-2 -2 2 exp -2 mul 5 sub)(0,-5)
\uput[l](!-2 -2 2 exp -2 mul 5 sub){$A$}
\uput[ul](0,-5){$B$}
\end{pspicture}
\end{multicols}
\end{minipage}
\end{center}
What happens to the average gradient as $A$ moves towards $B$? What happens to the
average gradient as $A$ moves away from $B$? What is the average gradient when $A$ overlaps
with $B$?
}

In Figure~\ref{fig:gradient11:point}, the gradient of the straight line that passes through points $A$ and $C$ changes as $A$ moves closer to $C$. At the point when $A$ and $C$ overlap, the straight line only passes through one point on the curve. Such a line is known as a tangent to the curve.

\begin{figure}[htbp]
\begin{center}
\begin{tabular}{cc}
\begin{pspicture}(-2,-1.4)(2,3)
%\psgrid[gridcolor=gray]
\uput[dl](2,3){(a)}
\psset{yunit=0.35,xunit=0.35}
\psaxes[labels=none,ticks=none]{<->}(0,0)(-3,-3)(3,7)
\psplot[linecolor=gray]{-3}{3}{x 2 exp 2 sub}
\psplot{-3.2}{-0.4}{-4 x mul 5 sub}
\uput[dl](0,7){$y$}
\uput[dl](3,0){$x$}
\pnode(!-3 -3 2 exp 2 sub){A}
\pnode(-1,-1){C}
\psdots(A)(C)
\uput[r](A){$A$}
\uput[l](C){$C$}
\end{pspicture}
&
\begin{pspicture}(-2,-1.4)(2,3)
%\psgrid[gridcolor=gray]
\uput[dl](2,3){(b)}
\psset{yunit=0.35,xunit=0.35}
\psaxes[labels=none,ticks=none]{<->}(0,0)(-3,-3)(3,7)
\psplot[linecolor=gray]{-3}{3}{x 2 exp 2 sub}
\psplot{-2.5}{0}{-3 x mul 4 sub}
\uput[dl](0,7){$y$}
\uput[dl](3,0){$x$}
\pnode(!-2 -2 2 exp 2 sub){A}
\pnode(-1,-1){C}
\psdots(A)(C)
\uput[r](A){$A$}
\uput[l](C){$C$}
\end{pspicture}
\\
\begin{pspicture}(-2,-1.4)(2,3)
%\psgrid[gridcolor=gray]
\uput[dl](2,3){(c)}
\psset{yunit=0.35,xunit=0.35}
\psaxes[labels=none,ticks=none]{<->}(0,0)(-3,-3)(3,7)
\psplot[linecolor=gray]{-3}{3}{x 2 exp 2 sub}
\psplot{-2}{-0.4}{-2 x mul -3 add}
\uput[dl](0,7){$y$}
\uput[dl](3,0){$x$}
\pnode(!-1 -1 2 exp 2 sub){A}
\pnode(-1,-1){C}
\psdots(A)(C)
\uput[r](A){$A$}
\uput[l](C){$C$}
\end{pspicture}
&
\begin{pspicture}(-2,-1.4)(2,3)
%\psgrid[gridcolor=gray]
\uput[dl](2,3){(d)}
\psset{yunit=0.35,xunit=0.35}
\psaxes[labels=none,ticks=none]{<->}(0,0)(-3,-3)(3,7)
\psplot[linecolor=gray]{-3}{3}{x 2 exp 2 sub}
\psplot{-1.5}{0.4}{-1.5 x mul 2.5 sub}
\uput[dl](0,7){$y$}
\uput[dl](3,0){$x$}
\pnode(!-0.5 -0.5 2 exp 2 sub){A}
\pnode(-1,-1){C}
\psdots(A)(C)
\uput[l](A){$A$}
\uput[l](C){$C$}
\end{pspicture}
\\
\end{tabular}
\caption{The gradient of the straight line between $A$ and $C$ changes as the point $A$ moves along the curve towards $C$. There comes a point when $A$ and $C$ overlap (as shown in (c)). At this point the line is a tangent to the curve.}
\label{fig:gradient11:point}
\end{center}
\end{figure}

We therefore introduce the idea of a gradient at a single point on a curve. The gradient at a point on a curve is simply the gradient of the tangent to the curve at the given point. 

\begin{wex}{Average Gradient}{Find the average gradient between two points $P(a;g(a))$ and $Q(a+h;g(a+h))$ on a curve $g(x)=x^2$. Then find the average gradient between $P(2;g(2))$ and $Q(4;g(4))$. Finally, explain what happens to the average gradient if $P$ moves closer to $Q$.\\}
{\westep{Label $x$ points}
\nequ{x_1=a}
\nequ{x_2=a+h}

\westep{Determine $y$ coordinates}
Using the function $g(x)=x^2$, we can determine:
\nequ{y_1=g(a)=a^2}
\begin{eqnarray*}
y_2&=&g(a+h)\\
&=&(a+h)^2\\
&=&a^2+2ah+h^2
\end{eqnarray*}

\westep{Calculate average gradient}
\begin{eqnarray}
\nonumber
\frac{y_2-y_1}{x_2-x_1}&=&\frac{(a^2+2ah+h^2)-(a^2)}{(a+h)-(a)}\\
\nonumber
&=&\frac{a^2+2ah+h^2-a^2}{a+h-a}\\
\nonumber
&=&\frac{2ah+h^2}{h}\\
\nonumber
&=&\frac{h(2a+h)}{h}\\
\label{eq:wexfin}
&=&2a+h
\end{eqnarray}

The average gradient between $P(a;g(a))$ and $Q(a+h;g(a+h))$ on the curve $g(x)=x^2$ is $2a+h$.\\

\westep{Calculate the average gradient between $P(2;g(2))$ and $Q(4;g(4))$}
We can use the result in (\ref{eq:wexfin}), but we have to determine what $a$ and $h$ are. We do this by looking at the definitions of $P$ and $Q$. The $x$-coordinate of $P$ is $a$ and the $x$-coordinate of $Q$ is $a+h$ therefore if we assume that $a=2$ and $a+h=4$, then $h=2$.

Then the average gradient is:
\nequ{2a+h=2(2)+(2)=6}

\westep{When $P$ moves closer to $Q$...}
When point $P$ moves closer to point $Q$, $h$ gets smaller. This means that the average gradient also gets smaller. When the point $Q$ overlaps with the point $P$ $h=0$ and the average gradient is given by $2a$.
}
\end{wex}

We now see that we can write the equation to calculate average gradient in a slightly different manner. If we have a curve defined by $f(x)$ then for two points $P$ and $Q$ with $P(a;f(a))$ and $Q(a+h;f(a+h))$, then the average gradient between $P$ and $Q$ on $f(x)$ is:
\begin{eqnarray*}
\frac{y_2-y_1}{x_2-x_1}&=&\frac{f(a+h)-f(a)}{(a+h)-(a)}\\
&=&\frac{f(a+h)-f(a)}{h}
\end{eqnarray*}
This result is important for calculating the gradient at a point on a curve and will be explored in greater detail in Grade 12.

\begin{eocexercises}{}
\begin{enumerate}
\item{}
\begin{enumerate}
\item{Determine the average gradient of the curve $f(x) = x(x+3)$ between $x = –5$ and $x = –3$.}
\item{Hence, state what you can deduce about the function $f$ between $x = –5$ and $x = –3$.}
\end{enumerate}
\item{A(1;3) is a point on $f(x)=3x^2$.}{
\begin{enumerate}
\item{Determine the gradient of the curve at point $A$.}
\item{Hence, determine the equation of the tangent line at $A$.}
\end{enumerate}
}
\item{Given: $f(x)=2x^2$.}{
\begin{enumerate}
\item{Determine the average gradient of the curve between $x=-2$ and $x=1$.}
\item{Determine the gradient of the curve of $f$ where $x=2$.}
\end{enumerate}}
\end{enumerate}



\insertpracticeinfo{3}
\end{eocexercises} 



% CHILD SECTION START 

