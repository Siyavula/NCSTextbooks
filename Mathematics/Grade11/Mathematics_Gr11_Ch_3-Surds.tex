\chapter{Surds}
\label{m:ng11}

\section{Surd Calculations}
%\begin{syllabus}
%\item Add, subtract, multiply and divide simple surds (e.g. $\sqrt{3}+\sqrt{12}=3\sqrt{3}$ and $\frac{\sqrt{2}}{2}=\frac{1}{\sqrt{2}}$).
%\end{syllabus}

There are several laws that make working with surds (or roots) easier. We will list them all and then explain where each rule comes from in detail.
\begin{eqnarray}
\label{eq:mn:s:1}
\sqrt[n]{a}\sqrt[n]{b}&=&\sqrt[n]{ab}\\
\label{eq:mn:s:2}
\sqrt[n]{\frac ab}&=&\frac{\sqrt[n]{a}}{\sqrt[n]{b}}\\
\label{eq:mn:s:3}
\sqrt[n]{a^m}&=&a^{\frac mn}
\end{eqnarray}

\subsection{Surd Law 1: $\sqrt[n]{a}\sqrt[n]{b}=\sqrt[n]{ab}$}
It is often useful to look at a surd in exponential notation as it allows us to use the exponential laws we learnt in Grade 10. In exponential notation, $\sqrt[n]{a}=a^{\frac{1}{n}}$ and $\sqrt[n]{b}=b^{\frac{1}{n}}$. Then,
\begin{eqnarray}
\label{eq:mn:s:1:exp}
\sqrt[n]{a}\sqrt[n]{b}&=&a^{\frac 1n}b^{\frac 1n}\\ \nonumber
&=&(ab)^{\frac 1n}\\ \nonumber
&=&\sqrt[n]{ab}
\end{eqnarray}

Some examples using this law:
\begin{enumerate}
\item{$\sqrt[3]{16}\times \sqrt[3]{4} \\= \sqrt[3]{64}\\=4$}
\item{$\sqrt{2}\times \sqrt{32}\\=\sqrt{64}\\=8$}
\item{$\sqrt{a^2b^3}\times \sqrt{b^5c^4}\\=\sqrt{a^2b^8c^4}\\= ab^4c^2$}
\end{enumerate}

\subsection{Surd Law 2: $\sqrt[n]{\frac ab}=\frac{\sqrt[n]{a}}{\sqrt[n]{b}}$}
If we look at $\sqrt[n]{\frac ab}$ in exponential notation and
apply the exponential laws then,
\begin{eqnarray}
\label{eq:mn:s:2:exp}
\sqrt[n]{\frac ab}&=&\left(\frac ab\right)^{\frac 1n}\\ \nonumber
&=&\frac{a^{\frac 1n}}{b^{\frac 1n}}\\ \nonumber
&=&\frac{\sqrt[n]{a}}{\sqrt[n]{b}}
\end{eqnarray}

Some examples using this law:
\begin{enumerate}
\item{$\sqrt{12}\div \sqrt{3}\\=\sqrt{4}\\=2$}
\item{$\sqrt[3]{24}\div \sqrt[3]{3}\\=\sqrt[3]{8}\\=2$}
\item{$\sqrt{a^2b^{13}}\div \sqrt{b^5}\\=\sqrt{a^2b^8}\\= ab^4$}
\end{enumerate}


\subsection{Surd Law 3: $\sqrt[n]{a^m}=a^{\frac mn}$}
If we look at $\sqrt[n]{a^m}$ in exponential notation and apply the exponential laws then,
\begin{eqnarray}
\label{eq:mn:s:3:exp}
\sqrt[n]{a^m}&=&(a^m)^{\frac 1n}\\ \nonumber
&=&a^{\frac mn}
\end{eqnarray}

For example,
\begin{eqnarray*}
\sqrt[6]{2^{3}} &=& 2^{\frac{3}{6}}\\
&=& 2^{\frac{1}{2}}\\
&=& \sqrt{2}
\end{eqnarray*}

\subsection{Like and Unlike Surds}
Two surds $\sqrt[m]{a}$ and $\sqrt[n]{b}$ are called \textit{like surds} if
$m=n$, otherwise they are called \textit{unlike surds}. For example $\sqrt{2}$ and $\sqrt{3}$ are like surds, however $\sqrt{2}$ and $\sqrt[3]{2}$ are unlike surds. An important thing to realise about the surd laws we have just learnt is that the surds in the laws are all like surds.

If we wish to use the surd laws on unlike surds, then we must first convert
them into like surds. In order to do this we use the formula
\begin{equation} 
\label{eq:mn:s:like}
\sqrt[n]{a^m}=\sqrt[bn]{a^{bm}}
\end{equation}
to rewrite the unlike surds so that $bn$ is the same for all the surds.

\begin{wex}{Like and Unlike Surds}{Simplify to like surds as far as possible, showing all steps:  $\sqrt[3]{3}\times \sqrt[5]{5}$}{
\westep{Find the common root}
\begin{eqnarray*}
&=&\sqrt[15]{3^5}\times \sqrt[15]{5^3}\\
\end{eqnarray*}
\westep{Use surd Law 1}
\begin{eqnarray*}
&=&\sqrt[15]{3^5.5^3}\\
&=&\sqrt[15]{243\times 125}\\
&=&\sqrt[15]{30 375}\\
\end{eqnarray*}
}
\end{wex}

\subsection{Simplest Surd Form}
In most cases, when working with surds, answers are given in simplest surd form.
For example,
\begin{eqnarray*}
\sqrt{50} &=& \sqrt{25 \times 2}\\
&=& \sqrt{25}\times \sqrt{2}\\
&=& 5\sqrt{2}
\end{eqnarray*}
$5\sqrt{2}$ is the simplest surd form of $\sqrt{50}$.

\begin{wex}{Simplest surd form}
{Rewrite $\sqrt{18}$ in the simplest surd form:}{
\westep{Convert the number $18$ into a product of it's prime factors}
\begin{eqnarray*}
\sqrt{18}&=& \sqrt{2 \times 9}\\
&=&\sqrt{2}\times \sqrt{3^{2}}\\
\end{eqnarray*}
\westep{Square root all squared numbers:}
\begin{eqnarray*}
 &=&3\sqrt{2}
\end{eqnarray*}

}
\end{wex}

\begin{wex}{Simplest surd form}
{Simplify: $\sqrt{147} + \sqrt{108}$}{
\westep{Simplify each square root by converting each number to a product of it's prime factors}
\begin{eqnarray*}
\sqrt{147} + \sqrt{108} &=& \sqrt{49\times 3} + \sqrt{36\times 3}\\
&=& \sqrt{7^{2}\times 3} + \sqrt{6^{2}\times 3}\\
\end{eqnarray*}
\westep{Square root all squared numbers}
\begin{eqnarray*}
&=& 7\sqrt{3} + 6\sqrt{3}
\end{eqnarray*}
 \westep{The exact same surds can be treated as "like terms" and may be added}
\begin{eqnarray*}
&=& 13\sqrt{3}
\end{eqnarray*}
}
\end{wex}

%Khan Academy video on simplifying surds: SIYAVULA-VIDEO:http://cnx.org/content/m30837/latest/#surd-1

\subsection{Rationalising Denominators}
It is useful to work with fractions, which have rational denominators instead of surd denominators. It is possible to rewrite any fraction, which has a surd in the denominator as a fraction which has a rational denominator. We will now see how this can be achieved.

Any expression of the form $\sqrt{a}+\sqrt{b}$ (where $a$ and $b$ are rational)
can be changed into a rational number by multiplying by $\sqrt{a}-\sqrt{b}$
(similarly $\sqrt{a}-\sqrt{b}$ can be rationalised by multiplying by
$\sqrt{a}+\sqrt{b}$). This is because
\begin{equation}
\label{eq:mn:s:rat}
(\sqrt{a}+\sqrt{b})(\sqrt{a}-\sqrt{b})=a-b
\end{equation}
which is rational (since $a$ and $b$ are rational).

If we have a fraction which has a denominator which looks like
$\sqrt{a}+\sqrt{b}$, then we can simply multiply the fraction by
$\frac{\sqrt{a}-\sqrt{b}}{\sqrt{a}-\sqrt{b}}$ to achieve a rational denominator. (Remember that $\frac{\sqrt{a}-\sqrt{b}}{\sqrt{a}-\sqrt{b}} = 1$)
\begin{eqnarray}
\label{eq:mn:s:rat:frac1}
\frac{c}{\sqrt a+\sqrt b}&=&\frac{\sqrt{a}-\sqrt{b}}{\sqrt{a}-\sqrt{b}}
\times\frac{c}{\sqrt a+\sqrt b}\\\nonumber
&=&\frac{c\sqrt{a}-c\sqrt{b}}{a-b}
\end{eqnarray}
or similarly
\begin{eqnarray}
\label{eq:mn:s:rat:frac2}
\frac{c}{\sqrt a-\sqrt b}&=&\frac{\sqrt{a}+\sqrt{b}}{\sqrt{a}+\sqrt{b}}
\times\frac{c}{\sqrt a-\sqrt b}\\\nonumber
&=&\frac{c\sqrt{a}+c\sqrt{b}}{a-b}
\end{eqnarray}

\begin{wex}{Rationalising the Denominator}
{Rationalise the denominator of: $\frac{5x - 16}{\sqrt{x}}$}{

\westep{GRationalise the denominator}
To get rid of $\sqrt{x}$ in the denominator, you can multiply it out by another $\sqrt{x}$. This \textit{rationalises} the surd in the denominator. Note that $\frac{\sqrt{x}}{\sqrt{x}} = 1$, thus the equation becomes rationalised by multiplying by $1$ (although its' value stays the same).
    \begin{eqnarray*}
\frac{5x - 16}{\sqrt{x}} \times \frac{\sqrt{x}}{\sqrt{x}}
\end{eqnarray*}
    \westep{Multiply out the numerators and denominators}
The surd is expressed in the numerator which is the prefered way to write expressions. (That's why denominators get rationalised.)
\begin{eqnarray*}
 & &\frac{5x\sqrt{x} - 16\sqrt{x}}{x}\\
 & = &\frac{(\sqrt{x})(5x - 16)}{x}
\end{eqnarray*}
}
\end{wex}

\begin{wex}{Rationalising the Denominator} 
{Rationalise the following: $\frac{5x - 16}{\sqrt{y} - 10}$}{
\westep{Rationalise the denominator}
\begin{eqnarray*}
\frac{5x - 16}{\sqrt{y} - 10} \times \frac{\sqrt{y} + 10}{\sqrt{y} + 10}
\end{eqnarray*}
    
    \westep{Multiply out the numerators and denominators}
\begin{eqnarray*}
\frac{5x\sqrt{y} - 16\sqrt{y} + 50x - 160}{y - 100}
\end{eqnarray*}
    
All the terms in the numerator are different and cannot be simplified and the denominator does not have any surds in it anymore.}
\end{wex}

\begin{wex}{Rationalise the denominator}
{Simplify the following: $\frac{y-25}{\sqrt{y} + 5}$}{

\westep{Rationalise the denominator}
    \begin{equation*}
\frac{y-25}{\sqrt{y} + 5} \times \frac{\sqrt{y} - 5}{\sqrt{y} - 5}
\end{equation*}
    \westep{Multiply out the numerators and denominators}
    \begin{eqnarray*}
\frac{y\sqrt{y} - 25\sqrt{y} - 5y + 125}{y - 25} &=& \frac{\sqrt{y}(y - 25) - 5(y-25)}{(y-25)}\\
&=& \frac{(y-25)(\sqrt{y} - 25)}{(y - 25)}\\
&=& \sqrt{y} - 25
\end{eqnarray*}
}
\end{wex}

%Khan Academy video on rationalising the denominator: SIYAVULA-VIDEO:http://cnx.org/content/m30837/latest/#surds-2

\begin{eocexercises}{}
\begin{enumerate}
\item{Expand:
\nequ{(\sqrt{x}-\sqrt{2})(\sqrt{x}+\sqrt{2})}}
\item{Rationalise the denominator:
\nequ{\frac{10}{\sqrt{x}-\frac{1}{x}}}}
\item{Write as a single fraction:
\nequ{\frac{3}{2\sqrt{x}}+\sqrt{x}
}}
\item{Write in simplest surd form:

\begin{enumerate}
\item $\sqrt{72}$ 
\item $\sqrt{45} + \sqrt{80}$ \\
\item $\dfrac{\sqrt{48}}{\sqrt{12}}$\\
\item $\dfrac{\sqrt{18} \div \sqrt{72}}{\sqrt{8}}$\\
\item $\dfrac{4}{(\sqrt{8} \div \sqrt{2})}$ 
\item$\dfrac{16}{(\sqrt{20} \div \sqrt{12})}$\\
\end{enumerate}

}
\item{Expand and simplify:
\nequ{(2+\sqrt{2})^2}}
\item{Expand and simplify:
\nequ{(2+\sqrt{2})(1+\sqrt{8})}}
\item{Expand and simplify:
\nequ{(1+\sqrt{3})(1+\sqrt{8}+\sqrt{3})}}
\item{Simplify, without use of a calculator:
\nequ{\sqrt{5}(\sqrt{45}+2\sqrt{80})}}
\item{Simplify:
\nequ{\sqrt{98x^6}+\sqrt{128x^6}}}
\item{Write the following with a rational denominator:
\nequ{\frac{\sqrt{5}+2}{\sqrt{5}}}}
\item{Simplify, without use of a calculator:
\nequ{\frac{\sqrt{98}-\sqrt{8}}{\sqrt{50}}}}

\item{Rationalise the denominator:
\nequ{\frac{y-4}{\sqrt{y}-2}}}
\item{Rationalise the denominator:
\nequ{\frac{2x-20}{\sqrt{y}-\sqrt{10}}}}
\item{Evaluate without using a calculator: $\biggl(2 - \dfrac{\sqrt{7}}{2}\biggr)^{\tfrac{1}{2}} \: \times \: \biggl(2 + \dfrac{\sqrt{7}}{2}\biggr)^{\tfrac{1}{2}}$} 

\item{Prove (without the use of a calculator) that:
\nequ{\sqrt{\frac{8}{3}}+5\sqrt{\frac{5}{3}}-\sqrt{\frac{1}{6}}=\frac{10\sqrt{15}+3\sqrt{6}}{6}}}




\item{The use of a calculator is not permissible in this question. Simplify completely by showing all your steps: $3^{-\tfrac{1}{2}}\biggl[\sqrt{12} + \sqrt[3]{(3\sqrt{3})}\biggr]$}

\item{Fill in the blank surd-form number on the right hand side of the equation which will make the following a true statement:  $ -3\sqrt{6} \times -2\sqrt{24} = - \sqrt{18} \times \ldots \ldots $}

\end{enumerate}



% Automatically inserted shortcodes - number to insert 17
\par \practiceinfo
\par \begin{tabular}[h]{cccccc}
% Question 1
(1.)	016q	&
% Question 2
(2.)	016r	&
% Question 3
(3.)	016s	&
% Question 4
(4.)	016t	&
% Question 5
(5.)	016u	&
% Question 6
(6.)	016v	\\ % End row of shortcodes
% Question 7
(7.)	016w	&
% Question 8
(8.)	016x	&
% Question 9
(9.)	016y	&
% Question 10
(10.)	016z	&
% Question 11
(11.)	0170	&
% Question 12
(12.)	0171	\\ % End row of shortcodes
% Question 13
(13.)	0172	&
% Question 14
(14.)	0173	&
% Question 15
(15.)	0174	&
% Question 16
(16.)	0175	&
% Question 17
(17.)	0176	&
\end{tabular}
% Automatically inserted shortcodes - number inserted 17
\end{eocexercises} 



% CHILD SECTION START 

