\chapter{Hyperbolic Functions and Graphs}
\label{m:fg:h11}

\section{Introduction}
In the previous chapter, we discussed the graphs of general quadratic
functions. In this chapter we will learn more about sketching and
interpreting the graphs of general hyperbolic functions.

\chapterstartvideo{aaa}

%\begin{syllabus}
%\item Demonstrate the ability to work with various types of functions.
%\item Recognise relationships between variables in terms of numerical, graphical, verbal and symbolic representations and convert flexibly between these representations (tables, graphs, words and formulae).
%\item Generate as many graphs as necessary, initially by means of point-by-point plotting, supported by available technology, to make and test conjectures and hence to generalise the effects of the parameters $a$ and $q$ on the graphs of functions including:
%\begin{eqnarray*}
%y=\frac{a}{x+p} + q\\
%\end{eqnarray*}
%\item Identify characteristics as listed below and hence use applicable characteristics to sketch graphs of functions including those listed above:
%\begin{itemize}
%\item domain and range;
%\item intercepts with the axes;
%\item turning points, minima and maxima;
%\item asymptotes;
%\item shape and symmetry;
%\item average gradient (average rate of change);
%\item intervals on which the function increases/decreases;
%\item the discrete or continuous nature of the graph.
%\end{itemize}
%\end{syllabus}

\section{Functions of the Form $y=\frac{a}{x+p} + q$}
This form of the hyperbolic function is slightly more complex than the form studied in Grade 10.
\begin{figure}[htbp]
\begin{center}
\begin{pspicture}(-5,-4)(5,4)
%\psgrid
\psset{yunit=0.75,xunit=0.75}
\psaxes(0,0)(-5,-5)(5,5)
\psplot[plotstyle=curve,arrows=<->]{-5}{-1.145}{x 1 add -1 exp 2 add}
\psplot[plotstyle=curve,arrows=<->]{-0.65}{5}{x 1 add -1 exp 2 add}
\psplot[linestyle=dotted,plotstyle=curve]{-5}{2}{x 3 add}
\psline[linestyle=dashed](-1,-5)(-1,5)
\psline[linestyle=dashed](-5,2)(5,2)
\psline[linestyle=dotted](-4,5)(5,-4)
\end{pspicture}
\caption{General shape and position of the graph of a function of the form $f(x)=\frac{a}{x+p} + q$. The asymptotes are shown as dashed lines.}
\label{fig:mf:g:hyperbola}
\end{center}
\end{figure}
\Activity{Investigation}{Functions of the Form $y=\frac{a}{x+p} + q$}{
\begin{enumerate}
\item{On the same set of axes, plot the following graphs:
\begin{enumerate}
\item{$a(x)=\frac{-2}{x+1} + 1$}
\item{$b(x)=\frac{-1}{x+1} + 1$}
\item{$c(x)=\frac{0}{x+1} + 1$}
\item{$d(x)=\frac{1}{x+1} + 1$}
\item{$e(x)=\frac{2}{x+1} + 1$}
\end{enumerate}
Use your results to deduce the effect of $a$.}
\item{On the same set of axes, plot the following graphs:
\begin{enumerate}
\item{$f(x)=\frac{1}{x-2} + 1$}
\item{$g(x)=\frac{1}{x-1} + 1$}
\item{$h(x)=\frac{1}{x+0} + 1$}
\item{$j(x)=\frac{1}{x+1} + 1$}
\item{$k(x)=\frac{1}{x+2} + 1$}
\end{enumerate}
Use your results to deduce the effect of $p$.}
\item{Following the general method of the above activities, choose your own values of $a$ and $p$ to plot five different graphs of $y=\frac{a}{x+p} + q$ to deduce the effect of $q$.}
\end{enumerate}}

You should have found that the sign of $a$ affects whether the graph is located in the first and third quadrants, or the second and fourth quadrants of Cartesian plane.

You should have also found that the value of $p$ affects whether the $x$-intercept is negative ($p>0$) or positive ($p<0$).

You should have also found that the value of $q$ affects whether the graph lies above the $x$-axis ($q>0$) or below the $x$-axis ($q<0$).

These different properties are summarised in Table~\ref{tab:mf:graphs:summaryhyp}. The axes of symmetry for each graph is shown as a dashed line.

\begin{table}[htb]
\begin{center}
\caption{Table summarising general shapes and positions of functions of the form $y=\frac{a}{x+p} + q$. The axes of symmetry are shown as dashed lines.}
\label{tab:mf:graphs:summaryhyp}
\begin{tabular}{|c|c|c||c|c|}\hline
&\multicolumn{2}{c||}{$p<0$}&\multicolumn{2}{c|}{$p>0$}\\\hline
& $a>0$&$a<0$& $a>0$&$a<0$\\\hline\hline
$q>0$&
\begin{pspicture}(-1.2,-1.2)(1.2,1.2)
%\psgrid
\psset{xunit=0.2,yunit=0.2}
\psaxes[arrows=<->,dx=0,Dx=10,dy=0,Dy=10](0,0)(-5,-5)(5,5)
\psplot[plotstyle=curve,arrows=<->]{-5}{0.85}{x 1 sub -1 exp 2 add}
\psplot[plotstyle=curve,arrows=<->]{1.35}{5}{x 1 sub -1 exp 2 add}
\psplot[linestyle=dotted,plotstyle=curve]{-4}{4}{x 1 add}
\psline[linestyle=dashed](-5,2)(5,2)
\psline[linestyle=dashed](1,-5)(1,5)
\psline[linestyle=dotted](-2,5)(5,-2)
\end{pspicture}
&
\begin{pspicture}(-1.2,-1.2)(1.2,1.2)
%\psgrid
\psset{xunit=0.2,yunit=0.2}
\psaxes[arrows=<->,dx=0,Dx=10,dy=0,Dy=10](0,0)(-5,-5)(5,5)
\psplot[plotstyle=curve,arrows=<->]{-5}{0.70}{x 1 sub -1 exp -1 mul 2 add}
\psplot[plotstyle=curve,arrows=<->]{1.2}{5}{x 1 sub -1 exp -1 mul 2 add}
\psplot[linestyle=dotted,plotstyle=curve]{-2}{4}{x neg 3 add}
\psline[linestyle=dashed](-5,2)(5,2)
\psline[linestyle=dashed](1,-5)(1,5)
\psline[linestyle=dotted](-5,-4)(4,5)
\end{pspicture}
&
\begin{pspicture}(-1.2,-1.2)(1.2,1.2)
%\psgrid
\psset{xunit=0.2,yunit=0.2}
\psaxes[arrows=<->,dx=0,Dx=10,dy=0,Dy=10](0,0)(-5,-5)(5,5)
\psplot[plotstyle=curve,arrows=<->]{-5}{-1.2}{x 1 add -1 exp 2 add}

\psplot[plotstyle=curve,arrows=<->]{-0.70}{5}{x 1 add -1 exp 2 add}
\psplot[linestyle=dotted,plotstyle=curve]{-4}{2}{x 3 add}
\psline[linestyle=dashed](-5,2)(5,2)
\psline[linestyle=dashed](-1,-5)(-1,5)
\psline[linestyle=dotted](-4,5)(5,-4)
\end{pspicture}
&
\begin{pspicture}(-1.2,-1.2)(1.2,1.2)
%\psgrid
\psset{xunit=0.2,yunit=0.2}
\psaxes[arrows=<->,dx=0,Dx=10,dy=0,Dy=10](0,0)(-5,-5)(5,5)
\psplot[plotstyle=curve,arrows=<->]{-5}{-1.3}{x 1 add -1 exp -1 mul 2 add}
\psplot[plotstyle=curve,arrows=<->]{-0.85}{5}{x 1 add -1 exp -1 mul 2 add}
\psplot[linestyle=dotted,plotstyle=curve]{-4}{4}{x neg 1 add}
\psline[linestyle=dashed](-1,-5)(-1,5)
\psline[linestyle=dashed](-5,2)(5,2)
\psline[linestyle=dotted](-5,-2)(2,5)
\end{pspicture}
\\\hline
$q<0$&
\begin{pspicture}(-1.2,-1.2)(1.2,1.2)
%\psgrid
\psset{xunit=0.2,yunit=0.2}
\psaxes[arrows=<->,dx=0,Dx=10,dy=0,Dy=10](0,0)(-5,-5)(5,5)
\psplot[plotstyle=curve,arrows=<->]{-5}{0.75}{x 1 sub -1 exp 2 sub}
\psplot[plotstyle=curve,arrows=<->]{1.35}{5}{x 1 sub -1 exp 2 sub}
\psplot[linestyle=dotted,plotstyle=curve]{-2}{4}{x 3 sub}
\psline[linestyle=dashed](-5,-2)(5,-2)
\psline[linestyle=dashed](1,-5)(1,5)
\psline[linestyle=dotted](-5,4)(4,-5)
\end{pspicture}
&
\begin{pspicture}(-1.2,-1.2)(1.2,1.2)
%\psgrid
\psset{xunit=0.2,yunit=0.2}
\psaxes[arrows=<->,dx=0,Dx=10,dy=0,Dy=10](0,0)(-5,-5)(5,5)
\psplot[plotstyle=curve,arrows=<->]{-5}{0.70}{x 1 sub -1 exp -1 mul 2 sub}
\psplot[plotstyle=curve,arrows=<->]{1.3}{5}{x 1 sub -1 exp -1 mul 2 sub}
\psplot[linestyle=dotted,plotstyle=curve]{-2}{4}{x neg 1 sub}
\psline[linestyle=dashed](-5,-2)(5,-2)
\psline[linestyle=dashed](1,-5)(1,5)
\psline[linestyle=dotted](-2,-5)(5,2)
\end{pspicture}
&
\begin{pspicture}(-1.2,-1.2)(1.2,1.2)
%\psgrid
\psset{xunit=0.2,yunit=0.2}
\psaxes[arrows=<->,dx=0,Dx=10,dy=0,Dy=10](0,0)(-5,-5)(5,5)
\psplot[plotstyle=curve,arrows=<->]{-5}{-1.3}{x 1 add -1 exp 2 sub}
\psplot[plotstyle=curve,arrows=<->]{-0.70}{5}{x 1 add -1 exp 2 sub}
\psplot[linestyle=dotted,plotstyle=curve]{-4}{2}{x 1 sub}
\psline[linestyle=dashed](-1,-5)(-1,5)
\psline[linestyle=dashed](-5,-2)(5,-2)
\psline[linestyle=dotted](-5,2)(2,-5)
\end{pspicture}
&
\begin{pspicture}(-1.2,-1.2)(1.2,1.2)
%\psgrid
\psset{xunit=0.2,yunit=0.2}
\psaxes[arrows=<->,dx=0,Dx=10,dy=0,Dy=10](0,0)(-5,-5)(5,5)
\psplot[plotstyle=curve,arrows=<->]{-5}{-1.3}{x 1 add -1 exp -1 mul 2 sub}
\psplot[plotstyle=curve,arrows=<->]{-0.75}{5}{x 1 add -1 exp -1 mul 2 sub}
\psplot[linestyle=dotted,plotstyle=curve]{-4}{2}{x neg 3 sub}
\psline[linestyle=dashed](-1,-5)(-1,5)
\psline[linestyle=dashed](-5,-2)(5,-2)
\psline[linestyle=dotted](-4,-5)(5,4)
\end{pspicture}
\\\hline
\end{tabular}
\end{center}
\end{table}

\subsection{Domain and Range}
For $y=\frac{a}{x+p} + q$, the function is undefined for $x=-p$. The domain is therefore\\
$\{x:x\in\mathbb{R},x\ne -p\}$.

We see that $y=\frac{a}{x+p} + q$ can be re-written as:
\begin{eqnarray*}
y&=&\frac{a}{x+p} + q\\
y-q&=&\frac{a}{x+p}\\
\mbox{If $x\ne -p$ then:}\quad (y-q)(x+p)&=&a\\
x+p&=&\frac{a}{y-q}
\end{eqnarray*}

This shows that the function is undefined at $y=q$. Therefore the range of $f(x)=\frac{a}{x+p} + q$ is $\{f(x):f(x)\in R, f(x) \neq q\}$.

For example, the domain of $g(x)=\frac{2}{x+1} + 2$ is $\{x:x\in\mathbb{R}, x\ne-1\}$ because $g(x)$ is undefined at $x=-1$.
\begin{eqnarray*}
y&=&\frac{2}{x+1} + 2\\
(y-2)&=&\frac{2}{x+1}\\
(y-2)(x+1)&=& 2\\
(x+1)&=&\frac{2}{y-2}
\end{eqnarray*}
We see that $g(x)$ is undefined at $y=2$. Therefore the range is $\{g(x):g(x)\in(-\infty;2)\cup(2;\infty)\}$.

\Exercise{Domain and Range}{
\begin{enumerate}
 
\item{Determine the range of $y = \frac{1}{x}+1$.}
\item{Given:$f(x)=\frac{8}{x-8}+4$. Write down the domain of $f$.}
\item{Determine the domain of $y = -\frac{8}{x+1} + 3$}
\end{enumerate}

% Automatically inserted shortcodes - number to insert 0
\par \practiceinfo
\par \begin{tabular}[h]{cccccc}
(1.) aaa &
(2.) aaa &
(3.) aaa &
\end{tabular}}
% Automatically inserted shortcodes - number inserted 0

\subsection{Intercepts}
For functions of the form, $y=\frac{a}{x+p} + q$, the intercepts with the $x$ and $y$ axis are calculated by setting $x=0$ for the $y$-intercept and by setting $y=0$ for the $x$-intercept.

The $y$-intercept is calculated as follows:
\begin{eqnarray}
y&=&\frac{a}{x+p} + q\\
y_{int}&=&\frac{a}{0+p} + q\\
&=&\frac{a}{p} + q
\end{eqnarray}

For example, the $y$-intercept of $g(x)=\frac{2}{x+1} + 2$ is given by setting $x=0$ to get:
\begin{eqnarray*}
y&=&\frac{2}{x+1} + 2\\
y_{int}&=&\frac{2}{0+1} + 2\\
&=&\frac{2}{1} + 2\\
&=&2+ 2\\
&=&4\\
\end{eqnarray*}

The $x$-intercepts are calculated by setting $y=0$ as follows:
\begin{eqnarray}
y=\frac{a}{x+p} + q\\
0&=&\frac{a}{x_{int}+p} + q\\
\frac{a}{x_{int}+p}&=&-q\\
a&=&-q(x_{int}+p)\\
x_{int}+p&=&\frac{a}{-q}\\
x_{int}&=&\frac{a}{-q}-p
\end{eqnarray}

For example, the $x$-intercept of $g(x)=\frac{2}{x+1} + 2$ is given by setting $x=0$ to get:
\begin{eqnarray*}
y&=&\frac{2}{x+1} + 2\\
0&=&\frac{2}{x_{int}+1} + 2\\
-2&=&\frac{2}{x_{int}+1}\\
-2(x_{int}+1)&=&2\\
x_{int}+1&=&\frac{2}{-2}\\
x_{int}&=&-1-1\\
x_{int}&=&-2
\end{eqnarray*}

\Exercise{Intercepts}{
\begin{enumerate}
\item{Given: $h(x) = \frac{1}{x+4}-2$. Determine the coordinates of the intercepts of $h$ with the $x$- and $y$-axes.}
\item{Determine the $x$-intercept of the graph of $y = \frac{5}{x} + 2$. Give the reason why there is no $y$-intercept for this function.}
\end{enumerate}


% Automatically inserted shortcodes - number to insert 2
\par \practiceinfo
\par \begin{tabular}[h]{cccccc}
% Question 1
(1.)	011q	&
% Question 2
(2.)	011r	&
\end{tabular}}
% Automatically inserted shortcodes - number inserted 2

\subsection{Asymptotes}
There are two asymptotes for functions of the form $y=\frac{a}{x+p} + q$. They are determined by examining the domain and range.

We saw that the function was undefined at $x=-p$ and for $y=q$. Therefore the asymptotes are $x=-p$ and $y=q$.

For example, the domain of $g(x)=\frac{2}{x+1} + 2$ is $\{x:x\in\mathbb{R}; x\ne-1\}$ because $g(x)$ is undefined at $x=-1$. We also see that $g(x)$ is undefined at $y=2$. Therefore the range is $\{g(x):g(x)\in(-\infty;2)\cup(2;\infty)\}$.

From this we deduce that the asymptotes are at $x=-1$ and $y=2$.

\Exercise{Asymptotes}{
\begin{enumerate}
\item{Given: $h(x) = \frac{1}{x+4}-2$.Determine the equations of the asymptotes of $h$.}
\item{Write down the equation of the vertical asymptote of the graph $y = \frac{1}{x-1}$.}
\end{enumerate}

% Automatically inserted shortcodes - number to insert 0
\par \practiceinfo
\par \begin{tabular}[h]{cccccc}
(1.) aaa &
(2.) aaa &
\end{tabular}}
% Automatically inserted shortcodes - number inserted 0

\subsection{Sketching Graphs of the Form $f(x)=\frac{a}{x+p} + q$}
In order to sketch graphs of functions of the form, $f(x)=\frac{a}{x+p} + q$, we need to calculate four characteristics:
\begin{enumerate}
\item{domain and range}
\item{asymptotes}
\item{$y$-intercept}
\item{$x$-intercept}
\end{enumerate}

For example, sketch the graph of $g(x)=\frac{2}{x+1} + 2$. Mark the intercepts and asymptotes.

We have determined the domain to be $\{x:x\in\mathbb{R}, x\ne-1\}$ and the range to be $\{g(x):g(x)\in(-\infty;2)\cup(2;\infty)\}$. Therefore the asymptotes are at $x=-1$ and $y=2$. The $y$-intercept $=4$ and the $x$-intercept $=-2$.

\begin{figure}[H]
\begin{center}
\begin{pspicture}(-5,-3)(5,6)
%\psgrid
\psset{yunit=0.75,xunit=0.75}
\psaxes[arrows=<->](0,0)(-5,-4)(5,7)
\psplot[plotstyle=curve,arrows=<->]{-5}{-1.33}{x 1 add -1 exp 2 mul 2 add}
\psplot[plotstyle=curve,arrows=<->]{-0.60}{5}{x 1 add -1 exp 2 mul 2 add}
\psline[linestyle=dashed](-1,-4)(-1,7)
\psline[linestyle=dashed](-5,2)(5,2)
\end{pspicture}
\caption{Graph of $g(x)=\frac{2}{x+1} + 2$.}
\label{fig:mf:g:hyperbolasketchexample}
\end{center}
\end{figure}

\Exercise{Graphs}{
\begin{enumerate}
\item{Draw the graph of $y = \frac{1}{x} + 2$.  Indicate the horizontal asymptote.}
\item{Given: $h(x) = \frac{1}{x+4}-2$. Sketch the graph of $h$ showing clearly the asymptotes and ALL intercepts with the axes.}
\item{Draw the graph of $y = \frac{1}{x}$  and  $y = - \frac{8}{x+1} + 3$  on the same system of axes.}
\item{Draw the graph of $y = \frac{5}{x-2,5} + 2$.  Explain your method.}
\item{Draw the graph of the function defined by $y = \frac{8}{x-8} + 4$. Indicate the asymptotes and intercepts with the axes.}
\end{enumerate}


% Automatically inserted shortcodes - number to insert 5
\par \practiceinfo
\par \begin{tabular}[h]{cccccc}
% Question 1
(1.)	011s	&
% Question 2
(2.)	011t	&
% Question 3
(3.)	011u	&
% Question 4
(4.)	011v	&
% Question 5
(5.)	011w	&
\end{tabular}}
% Automatically inserted shortcodes - number inserted 5

\begin{eocexercises}{}
\begin{enumerate}
\item{Plot the graph of the hyperbola defined by $y = \frac{2}{x}$  for $-4 \leq x \leq 4$.  Suppose the hyperbola is shifted $3$ units to the right and $1$ unit down.  What is the new equation then?}
\item{Based on the graph of $y = \frac{1}{x}$, determine the equation of the graph with asymptotes $y = 2$ and $x = 1$ and passing through the point ($2; 3$).}
\end{enumerate}



% Automatically inserted shortcodes - number to insert 2
\par \practiceinfo
\par \begin{tabular}[h]{cccccc}
% Question 1
(1.)	011x	&
% Question 2
(2.)	011y	&
\end{tabular}
% Automatically inserted shortcodes - number inserted 2
\end{eocexercises}



% CHILD SECTION START

