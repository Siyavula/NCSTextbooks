\chapter{Finance}
\label{m:f11}

\section{Introduction}
In Grade 10, the concepts of simple and compound interest were
introduced. Here we will extend those concepts, so it is a good idea to
revise what you've learnt. After you have mastered the techniques in this
chapter, you will understand depreciation and will learn how to
determine which bank is offering the best interest rate.

\chapterstartvideo{VMemn}

\section{Depreciation}
%\begin{syllabus}
%\item Use simple and compound decay formulae $(A = P(1 - ni)$ and $A = P(1 - i)n)$ to solve problems (including straight line depreciation and depreciation on a reducing balance) (link to Learning Outcome 2)..
%\end{syllabus}

It is said that when you drive a new car out of the dealership, it loses $20\%$ of its value, because it is now ``second-hand''. And from there on the value keeps falling, or \textit{depreciating}. Second hand cars are cheaper than new cars, and the older the car, usually the cheaper it is. If you buy a second-hand (or should we say \textit{pre-owned}!) car from a dealership, they will base the price on something called \textit{book value}.

The book value of the car is the value of the car taking into account the loss in value due to wear, age and use. We call this loss in value \textit{depreciation}, and in this section we will look at two ways of how this is calculated. Just like interest rates, the two methods of calculating depreciation are \textit{simple} and \textit{compound} methods.

The terminology used for simple depreciation is \textbf{straight-line depreciation} and for compound depreciation is \textbf{reducing-balance depreciation}.  In the straight-line method the value of the asset is reduced by the same constant amount each year. In compound depreciation or reducing-balance the value of the asset is reduced by the same percentage each year. This means that the value of an asset does not decrease by a constant amount each year, but the decrease is most in the first year, then by a smaller amount in the second year and by an even smaller amount in the third year, and so on.

\Extension{Depreciation}{You may be wondering why we need to calculate
depreciation. Determining the value of assets (as in the example of the second
hand cars) is one reason, but there is also a more financial reason for
calculating depreciation - tax! Companies can take depreciation into account as
an expense, and thereby reduce their taxable income. A lower taxable income
means that the company will pay less income tax to the Revenue Service.}

\section{Simple Decay (it really is simple!) or Straight-line depreciation}
Let us return to the second-hand cars. One way of calculating a depreciation amount would be to assume that the car has a limited useful life. Simple depreciation assumes that the value of the car decreases by an equal amount each year. For example, let us say the limited useful life of a car is 5 years, and the cost of the car today is R$60~000$. What we are saying is that after 5 years you will have to buy a new car, which means that the old one will be valueless at that point in time. Therefore, the amount of depreciation is calculated:
\nequ{\frac{\rm{R}60~000}{5~\rm{years}} = \rm{R}12~000\quad\mbox{per year}.}

The value of the car is then:
\begin{center}
\begin{tabular}{ccl}
End of Year 1&R$60~000 - 1\times($R$12~000)$&= R$48~000$\\
End of Year 2&R$60~000 - 2\times($R$12~000)$&= R$36~000$\\
End of Year 3&R$60~000 - 3\times($R$12~000)$&= R$24~000$\\
End of Year 4&R$60~000 - 4\times($R$12~000)$&= R$12~000$\\
End of Year 5&R$60~000 - 5\times($R$12~000)$&= R$0$\\
\end{tabular}
\end{center}
This looks similar to the formula for simple interest:
\begin{equation*}
\mbox{Total Interest after $n$ years} = n \times (P \times i)
\end{equation*}
where $i$ is the annual percentage interest rate and $P$ is the principal amount.

If we replace the word \textit{interest} with the word \textit{depreciation} and the word \textit{principal} with the words \textit{initial value} we can use the same formula:
\begin{equation*}
\mbox{Total depreciation after $n$ years} = n \times (P \times i)
\end{equation*}
Then the book value of the asset after $n$ years is:
\begin{eqnarray*}
\mbox{Initial Value - Total depreciation after $n$ years} &=& P - n \times (P \times i)\\
A &=&P(1-n\times i)
\end{eqnarray*}

For example, the book value of the car after two years can be simply calculated as follows:
\begin{eqnarray*}
\mbox{Book Value after 2 years} &=& P(1-n\times i)\\
&=&\rm{R}60~000(1-2\times 20\%)\\
&=&\rm{R}60~000(1-0,4)\\
&=&\rm{R}60~000(0,6)\\
&=&\rm{R}36~000
\end{eqnarray*}
as expected.

Note that the difference between the simple interest calculations and the simple decay calculations is that while the interest adds value to the principal amount, the depreciation amount reduces value!

\begin{wex}{Simple Decay method}{A car is worth R$240~000$ now.  If it depreciates at a rate of $15\%$ p.a. on a straight-line depreciation, what is it worth in 5 years' time?}{
\westep{Determine what has been provided and what is required}
\begin{eqnarray*}
P &=& R240~000\\
i &=& 0,15\\
n &=& 5\\
A~ &\mbox{is}& \mbox{required}
\end{eqnarray*}
\westep{Determine how to approach the problem}
\begin{eqnarray*}
A &=& P(1 - i\times n)\\
A &=& 240~000(1-(0,15\times 5))
\end{eqnarray*}
\westep{Solve the problem}
\begin{eqnarray*}
A &=& 240~000(1 - 0,75)\\
&=& 240~000\times 0,25\\
&=& 60~000
\end{eqnarray*}
\westep{Write the final answer}
In 5 years' time the car is worth R$60~000$}
\end{wex}

\begin{wex}{Simple Decay}{A small business buys a photocopier for R$12~000$.  For the tax return the owner depreciates this asset over 3 years using a straight-line depreciation method.  What amount will he fill in on his tax form after 1 year, after 2 years and then after 3 years?}{
\westep{Understanding the question}
The owner of the business wants the photocopier to depreciate to R$0$ after 3 years.
Thus, the value of the photocopier will go down by $12~000 \div 3 = \mbox{R}4~000$ per year.
\westep{Value of the photocopier after 1 year}
$12~000 - 4~000 = \mbox{R}8~000$
\westep{Value of the machine after 2 years}
$8~000 - 4~000 = \mbox{R}4~000$
\westep{Write the final answer}
$4~000 - 4~000 = 0$\\
After 3 years the photocopier is worth nothing}
\end{wex}

\Extension{Salvage Value}{Looking at the same example of our car with an
initial value of R$60~000$, what if we suppose that we think we would be able to
sell the car at the end of the 5 year period for R$10~000$? We call this amount
the ``Salvage Value".

We are still assuming simple depreciation over a useful life of 5 years, but now instead of depreciating the full value of the asset, we will take into account the salvage value, and will only apply the depreciation to the value of the asset that we expect not to recoup, i.e. R$60~000 - $R$10~000 = $R$50~000$.

The annual depreciation amount is then calculated as (R$60~000 - $R$10~000) / 5 = $R$10 000$

In general, the formula for simple (straight line) depreciation:
\begin{equation*}
\mbox{Annual Depreciation} = \frac{\mbox{Initial Value - Salvage Value}}{\mbox{Useful Life}}
\end{equation*}}

\Exercise{Simple Decay}{
\begin{enumerate}
\item{A business buys a truck for R$560~000$. Over a period of 10 years the value of the truck depreciates to R$0$ (using the straight-line method). What is the value of the truck after 8 years?}
\item{Shrek wants to buy his grandpa's donkey for R$800$. His grandpa is quite pleased with the offer, seeing that it only depreciated at a rate of 3\% per year using the straight-line method.  Grandpa bought the donkey 5 years ago.  What did grandpa pay for the donkey then ?}
\item{Seven years ago, Rocco's drum kit cost him R$12~500$. It has now been valued at R$2~300$. What rate of simple depreciation does this represent?}
\item{Fiona buys a DStv satellite dish for R$3~000$.  Due to weathering, its value depreciates simply at 15\% per annum. After how long will the satellite dish be worth nothing?}
\end{enumerate}


% Automatically inserted shortcodes - number to insert 4
\practiceinfo
\par \begin{tabular}[h]{cccccc}
% Question 1
(1.)	017h	&
% Question 2
(2.)	017i	&
% Question 3
(3.)	017j	&
% Question 4
(4.)	017k	&
\end{tabular}}
% Automatically inserted shortcodes - number inserted 4

\section{Compound Decay or Reducing-balance depreciation}
The second method of calculating depreciation is to assume that the value of the asset decreases at a certain annual rate, but that the initial value of the asset this year, is the book value of the asset at the end of last year.

For example, if our second hand car has a limited useful life of 5 years and it has an initial value of R$60~000$, then the interest rate of depreciation is $20\%$ ($100\%$/5 years). After 1 year, the car is worth:
\begin{eqnarray*}
\mbox{Book Value after first year} &=& P(1-n\times i)\\
&=&\rm{R}60~000(1-1\times 20\%)\\
&=&\rm{R}60~000(1-0,2)\\
&=&\rm{R}60~000(0,8)\\
&=&\rm{R}48~000
\end{eqnarray*}

At the beginning of the second year, the car is now worth R$48~000$, so after two years, the car is worth:
\begin{eqnarray*}
\mbox{Book Value after second year} &=& P(1-n\times i)\\
&=&\rm{R}48~000(1-1\times 20\%)\\
&=&\rm{R}48~000(1-0,2)\\
&=&\rm{R}48~000(0,8)\\
&=&\rm{R}38~400
\end{eqnarray*}

We can tabulate these values.
\begin{center}
\begin{tabular}{lll}
End of first year&R$60~000(1-1\times20\%)=$R$60~000(1-1\times20\%)^1$ &= R$48~000,00$\\
End of second year&R$48~000(1-1\times20\%)=$R$60~000(1-1\times20\%)^2$&= R$38~400,00$\\
End of third year&R$38~400(1-1\times20\%)=$R$60~000(1-1\times20\%)^3$&= R$30~720,00$\\
End of fourth year&R$30~720(1-1\times20\%)=$R$60~000(1-1\times20\%)^4$&= R$24~576,00$\\
End of fifth year&R$24~576(1-1\times20\%)=$R$60~000(1-1\times20\%)^5$&= R$19~608,80$\\
\end{tabular}
\end{center}
We can now write a general formula for the book value of an asset if the depreciation is compounded.
\begin{equation}
\mbox{Initial Value - Total depreciation after $n$ years} = P(1-i)^n
\end{equation}

For example, the book value of the car after two years can be simply calculated as follows:
\begin{eqnarray*}
\mbox{Book Value after 2 years: } A &=& P(1-i)^n\\
&=&\rm{R}60~000(1-20\%)^2\\
&=&\rm{R}60~000(1-0,2)^2\\
&=&\rm{R}60~000(0,8)^2\\
&=&\rm{R}38~400
\end{eqnarray*}
\mbox{as expected.}

Note that the difference between the compound interest calculations and the compound depreciation calculations is that while the interest adds value to the principal amount, the depreciation amount reduces value!
\clearpage

\begin{wex}{Compound Depreciation}{The flamingo population of the Berg river mouth is depreciating on a reducing balance at a rate of $12\%$ p.a. If there are now $3~200$ flamingos in the wetlands of the Berg river mouth, how many will there be in 5 years' time? Answer to three significant figures.}{
\westep{Determine what has been provided and what is required}
\begin{eqnarray*}
P &=& 3~200\\
i &=& 0,12\\
n &=& 5\\
A~ &\mbox{is}& \mbox{required}
\end{eqnarray*}
\westep{Determine how to approach the problem}
\begin{eqnarray*}
A &=& P(1 - i)^n\\
A &=& 3~200(1-0,12)^5
\end{eqnarray*}
\westep{Solve the problem}
\begin{eqnarray*}
A &=& 3~200(0,88)^5\\
%&=&3~200\times 0,527731916\\
&=& 1688,742134
\end{eqnarray*}
\westep{Write the final answer}
There would be approximately $1~690$ flamingos in 5 years' time.
}
\end{wex}

\begin{wex}{Compound Depreciation}{Farmer Brown buys a tractor for R$250~000$ which depreciates by $20\%$ per year using the compound depreciation method. What is the depreciated value of the tractor after 5 years?\\}{
\westep{Determine what has been provided and what is required}
\begin{eqnarray*}
P &=& R250~000\\
i &=& 0,2\\
n &=& 5\\
A~ &\mbox{is}& \mbox{required}
\end{eqnarray*}
\westep{Determine how to approach the problem}
\begin{eqnarray*}
A &=& P(1-i)^n\\
A &=& 250~000(1-0,2)^5
\end{eqnarray*}
\westep{Solve the problem}
\begin{eqnarray*}
A &=& 250~000(0,8)^5\\
%&=&250~000\times 0,32768\\
&=&81~920
\end{eqnarray*}
\westep{Write the final answer}
Depreciated value after 5 years is R$81~920$}
\end{wex}

\Exercise{Compound Depreciation}{
\begin{enumerate}
\item{On January 1, 2008 the value of my Kia Sorento is R$320~000$.
Each year after that, the car’s value will decrease $20\%$ of the previous year’s value.  What is the value of the car on January 1, 2012?}
\item{The population of Bonduel decreases at a reducing-balance rate of $9,5\%$ per annum as people migrate to the cities.  Calculate the decrease in population over a period of 5 years if the initial population was $2~178~000$.}
\item{A $20\ekg$ watermelon consists of $98\%$ water.  If it is left outside in the sun it loses $3\%$ of its water each day.  How much does it weigh after a month of 31 days?}
\item{A computer depreciates at $x\%$ per annum using the reducing-balance method. Four years ago the value of the computer was R$10~000$ and is now worth R$4~520$.  Calculate the value of $x$ correct to two decimal places.}
\end{enumerate}

% Automatically inserted shortcodes - number to insert 4
\par \practiceinfo
\par \begin{tabular}[h]{cccccc}
% Question 1
(1.)	017m	&
% Question 2
(2.)	017n	&
% Question 3
(3.)	017p	&
% Question 4
(4.)	017q	&
\end{tabular}}
% Automatically inserted shortcodes - number inserted 4

\section{Present and Future Values of an Investment or Loan}
\label{sec:m:f11:presentfuture}
%\begin{syllabus}
%\item Calculation of all variables in $A=P(1-i)^n$ (for $n$ by trial and error using a calculator).
%\end{syllabus}

\subsection{Now or Later}
When we studied simple and compound interest we looked at having a sum of money now, and calculating what it will be worth in the future. Whether the money was borrowed or invested, the calculations examined what the total money would be at some future date. We call these \textit{future values}.

It is also possible, however, to look at a sum of money in the future, and work out what it is worth now. This is called a \textit{present value}.

For example, if R$1~000$ is deposited into a bank account now, the future value is what that amount will accrue to by some specified future date. However, if R$1~000$ is needed at some future time, then the present value can be found by working backwards - in other words, how much must be invested to ensure the money grows to R$1~000$ at that future date?

The equation we have been using so far in compound interest, which relates the open balance ($P$), the closing balance ($A$), the interest rate ($i$ as a rate per annum) and the term ($n$ in years) is:
\begin{equation}
A = P\,.\,(1+i)^n
\label{m:f:closing}
\end{equation}

Using simple algebra, we can solve for $P$ instead of $A$, and come up with:
\begin{equation}
P = A\,.\,(1+i)^{-n}
\label{m:f:opening}
\end{equation}

This can also be written as follows, but the first approach is usually preferred.
\begin{equation}
%P = A / (1+i)^n
P = \frac{A}{(1+i)^n}
\label{m:f:opening2}
\end{equation}

Now think about what is happening here. In Equation \ref{m:f:closing}, we start off with a sum of money and we let it grow for $n$ years. In Equation~\ref{m:f:opening} we have a sum of money which we know in $n$ years time, and we ``unwind" the interest - in other words we take off interest for $n$ years, until we see what it is worth right now.

We can test this as follows. If I have R$1~000$ now and I invest it at $10\%$ for 5 years, I will have:
\begin{eqnarray*}
A &=& P\,.\,(1+i)^n\\
&=&\rm{R}1~000(1+10\%)^5\\
&=&\rm{R}1~610,51
\end{eqnarray*}
at the end. BUT, if I know I have to have R$1 610,51$ in 5 years time, I need to invest:
\begin{eqnarray*}
P &=& A\,.\,(1+i)^{-n}\\
&=&\rm{R}1~610,51(1+10\%)^{-5}\\
&=&\rm{R}1~000
\end{eqnarray*}
We end up with R$1~000$ which - if you think about it for a moment - is what we started off with. Do you see that?

Of course we could apply the same techniques to calculate a present value amount under simple interest rate assumptions - we just need to solve for the opening balance using the equations for simple interest.
\begin{quote}
{\begin{equation}
A = P (1 + i\times n)
\end{equation}

Solving for $P$ gives:
\begin{equation}
%P = A / (1 + i\times n)
P = \frac{A}{(1 + i\times n)}
\end{equation}

Let us say you need to accumulate an amount of R$1~210$ in 3 years time, and a bank account pays \textit{simple interest} of $7\%$. How much would you need to invest in this bank account today?
\begin{eqnarray*}
P &=& \frac{A}{1 + n\,.\, i}\\
&=& \frac{\rm{R}1~210}{ 1 + 3 \times 7\%}\\
&=& \rm{R}1~000
\end{eqnarray*}
Does this look familiar? Look back to the simple interest worked example in Grade 10. There we started with an amount of R$1~000$ and looked at what it would grow to in 3 years' time using simple interest rates. Now we have worked backwards to see what amount we need as an opening balance in order to achieve the closing balance of R$1~210$.}
\end{quote}

In practise, however, present values are usually always calculated assuming compound interest. So unless you are explicitly asked to calculate a present value (or opening balance) using simple interest rates, make sure you use the compound interest rate formula!

\Exercise{Present and Future Values}{
\begin{enumerate}
\item{After a 20-year period Josh's lump sum investment matures to an amount of R$313~550$.  How much did he invest if his money earned interest at a rate of $13,65\%$ p.a.compounded half yearly for the first 10 years, $8,4\%$  p.a. compounded quarterly for the next five years and $7,2\%$ p.a. compounded monthly for the remaining period ?}
\item{A loan has to be returned in two equal semi-annual instalments.  If the rate of interest is $16\%$ per annum, compounded semi-annually and each instalment is R$1~458$, find the sum borrowed.}
\end{enumerate}

% Automatically inserted shortcodes - number to insert 2
\par \practiceinfo
\par \begin{tabular}[h]{cccccc}
% Question 1
(1.)	017r	&
% Question 2
(2.)	017s	&
\end{tabular}}
% Automatically inserted shortcodes - number inserted 2

\section{Finding $i$}
\label{sec:m:f11:interest}
By this stage in your studies of the mathematics of finance, you have always known what interest rate to use in the calculations, and how long the investment or loan will last. You have then either taken a known starting point and calculated a future value, or taken a known future value and calculated a present value.

But here are other questions you might ask:
\begin{enumerate}
\item{I want to borrow R$2~500$ from my neighbour, who said I could pay back R$3~000$ in 8 months time. What interest is she charging me?}
\item{I will need R$450$ for some university textbooks in 1,5 years time. I currently have R$400$. What interest rate do I need to earn to meet this goal?}
\end{enumerate}

Each time that you see something different from what you have seen before, start off with the basic equation that you should recognise very well:
\begin{equation*}
A = P\,.\,(1+i)^n
\end{equation*}

If this were an algebra problem, and you were told to ``solve for $i$", you should be able to show that:
\begin{eqnarray*}
% A/P &=& (1+i)^n\\
% (1+i) &=& (A/P)^{1/n}\\
% i &=& (A/P)^{1/n} - 1
\frac{A}{P} &=& (1 + i)^n\\
\sqrt[n]{\frac{A}{P}} &=& 1 + i\\
\sqrt[n]{\frac{A}{P}} -1 &=& i\\
\therefore i &=& \sqrt[n]{\frac{A}{P}} -1 
\end{eqnarray*}

You do not need to memorise this equation, it is easy to derive any time you need it!

So let us look at the two examples mentioned above.

\begin{enumerate}
\item{Check that you agree that $P=$R$2~500$, $A=$R$3~000$, $n= \frac{8}{12}=\frac{2}{3}$. This means that:
\begin{eqnarray*}
i &=& \sqrt[\frac{2}{3}]{\frac{3000}{2500}} - 1\\
&=& 0,314534...\\
&=& 31,45\%
\end{eqnarray*}
Ouch! That is not a very generous neighbour you have.}

\item{Check that $P=$R$400$, $A=$R$450$, $n=1,5$
\begin{eqnarray*}
i &=& \sqrt[1,5]{\frac{450}{400}} - 1\\
&=& 0,0816871...\\
&=& 8,17\%
\end{eqnarray*}
This means that as long as you can find a bank which pays more than $8,17\%$ interest, you should have the money you need!}
\end{enumerate}

Note that in both examples, we expressed $n$ as a number of years ($\frac{8}{12}$ years, not $8$ because that is the number of months) which means $i$ is the annual interest rate. Always keep this in mind - keep years with years to avoid making silly mistakes.

\Exercise{Finding $i$}{
\begin{enumerate}
\item{A machine costs R$45~000$ and has a scrap value of R$9~000$ after 10 years.  Determine the annual rate of depreciation if it is calculated on the reducing balance method.}
\item{After 5 years an investment doubled in value.  At what annual rate was interest compounded?}
\end{enumerate}

% Automatically inserted shortcodes - number to insert 2
\par \practiceinfo
\par \begin{tabular}[h]{cccccc}
% Question 1
(1.)	017t	&
% Question 2
(2.)	017u	&
\end{tabular}}
% Automatically inserted shortcodes - number inserted 2

\section{Finding $n$ - Trial and Error}
\label{sec:m:f11:term}
By this stage you should be seeing a pattern. We have our standard formula, which has a number of variables:

\begin{equation*}
A = P\,.\,(1+i)^n
\end{equation*}

We have solved for $A$ (in Grade 10), $P$ (in Section \ref{sec:m:f11:presentfuture}) and $i$ (in Section \ref{sec:m:f11:interest}). This time we are going to solve for $n$. In other words, if we know what the starting sum of money is and what it grows to, and if we know what interest rate applies - then we can work out how long the money needs to be invested for all those other numbers to tie up.

This section will calculate $n$ by trial and error and by using a calculator. The proper algebraic solution will be learnt in Grade 12.

Solving for $n$, we can write:
\begin{eqnarray*}
A &=& P(1+i)^n\\
\frac{A}{P}&=&(1+i)^n
\end{eqnarray*}
% Now we have to examine the numbers involved to try to determine what a possible value of $n$ is. Refer to Table~\ref{tab:mn:s:perfectsquarecube} (on page \pageref{tab:mn:s:perfectsquarecube}) for some ideas as to how to go about finding $n$.
Now we have to examine the numbers involved to try to determine what a possible value of $n$ is. Refer to your Grade 10 notes for some ideas as to how to go about finding $n$.


\begin{wex}{Term of Investment - Trial and Error}{We invest R$3~500$ into a savings account which pays $7,5\%$ compound interest for an unknown period of time, at the end of which our account is worth R$4~044,69$. How long did we invest the money?\\}{
\westep{Determine what is given and what is required}
\begin{itemize}
\item{$P=$R$3~500$}
\item{$i=7,5\%$}
\item{$A=$R$4~044,69$}
\end{itemize}
We are required to find $n$.\\
\westep{Determine how to approach the problem}
We know that:
\begin{eqnarray*}
A &=& P(1+i)^n\\
\frac{A}{P}&=&(1+i)^n
\end{eqnarray*}

\westep{Solve the problem}
\begin{eqnarray*}
\frac{\rm{R}4~044,69}{\rm{R}3~500}&=&(1+7,5\%)^n\\
1,156&=&(1,075)^n
\end{eqnarray*}

We now use our calculator and try a few values for $n$.
\begin{center}
\begin{tabular}{|c|c|}\hline\hline
Possible $n$ & $1,075^n$\\
$1,0$ & $1,075$\\
$1,5$ & $1,115$\\
$2,0$ & $1,156$\\
$2,5$ & $1,198$\\
\end{tabular}
\end{center}
We see that $n$ is close to $2$.\\
\westep{Write final answer}
The R$3~500$ was invested for about 2 years.}
\end{wex}

\Exercise{Finding $n$ - Trial and Error}{
\begin{enumerate}
\item{A company buys two types of motor cars: The Acura costs R$80~600$ and the Brata R$101~700$, V.A.T. included. The Acura depreciates at a rate, compounded annually, of $15,3\%$ per year and the Brata at $19,7\%$, also compounded annually, per year.  After how many years will the book value of the two models be the same ?}
\item{The fuel in the tank of a truck decreases every minute by $5,5\%$ of the amount in the tank at that point in time.  Calculate after how many minutes there will be less than $30~l$ in the tank if it originally held $200~l$.}
\end{enumerate}

% Automatically inserted shortcodes - number to insert 2
\par \practiceinfo
\par \begin{tabular}[h]{cccccc}
% Question 1
(1.)	017v	&
% Question 2
(2.)	017w	&
\end{tabular}}
% Automatically inserted shortcodes - number inserted 2

\section{Nominal and Effective Interest Rates}
\label{m:f11:nominal}
%\begin{syllabus}
%\item Demonstrate an understanding of different periods of compounding growth and decay (including effective compounding growth and decay and including effective and nominal interest rates.
%\end{syllabus}

So far we have discussed annual interest rates, where the interest is quoted as a per annum amount. Although it has not been explicitly stated, we have assumed that when the interest is quoted as a per annum amount it means that the interest is paid once a year.

Interest however, may be paid more than just once a year, for example we could receive interest on a monthly basis, i.e. 12 times per year. So how do we compare a monthly interest rate, say, to an annual interest rate? This brings us to the concept of the \textit{effective annual interest rate}.

One way to compare different rates and methods of interest payments would be to compare the closing balances under the different options, for a given opening balance. Another, more widely used, way is to calculate and compare the \textit{effective annual interest rate} on each option. This way, regardless of the differences in how frequently the interest is paid, we can compare apples-with-apples.

For example, a savings account with an opening balance of R$1~000$ offers a compound interest rate of $1\%$ per month which is paid at the end of every month. We can calculate the accumulated balance at the end of the year using the formulae from the previous section. But be careful … our interest rate has been given as a monthly rate, so we need to use the same units (months) for our time period of measurement.

\Tip{Remember, the trick to using the formulae is to define the time period, and use the interest rate relevant to the time period.}

So we can calculate the amount that would be accumulated by the end of 1-year as follows:
\begin{eqnarray*}
\mbox{Closing Balance after 12 months}&=& P \times (1 + i)^n \\
&=& \rm{R}1~000 \times (1 + 1\%)^{12} \\
&=& \rm{R}1~126,83
\end{eqnarray*}
Note that because we are using a monthly time period, we have used $n=12$ months to calculate the balance at the end of one year.

The effective annual interest rate is an annual interest rate which represents the equivalent per annum interest rate assuming compounding.

It is the annual interest rate in our Compound Interest equation that equates to the same accumulated balance after one year. So we need to solve for the effective annual interest rate so that the accumulated balance is equal to our calculated amount of R$1~126,83$.

We use $i_{12}$ to denote the monthly interest rate. We have introduced this notation here to distinguish between the annual interest rate, $i$. Specifically, we need to solve for $i$ in the following equation:
\begin{eqnarray*}
P\times (1 + i) ^1 &=& P\times (1 + i_{12})^{12}\\
(1 + i) &=& (1 + i_{12})^{12}\quad\mbox{divide both sides by $P$}\\
i &=& (1 + i_{12})^{12}-1\quad\mbox{subtract $1$ from both sides}
\end{eqnarray*}

For the example, this means that the effective annual rate for a monthly rate $i_{12}=1\%$ is:
\begin{eqnarray*}
i &=& (1 + i_{12})^{12}-1\\
&=& (1 + 1\%)^{12}-1\\
&=&0,12683\\
&=&12,683\%
\end{eqnarray*}

If we recalculate the closing balance using this annual rate we get:
\begin{eqnarray*}
\mbox{Closing Balance after 1 year}&=& P \times (1 + i)^n \\
&=& \rm{R}1~000 \times (1 + 12,683\%)^{1} \\
&=& \rm{R}1~126,83
\end{eqnarray*}
which is the same as the answer obtained for 12 months.

Note that this is greater than simply multiplying the monthly rate by ($12 \times 1\% = 12\%$) due to the effects of compounding. The difference is due to interest on interest. We have seen this before, but it is an important point!

\subsection{The General Formula}
So we know how to convert a monthly interest rate into an effective annual interest. Similarly, we can convert a quarterly or semi-annual interest rate (or an interest rate of any frequency for that matter) into an effective annual interest rate.

For a quarterly interest rate of say $3\%$ per quarter, the interest will be paid four times per year (every three months). We can calculate the effective annual interest rate by solving for $i$:

\begin{equation*}
P(1 + i ) = P(1 + i_4 )^4
\end{equation*}

where $i_4$ is the quarterly interest rate.

So
$(1 + i ) = (1,03 )^4$ , and so $i = 12,55\%$. This is the effective annual interest rate.

In general, for interest paid at a frequency of $T$ times per annum, the follow equation holds:
\begin{equation}
P(1 + i) = P(1 + i_T)^{T}
\end{equation}
where $i_T$ is the interest rate paid $T$ times per annum.

\subsection{Decoding the Terminology}
Market convention however, is not to state the interest rate as say $1\%$ per month, but rather to express this amount as an annual amount which in this example would be paid monthly. This annual amount is called the nominal amount.

The market convention is to quote a nominal interest rate of ``$12\%$ per annum paid monthly" instead of saying (an effective) $1\%$ per month. We know from a previous example, that a nominal interest rate of $12\%$ per annum paid monthly, equates to an effective annual interest rate of $12,68\%$, and the difference is due to the effects of interest-on-interest.

So if you are given an interest rate expressed as an annual rate but paid more frequently than annual, we first need to calculate the actual interest paid per period in order to calculate the effective annual interest rate.

\begin{equation}
\mbox{monthly interest rate}=\frac{\mbox{Nominal interest Rate per annum}}{\mbox{number of periods per year}}
\end{equation}

For example, the monthly interest rate on $12\%$ interest per annum paid monthly, is:
\begin{eqnarray*}
\mbox{monthly interest rate}&=&\frac{\mbox{Nominal interest Rate per annum}}{\mbox{number of periods per year}}\\
&=& \frac{12\%}{12~\mbox{months}}\\
&=& 1\%~\mbox{per month}
\end{eqnarray*}
The same principle applies to other frequencies of payment.

\begin{wex}{Nominal Interest Rate}{Consider a savings account which pays a nominal interest at $8\%$ per annum, paid quarterly. Calculate (a) the interest amount that is paid each quarter, and (b) the effective annual interest rate.\\}
{\westep{Determine what is given and what is required}
We are given that a savings account has a nominal interest rate of $8\%$ paid quarterly. We are required to find:
\begin{itemize}
\item{the quarterly interest rate, $i_4$}
\item{the effective annual interest rate, $i$\\}
\end{itemize}

\westep{Determine how to approach the problem}
We know that:
\nequ{\mbox{quarterly interest rate}=\frac{\mbox{Nominal interest Rate per annum}}{\mbox{number of quarters per year}}}
and
\nequ{P(1 + i) = P(1 + i_T)^{T}}
where $T$ is 4 because there are 4 payments each year.\\

\westep{Calculate the monthly interest rate}
\begin{eqnarray*}
\mbox{quarterly interest rate}&=&\frac{\mbox{Nominal interest rate per annum}}{\mbox{number of periods per year}}\\
&=& \frac{8\%}{4~\mbox{quarters}}\\
&=& 2\%~\mbox{per quarter}
\end{eqnarray*}

\westep{Calculate the effective annual interest rate}
The effective annual interest rate ($i$) is calculated as:
\begin{eqnarray*}
(1 + i) &=& (1 + i_4)^{4}\\
(1+ i) &=& (1+ 2\%)^4\\
i &=& (1+ 2\%)^4 - 1\\
&=&8,24\%
\end{eqnarray*}
\westep{Write the final answer}
The quarterly interest rate is $2\%$ and the effective annual interest rate is $8,24\%$, for a nominal interest rate of $8\%$ paid quarterly.}
\end{wex}

\begin{wex}{Nominal Interest Rate}
{On their saving accounts, Echo Bank offers an interest rate of $18\%$ nominal,
paid monthly. If you save R$100$ in such an account now, how much would the amount
have accumulated to in 3 years' time?\\}
{
\westep{Determine what is given and what is required}
Interest rate is $18\%$
nominal paid monthly. There are 12 months in a year. We are working with a
yearly time period, so $n = 3$. The amount we have saved is R$100$, so $P =
100$. We need the accumulated value, $A$.

\westep{Recall relevant formulae} We know that
\begin{displaymath}
\mbox{monthly interest rate} =
\frac{\mbox{Nominal interest Rate per annum}}{\mbox{number of periods per
year}}
\end{displaymath}

for converting from nominal interest rate to effective interest rate, we have
\begin{displaymath}
	1 + i = (1 + i_T)^T
\end{displaymath}

and for calculating accumulated value, we have
\begin{displaymath}
	A = P \times (1 + i)^n
\end{displaymath}
	
\westep{Calculate the effective interest rate}
There are 12 month in a year, so
\begin{eqnarray*}
	i_{12} &=& \frac{\mbox{Nominal annual interest rate}}{12}\\
	&=& \frac{18\%}{12}\\
	&=& 1,5\% \mbox{ per month}
\end{eqnarray*}
and then, we have
\begin{eqnarray*}
	1 + i &=& (1 + i_{12})^{12}\\
	i &=& (1 + i_{12})^{12} - 1\\
	&=& (1 + 1,5\%)^{12} - 1\\
	&=& (1,015)^{12} - 1\\
	&=& 19,56\%
\end{eqnarray*}

\westep{Reach the final answer}
\begin{eqnarray*}
	A &=& P \times (1 + i)^n\\
	&=& 100 \times (1 + 19,56\%)^3\\
	&=& 100 \times 1,7091\\
	&=& 170,91
\end{eqnarray*}

\westep{Write the final answer}
The accumulated value is R$170,91$. (Remember to round off to the the nearest cent.)
}
\end{wex}

\Exercise{Nominal and Effect Interest Rates}{
\begin{enumerate}
\item{Calculate the effective rate equivalent to a nominal interest rate of $8,75\%$ p.a. compounded monthly.}
\item{Cebela is quoted a nominal interest rate of $9,15\%$ per annum compounded every four months on her investment of R$85~000$.  Calculate the effective rate per annum.}
\end{enumerate}

% Automatically inserted shortcodes - number to insert 2
\par \practiceinfo
\par \begin{tabular}[h]{cccccc}
% Question 1
(1.)	017x	&
% Question 2
(2.)	017y	&
\end{tabular}}
% Automatically inserted shortcodes - number inserted 2

\section{Formula Sheet}
As an easy reference, here are the key formulae that we derived and used during this chapter. While memorising them is nice (there are not many), it is the application that is useful. Financial experts are not paid a salary in order to recite formulae, they are paid a salary to use the right methods to solve financial problems.

\subsection{Definitions}
\begin{tabular}{ll}
$P$ &Principal (the amount of money at the starting point of the calculation)\\
$i$ &interest rate, normally the effective rate per annum\\
$n$ &period for which the investment is made\\
$i_T$ &the interest rate paid $T$ times per annum, i.e. $i_T = \frac{\mbox{Nominal Interest Rate}}{T}$
\end{tabular}

\subsection{Equations}

\begin{eqnarray*}
Simple~ Increase: A &=& P(1 + i \times n)\\
Compound~ Increase: A &=& P(1 + i)^n\\
Simple~ Decay: A &=& P(1 - i \times n)\\
Compound~ Decay: A &=& P(1 - i)^n\\
Effective~ Annual~ Interest~ Rate (i):  (1 + i) &=& (1 + i_T)^T
\end{eqnarray*}

\begin{eocexercises}{}
\begin{enumerate}
\item{Shrek buys a Mercedes worth R$385~000$ in 2007. What will the value of the Mercedes be at the end of 2013 if:
\begin{enumerate}
\item{the car depreciates at $6\%$ p.a. straight-line depreciation}
\item{the car depreciates at $12\%$ p.a. reducing-balance depreciation.}
\end{enumerate}}

\item{Greg enters into a 5-year hire-purchase agreement to buy a computer for R$8~900$. The interest rate is quoted as $11\%$ per annum based on simple interest. Calculate the required monthly payment for this contract.}

\item{A computer is purchased for R$16 000$. It depreciates at $15\%$ per annum.
\begin{enumerate}
\item{Determine the book value of the computer after 3 years if depreciation
is calculated according to the straight-line method.}
\label{m:f:computer}
\item{Find the rate, according to the reducing-balance method, that would
yield the same book value as in \ref{m:f:computer}) after 3 years.}
\end{enumerate}}

\item{Maggie invests R$12~500,00$ for 5 years at $12\%$ per annum compounded monthly for
the first 2 years and $14\%$ per annum compounded semi-annually for the next 3
years. How much will Maggie receive in total after 5 years?}

\item{Tintin invests R$120 000$. He is quoted a nominal interest rate of $7,2\%$ per annum compounded monthly.
\begin{enumerate}
\item{Calculate the effective rate per annum correct to \textit{three} decimal
places.}
\item{Use the effective rate to calculate the value of Tintin's investment if
he invested the money for 3 years.}
\item{Suppose Tintin invests his money for a total period of 4 years, but
after 18 months makes a withdrawal of R$20~000$, how much will he
receive at the end of the 4 years?}
\end{enumerate}}

\item{Paris opens accounts at a number of clothing stores and spends freely.  She gets herself into terrible debt and she cannot pay off her accounts.  She owes Hilton Fashion world R$5~000$ and the shop agrees to let Paris pay the bill at a nominal interest rate of $24\%$ compounded monthly.
\begin{enumerate}
\item{How much money will she owe Hilton Fashion World after two years?}
\item{What is the effective rate of interest that Hilton Fashion World is charging her?}
\end{enumerate}}
\end{enumerate}



% Automatically inserted shortcodes - number to insert 6
\practiceinfo
\par \begin{tabular}[h]{cccccc}
% Question 1
(1.)	017z	&
% Question 2
(2.)	0180	&
% Question 3
(3.)	0181	&
% Question 4
(4.)	0182	&
% Question 5
(5.)	0183	&
% Question 6
(6.)	0184	\\ % End row of shortcodes
\end{tabular}
% Automatically inserted shortcodes - number inserted 6
\end{eocexercises}



% CHILD SECTION START

