\chapter{Solving Quadratic Inequalities}
\label{m:se:qineq11}

\section{Introduction}
Now that you know how to solve quadratic equations, you are ready to learn how to solve quadratic inequalities.

\section{Quadratic Inequalities}
%\begin{syllabus}
%\item Solve quadratic inequalities in one variable and interpret the solution graphically
%\end{syllabus}

A \emph{quadratic inequality} is an inequality in one of the following forms:
\begin{eqnarray*}
ax^{2} + bx + c > 0\\
ax^{2} + bx + c \geq 0\\
ax^{2} + bx + c < 0\\
ax^{2} + bx + c \leq 0
\end{eqnarray*}

Solving a quadratic inequality corresponds to working out in what region the graph of a quadratic function lies above or below the $x$-axis.

\begin{wex}{Quadratic Inequality}{Solve the inequality $4x^{2} - 4x + 1 \leq 0$ and interpret the solution graphically.\\} 
{
\westep{Factorise the quadratic}
Let $f(x) = 4x^{2} - 4x + 1$. Factorising this quadratic function gives $f(x) = (2x - 1)^{2}$.\\

\westep{Re-write the original equation with factors}
\nequ{(2x - 1)^{2} \leq 0}
\westep{Solve the equation}
$f(x) = 0$ only when $x = \frac{1}{2}$.\\
\westep{Write the final answer}
This means that the graph of $f(x)=4x^{2} - 4x + 1$ touches the $x$-axis at $x=\frac{1}{2}$, but there are no regions where the graph is below the $x$-axis.
\newline
\westep{Graphical interpretation of solution}
% \newline
\begin{center}
\begin{pspicture}(-3,-1)(3,0.4)
%\psgrid[gridcolor=gray]
\psline{<->}(-3,0)(3,0)
\multirput(-2.5,0)(0.5,0){11}{\psline(0,-0.1)(0,0.1)}
\multido{\n=-2+1}{5}
{\usefont{T1}{ptm}{m}{n}\uput[d](\n,-0.2){$\n$}}
\psdot(0.5,0)
\usefont{T1}{ptm}{m}{n}
\uput[u](0.5,0.1){$x=\frac{1}{2}$}
\end{pspicture}
\end{center}
}
\end{wex}

\begin{wex}{Solving Quadratic Inequalities}{Find all the solutions to the inequality $x^{2} - 5x + 6 \geq 0$.\\}
{
\westep{Factorise the quadratic}
The factors of $x^{2} - 5x + 6$ are $(x - 3)(x - 2)$.\\

\westep{Write the inequality with the factors}
\begin{eqnarray*}
x^{2} - 5x + 6&\ge&0\\
(x - 3)(x - 2)&\ge&0
\end{eqnarray*}

\westep{Determine which ranges correspond to the inequality}
We need to figure out which values of $x$ satisfy the inequality. From the answers we have five regions to consider.
\newline
\begin{center}
\begin{pspicture}(0,0)(5,1)
\psline[arrows=<->](0,0.5)(5,0.5)
\psdots[dotsize=5pt](2,0.5)(3,0.5)
\multido{\n=1+1}{4}
{\usefont{T1}{ptm}{m}{n}
\uput[d](\n,0.4){$\n$}
\psline(\n,0.4)(\n,0.6)}
\uput[u](1,0.5){$A$}
\uput[u](2,0.5){$B$}
\uput[u](2.5,0.5){$C$}
\uput[u](3,0.5){$D$}
\uput[u](4,0.5){$E$}
\end{pspicture}
\end{center}

\westep{Determine whether the function is negative or positive in each of the regions}
% \newline
Let $f(x)=x^{2} - 5x + 6$. For each region, choose any point in the region and evaluate the function.
\begin{center}
\begin{tabular}{cccc}
& &$f(x)$ &sign of $f(x)$\\
Region A&$x<2$ &$f(1)=2$ &+\\
Region B&$x=2$ &$f(2)=0$ &+\\
Region C&$2<x<3$ &$f(2,5)= -2,5$ &-\\
Region D&$x=3$ &$f(3)=0$ &+\\
Region E&$x>3$ &$f(4)=2$ &+\\
\end{tabular}
\end{center}
We see that the function is positive for $x\le2$ and $x\ge 3$.\\
% \newline
\westep{Write the final answer and represent on a number line}

We see that $x^{2} - 5x + 6 \geq 0$ is true for $x\le2$ and $x\ge 3$.

\begin{center}
\begin{pspicture}(0,0)(5,1)
\psline[arrows=<->](0,0.5)(5,0.5)
\psdots[dotsize=5pt](2,0.5)(3,0.5)
\multido{\n=1+1}{4}
{\usefont{T1}{ptm}{m}{n}
\uput[d](\n,0.4){$\n$}
\psline(\n,0.4)(\n,0.6)}
\psline[linewidth=3pt]{->}(2,0.5)(0,0.5)
\psline[linewidth=3pt]{->}(3,0.5)(5,0.5)
\end{pspicture}
\end{center}
}
\end{wex}

\begin{wex}{Solving Quadratic Inequalities}{Solve the quadratic inequality $-x^{2} - 3x + 5 > 0$.\\}{
\westep{Determine how to approach the problem}
Let $f(x) = -x^{2} - 3x + 5$. $f(x)$ cannot be factorised so, use the quadratic formula to determine the roots of $f(x)$. The $x$-intercepts are solutions to the quadratic equation
\begin{eqnarray*}
-x^{2}- 3x + 5 &=& 0 \\
x^{2} + 3x - 5 &=& 0\\
\therefore x &=& \frac{-3 \pm \sqrt{(3)^{2} - 4(1)(-5)}}{2(1)} \\
&=& \frac{-3 \pm \sqrt{29}}{2}\\
x_1 &=& \frac{-3 - \sqrt{29}}{2} = -4,2\\
x_2 &=& \frac{-3 + \sqrt{29}}{2} = 1,2
\end{eqnarray*}

\westep{Determine which ranges correspond to the inequality}
We need to figure out which values of $x$ satisfy the inequality. From the answers we have five regions to consider.

\begin{center}
\begin{pspicture}(0,0)(5,1)
\psline[arrows=<->](0,0.5)(5,0.5)
\psdots[dotsize=5pt](2,0.5)(3,0.5)
\multido{\n=1+1}{4}
{%\uput[d](\n,0.4){\n}
\psline(\n,0.4)(\n,0.6)}
\uput[u](1,0.5){$A$}
\uput[u](2,0.5){$B$}
\uput[d](2,0.5){$-4,2$}
\uput[u](2.5,0.5){$C$}
\uput[u](3,0.5){$D$}
\uput[d](3,0.5){$1,2$}
\uput[u](4,0.5){$E$}
\end{pspicture}
\end{center}

\westep{Determine whether the function is negative or positive in each of the regions}
We can use another method to determine the sign of the function over different regions, by drawing a rough sketch of the graph of the function. We know that the roots of the function correspond to the $x$-intercepts of the graph. Let $g(x)=-x^{2} - 3x + 5$. We can see that this is a parabola with a maximum turning point that intersects the $x$-axis at $-4,2$ and $1,2$.\\

\begin{center}
\begin{pspicture}(-3.8,-2)(1.6,5.5)
%\psgrid
\psset{unit=0.75}
\psaxes{<->}(0,0)(-5,-2)(2,8)
\psplot{-4.5}{1.5}{x 2 exp neg x 3 mul sub 5 add}
\uput[ul](-4.2,0){$x_1$}
\uput[ur](1.2,0){$x_2$}
\end{pspicture}
\end{center}
It is clear that $g(x)>0$ for $x_1<x<x_2$\\

\westep{Write the final answer and represent the solution graphically}
$-x^{2} - 3x + 5>0$ for $-4,2<x<1,2$

\begin{center}
\begin{pspicture}(0,0)(5,1)
\psline[arrows=<->](0,0.5)(5,0.5)
\multido{\n=1+1}{4}
{%\uput[d](\n,0.4){\n}
\psline(\n,0.4)(\n,0.6)}
\uput[d](2,0.5){$-4,2$}
\uput[d](3,0.5){$1,2$}
\psline[linewidth=3pt](2,0.5)(3,0.5)
\pscircle[fillstyle=solid,fillcolor=white](2,0.5){0.1}
\pscircle[fillstyle=solid,fillcolor=white](3,0.5){0.1}
\end{pspicture}
\end{center}

}\end{wex}

When working with an inequality in which the variable is in the denominator, a different approach is needed.

\begin{wex}{Non-linear inequality with the variable in the denominator}
{Solve $\dfrac{2}{x+3} \le \dfrac{1}{x-3}$\\}
{
\westep{Subtract $\frac{1}{x-3}$ from both sides}
\nequ{\dfrac{2}{x+3} - \dfrac{1}{x-3} \le 0}
\westep{Simplify the fraction by finding LCD}
\begin{eqnarray*}
\dfrac{2(x-3)-(x+3)}{(x+3)(x-3)} \le 0 \\
\dfrac{x-9}{(x+3)(x-3)} \le 0
\end{eqnarray*}
\westep{Draw a number line for the inequality}

\begin{center}
% Generated with LaTeXDraw 1.9.3
% Tue Jan 01 20:09:06 CAT 2008
% \usepackage[usenames,dvipsnames]{pstricks}
% \usepackage{epsfig}
% \usepackage{pst-grad} % For gradients
% \usepackage{pst-plot} % For axes
\scalebox{1} % Change this value to rescale the drawing.
{
\begin{pspicture}(0,-0.66609377)(9.92,0.66609377)
\psline[linewidth=0.04cm](0.0,0.11890625)(9.9,0.13890626)
\psline[linewidth=0.04cm](1.14,0.09890625)(1.14,-0.24109375)
\psline[linewidth=0.04cm](3.8,0.13890626)(3.8,-0.22109374)
\psline[linewidth=0.04cm](8.9,0.13890626)(8.9,-0.22109374)
% \usefont{T1}{ptm}{m}{n}
\rput(1.1115625,0.38890624){undef}
% \usefont{T1}{ptm}{m}{n}
\rput(3.7915626,0.38890624){undef}
\usefont{T1}{ptm}{m}{n}
\rput(1.0614063,-0.47109374){$-3$}
\usefont{T1}{ptm}{m}{n}
\rput(3.7864063,-0.47109374){$3$}
\usefont{T1}{ptm}{m}{n}
\rput(8.870469,-0.51109374){$9$}
\usefont{T1}{ptm}{m}{n}
\rput(9.492657,0.34890625){$+$}
\usefont{T1}{ptm}{m}{n}
\rput(8.870625,0.46890625){$0$}
\usefont{T1}{ptm}{m}{n}
\rput(6.487344,0.38890624){$-$}
\usefont{T1}{ptm}{m}{n}
\rput(2.3926563,0.34890625){$+$}
\usefont{T1}{ptm}{m}{n}
\rput(0.20734376,0.36890626){$-$}
\end{pspicture} 
}
\end{center}
We see that the expression is negative for $x < -3$  or $3 < x \le 9$.\\
\westep{Write the final answer}
\nequ{x < -3 \quad or  \quad 3 < x \le 9}
}
\end{wex}

%Khan Academy video on quadratic inequalities - SIYAVULA-VIDEO:http://cnx.org/content/m30838/latest/#quadratic-inequalities-1
\begin{eocexercises}{}
Solve the following inequalities and show your answer on a number line:
\begin{enumerate}
\item{Solve: $x^2-x<12$.}
\item{Solve: $3x^2 > -x + 4$}
\item{Solve: $y^2 < -y - 2$}
\item{Solve: $-t^2 + 2t > -3$}
\item{Solve: $s^2 - 4s > -6$}
\item{Solve: $0\geq7x^2-x+8$}
\item{Solve: $0\geq -4x^2-x$}
\item{Solve: $0\geq6x^2$}
\item{Solve: $2x^2 + x + 6\leq0$}
\item{Solve for $x$ if: $\dfrac{x}{x-3} < 2$ and $x \neq 3$.}
\item{Solve for $x$ if: $\dfrac{4}{x-3} \leq 1$.}
\item{Solve for $x$ if: $\dfrac{4}{(x-3)^2} < 1$.}
\item{Solve for $x$: $\dfrac{2x-2}{x-3} > 3$}
\item{Solve for $x$: $\dfrac{-3}{(x-3)(x+1)} <0$}
\item{Solve: $(2x-3)^2 < 4$}
\item{Solve: $2x \leq \dfrac{15-x}{x}$}
\item{Solve for $x$: \quad $\dfrac{x^2 + 3}{3x - 2} \leq 0$}
\item{Solve: $x-2 \geq \dfrac{3}{x}$}
\item{Solve for $x$: $\dfrac{x^2+3x-4}{5+x^4} \leq 0$}
\item{Determine all real solutions: $\dfrac{x-2}{3-x} \geq 1$}
\end{enumerate}



\practiceinfo
\end{eocexercises}



% CHILD SECTION START

