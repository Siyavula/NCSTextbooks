\chapter{Independent and Dependent Events}
\label{m:p11}


%\begin{syllabus}
%\item Correctly identify dependent and independent events (e.g. from two-way contingency tables or Venn diagrams) and therefore appreciate when it is appropriate to calculate the probability of two independent events occurring by applying the product rule for independent events: P(A and B) = P(A).P(B).
%\item Use tree and Venn diagrams to solve probability problems (where events are not necessarily independent).
%\end{syllabus}

\section{Introduction}
In probability theory event are either independent or dependent. This chapter describes the differences and how each type of event is worked with.

\section{Definitions}
Two events are independent if knowing something about the value of one event does not give any information about the value of the second event. For example, the event of getting a "$1$" when a die is rolled and the event of getting a "$1$" the second time it is thrown are independent.

% \Definition{Independent Events}{Two events $A$ and $B$ are independent if when one of them happens, it doesn't affect whether the one happens or not.}

The probability of two independent events occurring, $P(A \cap B)$, is given by:
\equ{P(A \cap B) = P(A)\times P(B)}{eq:independentproduct}

\Definition{Independent events}{Events are said to be independent if the result or outcome of one event does not affect the result or outcome of the other event. So $P(A/C)=P(A)$, where $P(A/C)$ represents the probability of event $A$ after event $C$ has occured.}

\begin{wex}{Independent Events}{What is the probability of rolling a $1$ and then rolling a $6$ on a fair die?\\}{
\westep{Identify the two events and determine whether the events are independent or not}

Event $A$ is rolling a $1$ and event $B$ is rolling a $6$. Since the outcome of the first event does not affect the outcome of the second event, the events are independent.\\

\westep{Determine the probability of the specific outcomes occurring, for each event}

The probability of rolling a $1$ is $\frac{1}{6}$ and the probability of rolling a $6$ is $\frac{1}{6}$.

Therefore, $P(A)=\frac{1}{6}$ and $P(B)=\frac{1}{6}$.\\

\westep{Use equation~\ref{eq:independentproduct} to determine the probability of the two events occurring together.}

\begin{eqnarray*}
P(A \cap B) &=& P(A)\times P(B)\\
&=&\frac{1}{6} \times \frac{1}{6}\\
&=&\frac{1}{36}\\
\end{eqnarray*}

The probability of rolling a $1$ and then rolling a $6$ on a fair die is $\frac{1}{36}$.}
\end{wex}

Consequently, two events are dependent if the outcome of the first event affects the outcome of the second event.

\Definition{Dependent events}{Two events are dependent if the outcome of one event is affected by the outcome of the other event i.e. $P(A/C)\neq P(A)$.}

\begin{wex}{Dependent Events}{A cloth bag has four coins, one R$1$ coin, two R$2$ coins and one R$5$ coin. What is the probability of fisrt selecting a R$1$ coin and then selecting a R$2$ coin?\\}{
\westep{Identify the two events and determine whether the events are independent or not}

Event $A$ is selecting a R$1$ coin and event $B$ is next selecting a R$2$. Since the outcome of the first event affects the outcome of the second event (because there are less coins to choose from after the first coin has been selected), the events are dependent.\\

\westep{Determine the probability of the specific outcomes occurring, for each event}

The probability of first selecting a R$1$ coin is $\frac{1}{4}$ and the probability of next selecting a R$2$ coin is $\frac{2}{3}$ (because after the R$1$ coin has been selected, there are only three coins to choose from).

Therefore, $P(A)=\frac{1}{4}$ and $P(B)=\frac{2}{3}$.\\

\westep{Use equation~\ref{eq:independentproduct} to determine the probability of the two events occurring together.}

The same equation as for independent events are used, but the probabilities are calculated differently.

\begin{eqnarray*}
P(A \cap B) &=& P(A)\times P(B)\\
&=&\frac{1}{4} \times \frac{2}{3}\\
&=&\frac{2}{12}\\
&=&\frac{1}{6}\\
\end{eqnarray*}

The probability of first selecting a R$1$ coin followed by selecting a R$2$ coin is $\frac{1}{6}$.}
\end{wex}

\subsection{Identification of Independent and Dependent Events}
\subsubsection{Use of a Contingency Table}
A two-way contingency table (studied in an earlier grade) can be used to determine whether events are independent or dependent.

\Definition{two-way contingency table}{A two-way contingency table is used to represent possible outcomes when two events are combined in a statistical analysis.}

For example we can draw and analyse a two-way contingency table to solve the following problem.
\begin{wex}{Contingency Tables}
{A medical trial into the effectiveness of a new medication was carried out. $120$ males and $90$ females responded. Out of these $50$ males and $40$ females responded positively to the medication. 
\begin{enumerate}
\item Was the medication's success independent of gender? Explain.
\item Give a table for the independence of gender results.
\end{enumerate}}
{
\westep{Draw a contingency table}
\begin{center}
\begin{tabular}{|c|c|c|c|}
\hline
                   & Male & Female & Totals \\
\hline
Positive result    & $50$   & $40  $   & $90   $  \\
No Positive result & $70  $ & $50 $    & $120 $   \\
\hline
Totals             & $120$  & $90$     & $210$    \\
\hline
\end{tabular}
\end{center}
\westep{Work out probabilities}
\begin{align*}
\text{P(male).P(positive result)} =\frac{120}{210}=0,57\\
\newline
\text{P(female).P(positive result)} =\frac{90}{210}=0,43\\
\newline
\text{P(male and positive result)} =\frac{50}{210}=0,24
\end{align*}
\westep{Draw conclusion}
P(male and positive result) is the observed probability and P(male).P(positive result) is the expected probability. These two are quite different. So there is no evidence that the medication's success is independent of gender.\\
\westep{Gender-independent results}
To get gender independence we need the positive results in the same ratio as the gender. The gender ratio is: $120:90$, or $4:3$, so the number in the male and positive column would have to be $\frac{4}{7}$ of the total number of patients responding positively which gives $51,4$. This leads to the following table:
\begin{center}
\begin{tabular}{|c|c|c|c|}
\hline
                   & Male & Female & Totals \\
\hline
Positive result    & $51,4$   & $38,6$     & $90$     \\
No Positive result & $68,6$   & $51,4$     & $120$    \\
\hline
Totals             & $120$  & $9$0     & $210$    \\
\hline
\end{tabular}
\end{center}
}
\end{wex}

\subsubsection{Use of a Venn Diagram}
We can also use Venn diagrams to check whether events are dependent or independent. 

Also note that $P(A/C) = \frac{P(A\cap C)}{P(C)}$.
For example, we can draw a Venn diagram and a contingency table to illustrate and analyse the following example.
\begin{wex}{Venn diagrams and events}
{
A school decided that its uniform needed upgrading. The colours on offer were beige or blue or beige and blue. $40\%$ of the school wanted beige, $55\%$ wanted blue and $15\%$said a combination would be fine. Are the two events independent?\\
}
{
\westep{Draw a Venn diagram}
%\scalebox{1} % Change this value to rescale the drawing.
{
\begin{pspicture}(0,-3.49)(10.08,3.49)
\psframe[linewidth=0.04,dimen=outer](10.08,3.49)(0.0,-3.49)
\pscircle[linewidth=0.04,dimen=outer](3.31,-0.04){2.31}
\pscircle[linewidth=0.04,dimen=outer](6.93,-0.04){2.33}
% \usefont{T1}{ptm}{m}{n}
\rput(3.2478125,2.55){\small Beige}
% \usefont{T1}{ptm}{m}{n}
\rput(6.9478126,2.51){\small Blue}
% \usefont{T1}{ptm}{m}{n}
\rput(3.0676563,0.01){\small $0,25$}
% \usefont{T1}{ptm}{m}{n}
\rput(5.027656,-0.01){\small $0,15$}
% \usefont{T1}{ptm}{m}{n}
\rput(7.0829687,0.03){\small $0,4$}
% \usefont{T1}{ptm}{m}{n}
\rput(9.488125,-3.07){\small $0,2$}
% \usefont{T1}{ptm}{m}{n}
\rput(0.39484376,3.07){\small $S$}
\end{pspicture} 
}
\westep{Draw up a contingency table}
\begin{center}
\begin{tabular}{|c|c|c|c|}
\hline
         & Beige & Not Beige & Totals \\
\hline
Blue     & $0,15$  & $0,4$       & $0,55 $  \\
Not Blue & $0,25 $ & $0,2$       & $0,35$   \\
\hline
Totals   & $0,40$  & $0,6$       & $1$      \\
\hline
\end{tabular}
\end{center}
\westep{Work out the probabilities}
$P$(Blue)$=0,4$; $P$(Beige)$=0,55$; $P$(Both)$=0,15$; $P$(Neither)$=0,20$ \newline
Probability of choosing beige after blue is: \newline
\begin{eqnarray*}
P\left(\frac{\mbox{Beige}}{\mbox{Blue}}\right) & = & \frac{P\mbox{(Beige} \cap \mbox{Blue})}{P\mbox{(Blue)}} \\
& = & \frac{0,15}{0,55} \\
& = & 0,27 \\ 
\end{eqnarray*}
\westep{Solve the problem}
Since $P\left(\frac{\mbox{Beige}}{\mbox{Blue}}\right)$ the events are statistically dependent.
}
\end{wex}

\Extension{Applications of Probability Theory}
{
Two major applications of probability theory in everyday life are in risk assessment and in trade on commodity markets. Governments typically apply probability methods in environmental regulation where it is called ``pathway analysis'', and are often measuring well-being using methods that are stochastic in nature, and choosing projects to undertake based on statistical analyses of their probable effect on the population as a whole. It is not correct to say that statistics are involved in the modelling itself, as typically the assessments of risk are one-time and thus require more fundamental probability models, e.g. ``the probability of another 9/11''. A law of small numbers tends to apply to all such choices and perception of the effect of such choices, which makes probability measures a political matter.

A good example is the effect of the perceived probability of any widespread Middle East conflict on oil prices - which have ripple effects in the economy as a whole. An assessment by a commodity trade that a war is more likely vs. less likely sends prices up or down, and signals other traders of that opinion. Accordingly, the probabilities are not assessed independently nor necessarily very rationally. The theory of behavioral finance emerged to describe the effect of such groupthink on pricing, on policy, and on peace and conflict.

It can reasonably be said that the discovery of rigorous methods to assess and combine probability assessments has had a profound effect on modern society. A good example is the application of game theory, itself based strictly on probability, to the Cold War and the mutual assured destruction doctrine. Accordingly, it may be of some importance to most citizens to understand how odds and probability assessments are made, and how they contribute to reputations and to decisions, especially in a democracy.

Another significant application of probability theory in everyday life is reliability. Many consumer products, such as automobiles and consumer electronics, utilize reliability theory in the design of the product in order to reduce the probability of failure. The probability of failure is also closely associated with the product's warranty.
}

\begin{eocexercises}{}
\begin{enumerate}
\item In each of the following contingency tables give the expected numbers for the events to be perfectly independent and decide if the events are independent or dependent.
	\begin{enumerate}
	\item \begin{center}
\begin{tabular}{|c|c|c|c|}
\hline
           & Brown eyes & Not Brown eyes & Totals \\
\hline
Black hair & $50$         & $30$             & $80 $    \\
Red hair   & $70$         & $80$             & $150 $   \\
\hline
Totals     & $120$        & $110 $           & $230 $   \\
\hline
\end{tabular}
\end{center}

	\item \begin{center}
\begin{tabular}{|c|c|c|c|}
\hline
                    & Point A & Point B & Totals \\
\hline
Busses left late    & $15$      & $40$      & $55$     \\
Busses left on time & $25$      & $20 $     & $45$     \\
\hline
Totals              & $40 $     & $60$      & $100$    \\
\hline
\end{tabular}
\end{center}

	\item \begin{center}
\begin{tabular}{|c|c|c|c|}
\hline
                          & Durban & Bloemfontein & Totals \\
\hline
Liked living there        & $130$    & $30 $          & $160$    \\
Did not like living there & $140 $   & $200 $         & $340$    \\
\hline
Totals                    & $270$   &$ 230 $         & $500$    \\
\hline
\end{tabular}
\end{center}

	\item \begin{center}
\begin{tabular}{|c|c|c|c|}
\hline
                         & Multivitamin A & Multivitamin B & Totals \\
\hline
Improvement in health    & $400$            & $300$            & $700$    \\
No improvement in health & $140 $           &$ 120 $           & $260 $   \\
\hline
Totals                   & $540 $           & $420$            & $960$    \\
\hline
\end{tabular}

\end{center}
	\end{enumerate}
%\item A company has a probability of 0.4 of meeting their quota on time and a probability of 0.25 of meeting their quota late. Also there is a 0.10 chance of not meeting their quota on time. Use a Venn diagram and a contingency table to show the information and decide if the events are independent.
\item A study was undertaken to see how many people in Port Elizabeth owned either a Volkswagen or a Toyota. $3\%$ owned both, $25\%$ owned a Toyota and $60\%$ owned a Volkswagen. Draw a contingency table to show all events and decide if car ownership is independent.
\item Jane invested in the stock market. The probability that she will not lose all her money is 0,32. What is the probability that she will lose all her money? Explain.
\item If $D$ and $ F$ are mutually exclusive events, with $P(D')=0,3$ and $P(D$ or $F)=0,94$, find $P(F)$.
\item A car sales person has pink, lime-green and purple models of car $A$ and purple, orange and multicolour models of car $B$. One dark night a thief steals a car. 
	\begin{enumerate}
	\item What is the experiment and sample space? 
	\item Draw a Venn diagram to show this.
	\item What is the probability of stealing either a model of $A$ or a model of $B$?
	\item What is the probability of stealing both a model of $A$ and a model of $B$?
	\end{enumerate}
\item The probability of Event $X$ is $0,43$ and the probability of Event $Y$ is $0,24$. The probability of both occuring together is $0,10$. What is the probability that $X$ or $Y$ will occur (this inculdes $X$ and $Y$ occuring simultaneously)?  
\item$ P(H)=0,62$; $P(J)=0,39$ and$ P(H$ and $J)=0,31$. Calculate:
	\begin{enumerate}
	\item $P(H')$ 
	\item $P(H$ or $J)$
	\item $P(H'$ or $J')$
	\item $P(H'$ or $J)$
	\item $P(H'$ and $J')$
	\end{enumerate}
\item The last ten letters of the alphabet were placed in a hat and people were asked to pick one of them. Event $D$ is picking a vowel, Event $E$ is picking a consonant and Event $F$ is picking the last four letters. Calculate the following probabilities:
	\begin{enumerate}
	\item $P(F')$ 
	\item $P(F$ or $D)$
	\item $P($neither $E$ nor $F)$
	\item $P(D$ and $E)$
	\item $P(E$ and $F)$
	\item $P(E$ and $D')$
	\end{enumerate}
\item At Dawnview High there are $400$ Grade 12's. $270$ do Computer Science, $300$ do English and $50$ do Typing. All those doing Computer Science do English, $20$ take Computer Science and Typing and $35$ take English and Typing. Using a Venn diagram calculate the probability that a pupil drawn at random will take:
	\begin{enumerate}
	\item English, but not Typing or Computer Science 
	\item English but not Typing
	\item English and Typing but not Computer Science
	\item English or Typing
	\end{enumerate}   
\end{enumerate}


\insertpracticeinfo{10}
\end{eocexercises} 






% CHILD SECTION START 


\appendix
