\chapter{Quadratic Sequences}
\label{m:pin:g11}

%\begin{syllabus}
%\item Investigate number patterns (including but not limited to those where there is a constant second difference between consecutive terms in a number pattern, and the general term is therefore quadratic) and hence:
%\begin{itemize}
%\item make conjectures and generalisations
%\item provide explanations and justifications and attempt to prove conjectures.
%\end{itemize}
%\end{syllabus}

\section{Introduction}
In Grade 10 you learned about arithmetic sequences, where the
difference between consecutive terms is constant. In this chapter we
learn about quadratic sequences, where the difference between
consecutive terms is not constant, but follows its own pattern.

\chapterstartvideo{aaa}

\section{What is a \textit{Quadratic Sequence}?}

\Definition{Quadratic Sequence}{A quadratic sequence is a sequence of numbers in which the second difference between each consecutive term is constant. This called a common second difference.}

For example, 

\begin{equation}
\label{eq:mp:s:quadseq:1}
1; \: 2; \: 4; \: 7; \: 11; \: \ldots
\end{equation}
is a quadratic sequence. Let us see why ... 

The first difference is calculated by finding the difference between consecutive terms:
% \begin{eqnarray*}
% a_2 - a_1 &= 2 - 1 &= 1 \\
% a_3 - a_2 &= 4 - 2 &= 2 \\
% a_4 - a_3 &= 7 - 4 &= 3 \\
% a_5 - a_4 &= 11 - 7 &= 4
% \end{eqnarray*}

\begin{center}
% \begin{tikzpicture}[>=stealth,sloped]
%      \matrix (tree) [%
%        matrix of nodes,
%        minimum size=0.5cm,
%        column sep=0.5cm,
%        row sep=1cm,
%      ]
%      {
%  	1 &    & 2 &    & 4 &    & 7 &    & 11\\
%  	  & +1 &   & +2 &   & +3 &   & +4 &   \\         
%      };
% % -------------(FROM: tree-row-column) -- (TO: tree-row-column) node [.....] {LABEL};
%      \draw[-] (tree-2-2) -- (tree-1-1) node [midway,above] {};
%      \draw[-] (tree-2-2) -- (tree-1-3) node [midway,below] {};
%      \draw[-] (tree-2-4) -- (tree-1-3) node [midway,above] {};
%      \draw[-] (tree-2-4) -- (tree-1-5) node [midway,below] {};
%      \draw[-] (tree-2-6) -- (tree-1-5) node [midway,below] {};
%      \draw[-] (tree-2-6) -- (tree-1-7) node [midway,below] {};
%      \draw[-] (tree-2-8) -- (tree-1-7) node [midway,below] {};
%      \draw[-] (tree-2-8) -- (tree-1-9) node [midway,below] {};
% 
% \end{tikzpicture}
% \rput(5.7,-0.5){
\psmatrix[colsep=0.3cm,rowsep=0.3cm]
 	1 &    & 2 &    & 4 &    & 7 &    & 11\\
	  & +1 &   & +2 &   & +3 &   & +4\\
\endpsmatrix
\psset{nodesep=3pt,arrows=-}
\ncline{1,1}{2,2}
\ncline{1,3}{2,2}
\ncline{1,3}{2,4}
\ncline{1,5}{2,4}
\ncline{1,5}{2,6}
\ncline{1,7}{2,6}
\ncline{1,7}{2,8}
\ncline{1,9}{2,8}
% }
% \vspace{1.2cm}

\end{center}

We then work out the \textit{second differences}, which are simply obtained by taking the difference between the consecutive differences \{$1; \: 2; \: 3; \: 4; \: \ldots $\} obtained above:
%\begin{eqnarray*}
%2 - 1 &=& 1 \\
%3 - 2 &=& 1 \\
%4 - 3 &=& 1 \\
%\ldots
%\end{eqnarray*}

\begin{center}
% \rput(5.7,-0.5){
\psmatrix[colsep=0.3cm,rowsep=0.3cm]
 	1 &    & 2 &	 & 3 &     & 4 &\\
	  & +1 &    & +1 &    & +1 &\\
\endpsmatrix
\psset{nodesep=3pt,arrows=-}
\ncline{1,1}{2,2}
\ncline{1,3}{2,2}
\ncline{1,3}{2,4}
\ncline{1,5}{2,4}
\ncline{1,5}{2,6}
\ncline{1,7}{2,6}
% }
% \vspace{1.2cm}
\end{center}

We then see that the second differences are equal to $1$. Thus, Equation (\ref{eq:mp:s:quadseq:1}) is a \textit{quadratic sequence}. 

Note that the differences between consecutive terms (that is, the first differences) of a quadratic sequence form a sequence where there is a constant difference between consecutive terms. In the above example, the sequence of \{$1; \: 2; \: 3; \: 4; \: \ldots $\}, which is formed by taking the differences between consecutive terms of Equation (\ref{eq:mp:s:quadseq:1}), has a linear formula of the kind $ax+b$. 

\Exercise{Quadratic Sequences}{The following are examples of quadratic
sequences:
\begin{enumerate}[label=\textbf{\arabic*}.]
\item $3; \: 6; \: 10; \: 15; \: 21; \: \ldots\\$
\item $4; \: 9; \: 16; \: 25; \: 36; \: \ldots\\$
\item $7; \: 17; \: 31; \: 49; \: 71; \: \ldots\\$
\item $2; \: 10; \: 26; \: 50; \: 82; \: \ldots\\$
\item $31; \: 30; \: 27; \: 22; \: 15; \: \ldots$
\end{enumerate}

Calculate the common second difference for each of the above examples.


% Automatically inserted shortcodes - number to insert 0
\par \practiceinfo
\par \begin{tabular}[h]{cccccc}
(1.) aaa & (2.) aaa & (3.) aaa & (4.) aaa & (5.) aaa &
\end{tabular}}
% Automatically inserted shortcodes - number inserted 0

% -----------------------------------------------------------------

\subsubsection{General Case}
If the sequence is quadratic, the $n^{\rm th}$ term should be $T_n = an^2 + bn + c$

%....................USING TABLES..............................
% \begin{center}
% \begin{tabular}{ccccccc}
% TERMS & $a+b+c$ && $4a+2b+c$ && $9a+3b+c$ & \\
% $1^{\textsf{st}}$ difference && $3a+b$ && $5a+b$ && $7a+b$ \\ 
% $2^{\textsf{nd}}$ difference &&& $2a$ && $2a$ & \\
% \end{tabular}
% \end{center}

\begin{center}
$
\psmatrix[colsep=0.3cm,rowsep=0.3cm]
 	TERMS 				& a+b+c &      & 4a+2b+c &      & 9a+3b+c &      & 16a+4b+c &\\
 	1^{\textsf{st}} \mbox{ difference}	&	& 3a+b &         & 5a+b &         & 7a+b &\\
	2^{\textsf{nd}} \mbox{ difference}	&	&      &    2a   &      &    2a   &
\endpsmatrix
\psset{nodesep=3pt,arrows=-}
\ncline{1,2}{2,3}
\ncline{1,4}{2,3}
\ncline{1,4}{2,5}
\ncline{1,6}{2,5}
\ncline{1,6}{2,7}
\ncline{1,8}{2,7}
\ncline{2,3}{3,4}
\ncline{2,5}{3,4}
\ncline{2,5}{3,6}
\ncline{2,7}{3,6}
$
\end{center}

In each case, the second difference is $2a$.
This fact can be used to find $a$, then $b$ then $c$.

%%**********************EXAMPLE**************************

\begin{wex}{Quadratic sequence}%
% Question
{Write down the next two terms and find a formula for the $n^{\rm th}$ term of the sequence $5; 12; 23; 38;\ldots$}%
% Answer
{ 
%%-----------------------step---------------------------
\westep{Find the first differences between the terms} 

% %..................USING EQUATIONS...................
% \begin{eqnarray*}
% a_2 - a_1 &=& 12 - ~5 = 7\\
% a_3 - a_2 &=& 23 - 12 = 11\\
% a_4 - a_3 &=& 38 - 23 = 15
% \end{eqnarray*}

\rput(5.7,-0.5){
\psmatrix[colsep=0.3cm,rowsep=0.3cm]
 	5 &    & 12 &	  & 23 &     & 38 &\\
	  & +7 &    & +11 &    & +15 &\\
\endpsmatrix
\psset{nodesep=3pt,arrows=-}
\ncline{1,1}{2,2}
\ncline{1,3}{2,2}
\ncline{1,3}{2,4}
\ncline{1,5}{2,4}
\ncline{1,5}{2,6}
\ncline{1,7}{2,6}
}
\vspace{1.2cm}

i.e. $7 ; 11; 15$.\\
%%-----------------------step---------------------------
\westep{Find the second differences between the terms}

% %..................USING EQUATIONS...................
% \begin{eqnarray*}
% 11 - 7 = 4\\
% 15 - 11 = 4
% \end{eqnarray*}

\rput(5.7,-0.5){
\psmatrix[colsep=0.3cm,rowsep=0.3cm]
 	7 &    & 11 &	  & 15 & \\
	  & +4 &    & +4  & \\
\endpsmatrix
\psset{nodesep=3pt,arrows=-}
\ncline{1,1}{2,2}
\ncline{1,3}{2,2}
\ncline{1,3}{2,4}
\ncline{1,5}{2,4}
}
\vspace{1.2cm}

So the second difference is $4$.

Continuing the sequence, the differences between each term will be:

% %..................USING EQUATIONS...................
% \begin{eqnarray*}
% 15 + 4 = 19\\
% 19 + 4 = 23
% \end{eqnarray*}

\rput(5.7,-0.5){
\psmatrix[colsep=0.3cm,rowsep=0.3cm]
     ...15 &    & 19 &	   & 23... & \\
	   & +4 &    & +4  & \\
\endpsmatrix
\psset{nodesep=3pt,arrows=-}
\ncline{1,1}{2,2}
\ncline{2,2}{1,3}
\ncline{1,3}{2,4}
\ncline{2,4}{1,5}
}
\vspace{1.2cm}
%%-----------------------step---------------------------
\westep{Finding the next two terms}
The next two terms in the sequence willl be:

% %..................USING EQUATIONS...................
% \begin{eqnarray*}
% 38 + 19 = 57\\
% 57 + 23 = 80\\
% \end{eqnarray*}

\rput(5.7,-0.5){
\psmatrix[colsep=0.3cm,rowsep=0.3cm]
     ...38 &     & 57 &	     & 80... \\
	   & +19 &    & +23  & \\
\endpsmatrix
\psset{nodesep=3pt,arrows=-}
\ncline{1,1}{2,2}
\ncline{2,2}{1,3}
\ncline{1,3}{2,4}
\ncline{2,4}{1,5}
}
\vspace{1.2cm}

So the sequence will be:
$5; 12; 23; 38; 57; 80$.\\

%\westep{We now need to find the formula for this sequence.} 
%We know that the second difference is $4$. The start of the formula will therefore be $2n^2$.

%%-----------------------step---------------------------
\westep{Determine values for $a, b$ and $c$}
\begin{eqnarray*}
 2a &=& 4 \\
\textrm{which gives} \quad a &=& 2\\
\textrm{And} \quad  3a + b &=& 7\\   
\therefore \quad 3(2) + b &=& 7\\
 b &=& 7-6\\
 b &=& 1\\
\textrm{And} \quad  a + b + c  &=& 5\\  
\therefore \quad (2) + (1) + c &=& 5\\
 c &=& 5-3\\
 c &=& 2
\end{eqnarray*}

% \westep{We now need to work out the next part of the sequence.}
% If $n=1$, you have to get the value of term one, which is $5$ in this particular sequence.  The difference between $2n^2 = 2$ and original number ($5$) is $3$, which leads to $n+2$.\\
% Check if it works for the second term, i.e. when $n=2$.\\
% Then $2n^2=8$.  The difference between term two ($12$) and $8$ is $4$, which is can be written as $n+2$.\\
% So for the sequence $5; 12; 23; 38;\ldots$ the formula for the $n^{\rm th}$ term is $2n^2 + n + 2$.

%%-----------------------step---------------------------
\westep{Find the rule by substitution}
\begin{eqnarray*}
 T_n &=& ax^2 + bx + c\\
\therefore \quad T_n &=& 2n^2 + n + 2
\end{eqnarray*}
}
\end{wex}
%%*******************************************************

%%**********************EXAMPLE**************************

\begin{wex}{Quadratic Sequence}
{The following sequence is quadratic: $8; 22; 42; 68; \ldots$
Find the rule.}{
%%-----------------------step---------------------------
\westep{Assume that the rule is $an^2 +bn + c$}
% \begin{center}
% \begin{tabular}{ccccccccc}
% TERMS & $8$ && $22$ && $42$ && $68$ & \\
% $1^{\textsf{st}}$ difference && $14$ && $20$ && $26$ \\ 
% $2^{\textsf{nd}}$ difference &&& $6$ && $6$ && $6$ & \\
% \end{tabular}
% \end{center}
\rput(5.7,-0.9){
$
\psmatrix[colsep=0.3cm,rowsep=0.3cm]
 	TERMS 				& 8 &     & 22 &     & 42 &     & 68\\
 	1^{\textsf{st}} \mbox{ difference}	&   & +14 &    & +20 &    & +26 &\\
	2^{\textsf{nd}} \mbox{ difference}	&   &     & +6 &     & +6
\endpsmatrix
\psset{nodesep=3pt,arrows=-}
\ncline{1,2}{2,3}
\ncline{1,4}{2,3}
\ncline{1,4}{2,5}
\ncline{1,6}{2,5}
\ncline{1,6}{2,7}
\ncline{1,8}{2,7}
\ncline{2,3}{3,4}
\ncline{2,5}{3,4}
\ncline{2,5}{3,6}
\ncline{2,7}{3,6}
$
}
\vspace{2.0cm}
%%-----------------------step---------------------------
\westep{Determine values for $a, b$ and $c$}
\begin{eqnarray*}
 2a &=& 6\\
\textrm{which gives} \quad a &=& 3\\
\textrm{And} \quad  3a + b &=& 14\\   
\therefore \quad 9 + b &=& 14\\
 b &=& 5\\
\textrm{And} \quad  a + b + c  &=& 8\\  
\therefore \quad 3 + 5 + c &=& 8\\
 c &=& 0
\end{eqnarray*}
%%-----------------------step---------------------------
\westep {Find the rule by substitution}
\begin{eqnarray*}
 T_n &=& ax^2 + bx + c\\
\therefore \quad T_n &=& 3n^2 + 5n
\end{eqnarray*}
%%-----------------------step---------------------------
\westep{Check answer}
For
\begin{eqnarray*}
n = 1, ~~T_1 &=& 3(1)^2 + 5(1) = 8\\
n = 2, ~~T_2 &=& 3(2)^2 + 5(2) = 22\\
n = 3, ~~T_3 &=& 3(3)^2 + 5(3) = 42
\end{eqnarray*}
}
\end{wex}

%%?????????????????????EXTENSION????????????????????????
\Extension{Derivation of the $n^{\rm th}$-term of a Quadratic Sequence}{Let the
$n^{th}$-term for a quadratic sequence be given by
\begin{eqnarray}
\label{eq:mp:s:extras:1}
T_n = an^2 + bn + c
\end{eqnarray}
where $a$, $b$ and $c$ are some constants to be determined.
\begin{eqnarray*}
T_n &=& an^2 + bn + c
\end{eqnarray*}
\begin{eqnarray*}
T_1 &=& a(1)^2 + b(1) + c
\end{eqnarray*}
\begin{eqnarray}
\label{eq:mp:s:extras:2}
&=& a + b + c
\end{eqnarray}
\begin{eqnarray*}
T_2 &=& a(2)^2 + b(2) + c
\end{eqnarray*}
\begin{eqnarray}
\label{eq:mp:s:extras:3}
&=& 4a + 2b + c
\end{eqnarray}
\begin{eqnarray*}
T_3 &=& a(3)^2 + b(3) + c
\end{eqnarray*}
\begin{eqnarray}
\label{eq:mp:s:extras:4}
&=& 9a + 3b + c
\end{eqnarray}

The first difference ($d$) is obtained from
\begin{eqnarray*}
\textrm {Let } d & \equiv & T_2 - T_1 \\
\therefore d &=& 3a + b
\end{eqnarray*}
\begin{equation}
\label{eq:mp:s:extras:5}
\Rightarrow b = d - 3a
\end{equation}

The common second difference ($D$) is obtained from
\begin{eqnarray*}
D &=& (T_3 - T_2) - (T_2 - T_1) \\
&=& (5a + b) - (3a + b) \\
&=& 2a
\end{eqnarray*}
\begin{equation}
\label{eq:mp:s:extras:6}
\Rightarrow a = \frac{D}{2}
\end{equation}

Therefore, from (\ref{eq:mp:s:extras:5}),
\begin{equation}
\label{eq:mp:s:extras:7}
b = d - \frac{3}{2} \cdot D
\end{equation}

From (\ref{eq:mp:s:extras:2}),
\begin{equation*}
c = T_1 - (a + b) = T_1 - \frac{D}{2} - d + \frac{3}{2} \cdot D
\end{equation*}

\begin{equation}
\label{eq:mp:s:extras:8}
\therefore c = T_1 + D - d
\end{equation}

Finally, the general equation for the $n^{th}$-term of a quadratic sequence is
given by
\begin{equation}
\label{eq:mp:s:extras:9}
T_n = \frac{D}{2} \cdot {n^2} + (d - \frac {3} {2} \: D) \cdot n + (T_1 - d + D)
\end{equation}}
%%??????????????????????????????????????????????????????

%%**********************EXAMPLE**************************

\begin{wex}{Using a set of equations}
{Study the following pattern: $1; 7; 19; 37; 61; \ldots$
\begin{enumerate}
\item{What is the next number in the sequence?}
\item{Use variables to write an algebraic statement to generalise the pattern.}
\item{What will the $100^{th}$ term of the sequence be?}
\end{enumerate}
}{
%%-----------------------step---------------------------
\westep{The next number in the sequence}
The numbers go up in multiples of $6$\\
$1 + 6(1) = 7$,  then $7 + 6(2) = 19$\\
$19+ 6(3)=37$, then $37+6(4)=61$\\
Therefore $61 + 6(5) = 91$\\
The next number in the sequence is $91$.\\
%%-----------------------step---------------------------
\westep{Generalising the pattern}
% \begin{center}
% \begin{tabular}{cccccccccccc}
% TERMS & $1$ && $7$ && $19$ && $37$ && $61$ & \\
% $1^{\textsf{st}}$ difference && $6$ && $12$ && $18$ && $24$ \\ 
% $2^{\textsf{nd}}$ difference &&& $6$ && $6$ && $6$ && $6$& \\
% \end{tabular}
% \end{center}

\rput(5.7,-0.9){
$
\psmatrix[colsep=0.3cm,rowsep=0.3cm]
 	TERMS 				& 1 &    &  7 &     & 19 &     & 37 &     & 61 &\\
 	1^{\textsf{st}} \mbox{ difference}	&   & +6 &    & +12 &    & +18 &    & +24\\
	2^{\textsf{nd}} \mbox{ difference}	&   &    & +6 &     & +6 &     & +6
\endpsmatrix
\psset{nodesep=3pt,arrows=-}
\ncline{1,2}{2,3}
\ncline{1,4}{2,3}
\ncline{1,4}{2,5}
\ncline{1,6}{2,5}
\ncline{1,6}{2,7}
\ncline{1,8}{2,7}
\ncline{1,8}{2,9}
\ncline{1,10}{2,9}
\ncline{2,3}{3,4}
\ncline{2,5}{3,4}
\ncline{2,5}{3,6}
\ncline{2,7}{3,6}
\ncline{2,7}{3,8}
\ncline{2,9}{3,8}
$
}
\vspace{2.0cm}

The pattern will yield a quadratic equation since the second difference is
constant\\
Therefore $T_n = an^2 + bn + c$\\
For the first term: $n = 1$,  then $T_1 = 1$ \\
For the second term: $n = 2$, then $T_2 = 7$ \\  
For the third term:  $n = 3$,  then $T_3 = 19$ \\
etc....\\
%%-----------------------step---------------------------
\westep{Setting up sets of equations}
\begin{eqnarray*}
a+b+c &=& 1 \hspace{1cm} \rm {...eqn} (1)\\
4a + 2b + c &=& 7 \hspace{1cm} \rm {...eqn} (2)\\
9a + 3b + c &=& 19 \hspace{0.8cm} \rm {...eqn} (3)
\end{eqnarray*}
%%-----------------------step---------------------------
\westep{Solve the sets of equations}
\begin{eqnarray*}
&\rm{eqn} (2) - \rm{eqn} (1):& 3a + b = 6 \hspace{1cm} \rm {...eqn} (4)\\
&\rm{eqn} (3) - \rm{eqn} (2):& 5a + b = 12 \hspace{0.8cm} \rm {...eqn} (5)\\
&\rm{eqn} (5) - \rm{eqn} (4):& 2a = 6\\
&\therefore & a = 3, b = -3 ~and~ c = 1
\end{eqnarray*}
%%-----------------------step---------------------------
\westep{Final answer}
The general formula for the pattern is $T_n = 3n^2 - 3n + 1$\\
%%-----------------------step---------------------------
\westep{Term 100}
Substitute n with $100$:\\
$3(100)^2 - 3(100) + 1 = 29~701$\\
The value for term $100$ is $29~701$.
}
\end{wex}
%%*******************************************************

%%?????????????????????EXTENSION????????????????????????

\Extension{Plotting a graph of terms of a quadratic sequence}{Plotting $T_n$ vs.
$n$ for a quadratic sequence yields a parabolic graph. 

Given the quadratic sequence,
\begin{eqnarray*}
3; \: 6; \: 10; \: 15; \: 21; \: \ldots
\end{eqnarray*}

If we plot each of the terms vs. the corresponding index, we obtain a graph of a parabola.

%\begin{figure}[!htbp]
\begin{center}
\begin{pspicture}(-1,-1)(10,8)
%\psset{yunit=0.7,xunit=0.7}
%\psgrid[gridcolor=lightgray]
\rput(-1,0){
\psaxes[arrows=<->,dx=10,Dx=1,dy=10,Dy=0.5](1,0)(11,8)
\psplot[plotstyle=curve,arrows=->]{1}{10.5}{x 2 exp 0.05 mul x 0.25 mul add}
\psplot[plotstyle=dots,arrows=->,plotpoints=10]{1}{10}{x 2 exp 0.05 mul x 0.25 mul add}
\uput[l](0.4,5){\rotateleft{Term: $T_n$}}
\uput[r](1.4,0.35){$y$-intercept: $T_1$}
\multips(1,0)(1,0){10}{\psline(0,-.1)(0,.1)}
\multido{\n=1+1}{10}{\rput(\n,-0.35){\n}}
%\multido{\n=1+1}{9}{\rput(0.65,\n){$a_{\n}$}}
\rput(5,-.9){Index: $n$}

\psline(0.9,0.3)(1.1,0.3)
\psline(0.9,0.7)(1.1,0.7)
\psline(0.9,1.2)(1.1,1.2)
\psline(0.9,1.8)(1.1,1.8)
\psline(0.9,2.5)(1.1,2.5)
\psline(0.9,3.3)(1.1,3.3)
\psline(0.9,4.2)(1.1,4.2)
\psline(0.9,5.2)(1.1,5.2)
\psline(0.9,6.3)(1.1,6.3)
\psline(0.9,7.5)(1.1,7.5)
\rput(0.65,0.3){$a_1$}
\rput(0.65,0.7){$a_2$}
\rput(0.65,1.2){$a_3$}
\rput(0.65,1.8){$a_4$}
\rput(0.65,2.5){$a_5$}
\rput(0.65,3.3){$a_6$}
\rput(0.65,4.2){$a_7$}
\rput(0.65,5.2){$a_8$}
\rput(0.65,6.3){$a_9$}
\rput(0.65,7.5){$a_{10}$}
}
\end{pspicture}
%\caption{A plot of $a_n$ vs. $n$ for quadratic sequence \{3; 6; 10; 15; 21; \ldots\}.}
%\label{fig:mp:extra:quadratic}
\end{center}
%\end{figure}
}
%%??????????????????????????????????????????????????????

\begin{eocexercises}{}
\begin{enumerate}
\item{Find the first five terms of the quadratic sequence defined by:
\nequ{a_n=n^2+2n+1}}
\item{Determine which of the following sequences is a quadratic sequence by calculating the common second difference:
\begin{enumerate}
\item $6; 9; 14; 21; 30;\ldots$
\item $1; 7; 17; 31; 49;\ldots$
\item $8; 17; 32; 53; 80;\ldots$
\item $9; 26; 51; 84; 125;\ldots$
\item $2; 20; 50; 92; 146;\ldots$
\item $5; 19; 41; 71; 109;\ldots$
\item $2; 6; 10; 14; 18;\ldots$
\item $3; 9; 15; 21; 27;\ldots$
\item $10; 24; 44; 70; 102;\ldots$
\item $1; 2,5; 5; 8,5; 13;\ldots$
\item $2,5; 6; 10,5; 16; 22,5;\ldots$
\item $0,5; 9; 20,5; 35; 52,5;\ldots$
\end{enumerate}}
\item{Given $T_n= 2n^2$, find for which value of $n$, $T_n=242$}
\item{Given $T_n= (n - 4)^2$, find for which value of $n$, $T_n=36$}
\item{Given $T_n= n^2+4$, find for which value of $n$, $T_n=85$}
\item{Given $T_n= 3n^2$, find $T_{11}$}
\item{Given $T_n= 7n^2+4n$, find $T_{9}$}
\item{Given $T_n= 4n^2+3n-1$, find $T_{5}$}
\item{Given $T_n= 1,5n^2$, find $T_{10}$}
\item{For each of the quadratic sequences, find the common second difference, the formula for the general term and then use the formula to find $a_{100}$.
\begin{enumerate}
\item $4;7;12;19;28;\ldots$
\item $2;8;18;32;50;\ldots$
\item $7;13;23;37;55;\ldots$
\item $5;14;29;50;77;\ldots$
\item $7;22;47;82;127;\ldots$
\item $3;10;21;36;55;\ldots$
\item $3;7;13;21;31;\ldots$
\item $3;9;17;27;39;\ldots$
\end{enumerate}}

\end{enumerate}



% Automatically inserted shortcodes - number to insert 10
\par \practiceinfo
\par \begin{tabular}[h]{cccccc}
% Question 1
(1.)	0177	&
% Question 2
(2.)	0178	&
% Question 3
(3.)	0179	&
% Question 4
(4.)	017a	&
% Question 5
(5.)	017b	&
% Question 6
(6.)	017c	\\ % End row of shortcodes
% Question 7
(7.)	017d	&
% Question 8
(8.)	017e	&
% Question 9
(9.)	017f	&
% Question 10
(10.)	017g	&
\end{tabular}
% Automatically inserted shortcodes - number inserted 10
\end{eocexercises} 



% CHILD SECTION START 

