\chapter{Statistics}
\label{m:s11}

\section{Introduction}
This chapter gives you an opportunity to build on what you have learned in previous grades about data handling and probability. The work done will be mostly  of a practical nature. Through problem solving and activities, you will end up mastering further methods of collecting, organising, displaying and analysing data. You will also learn how to interpret data, and not always to accept the data at face value, because data is sometimes misused and abused in order to try to falsely prove or support a viewpoint. Measures of central tendency (mean, median and mode) and dispersion (range, percentiles, quartiles, inter-quartile, semi-inter-quartile range, variance and standard deviation) will be investigated. Of course, the activities involving probability will be familiar to most of you - for example, you may have played dice or card games even before you came to school. Your basic understanding of probability and chance gained so far will deepen to enable you to come to a better understanding of how chance and uncertainty can be measured and understood.  

\chapterstartvideo{VMfvd}

\section{Standard Deviation and Variance}
%\begin{syllabus}
%\item Calculating the variance and standard deviation of sets of data manually (for small sets of data) and using available technology (for larger sets of data), and representing results graphically using histograms and frequency polygons.
%\end{syllabus}

The measures of central tendency (mean, median and mode) and measures of dispersion (quartiles, percentiles, ranges) provide information on the data values at the centre of the data set and provide information on the spread of the data. The information on the spread of the data is however based on data values at specific points in the data set, e.g. the end points for range and data points that divide the data set into four equal groups for the quartiles. The behaviour of the entire data set is therefore not examined.

A method of determining the spread of data is by calculating a measure of the possible distances between the data and the mean. The two important measures that are used are called the \textit{variance} and the \textit{standard deviation} of the data set.

\subsection{Variance}
The variance of a data set is the average squared distance between the mean of the data set and each data value. An example of what this means is shown in Figure~\ref{fig:mdat:variance}. The graph represents the results of 100 tosses of a fair coin, which resulted in $45$ heads and $55$ tails. The mean of the results is $50$. The squared distance between the heads value and the mean is $(45-50)^2=25$ and the squared distance between the tails value and the mean is $(55-50)^2=25$. The average of these two squared distances gives the variance, which is $\frac{1}{2}(25+25)=25$.

\begin{figure}[H]
\begin{center}
\begin{pspicture}(-1,-1)(4,6)
\psset{unit=1.0cm}
\SpecialCoor
\psline(0,0)(3,0)
\psaxes[dy=0.5,Dy=5,dx=10](3,6)
\rput(0.75,-.25){Heads}
\rput(2.25,-.25){Tails}
\rput(-1.0,2.5){\rotateleft{Frequency (\%)}}
\rput(1.5,-.75){Face of Coin}
\psline[linewidth=3pt](0.75,0)(0.75,4.5)
\psline[linewidth=3pt](2.25,0)(2.25,5.5)
\psline[linewidth=1pt,linestyle=dotted](1.5,0)(1.5,5)
\psline[linestyle=dashed](0,4.5)(2.5,4.5)
\psline[linestyle=dashed](0,5.0)(2.5,5.0)
\psline[linestyle=dashed](0,5.5)(2.5,5.5)
\psline(0,0)(0,6)
\psline{|-|}(2.75,4.5)(2.75,5)
\psline{|-|}(2.75,5.5)(2.75,5)
\rput[l](2.95,5.25){Tails-Mean}
\rput[l](2.95,4.75){Heads-Mean}
\end{pspicture}
\end{center}
\caption{The graph shows the results of $100$ tosses of a fair coin, with $45$ heads and $55$ tails. The mean value of the tosses is shown as a vertical dotted line. The difference between the mean value and each data value is shown.}
\label{fig:mdat:variance}
\end{figure}

\subsubsection{Population Variance}
Let the population consist of $n$ elements $\{x_1;x_2;\ldots ;x_n\}$, with mean $\bar{x}$ (read as "x bar"). The variance of the population, denoted by $\sigma^2$, is the average of the square of the distance of each data value from the mean value.

\begin{equation}
\sigma^2 = \frac{(\sum(x - \bar{x}))^2}{n}.
\label{eq:popvariance}
\end{equation}

Since the population variance is squared, it is not directly comparable with the mean and the data themselves.

\subsubsection{Sample Variance}
Let the sample consist of the $n$ elements $\{x_1,x_2,\ldots,x_n\}$, taken from the population, with mean $\bar{x}$. The variance of the sample, denoted by $s^2$, is the average of the squared deviations from the sample mean:

\begin{equation}
s^2 = \frac{\sum(x - \bar{x})^2}{n-1}.
\end{equation}

Since the sample variance is squared, it is also not directly comparable with the mean and the data themselves.

A common question at this point is "Why is the numerator squared?" One answer is: to get rid of the negative signs. Numbers are going to fall above and below the mean and, since the variance is looking for distance, it would be counterproductive if those distances factored each other out.

\subsubsection{Difference between Population Variance and Sample Variance}
As seen a distinction is made between the variance, $\sigma^2$, of a whole population and the variance, $s^2$ of a sample extracted from the population.

When dealing with the complete population the (population) variance is a constant, a parameter which helps to describe the population. When dealing with a sample from the population the (sample) variance varies from sample to sample. Its value is only of interest as an estimate for the population variance.

\subsubsection{Properties of Variance}
The variance is never negative because the squares are always positive or zero. The unit of variance is the square of the unit of observation. For example, the variance of a set of heights measured in centimetres will be given in square centimeters. This fact is inconvenient and has motivated many statisticians to instead use the square root of the variance, known as the standard deviation, as a summary of dispersion.

\subsection{Standard Deviation}
Since the variance is a squared quantity, it cannot be directly compared to the data values or the mean value of a data set. It is therefore more useful to have a quantity which is the square root of the variance. This quantity is known as the standard deviation.

In statistics, the standard deviation is the most common measure of statistical dispersion. Standard deviation measures how spread out the values in a data set are. More precisely, it is a measure of the average distance between the values of the data in the set and the mean. If the data values are all similar, then the standard deviation will be low (closer to zero). If the data values are highly variable, then the standard variation is high (further from zero).

The standard deviation is always a positive number and is always measured in the same units as the original data. For example, if the data are distance measurements in metres, the standard deviation will also be measured in metres.

\subsubsection{Population Standard Deviation}
Let the population consist of $n$ elements $\{x_1;x_2;\ldots ;x_n\}$, with mean $\bar{x}$. The standard deviation of the population, denoted by $\sigma$, is the square root of the average of the square of the distance of each data value from the mean value.

\begin{equation}
\sigma = \sqrt{\frac{\sum(x - \bar{x})^2}{n}}
\label{eq:popstddev}
\end{equation}

\subsubsection{Sample Standard Deviation}
Let the sample consist of $n$ elements $\{x_1;x_2;\ldots,x_n\}$, taken from the population, with mean $\bar{x}$. The standard deviation of the sample, denoted by $s$, is the square root of the average of the squared deviations from the sample mean:

\begin{equation}
s = \sqrt{\frac{\sum(x - \bar{x})^2}{n-1}}
\end{equation}

It is often useful to set your data out in a table so that you can apply the formulae easily. For example to calculate the standard deviation of $\{57; 53; 58; 65; 48; 50; 66; 51\}$, you could set it out in the following way: \newline
\begin{eqnarray*}
\bar{x} & = & \frac{\textsf{sum of items}}{\textsf{number of items}} \\
&=& \frac{\sum {x}}{n} \\
&=& \frac{448}{8} \\
&=& 56
\end{eqnarray*}
\textbf{Note:} To get the deviations, subtract each number from the mean.
\begin{center}
\begin{tabular}{|l|l|l|}
\hline
\textbf{$X$} & \textbf{Deviation $(X - \bar{X})$} & \textbf{Deviation squared $(X - \bar{X})^2$} \bigstrut \\
\hline
$57$ & $1$ & $1$ \\
$53$ & $-3$ & $9$ \\
$58$ & $2$ & $4$ \\
$65$ & $9$ & $81$ \\
$48$ & $-8$ & $64$ \\
$50$ & $-6$ & $36$ \\
$66$ & $10$ & $100$ \\
$51$ & $-5$ & $25$ \\
\hline
$\sum{X} = 448$ & $\sum{x} = 0$ & $\sum{(X  - \bar{X})^2} = 320$ \bigstrut \\
\hline
\end{tabular}
\end{center}
\textbf{Note:} The sum of the deviations of scores about their mean is zero. This always happens; that is $(X - \bar{X}) = 0$, for any set of data. Why is this? Find out. \newline

Calculate the variance (add the squared results together and divide this total by the number of items).
\begin{eqnarray*}
\textbf{Variance} & = & \frac{\sum{(X - \bar{X})^2}}{n} \\
&=& \frac{320}{8} \\
&=& 40
\end{eqnarray*}
\begin{eqnarray*}
\textbf{Standard deviation} & = & \sqrt{\mbox{variance}} \\
&=& \sqrt{\frac{\sum{(X - \bar{X})^2}}{n}} \\
&=& \sqrt{\frac{320}{8}} \\
&=& \sqrt{40} \\
&=& 6.32 \\
\end{eqnarray*}

\subsubsection{Difference between Population Variance and Sample Variance}
As with variance, there is a distinction between the standard deviation, $\sigma$, of a whole population and the standard deviation, $s$, of sample extracted from the population.

When dealing with the complete population the (population) standard deviation is a constant, a parameter which helps to describe the population. When dealing with a sample from the population the (sample) standard deviation varies from sample to sample. \newline

\textbf{In other words, the standard deviation can be calculated as follows:}

\begin{enumerate}
\item Calculate the mean value $\bar{x}$.
\item For each data value $x_i$ calculate the difference $x_i - \bar{x}$ between $x_i$ and the mean value $\bar{x}$.
\item Calculate the squares of these differences.
\item Find the average of the squared differences. This quantity is the variance, $\sigma^2$.
\item Take the square root of the variance to obtain the standard deviation, $\sigma$.
\end{enumerate}

%Khan Academy video on standard deviation: SIYAVULA-VIDEO:http://cnx.org/content/m38858/latest/#statistics-2
\mindsetvid{Khan on std dev}{VMfvk}

\begin{wex}{Variance and Standard Deviation}{What is the variance and standard deviation of the population of possibilities associated with rolling a fair die?}
{\westep{Determine how many outcomes make up the population}
When rolling a fair die, the population consists of $6$ possible outcomes. The data set is therefore $x=\{1;2;3;4;5;6\}$. and $n=6$.\\

\westep{Calculate the population mean}
The population mean is calculated by:
\begin{eqnarray*}
\bar{x} &=& \frac{1}{6}(1+2+3+4+5+6)\\
&=& 3,5
\end{eqnarray*}

\westep{Calculate the population variance}
The population variance is calculated by:
\begin{eqnarray*}
\sigma^2&=& \frac{\sum(x-\bar{x})^2}{n}\\
&=&\frac{1}{6} (6,25+2,25+0,25+0,25+2,25+6,25)\\
&=& 2,917
\end{eqnarray*}

\westep{Alternately the population variance is calculated by:\\}
\begin{center}
\begin{tabular}{|l|l|l|}
\hline
$X$ & $(X - \bar{X})$ & $(X - \bar{X})^2$ \bigstrut \\
\hline
$1$ & $-2.5$ & $6.25$ \\
$2 $& $-1.5 $& $2.25 $\\
$3 $& $-0.5$ & $0.25 $\\
$4 $& $0.5$  & $0.25 $\\
$5$ & $1.5$  & $2.25 $\\
$6$ & $2.5$  & $6.25$ \\
\hline
$\sum{X} = 21$ & $\sum{x} = 0$ & $\sum{(X  - \bar{X})^2} = 17.5$ \bigstrut \\
\hline
\end{tabular}
\end{center}

\westep{Calculate the standard deviation}
The (population) standard deviation is calculated by:

\begin{eqnarray*}
\sigma &=& \sqrt{2,917} \\
&=& 1,708.
\end{eqnarray*}
Notice how this standard deviation is somewhere in between the possible deviations.}
\end{wex}

\subsection{Interpretation and Application}

A large standard deviation indicates that the data values are far from the mean and a small standard deviation indicates that they are clustered closely around the mean.

For example, each of the three samples ($0; 0; 14; 14$), ($0; 6; 8; 14$), and ($6; 6; 8; 8$) has a mean of $7$. Their standard deviations are $8,08$; $5,77$ and $1,15$ respectively. The third set has a much smaller standard deviation than the other two because its values are all close to $7$. The value of the standard deviation can be considered `large' or `small' only in relation to the sample that is being measured. In this case, a standard deviation of $7$ may be considered large. Given a different sample, a standard deviation of $7$ might be considered small.

Standard deviation may be thought of as a measure of uncertainty. In physical science for example, the reported standard deviation of a group of repeated measurements should give the precision of those measurements. When deciding whether measurements agree with a theoretical prediction, the standard deviation of those measurements is of crucial importance: if the mean of the measurements is too far away from the prediction (with the distance measured in standard deviations), then we consider the measurements as contradicting the prediction. This makes sense since they fall outside the range of values that could reasonably be expected to occur if the prediction were correct and the standard deviation appropriately quantified. (See prediction interval.)

\subsection{Relationship Between Standard Deviation and the Mean}

The mean and the standard deviation of a set of data are usually reported together. In a certain sense, the standard deviation is a ``natural'' measure of statistical dispersion if the centre of the data is measured about the mean. 

\Exercise{Means and standard deviations}
{
\begin{enumerate}
\item Bridget surveyed the price of petrol at petrol stations in Cape Town and Durban. The raw data, in rands per litre, are given below:
\begin{center}
\begin{tabular}{lllllll}
Cape Town & $3,96$ & $3,76$ & $4,00$ & $3,91$ & $3,69$ & $3,72$ \\
Durban    & $3,97$ & $3,81$ & $3,52$ & $4,08$ & $3,88$ & $3,68$ \\ 
\end{tabular}
\end{center}
	\begin{enumerate}
	\item Find the mean price in each city and then state which city has the lowest mean.
	\item Assuming that the data is a population find the standard deviation of each city's prices.
	\item Assuming the data is a sample find the standard deviation of each city's prices.
	\item Giving reasons which city has the more consistently priced petrol?
	\end{enumerate}
\item The following data represents the pocket money of a sample of teenagers. \newline
$150; ~300;~ 250;~ 270;~ 130;~ 80;~ 700;~ 500;~ 200;~ 220;~ 110;~ 320;~ 420;~ 140$. \newline
What is the standard deviation?
\item Consider a set of data that gives the weights of $50$ cats at a cat show.
	\begin{enumerate}
	\item When is the data seen as a population?
	\item When is the data seen as a sample?
	\end{enumerate}
\item Consider a set of data that gives the results of $20$ pupils in a class.
	\begin{enumerate}
	\item When is the data seen as a population?
	\item When is the data seen as a sample?
	\end{enumerate}  
\end{enumerate}



% Automatically inserted shortcodes - number to insert 4
\par \practiceinfo
\par \begin{tabular}[h]{cccccc}
% Question 1
(1.)	015e	&
% Question 2
(2.)	015f	&
% Question 3
(3.)	015g	&
% Question 4
(4.)	015h	&
\end{tabular}}
% Automatically inserted shortcodes - number inserted 4

\section[Graphical Representation of Measures of Central Tendency and Dispersion]{\Huge Graphical Representation of Measures of Central Tendency and Dispersion}
%\begin{syllabus}
%\item{Calculate and represent measures of central tendency and dispersion in univariate numerical data by:
%\begin{itemize}
%\item five number summary (maximum, minimum and quartiles);
%\item box and whisker diagrams;
%\item ogives;
%\end{itemize}}
%\end{syllabus}

The measures of central tendency (mean, median, mode) and the measures of dispersion (range, semi-inter-quartile range, quartiles, percentiles, inter-quartile range) are numerical methods of summarising data. This section presents methods of representing the summarised data using graphs.

\subsection{Five Number Summary}
One method of summarising a data set is to present a \textit{five number summary}. The five numbers are: minimum, first quartile, median, third quartile and maximum.

\subsection{Box and Whisker Diagrams}
A \textit{box and whisker} diagram is a method of depicting the five number summary, graphically.

The main features of the box and whisker diagram are shown in Figure~\ref{fig:mdat:s:boxwhiskerfeatures}. The box can lie horizontally (as shown) or vertically. For a horizontal diagram, the left edge of the box is placed at the first quartile and the right edge of the box is placed at the third quartile. The height of the box is arbitrary, as there is no $y$-axis. Inside the box there is some representation of central tendency, with the median shown with a vertical line dividing the box into two. Additionally, a star or asterix is placed at the mean value, centred in the box in the vertical direction. The whiskers which extend to the sides reach the minimum and maximum values.
\begin{figure}[H]
\begin{center}
\begin{pspicture}(-5,-2)(5,1)
% \psset{units=0.6}
%\psgrid
% \usefont{T1}{ptm}{m}{n}
\psframe(-2,-0.5)(2,0.5)
\psline[arrows=*-*](-4,0)(-2,0)
\psline[arrows=*-*](2,0)(4,0)
\rput(-4,-0.5){\begin{tabular}{c}minimum\\data value\end{tabular}}
\rput(4,-0.5){\begin{tabular}{c}maximum\\data value\end{tabular}}
\rput(-3,0.5){\begin{tabular}{c}first\\quartile\end{tabular}}
\psline[arrows=->](-2.4,0.6)(-2,0)
\rput(3,0.5){\begin{tabular}{c}third\\quartile\end{tabular}}
\psline[arrows=->](2.4,0.6)(2,0)
\psline(0,0.5)(0,-0.5)
\rput(0,0.8){median}

\psline[arrows=<->](-5,-1)(5,-1)
\rput(0,-1.8){Data Values}
\rput(0,-1.3){0}
\rput(2,-1.3){2}
\rput(-2,-1.3){-2}
\rput(4,-1.3){4}
\rput(-4,-1.3){-4}
\end{pspicture}
\end{center}
\caption{Main features of a box and whisker diagram}
\label{fig:mdat:s:boxwhiskerfeatures}
\end{figure}

\begin{wex}{Box and Whisker Diagram}{Draw a box and whisker diagram for the data set \\$x=\{1,25;~ 1,5;~ 2,5;~ 2,5;~ 3,1;~ 3,2;~ 4,1;~ 4,25;~ 4,75;~ 4,8; ~4,95;~ 5,1\}$.\\}{

\westep{Determine the five number summary}
Minimum = $1,25$\\
Maximum = $5,10$\\
Position of first quartile = between $3$ and $4$\\
Position of second quartile = between $6$ and $7$\\
Position of third quartile = between $9$ and $10$\\

Data value between $3$ and $4$ = $\frac{1}{2}(2,5+2,5)=2,5$\\
Data value between $6$ and $7$ = $\frac{1}{2}(3,2+4,1)=3,65$\\
Data value between $9$ and $10$ = $\frac{1}{2}(4,75+4,8)=4,775$\\

The five number summary is therefore: $1,25;~ 2,5;~ 3,65;~ 4,775;~ 5,10$.

\westep{Draw a box and whisker diagram and mark the positions of the minimum, maximum and quartiles.}

\begin{center}
\begin{pspicture}(0,-2)(6,2)
%\psgrid
\psline[arrows=*-*](1.25,0)(2.5,0) %left whisker
\psframe(2.5,-0.5)(4.775,0.5) %box
\psline[arrows=*-*](4.775,0)(4.95,0)%right whisker
\psline(3.65,0.5)(3.65,-0.5) %median

\rput(1.25,-0.75){\begin{tabular}{c}minimum\end{tabular}}
\rput(4.95,-0.75){\begin{tabular}{c}maximum\end{tabular}}
\psline[arrows=->](1.25,-0.6)(1.25,-0.1)
\psline[arrows=->](4.95,-0.6)(4.95,-0.1)

\rput(2.5,1.5){\begin{tabular}{c}first\\quartile\end{tabular}}
\psline[arrows=->](2.5,1.1)(2.5,0.6)
\rput(4.775,1.5){\begin{tabular}{c}third\\quartile\end{tabular}}
\psline[arrows=->](4.775,1.1)(4.775,0.6)
\rput(3.65,0.8){median}

\psline[arrows=<->](0,-1)(6,-1)
\rput(3.,-1.8){Data Values}
\rput(1,-1.3){1}
\rput(2,-1.3){2}
\rput(3,-1.3){3}
\rput(4,-1.3){4}
\rput(5,-1.3){5}

\end{pspicture}
\end{center}}
\end{wex}
%Khan Academy video on box and whisker plots: SIYAVULA-VIDEO:http://cnx.org/content/m38860/latest/#statistics-1
\mindsetvid{Khan on box and whisker}{VMfzi}
\Exercise{Box and whisker plots}
{
\begin{enumerate}
\item Lisa works as a telesales person. She keeps a record of the number of sales she makes each month. The data below show how much she sells each month. \newline
\textbf{$49; ~12; ~22; ~35; ~2; ~45;~ 60; ~48; ~19; ~1;~ 43; ~12$} \newline
Give a five number summary and a box and whisker plot of her sales. 
\item Jason is working in a computer store. He sells the following number of computers each month: \newline
\textbf{$27; ~39; ~3; ~15; ~43; ~27; ~19; ~54; ~65; ~23; ~45; ~16$} \newline
Give a five number summary and a box and whisker plot of his sales,
\item The number of rugby matches attended by 36 season ticket holders is as follows: \newline
\textbf{$15; ~11;~ 7;~ 34;~ 24;~ 22;~ 31;~ 12;~ 9$} \newline
\textbf{$12;~ 9;~ 1;~ 3;~ 15;~ 5;~ 8;~ 11;~ 2$} \newline
\textbf{$25;~ 2;~ 6; ~18; ~16; ~17; ~20; ~13; ~17$} \newline
\textbf{$14;~ 13;~ 11; ~5; ~3; ~2; ~23; ~26; ~40$} \newline
	\begin{enumerate}
	\item Sum the data.
	\item Using an appropriate graphical method (give reasons) represent the data.
	\item Find the median, mode and mean.
	\item Calculate the five number summary and make a box and whisker plot.
	\item What is the variance and standard deviation?
	\item Comment on the data's spread.
	\item Where are $95\%$ of the results expected to lie?
	\end{enumerate}
\item Rose has worked in a florists shop for nine months. She sold the following number of wedding bouquets: \newline
\textbf{$16;~ 14;~ 8;~ 12;~ 6; ~5; ~3; ~5; ~7$} \newline
	\begin{enumerate}
	\item What is the five-number summary of the data?
	\item Since there is an odd number of data points what do you observe when calculating the five-numbers?
	\end{enumerate}
\end{enumerate}


% Automatically inserted shortcodes - number to insert 4
\par \practiceinfo
\par \begin{tabular}[h]{cccccc}
% Question 1
(1.)	015i	&
% Question 2
(2.)	015j	&
% Question 3
(3.)	015k	&
% Question 4
(4.)	015m	&
\end{tabular}}
% Automatically inserted shortcodes - number inserted 4

\subsection{Cumulative Histograms}
Cumulative histograms, also known as ogives, are a plot of cumulative frequency and are used to determine how many data values lie above or below a particular value in a data set. The cumulative frequency is calculated from a frequency table, by adding each frequency to the total of the frequencies of all data values before it in the data set. The last value for the cumulative frequency will always be equal to the total number of data values, since all frequencies will already have been added to the previous total. The cumulative frequency is plotted at the upper limit of the interval.

For example, the cumulative frequencies for Data Set 2 are shown in Table~\ref{tab:mdat:s:cumulativeds2} and is drawn in Figure~\ref{fig:mdat:s:cumulativegraph}.

\begin{table}[htb]
\begin{center}
\begin{tabular}{|p{1.5cm}||p{1.5cm}|p{1.5cm}|p{1.5cm}|p{1.5cm}|p{1.5cm}|p{1.5cm}|}\hline
Intervals & $0<n\leq 1 $ & $1<n\leq 2 $ & $2<n\leq 3 $ & $3<n\leq 4 $ & $4<n\leq 5$ & $5<n\leq 6 $ \\ 
\hline
Frequency & $30$ &$32$ &$35$ &$34$ &$37$ &$32$\\
\hline
Cumulative Frequency & $30$ &$30 + 32$ &$30 + 32 + 35$ &$30 + 32 + 35 + 34$ &$30 + 32 + 35 + 34 + 37 $&$30 + 32 + 35 + 34 + 37 + 32$\\
\hline
& $30$ &$ 62$ & $97$ & $131$ & $168$ & $200$\\
\hline
\end{tabular}
\caption{Cumulative Frequencies for Data Set 2. \label{tab:mdat:s:cumulativeds2}}
\end{center}
\end{table}

\begin{figure}[htb]
\begin{center}
%\scalebox{1} % Change this value to rescale the drawing.
{
\begin{pspicture}(0,-3.1454444)(7.4096947,3.1195555)
\rput(1.3320833,-1.9580556){\psaxes[linewidth=0.028222222,arrowsize=0.05291667cm 2.0,arrowlength=1.4,arrowinset=0.4,dx=1.0cm,dy=1.0cm,Dy=40]{->}(0,0)(0,0)(6,5)}
\psdots[dotsize=0.127](2.3320832,-1.2080556)
\psdots[dotsize=0.127](3.3320832,-0.40805557)
\psdots[dotsize=0.127](4.332083,0.46694443)
\psdots[dotsize=0.127](5.332083,1.3169445)
\psdots[dotsize=0.127](6.332083,2.2419446)
\psdots[dotsize=0.127](7.332083,3.0419445)
\psline[linewidth=0.028222222](1.3320833,-1.9580556)(2.3320832,-1.2080556)(3.3320832,-0.40805557)(4.332083,0.46694443)(5.332083,1.3169445)(6.332083,2.2419446)(7.332083,3.0419445)
%%%%%%%\usefont{T1}{ptm}{m}{n}
\rput(0.05765625,0.40955555){$f$}
%%%%%%%\usefont{T1}{ptm}{m}{n}
\rput(4.695469,-2.9904444){Intervals}
\end{pspicture} 
}
\end{center}
\caption{Example of a cumulative histogram for Data Set 2.\label{fig:mdat:s:cumulativegraph}}
\end{figure}
Notice the frequencies plotted at the upper limit of the intervals, so the points $(30;1)$ $(62;2)$ $(97;3)$, etc have been plotted. This is different from the frequency polygon where we plot frequencies at the midpoints of the intervals.

\Exercise{Intervals}
{
\begin{enumerate}
\item Use the following data of peoples ages to answer the questions. \newline
$2;~ 5;~1;~ 76;~ 34;~ 23; ~65; ~22;~ 63;~ 45;~ 53; ~38 $\newline
$4; ~28; ~5; ~73; ~80; ~17; ~15; ~5; ~34; ~37;~ 45; ~56 $
	\begin{enumerate}
	\item Using an interval width of $8$ construct a cumulative frequency distribution 
	\item How many are below $30$?
	\item How many are below $60$?
	\item Giving an explanation state below what value the bottom $50\%$ of the ages fall
	\item Below what value do the bottom $40\%$ fall?
	\item Construct a frequency polygon and an ogive. 
	\item Compare these two plots
	\end{enumerate}
\item The weights of bags of sand in grams is given below (rounded to the nearest tenth): \newline
$50.1;~ 40.4; ~48.5;~ 29.4; ~50.2; ~55.3;~ 58.1;~ 35.3;~ 54.2;~ 43.5$ \newline
$60.1; ~43.9;~ 45.3; ~49.2; ~36.6; ~31.5; ~63.1; ~49.3;~ 43.4; ~54.1$ 
	\begin{enumerate}
	\item Decide on an interval width and state what you observe about your choice.
	\item Give your lowest interval. 
	\item Give your highest interval.
	\item Construct a cumulative frequency graph and a frequency polygon.
	\item Compare the cumulative frequency graph and frequency polygon.
	\item Below what value do $53\%$ of the cases fall?
	\item Below what value of $60\%$ of the cases fall?
	\end{enumerate}
\end{enumerate}


% Automatically inserted shortcodes - number to insert 2
\par \practiceinfo
\par \begin{tabular}[h]{cccccc}
% Question 1
(1.)	015n	&
% Question 2
(2.)	015p	&
\end{tabular}}
% Automatically inserted shortcodes - number inserted 2

\section{Distribution of Data}
%\begin{syllabus}
%\item Differentiate between symmetric and skewed data and make relevant deductions.
%\end{syllabus}

\subsection{Symmetric and Skewed Data}
The shape of a data set is important to know.
\Definition{Shape of a data set}
{This describes how the data is distributed relative to the mean and median.}
\begin{itemize}
\item Symmetrical data sets are balanced on either side of the median. 
\begin{center}
%\scalebox{1} % Change this value to rescale the drawing.
{
\begin{pspicture}(0,-0.54)(8.1,0.56)
 \psline[linewidth=0.04cm,dotsize=0.07cm 2.0]{**-}(0.0,0)(2.05,0)
\psline[linewidth=0.04cm,dotsize=0.07cm 2.0]{-**}(6.0,0)(8.05,0)
\psframe[linewidth=0.04,dimen=outer](6.02,0.54)(2.02,-0.54)
\psline[linewidth=0.04cm](4.02,0.54)(4.02,-0.5)
\end{pspicture} 
}
\end{center}
\item Skewed data is spread out on one side more than on the other. It can be skewed right or skewed left.
\begin{center}
%\scalebox{1} % Change this value to rescale the drawing.
{
\begin{pspicture}(0,-2.32)(8.28,2.28)
\psline[linewidth=0.04cm,dotsize=0.07cm 2.0]{**-}(0.0,-1.05)(3.04,-1.05)
\psline[linewidth=0.04cm,dotsize=0.07cm 2.0]{-**}(7.0,-1.05)(8.05,-1.05)
\psframe[linewidth=0.04,dimen=outer](7.0,-0.51)(3.0,-1.59)
\psline[linewidth=0.04cm](6.12,-0.53)(6.12,-1.59)
\psline[linewidth=0.04cm,dotsize=0.07cm 2.0]{**-}(0.18,1.76)(1.22,1.76)
\psline[linewidth=0.04cm,dotsize=0.07cm 2.0]{-**}(5.18,1.76)(8.26,1.76)
\psframe[linewidth=0.04,dimen=outer](5.18,2.30)(1.18,1.22)
\psline[linewidth=0.04cm](2.08,2.30)(2.08,1.22)
%%%%%%%\usefont{T1}{ptm}{m}{n}
\rput(3.24,0.75){skewed right}
%%%%%%%\usefont{T1}{ptm}{m}{n}
\rput(3.45,-2.16){skewed left}
\end{pspicture} 
}
\end{center}
\end{itemize}

\subsection{Relationship of the Mean, Median, and Mode}

The relationship of the mean, median, and mode to each other can provide some information about the relative shape of the data distribution. If the mean, median, and mode are approximately equal to each other, the distribution can be assumed to be approximately symmetrical.
With both the mean and median known, the following can be concluded:
\begin{itemize}
\item (mean - median) $\approx 0$ then the data is symmetrical 
\item (mean - median) $>0$ then the data is positively skewed (skewed to the right). This means that the median is close to the start of the data set.
\item (mean - median) $<0$ then the data is negatively skewed (skewed to the left). This means that the median is close to the end of the data set.
\end{itemize}

\Exercise{Distribution of Data}
{
\begin{enumerate}
\item Three sets of $12$ pupils each had test score recorded. The test was out of $50$. Use the given data to answer the following questions.
\begin{table}[htb]
\begin{center}
\begin{tabular}{|l|l|l|}
\hline
Set A & Set B & Set C \bigstrut \\ 
\hline
$25$ & $32$ & $43$ \\
$47$ & $34$ & $47$ \\
$15$ & $35$ & $16$ \\
$17$ & $32$ & $43$ \\
$16$ & $25$ & $38$ \\
$26$ & $16$ & $44$ \\c
$24$ & $38$ & $42$ \\
$27$ & $47$ & $50$ \\
$22$ & $43$ & $50$ \\
$24$ & $29$ & $44$ \\
$12$ & $18$ & $43$ \\
$31$ & $25$ & $42 $\\
\hline
\end{tabular}
\caption{Cumulative Frequencies for Data Set 2. \label{tab:mdat:s:cumulativeds2}}
\end{center}
\end{table}
\begin{enumerate}
%\item Make a stem and leaf plot for each set.
\item For each of the sets calculate the mean and the five number summary.
\item For each of the classes find the difference between the mean and the median. Make box and whisker plots on the same set of axes.
\item State which of the three are skewed (either right or left).
\item Is set $A$ skewed or symmetrical? 
\item Is set $C$ symmetrical? Why or why not?
\end{enumerate}
\item Two data sets have the same range and interquartile range, but one is skewed right and the other is skewed left. Sketch the box and whisker plots and then invent data ($6$ points in each set) that meets the requirements.
\end{enumerate}


% Automatically inserted shortcodes - number to insert 2
\par \practiceinfo
\par \begin{tabular}[h]{cccccc}
% Question 1
(1.)	015q	&
% Question 2
(2.)	015r	&
\end{tabular}}
% Automatically inserted shortcodes - number inserted 2

\section{Scatter Plots}
\label{ms:sp}

%\begin{syllabus}
%\item Represent bivariate numerical data as a scatter plot and suggest intuitively whether a linear, quadratic or exponential function would best fit the data (problems should include issues related to health, social, economic, cultural, political and environmental issues).
%\end{syllabus}

A scatter-plot is a graph that shows the relationship between two variables. We say this is bivariate data and we plot the data from two different sets using ordered pairs. For example, we could have mass on the horizontal axis (first variable) and height on the second axis (second variable), or we could have current on the horizontal axis and voltage on the vertical axis.

Ohm's Law is an important relationship in physics. Ohm's law describes the relationship between current and voltage in a conductor, like a piece of wire. When we measure the voltage (dependent variable) that results from a certain current (independent variable) in a wire, we get the data points as shown in Table~\ref{tab:ms:sp:ohm}.

\begin{table}[!ht]
\begin{center}
\caption{Values of current and voltage measured in a wire.}
\begin{tabular}{|c|c||c|c|}\hline
\textbf{Current} & \textbf{Voltage} & \textbf{Current} & \textbf{Voltage} \\\hline\hline
$0$ & $0.4$ & $2.4$ & $1.4$ \\\hline
$0.2$ & $0.3$ & $2.6$ & $1.6$ \\\hline
$0.4$ & $0.6$ & $2.8$ & $1.9$ \\\hline
$0.6$ & $0.6$ & $3$ & $1.9$ \\\hline
$0.8$ & $0.4$ & $3.2$ & $2$ \\\hline
$1$ & $1$ & $3.4$ & $1.9$ \\\hline
$1.2$ & $0.9$ & $3.6$ & $2.1$ \\\hline
$1.4 $& $0.7$ & $3.8$ & $2.1$ \\\hline
$1.6 $& $1$ & $4$ & $2.4$ \\\hline
$1.8$ & $1.1$ & $4.2$ & $2.4$ \\\hline
$2$ & $1.3$ & $4.4$ & $2.5$ \\\hline
$2.2$ & $1.1$ & $4.6$ & $2.5$ \\\hline
\end{tabular}
\label{tab:ms:sp:ohm}
\end{center}
\end{table}

When we plot this data as points, we get the scatter plot shown in Figure ~\ref{fig:ms:sp:ohm}.

\begin{figure}[htp]
\begin{center}
\begin{pspicture}(-1,-1)(5,3)
% \psset{units=0.6}
%\psgrid
\psaxes[arrows=<->](0,0)(-1,-1)(5,3)
\psdots(0.0,0.4)(0.2,0.3)(0.4,0.6)(0.6,0.6)(0.8,0.4)(1.0,1.0)(1.2,0.9)(1.4,0.7)(1.6,1.0)(1.8,1.1)(2.0,1.3)(2.2,1.1)(2.4,1.4)(2.6,1.6)(2.8,1.9)(3.0,1.9)(3.2,2.0)(3.4,1.9)(3.6,2.1)(3.8,2.1)(4.0,2.4)(4.2,2.4)(4.4,2.5)(4.6,2.5)(4.8,2.5)
\rput(-0.75,1.5){\rotateleft{Voltage}}
\rput(2.5,-0.75){Current}
\end{pspicture}
\caption{Example of a scatter plot}
\label{fig:ms:sp:ohm}
\end{center}
\end{figure}

If we are to come up with a function that best describes the data, we would have to say that a straight line best describes this data.
\pagebreak
\Extension{Ohm's Law}{Ohm's Law describes the relationship between current and voltage in a conductor. The gradient of the graph of voltage vs. current is known as the \textit{resistance} of the conductor.}

\Activity{Research Project}{Scatter Plot}{The function that best describes a set of data can take any form. We will restrict ourselves to the forms already studied, that is, linear, quadratic or exponential.
Plot the following sets of data as scatter plots and deduce the type of function that best describes the data. The type of function can either be quadratic or exponential.
\begin{enumerate}
\item \begin{tabular}{|c|c||c|c||c|c||c|c|}
\hline
\textbf{$x$} & \textbf{$y$} & \textbf{$x$} & \textbf{$y$} & \textbf{$x$} & \textbf{$y$} & \textbf{$x$} & \textbf{$y$} \\
\hline
\hline
$-5$ & $9.8$ & $0$ & $14.2$ & $-2.5$ &$ 11.9$ & $2.5$ & $49.3 $\\
\hline
$-4.5 $& $4.4$ &$ 0.5$ & $22.5$ &$ -2$ &$ 6.9$ & $3$ &$ 68.9$ \\
\hline
$-4$ & $7.6$ & $1$ & $21.5$ & $-1.5$ & $8.2$ & $3.5$ & $88.4 $\\
\hline
$-3.5$ & $7.9$ &$ 1.5$ & $27.5$ & $-1$ & $7.8$ & $4$ & $117.2$ \\
\hline
$-3$ &$ 7.5$ & $2$ & $41.9$ &$ -0.5 $& $14.4$ & $4.5$ & $151.4$ \\
\hline
\end{tabular}

\vspace{0.3cm}

\item \begin{tabular}{|c|c||c|c||c|c||c|c|} \hline
\textbf{$x$} & \textbf{$y$} & \textbf{$x$} & \textbf{$y$} & \textbf{$x$} & \textbf{$y$} & \textbf{$x$} & \textbf{$y$} \\
\hline\hline
$-5$ & $75$ & $0$ & $5$ & $-2.5$ & $27.5$ & $2.5$ & $7.5$ \\\hline
$-4.5$ & $63.5$ & $0.5$ & $3.5$ & $-2$ & $21$ & $3$ & $11$ \\\hline
$-4$ & $53$ & $1$ & $3$ & $-1.5$ & $15.5$ & $3.5$ & $15.5$ \\\hline
$-3.5$ & $43.5$ & $1.5$ & $3.5$ & $-1$ & $11$ & $4$ & $21$ \\\hline
$-3$ & $35$ & $2$ & $5$ & $-0.5$ & $7.5$ & $4.5$ & $27.5$ \\\hline
\end{tabular}

\vspace{0.3cm}

\item \begin{tabular}{|l||c|c|c|c|c|c|c|c|}\hline
Height (cm) & $147$ & $150$ & $152$ & $155$ & $157$ & $160$ & $163$ & $165$\\
& $168$ & $170$ & $173$ & $175$ & $178$ & $180$ & $183$&\\\hline
Weight (kg) & $52$ & $53$ & $54$ & $56$ & $57$ & $59$ & $60$ & $61$\\
& $63$ & $64$ & $66$ & $68$ & $70$ & $72$ & $74$&\\\hline
\end{tabular}
\end{enumerate}
}

\Definition{outlier}{A point on a scatter plot which is widely separated from the other points or a result differing greatly from others in the same sample is called an outlier.}
%Phet simulation for scatter plots: SIYAVULA-SIMULATION:http://cnx.org/content/m38861/latest/#curve-fitting
\mindsetvid{Phet sim for scatter plots}{VMgao}
\Exercise{Scatter Plots}
{
\begin{enumerate}
\item A class's results for a test were recorded along with the amount of time spent studying for it. The results are given below.
\begin{center}
\begin{tabular}{|l|l|}
\hline
Score (percent) & Time spent studying (minutes)  \\ 
\hline
$67 $&$ 100 $ \\
$55 $& $85 $ \\
$70 $& $150 $ \\
$90 $& $180$  \\
$45 $& $70 $ \\
$75 $& $160 $ \\
$50 $& $80 $ \\
$60 $& $90$  \\
$84 $& $110 $ \\
$30$ & $60 $ \\
$66$ & $96 $ \\
$96$ & $200$  \\
\hline
\end{tabular}
\end{center}

	\begin{enumerate}
	\item Draw a diagram labelling horizontal and vertical axes. 
	\item State with reasons, the cause or independent variable and the effect or dependent variable.
	\item Plot the data pairs
	\item What do you observe about the plot?
	\item Is there any pattern emerging? 
	\end{enumerate}
\item The rankings of eight tennis players is given along with the time they spend practising.

\begin{center}
\begin{tabular}{|l|l|}
\hline
Practise time (min) & Ranking  \\ 
\hline
$154$ & $5$ \\
$390$ & $1$ \\
$130$ & $6$ \\
$70 $ & $8$ \\
$240$ & $3$ \\
$280$ & $2$ \\
$175$ & $4$ \\
$103$ & $7$ \\
\hline
\end{tabular}
\end{center}

	\begin{enumerate}
	\item Construct a scatter plot and explain how you chose the dependent (cause) and independent (effect) variables.
	\item What pattern or trend do you observe?
	\end{enumerate}
\item Eight children's sweet consumption and sleep habits were recorded. The data is given in the following table.

\begin{center}
\begin{tabular}{|l|l|}
\hline
Number of sweets (per week) & Average sleeping time (per day) \\ 
\hline
$15$ & $4   $\\
$12$ & $4.5 $\\
$5 $ & $8   $\\
$3 $ & $8.5$ \\
$18$ & $3 $  \\
$23$ & $2$   \\
$11$ & $5$   \\
$4$  & $8$   \\
\hline
\end{tabular}
\end{center}

	\begin{enumerate}
	\item What is the dependent (cause) variable? 
	\item What is the independent (effect) variable?
	\item Construct a scatter plot of the data.
	\item What trend do you observe?
	\end{enumerate}
\end{enumerate}


% Automatically inserted shortcodes - number to insert 3
\par \practiceinfo
\par \begin{tabular}[h]{cccccc}
% Question 1
(1.)	015s	&
% Question 2
(2.)	015t	&
% Question 3
(3.)	015u	&
\end{tabular}}
% Automatically inserted shortcodes - number inserted 3

\section{Misuse of Statistics}
%\begin{syllabus}
%\item Identify potential sources of bias, errors in measurement, and potential uses and misuses of statistics and charts and their effects (a critical analysis of misleading graphs and claims made by persons or groups trying to influence the public is implied here).
%\item Effectively communicate conclusions and predictions that can be made from the analysis of data.
%\end{syllabus}

Statistics can be manipulated in many ways that can be misleading. Graphs need to be carefully analysed and questions must always be asked about 'the story behind the figures.' Potential manipulations are:
\begin{enumerate}
\item Changing the scale to change the appearance of a graph
\item Omissions and biased selection of data
\item Focus on particular research questions
\item Selection of groups
\end{enumerate}

\Activity{Investigation}{Misuse of statistics}
{
\begin{enumerate}
\item Examine the following graphs and comment on the effects of changing scale.

\begin{figure}[H]
\begin{center}
\scalebox{0.65} % Change this value to rescale the drawing.
{
\begin{pspicture}(0,0)(9,9)
\psaxes[linewidth=0.05,arrowsize=0.1cm 2.0,arrowlength=1.4,arrowinset=0.4,Ox=2002,dx=4.0cm,dy=1.0cm,Dx=1,Dy=2]{->}(0,0)(0,0)(8.5,8.5)
\psline[linewidth=0.05](0,1)(4,3)(8,6.5)
\psdots[dotsize=0.15](0,1)(4,3)(8,6.5)
\rput{90}(-1.0,4){earnings}
\rput(4,-1.0){years}
\end{pspicture} 
}
\end{center}
% \caption{Example of a cumulative histogram for Data Set 2.\label{fig:mdat:s:cumulativegraph}}
\end{figure}

% \begin{center}
% %\scalebox{1} % Change this value to rescale the drawing.
% {
% \begin{pspicture}(0,-4.5620313)(10.4125,4.5420313)
% %\usefont{T1}{ptm}{m}{n}
% \rput(1.2978125,3.9820313){\small 16}
% %\usefont{T1}{ptm}{m}{n}
% \rput(1.3196875,2.942031){\small 14}
% %\usefont{T1}{ptm}{m}{n}
% \rput(1.276406,1.9820313){\small 12}
% %\usefont{T1}{ptm}{m}{n}
% \rput(1.2792188,1.0020314){\small 10}
% %\usefont{T1}{ptm}{m}{n}
% \rput(1.3670313,0.0220311){\small 8}
% %\usefont{T1}{ptm}{m}{n}
% \rput(1.3657812,-0.9779689){\small 6}
% %\usefont{T1}{ptm}{m}{n}
% \rput(1.3617188,-2.0379689){\small 4}
% %\usefont{T1}{ptm}{m}{n}
% \rput(1.395625,-3.0379689){\small 2}
% %\usefont{T1}{ptm}{m}{n}
% \rput(1.3709375,-3.997969){\small 0}
% \psline[linewidth=0.04cm](1.7924999,-4.097969)(1.7924999,4.462031)
% \psline[linewidth=0.04cm](1.8124999,4.4820313)(1.6925,4.342031)
% \psline[linewidth=0.04cm](1.7924999,4.5220313)(1.9324999,4.362031)
% \psline[linewidth=0.04cm](1.7524999,-3.997969)(10.392501,-3.997969)
% \psline[linewidth=0.04cm](10.392501,-3.977969)(10.2325,-3.8979688)
% \psline[linewidth=0.04cm](10.352501,-3.977969)(10.2325,-4.117969)
% %\usefont{T1}{ptm}{m}{n}
% \rput(1.6856251,-4.397969){\small 2002}
% %\usefont{T1}{ptm}{m}{n}
% \rput(5.704531,-4.4179688){\small 2003}
% %\usefont{T1}{ptm}{m}{n}
% \rput(9.748905,-4.357969){\small 2004}
% \psline[linewidth=0.04cm](1.8124999,-3.977969)(1.8124999,-4.137969)
% \psline[linewidth=0.04cm](5.7925,-3.997969)(5.7925,-4.217969)
% \psline[linewidth=0.04cm](9.8724985,-3.977969)(9.8724985,-4.177969)
% \psline[linewidth=0.04cm](1.7724999,4.0020313)(1.6724999,4.0020313)
% \psline[linewidth=0.04cm](1.7924999,2.962031)(1.6925,2.962031)
% \psline[linewidth=0.04cm](1.7724999,1.9820313)(1.6724999,1.9620309)
% \psline[linewidth=0.04cm](1.7724999,1.0020314)(1.6125001,1.0220313)
% \psline[linewidth=0.04cm](1.7724999,0.0020311)(1.6724999,0.0020311)
% \psline[linewidth=0.04cm](1.7524999,-0.9779689)(1.5925,-0.9779689)
% \psline[linewidth=0.04cm](1.7724999,-2.0579689)(1.6125001,-2.0579689)
% \psline[linewidth=0.04cm](1.7524999,-3.0379689)(1.6524999,-3.017969)
% \psline[linewidth=0.04cm](1.7325001,-3.997969)(1.6524999,-3.997969)
% \psline[linewidth=0.04cm](1.7924999,-3.017969)(5.8324995,-1.0179689)
% \psline[linewidth=0.04cm](5.8324995,-1.0179689)(9.852501,2.5620313)
% \psdots[dotsize=0.06](5.8324995,-1.0579689)
% \psdots[dotsize=0.06](9.8724985,2.5820312)
% \psdots[dotsize=0.06](1.7524999,-2.977969)
% %\usefont{T1}{ptm}{m}{n}
% \rput(0.56359375,0.4920311){earnings}
% \end{pspicture} 
% }
% \end{center}

\begin{figure}[H]
\begin{center}
\scalebox{0.65} % Change this value to rescale the drawing.
{
\begin{pspicture}(0,0)(9,9)
\psaxes[linewidth=0.05,arrowsize=0.1cm 2.0,arrowlength=1.4,arrowinset=0.4,Ox=2002,dx=4.0cm,dy=1.0cm,Dx=1,Dy=10]{->}(0,0)(0,0)(8.5,8.5)
\psline[linewidth=0.05](0,0.2)(4,0.6)(8,1.3)
\psdots[dotsize=0.15](0,0.2)(4,0.6)(8,1.3)
\rput{90}(-1.0,4){earnings}
\rput(4,-1.0){years}
\end{pspicture} 
}
\end{center}
% \caption{Example of a cumulative histogram for Data Set 2.\label{fig:mdat:s:cumulativegraph}}
\end{figure}
\vspace{1cm}
% \begin{center}
% %\scalebox{1} % Change this value to rescale the drawing.
% {
% \begin{pspicture}(0,-4.8189063)(10.439531,4.7789063)
% %\usefont{T1}{ptm}{m}{n}
% \rput(1.3767188,4.2189064){\small 80}
% %\usefont{T1}{ptm}{m}{n}
% \rput(1.4048437,3.1789062){\small 70}
% %\usefont{T1}{ptm}{m}{n}
% \rput(1.3626562,2.2189062){\small 60}
% %\usefont{T1}{ptm}{m}{n}
% \rput(1.3610938,1.2389063){\small 50}
% %\usefont{T1}{ptm}{m}{n}
% \rput(1.3720312,0.27890626){\small 40}
% %\usefont{T1}{ptm}{m}{n}
% \rput(1.2835938,-0.70109373){\small 30}
% %\usefont{T1}{ptm}{m}{n}
% \rput(1.2701563,-1.8010937){\small 20}
% %\usefont{T1}{ptm}{m}{n}
% \rput(1.2973437,-2.8010938){\small 10}
% %\usefont{T1}{ptm}{m}{n}
% \rput(1.4151562,-3.7610939){\small 0}
% \psline[linewidth=0.04cm](1.8195312,-3.8610938)(1.8195312,4.6989064)
% \psline[linewidth=0.04cm](1.8395312,4.7189064)(1.7195313,4.578906)
% \psline[linewidth=0.04cm](1.8195312,4.7589064)(1.9595312,4.598906)
% \psline[linewidth=0.04cm](1.7795312,-3.7610939)(10.419532,-3.7610939)
% \psline[linewidth=0.04cm](10.419532,-3.7410936)(10.259531,-3.6610937)
% \psline[linewidth=0.04cm](10.379532,-3.7410936)(10.259531,-3.8810937)
% %\usefont{T1}{ptm}{m}{n}
% \rput(1.7282813,-4.1610937){\small 2002}
% %\usefont{T1}{ptm}{m}{n}
% \rput(5.7542186,-4.1810937){\small 2003}
% %\usefont{T1}{ptm}{m}{n}
% \rput(9.790468,-4.1210938){\small 2004}
% \psline[linewidth=0.04cm](1.8395312,-3.7410936)(1.8395312,-3.9010937)
% \psline[linewidth=0.04cm](5.8195314,-3.7610939)(5.8195314,-3.9810936)
% \psline[linewidth=0.04cm](9.89953,-3.7410936)(9.89953,-3.9410937)
% \psline[linewidth=0.04cm](1.7995312,4.2389064)(1.6995312,4.2389064)
% \psline[linewidth=0.04cm](1.8195312,3.1989062)(1.7195313,3.1989062)
% \psline[linewidth=0.04cm](1.7995312,2.2189062)(1.6995312,2.1989062)
% \psline[linewidth=0.04cm](1.7995312,1.2389063)(1.6395313,1.2589062)
% \psline[linewidth=0.04cm](1.7995312,0.23890625)(1.6995312,0.23890625)
% \psline[linewidth=0.04cm](1.7795312,-0.74109375)(1.6195313,-0.74109375)
% \psline[linewidth=0.04cm](1.7995312,-1.8210938)(1.6395313,-1.8210938)
% \psline[linewidth=0.04cm](1.7795312,-2.8010938)(1.6795312,-2.7810938)
% \psline[linewidth=0.04cm](1.7595313,-3.7610939)(1.6795312,-3.7610939)
% %\usefont{T1}{ptm}{m}{n}
% \rput(0.60359377,0.7289063){earnings}
% \psline[linewidth=0.04cm](1.7995312,-3.3810937)(5.8195314,-3.2410936)
% \psline[linewidth=0.04cm](5.8195314,-3.2410936)(9.839532,-2.3610938)
% %\usefont{T1}{ptm}{m}{n}
% \rput(5.7295313,-4.5910935){year}
% \end{pspicture} 
% }
% \end{center}

\item Examine the following three plots and comment on omission, selection and bias. Hint: What is wrong with the data and what is missing from the bar and pie charts?

\begin{center}
\begin{tabular}{|l|c|}
\hline
Activity & Hours \\ 
\hline
Sleep        & $8 $\\
Sports       & $2 $\\
School       & $7 $\\
Visit friend & $1$ \\
Watch TV     & $2$ \\
Studying     & $3$ \\
\hline
\end{tabular}
\end{center}

\begin{figure}[H]
\begin{center}
\scalebox{0.9}
{
% \psset{xunit=.44cm,yunit=.3cm}
\begin{pspicture}(0,0)(10,10)
\savedata{\Data}[1 7  2.5 8  4 3  5.5 2  7 1  8.5 2]
% \savedata{\activities}[school&sleep&studying&sports&visit friend&watch tv]
% \def\pshlabel#1{\tiny\activities(#1)}
% \psaxes[xAxisLabel=,xLabels={school,sleep,studying,sports,visitfriend,watchTV},ticks=y,xLabelsRot=45,Dx=1.5,Dy=1](10,10)
\psaxes[xLabels=.,ticks=y,Dx=1.5,Dy=1](10,10)
\listplot[shadow=false,linecolor=black,plotstyle=bar,barwidth=0.8cm,fillcolor=red,fillstyle=solid]{\Data}
\rput{-90}(1,-1){school}
\rput{-90}(2.5,-1){sleep}
\rput{-90}(4,-1){studying}
\rput{-90}(5.5,-1){sports}
\rput{-90}(7,-1){visit friend}
\rput{-90}(8.5,-1){watch TV}
\end{pspicture}
}
\end{center}
\end{figure}

% \begin{center}
% \scalebox{1} % Change this value to rescale the drawing.
% {
% \begin{pspicture}(0,0)(10,6)
% \definecolor{color1180b}{rgb}{0.8,0.8,0.8}
% \psaxes[linewidth=0.05,arrowsize=0.1cm 2.0,arrowlength=1.4,arrowinset=0.4,Ox=,dx=2.0cm,dy=1.0cm,Dx=1,Dy=1]{->}(0,0)(0,0)(12.5,10.5)
% \psline[linewidth=0.04cm](0.57109374,6.026972)(0.61109376,-4.3130283)
% \psline[linewidth=0.04cm](0.57109374,-4.3330283)(9.631094,-4.3130283)
% %%%\usefont{T1}{ptm}{m}{n}
% \rput(0.14890625,5.6269717){\small 10}
% %%%\usefont{T1}{ptm}{m}{n}
% \rput(0.22296876,4.5669713){\small 9}
% %%%\usefont{T1}{ptm}{m}{n}
% \rput(0.19078127,3.6069717){\small 8}
% %%%\usefont{T1}{ptm}{m}{n}
% \rput(0.28359374,2.6269717){\small 7}
% %%%\usefont{T1}{ptm}{m}{n}
% \rput(0.32328126,1.6269716){\small 6}
% %%%\usefont{T1}{ptm}{m}{n}
% \rput(0.35234374,0.5669717){\small 5}
% %%%\usefont{T1}{ptm}{m}{n}
% \rput(0.33390626,-0.41302827){\small 4}
% %%%\usefont{T1}{ptm}{m}{n}
% \rput(0.30921876,-1.3730285){\small 3}
% %%%\usefont{T1}{ptm}{m}{n}
% \rput(0.30984375,-2.3730285){\small 2}
% %%%\usefont{T1}{ptm}{m}{n}
% \rput(0.30953124,-3.4330285){\small 1}
% %%%\usefont{T1}{ptm}{m}{n}
% \rput(0.30671874,-4.3930283){\small 0}
% %%%\usefont{T1}{ptm}{m}{n}
% \rput{-89.557884}(8.074114,-2.0097408){\rput(3.0078402,-5.0547624){\small sleep}}
% %%%\usefont{T1}{ptm}{m}{n}
% \rput{-92.24784}(6.6021667,-3.6211305){\rput(1.5436302,-5.0002403){\small school}}
% %%%\usefont{T1}{ptm}{m}{n}
% \rput{-90.31768}(11.08398,0.96811926){\rput(6.0068107,-4.991975){\small sports}}
% %%%\usefont{T1}{ptm}{m}{n}
% \rput{-89.987595}(12.846535,2.4113781){\rput(7.6264052,-5.2345943){\small visit friend}}
% %%%\usefont{T1}{ptm}{m}{n}
% \rput{-89.03259}(9.498502,-0.68705446){\rput(4.3833113,-5.1549015){\small studying}}
% %%%\usefont{T1}{ptm}{m}{n}
% \rput{-91.91123}(14.631501,3.5999095){\rput(9.054456,-5.288088){watch TV}}
% \psframe[linewidth=0.04,dimen=outer,fillstyle=solid,fillcolor=color1180b](1.9510938,2.6869717)(1.2510937,-4.3330283)
% \psframe[linewidth=0.04,dimen=outer,fillstyle=solid,fillcolor=color1180b](3.4510937,3.6469717)(2.6710937,-4.3330283)
% \psframe[linewidth=0.04,dimen=outer,fillstyle=solid,fillcolor=color1180b](4.9710937,-1.3730285)(4.191094,-4.3330283)
% \psframe[linewidth=0.04,dimen=outer,fillstyle=solid,fillcolor=color1180b](6.4910936,-2.3530285)(5.751094,-4.3530283)
% \psframe[linewidth=0.04,dimen=outer,fillstyle=solid,fillcolor=color1180b](7.9710937,-3.3130286)(7.171094,-4.3330283)
% \psframe[linewidth=0.04,dimen=outer,fillstyle=solid,fillcolor=color1180b](9.451094,-2.3330286)(8.771093,-4.3130283)
% \end{pspicture} 
% }
% \end{center}
\vspace*{3cm}
\begin{figure}[H]
 \begin{center}
  \psset{unit=1cm}
  \scalebox{0.9}
  {
  \begin{pspicture}(-2.2,-2.2)(2.2,2.2)
  \pswedge[fillstyle=none,fillcolor=gray]{3.5}{70.4347826}{180}
  \pswedge[fillstyle=none,fillcolor=lightgray]{3.5}{-54.7826086}{70.4347826}
  \pswedge[fillstyle=none,fillcolor=darkgray]{3.5}{-101.7391304}{-54.7826086}
  \pswedge[fillstyle=none,fillcolor=darkgray]{3.5}{-133.0434782}{-101.7391304}
  \pswedge[fillstyle=none,fillcolor=darkgray]{3.5}{-148.6956521}{-133.0434782}
  \pswedge[fillstyle=none,fillcolor=darkgray]{3.5}{180}{-148.6956521}
  \SpecialCoor
  \psset{framesep=1.5pt}
  \uput{1.5}[125](0,0){school}  
  \uput{1.5}[5](0,0){sleep}
  \uput{1.5}[-78]{-78}(0,0){studying}
  \uput{1.5}[242]{62}(0,0){sports}
  \uput{1.5}[218]{38}(0,0){visit friend}
  \uput{1.5}[195]{15}(0,0){watch TV}
  \end{pspicture}
  }
 \end{center}
\end{figure}

% \begin{center}
% %\scalebox{1} % Change this value to rescale the drawing.
% {
% \begin{pspicture}(0,-3.33)(6.84,3.33)
% \psellipse[linewidth=0.04,dimen=outer](3.42,0.0)(3.42,3.33)
% \psline[linewidth=0.04cm](1.04,2.45)(3.42,-0.17)
% \psline[linewidth=0.04cm](3.42,-0.17)(6.72,-0.95)
% \psline[linewidth=0.04cm](3.46,-0.23)(3.98,-3.25)
% \psline[linewidth=0.04cm](3.44,-0.17)(3.34,-3.27)
% \psline[linewidth=0.04cm](3.42,-0.11)(1.06,-2.45)
% %%%%\usefont{T1}{ptm}{m}{n}
% \rput(4.3309374,1.51){\small sleep}
% %%%%\usefont{T1}{ptm}{m}{n}
% \rput(1.4854687,0.03){\small school}
% %%%%\usefont{T1}{ptm}{m}{n}
% \rput(2.341875,-2.25){\small sports}
% %%%%\usefont{T1}{ptm}{m}{n}
% \rput{-85.23191}(5.611834,1.5106634){\rput(3.6240625,-2.31){\small visit friend}}
% %%%%\usefont{T1}{ptm}{m}{n}
% \rput{-27.231009}(1.0881134,2.2749026){\rput(5.2235937,-1.09){\small studying}}
% \psline[linewidth=0.04cm](3.4,-0.11)(5.52,-2.55)
% %%%%\usefont{T1}{ptm}{m}{n}
% \rput{-65.12421}(4.359607,2.8378086){\rput(4.3995314,-2.01){\small watch TV}}
% \end{pspicture} 
% }
% \end{center}
\end{enumerate}
\vspace{1cm}
}

\Exercise{Misuse of Statistics}
{ The bar graph below shows the results of a study that looked at the cost of food compared to the income of a household in 1994.

\begin{figure}[H]
\begin{center}
\scalebox{1}
{
% \psset{xunit=.44cm,yunit=.3cm}
\begin{pspicture}(-1.5,-1.5)(11,6.5)
\savedata{\Data}[1 1  2.3 1  3.6 2  4.9 2  6.2 4  7.5 3  8.8 5  10.1 6]
% \savedata{\activities}[school&sleep&studying&sports&visit friend&watch tv]
% \def\pshlabel#1{\tiny\activities(#1)}
\psaxes[xAxisLabel=,xLabels=.,ticks=y,xLabelsRot=45,Dx=1.5,dy=1,Dy=2](11,6)
\listplot[shadow=false,linecolor=black,plotstyle=bar,barwidth=0.8cm,fillcolor=red,fillstyle=solid]{\Data}
\rput{-90}(1,-0.7){$<$5}
\rput{-90}(2.3,-0.7){5-10}
\rput{-90}(3.6,-0.7){10-15}
\rput{-90}(4.9,-0.7){15-20}
\rput{-90}(6.2,-0.7){20-30}
\rput{-90}(7.5,-0.7){30-40}
\rput{-90}(8.8,-0.7){40-50}
\rput{-90}(10.1,-0.7){$>$50}
\rput(5.5,-1.5){Income in 1994 (in thousands of rands)}
\rput{-90}(-1,3.25){Food bill (in thousands of rands)}
\end{pspicture}
}
\end{center}
\end{figure}

% \vspace{2cm}
% \begin{center}
% %\scalebox{1} % Change this value to rescale the drawing.
% {
% \begin{pspicture}(0,-4.1525)(10.159853,4.1325)
% \definecolor{color1180b}{rgb}{0.8,0.8,0.8}
% \psline[linewidth=0.04cm](1.0687593,4.1125)(1.0887593,-2.7475)
% \psline[linewidth=0.04cm](1.079853,-2.7875004)(10.1398535,-2.7675004)
% %%%\usefont{T1}{ptm}{m}{n}
% \rput(0.7576655,2.2124996){\small 10}
% %%%\usefont{T1}{ptm}{m}{n}
% \rput(0.77954054,1.2124999){\small 8}
% %%%\usefont{T1}{ptm}{m}{n}
% \rput(0.7920405,0.1924998){\small 6}
% %%%\usefont{T1}{ptm}{m}{n}
% \rput(0.86266553,-0.7875001){\small 4}
% %%%\usefont{T1}{ptm}{m}{n}
% \rput(0.79860306,-1.9075003){\small 2}
% %%%\usefont{T1}{ptm}{m}{n}
% \rput(0.815478,-2.7875004){\small 0}
% %%%\usefont{T1}{ptm}{m}{n}
% \rput{-89.557884}(6.0073156,-0.37899327){\rput(2.7964432,-3.2292342){\small 5-10}}
% %%%\usefont{T1}{ptm}{m}{n}
% \rput{-92.24784}(4.9755993,-1.4643598){\rput(1.7775458,-3.1347122){\small $<$5}}
% %%%\usefont{T1}{ptm}{m}{n}
% \rput{-90.31768}(8.404476,1.8107378){\rput(5.062601,-3.2864463){\small 15-20}}
% %%%\usefont{T1}{ptm}{m}{n}
% \rput{-89.987595}(9.518503,2.9680576){\rput(6.2364144,-3.289066){\small 20-30}}
% %%%\usefont{T1}{ptm}{m}{n}
% \rput{-89.03259}(7.201252,0.730758){\rput(3.932227,-3.3093731){\small 10-15}}
% %%%\usefont{T1}{ptm}{m}{n}
% \rput{-91.91123}(12.214403,5.254118){\rput(8.642746,-3.2925594){\small 40-50}}
% \psframe[linewidth=0.04,dimen=outer,fillstyle=solid,fillcolor=color1180b](2.139853,-1.7875)(1.4398531,-2.7875004)
% \psframe[linewidth=0.04,dimen=outer,fillstyle=solid,fillcolor=color1180b](3.2087593,-1.7875)(2.439853,-2.7875004)
% \psframe[linewidth=0.04,dimen=outer,fillstyle=solid,fillcolor=color1180b](4.319853,-0.7875)(3.539853,-2.8075004)
% \psframe[linewidth=0.04,dimen=outer,fillstyle=solid,fillcolor=color1180b](5.459853,-0.7675003)(4.719853,-2.7675004)
% \psframe[linewidth=0.04,dimen=outer,fillstyle=solid,fillcolor=color1180b](6.679853,1.2125)(5.8798532,-2.7475002)
% \psframe[linewidth=0.04,dimen=outer,fillstyle=solid,fillcolor=color1180b](7.939853,0.2325)(7.259853,-2.7875004)
% %%%\usefont{T1}{ptm}{m}{n}
% \rput(0.71672803,3.2124996){\small 12}
% %%%\usefont{T1}{ptm}{m}{n}
% \rput{-91.91123}(11.144077,4.218913){\rput(7.5985274,-3.2925594){\small 30-40}}
% %%%\usefont{T1}{ptm}{m}{n}
% \rput{-91.91123}(13.044959,6.257421){\rput(9.542277,-3.1925595){\small $>$50}}
% \psframe[linewidth=0.04,dimen=outer,fillstyle=solid,fillcolor=color1180b](8.959853,2.2325)(8.279853,-2.7475)
% \psframe[linewidth=0.04,dimen=outer,fillstyle=solid,fillcolor=color1180b](9.919853,3.2725)(9.239853,-2.7675004)
% %%%\usefont{T1}{ptm}{m}{n}
% \rput(5.187978,-3.9975){Income in 1994 (in thousands of rands)}
% %%%\usefont{T1}{ptm}{m}{n}
% \rput{-89.98055}(-0.63489014,0.9630194){\rput(0.15829054,0.8225){Food bill (in thousands of rands)}}
% \end{pspicture} 
% }
% \end{center}

\begin{center}
\begin{tabular}{|c|c|}
\hline
Income (thousands of rands) & Food bill (thousands of rands) \\ 
\hline
 $<5$  & $2$  \\
 $5-10$  & $2$  \\
 $10-15$ & $4$  \\
 $15-20$ & $4$  \\
 $20-30$ & $8$  \\
 $30-40$ & $6$  \\
 $40-50$ & $10$ \\
 $>50$ & $12$ \\
\hline
\end{tabular}
\end{center}

\begin{enumerate}
\item What is the dependent variable? Why?
\item What conclusion can you make about this variable? Why? Does this make sense? 
\item What would happen if the graph was changed from food bill in thousands of rands to percentage of income?
\item Construct this bar graph using a table. What conclusions can be drawn?
\item Why do the two graphs differ despite showing the same information?
\item What else is observed? Does this affect the fairness of the results? 
\end{enumerate}

% Automatically inserted shortcodes - number to insert 6
\practiceinfo
\par \begin{tabular}[h]{cccccc}
% Question 1
(1.)	015v	&
% Question 2
(2.)	015w	&
% Question 3
(3.)	015x	&
% Question 4
(4.)	015y	&
% Question 5
(5.)	015z	&
% Question 6
(6.)	0160	\\ % End row of shortcodes
\end{tabular}}
% Automatically inserted shortcodes - number inserted 6
\begin{eocexercises}{}
\begin{enumerate}
\item Many accidents occur during the holidays between Durban and Johannesburg. A study was done to see if speeding was a factor in the high accident rate. Use the results given to answer the following questions.

\begin{center}
\begin{tabular}{|l|l|}
\hline
Speed (km/h) & Frequency \\ 
\hline
 $60<x\leq 70$   & $3 $ \\
 $70<x\leq 80$   & $2 $ \\
 $80<x\leq 90$   & $6 $ \\
 $90<x\leq 100$  & $40$ \\
 $100<x\leq 110$ & $50$ \\
 $110<x\leq 120$ & $30$ \\
 $120<x\leq 130$ & $15$ \\
 $130<x\leq 140$ & $12$ \\
 $140<x\leq 150$ & $3 $ \\
 $150<x\leq 160$ & $2$  \\
\hline
\end{tabular}
\end{center}

	\begin{enumerate}
	\item Draw a graph to illustrate this information. 
	\item Use your graph to find the median speed and the interquartile range.
	\item What percent of cars travel more than $120~$km/h on this road?
	\item Do cars generally exceed the speed limit?
	\end{enumerate}

\item The following two diagrams (showing two schools contribution to charity) have been exaggerated. Explain how they are misleading and redraw them so that they are not misleading.

\begin{center}
%\scalebox{1} % Change this value to rescale the drawing.
{
\begin{pspicture}(0,-3)(8,3)
\psline[linewidth=0.04cm](0,-3)(0,0)(1,0)(1,-3)(0,-3)
\psline[linewidth=0.04cm](0,0)(1,1)(2,1)(1,0)

\psline[linewidth=0.04cm](2,1)(2,-2)(1,-3)
\psline[linewidth=0.04cm](0,-2)(1,-2)(2,-1)
\psline[linewidth=0.04cm](0,-1)(1,-1)(2,0)

\rput(0.5,-0.5){\small R100}
\rput(0.5,-1.5){\small R100}
\rput(0.5,-2.5){\small R100}

\psline[linewidth=0.04cm](6,-3)(4,-3)(4,1)(6,1)(6,-3)
\psline[linewidth=0.04cm](4,1)(6,3)(8,3)(6,1)

\psline[linewidth=0.04cm](8,3)(8,-1)(6,-3)
\psline[linewidth=0.04cm](4,-1)(6,-1)(8,1)

\rput(5,0){\small R200.00}
\rput(5,-2){\small R200.00}
\end{pspicture} 
}
\end{center}

\item The monthly income of eight teachers are given as follows: \newline
R$10~ 050$; R$14 ~300$; R$9 ~800$; R$15~ 000$; R$12~ 140$; R$13 ~800$; R$11~ 990$; R$12~ 900$.
	\begin{enumerate}
	\item What is the mean income and the standard deviation? 
	\item How many of the salaries are within one standard deviation of the mean?
	\item If each teacher gets a bonus of R$500$ added to their pay what is the new mean and standard deviation?
	\item If each teacher gets a bonus of $10\%$ on their salary what is the new mean and standard deviation? 
	\item Determine for both of the above, how many salaries are within one standard deviation of the mean.
	\item Using the above information work out which bonus is more beneficial financially for the teachers.
	\end{enumerate}
\end{enumerate}




% Automatically inserted shortcodes - number to insert 3
\par \practiceinfo
\par \begin{tabular}[h]{cccccc}
% Question 1
(1.)	0161	&
% Question 2
(2.)	0162	&
% Question 3
(3.)	0163	&
\end{tabular}
% Automatically inserted shortcodes - number inserted 3
\end{eocexercises} 



% CHILD SECTION START 

